% \section{Symbol der Synode von Nic�a}
\kapitel{Theologische Erkl�rung der Synode von Nicaea (Urk.~24)}
\label{ch:24}

\begin{footnotesize}
Die in Nicaea zusammengekommenen Bisch�fe waren beinahe 300 an der Zahl, verurteilten
die arianische H�esie und verbannten Arius und seine Mitstreiter. Schlie�lich legten sie zur Widerlegung 
jeder H�resie den kirchlichen Glauben schriftlich fest.
\end{footnotesize}

Das Folgende ist in Nicaea beschlossen worden:

Wir glauben an einen Gott, den Vater, den Allm�chtigen, den Sch�pfer aller sichtbaren und unsichtbaren Dinge;
und an einen Herrn Jesus Christus, den Sohn Gottes, als Eingeborener gezeugt aus dem Vater, das
hei�t aus dem Wesen des Vaters, Gott von Gott, Licht von Licht, wahrer Gott von wahrem Gott, gezeugt
und nicht geschaffen, wesenseins mit dem Vater, durch den alles wurde, was im Himmel und auf Erden ist,
der f�r uns Menschen und um unseres Heils willen herabstieg und Fleisch wurde, der Mensch geworden ist,
litt und am dritten Tag auferstand, aufstieg in die Himmel, der kommen wird, um die Lebenden
und die Toten zu richten; und an den heiligen Geist.
Die aber sagen, ">es war einmal, da� er nicht war"< oder ">er war nicht, bevor er gezeugt
wurde"< oder ">aus dem Nichts wurde er"< oder die behaupten, er sei aus einer anderen Hypostase oder
einem anderen Wesen, oder aber sagen, der Sohn Gottes sei geschaffen, wandelbar oder ver�nderlich, diese verdammt
die katholische und apostolische Kirche.
\clearpage