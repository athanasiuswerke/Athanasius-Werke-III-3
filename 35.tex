\kapitel{Fragmente eines Briefes (?) des Athanasius von Anazarba}
% \label{ch:35}
\label{ch:AthAnaz}
\begin{praefatio}
  \begin{description}
  \item[Vor Nicaea 325?]Die Datierung dieser Fragmente ist v�llig
    unklar. Falls sie zu dem Fragment eines Briefes des Athanasius von
    Anazarba\index[namen]{Athanasius!Bischof von Anazarba} an
    Alexander von Alexandrien\index[namen]{Alexander!Bischof von
      Alexandrien} geh�ren, das Athanasius von
    Alexandrien\index[namen]{Athanasius!Bischof von Alexandrien} in
    syn. 17 zitiert (= Dok \ref{ch:11} [Urk. 11]), fallen sie in die
    Zeit vor der Synode von Nicaea\index[synoden]{Nicaea!a. 325},
    d.\,h. ungef�hr in das Jahr 322. Sie k�nnten aber auch aus einer
    anderen Schrift stammen, da sich Athanasius von
    Anazarba\index[namen]{Athanasius!Bischof von Anazarba}
    offensichtlich mehrfach literarisch zu Thesen des
    Arius\index[namen]{Arius!Presbyter in Alexandrien} ge�u�ert hat
    und z.\,B. wahrscheinlich auch der Autor einer Homilie ist, die
    unter dem Namen des Athanasius von
    Alexandrien\index[namen]{Athanasius!Bischof von Alexandrien}
    �berliefert wurde (\cite{Tetz:Homilie}). Von ihm ist nur bekannt,
    da� er Bischof von Anazarba in Cilicia war und sp�ter Lehrer des
    A"etius\index[namen]{A"etius!Diakon in Alexandrien} wurde
    (Philost., h.\,e. III 15); daraus lassen sich jedoch keine
    chronologischen Hinweise ableiten.
  \item[�berlieferung]Die drei Fragmente sind im Codex Vaticanus
    latinus 5750 (p. 275) �berliefert, der zusammen mit Codex
    Ambrosianus E 147 sup. urspr�nglich einen Band mit 792 Bl�ttern
    bildete und aus der Bibliothek von Bobbio stammte. Es handelt sich
    um ein Palimpsest aus dem 7. Jh., bei dem zwei Handschriften mit
    der Korrespondenz zwischen Fronto und Marc Aurel und mit Scholien
    zu den Reden Ciceros f�r eine Abschrift der Akten der Synode von
    Chalcedon wiederverwendet wurden. Vor Ende des 7. Jh.s wurden etwa
    140 Seiten mit ">arianischen"< Texten �berschrieben. Gryson hat in
    seiner kritischen Edition (Scripta Arriana Latina, CChr.SL
    87,229-265) die anonym und titellos �berlieferten ">arianischen"<
    Fragmente neu sortiert und nach inhaltlichen Kriterien zwei
    voneinander zu unterscheidenden Schriften zugewiesen, die er
    ">Adversus Orthodoxos et Macedonianos"< (Fragment 1--12) und
    ">Instructio Verae Fidei"< (Fragment 13--23) nennt. F�r die erste
    Schrift, die Bez�ge zur sog. ersten sirmischen Formel zeigt, aber
    die Auseinandersetzungen um die Pneumatologie voraussetzt, nimmt
    er illyrische Herkunft an und datiert sie nach 380. Der Anonymus
    zitiert aus einem Text des Athanasius von
    Anazarba\index[namen]{Athanasius!Bischof von Anazarba}, der sich
    wiederum auf Dionys von Alexandrien\index[namen]{Dionysius!Bischof
      von Alexandrien} beruft (Frgm. 4 [\editioncite[235]{Gry},
    basierend auf \cite{DeBruyne}]).

    Die drei Athanasius von Anazarba zuzuschreibenden Fragmente sind
    �bersetzungen aus dem Griechischen, die dem griechischen Text Wort
    f�r Wort zu folgen scheinen.\footnoteA{Vgl. die R�ck�bersetzungen
      ins Griechische bei \cite[51]{Opitz:Dionys} und
      \cite[258]{Abramowski:Dionys}.} Seit ihrer Entdeckung wird das
    zweite Fragment als Zitat des Dionysius von Alexandrien
    diskutiert, wobei der �berlieferungszusammenhang trotz der
    theologischen Ablehnung einer Identifikationstheologie, die
    inhaltlich zu Dionysius passen w�rde, hier keine eindeutige
    Entscheidung zul��t. 
\item[Fundstelle]Codex Vaticanus lat. 5750, p. 275 (Frgm. 4 [\editioncite[235]{Gry}])
  \end{description}
\end{praefatio}
%
\begin{pairs}
\begin{Leftside}
\beginnumbering
\pstart
\noindent\kap{1}\edtext{\abb{}}{\killnumber\Cfootnote{\hskip -1em\latintext Cod. Vat. lat.
5750}}\specialindex{quellen}{chapter}{Codices!Vaticanus lat. 5750!p. 275}
\dots\ provisor, omnium iudex et
\edtext{despensator}{\Dfootnote{dispensator \textit{coni. Mai}}}, deus qui omnia creavit et construxit, qui
fecit omnia ex nihilo.
\pend
\pstart
\kap{2}\footnotesize{Iterum idem ipse Athanasius\edindex[namen]{Athanasius!Bischof von Anazarba}
antiquorum profert memoriam ac Dionisi\edindex[namen]{Dionysius!Bischof von Alexandrien}
episcopi, ut ostendat ante esse patrem quam filius generaretur,
dicens:}
\pend
\pstart
Ita pater quidem pater et non filius, non quia factus est, sed quia est, non ex aliquo,
sed in se permanens; filius autem et non pater, non quia erat, sed quia factus est, non de
se, sed ex eo qui eum fecit filii dignitatem sortitus est. \pend
\pstart
\kap{3}\footnotesize{Deinde ipse Athanasius:}\edindex[namen]{Athanasius!Bischof von Anazarba}
\pend
\pstart
Non enim se erigit filius contra patrem neque putat \edtext{\abb{\edtext{\abb{parem}}{\Dfootnote{\textit{coni. Bauer} paria \textit{cod.}}} esse cum
deo}}{\Afootnote{\latintext vgl. Phil 2,6}}\edindex[bibel]{Philipper!2,6|textit}, cedit
autem patri suo et fatetur docens omnes quia pater \edtext{maior}{\Dfootnote{\latintext
maior, maior \textit{coni. Gryson} \hskip 1ex maior se est, maior \textit{coni. de
Bruyne}}\lemma{\abb{maior}}\Afootnote{\latintext vgl. Io 14,28}}\edindex[bibel]{Johannes!14,28|textit},
autem non vastitate neque 
\edtext{\abb{magnitudine}}{\Dfootnote{\textit{coni. Mai} magnitudinem \textit{cod.}}}, quae quidem corporum propria sunt, sed perpetuitate et
innarrabili eius paterna ac generandi 
\edtext{\abb{virtute}}{\Dfootnote{\textit{coni. Mai} virtutem \textit{cod.}}}, et quia ipse quidem sempiternus est et in
se plenitudinem habens et a nullo vitam habens.
\pend
% \endnumbering
\end{Leftside}
\begin{Rightside}
\begin{translatio}
\beginnumbering
\pstart
\noindent\dots\ Gott, der Lenker, Richter und Verwalter von allem, der alles erschaffen und
errichtet, der alles aus nichts geschaffen hat.
\pend
\pstart
\footnotesize{Dann wiederum erinnert derselbe Athanasius an die Alten und an den Bischof Dionys,
um zu zeigen, da� der Vater war, bevor der Sohn gezeugt wurde, n�mlich:}
\pend
\pstart
So ist der Vater also Vater und nicht Sohn, nicht weil er gemacht ist, sondern weil er
existiert; er ist nicht aus einem anderen, sondern bleibt immer in sich; der Sohn aber ist
nicht Vater, nicht weil er (schon) war, sondern weil er gemacht worden ist; nicht aus sich
heraus, sondern aus jenem, der ihn gemacht hat, erhielt er die
Sohnesw�rde.
\pend
\pstart
\footnotesize{Hierauf derselbe Athanasius:}
\pend
\pstart
Der Sohn erhebt sich n�mlich nicht gegen den Vater, auch meint er nicht, Gott gleich zu
sein, vielmehr tritt er hinter seinem Vater zur�ck und bekennt und lehrt alle, da� der
Vater gr��er sei, aber nicht dem Umfang oder den Ausma�en nach, was doch Eigenschaften von
K�rpern sind, sondern nach der Ewigkeit und seiner unaussprechlichen v�terlichen und
zeugenden Kraft, und weil er ja selbst ewig ist und F�lle in sich selbst hat und von
niemandem das Leben bekommen hat.
\pend
\endnumbering
\end{translatio}
\end{Rightside}
\Columns
\end{pairs}
% \thispagestyle{empty}
%%% Local Variables: 
%%% mode: latex
%%% TeX-master: "dokumente_master"
%%% End: 
