%% Erstellt von uh
%% �nderungen:
%%%% 13.5.2004 Layout-Korrekturen (avs) %%%%
\kapitel{Berichte �ber Synoden in Antiochien des Jahres 327}
\label{ch:Antiochien326}
\section{Bericht �ber Verhandlungen gegen Asclepas von Gaza}
% \label{sec:36.1} 
\label{sec:Asclepas} 
\begin{praefatio}
  \begin{description}
  \item[327]Die Datierung ist allein aus einer Angabe in dem Schreiben
    der �stlichen Teilsynode von
    Serdica\index[synoden]{Serdica!a. 343} (Dok.
    \ref{sec:RundbriefSerdikaOst}) herzuleiten: \textit{dicimus autem
      Asclepan, qui ante decem et septem annos episcopos honore
      discinctus est}
    (\edpageref{Asclepas:Serdica},\lineref{Asclepas:Serdica}--\lineref{Asclepas2:Serdica}). Ist
    diese Angabe korrekt, so k�me man bei einer Datierung der Synode
    von Serdica in das Jahr 343 auf das Jahr 327. Da dieses Regest von
    einer Verurteilung des Asclepas\index[namen]{Asclepas!Bischof von
      Gaza} auf einer Synode in
    Antiochien\index[synoden]{Antiochien!a. 327} unter der Leitung des
    Eusebius von Caesarea\index[namen]{Eusebius!Bischof von Caesarea}
    ausgeht, ist die Absetzung des dortigen Bischofs
    Eustathius\index[namen]{Eustathius!Bischof von Antiochien}
    (s. folgendes Regest) wohl vorauszusetzen. Der Hinweis im
    �stlichen Synodalschreiben von Serdica (Dok.
    \ref{sec:RundbriefSerdikaOst},14), da�
    Athanasius\index[namen]{Athanasius!Bischof von Anazarba} dem
    zugestimmt habe, l��t eventuell ein sp�teres Datum vermuten, da
    Athanasius\index[namen]{Athanasius!Bischof von Anazarba} erst 328
    Bischof wurde, andererseits kann die Formulierung auch so gedeutet
    werden, da� Athanasius\index[namen]{Athanasius!Bischof von
      Anazarba} sp�ter die schon erfolgte Verurteilung
    best�tigte. Geh�ren die in Dok. \ref{sec:SerdicaRundbrief},6
    erw�hnten ">gef�lschten"< Briefe des Theognis von
    Nicaea\index[namen]{Theognis!Bischof von Nicaea} gegen
    Asclepas\index[namen]{Asclepas!Bischof von Gaza} (sowie gegen
    Athanasius\index[namen]{Athanasius!Bischof von Anazarba} und
    Markell\index[namen]{Markell!Bischof von Ancyra}) in diesen
    Zusammenhang, wird die Datierung problematisch, da
    Theognis\index[namen]{Theognis!Bischof von Nicaea} als in
    Nicaea\index[synoden]{Nicaea!a. 325} Verurteilter 327 nicht in der
    Position gewesen sein d�rfte, diese Verurteilung
    voranzutreiben. Eventuell geh�ren diese Briefe aber auch zu einer
    zweiten Verurteilung des Asclepas\index[namen]{Asclepas!Bischof
      von Gaza} aufgrund von Tumulten nach seiner R�ckkehr im Jahr 337
    (vgl. auch Dok.  \ref{sec:RundbriefSerdikaOst},10; Ath., fug. 3,3;
    h.\,Ar. 5,2; Soz., h.\,e. III 8,1; Socr., h.\,e.  II 15,2). Der
    Hinweis bei Thdt., h.\,e. I 29,7, da�
    Asclepas\index[namen]{Asclepas!Bischof von Gaza} auf der Synode in
    Tyrus\index[synoden]{Tyrus!a. 335} wegen falscher Lehre angeklagt
    worden sei, l��t sich mit den �brigen Regesten nicht in
    �bereinstimmung bringen.
  \item[�berlieferung] Vgl. die Einleitung zu
    Dok. \ref{sec:SerdicaRundbrief}.
  \item[Fundstelle] Dok. \ref{sec:SerdicaRundbrief},12 (Thdt.,
    h.\,e. II 8,26 [\editioncite[108,15--18]{Hansen:Thdt}]; Hil.,
    coll.\,antiar. B II 1,6 [\editioncite[118,3--6]{Hil:coll}]; Cod.\,Ver. LX f. 84b; Ath., apol.\,sec. 45,2 [\editioncite[122,3--6]{Opitz1935}])
  \end{description}
\end{praefatio}
\begin{pairs}
\begin{Leftside}
\beginnumbering
\selectlanguage{polutonikogreek}
\pstart
%%%%%%%%%%%%% aus Dok. 43.1, 12 %%%%%%%%%%%%%%%%%%%%%%%%%%%%%%%%%
%%%%%%%%%%%%% aktuellen Stand eingef�gt 1.5.2007 AvS %%%%%%%%%%%%%
\hskip -1,25em\edtext{\abb{}}{\killnumber\Cfootnote{\hskip -1em\latintext Thdt.(BAN+GS(s)=r LF=z T) Hil. Cod.\,Ver. Ath.(BKO  RE)}}\specialindex{quellen}{section}{Theodoret!h.\,e.!II 8,26}\specialindex{quellen}{section}{Hilarius!coll.\,antiar.!B II 1}\specialindex{quellen}{section}{Codices!Veronensis LX!f.
81a--86a}\specialindex{quellen}{section}{Athanasius!apol.\,sec.!42--47}
\edtext{\abb{ka`i}}{\Dfootnote{\latintext > Hil. \textit{etiam} Cod.\,Ver.}}
\edtext{>Asklhp~ac}{\Dfootnote{\latintext \textit{Asclepius} Hil.
\textit{Asclepas} Cod.\,Ver.}}\edindex[namen]{Asclepas!Bischof von Gaza}
\edtext{d`e}{\Dfootnote{\latintext \textit{sed} Hil. > Cod.\,Ver.}}
\edtext{<o sulleitourg`oc}{\Dfootnote{\latintext \textit{quoepiscopus noster}
Hil. \textit{conminister} Cod.\,Ver.}}
\edtext{pro'hnegken}{\Dfootnote{pros'hnegken \latintext Thdt.(BAszT)
\textit{protulit} Hil. Cod.\,Ver.}}
\edtext{<upomn'hmata}{\Dfootnote{\latintext \textit{acta} Hil. \textit{gesta}
Cod.\,Ver.}}
\edtext{gegenhm'ena}{\Dfootnote{gen'omena \latintext Ath. \textit{confecta} Cod.\,Ver. \textit{quae confecta sunt} Hil.}}
\edtext{>en >Antioqe'ia|}{\Dfootnote{>enant'ia \latintext Thdt.(s) \textit{apud
Anthiociam} Hil. \textit{aput Anthiochiam} Cod.\,Ver.}}\edindex[synoden]{Antiochien!a. 327},
\edtext{\edtext{par'ontwn}{\Dfootnote{par'ontwn ka`i \latintext Thdt.(A)
\greektext par`a \latintext Thdt.(BN)}}
\edtext{\abb{t~wn}}{\Dfootnote{\latintext > Thdt.(sFT)}}
kathg'orwn}{\Dfootnote{\latintext ] \textit{praesentibus adversariis} Hil.
\textit{praesentibus accusatoribus} Cod.\,Ver.}} ka`i E>useb'iou\edindex[namen]{Eusebius!Bischof von Caesarea} to~u >ap`o
Kaisare'iac; ka`i >ek t~wn >apof'asewn t~wn
\edtext{dikas'antwn}{\Dfootnote{dikast~wn \latintext Thdt.(B)
\textit{iudicandum} Cod.\,Ver. \textit{iudicatum} Hil.}}
\edtext{>episk'opwn}{\Dfootnote{\latintext \textit{episcopum} Hil.}}
\edtext{\edtext{>'edeixen}{\Dfootnote{\latintext ostendisse Hil.}}
<eaut`on}{\Dfootnote{>'edeixan a>ut`on \latintext Thdt.(s)}}
\edtext{>aj~won}{\Dfootnote{\latintext \textit{inreprehensibilem} Hil.
\textit{innocentem} Cod.\,Ver.}} e>~inai.
\pend
\endnumbering
\end{Leftside}
\begin{Rightside}
\begin{translatio}
\beginnumbering
\pstart
\noindent Auch der Mitdiener Asklepas brachte Aufzeichnungen herbei, die in
Antiochia in Gegenwart seiner Ankl�ger und des Eusebius von Caesarea verfa�t
worden waren.\footnoteA{Dies ist die
einzige Erw�hnung von Verhandlungen gegen Asclepas in Antiochien unter der Leitung des
Eusebius von Caesarea. Der zugrundeliegende Anla� bleibt im Dunkeln, insbesondere wenn die
Bemerkung im �stlichen Synodalschreiben von Serdica zutreffen sollte, da� Athanasius
seinerseits damals dieser Absetzung zugestimmt habe (s.\,o. Einleitung). Eventuelle
theologische Motive sind unbekannt, �berliefert ist nur, da� er sp�ter mit Paulus von
Konstantinopel Kontakt pflegte (vgl. Dok. \ref{sec:RundbriefSerdikaOst},21) und da� sich
die westliche Teilsynode von Serdica f�r eine Kirchengemeinschaft mit ihm ausgesprochen
hat gegen den ausdr�cklichen Protest aus dem Osten (vgl. Dok. \ref{sec:RundbriefSerdikaOst},25;
Dok. \ref{sec:SerdicaRundbrief},15). Asclepas konnte anscheinend nach seiner zweiten
R�ckkehr ca. 347 noch eine zeitlang unbehelligt Bischof bleiben (Soz., h.\,e. II 24; Socr.,
h.\,e. II 23).} Und er konnte anhand
der Aussagen der urteilenden Bisch�fe aufzeigen, da� er unschuldig war.
\pend
\endnumbering
\end{translatio}
\end{Rightside}
\Columns
\end{pairs}