% \section{Das Edikt gegen Arius}
\kapitel{Fragment eines Ediktes des Kaisers Konstantin gegen Arius und seine Anh�nger (Urk.~33)}
\label{ch:33}
\thispagestyle{empty}

Der Sieger Konstantin der Gro�e, Augustus, an die Bisch�fe und an das Volk!

\kapnum{1}Arius,\looseness=-1\ der Schlechte und Gottlose nachgeahmt hat, ertr�gt zu Recht dieselbe Schmach wie
jene. Wie n�mlich Porphyrius, der Feind der Gottesf�rchtigkeit, diverse gesetzwidrige Gesch�tze
gegen die Gottesverehrung aufgeboten und daf�r den entsprechenden Lohn gefunden hat, so da� er
von der Nachwelt verschm�ht wird und in �beraus schlechtem Ruf steht und seine gottlosen B�cher von
der Bildfl�che verschwunden sind, so gefiel es auch jetzt, Arius und seine Anh�nger ebenfalls
Porphyrianer zu nennen, damit sie ihren Namen nach denen tragen, deren Art sie nachgeahmt
haben.

\kapnum{2}Ferner aber, wenn irgendwelche von Arius verfa�ten Texte gefunden werden, sind diese dem Feuer
zu �bergeben, damit nicht nur das �bel seiner Lehre verschwindet, sondern auch �berhaupt keine
Erinnerung an ihn mehr verbleibt. Dar�berhinaus bestimme ich, da�, wenn jemand dabei ertappt wird,
wie er von Arius verfa�te Texte verbirgt und nicht sogleich hervorholt und ins Feuer wirft, diesen
die Todesstrafe treffen soll. Sobald er dessen �berf�hrt wird, soll er die Todesstrafe erleiden.

\textit{Und mit anderer Hand:} Gott m�ge euch bewahren, geliebte Br�der!