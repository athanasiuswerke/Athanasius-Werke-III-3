\kapitel[Bericht des Athanasius �ber Konflikte um die Wiederherstellung der Kircheneinheit in �gypten][Konflikte um die Wiederherstellung der Kircheneinheit in �gypten]{Bericht des Athanasius �ber Konflikte um die Wiederherstellung der Kircheneinheit in �gypten}
\thispagestyle{empty}
% \label{ch:37}
\label{ch:Arius}
\begin{praefatio}
  \begin{description}
  \item[328--335]Der Brief Konstantins\index[namen]{Konstantin,
      Kaiser} kann nur ungef�hr in die Zeit zwischen der Ernennung des
    Athanasius\index[namen]{Athanasius!Bischof von Alexandrien} zum
    Bischof (ind.\,ep.\,fest. 3: 27.4.328) und den Verhandlungen gegen
    Athanasius\index[namen]{Athanasius!Bischof von Alexandrien} auf
    den Synoden von Caesarea\index[synoden]{Caesarea!a. 334} und
    Tyrus\index[synoden]{Tyrus!a. 335} (334/335) datiert
    werden. Athanasius\index[namen]{Athanasius!Bischof von
      Alexandrien} zitiert hier aus einem Brief des Kaisers, der ihn
    laut seiner eigenen Einf�hrung dazu aufgefordert haben soll, die
    Anh�nger des Arius\index[namen]{Arius!Presbyter in Alexandrien}
    wieder in die Kirche aufzunehmen.  Es ist aber zu beachten, da�
    der zitierte Kaiserbrief ">die um Arius"< nicht namentlich erw�hnt
    und sich der Bezug auf die Arianer nur aus dem Kontext bei
    Athanasius\index[namen]{Athanasius!Bischof von Alexandrien} (und
    ihm folgend wohl bei Socrates) ergibt. Das Fragment ist vielleicht
    eher ein Teil eines Briefes, in dem
    Konstantin\index[namen]{Konstantin, Kaiser}
    Athanasius\index[namen]{Athanasius!Bischof von Alexandrien} die
    bereitwilligere Aufnahme ehemaliger Melitianer (vgl. den Plural im
    Text: \griech{to~ic boulom'enois, tinas a>ut~wn} und die
    Umschreibung: \griech{to~ic boulom'enois e>ic t`hn >ekklhs'ian
      e>iselje~in}) befahl, wie es auch der Kontext bei Sozomenus
    nahelegt. Zur Frage, ob Arius selbst zu dieser Zeit noch lebte
    oder nicht, vgl. die einleitenden Bemerkungen zur Chronologie.
  \item[�berlieferung]Socr., Soz., Anon.\,Cyz. und v.\,Ath., die nur
    das Brieffragment zitieren, gehen auf eine Quelle zur�ck und sind
    wohl von Athanasius abh�ngig.
  \item[Fundstelle] Ath., apol.\,sec. 59,4--60,1
    (\editioncite[139,18--140,15]{Opitz1935}; \S 2: Socr., h.\,e. I
    27,4 (\editioncite[75,14--18]{Hansen:Socr}), Soz., h.\,e. II
    22,5 (\editioncite[79,11--15]{Hansen:Soz}), Anon.\,Cyz. III
    14 (\editioncite[136,8--12]{Hansen:Gel}) und v.\,Ath. (BHG 185) 2
    (\cite[CCXXV A/B]{Athanasius:PG})
  \end{description}
\end{praefatio}

\begin{pairs}
\selectlanguage{polutonikogreek}
\begin{Leftside}
\beginnumbering
\pstart 
\hskip -1.25em\edtext{\abb{}}{\killnumber\Cfootnote{\hskip -1em\latintext
    Ath.(BKO RE)}}\specialindex{quellen}{chapter}{Athanasius!apol.\,sec.!59,4--60,1}
\kap{1}E>us'ebioc\edindex[namen]{Eusebius!Bischof von Nikomedien}
to'inun to~uto maj`wn ka`i proist'amenoc t~hc >areian~hc a<ir'esewc
p'empei ka`i >wne~itai to`uc Melitiano`uc\edindex[namen]{Melitianer}
>ep`i polla~ic >epaggel'iaic. ka`i g'inetai m`en a>ut~wn kr'ufa
f'iloc, sunt'attetai d`e a>uto~ic e>ic <`on >ebo'uleto kair'on. t`hn
m`en o>~un >arq`hn pros'epempe protr'epwn d'exasja'i me \edtext{to`uc
  per`i}{\Dfootnote{t`on \latintext Ath.(B)}}
>'Areion\edindex[namen]{Arianer}, ka`i >agr'afwc m`en >hpe'ilei,
gr'afwn d`e >hx'iou.  >epeid`h d`e >ant\-'ele\-gon m`h qr~hnai f'askwn
deqj~hnai to`uc a<'iresin >efeur'ontac kat`a t~hc >alhje'iac ka`i
>anajematisj'entac par`a t~hc o>ikoumenik~hc sun'odou poie~i ka`i
basil'ea moi gr'ayai t`on makar'ithn
Kwnstant~inon\edindex[namen]{Konstantin, Kaiser} >apeil`hn >'eq\-on\-ta,
e>i m`h l'aboimi to`uc per`i >'Areion\edindex[namen]{Arianer}, ta~ut'a
me paje~in, <`a pr'oteron ka`i n~un p'eponja. t`o to'inun m'eroc t~hc
>epistol~hc >esti to~uto ka`i palat~inoi
Sugkl'htioc\edindex[namen]{Syncletius!Agens in rebus} ka`i
Gaud'entioc\edindex[namen]{Gaudentius!Agens in rebus} >~hsan o<i
kom'isantec t`a gr'ammata; 
\pend 
\pstart
\edtext{\abb{}}{\xxref{3}{2}\Cfootnote{\latintext
    v.\,Ath.}}\specialindex{quellen}{chapter}{Vita Athanasii!BHG 185!2
  (PG 25, CCXXV A/B)}
\kap{2}\edtext{\textit{M'eroc}}{\lemma{\abb{}}\Dfootnote{M'eroc + t~hc
    \dt{Ath.(KE)}}}\edlabel{3} \textit{>epistol~hc}
\edtext{\textit{to~u}}{\lemma{\abb{}}\Dfootnote{to~u \dt{> Ath.(E)}}}
\textit{basil'ewc Kwnstant'inou}\edindex[namen]{Konstantin,
  Kaiser}\edindex[namen]{Constantinus, Kaiser|see{Konstantin, Kaiser}}
\pend
\pstart
\edtext{\abb{}}{\xxref{1}{2}\Cfootnote{\latintext  Socr.(MF=b AT Arm.) Soz.(BC=b) Anon.\,Cyz.(A)}}\specialindex{quellen}{chapter}{Socrates!h.\,e.!I 27,4}\specialindex{quellen}{chapter}{Sozomenus!h.\,e.!II 22,5}\specialindex{quellen}{chapter}{Anonymus Cyzicenus!h.\,e.!III 14}
>'Eqwn\edlabel{1}  to'inun t~hc >em~hc
\edtext{boul'hsewc}{\Dfootnote{boul~hc \latintext Socr.(bArm.) Anon.\,Cyz. v.\,Ath.}} t`o
gn'wrisma \edtext{<'apasi}{\Dfootnote{p~asi \latintext Socr. Soz. Anon.\,Cyz. v.\,Ath.}} to~ic
boulom'enoic e>ic t`hn >ekklhs'ian e>iselje~in >ak'wluton
\edtext{par'asqou}{\Dfootnote{par'asqe \latintext Anon.\,Cyz.}} t`hn e>'isodon. >e`an g`ar
\edtext{\abb{gn~w}}{\Dfootnote{\latintext > v.\,Ath.}}
\edtext{\abb{<wc}}{\Dfootnote{\latintext Ath.(R\corr)  Socr.(bA Arm.) Soz. Anon.\,Cyz.
\greektext + e>i \latintext Ath.(BKOR*E*) v.\,Ath. \greektext <wse`i >'h \latintext
Ath.(E\corr) Socr.(T)}} kek'wluk'ac \edtext{tinac
\edtext{\abb{a>ut~wn}}{\Dfootnote{\latintext >
Anon.\,Cyz.}}}{\lemma{\abb{}}\Dfootnote{\responsio\ a>ut~wn tinac \latintext Soz.}} t~hc
\edtext{>ekklhs'iac}{\Dfootnote{>ekklhsiastik~hc \latintext Anon.\,Cyz.}}
\edtext{\abb{metapoioum'enouc}}{\Dfootnote{+ p'istewc \latintext Anon.\,Cyz.}} >`h
\edtext{\abb{>ape~irxac}}{\Dfootnote{+ to`uc toio'utouc \latintext Anon.\,Cyz.}} t~hc
e>is'odou, \edtext{>apostel~w}{\Dfootnote{>apost'ellw \latintext
Socr.(M\textsuperscript{1}Arm.) \greektext >apostell~w \latintext Socr.(A)}}
\edtext{paraqr~hma}{\Dfootnote{paraut'ika \latintext Ath.(B)}} t`on
\edtext{\abb{ka`i}}{\Dfootnote{\latintext > Socr.(bA Arm.) Soz. Anon.\,Cyz.}} kajair'hsont'a
se >ex >em~hc kele'usewc ka`i \edtext{t~wn t'opwn}{\Dfootnote{t`on t'opon \latintext
Ath.(K) Socr.(M\textsuperscript{1}) v.\,Ath. \greektext to~u t'opou \latintext
Socr.(M\textsuperscript{r}) \greektext to~u t'opou to'utou \latintext Socr.(Arm.) +
\greektext se \latintext v.\,Ath.}} \edtext{metast'hsonta}{\Dfootnote{metast'hsanta
\latintext Ath.(E)}}.\edlabel{2}
\pend
\pstart
\kap{3}>Epeid`h to'inun \edtext{\abb{ka`i}}{\Dfootnote{\latintext > Ath.(E)}} basil'ea
gr'afwn >'epeijon mhdem'ian \edtext{e>~inai
koinwn'ian}{\lemma{\abb{}}\Dfootnote{\responsio\ koinwn'ian e~>inai \latintext Ath.(KO)}}
t~h| qristom'aqw| a<ir'esei pr`oc t`hn kajolik`hn >ekklhs'ian, t'ote loip`on E>us'ebioc\edindex[namen]{Eusebius!Bischof von Nikomedien}
t`on kair`on <`on sunef'wnhse met`a t~wn Melitian~wn\edindex[namen]{Melitianer} prof'erwn gr'afei ka`i pe'ijei
\edtext{to'utouc}{\Dfootnote{to'utoic \latintext Ath.(E)}} pl'asasjai pr'ofasin, <'in'',
<'wsper kat`a P'etrou\edindex[namen]{Petrus!Bischof von Alexandrien} ka`i >Aqill~a\edindex[namen]{Achillas!Bischof von Alexandrien} ka`i >Alex'androu\edindex[namen]{Alexander!Bischof von Alexandrien} memelet'hkasin, o<'utw ka`i kaj''
<hm~wn >epino'hswsi ka`i jrul'hswsi.
\pend
\endnumbering
\end{Leftside}
\begin{Rightside}
\begin{translatio}
\beginnumbering
\pstart
\noindent\kapR{1}Als Eusebius (von Nicomedia), der auch die arianische
H�resie anf�hrte, also dies erfuhr\footnoteA{Gemeint sind erneute Proteste der Melitianer
(vgl. Dok. \ref{ch:23} = Urk. 23) in Alexandrien, die sich wohl auf die Bischofswahl des Athanasius bezogen
haben d�rften (Soz., h.\,e. II 17,4; Philost., h.\,e. II 11; Epiph., haer. 68,7,3~f.; 69,11,4; vgl. Ath., apol.\,sec. 6,4).}, schickte er nach den Melitianern und kaufte sie mit vielen Versprechungen.
Und er wurde heimlich ihr Freund und verband sich mit ihnen so lange, wie es ihm
gefiel. Als erstes nun wandte er sich an mich und legte mir nahe, die um Arius
aufzunehmen, das hei�t m�ndlich drohte er mir, schriftlich bat er mich. Da ich aber
widersprach und erkl�rte, da� man nicht die aufnehmen darf, die im Widerspruch zur
Wahrheit eine H�resie erfinden und von der �kumenischen Synode verurteilt wurden, erwirkte
er, da� auch der Kaiser, der selige Konstantin, mir schrieb und damit drohte, da� ich
dieses erleiden werde, was ich zuvor erlitten habe und auch jetzt noch erleide\footnoteA{Athanasius
blickt auf seine wiederholten Vertreibungen aus Alexandrien und jahrelangen Aufenthalte im
Exil zur�ck. Da nicht eindeutig ist, wann Athanasius die apol.\,sec. verfa�t oder auch
erg�nzt hat, bleibt unklar, auf welchen Zeitpunkt sich dieses ">jetzt"< bezieht (2. oder
3. Exil?).}, falls ich die um Arius nicht aufnehme. Das Folgende ist ein Auszug aus
diesem Brief, und Syncletius und Gaudentius, die Palatini\footnoteA{Syncletius und
Gaudentius sind auch die �berbringer von Dok. \ref{ch:33} = Urk. 33 und Dok. \ref{ch:34} = Urk. 34; dort werden sie als
Magistrianoi (\textit{agentes in rebus}) bezeichnet.}, waren es, die
den Brief �berbrachten.
\pend
\pstart
\kapR{2}\textit{Auszug aus einem Brief Kaiser Konstantins:}
\pend
\pstart
Da du also Kenntnis von meinem Willen hast, gew�hre ungehinderten
Zutritt allen, die in die Kirche hinein wollen. Denn wenn ich erfahre, da� du welche von ihnen, die
nach der Kirchengemeinschaft streben, daran gehindert oder es verweigert hast, sie
einzulassen, werde ich sofort jemanden schicken, der dich auf meinen Befehl hin absetzen
und von deinem Sitz vertreiben wird.
\pend
\pstart
\kapR{3}Als ich daraufhin auch dem Kaiser schrieb\footnoteA{Ein entsprechender Brief ist
nicht �berliefert; vgl. die einleitenden Bemerkungen.} und ihn davon �berzeugte, da� die
christusfeindliche H�resie auf keinen Fall mit der katholischen Kirche Gemeinschaft habe,
da ergriff Eusebius also die Gelegenheit zu dem Zeitpunkt, den er mit den
Melitianern ausgemacht hatte, schrieb ihnen und
�berzeugte sie, einen Vorwand zu erfinden, um so, wie sie gegen Petrus,
Achillas und Alexander vorgegangen waren, auch gegen uns Pl�ne zu schmieden und Ger�chte zu
verbreiten.
\pend
\endnumbering
\end{translatio}
\end{Rightside}
\Columns
\end{pairs}
% \thispagestyle{empty}
%%% Local Variables: 
%%% mode: latex
%%% TeX-master: "dokumente_master"
%%% End: 
