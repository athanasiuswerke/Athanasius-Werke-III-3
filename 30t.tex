% \section{Brief der Presbyter Arius und Euzoius an Kaiser Konstantin}
\kapitel{Brief des Arius und des Euzoius an Kaiser Konstantin (Urk.~30)}
\label{ch:30}

Unseren ehrw�rdigsten und gottgeliebtesten Herrscher, Kaiser Konstantin, gr��en Arius und Euzoius.

\kapnum{1}Wie deine gottgeliebte Fr�mmigkeit geboten hat, Herr Kaiser, legen wir unseren
pers�nlichen Glauben vor und bekennen schriftlich vor Gott, da� wir und alle unsere Mitstreiter
folgenderma�en glauben:

\kapnum{2}Wir glauben an einen Gott, den Vater, den Allm�chtigen, und an den Herrn Jesus Christus, seinen
eingeborenen Sohn, der aus ihm vor allen Zeiten geboren worden ist, Gott, Wort, durch den alles
wurde, das im Himmel und das auf Erden, der herabstieg und Fleisch annahm, litt, auferstand und
hinaufstieg in den Himmel und der wieder kommen wird die Lebenden und die Toten zu richten.

\kapnum{3}Und an den heiligen Geist, die Auferstehung des Fleisches, an ein Leben im kommenden �on,
an das Himmelreich, an eine katholische Kirche Gottes vom Anfang bis zum Ende der Erde.

\kapnum{4}Diesen Glauben aber haben wir aus den heiligen Evangelien empfangen, wie es der Herr
seinen eigenen J�ngern sagte: ">Geht und lehrt alle V�lker, tauft sie auf den Namen des Vaters und
des Sohnes und des heiligen Geistes."< Falls wir aber dies nicht so glauben und nicht in Wahrheit
den Vater und den Sohn und den heiligen Geist annehmen, wie es die ganze katholische Kirche und die
Schriften lehren, an die wir in jeder Hinsicht glauben, so ist Gott unser Richter, sowohl jetzt als
auch an dem kommenden Tag.

\kapnum{5}Daher bitten wir deine Fr�mmigkeit, gottgeliebtester Kaiser, uns, die wir Anh�nger der
Kirche sind und am Glauben und Denken der Kirche und der heiligen Schriften festhalten, durch deine
friedensstiftende und gottesf�rchtige Fr�mmigkeit mit unserer Mutter, der Kirche also, wieder zu
vereinigen, da die Differenzen und die daraus erwachsenden Wortgefechte aufgehoben worden sind, so
da� wir und die Kirche nun miteinander Frieden halten und alle gemeinsam die gewohnten Gebete f�r
dein friedliches und frommes Kaisertum und f�r dein ganzes Geschlecht verrichten werden.