% \section{Regest des Synodalschreibens der Synode in Pal�stina}
\kapitel{Regeste eines Briefes (?) des Arius an Paulinus von Tyrus, Eusebius von Caesarea und an Patrophilus von Scythopolis und eines Briefes einer Synode in Palaestina (Urk.~10)}
\thispagestyle{empty}
\label{ch:10}

Als ihnen aber wider Erwarten ihre Bem�hungen nichts einbrachten, da
Alexander nicht nachgab, da wandte sich Arius an Paulinus, den Bischof von Tyrus, an
Eusebius, den Sohn des Pamphilus, der der Kirche von Caesarea in Palaestina vorstand, und an
Patrophilus von Scythopolis mit der Forderung, mit seinen Mitstreitern die Erlaubnis zu bekommen,
mit der auf seiner Seite stehenden Gemeinde Gottesdienst abzuhalten und dabei wie fr�her den Rang
von Presbytern einzunehmen. Es sei n�mlich Sitte in Alexandrien (wie es auch jetzt noch
ist), da�, w�hrend ein Bischof �ber allen stehe, die Presbyter f�r sich allein den
Kirchen vorstehen und die Gemeinden in ihnen versammeln. 
Sie aber kamen in Palaestina mit anderen Bisch�fen zusammen und bef�rworteten die Bitte des Arius
und geboten, sich wie fr�her zu versammeln, aber Alexander unterzuordnen und ihn zu bitten,
best�ndig am Frieden und der Gemeinschaft mit ihm teilzuhaben.