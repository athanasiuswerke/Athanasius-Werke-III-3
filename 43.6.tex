% \cleartooddpage
\section[Fragment des Briefes des Ossius von Cordoba und des Protogenes von
Serdica an Julius von Rom][Ossius von Cordoba und Protogenes von
Serdica an Julius von Rom]{Fragment des Briefes des Ossius von Cordoba und des Protogenes von
Serdica\\an Julius von Rom}
% \label{sec:43.2}
\label{sec:BriefOssiusProtogenes}
% \diskussionsbedarf
\begin{praefatio}
  \begin{description}
  \item[Herbst 343]Zum Datum vgl. Einleitung zu
    Dok. \ref{ch:SerdicaEinl}. Der f�hrende Bischof der ">westlichen"<
    Synode, Ossius\index[namen]{Ossius!Bischof von Cordoba}
    (vgl. Anm. zu Dok. \ref{sec:SerdicaRundbrief},5), und der
    Ortsbischof von Serdica schreiben an
    Julius\index[namen]{Julius!Bischof von Rom} von Rom, um die
    Abfassung einer l�ngeren Glaubenserkl�rung zu
    rechtfertigen. Wahrscheinlich handelt es sich um den Begleitbrief,
    dem die Glaubenserkl�rung als Anlage beigef�gt wurde.
  \item[�berlieferung]Der Text ist nur sehr schlecht im Codex
    Veronensis �berliefert (vgl. den textkritischen
    Apparat). S. \edpageref{lacuna1},\lineref{lacuna1} ist eine L�cke
    anzunehmen; ebenso ist der Schlu� des Briefes nicht �berliefert.
    Bei der uns vorliegenden lateinischen Fassung handelt es sich um
    eine �bersetzung aus dem Griechischen; darauf deutet die
    Konstruktion des Satzes S. \refpassage{ne-1}{ne-2} (\textit{ne
      quis} = \griech{<'ina mhde`ic} wg. \textit{et excludatur};
    \textit{ne fiat} wird dann wieder mit \textit{ne} konstruiert, da
    dies im Griechischen notwendig ist; f�r das Lateinische
    au�ergew�hnlich ist zudem die Sperrung zwischen \textit{coegit}
    und \textit{exponere}). Zur �berlieferung des Codex Veronensis
    vgl. Dok.\ref{sec:SerdicaRundbrief}. 
  \item[Fundstelle]\refpassage{cod01}{cod02} Codex Veronensis LX,
    f. 80b--81a; \refpassage{soz01}{soz02} Soz., h.\,e. III 12,6
    (\editioncite[116,19--25]{Hansen:Soz})
  \end{description}
\end{praefatio}
\begin{pairs}
\selectlanguage{latin}
\begin{Leftside}
% \beginnumbering
\pstart
\hskip -1.2em\kap{1,1}\edtext{\abb{}}{\xxref{cod01}{cod02}\Cfootnote{Cod.\,Ver.}}
\specialindex{quellen}{section}{Codices!Veronensis LX!f.
80b--81a}Dilectissimo\edlabel{cod01} fratri Iulio\edindex[namen]{Julius!Bischof von Rom}
Osius\edindex[namen]{Ossius!Bischof von Cordoba} \edtext{\abb{et Protogenes}}{\Dfootnote{\textit{cod.\corr} et pro
et Protogenes \textit{cod*}}}\edindex[namen]{Protogenes!Bischof von
Serdica}.
\pend
\pstart
Meminimus et tenemus et habemus illam scripturam 
\edtext{\abb{quae}}{\Dfootnote{\textit{coni. Ballerini} que \textit{cod.}}} continet catholicam fidem factam
aput Niceam\edindex[synoden]{Nicaea!a. 325} et consenserunt omnes qui aderant episcopi.
tres enim questiones 
\edtext{\abb{motae}}{\Dfootnote{\textit{coni. Ballerini} mote \textit{cod.}}} sunt:
\edtext{%
\edtext{\abb{quod}}{\Dfootnote{\textit{coni. Opitz} quad \textit{cod.}}}
erat quando non erat}{\lemma{quod erat quando non erat}\Dfootnote{quod erat aliquando quando non erat, et quia ex nullis existentibus factus est, aut ex alia substantia vel essentia dicunt esse convertibilem, aut mutabilem Filium Dei \textit{susp. Ballerini} \Ladd{quod de non exstantibus est filius deo
vel ex alia substantia et non ex deo et} quod erat \Ladd{aliquando} quando non erat
\textit{coni. Tetz}}} \edtext{\abb{\Ladd{\dots}}}{\Dfootnote{\textit{lacunam susp. Opitz}}}.\edlabel{lacuna1} 
\edtext{sed quoniam post hoc discipuli Arrii
\edtext{\abb{blasphemias}}{\Dfootnote{\textit{vel} blasphemi \textit{susp. Ballerini} blasphemiae \textit{cod.}}}
conmoverunt}{\lemma{\abb{\responsio\ sed quoniam \dots\ conmoverunt}}
\Dfootnote{\textit{ante} tres enim \dots\ \textit{coni. Opitz}}}, ratio 
\edtext{\abb{quaedam}}{\Dfootnote{\textit{coni. Ballerini} quedam \textit{cod.}}} coegit,
ne\edlabel{ne-1} quis ex illis tribus argumentis
\edtext{\abb{circumventus}}{\Dfootnote{\textit{cod.\corr} circummentus \textit{cod.*}}}
\edtext{\abb{renovet}}{\Dfootnote{\textit{coni. Ballerini} renovent \textit{cod.} renuerit
\textit{coni. Opitz} removeat \textit{coni. Tetz}}} fidem 
\edtext{et excludatur eorum spolium et
\edtext{\abb{ne fiat}}{\Dfootnote{\textit{cod.\corr} ne fia \textit{cod.*} nefas \textit{coni. Tetz}}}
\edtext{latior et longior}{\Dfootnote{latiorem et longiorem \textit{coni. Opitz} latius
et longius \textit{coni. Tetz}}},
\edtext{exponere}{\lemma{\abb{}} \Dfootnote{\textit{Tetz interpunxit post}
exponere}} priori
\edtext{consentientes}{\Dfootnote{consentientem \textit{coni. Opitz}}}}{\Dfootnote{ut excludatur eorum scholium adversus Nicaenam fidem, et fiat latior et longior expositio priori consentientes \textit{susp. Ballerini e Sozomeno}}}\edlabel{ne-2}.
\pend
\pstart
\kap{2}ut igitur nulla
\edtext{\abb{reprehensio}}{\Dfootnote{\textit{coni. Ballerini} reprehentio \textit{cod.}}} fiat, 
\edtext{\abb{haec}}{\Dfootnote{\textit{coni. Ballerini} hec \textit{cod.}}}
significamus 
\edtext{\abb{tuae}}{\Dfootnote{\textit{coni. Ballerini} tue \textit{cod.}}} bonitati, frater dilectissime.
\edtext{\abb{priora}}{\Dfootnote{\textit{coni. Opitz} plura \textit{cod.}}} placuerunt
firma esse et fixa et haec plenius cum quadam sufficientia veritatis
dictari, ut omnes docentes et caticizantes clarificentur et
\edtext{\abb{repugnantes}}{\Dfootnote{\textit{susp. Ballerini} prepugnantes \textit{cod}
propugnantes \textit{coni. Ballerini}}} obruantur et teneant catholicam et apostolicam
fidem.
\pend
\pstart
\noindent\edtext{\abb{\Ladd{\dots}}}{\Dfootnote{\textit{lacunam susp. Erl.}}}\edlabel{cod02}
\pend
% \endnumbering
\end{Leftside}
\begin{Rightside}
\begin{translatio}
\beginnumbering
\pstart
\noindent Ossius\looseness=-1\ und Protogenes an den vielgeliebten Bruder Iulius.
\pend
\pstart
Wir gedenken jener Schrift, die den in Nicaea beschlossenen Glauben enth�lt, halten an ihr
fest und bewahren sie, und alle Bisch�fe, die anwesend waren, haben zugestimmt. Drei
Fragestellungen sind n�mlich aufgeworfen worden: Da� (eine Zeit) war, als er nicht war,
\Ladd{\dots}.\footnoteA{Die L�cke ist entsprechend der
niz�nischen Anathematismen zu erg�nzen.} Aber weil danach die Sch�ler des Arius
Blasphemien aufgebracht haben, zwang ein gewisses Ma� an Vernunft dazu, da� wir diesen in �bereinstimmung mit
dem fr�heren Glauben erl�utern, damit nicht irgendeine Zusammenkunft infolge
jener drei Streitfragen den Glauben erneuert und damit ausgeschlossen wird, da� der Glaube zur
Beute jener Leute wird, und damit es nicht geschieht, da� er erweitert oder erg�nzt wird.
\pend
\pstart
Damit es also zu keinem Tadel kommt, weisen wir deine G�te, vielgeliebter Bruder, auf
folgendes hin: Man hat beschlossen, da� das Fr�here bekr�ftigt und festgelegt ist, ebenso da�
dies ausf�hrlicher gesagt und damit sozusagen der Wahrheit gen�ge getan wird. Dadurch sollen alle, die
lehren und unterweisen, erleuchtet und die, die Gegenwehr leisten, zum Schweigen gebracht
werden und alle am katholischen und apostolischen Glauben festhalten.
\pend
\pstart
\noindent\Ladd{\dots}
\pend
\endnumbering
\end{translatio}
\end{Rightside}
\Columns
\end{pairs}

\autor{Regest bei Sozomenus}
\begin{pairs}
\selectlanguage{polutonikogreek}
\begin{Leftside}
\pstart
\edtext{\abb{}}{\xxref{soz01}{soz02}\Cfootnote{\dt{Soz. (BC=b T)}}}\specialindex{quellen}{section}{Sozomenus!h.\,e.!III 12,6}
% \kap{2,1}>ex'ejento d`e ka`i a>uto`i
% \edtext{\abb{thnika~uta}}{\Dfootnote{\dt{> Soz.(T)}}}
% p'istewc graf`hn <et'eran, platut'eran m`en t~hc >en Nika'ia|\edindex[synoden]{Nicaea!a. 325}, ful'attousan d`e
% t`hn a>ut`hn di'anoian ka`i o>u par`a pol`u diall'attousan t~wn >eke'inhc
% \edtext{<rhm'atwn}{\Dfootnote{<rht~wn \dt{Soz.(T)}}\lemma{\abb{<rhm'atwn}}\Cfootnote{\dt{des. Soz.(T)}}}.
% \pend
% \pstart
% \selectlanguage{polutonikogreek}
\hskip -1.35em\kap{2,1}>am'elei\edlabel{soz01} <'Osioc\index[namen]{Ossius!Bischof von Cordoba} ka`i Prwtog'enhc\edindex[namen]{Protogenes!Bischof von Serdica}, o<`i t'ote \edtext{<up\-~hrqon
>'arqontec}{\Dfootnote{>~hrqon \dt{Soz.(C)}}}
t~wn >ap`o t~hc d'usewc >en Sardik~h|\index[synoden]{Serdica!a. 343} sunelhluj'otwn, de'isantec >'iswc, m`h
nomisje~i'en tisi kainotome~in t`a d'oxanta to~ic >en Nika'ia|\edindex[synoden]{Nicaea!a. 325}, >'egrayan
>Ioul'iw|\edindex[namen]{Julius!Bischof von Rom} ka`i >emart'uranto k'uria t'ade <hge~isjai, kat`a qre'ian d`e
safhne'iac t`hn a>ut`hn di'anoian plat~unai, <'wste m`h >eggen'esjai to~ic t`a
>Are'iou\edindex[namen]{Arianer} frono~usin >apokeqrhm'enoic t~h| suntom'ia| t~hc graf~hc e>ic >'atopon
<'elkein to`uc >ape'irouc dial'exewc.\edlabel{soz02}
\pend
% \endnumbering
\end{Leftside}
\begin{Rightside}
\begin{translatio}
\beginnumbering
\pstart
\noindent Im �brigen schrieben Ossius und Protogenes, die damals die Vorsitzenden der aus
dem Westen in Serdica Zusammengekommenen waren, vielleicht aus Furcht, da�
einige glauben k�nnten, sie wollten die Beschl�sse im Vergleich zu denen in
Nicaea erneuern, an Julius und bezeugten, da� sie diese f�r g�ltig hielten, da�
sie aber denselben Sachverhalt der Deutlichkeit wegen breiter dargestellt h�tten,
damit nicht die, die mit den Lehren des Arius sympathisieren, die M�glichkeit
h�tten, aufgrund der Knappheit der Urkunde die in Diskussionen Unerfahrenen in
Verlegenheit zu bringen.
\pend
\endnumbering
\end{translatio}
\end{Rightside}
\Columns
\end{pairs}
% \thispagestyle{empty}
%%% Local Variables: 
%%% mode: latex
%%% TeX-master: "dokumente_master"
%%% End: 
