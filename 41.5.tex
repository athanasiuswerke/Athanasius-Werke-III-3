%%%%%%%%%%%%%% Antiochenum I %%%%%%%%%%%%
\selectlanguage{german}
% \cleartooddpage
\section[Fragment eines Rundbriefes der Synode von Antiochien des Jahres 341 (1. antiochenische Formel)][Fragment eines Rundbriefes der Synode von Antiochien des Jahres 341]{Fragment eines Rundbriefes der Synode von Antiochien des Jahres 341\\(1. antiochenische Formel)}
\label{sec:AntI}
\begin{praefatio}
  \begin{description}
  \item[Anfang 341]Zur Datierung vgl. die Bemerkungen zu
    Dok. \ref{sec:BerichteAntiochien341}.  Dieses Textst�ck geh�rt zu
    einem Rundbrief, den die Synode an alle Bisch�fe verschickt hat,
    wie sowohl Socr., h.\,e. II 10,9 (\griech{ta~uta m`en >en t~h|
      pr'wth| >epistol~h| gr'ayantec to~ic kat`a p'olin >'epempon}
    [\editioncite[100,13]{Hansen:Socr}]) als auch Soz., h.\,e. III 5,5 (\griech{gr'ammata
      d`e diep'emyanto to~ic kat`a p'olin >episk'opoic} [\editioncite[196,10~f.]{Hansen:Soz}]) berichten. Hauptthema des Fragments ist die Abwehr
    des Vorwurfs, Arianer\index[namen]{Arianer} zu sein. Genau dieses
    hatte Athanasius\index[namen]{Athanasius!Bischof von Alexandrien}
    den �stlichen Bisch�fen unterstellt, zuletzt in seiner
    ep.\,encycl. (2,2~f. [\editioncite[171,3--7]{Opitz1935}]; 7,1 [\editioncite[176,11--13]{Opitz1935}]) von 339; so
    k�nnte dieser Rundbrief aus Antiochien\index[namen]{Antiochien}
    auch als Reaktion und Gegenst�ck zur \textit{Epistula encyclica} des
    Athanasius zu verstehen sein. Wahrscheinlich wurde dieser Brief
    auch den r�mischen Gesandten in
    Antiochien\index[namen]{Antiochien} zus�tzlich zur
    Antwort der Synode auf die Aufforderung des Julius mitgegeben,\index[namen]{Julius!Bischof
      von Rom} zu einer r�mischen Synode
    (vgl. Dok. \ref{sec:BriefJulius}und
    \ref{sec:BriefSynode341}) zu erscheinen. Nach Dok. \ref{sec:BriefJuliusII},22
    reisten diese Gesandten Januar 341 nach Rom\index[namen]{Rom} zur�ck. Die
    theologische Erkl�rung erweist sich z.\,T. als Kurzfassung der sogenannten zweiten
    antiochenischen Formel.
  \item[�berlieferung]Das Fragment des Rundbriefes �berliefern
    Athanasius und Socrates; Sozomenus bietet h.\,e. III 5,5--7 nur
    ein Regest. Die Quelle der Kirchenhistoriker d�rfte �ber
    Athanasius hinaus Sabinus von Heraclea sein, da sich nur bei Socr. und Soz. der Hinweis
    findet, da� es sich um einen Rundbrief handelt. Ferner k�nnte man
    aus dem Referat des Sozomenus (h.\,e. III 5,5~f.:
    \foreignlanguage{polutonikogreek}{piste'uein d`e sf~ac kat`a t`hn
      >ex >arq~hc paradoje~isan p'istin. e>~inai d`e ta'uthn <`hn
      <up'etaxan t~h| a>ut~wn >epistol~h|}\dots
    [\editioncite[106,13~f.]{Hansen:Soz}]) schlie�en, da� die eigentliche
    Glaubenserkl�rung der Synode (Dok. \ref{sec:AntII}), die bei Athanasius und Socrates ohne
    Unterbrechung gleich folgt (ab Z. \lineref{Par3}), ein Anhang zum
    Brief gewesen ist und erst von Athanasius und ihm folgend von
    Socrates an diese Stelle eingef�gt wurde. Wie in
    Dok. \ref{sec:AntII} ist auch hier die Fassung des Athanasius
    vorzuziehen.
  \item[Fundstelle]Ath., syn. 22,3--7 (\editioncite[248,30--249,8]{Opitz1935}); Socr.,
    h.\,e. II 10,4--8 (\editioncite[99,20--100,12]{Hansen:Socr})
  \end{description}
\end{praefatio}
\begin{pairs}
\selectlanguage{polutonikogreek}
\begin{Leftside}
% \beginnumbering
\pstart
\hskip -1.2em\edtext{\abb{}}{\killnumber\Cfootnote{\hskip -1em\latintext Ath.(BKPO R) Socr. (MF=b
AT)}}\specialindex{quellen}{section}{Athanasius!syn.!22,3--7}\specialindex{quellen}{section}{Socrates!h.\,e.!II 10,4--8}
\kap{1}\looseness=-1\ <Hme~ic o>'ute >ak'oloujoi >Are'iou\edindex[namen]{Arius!Presbyter in Alexandrien}
geg'onamen -- p~wc
g`ar >ep'iskopoi >'ontec \edtext{>akoloujo~umen} {\Dfootnote{>akolouj'hsomen \latintext
Socr.}} presbut'erw|? -- o>'ute >'allhn tin`a p'istin par`a t`hn >ex >arq~hc
\edtext{paradoje~isan}{\Dfootnote{>ekteje~isan \latintext Socr.(bA)}}
\edtext{>edex'ameja}{\Dfootnote{>exedex'ameja \latintext Socr.(T)}}, 
\kap{2}\edtext{>all`a
ka`i a>uto`i}{\Dfootnote{>all'' a>uto`i \latintext Socr.(T) \greektext >all`a ka`i <hme~ic
\latintext Socr.(bA)}} >exetasta`i ka`i dokimasta`i t~hc p'istewc a>uto~u gen'omenoi
m~allon a>ut`on proshk'ameja \edtext{>'hper}{\Dfootnote{e>'iper \latintext Socr.(T)}}
>hkolouj'hsamen; \edtext{gn'wsesje d`e}{\Dfootnote{ka`i gn'wsesje \latintext Socr.(bA)}} 
>ap`o t~wn legom'enwn.
\pend
\pstart
\kap{3}\edlabel{Par3}memaj'hkamen g`ar \edtext{\abb{>ex >arq~hc}}{\Dfootnote{\latintext >
Socr.(T)}} e>ic \edtext{\abb{<'ena}}{\Afootnote{\latintext vgl. 1Cor 8,6; Eph
4,6}}\edindex[bibel]{Korinther I!8,6|textit}\edindex[bibel]{Epheser!4,6|textit}
\edtext{\abb{je`on
t`on}}{\Dfootnote {\latintext > Socr.(T)}} t~wn <'olwn je`on piste'uein, t`on
\edtext{\abb{p'antwn}}{\Dfootnote{+ t~wn \latintext Socr.(T)}} noht~wn te ka`i a>isjht~wn
dhmiourg'on te ka`i pronoht'hn, 
\kap{4}ka`i e>ic <'ena u<i`on to~u jeo~u
\edtext{\abb{monogen~h}}{\Afootnote{\latintext Io 1,14.18; 3,16; 1Io
4,9}}\edindex[bibel]{Johannes!1,14}\edindex[bibel]{Johannes!1,18}\edindex[bibel]{Johannes!3,16}\edindex[bibel]{Johannes I!4,9}, pr`o
\edtext{\abb{p'antwn}}{\Dfootnote{+ t~wn \latintext
Socr.}} \edtext{\abb{a>i'wnwn}}{\Afootnote{\latintext vgl. 1Cor
2,7}}\edindex[bibel]{Korinther I!2,7|textit} <up'arqonta ka`i sun'onta t~w| gegennhk'oti
a>ut`on
patr'i, \edtext{\abb{di> o~<u \edtext{\abb{ka`i}}{\Dfootnote{\latintext > Socr.(T)}} t`a
p'anta >eg'eneto}}{\Afootnote{\latintext vgl. Io 1,3; 1Cor 8,6; Col 1,16; Hebr
1,2}}\edindex[bibel]{Johannes!1,3|textit}\edindex[bibel]{Korinther I!8,6|textit}\edindex[bibel]{Kolosser!1,16|textit}\edindex[bibel]{Hebraeer!1,2|textit}, t`a
\edtext{\abb{te}}{\Dfootnote{\latintext > Socr.(bA)}} \edtext{\abb{<orat`a ka`i t`a
>a'orata}}{\Afootnote{\latintext Col 1,16}}\edindex[bibel]{Kolosser!1,16}, t`on ka`i
\edtext{\abb{>ep> \edtext{\abb{>esq'atwn}}{\Dfootnote{+ t~wn \latintext Socr.}}
<hmer~wn}}{\Afootnote{\latintext vgl. Hebr 1,2}}\edindex[bibel]{Hebraeer!1,2|textit} kat''
e>udok'ian to~u patr`oc katelj'onta ka`i \edtext{\abb{s'arka}}{\Afootnote{\latintext vgl.
Io 1,14; Rom 1,3;
8,3}}\edindex[bibel]{Johannes!1,14|textit}\edindex[bibel]{Roemer!1,3|textit}\edindex[bibel]{Roemer!8,3|textit} >ek
\edtext{\abb{t~hc}}{\Dfootnote{+ <ag'iac \latintext Socr.(bA)}}
\edtext{\abb{parj'enou}}{\Afootnote{\latintext vgl. Mt
1,23; Lc 1,27.34~f.}}\edindex[bibel]{Matthaeus!1,23|textit}\edindex[bibel]{Lukas!1,27|textit}
\edindex[bibel]{Lukas!1,34~f.|textit}
>aneilhf'ota ka`i \edlabel{jo434-1}p~asan t`hn patrik`hn a>uto~u
\edtext{bo'ulhsin}{\Dfootnote{boul`hn \latintext Socr.}}
\edtext{\abb{sunekpeplhrwk'ota}}{\Afootnote{\latintext vgl. Io
4,34}}\edindex[bibel]{Johannes!4,34|textit}\edlabel{jo434-2},
\edtext{\abb{peponj'enai}}{\Afootnote{\latintext vgl. 1Petr 2,21; 3,18; Act
3,18}}\edindex[bibel]{Petrus I!2,21|textit}\edindex[bibel]{Petrus I!3,18|textit}\edindex[bibel]{Apostelgeschichte!3,18|textit} ka`i
\edtext{\abb{>eghg'erjai}}{\Afootnote{\latintext vgl. Mt 16,21par; 1Cor 15,4; Rom 4,24; Eph
1,20}}\edindex[bibel]{Matthaeus!16,21par|textit}\edindex[bibel]{Korinther I!15,4|textit}\edindex[bibel]{Roemer!4,24|textit}\edindex[bibel]{Epheser!1,20|textit} ka`i
\edtext{\abb{e>ic o>urano`uc
>anelhluj'enai}}{\Afootnote{\latintext vgl. Mc 16,19; Act 1,2; 1Petr
3,22}}\edindex[bibel]{Markus!16,19|textit}\edindex[bibel]{Apostelgeschichte!1,2|textit}
\edindex[bibel]{Petrus I!3,22|textit} ka`i \edtext{\abb{>en dexi\aci\  to~u patr`oc
kaj'ezesjai}}{\Afootnote{\latintext
vgl. Ps 110,1; Eph 1,20; Col 3,1; 1Petr 3,22; Hebr 1,3; Mc
16,19}}\edindex[bibel]{Psalmen!110,1|textit}\edindex[bibel]{Epheser!1,20|textit}\edindex[bibel]{Kolosser!3,1|textit}\edindex[bibel]{Petrus I!3,22|textit}\edindex[bibel]{Hebraeer!1,3|textit}\edindex[bibel]{Markus!16,19|textit} ka`i
\edtext{\abb{p'alin}}{\Dfootnote{\latintext > Socr.(bA)}} >erq'omenon
\edtext{\abb{kr~inai z~wntac ka`i nekro`uc}}{\Afootnote{\latintext 2Tim 4,1; 1Petr 4,5;
vgl. Io 5,22; Act 10,42}}\edindex[bibel]{Timotheus II!4,1}\edindex[bibel]{Petrus I!4,5}\edindex[bibel]{Johannes!5,22|textit}\edindex[bibel]{Apostelgeschichte!10,42|textit}, ka`i
diam'enonta basil'ea ka`i je`on e>ic to`uc a>i~wnac.
\pend
\pstart
\kap{5}piste'uomen \edtext{\abb{d`e}}{\Dfootnote{\latintext > Socr.}} ka`i e>ic t`o
<'agion pne~uma; e>i \edtext{\abb{d`e}}{\Dfootnote{\latintext > Socr.(T)}} de~i
prosje~inai piste'uomen ka`i per`i sark`oc >anast'asewc ka`i \edtext{\abb{zw~hc
a>iwn'iou}}{\Afootnote{\latintext vgl. Gal 6,8; 1Tim 1,16; Jud
21}}\edindex[bibel]{Galater!6,8|textit}\edindex[bibel]{Timotheus I!1,16|textit}\edindex[bibel]{Judas!21|textit}.
\pend
% \endnumbering
\end{Leftside}
\begin{Rightside}
\begin{translatio}
\beginnumbering
\pstart
\noindent\kapR{1}Wir sind weder Gefolgsleute des Arius geworden~-- denn wie k�nnten wir als
Bisch�fe einem Presbyter folgen?~-- noch haben wir einen anderen Glauben als den angenommen, der von
Anfang an �berliefert ist, 
\kapR{2}sondern es haben, da wir auch selbst Pr�fer und Gutachter
seines Glaubens gewesen sind, vielmehr wir ihn zugelassen,\footnoteA{Das bezieht sich
zur�ck auf die Rehabilitierung des Arius, vgl. Dok. \ref{ch:29} = Urk. 29; Dok. \ref{ch:30} = Urk. 30; Dok. \ref{ch:32} = Urk. 32; Dok. \ref{ch:Arius} und die einleitenden Bemerkungen zur Chronologie.} als da� wir ihm
gefolgt sind; ihr werdet es aber aus dem Gesagten erkennen.
\pend
\pstart
\kapR{3}Denn wir haben von Anfang an gelernt, an einen Gott, den Gott des Alls\footnoteA{Die
sonst �blichen Attribute ">Vater"<, ">Allm�chtiger"< fehlen.}, zu glauben, den Sch�pfer
und Bewahrer\footnoteA{Diese beiden Attribute stehen auch in der sog. 2. antiochenischen Formel (Dok. \ref{sec:AntII},1,1, dort ist
zus�tzlich die �bliche Beschreibung \griech{poiht'hn} eingef�gt).} aller vorstellbaren und wahrnehmbaren Dinge,
\kapR{4}und an einen eingeborenen Sohn Gottes, der vor allen Zeiten existiert und mit
seinem Erzeuger, dem Vater, zusammen ist\footnoteA{Eine Kurzfassung der antimarkellischen
Aussagen, wie sie in der sog. 2. antiochenischen Formel (Dok. \ref{sec:AntII},1,6) ausf�hrlich angeh�ngt werden, die so nat�rlich auch eine Abgrenzung zum Arianismus bedeuten; vgl. auch den entsprechenden Passus bei
Theophronius \griech{ka`i >'onta pr`oc t`on je`on >en <upost'asei} in Dok. \ref{sec:AntIII},3.},
durch den auch alles wurde, das Sichtbare und das Unsichtbare, der in den letzten Tagen
nach dem Willen des Vaters herabkam und Fleisch aus der Jungfrau annahm und den ganzen
v�terlichen Willen erf�llt hat,\footnoteA{Vgl. Dok. \ref{sec:AntII},4,1,4
(\refpassage{jo638-1}{jo638-2}).} der gelitten hat, auferweckt
worden ist, in den Himmel aufgestiegen ist, zur Rechten des Vaters sitzt und wieder kommt, um die Lebenden und die Toten zu richten, und der Herrscher und Gott auf ewig bleibt.
\pend
\pstart
\kapR{5}Wir glauben aber auch an den heiligen Geist; wenn aber noch etwas hinzuzuf�gen
ist, glauben wir auch an die Auferstehung des Fleisches und das ewige
Leben\footnoteA{Diese Aussagen, hier etwas salopp nachgereicht, finden sich nur im
Bekenntnis des Markell in seinem Brief (Dok. \ref{sec:MarkellJulius},11) und in dem
Bekenntnis, das Arius zu seiner Rehabilitierung vorlegte (Dok. \ref{ch:30} = Urk. 30,3).}.
\pend
\endnumbering
\end{translatio}
\end{Rightside}
\Columns
\end{pairs}
% \thispagestyle{empty}
\selectlanguage{german}
% \clearpage