\kapitel{Fragmente des Tomus des Alexander von Alexandrien an alle Bisch�fe (Urk.~15)}
\label{ch:15}
\begin{center}
Von Alexander aus Alexandrien
\end{center}
\begin{footnotesize}
  Aus dem Tomus, der von dem Papa, dem Erzbischof von Alexandrien,
  geschrieben wurde �ber den rechten Glauben an die gottliebenden
  Bisch�fe an allen Orten, welchem auch die gottliebenden Bisch�fe,
  die an der Zahl etwa 200 waren, zugestimmt haben, indem sie die Hand
  hoben, da� sie (ihn) so ann�hmen. Dasselbe Schreiben wurde nun vor
  allem geschrieben gegen die Ruchlosigkeit des Arius und jener, die
  sich mit ihm absonderten, sodann aber �ber den katholischen Glauben,
  so wie es jedermann ziemt, ihn festzuhalten, und da� die heilige
  Jungfrau Gottesgeb�rerin sei.
\end{footnotesize}
 
\kapnum{1}Meinen Herrn und Amtsbruder, den Geliebten meiner Seele, Melitius, und die �brigen
Bisch�fe der katholischen Kirche gr��t Alexander in Gott.
 
\begin{footnotesize}Und nach dem Beginn:\end{footnotesize}
 
\kapnum{2}Und zum rechten Glauben, dem �ber den Vater und den Sohn, gem�� dem, was uns die
Schriften lehren: Den einen heiligen Geist bekennen wir und die eine katholische Kirche
und die Auferstehung der Toten, dessen Anfang unser Herr und Heiland Jesus Christus
gewesen ist, indem er Fleisch anzog von der Gottesgeb�rerin Maria, da� er zu dem
Geschlecht der Menschen komme, indem er starb, auferstand aus der Unterwelt und aufstieg
in den Himmel und sich setzte zur Rechten der Majest�t.
 
\kapnum{3}Dies habe ich zum Teil brief"|lich niedergelegt, wobei ich es unterlie�, alles einzelne davon genau aufzuschreiben, damit ihr nicht eure g�ttliche Last zu tragen verge�t.
Dies haben wir gelehrt. Dies haben wir verk�ndigt. Dies sind die apostolischen
Lehren der Kirche. Zu denen gerieten die Anh�nger des Arius und
des Achillas und die, die sich mit ihnen von der Kirche abgespalten haben, in Gegensatz; denn sie lehren Fremdes im Vergleich zu unserer rechten Lehre, gem�� dem seligen Paulus, der sagte: ">Wenn jemand euch etwas anderes verk�ndigt, als das, was ihr empfangen habt, der sei verflucht."<
 
\begin{footnotesize}Und nach anderem.\end{footnotesize}
 
\kapnum{4}Denn auch diesem Wort gem��~-- n�mlich jenes ">Im Anfang war das Wort"<,
verleugnen sie, und jenes: ">Christus ist Gottes Kraft und Gottes Weisheit"< oder ">Er ist
das Wort und die Weisheit des Vaters"<, lehren sie nicht. Oder da� Gott nicht (schon)
immer die Weisheit und das Wort gezeugt habe, glauben sie. Dieses aber auch (nur) zu
denken, kommt einer Seele zu, die ungl�ubig und fern ist von den J�ngern Christi.
 
\begin{footnotesize}Und nach der Unterschrift derjenigen von ganz Aegyptus und der Theba"is und
Libyas und der Pentapolis und von den oberen Orten mit denen von Palaestina und von
Arabia und von Achaia und von Thracia und vom Hellespont und von Asia und Caria und
Lycia und Lydia und Phrygia und Pamphylia und Galatia und Pisidia und von Pontus und
Polemoniacus, und von Cappadocia und von Armenia unterschrieb auch Philogonius, der
Bischof von Antiochia in Syria [und alle vom Osten, die gottliebenden Bisch�fe von
Mesopotamia und von Augusto-Euphratesia und von Cilicia und von Isauria und Phoenice].
\end{footnotesize}
 
\kapnum{5}Ich, Philogonius, der Bischof der katholischen Kirche von Antiochia, indem ich
den Glauben, der im Schreiben meines Herrn und Seelenfreundes Alexander ist, �beraus
lobe und mit ihm und mit der �bereinkunft der heiligen Ordnung derjenigen �bereinstimme,
welche eines Sinnes sind, unterschrieb und alle diejenigen, welche im Osten sind, was oben
geschrieben steht.