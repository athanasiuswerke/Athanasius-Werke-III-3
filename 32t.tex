% \section{Brief Kaiser Konstantins an Alexander von Alexandrien}
\kapitel{Brief des Kaisers Konstantin an Alexander von Alexandrien (Urk.~32)}
\label{ch:32}

Der Sieger Konstantin der Gro�e, Augustus, an den Vater Alexander, den Bischof.

\kapnum{1}Und jetzt wird also der sch�dliche Neid mit gottlosen Spitzfindigkeiten f�r einen
Aufschub zur�ckbellen. Was bedeuten also die gegenw�rtigen Verh�ltnisse? Beschlie�en wir andere Dinge,
als wie es vom heiligen Geist durch euch beschlossen worden ist, geliebter Bruder?

\kapnum{2}Ich sage, da� Arius, jener Arius, zu mir, dem Augustus, kommen soll, auf Zuraten von
vielen hin, da er versprach, das von unserem katholischen Glauben zu denken, was von euch auf der
Synode in Nicaea bestimmt und festgelegt worden ist, wo auch ich dabei war und mitbestimmt und mich
mit euch um eure Angelegenheiten gek�mmert habe.

\kapnum{3}Sofort darauf kam er also zu uns, zusammen mit Euzoius, sobald sie vom Befehl des
kaiserlichen Willens erfahren hatten. Ich erforschte also mit ihnen in  Anwesenheit von vielen das
Wort des Lebens. Ich bin jener Mensch, der seinen Sinn mit reinem Glauben auf Gott ausgerichtet hat.
Ich bin es, der euch unterst�tzt, der ich mein ganzes Streben auf unseren Frieden und Einm�tigkeit
gerichtet habe.

\begin{footnotesize}
Und nach anderem:
\end{footnotesize}

\kapnum{4}Ich schickte sie also los und erinnerte euch nicht nur, sondern bat euch auch, diese
bittenden Menschen wieder aufzunehmen. Wenn ihr also herausfindet, da� sie nach dem in Nicaea
festgelegten, rechten, immer lebendigen, apostolischen Glauben streben~-- sie haben n�mlich bei uns
bekr�ftigt, so zu denken~-- , dann sorgt euch um alle, ich bitte euch. Denn wenn ihr euch
in F�rsorge f�r diese �bt, dann werdet ihr wohl den Ha� durch Eintracht besiegen.

\kapnum{5}Ich bitte euch also, helft der Eintracht, bringt die G�ter der Freundschaft zu denen, die
die Angelegenheiten des Glaubens nicht durchschauen, bringt mir die Dinge zu Ohren, die ich will und
erstrebe, Frieden und Eintracht unter euch allen.

Gott m�ge dich beh�ten, verehrtester Vater!
\clearpage