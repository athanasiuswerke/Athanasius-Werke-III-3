%%%%%%%%%%%%%%%%%%%%%%%%%%%%%
%%% Konfiguration f�r AW III
%%%%%%%%%%%%%%%%%%%%%%%%%%%%%
%%% Zusammenf�hrung von ledmac.tex und memoir.tex
%%% Mi 25 Apr 2007 23:01:23 CEST  AvS
%%% Umstellung von jurabib auf biblatex
%%% Mi Mai 23 23:09:47 2007 AvS
%%%%%%%%%%%%%%%%%%%%%%%%%%%%%
\makeatletter
%%%%%%%%% Geladene Pakete %%%%%%%%%%%
\usepackage[latin,polutonikogreek,german]{babel}
\usepackage[latin1]{inputenc}
\usepackage[T1]{fontenc}
\usepackage{etex}
\usepackage{textcomp}
\usepackage{teubner}
\usepackage{ledmac}
\usepackage{ledpar}
\usepackage{longtable,booktabs}
\usepackage{multicol}
\usepackage{tikz}
\usepackage[babel]{csquotes}
%%%%%%% Schriften %%%%%%%%%%%%%%%
\usepackage{AGaramondPro}
\usepackage[neohellenic]{psgreek}
\renewcommand{\ttdefault}{pcr}
\renewcommand{\sfdefault}{MyriadProJ}
%%%%%% biblatex-Konfiguration %%%%%%%%%%
\usepackage[style=authortitle-comp,hyperref]{biblatex}
%%
\DeclareNameFormat{default}{%
  \usebibmacro{name:first-last}{#1}{#4}{#5}{#7}%
  \usebibmacro{name:delim}}
\DeclareNameFormat{sortname}{%
  \ifnum\value{listcount}=1\relax
    \usebibmacro{name:last-first}{#1}{#4}{#5}{#7}%
    \usebibmacro{name:revsdelim}%
  \else
    \usebibmacro{name:first-last}{#1}{#4}{#5}{#7}%
    \usebibmacro{name:delim}%
  \fi}
%% Komma als Trenner
\renewcommand*{\newunitpunct}{\addcomma\space}
%% Delim zwischen Namens
\renewcommand*{\multinamedelim}{\addslash}
%% Delim vor letztem Namen
\renewcommand*{\finalnamedelim}{\addslash}
\renewcommand*{\revsdnamedelim}{}
%% Format des Titels
\DeclareFieldFormat{citetitle}{#1\isdot}
\DeclareFieldFormat{citetitle:article}{#1\midsentence}
\DeclareFieldFormat{citetitle:book}{#1\isdot}
\DeclareFieldFormat{citetitle:booklet}{#1\isdot}
\DeclareFieldFormat{citetitle:collection}{#1\isdot}
\DeclareFieldFormat{citetitle:inbook}{#1\midsentence}
\DeclareFieldFormat{citetitle:incollection}{#1\midsentence}
\DeclareFieldFormat{citetitle:inproceedings}{#1\midsentence}
\DeclareFieldFormat{citetitle:manual}{#1\isdot}
\DeclareFieldFormat{citetitle:misc}{#1\isdot}
\DeclareFieldFormat{citetitle:online}{#1\isdot}
\DeclareFieldFormat{citetitle:proceedings}{#1\isdot}
\DeclareFieldFormat{citetitle:thesis}{#1\midsentence}
\DeclareFieldFormat{citetitle:unpublished}{#1\midsentence}
\DeclareFieldFormat{citetitle:customa}{#1\isdot}
\DeclareFieldFormat{citetitle:customb}{#1\isdot}
\DeclareFieldFormat{citetitle:customc}{#1\isdot}
\DeclareFieldFormat{citetitle:customd}{#1\isdot}
\DeclareFieldFormat{citetitle:custome}{#1\isdot}
\DeclareFieldFormat{citetitle:customf}{#1\isdot}
\DeclareFieldFormat{booktitle}{#1}
\DeclareFieldFormat{journal}{#1}
\DeclareFieldFormat{maintitle}{#1}
\DeclareFieldFormat{title}{#1\isdot}
\DeclareFieldFormat{title:article}{#1\midsentence}
\DeclareFieldFormat{title:book}{#1\isdot}
\DeclareFieldFormat{title:booklet}{#1\isdot}
\DeclareFieldFormat{title:collection}{#1\isdot}
\DeclareFieldFormat{title:inbook}{#1\midsentence}
\DeclareFieldFormat{title:incollection}{#1\midsentence}
\DeclareFieldFormat{title:inproceedings}{#1\midsentence}
\DeclareFieldFormat{title:manual}{#1\isdot}
\DeclareFieldFormat{title:misc}{#1\isdot}
\DeclareFieldFormat{title:online}{#1\isdot}
\DeclareFieldFormat{title:proceedings}{#1\isdot}
\DeclareFieldFormat{title:thesis}{#1\midsentence}
\DeclareFieldFormat{title:unpublished}{#1\midsentence}
\DeclareFieldFormat{title:customa}{#1\isdot}
\DeclareFieldFormat{title:customb}{#1\isdot}
\DeclareFieldFormat{title:customc}{#1\isdot}
\DeclareFieldFormat{title:customd}{#1\isdot}
\DeclareFieldFormat{title:custome}{#1\isdot}
\DeclareFieldFormat{title:customf}{#1\isdot}
\DeclareFieldFormat{postnote}{%
  \ifpages{#1}
    {}
    {}%
  #1}
%% Anpassung citecommand \cite
\newbibmacro*{ecite}{%
  \let\cbx@tempa\empty
  \iffieldundef{shorthand}
    {\ifnameundef{labelname}
       {}
       {\printnames{labelname}%
        \def\cbx@tempa{\addcomma\space}}%
     \usebibmacro{cite:title}}%
    {\usebibmacro{cite:shorthand}}}
\renewbibmacro*{cite}{%
  \let\cbx@tempa\empty
    {\ifnameundef{labelname}
       {}
       {\printnames{labelname}%
        \def\cbx@tempa{\addcomma\space}}%
     \usebibmacro{cite:title}}%
}
%%%%%%%%%%%%%%%%% Neues citecommand \conicite
\newbibmacro*{ccite}{%
  \let\cbx@tempa\empty
\usebibmacro{cite:shorthand}\addcomma\space\usebibmacro{cite:title}}
\DeclareCiteCommand{\conicite}
  {\usebibmacro{citeinit}%
   \usebibmacro{prenote}}
  {\usebibmacro{citeindex}%
   \usebibmacro{ccite}}
  {\multicitedelim}
  {\usebibmacro{postnote}}
%% Neues citecommand \editioncite
\newbibmacro*{postnoteA}{%
  \iffieldundef{postnote}
    {}
    {\printfield{postnote}}}
\DeclareCiteCommand{\editioncite}
  {\usebibmacro{citeinit}%
   \usebibmacro{prenote}}
  {\usebibmacro{postnoteA}\addspace%
  \usebibmacro{citeindex}%
   \usebibmacro{ecite}}
  {\multicitedelim}
  {}
%%%%%%%%%% kein in: bei Zeitschriftenartikeln
\DeclareBibliographyDriver{article}{%
  \usebibmacro{bibindex}%
  \usebibmacro{author}%
  \newunit\newblock
  \usebibmacro{title+stitle}%
  \newunit\newblock
  %\usebibmacro{in:}%
  \usebibmacro{journal+issue+year+pages}%
  \newunit\newblock
  \printfield{note}%
  \newunit\newblock
  \printfield{issn}%
  \newunit\newblock
  \printfield{doi}%
  \newunit\newblock
  \printfield{addendum}%
  \newunit\newblock
  \usebibmacro{url+date}%
  \newunit\newblock
  \usebibmacro{pageref}%
  \usebibmacro{finentry}}
\bibliography{dokumente_master}
\defbibheading{edition}{\section*{Editionen}}
\defbibheading{literatur}{\section*{Sekund�rliteratur}}
%%%%%%%%% Hyperref
\usepackage[implicit=true,plainpages=false,pdfpagelabels]{hyperref}
\hypersetup{%
pdftitle = {Dokumente zum arianischen Streit},
pdfsubject = {Edition},
pdfkeywords = {Dokumente zum arianischen Streit, Edition},
pdfauthor = {\textcopyright\ 2007 Edition Athanasius Werke},
}
\usepackage{memhfixc}
\memhyperindexfalse
\usepackage{microtype}
\title{Dokumente zum arianischen Streit}
\author{Hanns Christof Brennecke\\Uta Heil\\Annette von Stockhausen\\Angelika Wintjes}
\date{\today}
%%%%%%%%%%%% Indices erstellen %%%%%%%%%%%%%%%%%%%%
% \makeindex[bibel]
% \makeindex[quellen]
% \makeindex[synoden]
% \makeindex[namen]

%%%%%%%%% Seitenlayout %%%%%%%%%%%%%
\settrimmedsize{297mm}{210mm}{*}
\setlength{\trimtop}{0pt}
\setlength{\trimedge}{\stockwidth}
\addtolength{\trimedge}{-\paperwidth}
\settypeblocksize{22.5cm}{14.5cm}{*}
\setulmargins{3.5cm}{*}{*}
\setlrmargins{*}{*}{1}
\setmarginnotes{10pt}{11pt}{\onelineskip}
\setheadfoot{\onelineskip}{2\onelineskip}
\setheaderspaces{*}{\onelineskip}{*}
\checkandfixthelayout

%%%%%%%%% Seitenstil (Kopfzeilenlayout) %%%%%%
\makepagestyle{dokumente}
\makepsmarks{dokumente}{%
  \let\@mkboth\markboth
  \def\chaptermark##1{\markboth{\thechapter. \ ##1}{\thechapter. \ ##1}}    % left mark & right marks
  \def\sectionmark##1{\markright{%
    \ifnum \c@secnumdepth>\z@
      \thesection. \ %
    \fi
    ##1}}
  \def\tocmark{\markboth{\contentsname}{\contentsname}}
  \def\lofmark{\markboth{\listfigurename}{\listfigurename}}
  \def\lotmark{\markboth{\listtablename}{\listtablename}}
  \def\bibmark{\markboth{\bibname}{\bibname}}
  \def\indexmark{\markboth{\indexname}{\indexname}}
}
\makeevenhead{dokumente}{\thepage}{\small\leftmark}{}
\makeoddhead{dokumente}{}{\small\rightmark}{\thepage}
% \makeevenfoot{dokumente}{}{\tiny\texttt{\dt{Probeausdruck Athanasius Werke III/3 -- Stand: \today\ -- Blatt \thesheetsequence\ von \thelastsheet}}}{}
% \makeoddfoot{dokumente}{}{\tiny\texttt{\dt{Probeausdruck Athanasius Werke III/3 -- Stand: \today\ -- Blatt \thesheetsequence\ von \thelastsheet}}}{}

\pagestyle{dokumente}
\aliaspagestyle{chapter}{empty}

%%%%%%%% �berschriften %%%%%%%%%%%%%%
%%%%%%%% Kapitel %%%%%%%%%%%%%%%%%
\makechapterstyle{ath}{
  \renewcommand{\printchaptername}{}
  \renewcommand{\chapternamenum}{}
  \renewcommand{\chaptitlefont}{\centering\Large}
  \renewcommand{\chapnumfont}{\chaptitlefont}
  \renewcommand{\printchapternum}{\centering \chapnumfont \thechapter\space}
  \renewcommand{\afterchaptertitle}{\par\nobreak\vskip \afterchapskip}
}
\chapterstyle{ath}
%%%%%%%% Neuer Kapitelbefehl f�r �Urkunden� %%%
%\let\kapitel\chapter
%\renewcommand{\clearforchapter}{\par}  % Chapter beginnt nicht neue Seite
\newcommand\kapitel{%
  \ifartopt\par\else
%     \clearforchapter
\par
    \thispagestyle{chapter}
    \global\@topnum\z@
  \fi
  \@afterindentfalse
  \@ifstar{\@m@mschapter}{\@m@mchapter}}
%%%%%%%% Sections u.s.w. %%%%%%%%%%%%%%
\setsecheadstyle{\large\centering}
\setsubsecheadstyle{\normalsize\centering}
\setsubsubsecheadstyle{\itshape\centering}
\setparaheadstyle{\itshape}
%%%%%%%%%%% Abst�nde vor Kapitel etc. %%%
\setlength{\beforechapskip}{2\baselineskip}
\setlength{\midchapskip}{.5\baselineskip}
\setlength{\afterchapskip}{\baselineskip}
\setbeforesecskip{2\baselineskip}
\setbeforeparaskip{0\baselineskip}
\setaftersecskip{\baselineskip}
%%%%%%%% Nummerierung nur bis section %%%%%%
\maxsecnumdepth{section}

%%%%%%% Formatierungen f�r Edition %%%%%%%%
%%%%%%% Handschriftenbeschreibung Siglenliste %%
\newenvironment{codices}%
	{\begin{longtable}[l]{p{1.6cm}p{10cm}p{2cm}}}
	{\end{longtable}\ignorespacesafterend}
\newenvironment{konjektoren}%
	{\begin{longtable}[l]{p{3cm}p{8cm}p{2cm}}}
	{\end{longtable}\ignorespacesafterend}
\newcommand{\codex}[3]{%
	#1% Sigle
	& #2% Name
	& #3% Datierung
	\\
}
%%%%%%% Paragraphenangabe in nicht-nummeriertem Text (�bersetzungen aus AW III/1+2)
\newcommand{\kapnum}[1]{\noindent\llap{\makebox[.5cm][l]{\footnotesize #1}}\ifthenelse{\equal{#1}{1}}{\noindent}{\quad}}
%%%%%%% Paragraphenangabe im nummerierten Text %%
\newcommand{\kap}[1]{\ledsidenote{\dt{#1}}}
\newcommand{\kapR}[1]{} %% Befehl f�r die Kapitelz�hlung rechts - momentan nicht ausgegeben

%%%%%%% Abk�rzungen f�r den textkritischen Apparat %%
\newcommand{\dt}{\foreignlanguage{german}}
\newcommand{\griech}{\foreignlanguage{polutonikogreek}}
\newcommand{\mg}{\textsuperscript{\,mg}}
\newcommand{\slin}{\textsuperscript{\,sl}}
\newcommand{\corr}{\textsuperscript{\,c}}
\newcommand{\ras}{\textsuperscript{\,ras}}
\newcommand{\ts}{\textsuperscript}
\newcommand{\tsub}{\textsubscript}
%%%%%% Neudefinition von teubner.sty-Abk�rzungen %%%%%%%%%%
\renewcommand{\Ladd}[1]{\dt{<}#1\dt{>}}%
\renewcommand{\ladd}[1]{{[}%
    {#1\/}{]}}%
\renewcommand{\dBar}{\textbardbl}
%%%%%% Linie �ber Buchstaben %%%%%%%%%%
\usepackage{ulem}
\protected\def\oline{\bgroup \ULdepth=-1.9ex \ULset}

%%%%%%% Definition einer Praefatio-Umgebung %%%%
\newenvironment{praefatio}%
{\footnotesize}%
{\ignorespacesafterend}%
%%%%%%%%%%%%% Listen enger gesetzt %%%%%%%%%%%

%%%%%%% Description-Umgebung so umdefiniert, dass folgende Zeilen nicht mehr eingezogen werden
\renewenvironment{description}%
               {\list{}{\labelwidth\z@ \setlength{\leftmargin}{0em}\setlength{\parsep}{0pt}
               \let\makelabel\descriptionlabel}\tightlist}%
               {\endlist\ignorespacesafterend}
% \renewcommand*{\descriptionlabel}[1]{\hspace\labelsep\normalfont\itshape #1}
%%%%%% �bersetzungs-Umgebung %%%%%%%%%%%
\newenvironment{translatio}
{\small\begin{otherlanguage}{german}}%
{\end{otherlanguage}}

%%%%%% Befehl, um den Autor eines Textst�ckes angeben zu k�nnen %%%
\newcommand{\autor}[1]{%
  \subsection{#1}
    }

%%%%%%% Fu�noten %%%%%%%%%%%%%%%%%%
%%%%%%% �Normale� Fu�noten %%%%%%%%%%%%
\setlength{\footmarksep}{0pt}
\newlength{\myFootnoteWidth}\newlength{\myFootnoteLabel}
 \setlength{\myFootnoteLabel}{.6cm}%  <-- can be changed to any valid value
\setlength{\footmarkwidth}{.6cm}%\myFootnoteLabel}
\footmarkstyle{\makebox[\myFootnoteLabel][l]{#1}}
\setlength{\thanksmarkwidth}{1em}
\setlength{\thanksmarksep}{0em}
\renewcommand{\@makefntext}[1]{\makefootmark #1}

%%%%%% Umdefinition von footnoteA (Hist. Kommentar) %%%%
% \footparagraph{A}
\renewcommand{\footnoteA}[1]{%
  \stepcounter{footnoteA}%
  \protected@xdef\@thefnmarkA{\thefootnoteA\noexpand}%
  \@footnotemarkA
  \vfootnoteA{A}{\normalfont #1}\m@mmf@prepare}
\renewcommand{\thefootnoteA}{\alph{footnoteA}}
\usepackage{perpage}
\MakePerPage{footnoteA}

%%%%%%% Ledmac-Spezifisches %%%%%%%%%%%%
%%%%%%% meist von Dirk-Jan Dekker %%%%%%%%%%
\lefthyphenmin=3
\righthyphenmin=3
%%%%%%%%%%%%%% Zeilennummer l�schen  %%%%%%%%%%%%%%%%%
\def\printlines#1|#2|#3|#4|#5|#6|#7|{\begingroup
 \l@d@pnumfalse \l@d@dashfalse
 \ifbypage@
 \ifnum#4=#1 \else
 \l@d@pnumtrue
 \l@d@dashtrue
 \fi
 \fi
 \ifl@d@pnum \l@d@elintrue \else \l@d@elinfalse \fi
 \ifnum#2=#5 \else
 \l@d@elintrue
 \l@d@dashtrue
 \fi
 \l@d@ssubfalse
 \ifnum#3=0 \else
 \l@d@ssubtrue
 \fi
 \l@d@eslfalse
 \ifnum#6=0 \else
 \ifnum#6=#3
 \ifl@d@elin \l@d@esltrue \else \l@d@eslfalse \fi
 \else
 \l@d@esltrue
 \l@d@dashtrue
 \fi
 \fi
 \ifnum#2=-1 \ledplinenumfalse \fi     % This line is new
 \ifl@d@pnum #1\fullstop\fi
 \ifledplinenum \linenumr@p{#2}\else \symplinenum\fi
 \ifl@d@ssub \fullstop \sublinenumr@p{#3}\fi
 \ifl@d@dash \endashchar\fi
 \ifl@d@pnum #4\fullstop\fi
 \ifl@d@elin \linenumr@p{#5}\fi
 \ifl@d@esl \ifl@d@elin \fullstop\fi \sublinenumr@p{#6}\fi
 \endgroup}

 \newcommand{\killnumber}{\linenum{|-1|||-1||}}
 %%%%%%%%%%%%%%%%%% Zeilennummer l�schen Ende %%%%%%%%%%%%%%%%%%%%
\renewcommand*{\para@vfootnote}[2]{%
\insert\csname #1footins\endcsname
\bgroup
\notefontsetup
\footsplitskips
\l@dparsefootspec #2\ledplinenumtrue  %%%% FIRST ADDED LINE %%%%%%%%%%%%%%
\ifnum\@nameuse{previous@#1@number}=\l@dparsedstartline\relax
\ledplinenumfalse
\fi
\ifnum\previous@page=\l@dparsedstartpage\relax
\else \ledplinenumtrue \fi
\ifnum\l@dparsedstartline=\l@dparsedendline\relax
\else \ledplinenumtrue \fi
\expandafter\xdef\csname previous@#1@number\endcsname{\l@dparsedstartline}
\xdef\previous@page{\l@dparsedstartpage}  %%%% LAST ADDDED LINE %%%%%%%%%%
\setbox0=\vbox{\hsize=\maxdimen
\noindent\csname #1footfmt\endcsname#2}%
\setbox0=\hbox{\unvxh0}%
\dp0=0pt
\ht0=\csname #1footfudgefactor\endcsname\wd0
\box0
\penalty0
\egroup}

\footparagraph{A}
\footparagraph{B}
\footparagraph{C}
\footparagraph{D}


% \def\zparafootfmt#1#2#3{%
% \normal@pars
% \hskip -1.8ex�\parfillskip=0pt plus1fil
% \notetextfont�#3\penalty-10 }
%
% \let\Cfootfmt=\zparafootfmt

% \let\Afootnoterule=\relax
% \let\Bfootnoterule=\relax
% \let\Cfootnoterule=\relax
% \let\Dfootnoterule=\relax
% \addtolength{\skip\Afootins}{0.5mm}
% \addtolength{\skip\Bfootins}{0.5mm}
% \addtolength{\skip\Cfootins}{0.5mm}
% \addtolength{\skip\Dfootins}{0.5mm}
\setlength{\skip\Afootins}{2.5mm}
\setlength{\skip\Bfootins}{2.5mm}
\setlength{\skip\Cfootins}{2.5mm}
\setlength{\skip\Dfootins}{2.5mm}
%\count\Afootins=925
%\count\Bfootins=825
%\count\Cfootins=925
% \count\Dfootins=875
\renewcommand*{\footfudgefiddle}{78}

\newcommand*{\previous@A@number}{-1}
\newcommand*{\previous@B@number}{-1}
\newcommand*{\previous@C@number}{-1}
\newcommand*{\previous@D@number}{-1}
\newcommand*{\previous@page}{-1}

%% NO \rbracket IN FRONT OF om./inv./add. ETC.

\newcommand{\abb}[1]{#1%
        \let\rbracket\nobrak\relax}
\newcommand{\nobrak}{\textnormal{}}
\newcommand{\morenoexpands}{%
        \let\abb=0%
}

\newcommand{\Aparafootfmt}[3]{%
  \normal@pars\footnotesize
  \parindent=0pt \parfillskip=0pt plus1fil
  \notenumfont\printlines#1|%
\ifledplinenum
\enspace%
\else
{\dBar \hskip .8em plus0em minus.4em}%
\fi
  \select@lemmafont#1|#2\rbracket\enskip
  \notetextfont
  #3\penalty-10\hskip 1em plus 4em minus.4em\relax }

\newcommand{\Bparafootfmt}[3]{%
  \normal@pars\footnotesize
  \parindent=0pt \parfillskip=0pt plus1fil
  \notenumfont\printlines#1|%
\ifledplinenum
\enspace%
\else
{\dBar \hskip .8em plus0em minus.4em}%
\fi
  \select@lemmafont#1|#2\rbracket\enskip
  \notetextfont
  #3\penalty-10\hskip 1em plus 4em minus.4em\relax }

\newcommand{\Cparafootfmt}[3]{%
  \normal@pars\footnotesize
  \parindent=0pt \parfillskip=0pt plus1fil
  \notenumfont\printlines#1|%
\ifledplinenum
\enspace%
\else
{}% keine dbar
\fi
  \select@lemmafont#1|#2\rbracket\enskip
  \notetextfont
  #3\penalty-10\hskip 1em plus 4em minus.4em\relax }

\newcommand{\Dparafootfmt}[3]{%
  \normal@pars\footnotesize
  \parindent=0pt \parfillskip=0pt plus1fil
  \notenumfont\printlines#1|%
\ifledplinenum
\enspace%
\else
{\dBar \hskip .8em plus0em minus.4em}%
\fi
  \select@lemmafont#1|#2\rbracket\enskip
  \notetextfont
  #3\penalty-10\hskip 1em plus 4em minus.4em\relax }
%% END DEFINITION OF \abb


\let\Afootfmt=\Aparafootfmt
\let\Bfootfmt=\Bparafootfmt
\let\Cfootfmt=\Cparafootfmt
\let\Dfootfmt=\Dparafootfmt

%%%%%%%%% Referenzierung von Textpassagen %%%%%%%%%%%%%%%
%% Dirk-Jan Dekker in comp.text.tex: cross-referencing entire passages in ledmac
\newcommand{\refpassage}[2]{%
   \xpageref{#1},\xlineref{#1}%
   \ifnum\xpageref{#1}=\xpageref{#2}
      \ifnum\xlineref{#1}=\xlineref{#2}
      \else
         \endashchar\xlineref{#2}%
      \fi
   \else
      \endashchar\xpageref{#2},\xlineref{#2}%
   \fi
}

%%%%%%%% Ma�e und Einstellungen %%%%%%%%%%%%%%%%%%%
\lineation{page}
\sidenotemargin{left}
\linenummargin{right}
\setlength{\linenumsep}{0.4pc}
\setlength{\parindent}{1em}
\setlength\parskip{0em} 
   \tolerance1414
   \hbadness1414
   \vbadness1414
\emergencystretch35pt
   \hfuzz0.5pt
\widowpenalty=10000
   \vfuzz \hfuzz
 \clubpenalty=10000

\renewcommand{\symplinenum}{}   % evt. \textbar of $\|$ invullen
%%%%%%% Schriftgr��en %%%%%%%%%%%%%%%%%%%
\renewcommand*{\notenumfont}{\bfseries\latintext}
\newcommand*{\notetextfont}{\footnotesize}
\renewcommand*{\numlabfont}{\normalfont\scriptsize\latintext}

%%%%%%% Paralleltext  %%%%%%%%%%%%
\maxchunks{300}
\renewcommand*{\Rlineflag}{}
% \renewcommand*{\leftlinenum}{}
% \renewcommand*{\rightlinenum}{}
\renewcommand*{\leftlinenumR}{}
\renewcommand*{\rightlinenumR}{}
%%%%%%% Spaltenabstand %%%%%%%%%%
\setlength{\Lcolwidth}{.45\textwidth}
\setlength{\Rcolwidth}{.51\textwidth}
%\setlength{\Lcolwidth}{65mm}
%\setlength{\Rcolwidth}{75mm}

%%%%%%% TOC %%%%%%%%%%%%%%%%%%%%%
%%%%%%% Style des Titels u. d. Kapitel�berschriften %%%
\renewcommand{\printtoctitle}[1]{\centering\normalfont\Large #1}
\renewcommand{\cftchapterfont}{\raggedright\normalfont\large}
\renewcommand{\cftchapterpagefont}{\normalfont}
%%%%%%% Abst�nde zwischen Nummer und Titel
\setlength{\cftchapternumwidth}{3em}
\setlength{\cftsectionnumwidth}{3em}
\setlength{\cftsectionindent}{0em}
%%%%%%% Punkte %%%%%%%%%%%%%%%%%%%%
\renewcommand{\cftsectiondotsep}{\cftdotsep}
\renewcommand{\cftchapterdotsep}{\cftdotsep}

%%%%%%% Captions %%%%%%%%%%%%%%%%%%%
\captionnamefont{\scriptsize}
\captiontitlefont{\scriptsize}

%%%%%%% Index %%%%%%%%%%%%%%%%%%%%%
\renewcommand{\pagelinesep}{,} % Trenner zw. Seite u. Zeile
\renewcommand*{\see}[2]{\textit{\seename} #1}
%%%%%%% Dreispaltig und sonstige Index-Formatierungen 
\setlength{\indexcolsep}{1cm}
% \setlength{\indexrule}{.1pt}

\renewenvironment{theindex}{%
  \begin{multicols}{3}[\section*{\indexname}\preindexhook][10\baselineskip]%  
 \raggedcolumns
\raggedright\footnotesize
  \indexmark
  \setlength{\columnseprule}{\indexrule}
  \setlength{\columnsep}{\indexcolsep}
  \ifnoindexintoc\else
    \phantomsection
    \addcontentsline{toc}{section}{\indexname}
  \fi
%   \thispagestyle{chapter}
  \parindent\z@
  \parskip\z@ \@plus .3\p@\relax
  \let\item\@idxitem}
{\end{multicols}}
\renewcommand{\@idxitem}  {\par\hangindent 40\p@}
\renewcommand{\subitem}   {\par\hangindent 20\p@}
\renewcommand{\subsubitem}{\par\hangindent 40\p@}
\renewcommand{\indexspace}{\par \vskip 20\p@ \@plus5\p@ \@minus3\p@\relax}

%%%%%%% Diskussionsbedarf %%%%%%%%%%%%%%%
\usepackage{framed}
% \newcommand{\diskussionsbedarf}{\begin{framed}\bfseries\centering Diskussionsbedarf\end{framed}}
%\newcommand{\frage}[1]{\LitNil\LitNil\texttt{#1}\LitNil\LitNil\ }
\newcommand{\diskussionsbedarf}{}
\newcommand{\frage}[1]{#1}
\makeatother