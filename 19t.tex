% \section{Brief des Narcissus von Neronias an Chrestus, Euphronius und Eusebius}
\kapitel[Markell von Ancyra �ber einen Brief des Narcissus von Neronias an Chrestus, Euphronius und Eusebius (Urk.~19)][Markell von Ancyra �ber einen Brief des Narcissus von Neronias (Urk.~19)]{Markell von Ancyra �ber einen Brief des Narcissus von Neronias an Chrestus, Euphronius und Eusebius (Urk.~19)}
\label{ch:19}

\kapnum{1}">Mir fiel n�mlich der Brief des Narcissus, des Vorstehers von Neronias, in die H�nde,
den er an einen gewissen Chrestus, an Euphronius und an Eusebius geschrieben hatte, in dem er schreibt, da� der Bischof Ossius ihn gefragt hatte, ob er wie Eusebius aus Palaestina sage, es gebe zwei Wesen; und daraus habe ich erfahren, da� er geantwortet hat, er glaube, es gebe drei Wesen."<

\kapnum{2} \dots\ Und danach wendet er (Markell) sich Narcissus zu und sagt:
">Auch wenn jemand das sagt und einen ersten und zweiten Gott einf�hrt, wie es Narcissus in seinen
Schriften formuliert hat (\,\dots), weil n�mlich er und sein Vater zwei seien, so haben wir zum Teil
schon geh�rt, was der Herr selbst und die heiligen Schriften (dar�ber) bezeugen. Wenn also Narcissus
deswegen das Wort der Kraft nach vom Vater abtrennen will, so soll er wissen, da� der Prophet, der
geschrieben hat, wie Gott sprach: \frq La�t uns den Menschen machen nach unserem Bild und unserer
Gleichheit\flq, selbst auch geschrieben hat: \frq Und Gott machte den Menschen\flq ."<
\clearpage