%%%% Input-Datei OHNE TeX-Pr�ambel %%%%
\kapitel[Fragmente zweier Briefe des Theognis von Nicaea]{Fragmente zweier Briefe des
Theognis von Nicaea}
% \label{ch:38}
\label{ch:TheognisNiz}
\begin{praefatio}
  \begin{description}
  \item[Vor Nicaea 325?] Im Text bietet sich kein
    Anhaltspunkt f�r eine Datierung. Es kann jedoch vermutet werden,
    da� die Fragmente der beiden Briefe in die Zeit vor der Synode von
    Nicaea\index[synoden]{Nicaea!a. 325} 325 geh�ren, als viele Briefe
    pro und contra Arius\index[namen]{Arius!Presbyter in Alexandrien}
    und dessen theologische Ideen an Alexander von
    Alexandrien\index[namen]{Alexander!Bischof von Alexandrien}
    verschickt wurden (vgl. Dok. \ref{ch:7} = Urk. 7; Dok. \ref{ch:9} =
    Urk. 9; Dok. \ref{ch:11} = Urk. 11; Dok. \ref{ch:12} = Urk. 12 und
    die Einleitung zur Chronologie), da es sich beim Adressaten
    (\textit{ad papam}) wohl um den alexandrinischen Bischof
    handelt. �ber Theognis\index[namen]{Theognis!Bischof von Nicaea}
    wissen wir ferner, da� er sich auf der Synode von Nicaea geweigert
    hatte, die Verurteilung des Arius\index[namen]{Arius!Presbyter in
      Alexandrien} zu unterzeichnen (Soz., h.\,e. I 21,3;
    Dok. \ref{ch:27} = Urk. 27; Dok. \ref{ch:31} = Urk. 31), und
    daraufhin selbst verbannt wurde. An der Synode von
    Tyrus\index[synoden]{Tyrus!a. 335}, die zur Absetzung des
    Athanasius\index[namen]{Athanasius!Bischof von Anazarba} f�hrte,
    war er als Mitglied der Mareotis-Kommission
    (vgl. \ref{sec:BriefJuliusII},32; \ref{sec:SerdicaRundbrief},6;
    Soz., h.\,e. II 25,19) ma�geblich beteiligt.
  \item[�berlieferung]Vgl. Dok. \ref{ch:AthAnaz}. Die beiden
    Briefexzerpte sind ebenfalls �bersetzungen aus dem
    Griechischen. Folgt man der Konjektur von \cite{DeBruyne}, so sind
    sie Theognis von Nicaea\index[namen]{Theognis!Bischof von Nicaea}
    zuzuordnen. Der letzte Satz des dritten Exzerpts geh�rt gegen
    Gryson noch zum Zitat, weil der Satz ebenfalls auf den Beginn
    eines Werkes verweist und weil das Zitat wahrscheinlich am Ende
    des Fragments abbricht, da noch gar keine Aussage zu dem
    angesprochenen Thema getroffen worden ist.
  \item[Fundstelle]Codex Vaticanus lat. 5750, p. 275~f. (Frgm. 4
    [\editioncite[235~f.]{Gry}])
  \end{description}
\end{praefatio}
\begin{pairs}
\selectlanguage{latin}
\begin{Leftside}
% \beginnumbering
\pstart
\noindent\kap{1}\edtext{\abb{}}{\killnumber\Cfootnote{\hskip -1em Cod. Vat. lat.
5750}}\specialindex{quellen}{chapter}{Codices!Vaticanus lat. 5750!p. 275~f.}%
\footnotesize{Similiter etiam Bithenus episcopus
\edtext{\abb{Teognius}}{\lemma{\abb{\normalfont{Teognius}}}\Dfootnote{\textit{coni. de
Bruyne} \normalfont{et cognius} \textit{cod.} \normalfont{et cognitus} \textit{coni.
Mai}}}\edindex[namen]{Theognis!Bischof von Nicaea} ad papam}:
\pend
\pstart
ergo filium genitum dicimus,
filius autem ingenitus numquam fieri potest; solum autem patrem scientes ingenitum de
sanctis \edtext{\abb{scripturis}}{\Dfootnote{+ illum solum adoramus \textit{add. de
Bruyne}}}: veneramur autem filium, quia apud nos certum est hanc eius gloriam ad patrem
ascendere.
\pend
\pstart
\kap{2}\footnotesize{et post pauca idem}:
\pend
\pstart
cum ergo \edtext{maiorem se
\edtext{\abb{patre\Ladd{m}}}{\Dfootnote{\textit{coni. de Bruyne} patre \textit{cod.} pater
\textit{coni. Mai}}}}{\Afootnote{vgl. Io 14,28}}\edindex[bibel]{Johannes!14,28|textit} ostendat,
certum esse \edtext{\abb{quia}}{\Dfootnote{+ pater est deus \textit{add. de
Bruyne}}} non solum \edtext{\abb{operatione\ladd{m}}}{\Dfootnote{\textit{coni. Wintjes}
secundum operationem \textit{coni. Gryson} ob rationem \textit{susp. Mai}}} creaturae, sed
quia ingenitus est.
\pend
\pstart
\kap{3}\footnotesize{similiter idem ipse in alia epistula}:
\pend
\pstart
de patre autem et \edtext{filio}{\Dfootnote{fili \textit{cod.}}} \edtext{dicere}{\Dfootnote{diceris \textit{cod.} dicere iusta \textit{coni. de Bruyne}}}, sicut scis, super \edtext{\abb{nubeculam}}{\Dfootnote{\textit{coni. de Bruyne} baculum \textit{cod.}}} est ambulare. ergo vere primum \edtext{deprecabor}{\Dfootnote{deprecamur \textit{coni. Mai}}} dominum, ut veniam mihi tribuat propter necessitatem, et ita de his incipiam non de plurimis quaestionibus, neque per circuitum, sed per compendium.
\pend
\endnumbering
\end{Leftside}
\begin{Rightside}
\begin{translatio}
\beginnumbering
\pstart
\noindent\kapR{1}\footnotesize{Ebenso wendet sich auch der bithynische Bischof Theognis an den Papas:}
\pend
\pstart
Also nennen wir den Sohn gezeugt, einen ungezeugten Sohn dagegen kann es niemals geben.
Wir wissen indes aus den heiligen Schriften, da� der Vater allein ungezeugt ist: Den Sohn
aber verehren wir, weil f�r uns feststeht, da� seine Herrlichkeit zum Vater aufsteigt.
\pend
\pstart
\kapR{2}\footnotesize{Und wenig sp�ter sagt derselbe:}
\pend
\pstart
Wenn er also zeigt, da� der Vater gr��er
ist als er, so steht fest, da� er es nicht nur auf Grund des Hervorbringens der Sch�pfung
ist, sondern auch weil er ungezeugt ist.
\pend
\pstart
\kapR{3}\footnotesize{Ebenso sagt derselbe selbst in einem anderen Brief:}
\pend
\pstart
Aber �ber den Vater und den
Sohn zu sprechen hei�t, wie du wei�t, �ber Wolken zu gehen. Also werde ich wahrlich zuerst
den Herrn bitten, mir wegen der Notwendigkeit zu verzeihen, und so werde ich beginnen, �ber
diese Fragen, nicht �ber die vielen anderen zu schreiben, und nicht mit vielen Umschweifen,
sondern kurz und knapp.
\pend
\endnumbering
\end{translatio}
\end{Rightside}
\Columns
\end{pairs}
% \thispagestyle{empty}
%%% Local Variables: 
%%% mode: latex
%%% TeX-master: "dokumente_master"
%%% End: 
