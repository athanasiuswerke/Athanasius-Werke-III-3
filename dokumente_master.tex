%%%%%%%%%%%%%%%% Version 0.6 der Masterdatei %%%%%%%%%%%%%%%%%%%%%%%%%
%% �nderungen:
%% 25.4.2007 v0.6 Umstellung von memoir.tex und ledmac.tex auf config.tex
%% 13.12.2005 v0.5 Bibelstellen-Index eingef�gt
%% 9.2.2005 v0.4 Umnummerierung und �nderung der Verweise
%% 28.4.2004 v0.3 \chapter und \section in der Input-Datei, nicht mehr in der Masterdatei
\documentclass[a4paper,11pt,openany]{memoir}
%%%%%%%%%%%%%%%%%%%%%%%%%%%%%
%%% Konfiguration f�r AW III
%%%%%%%%%%%%%%%%%%%%%%%%%%%%%
%%% Zusammenf�hrung von ledmac.tex und memoir.tex
%%% Mi 25 Apr 2007 23:01:23 CEST  AvS
%%% Umstellung von jurabib auf biblatex
%%% Mi Mai 23 23:09:47 2007 AvS
%%%%%%%%%%%%%%%%%%%%%%%%%%%%%
\makeatletter
%%%%%%%%% Geladene Pakete %%%%%%%%%%%
\usepackage[latin,polutonikogreek,german]{babel}
\usepackage[latin1]{inputenc}
\usepackage[T1]{fontenc}
\usepackage{etex}
\usepackage{textcomp}
\usepackage{teubner}
\usepackage{ledmac}
\usepackage{ledpar}
\usepackage{longtable,booktabs}
\usepackage{multicol}
\usepackage{tikz}
\usepackage[babel]{csquotes}
%%%%%%% Schriften %%%%%%%%%%%%%%%
\usepackage{AGaramondPro}
\usepackage[neohellenic]{psgreek}
\renewcommand{\ttdefault}{pcr}
\renewcommand{\sfdefault}{MyriadProJ}
%%%%%% biblatex-Konfiguration %%%%%%%%%%
\usepackage[style=authortitle-comp,hyperref]{biblatex}
%%
\DeclareNameFormat{default}{%
  \usebibmacro{name:first-last}{#1}{#4}{#5}{#7}%
  \usebibmacro{name:delim}}
\DeclareNameFormat{sortname}{%
  \ifnum\value{listcount}=1\relax
    \usebibmacro{name:last-first}{#1}{#4}{#5}{#7}%
    \usebibmacro{name:revsdelim}%
  \else
    \usebibmacro{name:first-last}{#1}{#4}{#5}{#7}%
    \usebibmacro{name:delim}%
  \fi}
%% Komma als Trenner
\renewcommand*{\newunitpunct}{\addcomma\space}
%% Delim zwischen Namens
\renewcommand*{\multinamedelim}{\addslash}
%% Delim vor letztem Namen
\renewcommand*{\finalnamedelim}{\addslash}
\renewcommand*{\revsdnamedelim}{}
%% Format des Titels
\DeclareFieldFormat{citetitle}{#1\isdot}
\DeclareFieldFormat{citetitle:article}{#1\midsentence}
\DeclareFieldFormat{citetitle:book}{#1\isdot}
\DeclareFieldFormat{citetitle:booklet}{#1\isdot}
\DeclareFieldFormat{citetitle:collection}{#1\isdot}
\DeclareFieldFormat{citetitle:inbook}{#1\midsentence}
\DeclareFieldFormat{citetitle:incollection}{#1\midsentence}
\DeclareFieldFormat{citetitle:inproceedings}{#1\midsentence}
\DeclareFieldFormat{citetitle:manual}{#1\isdot}
\DeclareFieldFormat{citetitle:misc}{#1\isdot}
\DeclareFieldFormat{citetitle:online}{#1\isdot}
\DeclareFieldFormat{citetitle:proceedings}{#1\isdot}
\DeclareFieldFormat{citetitle:thesis}{#1\midsentence}
\DeclareFieldFormat{citetitle:unpublished}{#1\midsentence}
\DeclareFieldFormat{citetitle:customa}{#1\isdot}
\DeclareFieldFormat{citetitle:customb}{#1\isdot}
\DeclareFieldFormat{citetitle:customc}{#1\isdot}
\DeclareFieldFormat{citetitle:customd}{#1\isdot}
\DeclareFieldFormat{citetitle:custome}{#1\isdot}
\DeclareFieldFormat{citetitle:customf}{#1\isdot}
\DeclareFieldFormat{booktitle}{#1}
\DeclareFieldFormat{journal}{#1}
\DeclareFieldFormat{maintitle}{#1}
\DeclareFieldFormat{title}{#1\isdot}
\DeclareFieldFormat{title:article}{#1\midsentence}
\DeclareFieldFormat{title:book}{#1\isdot}
\DeclareFieldFormat{title:booklet}{#1\isdot}
\DeclareFieldFormat{title:collection}{#1\isdot}
\DeclareFieldFormat{title:inbook}{#1\midsentence}
\DeclareFieldFormat{title:incollection}{#1\midsentence}
\DeclareFieldFormat{title:inproceedings}{#1\midsentence}
\DeclareFieldFormat{title:manual}{#1\isdot}
\DeclareFieldFormat{title:misc}{#1\isdot}
\DeclareFieldFormat{title:online}{#1\isdot}
\DeclareFieldFormat{title:proceedings}{#1\isdot}
\DeclareFieldFormat{title:thesis}{#1\midsentence}
\DeclareFieldFormat{title:unpublished}{#1\midsentence}
\DeclareFieldFormat{title:customa}{#1\isdot}
\DeclareFieldFormat{title:customb}{#1\isdot}
\DeclareFieldFormat{title:customc}{#1\isdot}
\DeclareFieldFormat{title:customd}{#1\isdot}
\DeclareFieldFormat{title:custome}{#1\isdot}
\DeclareFieldFormat{title:customf}{#1\isdot}
\DeclareFieldFormat{postnote}{%
  \ifpages{#1}
    {}
    {}%
  #1}
%% Anpassung citecommand \cite
\newbibmacro*{ecite}{%
  \let\cbx@tempa\empty
  \iffieldundef{shorthand}
    {\ifnameundef{labelname}
       {}
       {\printnames{labelname}%
        \def\cbx@tempa{\addcomma\space}}%
     \usebibmacro{cite:title}}%
    {\usebibmacro{cite:shorthand}}}
\renewbibmacro*{cite}{%
  \let\cbx@tempa\empty
    {\ifnameundef{labelname}
       {}
       {\printnames{labelname}%
        \def\cbx@tempa{\addcomma\space}}%
     \usebibmacro{cite:title}}%
}
%%%%%%%%%%%%%%%%% Neues citecommand \conicite
\newbibmacro*{ccite}{%
  \let\cbx@tempa\empty
\usebibmacro{cite:shorthand}\addcomma\space\usebibmacro{cite:title}}
\DeclareCiteCommand{\conicite}
  {\usebibmacro{citeinit}%
   \usebibmacro{prenote}}
  {\usebibmacro{citeindex}%
   \usebibmacro{ccite}}
  {\multicitedelim}
  {\usebibmacro{postnote}}
%% Neues citecommand \editioncite
\newbibmacro*{postnoteA}{%
  \iffieldundef{postnote}
    {}
    {\printfield{postnote}}}
\DeclareCiteCommand{\editioncite}
  {\usebibmacro{citeinit}%
   \usebibmacro{prenote}}
  {\usebibmacro{postnoteA}\addspace%
  \usebibmacro{citeindex}%
   \usebibmacro{ecite}}
  {\multicitedelim}
  {}
%%%%%%%%%% kein in: bei Zeitschriftenartikeln
\DeclareBibliographyDriver{article}{%
  \usebibmacro{bibindex}%
  \usebibmacro{author}%
  \newunit\newblock
  \usebibmacro{title+stitle}%
  \newunit\newblock
  %\usebibmacro{in:}%
  \usebibmacro{journal+issue+year+pages}%
  \newunit\newblock
  \printfield{note}%
  \newunit\newblock
  \printfield{issn}%
  \newunit\newblock
  \printfield{doi}%
  \newunit\newblock
  \printfield{addendum}%
  \newunit\newblock
  \usebibmacro{url+date}%
  \newunit\newblock
  \usebibmacro{pageref}%
  \usebibmacro{finentry}}
\bibliography{dokumente_master}
\defbibheading{edition}{\section*{Editionen}}
\defbibheading{literatur}{\section*{Sekund�rliteratur}}
%%%%%%%%% Hyperref
\usepackage[implicit=true,plainpages=false,pdfpagelabels]{hyperref}
\hypersetup{%
pdftitle = {Dokumente zum arianischen Streit},
pdfsubject = {Edition},
pdfkeywords = {Dokumente zum arianischen Streit, Edition},
pdfauthor = {\textcopyright\ 2007 Edition Athanasius Werke},
}
\usepackage{memhfixc}
\memhyperindexfalse
\usepackage{microtype}
\title{Dokumente zum arianischen Streit}
\author{Hanns Christof Brennecke\\Uta Heil\\Annette von Stockhausen\\Angelika Wintjes}
\date{\today}
%%%%%%%%%%%% Indices erstellen %%%%%%%%%%%%%%%%%%%%
% \makeindex[bibel]
% \makeindex[quellen]
% \makeindex[synoden]
% \makeindex[namen]

%%%%%%%%% Seitenlayout %%%%%%%%%%%%%
\settrimmedsize{297mm}{210mm}{*}
\setlength{\trimtop}{0pt}
\setlength{\trimedge}{\stockwidth}
\addtolength{\trimedge}{-\paperwidth}
\settypeblocksize{22.5cm}{14.5cm}{*}
\setulmargins{3.5cm}{*}{*}
\setlrmargins{*}{*}{1}
\setmarginnotes{10pt}{11pt}{\onelineskip}
\setheadfoot{\onelineskip}{2\onelineskip}
\setheaderspaces{*}{\onelineskip}{*}
\checkandfixthelayout

%%%%%%%%% Seitenstil (Kopfzeilenlayout) %%%%%%
\makepagestyle{dokumente}
\makepsmarks{dokumente}{%
  \let\@mkboth\markboth
  \def\chaptermark##1{\markboth{\thechapter. \ ##1}{\thechapter. \ ##1}}    % left mark & right marks
  \def\sectionmark##1{\markright{%
    \ifnum \c@secnumdepth>\z@
      \thesection. \ %
    \fi
    ##1}}
  \def\tocmark{\markboth{\contentsname}{\contentsname}}
  \def\lofmark{\markboth{\listfigurename}{\listfigurename}}
  \def\lotmark{\markboth{\listtablename}{\listtablename}}
  \def\bibmark{\markboth{\bibname}{\bibname}}
  \def\indexmark{\markboth{\indexname}{\indexname}}
}
\makeevenhead{dokumente}{\thepage}{\small\leftmark}{}
\makeoddhead{dokumente}{}{\small\rightmark}{\thepage}
% \makeevenfoot{dokumente}{}{\tiny\texttt{\dt{Probeausdruck Athanasius Werke III/3 -- Stand: \today\ -- Blatt \thesheetsequence\ von \thelastsheet}}}{}
% \makeoddfoot{dokumente}{}{\tiny\texttt{\dt{Probeausdruck Athanasius Werke III/3 -- Stand: \today\ -- Blatt \thesheetsequence\ von \thelastsheet}}}{}

\pagestyle{dokumente}
\aliaspagestyle{chapter}{empty}

%%%%%%%% �berschriften %%%%%%%%%%%%%%
%%%%%%%% Kapitel %%%%%%%%%%%%%%%%%
\makechapterstyle{ath}{
  \renewcommand{\printchaptername}{}
  \renewcommand{\chapternamenum}{}
  \renewcommand{\chaptitlefont}{\centering\Large}
  \renewcommand{\chapnumfont}{\chaptitlefont}
  \renewcommand{\printchapternum}{\centering \chapnumfont \thechapter\space}
  \renewcommand{\afterchaptertitle}{\par\nobreak\vskip \afterchapskip}
}
\chapterstyle{ath}
%%%%%%%% Neuer Kapitelbefehl f�r �Urkunden� %%%
%\let\kapitel\chapter
%\renewcommand{\clearforchapter}{\par}  % Chapter beginnt nicht neue Seite
\newcommand\kapitel{%
  \ifartopt\par\else
%     \clearforchapter
\par
    \thispagestyle{chapter}
    \global\@topnum\z@
  \fi
  \@afterindentfalse
  \@ifstar{\@m@mschapter}{\@m@mchapter}}
%%%%%%%% Sections u.s.w. %%%%%%%%%%%%%%
\setsecheadstyle{\large\centering}
\setsubsecheadstyle{\normalsize\centering}
\setsubsubsecheadstyle{\itshape\centering}
\setparaheadstyle{\itshape}
%%%%%%%%%%% Abst�nde vor Kapitel etc. %%%
\setlength{\beforechapskip}{2\baselineskip}
\setlength{\midchapskip}{.5\baselineskip}
\setlength{\afterchapskip}{\baselineskip}
\setbeforesecskip{2\baselineskip}
\setbeforeparaskip{0\baselineskip}
\setaftersecskip{\baselineskip}
%%%%%%%% Nummerierung nur bis section %%%%%%
\maxsecnumdepth{section}

%%%%%%% Formatierungen f�r Edition %%%%%%%%
%%%%%%% Handschriftenbeschreibung Siglenliste %%
\newenvironment{codices}%
	{\begin{longtable}[l]{p{1.6cm}p{10cm}p{2cm}}}
	{\end{longtable}\ignorespacesafterend}
\newenvironment{konjektoren}%
	{\begin{longtable}[l]{p{3cm}p{8cm}p{2cm}}}
	{\end{longtable}\ignorespacesafterend}
\newcommand{\codex}[3]{%
	#1% Sigle
	& #2% Name
	& #3% Datierung
	\\
}
%%%%%%% Paragraphenangabe in nicht-nummeriertem Text (�bersetzungen aus AW III/1+2)
\newcommand{\kapnum}[1]{\noindent\llap{\makebox[.5cm][l]{\footnotesize #1}}\ifthenelse{\equal{#1}{1}}{\noindent}{\quad}}
%%%%%%% Paragraphenangabe im nummerierten Text %%
\newcommand{\kap}[1]{\ledsidenote{\dt{#1}}}
\newcommand{\kapR}[1]{} %% Befehl f�r die Kapitelz�hlung rechts - momentan nicht ausgegeben

%%%%%%% Abk�rzungen f�r den textkritischen Apparat %%
\newcommand{\dt}{\foreignlanguage{german}}
\newcommand{\griech}{\foreignlanguage{polutonikogreek}}
\newcommand{\mg}{\textsuperscript{\,mg}}
\newcommand{\slin}{\textsuperscript{\,sl}}
\newcommand{\corr}{\textsuperscript{\,c}}
\newcommand{\ras}{\textsuperscript{\,ras}}
\newcommand{\ts}{\textsuperscript}
\newcommand{\tsub}{\textsubscript}
%%%%%% Neudefinition von teubner.sty-Abk�rzungen %%%%%%%%%%
\renewcommand{\Ladd}[1]{\dt{<}#1\dt{>}}%
\renewcommand{\ladd}[1]{{[}%
    {#1\/}{]}}%
\renewcommand{\dBar}{\textbardbl}
%%%%%% Linie �ber Buchstaben %%%%%%%%%%
\usepackage{ulem}
\protected\def\oline{\bgroup \ULdepth=-1.9ex \ULset}

%%%%%%% Definition einer Praefatio-Umgebung %%%%
\newenvironment{praefatio}%
{\footnotesize}%
{\ignorespacesafterend}%
%%%%%%%%%%%%% Listen enger gesetzt %%%%%%%%%%%

%%%%%%% Description-Umgebung so umdefiniert, dass folgende Zeilen nicht mehr eingezogen werden
\renewenvironment{description}%
               {\list{}{\labelwidth\z@ \setlength{\leftmargin}{0em}\setlength{\parsep}{0pt}
               \let\makelabel\descriptionlabel}\tightlist}%
               {\endlist\ignorespacesafterend}
% \renewcommand*{\descriptionlabel}[1]{\hspace\labelsep\normalfont\itshape #1}
%%%%%% �bersetzungs-Umgebung %%%%%%%%%%%
\newenvironment{translatio}
{\small\begin{otherlanguage}{german}}%
{\end{otherlanguage}}

%%%%%% Befehl, um den Autor eines Textst�ckes angeben zu k�nnen %%%
\newcommand{\autor}[1]{%
  \subsection{#1}
    }

%%%%%%% Fu�noten %%%%%%%%%%%%%%%%%%
%%%%%%% �Normale� Fu�noten %%%%%%%%%%%%
\setlength{\footmarksep}{0pt}
\newlength{\myFootnoteWidth}\newlength{\myFootnoteLabel}
 \setlength{\myFootnoteLabel}{.6cm}%  <-- can be changed to any valid value
\setlength{\footmarkwidth}{.6cm}%\myFootnoteLabel}
\footmarkstyle{\makebox[\myFootnoteLabel][l]{#1}}
\setlength{\thanksmarkwidth}{1em}
\setlength{\thanksmarksep}{0em}
\renewcommand{\@makefntext}[1]{\makefootmark #1}

%%%%%% Umdefinition von footnoteA (Hist. Kommentar) %%%%
% \footparagraph{A}
\renewcommand{\footnoteA}[1]{%
  \stepcounter{footnoteA}%
  \protected@xdef\@thefnmarkA{\thefootnoteA\noexpand}%
  \@footnotemarkA
  \vfootnoteA{A}{\normalfont #1}\m@mmf@prepare}
\renewcommand{\thefootnoteA}{\alph{footnoteA}}
\usepackage{perpage}
\MakePerPage{footnoteA}

%%%%%%% Ledmac-Spezifisches %%%%%%%%%%%%
%%%%%%% meist von Dirk-Jan Dekker %%%%%%%%%%
\lefthyphenmin=3
\righthyphenmin=3
%%%%%%%%%%%%%% Zeilennummer l�schen  %%%%%%%%%%%%%%%%%
\def\printlines#1|#2|#3|#4|#5|#6|#7|{\begingroup
 \l@d@pnumfalse \l@d@dashfalse
 \ifbypage@
 \ifnum#4=#1 \else
 \l@d@pnumtrue
 \l@d@dashtrue
 \fi
 \fi
 \ifl@d@pnum \l@d@elintrue \else \l@d@elinfalse \fi
 \ifnum#2=#5 \else
 \l@d@elintrue
 \l@d@dashtrue
 \fi
 \l@d@ssubfalse
 \ifnum#3=0 \else
 \l@d@ssubtrue
 \fi
 \l@d@eslfalse
 \ifnum#6=0 \else
 \ifnum#6=#3
 \ifl@d@elin \l@d@esltrue \else \l@d@eslfalse \fi
 \else
 \l@d@esltrue
 \l@d@dashtrue
 \fi
 \fi
 \ifnum#2=-1 \ledplinenumfalse \fi     % This line is new
 \ifl@d@pnum #1\fullstop\fi
 \ifledplinenum \linenumr@p{#2}\else \symplinenum\fi
 \ifl@d@ssub \fullstop \sublinenumr@p{#3}\fi
 \ifl@d@dash \endashchar\fi
 \ifl@d@pnum #4\fullstop\fi
 \ifl@d@elin \linenumr@p{#5}\fi
 \ifl@d@esl \ifl@d@elin \fullstop\fi \sublinenumr@p{#6}\fi
 \endgroup}

 \newcommand{\killnumber}{\linenum{|-1|||-1||}}
 %%%%%%%%%%%%%%%%%% Zeilennummer l�schen Ende %%%%%%%%%%%%%%%%%%%%
\renewcommand*{\para@vfootnote}[2]{%
\insert\csname #1footins\endcsname
\bgroup
\notefontsetup
\footsplitskips
\l@dparsefootspec #2\ledplinenumtrue  %%%% FIRST ADDED LINE %%%%%%%%%%%%%%
\ifnum\@nameuse{previous@#1@number}=\l@dparsedstartline\relax
\ledplinenumfalse
\fi
\ifnum\previous@page=\l@dparsedstartpage\relax
\else \ledplinenumtrue \fi
\ifnum\l@dparsedstartline=\l@dparsedendline\relax
\else \ledplinenumtrue \fi
\expandafter\xdef\csname previous@#1@number\endcsname{\l@dparsedstartline}
\xdef\previous@page{\l@dparsedstartpage}  %%%% LAST ADDDED LINE %%%%%%%%%%
\setbox0=\vbox{\hsize=\maxdimen
\noindent\csname #1footfmt\endcsname#2}%
\setbox0=\hbox{\unvxh0}%
\dp0=0pt
\ht0=\csname #1footfudgefactor\endcsname\wd0
\box0
\penalty0
\egroup}

\footparagraph{A}
\footparagraph{B}
\footparagraph{C}
\footparagraph{D}


% \def\zparafootfmt#1#2#3{%
% \normal@pars
% \hskip -1.8ex�\parfillskip=0pt plus1fil
% \notetextfont�#3\penalty-10 }
%
% \let\Cfootfmt=\zparafootfmt

% \let\Afootnoterule=\relax
% \let\Bfootnoterule=\relax
% \let\Cfootnoterule=\relax
% \let\Dfootnoterule=\relax
% \addtolength{\skip\Afootins}{0.5mm}
% \addtolength{\skip\Bfootins}{0.5mm}
% \addtolength{\skip\Cfootins}{0.5mm}
% \addtolength{\skip\Dfootins}{0.5mm}
\setlength{\skip\Afootins}{2.5mm}
\setlength{\skip\Bfootins}{2.5mm}
\setlength{\skip\Cfootins}{2.5mm}
\setlength{\skip\Dfootins}{2.5mm}
%\count\Afootins=925
%\count\Bfootins=825
%\count\Cfootins=925
% \count\Dfootins=875
\renewcommand*{\footfudgefiddle}{78}

\newcommand*{\previous@A@number}{-1}
\newcommand*{\previous@B@number}{-1}
\newcommand*{\previous@C@number}{-1}
\newcommand*{\previous@D@number}{-1}
\newcommand*{\previous@page}{-1}

%% NO \rbracket IN FRONT OF om./inv./add. ETC.

\newcommand{\abb}[1]{#1%
        \let\rbracket\nobrak\relax}
\newcommand{\nobrak}{\textnormal{}}
\newcommand{\morenoexpands}{%
        \let\abb=0%
}

\newcommand{\Aparafootfmt}[3]{%
  \normal@pars\footnotesize
  \parindent=0pt \parfillskip=0pt plus1fil
  \notenumfont\printlines#1|%
\ifledplinenum
\enspace%
\else
{\dBar \hskip .8em plus0em minus.4em}%
\fi
  \select@lemmafont#1|#2\rbracket\enskip
  \notetextfont
  #3\penalty-10\hskip 1em plus 4em minus.4em\relax }

\newcommand{\Bparafootfmt}[3]{%
  \normal@pars\footnotesize
  \parindent=0pt \parfillskip=0pt plus1fil
  \notenumfont\printlines#1|%
\ifledplinenum
\enspace%
\else
{\dBar \hskip .8em plus0em minus.4em}%
\fi
  \select@lemmafont#1|#2\rbracket\enskip
  \notetextfont
  #3\penalty-10\hskip 1em plus 4em minus.4em\relax }

\newcommand{\Cparafootfmt}[3]{%
  \normal@pars\footnotesize
  \parindent=0pt \parfillskip=0pt plus1fil
  \notenumfont\printlines#1|%
\ifledplinenum
\enspace%
\else
{}% keine dbar
\fi
  \select@lemmafont#1|#2\rbracket\enskip
  \notetextfont
  #3\penalty-10\hskip 1em plus 4em minus.4em\relax }

\newcommand{\Dparafootfmt}[3]{%
  \normal@pars\footnotesize
  \parindent=0pt \parfillskip=0pt plus1fil
  \notenumfont\printlines#1|%
\ifledplinenum
\enspace%
\else
{\dBar \hskip .8em plus0em minus.4em}%
\fi
  \select@lemmafont#1|#2\rbracket\enskip
  \notetextfont
  #3\penalty-10\hskip 1em plus 4em minus.4em\relax }
%% END DEFINITION OF \abb


\let\Afootfmt=\Aparafootfmt
\let\Bfootfmt=\Bparafootfmt
\let\Cfootfmt=\Cparafootfmt
\let\Dfootfmt=\Dparafootfmt

%%%%%%%%% Referenzierung von Textpassagen %%%%%%%%%%%%%%%
%% Dirk-Jan Dekker in comp.text.tex: cross-referencing entire passages in ledmac
\newcommand{\refpassage}[2]{%
   \xpageref{#1},\xlineref{#1}%
   \ifnum\xpageref{#1}=\xpageref{#2}
      \ifnum\xlineref{#1}=\xlineref{#2}
      \else
         \endashchar\xlineref{#2}%
      \fi
   \else
      \endashchar\xpageref{#2},\xlineref{#2}%
   \fi
}

%%%%%%%% Ma�e und Einstellungen %%%%%%%%%%%%%%%%%%%
\lineation{page}
\sidenotemargin{left}
\linenummargin{right}
\setlength{\linenumsep}{0.4pc}
\setlength{\parindent}{1em}
\setlength\parskip{0em} 
   \tolerance1414
   \hbadness1414
   \vbadness1414
\emergencystretch35pt
   \hfuzz0.5pt
\widowpenalty=10000
   \vfuzz \hfuzz
 \clubpenalty=10000

\renewcommand{\symplinenum}{}   % evt. \textbar of $\|$ invullen
%%%%%%% Schriftgr��en %%%%%%%%%%%%%%%%%%%
\renewcommand*{\notenumfont}{\bfseries\latintext}
\newcommand*{\notetextfont}{\footnotesize}
\renewcommand*{\numlabfont}{\normalfont\scriptsize\latintext}

%%%%%%% Paralleltext  %%%%%%%%%%%%
\maxchunks{300}
\renewcommand*{\Rlineflag}{}
% \renewcommand*{\leftlinenum}{}
% \renewcommand*{\rightlinenum}{}
\renewcommand*{\leftlinenumR}{}
\renewcommand*{\rightlinenumR}{}
%%%%%%% Spaltenabstand %%%%%%%%%%
\setlength{\Lcolwidth}{.45\textwidth}
\setlength{\Rcolwidth}{.51\textwidth}
%\setlength{\Lcolwidth}{65mm}
%\setlength{\Rcolwidth}{75mm}

%%%%%%% TOC %%%%%%%%%%%%%%%%%%%%%
%%%%%%% Style des Titels u. d. Kapitel�berschriften %%%
\renewcommand{\printtoctitle}[1]{\centering\normalfont\Large #1}
\renewcommand{\cftchapterfont}{\raggedright\normalfont\large}
\renewcommand{\cftchapterpagefont}{\normalfont}
%%%%%%% Abst�nde zwischen Nummer und Titel
\setlength{\cftchapternumwidth}{3em}
\setlength{\cftsectionnumwidth}{3em}
\setlength{\cftsectionindent}{0em}
%%%%%%% Punkte %%%%%%%%%%%%%%%%%%%%
\renewcommand{\cftsectiondotsep}{\cftdotsep}
\renewcommand{\cftchapterdotsep}{\cftdotsep}

%%%%%%% Captions %%%%%%%%%%%%%%%%%%%
\captionnamefont{\scriptsize}
\captiontitlefont{\scriptsize}

%%%%%%% Index %%%%%%%%%%%%%%%%%%%%%
\renewcommand{\pagelinesep}{,} % Trenner zw. Seite u. Zeile
\renewcommand*{\see}[2]{\textit{\seename} #1}
%%%%%%% Dreispaltig und sonstige Index-Formatierungen 
\setlength{\indexcolsep}{1cm}
% \setlength{\indexrule}{.1pt}

\renewenvironment{theindex}{%
  \begin{multicols}{3}[\section*{\indexname}\preindexhook][10\baselineskip]%  
 \raggedcolumns
\raggedright\footnotesize
  \indexmark
  \setlength{\columnseprule}{\indexrule}
  \setlength{\columnsep}{\indexcolsep}
  \ifnoindexintoc\else
    \phantomsection
    \addcontentsline{toc}{section}{\indexname}
  \fi
%   \thispagestyle{chapter}
  \parindent\z@
  \parskip\z@ \@plus .3\p@\relax
  \let\item\@idxitem}
{\end{multicols}}
\renewcommand{\@idxitem}  {\par\hangindent 40\p@}
\renewcommand{\subitem}   {\par\hangindent 20\p@}
\renewcommand{\subsubitem}{\par\hangindent 40\p@}
\renewcommand{\indexspace}{\par \vskip 20\p@ \@plus5\p@ \@minus3\p@\relax}

%%%%%%% Diskussionsbedarf %%%%%%%%%%%%%%%
\usepackage{framed}
% \newcommand{\diskussionsbedarf}{\begin{framed}\bfseries\centering Diskussionsbedarf\end{framed}}
%\newcommand{\frage}[1]{\LitNil\LitNil\texttt{#1}\LitNil\LitNil\ }
\newcommand{\diskussionsbedarf}{}
\newcommand{\frage}[1]{#1}
\makeatother
\begin{document}
%%%%%%%%%%%%%%%%%%%%%%%%%%%%%%%%%%%%%%%%%%%%%
%%%%%%%%%%%%%%%%%% TITELEI %%%%%%%%%%%%%%%%%%
%%%%%%%%%%%%%%%%%%%%%%%%%%%%%%%%%%%%%%%%%%%%%
% \begin{titlingpage}
% \thispagestyle{empty}
% %%%%%%%%% Schmutzblatt
% \begin{center}
% \large{ATHANASIUS WERKE}\par
% \normalsize{DRITTER BAND}\par
% \normalsize{ZWEITER TEIL}
% \end{center}
% \clearpage
% %%%%%%%%%%%%%% Reihenseite
% \thispagestyle{empty}
% \begin{center}
% \Huge{ATHANASIUS WERKE}\\
% \vspace*{1cm}
% \large{DRITTER BAND}\\
% \large{ZWEITER TEIL}\\
% \vspace*{1cm}
% \LARGE{DOKUMENTE ZUR GESCHICHTE DES ARIANISCHEN STREITES}\\
% \vspace*{1cm}
% \large{HERAUSGEGEBEN IM AUFTRAG DER BERLIN-BRANDENBURGISCHEN AKADEMIE DER
% WISSENSCHAFTEN}\par
% \large{VON HANNS CHRISTOF BRENNECKE, UTA HEIL, ANNETTE VON STOCKHAUSEN UND ANGELIKA
% WINTJES}\par
% \vspace*{2cm}
% \large{3. LIEFERUNG}
% \end{center}
% \vspace*{\fill}
% \begin{center}
%  \includegraphics{degruyter_signet}
%  % degruyter_signet.png: 555x1038 pixel, 1200dpi, 1.17x2.20 cm, bb=0 0 33 62
% \end{center}
% \begin{center}
% \large{WALTER DE GRUYTER \LitNil\ BERLIN \LitNil\ NEW YORK}\par
% \large{2007}
% \end{center}
% \clearpage
% %%%%%%%%%%%%% Titelseite
% \thispagestyle{empty}
% \begin{center}
% \Huge{ATHANASIUS WERKE}\\
% \vspace*{1cm}
% \large{DRITTER BAND}\\
% \large{ZWEITER TEIL}\\
% \vspace*{1cm}
% \LARGE{DOKUMENTE ZUR GESCHICHTE DES ARIANISCHEN STREITES}\\
% \vspace*{2cm}
% \large{3. LIEFERUNG}\par
% \large{BIS ZUR EKTHESIS MAKROSTICHOS}\par
% \vspace*{1cm}
% \large{HERAUSGEGEBEN VON\\HANNS CHRISTOF BRENNECKE, UTA HEIL, ANNETTE VON
% STOCKHAUSEN UND ANGELIKA WINTJES}
% \end{center}
% \vspace*{\fill}
% \begin{center}
%  \includegraphics{degruyter_signet}
%  % degruyter_signet.png: 555x1038 pixel, 1200dpi, 1.17x2.20 cm, bb=0 0 33 62
% \end{center}
% \begin{center}
% \large{WALTER DE GRUYTER \LitNil\ BERLIN \LitNil\ NEW YORK}\par
% \large{2007}
% \end{center}
% \clearpage
% % copyright page
% \begingroup
% 	\begin{center}
% 	\footnotesize
% 	\begin{tabular}{ll}
% 	Stand: \today
% 	\end{tabular}
% 	\end{center}
% \vspace*{\fill}
% 	\begin{center}
% 	\footnotesize
% 	Die Deutsche Bibliothek -- CIP-Einheitsaufnahme\\
% 	\vspace*{1cm}
% 	\textbf{Athanasius <Alexandrinus>:}\\
% 	Werke / Athanasius.\\
% 	hg. im Auftrag der Berlin-Brandenburgischen Akademie der Wissenschaften\\
% 	von Hanns Christof Brennecke, Uta Heil, Annette von Stockhausen, Angelika
% Wintjes\\
% 	-- Berlin ; New York : de Gruyter\\
% 	Teilw. griech. -- Teilw. in griech. Schr.\\
% 	Band III. Teil 2. Lieferung 3\\
% 	Dokumente zum arianischen Streit\\
% 	2007\\
% 	\vspace*{1cm}
% 	ISBN 978-3-11-019104-2\\
% 	\vspace*{1cm}
% 	\textcopyright{} 2007 by Walter de Gruyter GmbH \& Co., D-10785 Berlin\\
% 	\end{center}
% \endgroup
% \end{titlingpage}
%%%%%%%%%%%%%%%%%%%%%%%%%%%%%%%%%%%%%%%%%%%%
%%%%%%%%%%%%%%%%% VORWORT %%%%%%%%%%%%%%%%%%%%%
%%%%%%%%%%%%%%%%%%%%%%%%%%%%%%%%%%%%%%%%%%%%
\frontmatter
\pagenumbering{roman}
\setcounter{page}{5}
\chapter*{Vorwort}
\markboth{Vorwort}{Vorwort}

Im sogenannten ">arianischen Streit"< geht es um die Trinit�t,
d.\,h. um die Frage, wie die Einheit und Dreiheit von Gott Vater, dem
Sohn und dem heiligen Geist zu verstehen und zu beschreiben
ist. Dieser Streit pr�gt das gesamte vierte Jahrhundert und reicht
teilweise bis in das sechste Jahrhundert hinein. Im Verlauf der
Auseinandersetzungen wurde eine F�lle von gr��eren theologischen
Abhandlungen verfa�t, die inzwischen gr��tenteils als kritische Editionen 
vorliegen. Nicht minder bedeutsam sind Dokumente wie
Synodalentscheidungen, Briefe von beteiligten Personen oder auch der
jeweiligen Kaiser, die in dieser Ausgabe gesammelt werden.

In den Jahren 1934 und 1935 erschienen bereits im Rahmen des
deutsch{}-amerikanischen Forschungsprojektes einer ersten kritischen
Edition der Werke des Athanasius von Alexandrien, das seit 1930 von
der Kirchenv�terkommission der damaligen Preu�ischen Akademie der
Wissenschaften in Berlin unter Leitung von Hans Lietzmann betreut
wurde, zwei insgesamt nur 75 Seiten umfassende Faszikel ">Urkunden zur
Geschichte des arianischen Streites"< als Auftakt zu der geplanten
Ausgabe der Werke des alexandrinischen Bischofs. Mit der Edition
dieser ">Urkunden zur Geschichte des arianischen Streites"< hatten die
Herausgeber der Athanasius{}-Ausgabe, Hans Lietzmann und Robert Casey,
Lietzmanns Berliner Sch�ler Hans{}-Georg Opitz beauftragt, dem auch
Band II der Ausgabe, der die sogenannten Apologien enthalten
sollte,\footnote{\cite{Opitz1935}; \cite{Brennecke2006}; vgl. auch die
  Habilitationsschrift von \textcite{Opitz:Untersuchungen}.}  �bertragen worden war.

Das urspr�nglich gar nicht geplante Projekt einer eigenen Sammlung von
">Urkunden zum arianischen Streit"< ging auf Anregungen von Eduard
Schwartz zur�ck. Schwartz selbst hatte seit Beginn des Jahrhunderts
begonnen, Dokumente zur Kirchengeschichte des 4.  Jahrhunderts und vor
allem zur Lebensgeschichte des Athanasius im Zusammenhang seiner
umfangreichen Athanasiusstudien zusammenzustellen.\footnote{Der dritte
  Band seiner 1959 postum von W. Eltester und H.{}-D. Altendorf
  herausgegebenen Gesammelten Schriften vereinigt die wichtigsten
  Arbeiten zur Geschichte des Athanasius und des arianischen Streites,
  die seit Beginn des 20. Jahrhunderts erschienen waren; vgl.
  \cite{Schwartz:GesIII}.}  Als ">Urkunden"< sah Schwartz in erster
Linie Briefe der an den Auseinandersetzungen Beteiligten sowie Texte
von Synoden und Berichte dar�ber an, au�erdem nat�rlich auch
Regesten. Deren kritische Edition und historische wie theologische
Kommentierung k�nne �berhaupt erst die Darstellung einer Geschichte
des vierten Jahrhunderts erm�glichen.

Ausgehend von der von Schwartz immer wieder unterstrichenen, in der
�lteren Forschung nicht selten vernachl�ssigten Tatsache, da� in den
Werken des Athanasius und anderer Beteiligter an den theologischen und
kirchenpolitischen Auseinandersetzungen eine gro�e Zahl von Dokumenten
�berliefert ist, war es das Ziel, diese, losgel�st von ihrem
sekund�ren literarischen �berlieferungszusammenhang, kritisch zu
edieren und zu kommentieren.


Nach den Protokollen der Kirchenv�terkommission der Preu�ischen
Akademie der Wissenschaften plante Opitz urspr�nglich, angeregt von
Eduard Schwartz, eine Sammlung von Dokumenten zur Geschichte des
Athanasius als Band III der Athanasiusausgabe zusammenzustellen. Noch
vor Erscheinen des ersten Faszikels 1934 wurde von ihm offensichtlich
in Absprache mit den Herausgebern aber das Konzept dahingehend
abge�ndert, alle relevanten Dokumente zum arianischen Streit bis zum
2.  �kumenischen Konzil von Konstantinopel 381 zu edieren und zu
kommentieren.\footnote{Laut Prospekt des Verlages Walter de Gruyter \&
  Co. aus dem Jahre 1934 waren folgende Faszikel geplant: 1. Von den
  Anf�ngen bis zum Tode Alexanders von Alexandrien (318--328), 2. Von
  der Inthronisation des Athanasius bis zur Synode von Serdica
  (328--343), 3. Von Serdica bis zur Synode von Alexandrien 362
  (343--362), 4. Von 362 bis zum 2. Konzil in Konstantinopel (381).}

In den ersten beiden Faszikeln hatte Opitz, der dar�ber in st�ndigem
brief"|lichen Austausch stand, wie die an Eduard Schwartz gerichteten
Briefe zeigen,\footnote{Die Briefe befinden sich in dem in der
  M�nchener Staatsbibliothek aufbewahrten Nachla� von Eduard Schwartz
  (Bayerische Staatsbibliothek M�nchen, Schwartziana IIA), einige
  Briefe von Schwartz an Opitz befinden sich im Nachla� von
  Hans{}-Georg Opitz, der in der Athanasius{}-Arbeitsstelle in
  Erlangen aufbewahrt wird. Der Hauptteil der Briefe an Opitz ist mit
  dem gr��ten Teil seines Nachlasses den Wirren des zweiten
  Weltkrieges zum Opfer gefallen.} insgesamt 34 Dokumente von den
(nicht sicher zu datierenden) Anf�ngen der Auseinandersetzung um Arius
bis zur Nachgeschichte der Synode von Nicaea ediert und
kommentiert. Die von ihm erstellten und erhaltenen Listen f�r die in
Zukunft zu edierenden Texte deuten auf eine Konzeption, die von uns
in dieser Form f�r nicht fortsetzbar gehalten wird: Die von Opitz f�r die Edition
aufgenommenen St�cke beschr�nken sich nicht auf den eigentlichen
arianischen Streit, sondern umfassen auch allgemein Dokumente zur Kirchenpolitik des vierten Jahrhunderts. Dies geschah in enger
Anlehnung an das kirchenhistorische Konzept von Eduard Schwartz, der
davon ausging, da� in der Sp�tantike Kirchen{}- und Reichspolitik
unentwirrbar miteinander verwoben seien.

Wilhelm Schneemelcher, der nach dem Tode von Hans{}-Georg Opitz im
Jahr 1941 nach dem Krieg die Fortsetzung der Herausgabe der ">Urkunden
zur Geschichte des arianischen Streites"< �bernommen hatte, behielt
dieses Konzept in seiner unver�ffentlichten G�ttinger Habilitation von
1949\footnote{\cite{Schneemelcher1949}.} im Wesentlichen bei und
setzte die Urkundensammlung bis zum Tode Konstantins (337) fort,
verfolgte dann aber das ganze Projekt nicht weiter.

Nachdem die Erlanger Athanasius{}-Arbeitsstelle die Fortsetzung der
">Urkunden zur Geschichte des arianischen Streites"< �bernommen hatte,
haben wir uns entschlossen, die Auswahl der zu edierenden und
kommentierenden Texte auf den eigentlichen arianischen Streit, also
die theologischen und kirchenpolitischen Auseinandersetzungen um die
Ausbildung einer christlichen Trinit�tslehre, zu
beschr�nken. Dokumente zu den nur mittelbar mit dem trinitarischen
Streit verbundenen personalpolitischen Auseinandersetzungen sowie alle
Dokumente, die zwar mit Athanasius, aber nicht unbedingt mit dem
trinitarischen Streit verbunden sind, ferner die Texte zum
melitianischen Schisma wurden deshalb hier ausgelassen und sind f�r
eine sp�tere Sammlung geplant.

Das vorliegende Faszikel setzt chronologisch die Arbeit von Opitz fort
und beginnt mit den Nachrichten �ber die Wiederherstellung der Einheit
der Kirche in �gypten (Dok. \ref{ch:Arius}) und die Synode von
Jerusalem im Jahre 335 (Dok. \ref{ch:Jerusalem335}). Der inhaltliche
Schwerpunkt liegt auf den r�mischen und antiochenischen Synoden der
Jahre 340/41 (Dok. \ref{sec:BriefJulius}--\ref{sec:BriefJuliusII}) und
der Synode von Serdica 343
(Dok. \ref{ch:SerdicaEinl}--\ref{sec:NominaepiscSerdikaOst}). Den
Abschlu� dieses Faszikels bildet die ">Ekthesis makrostichos"< aus dem
Jahre 344 (Dok. \ref{ch:Makrostichos}), die noch in die unmittelbare
Nachgeschichte der Synode von Serdica geh�rt. Die folgenden Faszikel
sollen die Dokumente �ber das 2. �kumenische Konzil von 381 hinaus bis
zum Ende der sogenannten germanischen arianischen (besser:
hom�ischen) Landeskirchen an der Wende zum 7.  Jahrhundert beinhalten.

Abweichend von dem von Schwartz inaugurierten und von Opitz
durchgef�hrten Konzept haben wir uns entschlossen, den seit nunmehr
siebzig Jahren f�r diese Sammlung eingeb�rgerten Begriff ">Urkunden"<
aufzugeben. Die mit dem vorliegenden Faszikel beginnende Fortsetzung
soll unter dem Titel ">Dokumente zur Geschichte des arianischen
Streites"< erscheinen,\footnote{Vgl. \cite{Schwartz:Dokumente}.} 
weil der Begriff ">Urkunde"< in der
Geschichtswissenschaft im
rein juristischen Sinne verwendet wird, so da� der viel umfassendere Begriff
">Dokument"< angemessener erscheint.

In der Pr�sentation der Dokumente sind wir weithin den Vorgaben von
Opitz gefolgt. Den Texten vorangestellt ist aber je eine einleitende
Bemerkung zu Datierung und �berlieferung. Neu gegen�ber Opitz ist
au�erdem die deutsche �bersetzung, die heute notwendig ist, wenn eine
solche Dokumentensammlung von Altertumswissenschaftlern und Theologen
benutzt werden soll.

Der �u�erst knappe Kommentar, der in erster Linie dem historischen
Verst�ndnis dienen soll, ist der �bersetzung beigef�gt. Hier wurde auf
eine Auseinandersetzung mit der je schnell veraltenden
Sekund�rliteratur so weit wie irgend m�glich verzichtet und deswegen
vorwiegend auf weiterf�hrende Quellentexte verwiesen.

Parallel�berlieferungen hatte Opitz, wenn es sich dabei um
�bersetzungen in andere Sprachen des antiken Christentums handelte,
getrennt dargeboten, unterschiedliche �berlieferungen in ein und
derselben Sprache dagegen im kritischen Apparat vermerkt und zur
Textkonstitution herangezogen. Auch dieses Verfahren geht letztlich
auf die Methode von Eduard Schwartz bei der Herausgabe der Akten der
�kumenischen Konzile zur�ck, wo Schwartz die verschiedenen Sammlungen
ebenfalls jeweils getrennt voneinander ediert hatte. Hier erscheint
dieses Verfahren aber als nicht angemessen, da es sich nicht um
unterschiedliche Sammlungen mit je v�llig eigener
�berlieferungsgeschichte handelt. Wir haben den Versuch unternommen,
aus den verschiedenen �berlieferungszusammenh�ngen je einen Text zu
rekonstruieren. Denn bei der Dokumentensammlung geht es um eine
kritische Edition des einzelnen Dokuments in seiner zu
rekonstruierenden urspr�nglichen Form, d.\,h. unter Absehung
eventueller redaktioneller Zus�tze, Auslassungen etc. durch den
tradierenden Autor, der nicht selten seiner Tendenz entsprechend
Dokumente in einen anderen Kontext gestellt, uminterpretiert oder
ver�ndert hat. Diese Rekonstruktion ist auch bei Dokumenten, die nur
bei einem Autor �berliefert sind, durchzuf�hren, selbst wenn sie
bereits kritisch ediert sind, da auch in diesem Falle redaktionelle
Eingriffe wie Auslassungen oder Hinzuf�gungen denkbar sind, auch wenn
sie bei einer solchen �berlieferungslage nicht so gut nachweisbar
sind.  Ebenfalls im Gegensatz zu Opitz haben wir schlie�lich auf
R�ck�bersetzungen in die angenommene urspr�ngliche Sprache eines
Dokuments verzichtet.\footnote{Eine aus Gr�nden der Praktikabilt�t
  gemachte Ausnahme ist der textkritische Apparat bei dem auf syrisch
  �berlieferten Text der Ekthesis der ">�stlichen"< Synode von Serdica
  (Dok. \ref{sec:BekenntnisSerdikaOst}).}

In Anlehnung und Fortf�hrung der Untersuchungen von Eduard Schwartz
hatte Hans{}-Georg Opitz Untersuchungen zur Chronologie der Anf�nge
des arianischen Streites als Vorarbeit
ver�ffentlicht\footnote{\cite{Opitz:Zeitfolge}.} und danach die
Reihenfolge der in den beiden ersten Faszikeln erschienenen Dokumente
festgelegt. Durch Opitz' Ausgabe ist diese von Schneemelcher im
Prinzip �bernommene und modifizierte
Chronologie\footnote{\cite{Schneemelcher:Chronologie}.} w�hrend des
letzten halben Jahrhunderts fast kanonisch geworden. Die vielfach erst
durch die Quellenedition von Opitz neu angesto�ene Forschung macht nun
deutlich, da� die auf Forschungen der ersten H�lfte des vergangenen
Jahrhunderts zur�ckgehende Chronologie so nicht mehr �bernommen werden
kann, wie seither zahlreiche Arbeiten zeigen. Auf der Basis der
Diskussionen des letzten halben Jahrhunderts haben wir neue
�berlegungen zur Chronologie zusammengestellt, mit denen nach der
langen Unterbrechung das Werk von Opitz fortgesetzt werden soll. Uta
Heil ist es zu danken, da� daraus eine eigene Untersuchung zur
Chronologie des arianischen Streites geworden ist.

Auf Wunsch des Verlages haben wir dar�ber hinaus der Fortsetzung der
Dokumentensammlung von Opitz eine deutsche �bersetzung der von Opitz
in den beiden ersten Faszikeln bereits edierten Texte vorangestellt,
hier aber auf einen die Forschung der letzten Jahrzehnte
ber�cksichtigenden Kommentar im einzelnen verzichtet. Das dazu
Notwendige ist in den �berlegungen zu einer neuen Chronologie der
Ereignisse gesagt. Aufgrund unserer neuen Chronologie haben wir auch
die �bersetzungen der von Opitz edierten Texte neu sortiert und
numeriert (Dok. 1--37), wobei zur Orientierung die Z�hlung von Opitz
je mit angegeben wird.\footnote{Vgl. auch die Synopse auf
  S. \pageref{synopse}.}

Da eine derartige Dokumentenedition auf sowohl vorkritische als auch
moderne kritische Editionen, wo sie zur Verf�gung stehen,
zur�ckgreifen mu�, haben wir die nach sehr unterschiedlichen Kriterien
angewandte Zeichensetzung der verschiedenen Ausgaben vereinheitlicht,
ohne das im Apparat je zu vermerken; dasselbe gilt f�r die
stillschweigende Korrektur eindeutiger Rechtschreibfehler.

\vspace*{\baselineskip}
\noindent Die Deutsche Forschungsgemeinschaft hat es durch gro�z�gige
Unterst�tzung m�glich gemacht, dieses in den politischen Katastrophen
des 20. Jahrhunderts liegengebliebene wissenschaftliche Projekt wieder
aufzunehmen, fortzuf�hren und vielleicht zu vollenden.  Ihr sei daf�r
an dieser Stelle ausdr�cklich gedankt. Die
Friedrich{}-Alexander{}-Universit�t Erlangen{}-N�rnberg hat das
Projekt in vieler Hinsicht unterst�tzt. Ihr und besonders dem Kanzler
der Universit�t, Herrn Thomas Sch�ck, sei ebenfalls gedankt. Die
Berlin{}-Brandenburgische Akademie hat als Nachfolgerin der ehemaligen
Preu�ischen Akademie der Wissenschaften weiterhin die Schirmherrschaft
�ber die Ausgabe der Werke des Athanasius �bernommen; die Patristische
Kommission der Akademien der Wissenschaften hat uns f�r die
schwierigen Korrekturarbeiten einen namhaften Betrag zur Verf�gung
gestellt. Beiden sei ebenfalls daf�r gedankt.

Wir danken Prof. Dr. Michael Durst daf�r, da� er uns eine vorl�ufige
Fassung seiner Edition der Schrift De synodis des Hilarius von
Poitiers zur Verf�gung gestellt hat.  PD Pfarrer Dr. M. Westerhoff hat
sich in besonderer Weise um die syrisch �berlieferten Dokumente
verdient gemacht, Frau H. Erlwein die mit einem derartigen
Drittmittelprojekt verbundene Verwaltungsarbeit zus�tzlich auf sich
genommen, A. Bauer und G. Schwarz die m�hselige Korrekturarbeit,
Ch. Hofmann und Ch. M�ller haben als studentische Hilfskr�fte unsere
Arbeit in vieler Hinsicht unterst�tzt. Allen sei f�r ihren Einsatz und
das Engagement gedankt.


\vspace*{\baselineskip}
\noindent Erlangen, im Mai 2007

\noindent Hanns Christof Brennecke, Uta Heil, Annette von Stockhausen
und Angelika Wintjes

\cleardoublepage
\selectlanguage{german}
\renewcommand{\contentsname}{Inhalt}
\tableofcontents*
\chapter*[Verzeichnis der verwendeten Siglen und Abk�rzungen]{Verzeichnis der verwendeten Siglen und Abk�rzungen}
\addcontentsline{toc}{chapter}{Verzeichnis der verwendeten Siglen und Abk�rzungen}
\markboth{Verzeichnis der verwendeten Siglen und Abk�rzungen}{Verzeichnis der verwendeten Siglen und Abk�rzungen}
\subsection*{Anonymus Cyzicenus (Anon.\,Cyz.)}
\begin{codices}
	\codex{A}{Codex Ambrosianus gr. M 88 sup. (534)}{s. XIII}
\end{codices}

\subsection*{Athanasius (Ath.)}
\begin{codices}
 	\codex{syn.}{De synodis}{}
	\codex{h.\,Ar.}{Historia Arianorum}{}
	\codex{apol.\,sec.}{Apologia secunda}{}
\end{codices}

\begin{codices}
	\codex{B}{Codex Basiliensis gr. A III 4}{s. XIII}
	\codex{K}{Codex Athous Vatopediou 5/6}{s. XIV [2. V.]}
	\codex{P}{Codex Parmensis Pal. 10}{s. XIII [2. H.]}
	\codex{O}{Codex Scorialensis gr. X II 11 (371)}{s. XIII--XIV}
	\codex{R}{Codex Parisinus gr. 474}{s. XI}
	\codex{E}{Codex Scorialensis gr. \griech{W} III 15 (548)}{s. XII--XIII}
\end{codices}

\subsection*{Epiphanius (Epiph.)}
\begin{codices}
 	\codex{haer.}{Adversus haereses}{}
\end{codices}

\begin{codices}
	\codex{J}{Codex Jenensis Bos. f. 1}{a. 1304}
\end{codices}

\subsection*{Eusebius von Caesarea (Eus.)}
\begin{codices}
 	\codex{Marcell.}{Contra Marcellum}{}
\end{codices}

\begin{codices}
	\codex{V}{Codex Athous Vatopediou 180}{s. XIV in.}
\end{codices}
 
\subsection*{Hilarius (Hil.)}
\begin{codices}
	\codex{coll.\,antiar.}{Hilarius, Collectanea antiariana Parisina}{}
	\codex{syn.}{Hilarius, De synodis}{}
        \codex{ad Const.}{Hilarius, Liber I ad Constantium}{}
\end{codices}
 
\subsubsection*{Collectanea antiariana Parisina}
\begin{praefatio}
Die Zuweisung von Siglen durch A. Feder ist problematisch und wurde daher nicht grunds�tzlich beibehalten. Bei der Sigle \textit{Cm1} handelt es sich um von Feder f�r textkritisch relevant eingesch�tzte Korrekturen durch den Schreiber der Handschrift C selbst, was daher hier als C\corr\ angegeben wird. Bei der Sigle \textit{Cm2} handelt es sich um handschriftlich eingetragene Konjekturen von N. Faber in der Handschrift C, bei der Sigle \textit{Fab} um Lesarten aus der Edition von N. Faber von 1598, die hier beide mit \textit{coni. Faber} wiedergegeben werden. Die Lesarten, die von Feder unter der Sigle \griech{g} angegeben werden und bei denen es sich um von P. Coustant in dessen Edition von 1693 �bernommene Konjekturen N. Fabers handelt, werden ebenfalls mit \textit{coni. Faber} wiedergegeben.

Die Herkunft der von A. Feder in den Text aufgenommenen Varianten, die nicht durch A belegt sind, ist nicht mehr genau nachzuvollziehen, da Feder selbst nur Hinweise gibt (\cite[LXI]{Hil:coll}). Auf die Nachpr�fung aller bisherigen Editionen wurde hier verzichtet, da im Rahmen der Edition der Dokumente zum arianischen Streit eine dringend notwendige Neuedition der Collectanea antiariania Parisina nicht zu leisten war. Soweit m�glich wurde hier dennoch gegen�ber Feder eine Kl�rung des Befundes herbeigef�hrt.

Lesarten der beiden Apographa der Handschrift A (C und T) werden in Dok. \ref{ch:Konstantinopel336}, \ref{sec:B}, \ref{sec:RundbriefSerdikaOst} und \ref{sec:NominaepiscSerdikaOst} im Apparat als Konjekturen vermerkt, da davon auszugehen ist, da� es sich um solche handelt.

\end{praefatio}

\begin{codices}
	\codex{Hil.}{Hilarius nach Handschrift A}{}
	\codex{A}{Codex Parisinus Armam. 483}{s. IX}
	\codex{C}{Codex Parisinus lat. 1700 (fr�her Colbert. 2568. Reg. 3982.3.3)}{s.
	XVI}
	\codex{T}{Codex Pithoeanus}{s. XV}
	\codex{S}{Lesarten aus Codex Remensis S. Remigii anni 361}{s. XII}
	\codex{}{}{}
	\codex{\textit{\greektext P}}{\textit{Collectio Sanblasiana}}{}
	\codex{B\tsub{1}}{Codex Coloniensis CCXIII, App. V}{s. VIII}
	\codex{B\tsub{2}}{Codex Sanblasianus S. Pauli ap. Carinth. XXV a/7}{s. VII}
	\codex{B\tsub{3}}{Codex Parisinus lat. 3836}{s. VIII}
	\codex{B\tsub{4}}{Codex Lucensis 490}{s. VIII}
	\codex{B\tsub{5}}{Codex Parisinus lat. 4279}{s. IX}
	\codex{B\tsub{6}}{Codex Parisinus lat. 1455}{s. IX}
	\codex{}{}{}
	\codex{Di}{Codex Monacensis 5508 (Collectio cod. Diessen)}{s. IX}
	\codex{}{}{}
	\codex{\textit{M}}{\textit{Collectio cod. S. Mauri}}{}
	\codex{M\tsub{1}}{The Hague, Mus. Meerm.-Westr. 10.B.4 (9)}{s. VIII}
	\codex{M\tsub{2}}{Codex Parisinus lat. 1451}{s. VIII}
	\codex{M\tsub{3}}{Codex Vaticanus Reg. lat. 1127}{s. IX}
	\codex{}{}{}
	\codex{\textit{E}}{\textit{Collectio Hadriana aucta}}{}
	\codex{E\tsub{2}}{Codex Monacensis 14008}{s. IX/X}
	\codex{E\tsub{4}}{Codex Vaticanus lat. 5845}{s. X}
	\codex{H\tsub{2}}{Codex Vallicellianus A 5}{s. IX/X}
	\codex{H\tsub{3}}{Codex Vercellensis LXXVI}{s. X}
	\codex{H\tsub{4}}{Codex Vaticanus lat. 1353}{s. XII}
	\codex{}{}{}
	\codex{\textit{\greektext D}}{\textit{Collectio Dionysio-Hadriana}}{}
	\codex{D\tsub{1}}{Codex Parisinus lat. 8921}{s. VIII}
	\codex{D\tsub{2}}{Codex Parisinus lat. 11710}{s. IX}
	\codex{D\tsub{3}}{Codex Parisinus lat. 3840}{s. IX}
	\codex{D\tsub{4}}{Codex Parisinus lat. 11711}{s. IX}
	\codex{D\tsub{5}}{Codex Monacensis 6355}{s. IX/X}
	\codex{D\tsub{6}}{Codex Monacensis 6242}{s. X}
	\codex{D\tsub{7}}{Codex Monacensis 5258}{s. X}
	\codex{D\tsub{8}}{Codex Lucensis 125}{s. X}
	\codex{}{}{}
	\codex{\textit{\greektext F}}{\textit{Collectio codicis Vaticani}}{}
	\codex{V\tsub{1}}{Codex Vaticanus Barberinianus 679 (ol. XIV 52)}{s. VIII}
	\codex{V\tsub{2}}{Codex Vaticanus lat. 1342}{s. IX/X}
	\codex{}{}{}	
	\codex{J}{Codex Oxoniensis e Museo 101 (Collectio cod. Iustelli)}{s. VII}
	\codex{P}{Codex Parisinus lat. 3858 C (Collectio cod. Parisini)}{s. XIII}
	\codex{R}{Codex Ambrosianus lat. S 33 sup. (Collectio Dionysiana Bobiensis)}{s. X}
	\codex{}{}{}
	\codex{\greektext a}{\greektext P\latintext DiJ\greektext F}{}
	\codex{\greektext b}{\greektext D\latintext ER}{}
\end{codices}
 
\subsubsection*{Liber I ad Constantium}
\begin{codices}
	\codex{B}{Codex Vaticanus Basilicanus S. Petri D. 182}{s. VI}
	\codex{C}{Codex Parisinus Nouv. acq. 1454}{s. X}
	\codex{J}{Codex Salisburgensis S. Petri a. XI. 2}{s. XI/XII}
	\codex{E}{Codex Bernensis 100}{s. XII}
	\codex{L}{Codex Zwettlensis 33}{s. XII}
	\codex{M}{Codex Monacensis 169 (Lib. H. Schedelii)}{s. XII}
	\codex{O}{Codex Monacensis 21528 (Weihenstef. 28)}{s. XII}
	\codex{W}{Codex Vindobonensis lat. 684}{s. XII}
	\codex{G}{Codex Burdigalensis 112 (B. Mariae Syluae Maioris)}{s. XII}
	\codex{T}{Codex Cantabrigiensis 345}{s. XII}
\end{codices}

\subsection*{Socrates (Socr.)}
\begin{codices}
 	\codex{h.\,e.}{Historia ecclesiastica}{}
\end{codices}

\begin{codices}
	\codex{M}{Codex Laurentianus Plut. LXX 7}{s. X}
	\codex{}{M\ts{1} erste Hand, M\corr\ Korrektor(en), M\ts{r} j�ngerer Korrektor}{}
	\codex{F}{Codex Laurentianus Plut. LXIX 5}{s. XI}
	\codex{A}{Codex Athous Xeropotamou 226 (2559)}{s. XIV}
	\codex{T}{Codex Marcianus gr. 917 (344)}{s. XIII ex.}
	\codex{Arm.}{Altarmenische �bersetzung}{s. VI/VII}
\end{codices}
 
\subsection*{Sozomenus (Soz.)}
\begin{codices}
 	\codex{h.\,e.}{Historia ecclesiastica}{}
\end{codices}

\begin{codices}
	\codex{B}{Codex Oxoniensis  Baroccianus 142}{s. XIV in.}
	\codex{C}{Codex Alexandrinus 60 (olim Cairensis 86)}{s. XIII}
	\codex{T}{Codex Marcianus gr. 917 (344)}{s. XIII ex.}
	\codex{Cod. Nic.}{Codex Nicet.}{}
	\codex{Cast.}{Castellanus}{}
\end{codices}
 
 
\subsection*{Theodoret (Thdt.)}
\begin{codices}
 	\codex{h.\,e.}{Historia ecclesiastica}{}
\end{codices}

\begin{codices}
	\codex{B}{Codex Oxoniensis  Auct. E IV 18 (misc. 61)}{s. X}
	\codex{A}{Codex Oxoniensis  Auct. E II 14 (misc. 42)}{s. XI}
	\codex{N}{Athous Vatopediou 211}{s. XIII}
	\codex{H}{Codex Parisinus gr. 1442}{s. XIII}
	\codex{G}{Codex Angelicus gr. 41}{s. XII/XIII}
	\codex{S}{Codex Scorialensis gr. \griech{X} III 14}{s. XII}
	\codex{L}{Codex Laurentianus Plut. X 18}{s. XI}
	\codex{F}{Codex Parisinus gr. 1433}{s. XI/XII}
	\codex{T}{Codex Marcianus gr. 344}{s.XIII}
	\codex{V}{Codex Vaticanus gr. 628}{s. XI}
	\codex{r}{NGS}{}
	\codex{s}{GS}{}
	\codex{z}{LF}{}
	\codex{Cass.}{Cassiodor}{}
\end{codices}

\subsection*{Codex Veronensis LX}
\begin{praefatio}
 F�r den Codex Veronensis LX wurde, sofern von den bisherigen Herausgebern nicht eindeutige Schreibfehler verbessert wurden, auf den Text des Codex selbst zur�ckgegriffen. Dabei wurden die vom heutigen Gebrauch abweichenden Schreibweisen z.\,T. beibehalten. Damit sich in Dok. \ref{sec:SerdicaWestBekenntnis}, f�r das viele Konjekturen vorliegen, der Leser jedoch selbst ein Bild machen kann, wurden hier die notwendigen und eindeutigen Korrekturen dennoch gekennzeichnet. In Dok. \ref{sec:BriefOssiusProtogenes}, \ref{sec:BriefAthAlexParem}, \ref{sec:BriefSerdikaMareotis} und \ref{sec:BriefAthMareotis} werden zus�tzlich die Editionen des Textes ber�cksichtigt.
 
\end{praefatio}

\begin{codices}
 	\codex{Cod.Ver.}{Codex Veronensis LX}{s. VIII}
\end{codices}

\subsection*{Konjektoren\protect\footnote{Aufl�sung der Kurztitel im Literaturverzeichnis ab S. \pageref{literatur}}}
\begin{konjektoren}
	\codex{Abramowski}{\cite{Abramowski:Arianerrede}}{}
	\codex{Badius}{bei \cite{Hil:coll}}{}
	\codex{Baronius}{bei \cite{Hil:coll}}{}
	\codex{Bauer}{Angela Bauer}{}
	\codex{Ballerini}{\cite{Ballerini}}{}
	\codex{de Bruyne}{\cite{DeBruyne}}{}
	\codex{Bidez}{\cite{Hansen:Soz}}{}
	\codex{Binius}{bei \cite{Hil:coll}}{}  
	\codex{Christo\-phor\-son}{bei \cite{Hansen:Thdt}}{}
	\codex{Commelin}{\cite{Commelin}}{}
	\codex{Cornarius}{\cite{Corn}}{}
	\codex{Coustant}{\cite{Coustant}}{}
	\codex{Crabbe}{bei \cite{Hil:coll}}{}
	\codex{Cochlaeus}{bei \cite{Hil:coll}}{}
	\codex{Coleti}{bei \cite{Hil:coll}}{}   
	\codex{Dindorf}{\cite{Din}}{}
	\codex{Ed. regia}{bei \cite{Hil:coll}}{}
	\codex{Engelbrecht}{bei \cite{Hil:coll}}{}
	\codex{Erasmus}{bei \cite{Hil:coll}}{}
	\codex{Erl.}{Arbeitsstelle Athanasius Werke, Erlangen}{}
	\codex{Faber}{\cite{Faber}}{}
	\codex{Farlati}{bei \cite{Hil:coll}}{}
	\codex{Feder}{\cite{Hil:coll}}{}
	\codex{Gelzer}{bei \cite{Hansen:Thdt}}{}
	\codex{Gillotius}{bei \cite{Hil:coll}}{}
	\codex{Gryson}{\cite{Gry}}{}
	\codex{Hansen}{f�r Socr.: \cite{Hansen:Socr}}{}
	\codex{Hansen}{f�r Soz.: \cite{Hansen:Soz}}{}
	\codex{Hardouin}{bei \cite{Hil:coll}}{}  
	\codex{Holl}{\cite{Epi}}{}
 	\codex{J�licher}{bei \cite{Hansen:Thdt}}{}
	\codex{Klostermann}{bei \cite{Epi}}{}
	\codex{Koetschau}{bei \cite{Hansen:Thdt}}{}
	\codex{Labbe}{bei \cite{Hil:coll}}{}
	\codex{LeQuien}{bei \cite{Hil:coll}}{} 
	\codex{Loofs}{\cite{Loofs:Glaubensbekenntnis}}{}
	\codex{Lypsius}{bei \cite{Hil:coll}}{}
	\codex{Maffei}{\cite{Maffei}}{}
	\codex{Mai}{\cite{Mai}}{}
	\codex{Mansi}{bei \cite{Hil:coll}}{}
	\codex{Montfaucon}{\cite{Mont}}{}
	\codex{Oberth�r}{bei \cite{Hil:coll}}{}  
	\codex{Opitz}{\cite{Opitz1935}}{}
	\codex{Opitz}{bei \cite{Turner}}{}
	\codex{Parmentier}{\cite{Hansen:Thdt}}{}
	\codex{Petavius}{\cite{Pet}}{}
	\codex{Primmer}{bei \cite{Hansen:Soz}}{}
	\codex{Rettberg}{\cite{Rett}}{}
	\codex{Scheidweiler}{bei \cite{Hansen:Thdt}}{}
	\codex{Scheidweiler}{\cite{Scheidweiler:Byz}}{}
	\codex{Schwartz}{bei \cite{Opitz1935}}{}
	\codex{Schwartz}{bei \cite{Hansen:Thdt}}{}
	\codex{Schwartz}{bei \cite{Hansen:Soz}}{}
	\codex{St�hlin}{\cite{Staehlin:Rez}}{}
	\codex{Stockhausen}{Annette von Stockhausen}{}
	\codex{Tillemont}{bei \cite{Hil:coll}}{}
	\codex{Ulrich}{\cite{Ulrich1994}}{}
	\codex{Tetz}{\cite{Tetz:Ante}; f�r Dok.\ref{sec:AntIII} \cite{Tetz:Kirchweih}}{}
	\codex{Thompson}{\cite{Thompson:Papal}}{}
	\codex{Valois}{bei \cite{Hansen:Thdt}}{}
	\codex{Wickham}{\cite{Wickham:Hilary}}{}
	\codex{Wilmart}{\cite{Wilmart:Constantium}}{}
	\codex{Wintjes}{Angelika Wintjes}{}
\end{konjektoren}

\chapter*{Gestaltung der Edition}
\addcontentsline{toc}{chapter}{Gestaltung der Edition}
\markboth{Gestaltung der Edition}{Gestaltung der Edition}
    Grunds�tzlich werden alle Varianten angegeben, soweit sie
    anhand der vorliegenden kritischen Textausgaben nachzuvollziehen
    sind (vgl. v.\,a. oben die Bemerkung zu Hil., coll.\,antiar.). Ist der Text bei mehreren Autoren �berliefert, so wird dann auf eine genauere Aufschl�sselung der handschriftlichen �berlieferung der einzelnen Autoren verzichtet, wenn sonst dadurch die Lesbarkeit des textkritische Apparates in zu hohem Ma�e beeintr�chtigt w�re. In diesem Fall werden auch irrelevante
    Abweichungen der Textzeugen voneinander wie Schreibvarianten und offensichtliche Schreibfehler nicht ber�cksichtigt. Dies betrifft in erster Linie die Dokumente \ref{sec:SerdicaRundbrief}--\ref{sec:NominaepiscSerdikaOst}.

Itazismen oder das Setzen bzw. Weglassen von \griech{n}-ephelkystikon und von \griech{s} wurden generell nicht ber�cksichtigt.
Wo m�glich, wurden die Varianten in einem negativen Apparat verzeichnet, nur an einigen Stellen wurde um der Klarheit willen ein positiver Apparat vorgezogen. 

    Bei Namen wurden grunds�tzlich folgende Varianten nicht
    ber�cksichtigt: b/v/f, b/p, g/c, t/d, ti/ci, c/ch, t/th, ph/f,
    ae/e, Verdopplungen von Buchstaben.

Bei der Einarbeitung von Neben�berlieferungen aus einer anderen Sprache wurden nicht ber�cksichtigt: 
Umstellungen im Satzbau, kleinere
    grammatikalische Abweichungen (verschiedene Tempora, verschiedene
    M�glichkeiten der Auf"|l�sung von Partizipien, z.\,T. Wechsel von
    Singular und Plural, Wechsel von Konjunktionalsatz und AcI) und
    verschiedene W�rter bei gleicher Semantik.

    Es wurden dagegen angegeben: Varianten, die f�r die Abh�ngigkeit der
    Texte relevant sind; semantisch abweichende W�rter; Erg�nzungen
    und Auslassungen; grammatikalische Abweichungen, die eine
    wirkliche Sinn�nderung bedeuten. Wenn bereits in der Haupt�berlieferung
    kleinere Varianten vorhanden sind, wurden oft zum
    Vergleich auch alle Neben�berlieferungen angegeben.
\chapter*{Bemerkungen zur Chronologie des arianischen Streits\\bis zum Tod des Arius}
\addcontentsline{toc}{chapter}{Bemerkungen zur Chronologie des arianischen Streits bis zum Tod des Arius} 
\markboth{Bemerkungen zur Chronologie des arianischen Streits bis zum Tod des Arius}{Bemerkungen zur Chronologie des arianischen Streits bis zum Tod des Arius}
\setcounter{footnote}{0}
\section*{Die Anf�nge des arianischen Streits}
\noindent Als Hans-Georg Opitz 1934--1935 zwei Faszikel mit ">Urkunden zum
arianischen Streit"< (Athanasius Werke III) ver�ffentlichte, mu�te er
mit der Sortierung der Dokumente auch �ber chronologische Fragen
entscheiden. Gl�cklicherweise begr�ndete er sein Verst�ndnis von der
Zeitfolge des arianischen Streits noch 1934 in einem
Aufsatz,\footnote{\cite{Opitz:Zeitfolge}.} sein Tod im Zweiten Weltkrieg verhinderte 
die Ver�ffentlichung einer Einleitung zu
seiner Edition der Werke des Athanasius. Diese
Urkundensammlung und der Aufsatz von Opitz bilden seither die
Grundlage f�r das Verst�ndnis der Chronologie des ">arianischen"<
Streits, und seine Ergebnisse sind in viele Handb�cher und
Gesamtdarstellungen �bernommen
worden.\footnote{\cite[93--103]{Lietzmann:Geschichte};
  \cite[357]{Grillmeier:Jesus}; \cite[53--58]{Thuemmel:Kirche};
  \cite[126--129]{Lorenz:Osten}; \cite[291.
  294--300]{Markschies:Entstehen};
  \cite[129--138]{Hanson:Search}; \cite[699--703]{Ritter:Arianismus};
  \cite[243--245]{Frank:Lehrbuch};
  \cite[6--19.190~f.]{Quasten:GoldenAge};
  \cite[202--217]{Barnes:Constantine}; \cite[144~f.147]{Ritter:Dogma};
  nur pauschal ohne chronologische Details \cite[26 Anm.
  1.]{Simonetti:crisi}}

Einerseits legt Opitz seiner Rekonstruktion die beiden Berichte �ber
den Streitverlauf bei Sozomenus (h.\,e. I 15\footnote{Sozomenus ist
  hier ausf�hrlicher als Socrates, h.\,e. I 5~f.}) und Epiphanius
(haer. 69,3--5) zugrunde, andererseits f�gt er folgende Ereignisse in
diese Berichte ein: erstens die von Eduard Schwartz identifizierte
antiochenische Synode mit der dazugeh�rigen theologischen Erkl�rung
(Dok. \ref{ch:18}~=
Urk. 18)\footnote{\cite[134--155]{Schwartz:Dokumente}.} und zweitens
das Synodenverbot des Licinius, welches er in die Zeitspanne zwischen
322 bis Herbst 324 n.\,Chr. (Sieg des Konstantin �ber
Licinius\footnote{Darauf da� der Sieg �ber Licinius nicht 323
  n.\,Chr., sondern 324 n.\,Chr. zu datieren ist, wies schon Otto
  Seeck (\cite[433 Anm. 3]{Seeck:Urkundenfaelschungen} und
  \cite[493--501]{Seeck:Rheinisch}) hin, gegen
  \cite[540]{Schwartz:Konstantin}, und
  \cite[232]{Mommsen:Chronica}. Vgl. die l�ngere Diskussion dar�ber
  zwischen Seeck und Mommsen: \cite{Mommsen1897}; \cite{Seeck1901};
  \cite{Mommsen1901}; \cite{Seeck1902}; \cite{Mommsen1902}. Auf Seeck
  berief sich auch \cite[132]{Opitz:Zeitfolge}, der nochmals
  ausf�hrlich die Belege f�r den Sieg �ber Licinius im Jahr 324
  n.\,Chr. angibt (\cite[133--142]{Opitz:Zeitfolge}).})  ansetzte. Da
damit die Zeitspanne zwischen Herbst 324 n.\,Chr. und der nicaenischen
Synode im Fr�hsommer 325 n.\,Chr. zu kurz geriet, um alle
�berlieferten Ereignisse und Dokumente unterzubringen, datierte er
Schritt f�r Schritt r�ckw�rts rechnend das nach seinem Verst�ndnis
erste Dokumente, einen Brief des Arius an Eusebius von Nikomedien, in
das Jahr 318. Dies pa�te nun andererseits zu der Tatsache, da� jener
Eusebius zwischen 315 und 317 n.\,Chr. Bischof von Nikomedien geworden
sein mu�. Demzufolge ergab sich f�r Opitz folgender Streitverlauf:

Die\looseness=-1\ Auseinandersetzungen begannen in Alexandrien zwischen dem
Presbyter Arius und dem dortigen Bischof Alexander, und nach einigem
Z�gern exkommunizierte Alexander Arius und seine Anh�nger auf einer
Synode der ">100"< Bisch�fe (wie in Dok. \ref{ch:1}~= Urk. 1 und
Dok. \ref{sec:4b}~= Urk. 4b berichtet). Arius gewann trotzdem weitere
Anh�nger (z.B. eine Gruppe um einen Presbyter namens Pistus), reiste
selbst nach Palaestina und schrieb von dort an Eusebius von Nikomedien
mit der Bitte um Unterst�tzung (Dok. \ref{ch:1}~= Urk. 1, 318
n.\,Chr.; Dok. \ref{ch:2}~= Urk. 2 ist evtl. die entsprechende
Antwort). Darauf reagierte Alexander wiederum mit einem Rundbrief, in
dem er vor den Umtrieben der ">Arianer"< warnte (Dok. \ref{sec:4b}~=
Urk. 4b); auch an �gyptische Kleriker wandte er sich nochmals
(Dok. \ref{sec:4a}~= Urk. 4a). Eusebius von Nikomedien wiederum
unterst�tzte tats�chlich Arius; eine bithynische Synode best�tigte die
Rechtgl�ubigkeit des Arius (Dok. \ref{ch:5}~= Urk. 5). Von dort aus
schrieb Arius an Alexander (Dok. \ref{ch:6}~= Urk. 6), um einen
theologischen Konsens zu erreichen. Als dies nicht gelang, wandte sich
Arius wieder nach Palaestina (Dok. \ref{ch:7}~= Urk. 7;
Dok. \ref{ch:8}~= Urk. 8; Dok. \ref{ch:9}~= Urk. 9 geh�ren in diesen
Zusammenhang), und eine dortige Synode (Dok. \ref{ch:10}~= Urk. 10)
forderte Alexander auf, ihn wieder einzusetzen (auch
Dok. \ref{ch:11}~= Urk. 11; Dok. \ref{ch:12}~= Urk. 12;
Dok. \ref{ch:13}~= Urk. 13). Das nun erlassene Synodenverbot
ausnutzend kehrte Arius nach Alexandrien zur�ck und bildete eine
Sondergemeinde (wie in Dok. \ref{ch:14}~= Urk. 14
berichtet). Alexander reagierte mit weiteren Briefen
(Dok. \ref{ch:14}~= Urk. 14; Dok. \ref{ch:15}~= Urk. 15 und
Dok. \ref{ch:16}~= Urk. 16), die Situation aber blieb verworren. Nun
griff Konstantin nach seinem Sieg �ber Licinius in den Streitverlauf
ein (Dok. \ref{ch:17}~= Urk. 17) und schickte seinen theologischen
Berater Ossius von Cordoba nach Alexandrien. Eine alexandrinische
Synode konnte den Streit nicht l�sen, eine antiochenische
(Dok. \ref{ch:18}~= Urk. 18; 325 n.\,Chr.) traf eine Vorentscheidung;
die nicaenische Synode schlie�lich exkommunizierte Arius und seine
engsten Getreuen.

\paragraph{Ein R�ckblick auf Diskussionen vor Opitz} Opitz kam durch seine
Anordnung des Urkundenmaterials vor und nach dem Synodenverbot des
Licinius zu der Schlu�folgerung, da� der Streit nicht erst nach dem
Sieg des Konstantin �ber Licinius ungef�hr 18 Monate vor der Synode
von Nicaea begonnen haben konnte, sondern zu diesem Zeitpunkt schon
mehrere Jahre dauern mu�te (die sogenannte lange Chronologie). Nur so lie�en
sich die vielen Dokumente und die Reisen der ">Arianer"< 
einordnen. Damit best�tigte Opitz die alte Ansicht von
Tillemont\footnote{\cite[737--740]{Tillemont:Memoires}. Ebenfalls eine
  lange Chronologie vertraten schon \cite[32--38]{Gwatkin:Studies};
  \cite[5--15]{Gwatkin:Controversy}; \cite{Seeck:Untersuchungen}; und
  \cite{Snellman:Anfang}; \cite[37~f]{Urbina:Histoire}.  Auch Gwatkin
  l��t den Streit 318 n.\,Chr. beginnen, obwohl f�r ihn die
  antichristlichen Ma�nahmen des Licinius unerheblich waren.}  aus dem
ausgehenden 17. Jahrhundert und grenzte sich in dieser Frage auch von Eduard
Schwartz ab, der sich f�r die sogenannte k�rzere Chronologie
aussprach und alle Ereignisse erst nach dem Ende des Licinischen
Synodenverbots 323 n.\,Chr. datierte,\footnote{\cite[165--168]{Schwartz:Dokumente}.} die sich so in einen Zeitraum von nur wenigen Monaten bis zur Synode von Nicaea ereignet haben m�ssen. Basis f�r diese
k�rzere Chronologie sind die alte Datierung des Sieges Konstantins
�ber Licinius (nach \cite[232]{Mommsen:Chronica}) in das Jahr 323
n.\,Chr. und die Entscheidung, den Bericht des Eusebius von Caesarea,
v.\,C. II 61--73, zugrundezulegen, der den arianischen Streit als
�berraschend und sich schnell ausbreitend schildert.

Eine kurze Chronologie von nur einigen Monaten nahm auch Pierre
Batiffol\footcite{Batiffol:paix} an, der hierin Schwartz zustimmte,
ansonsten aber eher dem Bericht bei Sozomenus (h.\,e. I 15) folgte, da
diesem eine glaubw�rdige Quelle (Sabinus)
zugrundeliege.\footnote{\cite{Batiffol:Sozomene};
  s.\,u. Anm. \ref{fn:27}.} Auch Gustave
Bardy\footcite{Bardy:Histoire} folgt in der Datierung des
Streitanfangs auf 323 n.\,Chr. E. Schwartz,\footcite[71
Anm. 2]{Bardy:Histoire} nicht Opitz, obwohl er wie Opitz den Sieg des
Konstantin �ber Licinius in das Jahr 324
n.\,Chr. datiert\footnote{\cite[58]{Bardy:Histoire}. Auch A. Robertson
  geht in der Einleitung zu seiner �bersetzung der Werke des
  Athanasius von einer k�rzeren Chronologie aus, vgl. \cite[XVI Anm.
  1]{Robertson:Athanasius}: 321 verurteilte eine �gyptische Synode
  Arius; Arius verl��t Alexandrien nach Palaestina und Nikomedien;
  eine bithynische Synode und Eusebius von Nikomedien sprechen sich
  f�r Arius aus, Alexander verschickt Dok. \ref{sec:4b}~= Urk. 4b;
  Schisma des Colluthus, Brief des Alexander an seinen Namensvetter
  (Dok. \ref{ch:14}~= Urk. 14); Eingreifen des Konstantin.}~-- das
Licinische Synodenverbot spielt f�r Bardy keine Rolle.

Abgesehen von diesen �u�eren Daten des Streitanfangs wurde vor Opitz
auch schon �ber die innere Reihenfolge der vorhandenen Dokumente
diskutiert, insbesondere �ber die Anordnung der Briefe des
alexandrinischen Bischofs Alexander (bei Opitz nun Dok. \ref{sec:4a}~=
Urk. 4a, Dok. \ref{sec:4b}~= Urk. 4b, Dok. \ref{ch:14}~= Urk. 14,
Dok. \ref{ch:15}~= Urk. 15). Im Mittelpunkt dieser �lteren Diskussion
um die Briefe des Alexander standen die jeweils aufgelisteten
verurteilten Arianer, deren Bezeichnung als Presbyter und Diakone
nicht �bereinstimmt. Schon Valesius folgerte daraus, da� die Diakone
aus Dok. \ref{ch:14}~= Urk. 14 nun in den Presbyterrang aufgestiegen
seien (Dok. \ref{sec:4b}~= Urk. 4b) und da� folglich
Dok. \ref{ch:14}~= Urk. 14 vor Dok. \ref{sec:4b}~= Urk. 4b anzusetzen
sei.\footnote{\cite[1529~f]{Valesius:Adnotationes}.} Dieser These
widersprach Eduard Schwartz, der eine Verderbung der
Namensliste annahm.\footnote{\cite[165]{Schwartz:Dokumente}; auf
  S. 162--165 erkl�rt Schwartz ausf�hrlich die diversen Listen
  bzw. Z�hlungen der Arianer: Epiph., haer. 69,3; Soz., h.\,e.  I
  15,7; Dok. \ref{sec:4a}~= Urk. 4a; Dok. \ref{sec:4b}~= Urk. 4b;
  Dok. \ref{ch:6}~= Urk. 6; Dok. \ref{ch:14}~= Urk. 14;
  Dok. \ref{ch:16}~= Urk. 16.}  Sigismund
Rogala\footnote{\cite{Rogala:Anfaenge}. Die Arbeit von Rogala ist ganz
  und gar eine Auseinandersetzung mit Thesen von O. Seeck (s.\,u.).}
wiederum setzte Dok. \ref{ch:14}~= Urk. 14 wieder vor
Dok. \ref{sec:4b}~= Urk. 4b an, da erstens am Anfang der l�ngere,
ausf�hrlichere Brief (Dok. \ref{ch:14}~= Urk. 14) gestanden habe,
zweitens die Situation in Dok. \ref{sec:4b}~= Urk. 4b weiter
fortgeschritten erscheine und drittens Dok. \ref{ch:14}~= Urk. 14
urspr�nglich noch kein Ketzerverzeichnis habe. Ferner bezweifelte
Rogala die Identit�t des in Dok. \ref{ch:14}~= Urk. 14 erw�hnten
schismatischen Colluthus mit dem Colluthus, der Dok. \ref{sec:4b}~=
Urk. 4b unterschreibt. Auf die Thesen von Rogala reagierte wiederum
Viktor Hugger\footnote{\cite{Hugger:Briefe}.}, der Dok. \ref{sec:4b}~=
Urk. 4b eindeutig als ein Dokument der Fr�hzeit vor Dok. \ref{ch:14}~=
Urk. 14 ansah: die Arianer seien erst k�rzlich aufgetaucht, Alexander
habe sie zun�chst schweigend �bergehen wollen, nur die Einmischung des
Eusebius von Nikomedien habe eine Verurteilung erzwungen, der Brief
warne auch nur vor Agitationen der Arianer au�erhalb �gyptens. Hugger
wiederholte damit Argumente, die schon Paavo
Snellman\footcite[71--75]{Snellman:Anfang} formuliert hatte. Auch Otto
Seeck\footcite[432--433]{Seeck:Urkundenfaelschungen} analysierte die
Namenslisten der erw�hnten H�retiker und nahm diese sogar zum Anla�,
eine zeitweilige R�ckkehr von verurteilten Arianern in die
alexandrinische Kirche zu postulieren (auch er setzt
Dok. \ref{ch:14}~= Urk. 14 vor Dok. \ref{sec:4b}~= Urk. 4b an), da
diese im Rang aufgestiegen seien, was nur unter Alexander m�glich
gewesen w�re. Seeck baute diese Ansicht ferner zu der These aus, da�
Alexander unter dem Druck des Licinius die Arianer wieder in die
Kirchengemeinschaft aufgenommen
habe,\footnote{\label{fn:27}\cite[13--19.340--344]{Seeck:Untersuchungen}. Seine
  in diesem Aufsatz ge�u�erte Vermutung, da� Licinius zu einer Synode
  nach Nicaea geladen habe (S. 340~f.), nahm er in seinem sp�teren
  Beitrag wieder zur�ck (S. 433). Auch seine These, da� die
  Informationen, die Sozomenus in seinem Bericht �ber die Anf�nge des
  arianischen Streits liefert, aus einem Bericht des Ossius an
  Konstantin nach dessen Besuch in Alexandrien (S. 325--327) stammen,
  wurde bald von Batiffol widerlegt (s.\,o.), der Sabinus als Quelle
  nachwies.  Vielleicht mu� man an dieser Stelle aber gar nicht
  Sabinus bem�hen, da Sozomenus auch die Text- bzw. Briefsammlungen
  gekannt haben k�nnte, die im Verlauf der Anf�nge dieses Streits
  zusammengestellt wurden (vgl. Socr., h.\,e. I 6,41). Seecks Ansicht,
  Dok. \ref{sec:4a}~= Urk. 4a, die sog.  Depositio Arii, sei eine
  F�lschung des Athanasius (S. 50~f.), erfuhr ebenfalls keine
  Zustimmung: \cite[28--37]{Rogala:Anfaenge}.} was aber schon bald von
Snellman\footcite[98--115]{Snellman:Anfang} und
Rogala\footcite[37--74]{Rogala:Anfaenge} ausf�hrlich widerlegt
wurde. Gerhard Loeschcke\footcite{Loeschcke:Chronologie} nun ging
ebenfalls wie schon Schwartz von einer Verderbung der H�retiker-Listen
aus und beschrieb ferner, wie die Verschreibungen entstanden sein
k�nnten: eine senkrechte Namensliste sei f�lschlicherweise waagerecht
gelesen worden. Sein Vorschlag wurde schlie�lich von Opitz f�r die
Textrekonstruktion herangezogen (\editioncite[29 Apparat]{Opitz:Urk}), so da� Opitz ohne
gro�e Diskussion wieder Dok. \ref{sec:4b}~= Urk. 4b vor
Dok. \ref{ch:14}~= Urk. 14 stellt, wie es aus seiner Nummerierung ja
auch ersichtlich ist.

Gustave Bardy votierte damals aus anderen Gr�nden f�r die Priorit�t
von Dok. \ref{ch:14}~= Urk. 14 vor Dok. \ref{sec:4b}~= Urk.  4b:
Alexander habe n�mlich die Thesen des Arius in Dok. \ref{sec:4b}~=
Urk. 4b unter R�ckgriff auf dessen Schrift Thalia formuliert, die ihm
offenbar in Dok. \ref{ch:14}~= Urk. 14 noch unbekannt
war.\footnote{\cite{Bardy:Alexandre}; vgl. auch
  \cite{Bardy:Lucien}.} Bardy versuchte damit aus einer neuen
Perspektive heraus eine Antwort auf die schon �ltere Frage, welche
Reihenfolge den Briefen des Alexander zuzuweisen sei, zu geben. Opitz
selbst lehnte diese Bez�ge zur Thalia des Arius ab,\footcite[150
Anm. 88]{Opitz:Zeitfolge} da die Thalia nicht Grundlage f�r das
Referat in Dok. \ref{sec:4b}~= Urk. 4b, sondern umgekehrt das Referat
in Dok. \ref{sec:4b}~= Urk. 4b selbst die Grundlage f�r die sp�teren
Referate bzw. Zusammenfassungen arianischer Thesen bei Athanasius
sei. Sp�ter aber wurde diese Frage wieder diskutiert, besonders R.
Williams stellte f�r seine chronologische Sortierung die alte These
von Bardy wieder in den Mittelpunkt (s.\,u.). Auch sp�ter blieb Bardy
bei seiner Umstellung der
Alexanderbriefe,\footnote{\label{fn:Bardy}\cite[76
  Anm. 3]{Bardy:Histoire}.} ohne aber auf Opitz n�her einzugehen, so
da� sich f�r ihn folgender Streitverlauf ergab: Nach mehreren
Gespr�chen in Alexandrien lehnte Arius es ab, sich Alexander zu
beugen, und wurde daher auf der Synode der Hundert verurteilt; Arius
reiste daraufhin nach Palaestina, sp�ter von dort nach Nikomedien, wo
er sowohl sein Bekenntnis (Dok. \ref{ch:6}~= Urk. 6) als auch die
Thalia verfa�te; Alexander reagierte nun mit zwei Schreiben: erst mit
Dok. \ref{ch:14}~= Urk. 14 und, nachdem er die Thalia gelesen hatte,
mit Dok. \ref{sec:4b}~= Urk. 4b; Streitschriften wurden gewechselt und
gesammelt, und nach Synoden in Bithynia und Palaestina kehrte Arius
nach Alexandrien zur�ck. F�r Bardy folgen also beide Briefe Alexanders
ohne gro�en Abstand hintereinander.

\paragraph{Reaktionen auf die Thesen von Opitz} Kritisiert wurde der Vorschlag
von Opitz bald von W.
Telfer\footnote{\cite{Telfer:ArianControversy}. Vgl. schon
  \cite{Telfer:Arius}.}, der ebenfalls von einer wesentlich k�rzeren
Chronologie ausging. Erstens stellte Telfer infrage, ob tats�chlich
Reisen der Arianer stattgefunden haben~-- Epiphanius habe dies nur aus
Dok. \ref{sec:4b}~= Urk. 4b
herausgelesen\footcite[131]{Telfer:Arius}~-- und verk�rzte zweitens
die Zeitspanne des Synodenverbots des Licinius von 322--324 auf April
bis September 324, so da� f�r ihn der Streit erst knapp zwei Jahre vor
Nicaea begonnen habe. Der Bericht Soz., h.\,e. I 15 sei ferner nur mit
Vorsicht zu verwenden, da diesem die tendenzi�se Darstellung der
Synodalsammlung des Sabinus zugrundeliege, in welcher wohl die
andauernde Starrsinnigkeit des Alexander gegen wiederholte
Auf"|forderungen der Eusebianer, Arius wieder aufzunehmen, betont
worden sei. So seien die Ereignisse in Wirklichkeit sicher viel
schneller und zum Teil auch gleichzeitig geschehen, anders, als es
Sozomenus darstelle. Aus Dok. \ref{ch:7}~= Urk. 7, einem Brief des
Eusebius von Caesarea an Alexander, besonders aus dem Satz
\griech{<'ora e>i m`h e>ujuc p'alin a>uto~ic >aform`h d'idotai e>ic
  t`o >epilab'esjai ka`i diab'allein <orm~asjai <'osa ka`i j'elousi}
(15,1~f.), schlo� Telfer, da� zuvor ein theologischer Konsens erreicht
worden sei, und zwar anscheinend auf der Basis von Dok. \ref{ch:6}~=
Urk. 6, einer theologischen Erkl�rung des Arius.  Diesen Konsens, der
mit Eusebius auf einer pal�stinischen Synode (Dok. \ref{ch:10}~=
Urk. 10) gefunden worden sei, drohe Alexander aber mit seinem Brief
Dok. \ref{sec:4b}~= Urk. 4b zu zerst�ren, da er Arius darin falsche
Aussagen unterstelle, was ihm Eusebius nun mit Dok. \ref{ch:7}~=
Urk. 7 vorwerfe. Offensichtlich habe aber die bithynische Synode unter
Eusebius von Nikomedien, die anders als in Caesarea nicht vermittelnd,
sondern einseitig pro Arius entschieden habe, Alexander dazu
angestiftet, den Konsens aufzubrechen und auf einer eigenen
alexandrinischen Synode Arius und seine Anh�nger zu verurteilen. Diese
Synode setzte Telfer vor dem von ihm auf April bis September 324
datierten Synodenverbot an, also im M�rz 324, demnach m�sse Alexander
etwa Februar 324 von den Aktivit�ten des Eusebius von Nikomedien
erfahren haben. Entsprechend sei die bithynische Synode
Oktober/November 323 anzusetzen. So reiche es, wenn der Streit Juli
323 anfange: Am Beginn standen Disputationen in Alexandrien, die
zun�chst zu einem Kirchenausschlu� des Arius und seiner Anh�nger im
September 323 f�hrten, die dann zwar Ende des Jahres auf Vermittlung
Caesareas hin aufgrund eines Kompromisses wiedereingegliedert wurden,
aber schlie�lich erneut von Alexander auf einer �gyptischen Synode
(M�rz 324) exkommuniziert wurden, nachdem Alexander davon erfahren
hatte, da� Eusebius von Nikomedien sich f�r Arius eingesetzt hatte. In
einem sp�teren Aufsatz\footcite{Telfer:Sozomen} verteidigte Telfer
seine grunds�tzlichen Annahmen, passte aber die Darstellung des
Streitverlaufes dem Bericht bei Sozomenus (h.\,e. I 15) an, da diesem
doch eine glaubw�rdige Quelle, n�mlich Sabinus von Heraclea (s.\.o.),
zugrundeliege. So verlegte er jetzt seinen vermuteten Konsens
(zusammen mit Dok. \ref{ch:2}~= Urk. 2, Dok. \ref{ch:3}~= Urk. 3,
Dok. \ref{ch:11}~= Dok. 11, Dok. \ref{ch:12}~= Urk. 12 und
Dok. \ref{ch:13}~= Urk. 13, die in diesen Zusammenhang geh�rten) in
eine fr�here Zeit w�hrend der anf�nglichen Debatten um Arius in
Alexandrien vor dessen Ausschlu� auf einer �gyptischen Synode
(Dok. \ref{sec:4a} und \ref{sec:4b}~= Urk. 4a/b). Dok. \ref{ch:6}~=
Urk. 6 sei ebenfalls ein Dokument dieser Fr�hzeit. Nach der
alexandrinischen Synode folge Dok. \ref{ch:1}~= Urk. 1 und
Dok. \ref{ch:7}~= Urk. 7, anschlie�end Dok. \ref{ch:5}~= Urk. 5, dann
Dok. \ref{ch:8}~= Urk. 8, Dok. \ref{ch:9}~= Urk. 9 und
Dok. \ref{ch:10}~= Urk. 10 (die bithynische und palaestinische Synode
folgen also jetzt wieder der alexandrinischen Synode); schlie�lich
Dok. \ref{ch:14}~= Urk. 14.

Auf die Argumente von Telfer reagierten wiederum Norman
H. Baynes\footcite{Baynes:Sozomen} und Wilhelm
Schneemelcher\footcite{Schneemelcher:Chronologie}, die beide gegen
Telfer den Vorschlag von Opitz verteidigten. Baynes wies auf CTh XVI
2,5 als Beleg daf�r, da� auch Dezember 323 das Synodenverbot gegolten
habe, und betonte die Bedeutsamkeit dieser Ma�nahme des Licinius, die
Telfer zu Unrecht herabmindere. Zus�tzlich bem�ngelte Baynes, da� nach
der Darstellung des Sozomenus/Sabinus die Person des Alexander
keineswegs besonders starrsinnig sei, Alexander reagiere im Gegenteil
nur verz�gert und wolle gerade eine Verurteilung des Arius zun�chst
vermeiden, schreibe daher den Rundbrief auch erst nach der
Intervention des Eusebius von Nikomedien. Insbesondere die
Interpretation des einen Satzes aus Dok. \ref{ch:7}~= Urk. 7 bei
Telfer kritisierte Baynes, weil aus ihm keineswegs ein zuvor gefundener
theologischer Konsens herausgelesen werden d�rfe. Eusebius kritisiere
in diesem Brief allein, da� Alexander Aussagen des Arius (!) falsch
wiedergebe. Schneemelcher wies darauf hin, da� Telfer weder die
�berraschung des Kaisers �ber den Streit (in Dok. \ref{ch:17}~=
Urk. 17) noch die von Eusebius von Caesarea berichtete Pl�tzlichkeit
(v.\,C. II 61~f.) als Argumente f�r eine kurze Chronologie verwenden
k�nne, da beides rhetorischen Charakter trage.  Sowohl Baynes als auch
Schneemelcher bem�ngeln, da� Telfer den Beginn des arianischen Streits
sp�ter ansetzen m�chte, um noch vorher das apologetische Doppelwerk
des Athanasius (gent.; inc.) anzusetzen, in dem bekanntlich Hinweise
auf diesen Streit fehlen. Nun sei aber die Datierung dieses
Doppelwerkes umstritten und werde von vielen sogar weitaus sp�ter in
die Zeit des ersten Exils des Athanasius in Trier angesetzt (335--337
n.\,Chr.), so da� dieses Werk f�r ein chronologisches Verst�ndnis der
Anf�nge des arianischen Streits unbrauchbar sei.

Wieder anders als Telfer, aber auch in Kritik an Opitz, rekonstruiert
�phrem Boularand die Anf�nge des arianischen
Streits.\footcite[21--37]{Boularand:Heresie} Er stimmte Bardy zu, die
Reihenfolge der beiden Briefe des Alexander umzukehren, datierte wie
Opitz das Ende des Synodenverbots 324 n.\,Chr., setzte aber anders als
Telfer die Reiset�tigkeit des Arius und seiner Anh�nger voraus, auch
wenn er die Zeitfolge im Unterschied zu Opitz etwas straffte, so da�
f�r ihn der Streit 322 n.\,Chr. in Alexandrien begann.

Ohne auf die chronologischen Untersuchungen von Opitz einzugehen,
stellte Thomas A. Kopecek die Anf�nge des arianischen Streits
dar;\footcite[3--48]{Kopecek:History} er richtete sich aber in seiner
Darstellung offensichtlich nach Boularand \footcite[4
Anm. 2]{Kopecek:History} und G. Bardy \footcite[10
Anm. 1]{Kopecek:History}. So begann f�r Kopecek wie f�r Boularand der
Streit erst 322 n.\,Chr., und nach zwei Anh�rungen in Alexandrien
exkommunizierte Alexander Arius und seine Anh�nger. Da Widerstand
blieb, kam es zur Synode der ">Hundert"< in Alexandrien und
schlie�lich zur Vertreibung des Arius aus der Stadt. Arius reiste nach
Syria (Dok. \ref{ch:3}~= Urk. 3); aufgrund dieser Ausweitung wandte
sich auch Alexander an Kollegen au�erhalb Alexandriens
(Dok. \ref{ch:14}~= Urk. 14); nun wiederum richtete sich Arius an
Eusebius von Nikomedien (Dok. \ref{ch:1}~= Urk. 1 und
Dok. \ref{ch:2}~= Urk. 2), reiste dorthin und verfa�te die Thalia. Der
arianische Unterst�tzerkreis wuchs, daher ver�ffentlichte Alexander
die fr�here Verurteilung des Arius (Dok. \ref{sec:4b}~=
Urk. 4b). Jetzt berief auch Eusebius von Nikomedien eine Synode zugunsten des Arius ein (Dok. \ref{ch:5}~= Urk. 5), f�r die Arius eine theologische
Erkl�rung verfa�te, die er auch an Alexander verschickte
(Dok. \ref{ch:6}~= Urk. 6). Eine palaestinische Synode best�tigte
ebenfalls die Rechtgl�ubigkeit des Arius (Dok. \ref{ch:10}~= Urk. 10)
und Eusebius von Caesarea kritisierte Alexander in einem Brief
(Dok. \ref{ch:7}~= Urk. 7). Auf diese Weise mannigfaltig unterst�tzt
kehrte Arius nach Alexandrien zur�ck, Alexander aber verweigerte
weiterhin die Gemeinschaft mit ihm. Nun griff Konstantin in den Streit
ein (Dok. \ref{ch:17}~= Urk. 17; Dok. \ref{ch:18}~= Urk. 18 etc.). Die
Abwesenheit des Paulinus von Tyrus auf der antiochenischen Synode sei
der Grund f�r die Auf"|forderung des Eusebius von Nikomedien, nicht
l�nger zu schweigen, sondern sich f�r Arius einzusetzen
(Dok. \ref{ch:8}~= Urk. 8). Das chronologische Verst�ndnis
unterscheidet sich also nicht nur dadurch, da� die Zeitspanne viel
k�rzer gefa�t wird, sondern auch durch Umstellung der
Dok. \ref{sec:4b}~= Urk. 4b und Dok. \ref{ch:14}~= Urk. 14;
Dok. \ref{ch:1}~= Urk. 1; Dok. \ref{ch:2}~= Urk. 2 und
Dok. \ref{ch:3}~= Urk. 3; Dok. \ref{ch:8}~= Urk. 8.

1981 wurde auf einem Kolloquium in Berkeley �ber den arianischen
Streit am Rande auch �ber chronologische Fragen
diskutiert.\footcite{Kannengiesser:HolyScripture} Ch. Kannengie�er
akzeptierte hier die Opitzsche Reihenfolge der Dokumente, beobachtete
aber, da� Dok. \ref{ch:6}~= Urk. 6 das nach Opitz vorausgehende
Dok. \ref{sec:4b}~= Urk. 4b v�llig ignoriere, Dok. \ref{sec:4b}~=
Urk. 4b wiederum auch Dok. \ref{ch:1}~= Urk.
  1.\footcite[9]{Kannengiesser:HolyScripture} T. Kopecek unterst�tzte
  auf diesem Kolloquium diese Beobachtungen und verwies in seiner
  Response auf seinen eigenen chronologischen
  Vorschlag,\footcite[51]{Kopecek:Response} nach dem er zwar auch
  Dok. \ref{ch:1}~= Urk. 1 vor Dok. \ref{sec:4b}~= Urk. 4b und beide
  vor Dok. \ref{ch:6}~= Urk. 6 ansetzte, aber vor Dok. \ref{sec:4b}~=
  Urk. 4b die Thalia einf�gte, worauf sich Alexander in seinem Brief
  Dok. \ref{sec:4b}~= Urk. 4b bezogen habe. Auch G.\,C. Stead ging auf
  chronologische Fragen ein, besonders auf Dok. \ref{ch:1}~= Urk. 1,
  und wies darauf hin, da� dieses Dokument erstens davon ausgehe, da�
  Arius aus Alexandrien vertrieben worden sei, zweitens, da� Alexander
  schon mit Bisch�fen in Antiochia Kontakt habe und dort einige
  exkommuniziert worden seien, und da� drittens diese Ereignisse nur
  durch eine bereits vorausgehende Kontaktsuche des Arius dort
  verursacht worden sein k�nnten.\footcite[73]{Stead:Response} Ohne
  aber diese �berlegungen weiterzuverarbeiten, referierte Stead in
  einem sp�teren Aufsatz Dok. \ref{ch:1}~= Urk. 1 als erstes Werk des
  Arius, Dok. \ref{ch:6}~= Urk. 6 als
  zweites.\footcite[24]{Stead:Arius}

  In einem weiteren Aufsatz ging Stead n�her auf Dok. \ref{sec:4b}~=
  Urk. 4b ein und wies darin diesen Text aufgrund stilistischer Beobachtungen
  nicht Alexander, sondern seinem Diakon und Nachfolger Athanasius  zu.\footnote{\cite{Stead:Athanasius}. Dies ist eine schon
    �ltere Idee, wie Stead selbst berichtet, vorgeschlagen von:
    \cite[297]{Newman:HistoricalTracts}; \cite{Moehler:Athanasius};
    \cite[68]{Robertson:Athanasius}. Auch diese �lteren Autoren wiesen
    schon auf stilistische und argumentative Unterschiede hin, wie sie
    Stead jetzt im Detail ausf�hrlicher beschreibt.} Dennoch blieb
  Stead dabei, dieses Schreiben wie Opitz in die Anfangsphase des
  arianischen Streits einzuordnen.\footcite[91
  Anm. 23]{Stead:Athanasius} Ganz andere Konsequenzen zog
  L. Abramowski aus den Thesen von Stead; sie m�chte auch diesen Text
  ganz aus der Vorgeschichte von Nicaea streichen, da vor dem nicaenischen
  \griech{<omoo'usioc} niemand \griech{<'omoioc kat'' o>us'ian}
  formuliert haben d�rfte, und etwa zehn Jahre sp�ter datieren. Er
  sei n�mlich von Athanasius erst nach der R�ckkehr des Eusebius von
  Nikomedien aus dem westlichen Exil 328 verfa�t worden,\footcite[408
  Anm. 36]{Abramowski:Arianerrede} so da� erst jetzt die Bemerkung in
  Dok. \ref{sec:4b}~= Urk. 4b sinnvoll sei, Eusebius erneuere seine
  alte schlechte Gesinnung. Diese �berlegungen scheinen aber
  angesichts der Tatsache, da� sich Eusebius von Caesarea in
  Dok. \ref{ch:7}~= Urk. 7 direkt auf Dok. \ref{sec:4b}~= Urk. 4b
  bezieht, doch unwahrscheinlich zu sein.

  Einen neuen Vorschlag zur Chronologie der Anf�nge des arianischen
  Streits legte 1987 R.  Williams in seiner Monographie �ber Arius
  vor.\footcite{Williams:Arius} Er besprach (in ">Part I Arius and the
  Nicene Crisis, B Documents and Dating"<) ausf�hrlich Probleme der
  Opitzschen Chronologie und ordnete schlie�lich die meisten Dokumente
  ganz anders an.\footnote{Vgl. die tabellarische �bersicht
    \cite[58~f.]{Williams:Arius}} So stellte Williams die beiden
  Ariusbriefe Dok. \ref{ch:1}~= Urk. 1 und Dok. \ref{ch:6}~= Urk. 6
  um: Dok. \ref{ch:6}~= Urk. 6 geh�re an den Beginn der
  Auseinandersetzungen, sei die erste schriftliche Erkl�rung des
  Arius, verfa�t f�r die anf�ngliche Anh�rung in Alexandrien (ca. 320
  n.\,Chr.). Auch die Reihenfolge der beiden belegten au�er�gyptischen
  Synoden drehte er um, da Arius nach seiner alexandrinischen
  Verurteilung (321 n.\,Chr.) zuerst Unterst�tzung in Palaestina
  (Dok. \ref{ch:10}~= Urk. 10) gesucht und erst im Anschlu� daran
  Kontakt zu Eusebius von Nikomedien gekn�pft habe (Dok. \ref{ch:1}~=
  Urk. 1). Drittens sei der Brief des Alexander Dok. \ref{ch:14}~=
  Urk. 14 sein erster und als Reaktion auf diese palaestinische
  Unterst�tzung zu verstehen: Alexander suche nun auch selbst Anh�nger
  in seiner Sache. Nach der f�r Arius erfolgreichen Ausweitung der
  Streitsache (Dok. \ref{ch:3}~= Urk. 3; Dok. \ref{ch:1}~= Urk. 1 und
  Dok. \ref{ch:2}~= Urk. 2; Dok. \ref{ch:11}~= Urk. 11;
  Dok. \ref{ch:13}~= Urk. 13) berufe Alexander nun die sogenannte
  Synode der hundert Bisch�fe ein (Dok. \ref{ch:15}~= Urk. 15).

  Williams ging also f�r die Anf�nge des arianischen Streits von drei
  �gyptischen Synoden aus: eine erste alexandrinische 321 n.\,Chr.,
  eine �gyptische 323 n.\,Chr. und drittens die Synode unter Ossius
  von Cordoba 325 n.\,Chr. Die bithynische Synode (Dok. \ref{ch:5}~=
  Urk. 5) datierte Williams erst nach dem Synodenverbot des Licinius
  (324 n.\,Chr.). Jetzt greife auch Konstantin in den Streit ein und
  schicke Ossius nach Alexandrien. Der Rundbrief des Alexander
  Dok. \ref{sec:4b}~= Urk. 4b sei das Ergebnis eben dieser
  alexandrinischen Versammlung unter Ossius Winter 324/325
  n.\,Chr. Schlie�lich datierte Williams auch Dok. \ref{ch:7}~=
  Urk. 7, den Brief des Eusebius von Caesarea, sehr sp�t, n�mlich nach
  dieser letzten alexandrinischen und sogar noch nach der
  antiochenischen Synode Anfang 325 n.\,Chr. Er sei als Versuch in
  letzter Minute zu verstehen, die theologischen Positionen im Vorfeld
  der Synode von Nicaea zu kl�ren.

  In Bezug auf die �u�eren Daten unterscheidet sich Williams nicht
  erheblich von Opitz. Den ersten Brief des Konstantin an Arius und
  Alexander (Dok. \ref{ch:17}~= Urk. 17) datierte Williams etwas
  sp�ter als Opitz, und zwar um Weihnachten 324, also w�hrend des
  Aufenthalts des Konstantin in Nikomedien im Zusammenhang seiner
  Reise in den Osten (November 324 bis Februar 325), und berief sich
  hierf�r auf T.\,D. Barnes\footnote{\cite[54--56]{Barnes:Emperor};
    \cite[212--214]{Barnes:Constantine}.}. Die Zeitspanne des
  Synodenverbots verk�rzte Williams unter Berufung auf
  Boularand\footnote{\cite[24]{Boularand:Heresie}.} auf etwa 16
  Monate, da es erst Sommer 323 erlassen worden sei. Licinius habe es
  erlassen, um eine christliche ">5. Kolonne"< zu verhindern, nachdem
  Konstantin im Fr�hsommer die Grenzen verletzt hatte.\footcite[292
  Anm. 12]{Barnes:Constantine} Entsprechend reichte es f�r Williams
  aus, das erste Dokument 320 n.\,Chr. und nicht wie Opitz 318
  n.\,Chr. zu datieren.

  Die vielen �nderungen der Reihenfolge der Dokumente begr�ndete
  Williams mit Beobachtungen an den Texten selbst. Im Zentrum stehen
  die beiden Briefe des Alexander von Alexandrien (Dok. \ref{sec:4b}~=
  Urk. 4b und Dok. \ref{ch:14}~= Urk. 14). Dok. \ref{sec:4b}~= Urk. 4b
  wird von einem gewissen Presbyter Colluthus unterschrieben; in
  Dok. \ref{ch:14}~= Urk. 14 wird aber von seinem Schisma berichtet,
  und zwar habe sich Colluthus dar�ber beschwert, da� Alexander
  gegen�ber Arius zu tolerant auftrete. Dies sei aber nach der
  alexandrinischen Synode und der Verurteilung des Arius nicht
  erkl�rbar, so da� Dok. \ref{ch:14}~= Urk. 14 vor Dok. \ref{sec:4b}~=
  Urk. 4b anzusetzen sei. Ferner sei es merkw�rdig, da�
  Dok. \ref{ch:14}~= Urk. 14 nur von drei syrischen Bisch�fen
  berichte, die Arius unterst�tzten, und nicht von den Aktivit�ten des
  Eusebius von Nikomedien (Dok. \ref{sec:4b}~= Urk. 4b). Dar�berhinaus
  sei die Bemerkung in Dok. \ref{sec:4b}~= Urk. 4b, Eusebius von
  Nikomedien erneuere seine alte schlechte Gesinnung, auf die
  Unterbrechung der Aktivit�ten wegen des Synodenverbots zu
  beziehen. Auch passe die in Dok. \ref{sec:4b}~= Urk. 4b erw�hnte
  Friedenszeit besser in die Phase nach dem Sieg des Konstantin �ber
  Licinius und der Aufhebung des Synodenverbots.  Viertens seien die
  Parallelen zwischen Dok. \ref{sec:4b}~= Urk. 4b und der Thalia so
  offensichtlich, da� Urk.  4b ohne Zweifel die Thalia voraussetze~--
  dann sei es aber merkw�rdig, da� diese theologischen Inhalte in
  Dok. \ref{ch:14}~= Urk. 14 wieder vergessen seien.\footnote{Mit
    diesem Hinweis datierte schon G. Bardy
    (s.\,o. Anm. \ref{fn:Bardy}) Dok. \ref{sec:4b}~= Urk. 4b vor
    Dok. \ref{ch:14}~= Urk. 14.} Geh�rt nun Dok. \ref{ch:14}~=
  Urk. 14 vor Dok. \ref{sec:4b}~= Urk. 4b, dann habe Arius am
  Beginn der Auseinandersetzung nach der alexandrinischen Verurteilung
  zuerst versucht, Kontakt nach Syria/Palaestina aufzunehmen (daher
  Dok. \ref{ch:10}~= Urk. 10 vor Dok. \ref{ch:5}~= Urk. 5) und sich erst in
  zweiter Linie danach von Caesarea aus nach Nikomedien
  gewandt. Ferner k�nne es gut m�glich sein, da� Dok. \ref{sec:4b}~=
  Urk. 4b ein Dokument der Synode unter Ossius in Alexandrien Anfang
  325 ist, da nach Ath., apol.\,sec. 76 Colluthus auf einer Synode mit
  Ossius wieder in die Kirche eingegliedert worden sei, was seine
  Unterschrift an exponierter Stelle erkl�re.

  Eine erste kritische Reaktion auf diesen neuen Vorschlag von
  Williams findet sich bei Thomas
  B�hm.\footnote{\cite{Boehm:Christologie}.} Er referierte zwar
  zun�chst zustimmend Williams Chronologie, bezweifelte dann aber, da�
  Dok. \ref{sec:4b}~= Urk. 4b der alexandrinischen Synode unter Ossius
  zuzuweisen ist, und schlug eine andere Datierung f�r diesen
  Rundbrief vor, in zeitlicher N�he zur etwa 323 n.\,Chr. verfa�ten
  Thalia und vor Dok. \ref{ch:14}~= Urk. 14.

  Gegen diese neue Chronologie von R. Williams bezog auch Uta Heil
  schon 1990 Stellung,\footnote{\cite{Loose:Chronologie}; zustimmend
    \cite[114--116]{Boehm:Aspekte}; \cite[296]{Markschies:Entstehen};
    \cite[C126-128]{Lorenz:Osten}.} um damals die Reihung von Opitz zu
  verteidigen. Der Protest des Colluthus k�nne sich auch gegen die
  R�ckkehr der Arianer w�hrend des Synodenverbots gerichtet haben;
  Dok. \ref{sec:4b}~= Urk. 4b verliere ferner kein Wort �ber eine
  Beilegung des Schismas des Colluthus; die in Dok. \ref{ch:14}~=
  Urk. 14 erw�hnte Friedenszeit sei als rhetorisches Mittel nicht
  unbedingt ein Hinweis auf eine echte Friedenszeit, also auf die
  Fr�hzeit vor dem Synodenverbot; die erneuerte schlechte Gesinnung
  des Eusebius von Nikomedien in Dok. \ref{sec:4b}~= Urk.  4b beziehe
  sich auf dessen vorhergehenden Bischofswechsel und Karrierestreben,
  nicht auf die durch das Synodenverbot unterbrochenen Aktivit�ten des
  Eusebius f�r die Arianer. Dok. \ref{sec:4b}~= Urk. 4b passe
  insgesamt besser in die Fr�hzeit des arianischen Streits und
  Dok. \ref{ch:14}~= Urk. 14 setze einen bereits versandten Rundbrief
  voraus, auf den Alexander schon Antworten aus Syria, Pamphylia,
  Lycia, Asia und Cappadocia erhalten habe.

  Williams reagierte darauf wiederum in seiner zweiten Auf"|lage
  seiner Monographie �ber Arius, ohne aber seine Meinung zu
  �ndern.\footcite[252--255]{Williams:Arius} Hier verwies er noch auf
  folgende zus�tzliche Beobachtungen: Nach der Vorstellung von Opitz
  kehre Arius wieder nach Alexandrien zur�ck und verursache dadurch
  die Zust�nde, wie sie in Dok. \ref{ch:14}~= Urk. 14 geschildert
  werden, was Williams aber infrage stellt. Er meint, die Synode in
  Palaestina (Dok. \ref{ch:10}~= Urk. 10) k�nne nur erlaubt haben,
  Sondergemeinden in Palaestina, und nicht in Alexandrien zu
  bilden. Au�erdem sei es merkw�rdig, da� in dem Brief des Alexander
  von Alexandrien an Alexander von Byzantium (Dok. \ref{ch:14}~=
  Urk. 14) nichts von den Aktivit�ten seines nachbarlichen Kollegen
  Eusebius von Nikomedien zu lesen sei, so da� Dok. \ref{ch:14}~=
  Urk. 14 eben an den Anfang des Streits vor diesen Aktivit�ten
  geh�re. Ferner wiederholte Williams nochmals die Beobachtung, da�
  Dok \ref{sec:4b}~= Urk. 4b wohl auf die Thalia, die Arius auf Anregung des Eusebius
  von Nikomedien verfa�t habe, zur�ckgreife.

  Eine ausf�hrliche Analyse von Leben und Werk des Arius bietet nun
  Winrich L�hr in seinem l�ngeren, zweigeteilten
  Aufsatz,\footnote{\cite{Loehr2005}; \cite{Loehr2006}.} dessen erster
  Teil historische und chronologische und dessen zweiter Teil
  theologische und dogmengeschichtliche Probleme behandelt. Auch L�hr
  setzt sich sehr kritisch mit dem chronologischen Ger�st von Williams
  auseinander und legt seinen Schwerpunkt auf drei Hauptprobleme:
  Erstens die Reihung der beiden Briefe des Alexander
  (Dok. \ref{sec:4b}~= Urk. 4b und Dok. \ref{ch:14}~= Urk. 14),
  zweitens das relative Verh�ltnis der drei Dok. \ref{ch:1}~= Urk. 1,
  Dok. \ref{sec:4b}~= Urk. 4b und Dok. \ref{ch:6}~= Urk. 6 zueinander
  und schlie�lich die Relevanz der Thalia f�r diese Texte. 
Zum ersten argumentiert auch L�hr f�r die Priorit�t von Dok. \ref{sec:4b}~=
  Urk. 4b vor Dok. \ref{ch:14}~= Urk. 14, da letztere ein sp�teres
  Diskussionsstadium widerspiegele.\footcite[543~f]{Loehr2005} Das
  Schisma des Colluthus d�rfe nicht chronologisch verwertet werden, da
  es wahrscheinlich gar nicht in Beziehung zur arianischen Frage
  st�nde,\footcite[544~f]{Loehr2005} was meines Erachtens eine
  glaubw�rdige Interpretation zu sein scheint. Dok. \ref{ch:15}~=
  Urk. 15 d�rfe ferner nicht einer gesonderten Synode zugewiesen
  werden, sondern sei ein Rundschreiben des Alexander und stehe in
  Zusammenhang mit Dok. \ref{ch:14}~= Urk. 14. Es sei dar�berhinaus
  nicht �berraschend, wenn Alexander in seinem zweiten Brief
  Dok. \ref{ch:14}~= Urk. 14 sein Referat arianischer Thesen nicht
  wiederhole, nachdem Eusebius von Caesarea es (in Dok. \ref{ch:7}~=
  Urk. 7) als inkorrekt kritisiert habe.\footcite[548]{Loehr2005} F�r
  L�hr sieht, anders als f�r Williams, die Zusammenfassung der Lehre
  des Arius in Dok. \ref{ch:14}~= Urk. 14 k�rzer, einfacher und klarer
  strukturiert aus.\footcite[550]{Loehr2005} Zum zweiten beurteilt wie Stead
  auch L�hr Dok. \ref{ch:1}~= Urk. 1 als Dokument eines
  sp�teren Stadiums; auch er w�rde nicht diesen Text an den Beginn
  stellen,\footcite[553~f]{Loehr2005} sondern eher Dok. \ref{ch:6}~=
  Urk. 6. Auch L�hr dreht somit auch die Reihenfolge der beiden Briefe
  des Arius um. Das Verh�ltnis von Dok. \ref{sec:4b}~= Urk. 4b zu
  Dok. \ref{ch:6}~= Urk. 6 lie�e sich aber nicht genauer bestimmen, da
  Alexander in Dok. \ref{sec:4b}~= Urk. 4b sich nicht direkt auf die
  Erkl�rung der Arius, Dok. \ref{ch:6}~= Urk. 6, beziehe. Beide Texte
  seien aber ohne Zweifel Dokumente der fr�hen Zeit. Zum dritten d�rfe die
  Verbindung zwischen der Thalia und dem Referat arianischer Thesen in
  Dok. \ref{sec:4b}~= Urk. 4b nicht zu eng gezogen werden, da
  sich erstens Dok. \ref{sec:4b}~= Urk. 4b auch auf eine m�ndliche
  Debatte beziehe und zweitens Arius sich gewi� mehrmals zu diesem Thema
  ge�u�ert haben d�rfte.\footcite[556~f]{Loehr2005}

  Wie sind nun die vielen Beobachtungen, Vorschl�ge und Kritiken zu
  beurteilen? Sicher ist grunds�tzlich zu beachten, da� nur ein
  Ausschnitt der tats�chlichen Briefe und Glaubenserkl�rungen
  �berliefert ist. Daher mu� manches l�ckenhaft bleiben. Zweitens ist
  die Beobachtung von Telfer, da� einige �berlieferte Dokumente
  eventuell auch gleichzeitig anzusetzen seien, gewi� zutreffend, so
  da� die Ereigniskette, die die Liste der Dokumente suggeriert, nur
  mit Vorsicht zu betrachten ist. Drittens m�ssen manche Fragen und
  Probleme ungekl�rt bleiben. Zusammenfassend lassen sich folgende
  Probleme bei den Dokumenten benennen:

  In Hinblick auf die �u�eren Daten ist eine �bereinkunft �ber die
  Datierung des Endes des Synodenverbots bzw. des Sieges des
  Konstantin �ber Licinius erreicht, so da� heute niemand mehr die
  Dokumente des arianischen Streits zwischen Synodenverbot und der
  Synode von Nicaea ansetzt. Unklar ist jedoch noch immer, wie lange
  das Synodenverbot in Kraft war, wovon die Datierung des Anfangs des
  arianischen Streits abh�ngt. Wie weit mu� man zur�ckgehen? Bis 318
  n.\,Chr. wie Opitz, oder reichen 320 n.\,Chr. bzw. 322 n.\,Chr.? Zu
  ber�cksichtigen ist, da� aus der Anfangszeit der
  inneralexandrinischen Diskussionen keine Dokumente �berliefert sind,
  allein der Bericht bei Sozomenus legt nahe, da� ziemlich lange schon
  die Auseinandersetzungen geschwelt hatten, bis Alexander schlie�lich
  eine Entscheidung traf. So k�nnte das Datum 318 n.\,Chr. zutreffen,
  ohne da� wir aber aus dieser Zeit schon ein zu datierendes Dokument
  h�tten (s.\,u.).

  Wann und wohin reiste Arius? Oder reiste er gar nicht? Reiste er
  nach Palaestina, dann nach Syria, schlie�lich wieder nach Palaestina
  und kehrte letztlich nach Alexandrien zur�ck?  Snellman sah in der
  Namensliste aus Dok. \ref{ch:1}~= Urk. 1 sogar eine Reiseroute des
  Arius.\footcite[80~f]{Snellman:Anfang} Oder reiste er nur nach
  Palaestina und blieb dort?  Oder blieb er gar in Alexandrien, wie
  Telfer annimmt? Alexander berichtet in Dok. \ref{ch:14}~= Urk. 14,
  es best�nde die Gefahr, ">Arianer"< w�rden in andere St�dte
  eindringen, sei es pers�nlich oder durch Briefe (Dok. \ref{ch:14}~=
  Urk. 14,2.58). Auch Dok. \ref{sec:4b}~= Urk. 4b warnt davor, Arianer
  aufzunehmen (Dok. \ref{sec:4b}~= Urk. 4b,20) und Arius schreibt in
  Dok. \ref{ch:1}~= Urk. 1,2, Alexander verfolge und vertreibe ihn und
  seine Anh�nger aus der Stadt. Bezieht sich dieses ">Durchwandern der
  St�dte"< auf Arius oder nur auf die ">Brieftr�ger"<, die um
  unterst�tzende Briefe oder Unterschriften ersuchten? K�rzt man die
  Reiset�tigkeit des Arius, so m�ssen die Dokumente nicht parallel zu
  einer angenommenen Reiseroute des Arius angeordnet werden.

  Die Frage nach den Reisen des Arius betrifft unmittelbar das
  Verst�ndnis von Dok. \ref{ch:6}~= Urk. 6, der theologischen
  Erkl�rung des Arius. Viel h�ngt davon ab, wie man die einleitende
  Bemerkung bei Epiphanius, haer. 69,7, beurteilt, Arius habe
  Dok. \ref{ch:6}~= Urk. 6 in Nikomedien verfa�t und von dort an
  Alexander verschickt. Telfer kritisierte, es sei unwahrscheinlich,
  da� Arius zusammen mit seinen Anh�ngern, Presbytern, Diakonen und
  zwei libyschen Bisch�fen (so die Unterschriften unter
  Dok. \ref{ch:6}~= Urk. 6), durch Syria und Kleinasien
  reise. Williams gab zwar zu bedenken, da� jene Unterschriften auch
  aus Dok. \ref{sec:4b}~= Urk. 4b dort eingetragen worden sein
  k�nnten, wollte aber auch Dok. \ref{ch:6}~= Urk. 6 als fr�hes
  Dokument noch vor einer alexandrinischen Verurteilung verstehen, da
  jeder Hinweis auf eine Verurteilung oder Verbannung fehle. Auch
  Stead wies darauf hin, da� Dok. \ref{ch:6}~= Urk. 6 nichts von den
  Vorg�ngen, die Dok. \ref{sec:4b}~= Urk. 4b berichtet, zu kennen
  scheine. So legt der Text von Dok. \ref{ch:6}~= Urk. 6 nahe, der
  einleitenden Bemerkung bei Epiphanius nicht zu viel Gewicht
  beizumessen; Epiphanius kam zu seiner Vermutung eventuell aufgrund
  des zuvor zitierten Dok. \ref{ch:1}~= Urk. 1, dem Brief des Arius an
  Eusebius von Nikomedien. Die chronologische Beurteilung von
  Dok. \ref{ch:6}~= Urk. 6 beeinflu�t auch das theologische
  Verst�ndnis dieses Dokuments: Ist es ein gem��igtes, vermittelndes
  Schreiben des Arius an Alexander, um eine Verst�ndigung zu
  erreichen, oder gibt es kompromi�los Thesen des Arius wieder?

  Besonders wichtig und auch schwierig f�r die Frage nach den Reisen
  des Arius ist Dok. \ref{ch:10}~= Urk. 10, der Bericht �ber die
  Synode in Palaestina~-- wie ist diese Notiz zu beurteilen? Ohne
  Zweifel spricht sich die Synode f�r Arius aus, aber was gestattet
  sie?  ">Anh�nger-"<Gemeinden in Palaestina oder in Alexandrien?
  Werden tats�chlich eine Art von Exil-Gemeinden in Palaestina
  zugelassen, wie Williams vermutet (">�migr� congregations"<,
  \cite[51]{Williams:Arius}), wenn gleichzeitig empfohlen wird, sich
  Alexander unterzuordnen, oder ist das nur sinnvoll f�r
  alexandrinische Gemeinden? Ferner berichtet Sozomenus hier, da� die
  Synode Arius und seinen Anh�ngern zugestehe, ">wie fr�her"<, also
  wie vor dem Ausbruch des Streits und der Exkommunikation,
  Gottesdienste abzuhalten und ">wie fr�her"< als Presbyter zu gelten,
  auch wenn sie sich dabei Alexander unterordnen sollen. Damit k�nnen
  eigentlich nur alexandrinische Verh�ltnisse angesprochen
  sein.\footnote{So auch \cite[558~f.]{Loehr2005}} Ist diese
  Interpretation korrekt, dann beschwert sich Alexander in
  Dok. \ref{ch:14}~= Urk. 14 zu Recht �ber die unrechtm��ige
  Einmischung von au�en durch syrische Bisch�fe, was die Verh�ltnisse
  verschlimmert habe (Dok. \ref{ch:14}~= Urk. 14,37\footnote{Gerade
    diesen Bezug lehnt \cite[51]{Williams:Arius} ab.}).

  Zus�tzlich zu diesen Reisen bleibt die Chronologie der
  Dok. \ref{sec:4b}~= Urk. 4b und Dok. \ref{ch:14}~= Urk. 14 ein
  Problem. So berichten Dok. \ref{sec:4b}~= Urk. 4b und
  Dok. \ref{ch:14}~= Urk. 14 beide �ber Geschehnisse in Alexandrien,
  die sich nicht decken.  Das in Dok. \ref{sec:4b}~= Urk. 4b
  Berichtete (bes. die Einmischung des Eusebius von Nikomedien) fehlt
  in Dok. \ref{ch:14}~= Urk. 14; umgekehrt vermi�t man wiederum die
  Einflu�nahme des Eusebius von Caesarea und weiterer Syrer aus
  Dok. \ref{ch:14}~= Urk. 14 in Dok. \ref{sec:4b}~= Urk. 4b, wenn man
  die Reihenfolge umdreht. Schwierig zu beurteilen ist auch, warum die
  Referate �ber arianische Thesen in den beiden Dokumenten so
  unterschiedlich ausfallen. Es ist aber gewi� methodisch
  problematisch, aus der Art und Weise der Pr�sentation der Thesen der
  Arianer chronologische Schl�sse zu ziehen. Ferner ist es ein
  modernes Geschmacksurteil, die Reihung der Thesen der Arianer in
  Dok. \ref{sec:4b}~= Urk. 4b besser strukturiert und durchdacht zu
  halten.

  Eine weitere damit zusammenh�ngende Frage, die auch unmittelbar die
  Chronologie betrifft, ist die nach der Thalia des Arius. Meistens
  wird davon ausgegangen, Arius habe diese Schrift in Nikomedien auf
  Anregung des dortigen Eusebius hin verfa�t. Kannengie�er verlegte
  die Abfassung aufgrund einer anderen �bersetzung von Ath., syn. 15,
  nach Alexandrien vor die Abreise des
  Arius.\footcite{Kannengiesser:Arius} Auch Lorenz datierte die
  Abfassung der Thalia des Arius auf diesen fr�hen Zeitpunkt, da in
  der Thalia die theologischen Thesen des Arius weniger ausgereift
  erscheinen.\footcite{Lorenz:Arius} Schon G. Bardy, wie auch jetzt
  wieder R. Williams, begr�ndete die Umstellung von
  Dok. \ref{sec:4b}~= Urk. 4b und Dok. \ref{ch:14}~= Urk. 14 mit der
  Existenz der Thalia, welche in Dok. \ref{sec:4b}~= Urk. 4b
  vorausgesetzt sei, in Dok. \ref{ch:14}~= Urk. 14 jedoch nicht. Alle
  diese �berlegungen, das Verh�ltnis zwischen Dok. \ref{sec:4b}~=
  Urk. 4b, Dok. \ref{ch:14}~= Urk. 14 und der Thalia zu bestimmen,
  sind sehr problematisch. Es ist methodisch fragw�rdig, aufgrund eines
  bestimmten Verst�ndnisses der Anf�nge des arianischen Streits die
  Thalia zu datieren und diese Datierung dann wieder f�r eine
  zeitliche Bestimmung von Texten aus dem arianischen Streit
  heranzuziehen. Zweitens ergibt sich aus Dok. \ref{sec:4b}~= Urk. 4b
  selbst nicht, da� die Thalia die Quelle f�r die dort
  (Dok. \ref{sec:4b}~= Urk. 4b,7--10) referierten Thesen des Arius
  ist, vielmehr wird mehrfach auf eine Gespr�chssituation hingewiesen
  (bes. Dok. \ref{sec:4b}~= Urk. 4b,10), so da� doch
  h�chstwahrscheinlich die alexandrinischen theologischen Gespr�che
  und evtl. dort angefertigte Protokolle Grundlage f�r
  Dok. \ref{sec:4b}~= Urk. 4b bilden d�rften.\footnote{�hnlich auch
    \cite[556~f.]{Loehr2005}} Dreht man die Reihung von
  Dok. \ref{sec:4b}~= Urk. 4b und Dok. \ref{ch:14}~= Urk. 14 um, dann
  stellt sich dasselbe Problem eigentlich auch f�r Dok. \ref{ch:18}~=
  Urk. 18 (das Schreiben der antiochenischen Synode), die ja auf jeden
  Fall nach Dok. \ref{sec:4b}~= Urk. 4b anzusetzen ist, aber ebenfalls
  wie Dok. \ref{ch:14}~= Urk. 14 Thesen aus der Thalia des Arius
  �bergeht.\footnote{Dies beobachtete \cite[547~f.]{Loehr2005}}

  Wichtig scheinen die Beobachtungen von Stead zu Dok. \ref{ch:1}~=
  Urk. 1 zu sein. Hier ist Arius bereits verurteilt und aus der Stadt
  vertrieben, hier sind bereits mehrere syrisch-pal�stinische Bisch�fe
  von Alexander verurteilt worden, offensichtlich weil sie sich in
  irgendeiner Form f�r Arius ausgesprochen hatten. So geh�rt dieser
  Brief sicherlich nicht an den Beginn des arianischen Streits,
  sondern setzt schon eine �berregionale Entwicklung
  voraus. Dar�berhinaus ist zu fragen, ob dieser Brief der erste des
  Arius an Eusebius von Nikomedien sein mu�.  Vielmehr erinnert Arius
  Eusebius doch an dessen schon vorher ausgesprochene Gemeinschaft
  bzw.  Unterst�tzung (Dok. \ref{ch:1}~= Urk. 1,2); er spricht die
  Verh�ltnisse auch nur kurz an, erw�hnt den verfolgenden Bischof ohne
  Namensnennung (\editioncite[1,7]{Opitz:Urk}) und beruft sich darauf, da� Eusebius
  eigentlich schon �ber die Streitsache informiert sei
  (Dok. \ref{ch:3}~= Urk. 3,6: \griech{di`a to~uto diwk'omeja, loip`on
    s`u o~>idac}). Der eigentliche Anla� dieses Briefes ist nun aber,
  da� auch Eusebius von Caesarea, Theodotus von Laodicea, Paulinus von
  Tyrus, Athanasius von Anazarba, Gregorius von Berytus und A"etius
  von Lydda verurteilt worden sind~-- nicht aber ">H�retiker"< wie
  Philogonius von Antiochia, Hellanicus von Tripolis und Macarius von
  Jerusalem --, wor�ber Arius den Eusebius von Nikomedien mit diesem
  Brief in Kenntnis setzen will. Mit dem Begriff ">Verurteilung"<
  scheint Arius die Situation etwas �bertrieben darzustellen, da die
  nachfolgenden Synoden von Antiochia und Nicaea nichts von dieser
  Verurteilung zu wissen scheinen.\footnote{�hnlich auch
    \cite[553]{Loehr2005}.} Er kann nur gemeint haben, da� Alexander
  sie als Unterst�tzer des Arius auch als verurteilt betrachtet
  (mittels Dok. \ref{ch:14}~= Urk. 14 und Dok. \ref{ch:15}~= Urk. 15;
  s.\,u.). Arius ist �berzeugt, in Eusebius von Nikomedien einen
  Gesinnungsgenossen zu haben, so da� er in Dok. \ref{ch:1}~=
  Urk. 1,4--5 seine Thesen nur noch einmal in Kurzfassung referiert,
  und zwar wohl nicht um nachzufragen, ob der Nidomedier mit ihm
  �bereinstimme, sondern um den schon bestehenden Konsens zu erhalten
  und an die gemeinsamen Thesen zu erinnern. Trotz dieser K�rze passen
  die in Dok. \ref{ch:1}~= Urk. 1 erw�hnten Stichworte inhaltlich zu
  Dok. \ref{ch:14}~= Urk. 14: Alexander lehre, da� der Vater immer
  Vater, der Sohn immer Sohn sei (vgl. Dok. \ref{ch:14}~=
  Urk. 14,15~ff., bes. 26); ferner lehre er die ewige Zeugung des
  Sohnes aus dem Vater (vgl. Dok. \ref{ch:14}~= Urk. 14,52); Alexander
  kritisiere, der Sohn sei aus nichts entstanden
  (vgl. Dok. \ref{ch:14}~= Urk. 14,15.22). In beiden Dokumenten steht
  au�erdem die Frage, ob auch der Sohn ungezeugt sei, im Mittelpunkt
  (Dok. \ref{ch:14}~= Urk. 14,19.44~f.47~f.52; vgl. auch
  Dok. \ref{ch:8}~= Urk. 8,3). Diese inhaltlichen Konvergenzen
  sprechen f�r eine auch zeitliche N�he der beiden Dokumente.
  Schlie�lich mu� noch besonders die Erw�hnung des Paulinus von Tyrus
  in Dok. \ref{ch:1}~= Urk. 1 beachtet werden. Offensichtlich hat
  Paulinus sich in irgendeiner Form positiv zu den Thesen des Arius
  ge�u�ert und so eine Verurteilung auf sich gezogen; dann mu� aber
  der Brief des Eusebius von Nikomedien (Dok. \ref{ch:8}~= Urk. 8) und
  auch der Brief des Paulinus (Dok. \ref{ch:9}~= Urk. 9) diesem
  Dok. \ref{ch:1}~= Urk. 1 vorausgehen.  Es w�re n�mlich sinnlos, da�
  Eusebius Paulinus auf"|fordert, endlich Stellung zu beziehen
  (Dok. \ref{ch:8}~= Urk. 8), nachdem er bereits diesbez�glich laut
  Dok. \ref{ch:1}~= Urk. 1 verurteilt worden ist! Arius z�hlt in
  Dok. \ref{ch:1}~= Urk. 1 Philogonius zu denen, die nicht verurteilt
  worden sind. Nun ist Dok. \ref{ch:15}~= Urk. 15 mit Fragmenten aus
  einem Rundbrief des Alexander gerade in der Fassung mit der
  Unterschrift des Philogonius von Antiochia �berliefert
  (Dok. \ref{ch:15}~= Urk. 15,5), so da� zu �berlegen ist, ob dieses
  Dokument nicht genau dasjenige ist, welches Alexander verschicken
  lie� mit der Auf"|forderung, es zu unterschreiben, so da� jeder, der
  es nicht unterschreibt, als verurteilt zu gelten habe wie Arius
  selbst. Eben darauf scheint sich Arius in Dok. \ref{ch:1}~= Urk. 1
  zu beziehen, so da� wahrscheinlich dieses Dokument wirklich nicht an
  den Anfang, sondern an einen sp�teren Zeitpunkt zu setzen ist.

  Es sind zwei Rundbriefe des Alexander �berliefert, einmal
  Dok. \ref{sec:4b}~= Urk. 4b, ferner jenes Dok. \ref{ch:15}~=
  Urk. 15. In Urk.  4b teilt Alexander allen Mitbisch�fen mit, da� in
  Alexandrien Arius und seine Anh�nger verurteilt worden sind, um alle
  Adressaten vor weiteren arianischen Umtrieben zu warnen,
  insbesondere da Eusebius von Nikomedien sich f�r sie
  einsetze. Dieser Rundbrief scheint von Alexander ein weiteres Mal
  Verwendung gefunden zu haben, da er in Dok. \ref{sec:4a}~= Urk. 4a
  die Versendung dieses Rundbriefes voraussetzt, nun aber, da zwei
  Presbyter und vier Diakone aus Alexandrien und der Mareotis dieser
  Absetzung des Arius und seiner Anh�nger ebenfalls nicht zustimmten,
  erneut seine Position in Alexandrien und der Mareotis absichern
  will. So ist also Dok. \ref{sec:4b}~= Urk. 4b mit Unterschriften der
  Presbyter und Diakone aus Alexandrien und der Mareotis �berliefert,
  wie Alexander es in Dok. \ref{sec:4a}~= Urk. 4a fordert. Von dem
  zweiten Rundbrief (Dok. \ref{ch:15}~= Urk. 15) ist leider nur wenig
  �berliefert, da im christologischen Streit nur der Abschnitt mit dem
  Bekenntnis zur Gottesgeb�rerin Maria und der Unterschrift des
  Antiocheners Philogonius interessierte.  Dieser Brief wurde
  verschickt, um Unterschriften zu sammeln: Laut einleitender
  Bemerkung kamen etwa 200 Bisch�fe zusammen, laut Referat in
  Dok. \ref{ch:15}~= Urk. 15,4 unterzeichneten Bisch�fe aus Aegyptus,
  Theba"is, Libya, Pentapolis, Palaestina, Arabia, Achaia, Thracia,
  Hellespont, Asia, Caria, Lycia, Lydia, Phrygia, Pamphylia, Galatia,
  Pisidia, Pontus, Polemoniacus, Cappadocia, Armenia und Philogonius
  aus Antiochia. Warum sollte dieser Rundbrief nicht der in
  Dok. \ref{ch:14}~= Urk. 14 erw�hnte sein, den nach
  Dok. \ref{ch:14}~= Urk. 14,59 Bisch�fe aus Aegyptus, Theba"is,
  Libya, Pentapolis, Syria, Lycia, Pamphylia, Asia und Cappadocia
  schon unterschrieben hatten? Der Zusammenhang beider Texte f�llt
  sofort auf, wenn man Dok. \ref{ch:14}~= Urk. 14,53--56 parallel zu
  Dok. \ref{ch:15}~= Urk. 15,2~f. liest, worauf schon Opitz im Apparat
  zum Text verwies.\footnote{Vgl. auch die �bersicht bei
    \cite[20--28]{Rogala:Anfaenge}.} Dok. \ref{ch:14}~= Urk. 14 ist
  offenbar eine ausf�hrlichere Darstellung der Aussagen von
  Dok. \ref{ch:15}~= Urk. 15, die Alexander verfa�te, da er von seinem
  Namensvetter noch keine Antwort erhalten hatte. Beide Texte weisen
  au�erdem eine gewisse N�he zu Dok. \ref{ch:18}~= Urk. 18 auf
  (vgl. bes.  Dok. \ref{ch:14}~= Urk. 14,53--55; Dok. \ref{ch:15}~=
  Urk. 15,2; Dok. \ref{ch:18}~= Urk. 18,12), was auch einen
  zeitlichen Zusammenhang nahelegt, wie schon Markschies unter
  Berufung auf Abramowski bemerkte.\footcite[291 Anm.
  81]{Markschies:Entstehen} Vergleichbar sind auch die Abschnitte
  Dok. \ref{ch:14}~= Urk. 14,47 und Dok. \ref{ch:18}~= Urk. 18,9, in
  denen die unaussprechliche Art und Weise der Zeugung des Sohnes
  behandelt wird (unter Hinweis auf Mt 11,27; anders als in
  Dok. \ref{sec:4b}~= Urk. 4b,8), sowie Dok. \ref{ch:14}~=
  Urk. 14,29--31 und Dok. \ref{ch:18}~= Urk. 18,10.13 �ber die
  unwandelbare nat�rliche Sohnschaft. Hinzuweisen ist auch auf die
  Beobachtung von Skarsaune\footcite[46~f]{Skarsaune:neglecteddetail},
  da� die Anathematismen des Nicaenums inhaltlich die
  H�resiebeschreibung aus Dok. \ref{ch:14}~= Urk. 14,10~f. wieder
  aufnehmen.

  Dok. \ref{ch:7}~= Urk. 7, der Brief des Eusebius von Caesarea an
  Alexander, ist chronologisch insofern bedeutend, als Eusebius hier
  sowohl Dok. \ref{sec:4b}~= Urk. 4b, den Rundbrief des Alexander, als
  auch Dok. \ref{ch:6}~= Urk. 6, die theologische Erkl�rung des Arius,
  voraussetzt, da Eusebius in dem hier �berlieferten Fragment die
  Aussagen des Alexander �ber Arius aus Dok. \ref{sec:4b}~= Urk. 4b
  mit dessen eigenen Aussagen aus Dok. \ref{ch:6}~= Urk. 6
  vergleicht. Da� dies nicht einen vorgefundenen Konsens voraussetzt,
  wie Telfer annahm, wurde oben schon erl�utert. Nun stellt Williams
  Dok. \ref{ch:6}~= Urk. 6 als erstes Dokument ganz an den Anfang und
  Dok. \ref{ch:7}~= Urk. 7 als letztes kurz vor Nicaea, da er
  Dok. \ref{sec:4b}~= Urk. 4b so sp�t ansetzt, Dok. \ref{ch:7}~=
  Urk. 7 aber nur danach geschrieben worden sein kann. Es erscheint
  aber unwahrscheinlich, da� diese Texte so weit auseinanderzuziehen
  sind, auch ist es wenig glaubw�rdig, da� Eusebius als in Antiochia
  vorl�ufig Verurteilter in der Position gewesen w�re, eine
  Vermittlungsinitiative zu ergreifen.

  Es ist nicht �berzeugend, wie Williams Dok. \ref{sec:4b}~= Urk. 4b
  als Dokument der Synode unter Ossius in Alexandrien 324/325
  anzusehen. Auf dieser Synode wurde offensichtlich der arianische
  Streit in Alexandrien nicht geschlichtet; das Schisma des Colluthus
  aber konnte beigelegt werden (Ath., apol.\,sec. 74,4; 76,3: Ischyras
  als von Colluthus geweihter Presbyter wird wieder Laie;
  vgl. apol.\,sec. 12,1). Nun unterschreibt Colluthus zwar an
  exponierter Stelle Dok. \ref{sec:4b}~= Urk.  4b, aber weder der kaiserliche Gesandte
  Ossius wird in Dok. \ref{sec:4b}~= Urk. 4b erw�hnt noch die Einigung
  �ber das Schisma des Colluthus, daf�r aber die von 100 �gyptischen
  Bisch�fen entschiedene Exkommunikation des Arius und seiner
  Anh�nger. Diese wird als erstmalige Verurteilung berichtet, wogegen
  dies doch laut Williams schon die dritte alexandrinische Versammlung
  gegen die Arianer sein soll. Ist also entgegen Williams
  Dok. \ref{sec:4b}~= Urk. 4b doch in die Fr�hzeit zu setzen und auch
  Dok. \ref{ch:6}~= Urk. 6 ein Dokument dieser Zeit, so ist das
  Verh�ltnis dieser beiden Texte zueinander zu
  kl�ren. Dok. \ref{ch:6}~= Urk. 6 wei� nichts von einer Verurteilung
  der Anh�nger des Arius, die Dok. \ref{sec:4b}~= Urk. 4b berichtet,
  und bezieht sich auch nicht direkt auf die Thesen, die laut
  Dok. \ref{sec:4b}~= Urk. 4b Arius lehren solle.  Umgekehrt beruft
  sich der Verfasser von Dok. \ref{sec:4b}~= Urk. 4b in dem Referat
  arianischer Thesen auch nicht auf Dok. \ref{ch:6}~= Urk. 6, sondern
  vielmehr auf �u�erungen, die wahrscheinlich im Zusammenhang der
  theologischen Gespr�che in Alexandrien getroffen worden sind. So
  w�re es eine M�glichkeit, Dok. \ref{ch:6}~= Urk. 6 als
  vorbereitendes Dokument dieser Diskussionen zu verstehen;
  Dok. \ref{sec:4b}~= Urk. 4b nimmt dann aber direkt Bezug auf
  Aussagen w�hrend dieser theologischen Diskussionen.

  Williams gewichtet aufgrund seiner neuen Chronologie die Rollen der
  beiden Eusebe neu.  Zuerst habe Eusebius von Caesarea Arius
  unterst�tzt (Dok. \ref{ch:10}~= Urk. 10: Synode in Palaestina;
  Dok. \ref{ch:3}~= Urk. 3: Eusebius schreibt an Euphration), Eusebius
  von Nikomedien habe sich erst danach f�r Arius eingesetzt
  (Dok. \ref{ch:1}~= Urk. 1 und Dok. \ref{ch:2}~= Urk. 2: Briefwechsel
  mit Arius; Dok. \ref{ch:8}~= Urk. 8: Eusebius schreibt an Paulinus;
  Dok. \ref{ch:5}~= Urk. 5: bithynische Synode). Dagegen ist aber
  einzuwenden, da� gerade in der Schlu�phase vor Nicaea, in die
  Williams Dok. \ref{sec:4b}~= Urk. 4b mit den darin erw�hnten
  Aktivit�ten des Eusebius von Nikomedien verlegt, Eusebius von
  Caesarea derjenige ist, der treu weiterhin Arius unterst�tzt und
  sogar eine vorl�ufige Verurteilung in Antiochia in Kauf nimmt. In
  Dok. \ref{ch:18}~= Urk. 18 liest man nichts von Eusebius von
  Nikomedien. Vielleicht ist es doch wahrscheinlicher, da� Eusebius
  von Nikomedien sich schon anfangs sehr f�r Arius eingesetzt
  hat. Ferner ist es insgesamt problematisch, die Aktivit�ten der
  beiden Eusebe chronologisch gegeneinander auszuspielen.
  Anhaltspunkt daf�r bietet allein die Darstellung bei Soz., h.\,e. I
  15,9--12, der erst von den Aktivit�ten des Nikomediers berichtet und
  dann zur palaestinischen Synode unter Eusebius von Caesarea
  �bergeht. Generell wird dem Bericht bei Sozomenus gro�e
  Glaubw�rdigkeit attestiert; das Berichtete hat sich wahrscheinlich
  aber auch zeitlich �berschnitten.

\section*{Zusammenfassung}
\noindent Aus diesen �berlegungen ergibt sich nun folgender neuer Vorschlag
einer Chronologie der Anf�nge des arianischen Streites:

\begin{longtable}[l]{p{10cm}p{4cm}}
  % \toprule
  % \endhead
  % \bottomrule
  % \endlastfoot
  \textit{Erste Phase: Der Streit in Alexandrien} & \\
  Der Streit bricht in Alexandrien aus. & \\
  Arius wendet sich wahrscheinlich schon fr�h an Eusebius von
  Nikomedien (pers�nliche Kontakte,
  als Lucianus-Verehrer). & nicht �berliefert\\
  Es gibt zwei Versammlungen bzw. theologische Diskussionen in
  Alexandrien. & Soz., h.\,e. I
  15,4--6 \\
  Arius verfa�t eine Glaubenserkl�rung daf�r. & Dok. \ref{ch:6}~= Urk. 6 \\
  Alexander verlangt schlie�lich einen Konsens gegen die Arianer. & \\
  Alexander schreibt~-- ohne Erfolg zu haben~-- eine Mahnung an die Arianer, unterschrieben von den Presbytern und Diakonen Alexandriens und der Mareotis, der Gottlosigkeit abzusagen. & in Dok. \ref{sec:4a}~= Urk. 4a erw�hnt; nicht �berliefert \\
  Versammlung der 100 Bisch�fe in �gypten: Arius und seine Anh�nger werden abgesetzt und exkommuniziert. & \\
  Widerspruch kommt besonders von Eusebius von Nikomedien. & \\
  Alexander schreibt eine Mitteilung �ber die Absetzung an alle
  Mitdiener. & Dok. \ref{sec:4b}~= Urk. 4b (evtl. ist
  Dok. \ref{ch:16}~= Urk. 16
  das entsprechende Exemplar nach Rom) \\
  Aber aus �gypten verweigern Chares, Pistus (Presbyter, er hat
  Dok. \ref{ch:6}~= Urk. 6
  unterschrieben), Serapion, Perammon, Zosimus und Irenaeus (Dok. \ref{sec:4a}~= Urk. 4a erw�hnt) die Unterschrift. & Vgl. Soz., h.\,e. I 15,7. \\
  Eine erneute Versammlung mit einer Rede von Alexander findet in Alexandrien statt: Alle sollen der Absetzung derer um Arius und derer um Pistus zustimmen. & Dok. \ref{sec:4a}~= Urk. 4a \\
  Die Presbyter und Diakone Alexandriens und der Mareotis
  unterschreiben. & Dok. \ref{sec:4b}~= Urk. 4b mit
  Unterschriften \\
%%
  & \\
  \textit{Zweite Phase nach der alexandrinischen Verurteilung:
    Ausweitung des Konflikts
    au�erhalb �gyptens}\footnote{Die Reihenfolge folgender Briefe bleibt ungenau, da manches gleichzeitig geschieht und sicher noch viele andere entsprechende Briefe existiert haben. Unklar bleibt, wohin Arius tats�chlich pers�nlich gereist ist.} & \\
  Arius kn�pft wieder Kontakt zu Eusebius von Nikomedien: Synode in Bithynia, es sollen unterst�tzende Briefe folgen. & Dok. \ref{ch:5}~= Urk. 5 \\
  Eusebius von Nikomedien schreibt an Paulinus von Tyrus, er solle
  nicht l�nger
  schweigen; & Dok. \ref{ch:8}~= Urk. 8 \\
  Paulinus von Tyrus schreibt nun an Alexander. & Dok. \ref{ch:9}~= Urk. 9\\
  Georg schreibt aus Antiochia an Arius und an Alexander und will
  zwischen den beiden Kontrahenten vermitteln. & Dok. \ref{ch:12}~=
  Urk. 12 und Dok. \ref{ch:13}~= Urk. 13 (diese Briefe k�nnten aber
  auch noch in die erste Phase vor der Verurteilung
  geh�ren.) \\
  Arius wendet sich an Eusebius von Caesarea, Paulinus von Tyrus und
  Patrophilus von
  Scythopolis. & \\
  Synode in Palaestina: Arius darf als Presbyter in Alexandrien
  auftreten,
  soll sich aber seinem Bischof Alexander unterordnen. & Dok. \ref{ch:10}~= Urk. 10 \\
  Eusebius von Caesarea kritisiert unsauberes Vorgehen des Alexander:
  er vergleicht
  Aussagen in Dok. \ref{ch:6}~= Urk. 6 mit dem Referat in Dok. \ref{sec:4b}~= Urk. 4b und bem�ngelt falsches Zitieren in seinem Brief an Alexander. & Dok. \ref{ch:7}~= Urk. 7 \\
  Eusebius von Caesarea schreibt an Euphration von Balanaea �ber entsprechende theologische Fragen. & Dok. \ref{ch:3}~= Urk. 3 \\
  Evtl. geh�ren die Briefe des Athanasius von Anazarba und Theognis von Niz�a in diesen Kontext (vgl. die Einleitungen zu Dok. \ref{ch:AthAnaz}; \ref{ch:TheognisNiz}).& Dok. 13; 14 \\
%%
  & \\
  \textit{Dritte Phase: Zeit des Synodenverbots} & \\
  Die Auseinandersetzung spitzt sich zu; chaotische Zust�nde entstehen
  in Alexandrien: es bildet sich eine Gemeinde der
  Anh�nger des Arius (was Dok. \ref{ch:8}~= Urk. 8 im Prinzip erm�glichte;
  vgl. Dok. \ref{ch:14}~= Urk. 14,3.7),
  Alexander will Arianer aus Alexandrien vertreiben (in Dok. \ref{ch:14}~= Urk. 14,6.8 berichtet, vgl. die Klage des Arius �ber Verfolgungen in Dok. \ref{ch:1}~= Urk. 1); evtl. Zusammenhang mit der Abspaltung des Colluthus. & \\
  Alexander will die Lage im Osten mit dem Rundbrief kl�ren: Wer nicht
  unterschreibt, gilt als Arianer und als verurteilt; das Exemplar
  mit der Unterschrift des Philogonius von Antiochia ist �berliefert. & Dok. \ref{ch:15}~= Urk. 15, mit 200 Unterschriften \\
  Arius schreibt an Eusebius von Nikomedien: Eusebius von Caesarea,
  Theodotus, Paulinus, Gregorius,
  A"etius und alle im Osten sind verurteilt. & Dok. \ref{ch:1}~= Urk. 1\\
  Antwort des Eusebius von Nikomedien.\footnote{Eventuell geh�rt das Fragment aus Eusebius von Nikomedien, Dok. \ref{ch:21}~= Urk. 21, auch in diesen Kontext. Auch wenn diese Aussage auf der Synode von Nicaea 325 zitiert wurde, kann sie trotzdem schon aus einem bereits zuvor verfa�ten Brief stammen.} & Dok. \ref{ch:2}~= Urk. 2?\\
  Alexander will Beteiligung des Namenskollegen aus Byzantium
  erreichen, dem er
  schon Dok. \ref{ch:15}~= Urk. 15 geschickt hatte (durch den Diakon Api). & Dok. \ref{ch:14}~= Urk. 14 \\
%%
  & \\
  \textit{Vierte Phase: Das Ende des Synodenverbots und Eingreifen Konstantins in den
    Streit} & \\
  Brief Konstantins an Alexander und Arius mit der Aufforderung zur Beilegung des Streites. & Dok. \ref{ch:17}~= Urk. 17 \\
  Reise des Ossius von Cordoba, Synode in Alexandrien: Anh�nger des Colluthus verlieren ihre kirchlichen �mter (Ischyras!). & \\
  Synode in Antiochia. & Dok. \ref{ch:18}~= Urk. 18 \\
\end{longtable}

\section*{Die Synode von Antiochia 325. Eine Auseinandersetzung mit
  H. Strutwolf}
\noindent Zu diskutieren bleibt nun noch das von Schwartz entdeckte Schreiben
der Synode von Antiochia (325 n.\,Chr.; Dok. \ref{ch:18}~=
Urk. 18\footcite[134--155]{Schwartz:Dokumente}), da die Authentizit�t
dieses Textes j�ngst wieder bestritten wurde. Die Edition von Schwartz
basierte nur auf einer Handschrift (Codex Parisinus syr. 62), in
welcher der Text zwischen Kanonessammlungen eingef�gt war; inzwischen
sind aber zwei weitere Handschriften entdeckt
worden,\footnote{\cite{Nau:Litterature} (Cod. Vat. Borg. syr. 148);
  \cite{Chadwick:Ossius} (Cod. Mingana syr. 8). Der Cod. Mingana
  syr. 8 best�tigte die alte Konjektur von A.\,I.  Brilliantov (1911),
  da� der Name der ersten Unterschrift nicht Eusebius, sondern Ossius
  lauten mu�.} ferner sind die Bemerkungen von L. Abramowski zur
griechischen R�ck�bersetzung von E. Schwartz zu
beachten.\footcite{Abramowski:Synode}

Schon\looseness=-1\ Adolf von Harnack widersprach der Einsch�tzung von Schwartz, in
dem Dokument einen Text aus der Zeit kurz vor der Synode von Nicaea zu
sehen,\footnote{\cite[477--491]{Harnack:Synode}; daraufhin
  \cite{Schwartz:Antiochia}; und wieder
  \cite[401--425]{Harnack:Synode}; ferner f�r eine antiochenische
  Synode \cite{Seeberg:Synode}. Zu dieser Diskussion vgl. die
  Zusammenfassung von \cite{Cross:Council}.} und auch sp�ter wurde
gelegentlich die Existenz dieser Synode bestritten bzw. dieses
Dokument als F�lschung erkl�rt.\footnote{\cite[29.45~f.]{Urbina:Nizaea};
  \cite{Holland:Synode}; dagegen: \cite{Pollard:Eusebius};
  \cite{Abramowski:Synode}; vgl. auch: \cite{Nyman:Synod}.}  Bem�ngelt
wurde damals, da� der Text nur syrisch �berliefert sei, ferner da� an
keiner anderen Stelle in der altkirchlichen Literatur auf eine
Exkommunikation des Eusebius Bezug genommen werde, haupts�chlich aber, 
da� theologisch Eusebius von Caesarea diesem Text zugestimmt haben
m�sse bzw. ">da� eine Unterzeichnung von Ant Eusebius keine besonderen
Schwierigkeiten gemacht haben konnte"<\footcite[170]{Holland:Synode},
ohne aber genauer den Sitz im Leben einer etwaigen F�lschung zu
kl�ren. Einzig Harnack mutma�te, da� es sich um eine F�lschung aus dem
6./7. Jahrhundert handele, welche die nachnicaenische Absetzung des
Eustathius von Antiochia durch Eusebius von Caesarea r�chen wolle
mittels einer erfundenen Absetzung eben jenes Eusebius durch
Eustathius kurz vor Nicaea 325 n.\,Chr.


Holger Strutwolf greift nun in seiner Monographie �ber Eusebius von
Caesarea\footcite{Strutwolf:Trinitaetstheologie} diese alte Idee von
Harnack leicht ver�ndert wieder auf~-- ohne Harnack allerdings zu
nennen (!)~-- und urteilt, Dok. \ref{ch:18}~= Urk. 18 sei ein Werk
antieusebianischer antiochenischer Kreise, genauer gesagt der kleinen
Eustathianergemeinde in Antiochia, ">die mit der F�lschung oder
Verf�lschung eines Synodalbriefs einer antiochenischen Synode die
Gegner ihres verehrten Eustathius diskreditieren wollten, indem sie in
spiegelbildlicher Umkehrung der Ereignisse von 326 die Ankl�ger zu
Verurteilten machen."<\footcite[39~f]{Strutwolf:Trinitaetstheologie}
Das Dokument sei in Antiochia in den fr�hen f�nfziger Jahren gef�lscht
worden, da einerseits neuarianische Lehre vorauszusetzen sei (die
Aussage, der Sohn sei nicht Abbild des Willens, sondern der
v�terlichen Hypostase, sei gegen die neuarianische These gerichtet,
der Sohn sei \griech{<'omoioc kat`a bo'ulhsin ka`i >en'ergeian,
  >an'omoion kat'' o>us'ian}) und andererseits das
\griech{<omoo'usioc} noch nicht zum unverzichtbaren Schibboleth der
Rechtgl�ubigkeit geworden sei; theologisch handele es sich um eine
Weiterentwicklung athanasianischer Ans�tze. Konkreter Anla� sei die
Weihe des A"etius zum Diakon durch Leontius von Antiochia. Strutwolf
bestreitet nicht, da� �berhaupt eine Synode in Antiochia kurz vor der
nicaenischen stattgefunden habe. Er nimmt vielmehr an, da� die
Eustathianergemeinde den urspr�nglichen Synodalbrief, der
haupts�chlich kirchenrechtliche Fragen angesprochen und neben der
Bischofsliste auch den Hinweis auf die bevorstehende Synode von Ancyra
enthalten habe, entsprechend erweitert habe, und weist auf folgende ">nur
literarkritisch zu interpretierende
Spannungen"<\footcite[36]{Strutwolf:Trinitaetstheologie} hin: die
Bischofsliste (Dok. \ref{ch:18}~= Urk. 18,1) stimme nicht mit der
Liste der Provinzen (Dok. \ref{ch:18}~= Urk. 18,3) �berein; der Text
beginne in der ersten Person Singular (Ossius) und wechsele (ab
Dok. \ref{ch:18}~= Urk. 18,4) zur ersten Person Plural; zu Beginn
hei�e es, Ossius berufe wegen Unordnung eine Synode ein, anschlie�end
komme man aber erst aufgrund der Synode dazu, Unordnungen zu
entdecken; der Text wechsele zwischen der Behandlung theologischer
(Dok. \ref{ch:18}~= Urk. 18,2--3.5~ff.) und disziplinarischer
(Dok. \ref{ch:18}~= Urk. 18,4) Fragen hin und her.

Strutwolf f�hrt jedoch diese literarkritischen Beobachtungen nicht zu
einer klaren Scheidung von prim�ren und sekund�ren Textabschnitten
weiter, was auch schwierig w�re, da in den Passagen im Singular sowohl
Theologisches als auch Disziplinarisches angesprochen wird
bzw. Theologisches sowohl in den Singular- als auch in den
Plural-Passagen vorkommt. Ferner wechseln von Beginn des Textes an
Ich- mit Wir-Passagen, die sich ganz nat�rlich aus dem Duktus ergeben,
da Ossius in seinem Bericht genau zwischen seinen eigenen Aktivit�ten
und denen der Synode unterscheidet, wie es deutlich in
Dok. \ref{ch:18}~= Urk. 18,2 hei�t: ">ist es folgerichtig, da� auch deiner Liebe diese Angelegenheiten bekannt w�rden, die sowohl von mir als auch von unseren heiligen, seelenverwandten Br�dern und Mitdienern verhandelt wie auch verabschiedet wurden."< So berichtet Ossius also, wie er in die Stadt Antiochia
gekommen sei und die Kirche in Unordnung vorgefunden habe, da� er
daraufhin f�r eine Synode Bisch�fe aus benachbarten Gegenden
einberufen habe (Dok. \ref{ch:18}~= Urk. 18,3), da� aber auch dort
�berall vergleichbares Chaos herrsche wegen der verhinderten Synoden
(Dok. \ref{ch:18}~= Urk. 18,4). Danach habe man zun�chst �ber den
theologischen Dissens debattiert (Dok. \ref{ch:18}~=
Urk. 18,5\footnote{Auch in der historischen Notiz im Anschlu� an
  dieses Dokument (vgl. \cite[143]{Schwartz:Dokumente}) ist von
  theologischen und disziplinarischen Themen die Rede.}) und sich
schlie�lich der Position des Alexander von Alexandrien angeschlossen
(Dok. \ref{ch:18}~= Urk. 18,7; es ist nicht ersichtlich, wie Strutwolf
schlie�en kann, da� die Verurteilung des Arius nicht mehr zur Debatte
stehe, sondern schon beschlossene Sache sei), die nun zusammenfassend
mitgeteilt werde (Dok. \ref{ch:18}~= Urk. 18,8--13), inklusive der
vorl�ufigen Verurteilung der drei Bisch�fe (Dok. \ref{ch:18}~=
Urk. 18,14), die sich aber auf der bevorstehenden Synode von Ancyra
rechtfertigen k�nnen (Dok. \ref{ch:18}~= Urk. 18,15). So l��t sich
dieser Text eigentlich ohne Br�che und Spannungen lesen.

Problematisch ist es, wenn Strutwolf nur aufgrund einer kurzen Passage
("> Denn er ist Bild 
nicht aus dem Willen noch von etwas anderem, sondern aus eben der v�terlichen Hypostase."< [Dok. \ref{ch:18}~= Urk. 18,11]) davon
ausgeht, da� die neuarianische Lehre vorauszusetzen sei. Auch schon
vor Nicaea wurde die Rolle des v�terlichen Willens bei der Zeugung des
Sohnes diskutiert, und an dieser Stelle liegt nur ein etwas
erweitertes Schriftzitat (Hebr 1,3) vor.  Entsprechend ist auch die
Ablehnung im Synodalschreiben, der Sohn sei nicht ">durch Willen
oder durch Setzung"< (Dok. \ref{ch:18}~= Urk. 18,10]) gezeugt, zu verstehen. Es gibt einzelne Passagen schon in
den vornicaenischen Texten, aus denen deutlich wird, da� die Herkunft
des Sohnes aus dem Willen des Vaters gegen die Vorstellung einer
physischen Zeugung des Sohnes aus dem Vater betont worden ist, so da�
diese Formulierung im Synodalschreiben von Antiochia nicht so sehr
�berrascht (vgl. Dok. \ref{ch:1}~= Urk. 1,4 [\editioncite[3,1~f.]{Opitz:Urk}];
Dok. \ref{ch:6}~= Urk. 6,2~f. [\editioncite[12,8~f.; 13,4]{Opitz:Urk}]; Dok. \ref{ch:8}~=
Urk. 8,7 [\editioncite[17,3~f.]{Opitz:Urk}]), auch wenn selbstverst�ndlich in der sp�teren Zeit
nach Nicaea dar�ber noch ausf�hrlich diskutiert wurde.

Nicht so schwierig ist ferner, da� nicht alle in Dok. \ref{ch:18}~=
Urk. 18,1 genannten Bisch�fe sich den in Dok. \ref{ch:18}~= Urk. 18,3
genannten Provinzen zuordnen lassen, da dort nur die benachbarten als
wichtigste Provinzen genannt werden.

So erscheint insgesamt die Einsch�tzung von Strutwolf, der Text sei
eine Racheaktion der kleinen Eustathianergemeinde, explizit gegen
Leontius gerichtet, theologisch aber eine Weiterentwicklung
athanasianischer (nicht marcellischer) Elemente, sehr konstruiert und
nur unzureichend begr�ndet.\footnote{Zu beachten bleibt dar�berhinaus,
  da� Strutwolf infolgedessen auch die Rolle des Eusebius von Caesarea
  auf der Synode von Nicaea (325 n.\,Chr.)  neu gewichtet, der, so
  Strutwolf, nicht nur der bedeutendste Bischof der Synode sei,
  sondern dessen Bekenntnis auch die Grundlage f�r das Nicaenum bilde
  (\cite[44--61]{Strutwolf:Trinitaetstheologie}).} Die Kritik, man
h�re sp�ter nichts von dieser Verurteilung des Eusebius, ist so
problematisch wie alle Argumente e silentio. So k�nnte man Strutwolf
entgegenhalten, da�, wenn dieser Text wirklich Anfang der 50er Jahre
gef�lscht worden sei, sich Athanasius in seiner Schrift De decretis
Nicaenae synodi auf die druckfrisch gef�lschte Verurteilung des
Eusebius h�tte beziehen m�ssen, da er hier gerade mit der umstrittenen
Haltung des Eusebius zum Nicaenum argumentiert (decr. 3).  Nat�rlich
bleibt das Ereignis einer ">vorl�ufigen Verurteilung"< ein einmaliger,
ungew�hnlicher Vorgang, aber die besondere Situation einer kurz
bevorstehenden erstmaligen gro�en Synode, vom Kaiser einberufen,
erforderte besondere Ma�nahmen. So bleiben wir bei der Einsch�tzung, da�
Dok. \ref{ch:18}~= Urk. 18 ein echtes Synodalschreiben einer antiochenischen Synode unmittelbar vor der Synode von Nicaea ist.


\section*{Die letzten Lebensjahre des Arius}
\label{chron:Arius}
\enlargethispage{\baselineskip}
\noindent Opitz stellt an den Schlu� der beiden von ihm noch herausgegebenen
Faszikel zwei Dokumente, die eine Verurteilung des Arius dokumentieren:
das Edikt gegen Arius (Dok. \ref{ch:33}~= Urk. 33) und einen Brief des
Kaisers Konstantin an Arius und seine Anh�nger (Dok. \ref{ch:34}~=
Urk. 34). Er �berspringt damit, da er Dok. \ref{ch:32}~= Urk. 32 in
das Jahr 328, Dok. \ref{ch:33}~= Urk. 33 und Dok. \ref{ch:34}~=
Urk. 34 in das Jahr 333 datiert, einen Zeitraum von f�nf Jahren, ">um
den �berlieferungsgeschichtlichen Rahmen der Urkunden zu
ber�cksichtigen"<, wie er selbst erkl�rt.\footnote{\editioncite[76]{Opitz:Urk}. Genauere
  Hintergr�nde f�r diese chronologische Einordnung gibt Opitz nicht;
  \cite{Opitz:Zeitfolge} deckt nur die Jahre bis 328 ab.} Beide Texte
sind n�mlich, wie andere Dokumente vorher schon, in den Handschriften
im Anhang zu De decretis Nicaenae synodi und beim 
Anonymus Cyzicenus �berliefert. Nun stellt sich jedoch bei Betrachtung aller
Schriften des und an Arius nach der Synode von Nicaea 325, zusammen
mit Dok. 38 (Konflikte um die Wiederherstellung der Kircheneinheit in
�gypten) und Dok. 39 (Die Synode von Jerusalem 335), insgesamt die
Frage, was sich in den letzten Lebensjahren des Arius
ereignete. Insbesondere mu� noch einmal �berpr�ft werden, ob von einer
mehrfachen Verurteilung und Rehabilitierung des Arius auszugehen ist.

�blicherweise werden die Ereignisse seit Schwartz und Opitz wie folgt dargestellt: Arius wurde
zwar auf der Synode von Nicaea 325 verurteilt, zusammen mit den zwei
treuen Anh�ngern Theonas von Marmarice und Secundus von Ptolema"is
(Dok. \ref{ch:23}~= Urk. 23,5), konnte aber gegen Ende 327 seine
Rehabilitierung erreichen, nachdem er an den kaiserlichen Hof in
Nikomedien geladen worden war (Dok. \ref{ch:29}~= Urk. 29) und dem
Kaiser ein schriftliches Bekenntnis vorgelegt hatte
(Dok. \ref{ch:30}~= Urk. 30). Es bleibt zu vermuten, da� nicht nur der
Kaiser die Verurteilung des Arius aufhob, sondern zus�tzlich eine
Synode dies beschlo�, da sich nun auch Eusebius von Nikomedien und
Theognis von Nicaea an eine Synode mit der Bitte um ihre eigene
Rehabilitation wandten (Dok. \ref{ch:31}~= Urk. 31) und sich darauf
bezogen, da� die Synodalen auch Arius zur�ckberufen h�tten (Urk.
31,4: \dots~\griech{<op'ote a>ut`on t`on >ep`i to'utoic >enag'omenon
  <'edoxe t~h| <um~wn e>ulabe'ia| filanjrwpe'usasjai ka`i
  >anakal'esasjai} [\editioncite[65,15~f.]{Opitz:Urk}]). Wann und wo eine
derartige Synode stattgefunden haben soll oder ob verschiedene
Provinzsynoden gemeint sind, bleibt unklar, da die Hinweise sp�rlich
und undeutlich sind (vgl. noch Thdt., h.\,e. I 26,1; Eusebius,
v.\,C. III 23; Anon.Cyz. II 15,7; Philost., h.\,e. II 7;
Dok. \ref{ch:34}~= Urk. 34).  Konstantin forderte daraufhin Alexander
von Alexandrien auf, Arius wieder in die Kirchengemeinschaft
aufzunehmen (Dok. \ref{ch:32}~= Urk. 32). Dieser verweigerte sich
diesem Anliegen jedoch ebenso wie der im Juni 328 geweihte neue
Bischof Athanasius~-- und Konstantin lie� die Dinge zun�chst auf sich
beruhen. Dann jedoch, im Jahr 333, schrieb Arius erneut an den Kaiser,
um seine Wiedereinsetzung zu erreichen, beschwerte sich �ber die
ungekl�rte Situation, erreichte jedoch mit der Drohung, eine
Sonderkirche zu bilden, die gegenteilige Reaktion beim Kaiser, der
sein Ansinnen br�sk ablehnte (Dok. \ref{ch:34}~= Urk. 34) und ein
Edikt gegen ihn verh�ngte (Dok. \ref{ch:33}~= Urk. 33). Dennoch
enth�lt der Kaiserbrief (Dok. \ref{ch:34}~= Urk. 34) auch eine
Einladung, an den kaiserlichen Hof zu kommen. Arius leistete diesem
Folge und konnte nochmals rehabilitiert werden. Konstantin schrieb nun
an die Synodalen von Jerusalem, sie sollen den Fall des Arius kl�ren,
nachdem er selbst schon dessen Glauben f�r rechtm��ig erachtet habe.
Auf der Synode von Jerusalem 335 wurde Arius wunschgem��
rehabilitiert, aber bei seiner R�ckkehr nach �gypten kam es zu
Tumulten. Daraufhin wurde Arius nochmals nach Konstantinopel befohlen,
rehabilitiert, starb aber, bevor diese Rehabilitierung in Alexandrien
durchgesetzt werden konnte, noch in Konstantinopel
(ep.\,Aeg.\,Lib. 18~f.; ep.\,mort.\,Ar.
2~f.).\footnote{\cite[264~f.]{Hanson:Search};
  \cite[49~f.]{Boehm:Christologie}; \cite[118--124.161]{Arnold:Early};
  \cite[138~f.142~f.]{Lorenz:Osten}; \cite[74--81]{Williams:Arius};
  \cite[99--134]{Simonetti:crisi}; \cite[100--103]{Ayres:Nicaea}.}

\enlargethispage{\baselineskip}
Diese Darstellung beruht einerseits auf der Berichterstattung der
Kirchenhistoriker Socrates und Sozomenus, andererseits auf
Forschungsergebnissen von
Schwartz\footcite[200--260]{Schwartz:Nicaea}, der in zwei Punkten die
chronologische Einordnung der Kirchenhistoriker umgestellt hatte.
Erstens: Nach den Kirchenhistorikern ist das konstantinische Edikt
gegen Arius (Dok. \ref{ch:33}~= Urk. 33) in den Zusammenhang der
Synode von Nicaea 325 zu verorten (Socr., h.\,e. I 9,30: hier wird
Dok. \ref{ch:33}~= Urk. 33 zitiert; Soz., h.\,e. I 21,4: hier wird
Dok. \ref{ch:33}~= Urk. 33 referiert); Schwartz jedoch datiert diesen
Text zusammen mit dem Brief des Konstantin Dok. \ref{ch:34}~= Urk. 34
in das Jahr 333, da in Dok. \ref{ch:34}~= Urk. 34 ein Parterius als
�gyptischer Pr�fekt angegeben wird, der f�r dieses Jahr bezeugt
ist.\footnote{\cite[202]{Schwartz:Nicaea};
  vgl. \cite[109]{Hubner1952}.} Aus dieser Umstellung ergibt sich die
Konsequenz, da� Arius ein zweites Mal verurteilt worden sein mu�te, so
da� sich nicht ein l�ngerer Rehabilitierungsproze� (wegen der
andauernden Weigerung des Alexander und des Athanasius, Arius
wiederaufzunehmen), sondern ein Wechsel von Verurteilung,
Rehabilitierung, Verurteilung und erneuter Rehabilitierung
ergibt. Zweitens: Socrates und Sozomenus setzen Dok. \ref{ch:29}~=
Urk. 29 und Dok. \ref{ch:30}~= Urk. 30 in die Zeit nach der Bischofswahl des Athanasius an, und
zwar in die Vorgeschichte der Synoden von Tyrus und Jerusalem, auf
denen Athanasius verurteilt und Arius rehabilitiert wurde (Socr.,
h.\,e. I 25~f.; Soz., h.\,e. II 27). Schwartz jedoch datiert diese beiden
Texte fr�her, in den Zusammenhang einer sogenannten Nachsynode von
Nicaea im Jahr 327, da schon vor der Bischofswahl des Athanasius, auch
vor der R�ckkehr des Eusebius von Nikomedien Arius rehabilitiert
worden sei. Die Darstellung bei den Kirchenhistorikern, da� auf der
Synode von Jerusalem �ber Arius und Euzoius verhandelt wurde, beruhe
auf der falschen Einordnung der Dok. \ref{ch:29}~= Urk. 29 und
Dok. \ref{ch:30}~= Urk. 30, die tats�chlich ohne Zweifel fr�her zu
datieren sind (zwischen 325 und 328).\label{chron:irrtum}

% \clearpage
Folgende Probleme ergeben sich aber mit den beiden Dok. \ref{ch:33}~=
Urk. 33 und Dok. \ref{ch:34}~= Urk. 34, insbesondere wenn man auch die
entsprechenden sp�teren Dokumente bis zur Synode von Jerusalem 335
ber�cksichtigt (Dok. \ref{ch:Arius} und \ref{ch:Jerusalem335}):  In
Dok. \ref{ch:Arius}, welches von Athanasius herangezogen wird, um zu
belegen, da� Konstantin ihn dazu aufgefordert habe, Arius wieder
aufzunehmen, ist nur von ">denen um Arius"< die Rede. In dem zitierten
Teil des Kaiserbriefes ist noch nicht einmal dies erkennbar, es hei�t
hier nur allgemein ">welche, die in die Kirchengemeinschaft
aufgenommen werden wollen"<, was sogar die Melitianer meinen k�nnte
(s. die Einleitung zu Dok. \ref{ch:Arius}). Auch in dem Brief der
Synode von Jerusalem an die �gypter sind nur ">die um Arius"< bzw. die
Presbyter des Arius angesprochen, was den Eindruck best�tigt, da� es
im Jahr 335 nicht mehr um Arius pers�nlich ging, sondern nur noch um
">�briggebliebene"< Anh�nger des Arius in �gypten. Ferner ist zu
beachten, da� Athanasius in seinen polemischen Berichten
(ep.\,Aeg.\,Lib. 18~f.; ep.\,mort.\,Ar.) �ber den pl�tzlichen Tod des
H�retikers dieses Ereignis in die Zeit des konstantinopler Bischofs
Alexander datiert, was doch trotz aller Polemik glaubhaft sein
d�rfte. Jener Alexander mu� jedoch noch vor der Synode von Tyrus
verstorben sein, da dort schon die Teilnahme von Paulus als Bischof
von Konstantinopel belegt ist
(s. Dok. \ref{sec:RundbriefSerdikaOst},13). Sind diese Beobachtungen
korrekt, dann wurde auf der Synode von Jerusalem nicht mehr �ber Arius
verhandelt, da er schon verstorben war. Auch Theodoret berichtet
�ber den Tod des Arius unmittelbar im Anschlu� an die Synode von
Nicaea (h.\,e. I 14).

Dieser Befund wird auch durch die Ereignisse erg�nzt, die zur
Verurteilung des Athanasius gef�hrt hatten. Hier spielt n�mlich die
Frage, ob Arius wieder aufgenommen wurde oder nicht bzw. ob sich
Athanasius diesem Anliegen verweigert habe, �berhaupt keine Rolle,
weder w�hrend der synodalen Verhandlungen in Tyrus noch w�hrend der
Gespr�che mit dem Kaiser in Konstantinopel. Die von Schwartz
vorgenommene neue zeitliche Einordnung der kaiserlichen Verurteilung
des Arius (Dok. \ref{ch:33}~= Urk. 33; Dok. \ref{ch:34}~= Urk. 34)
basiert allein auf der Erw�hnung des Pr�fekten �gyptens, Paterius. Da
aber die Liste der Pr�fekten �gyptens in den Jahren zwischen Nicaea 325
und 328 l�ckenhaft ist (vgl.  \cite{Hubner1952}), k�nnte ein Paterius
auch vorher schon einmal Pr�fekt gewesen sein, so da� die zeitliche
Einordnung von Dok. \ref{ch:33}~= Urk. 33; Dok. \ref{ch:34}~= Urk. 34
in das Jahr 333 nicht zwingend ist. Oder man k�nnte die M�glichkeit in
Rechnung stellen, da� Athanasius sich dieses Dokument
(Dok. \ref{ch:34}~= Urk. 34) zu diesem sp�teren Zeitpunkt noch einmal
beschaffen lie�, um seine Position gegen die Anh�nger des Arius vor
der Synode von Caesarea 333 zu st�rken. Dieses kaiserliche Edikt gegen
Arius (Dok. \ref{ch:33}~= Urk. 33) k�nnte sehr gut, wie es die
Kirchenhistoriker berichten, in die Zeit direkt nach Nicaea geh�ren
und w�re dann als kaiserliche Best�tigung der kirchlichen Verurteilung
durch die Synode anzusehen. Der Brief Dok. \ref{ch:34}~= Urk. 34 mit
der Auf"|forderung an Arius am Schlu�, doch noch einmal an den Hof zu
kommen, um dar�ber zu verhandeln, pa�t gut zu Dok. \ref{ch:29}~=
Urk. 29, wo eben eine schon vorausgehende Einladung an den Hof
angesprochen wird.  Die Aussagen in Dok. \ref{ch:34}~= Urk. 34, aus
denen man einen Brief des Arius an den Kaiser in Ans�tzen
rekonstruieren kann, passen gut in diese fr�here Zeit knapp vor
Nicaea. Es ist durchaus m�glich, da� Arius versucht hat, mit einem
pers�nlichen Brief an den Kaiser die drohende synodale Verurteilung und ihr folgende Verbannung abzuwenden, und da� er darin auch erw�hnt
hat, wie gro� sein R�ckhalt in Libya sei. Die nur unscharf zu
erkennenden theologischen Aussagen passen ebenfalls besser in die Zeit
vor Dok. \ref{ch:30}~= Urk. 30 als danach.

So ist davon auszugehen, da� Arius erstens nach der Synode von Nicaea
nicht noch einmal verurteilt worden ist, was auch in der gesamten
�berlieferung nicht erw�hnt wird, und da� er zweitens bald wieder
rehabilitiert wurde (Dok. \ref{ch:29}~= Urk. 29; Dok. \ref{ch:30}~=
Urk. 30), aber noch verstarb, bevor dies umgesetzt werden konnte. Das
geschah wohl schon vor der Wahl des Athanasius zum Bischof, da es bei
ihm nur noch um die Aufnahme derer ">um Arius"< ging. Ferner ist das
kaiserliche Edikt gegen Arius (Dok. \ref{ch:33}~= Urk. 33) gewi�
fr�her, und zwar kurz nach der Synode von Nicaea 325 zu datieren, und
viel spricht daf�r, den Brief des Arius, der dem Kaiserbrief
Dok. \ref{ch:34}~= Urk. 34 voranging, genau zwischen der synodalen und
kaiserlichen Verurteilung anzusetzen. Sp�ter, in den 30er Jahren, ging
es nur noch um die Wiedereingliederung der ehemaligen Anh�nger des
Arius (vgl. Dok. \ref{ch:Arius}; \ref{ch:Jerusalem335}).


\cleardoublepage
%%%%%%%%%%%%%%%%%%%%%%%%%%%%%%%%%%%%%%%%%%%%
%%%%%%%%%%%%%%%%% HAUPTTEXT %%%%%%%%%%%%%%%%%%
%%%%%%%%%%%%%%%%%%%%%%%%%%%%%%%%%%%%%%%%%%%%
\mainmatter
\aliaspagestyle{chapter}{dokumente}
%% �bersetzung der in den bereits erschienenen Faszikeln edierten Texte %%%%%%%%%%%
\setcounter{page}{77}
\input{6t.tex}
\input{4t.tex}
% \section{Regest eines Synodalschreibens einer Synode in Bithynien}
\kapitel{Regest eines Rundbriefes einer Synode in Bithynien (Urk. 5)}
\label{ch:5}

Und sie versammelten eine Synode in Bithynien und schrieben an die Bisch�fe �berall, da� sie mit
denen um Arius, da sie rechtgl�ubig seien, Gemeinschaft halten sollten, ferner da� sie auch Alexander
dazu bringen sollten, mit ihnen Gemeinschaft zu halten.
\clearpage
\input{8t.tex}
\input{9t.tex}
% \section{Brief des Presbyters Georgius an Alexander von Alexandrien}
\kapitel{Fragment eines Briefes des Presbyters Georgius an Alexander von Alexandrien (Urk.~12)}
\label{ch:12}

\begin{footnotesize}
Georg aber, der jetzt in Laodicea ist, damals Presbyter in Alexandrien war und sich in Antiochien
aufhielt, schrieb an den Bischof Alexander:
\end{footnotesize}

Tadele nicht Arius und seine Mitstreiter, wenn sie sagen, es war einmal, da� der Sohn Gottes nicht war; denn auch
Jesaja wurde Sohn des Amos, und doch war Amos, bevor Jesaja wurde, Jesaja aber war nicht vorher,
sondern wurde erst danach.
\clearpage
% \section{Brief des Presbyters Georgius an die Arianer in Alexandrien}
\kapitel{Fragment eines Briefes des Presbyters Georgius an die ">Arianer"< in Alexandrien (Urk.~13)}
\label{ch:13}
\thispagestyle{empty}

\begin{footnotesize}
Er schrieb aber an die Arianer:
\end{footnotesize}

Was tadelt ihr den Papas Alexander, wenn er sagt, der Sohn ist aus dem Vater? Denn auch ihr f�rchtet
euch nicht zu sagen, da� der Sohn auch aus Gott ist. Denn wenn der Apostel schrieb: ">Alles ist aus
Gott"<, und offensichtlich ist, da� alles aus nichts gemacht wurde, dann ist auch der Sohn
Gesch�pf und eines der gemachten Dinge. Also d�rfte doch wohl gesagt werden, da� der Sohn so aus
Gott ist, wie auch von allen Dingen gesagt wird, da� sie aus Gott sind.
% \section{Regest des Synodalschreibens der Synode in Pal�stina}
\kapitel{Regeste eines Briefes (?) des Arius an Paulinus von Tyrus, Eusebius von Caesarea und an Patrophilus von Scythopolis und eines Briefes einer Synode in Palaestina (Urk.~10)}
\thispagestyle{empty}
\label{ch:10}

Als ihnen aber wider Erwarten ihre Bem�hungen nichts einbrachten, da
Alexander nicht nachgab, da wandte sich Arius an Paulinus, den Bischof von Tyrus, an
Eusebius, den Sohn des Pamphilus, der der Kirche von Caesarea in Palaestina vorstand, und an
Patrophilus von Scythopolis mit der Forderung, mit seinen Mitstreitern die Erlaubnis zu bekommen,
mit der auf seiner Seite stehenden Gemeinde Gottesdienst abzuhalten und dabei wie fr�her den Rang
von Presbytern einzunehmen. Es sei n�mlich Sitte in Alexandrien (wie es auch jetzt noch
ist), da�, w�hrend ein Bischof �ber allen stehe, die Presbyter f�r sich allein den
Kirchen vorstehen und die Gemeinden in ihnen versammeln. 
Sie aber kamen in Palaestina mit anderen Bisch�fen zusammen und bef�rworteten die Bitte des Arius
und geboten, sich wie fr�her zu versammeln, aber Alexander unterzuordnen und ihn zu bitten,
best�ndig am Frieden und der Gemeinschaft mit ihm teilzuhaben.
\input{7t.tex}
% \section{Brief des Euseb von Caesarea an Euphration von Balane�
\kapitel{Fragmente eines Briefes des Eusebius von Caesarea an Euphration von Balaneae (Urk. 3)}
\label{ch:3}

\kapnum{1}(Brief) des Eusebius, des Pamphilos' Sohn, an Euphration, dessen Anfang (lautet): Meinem Herrn
bekenne ich nach aller Gnade.

\begin{footnotesize}
Und nach weiteren Worten:
\end{footnotesize}

Denn wir sagen nicht, da� der Sohn zusammen mit dem Vater existiert, sondern da� der Vater vor dem
Sohn existiert.
Denn falls sie zusammen existierten, wie kann da der Vater Vater und der Sohn Sohn sein? Oder wie
ist der eine Erster, der andere aber Zweiter? Und der eine ungezeugt, der andere aber gezeugt? Denn
zwei, die einander genau gleich sind und zusammen existieren, d�rften der gleichen Ehre wert gehalten werden und sind
entweder beide, wie ich sagte, ungezeugt oder beide gezeugt. Doch keines von beiden ist wahr; denn
weder das Ungezeugte noch das Gezeugte d�rften beide zugleich existieren, sondern das eine wird f�r das Erste und
f�r st�rker sowohl der Ordnung als auch der Ehre nach als das Zweite gehalten, da es ja f�r das Zweite
auch zur Ursache sowohl f�r das Sein als auch das So-Sein wurde.

\kapnum{2}Abgesehen (davon) hat der Sohn Gottes selbst, der besser als alle genau sich auskennt und
wei�, da� er selbst ein anderer als der Vater und geringer und hervorgegangen ist, ganz und gar
fromm dies auch uns gelehrt und gesagt: ">Der Vater, der mich gesandt hat, ist gr��er als ich."<

\begin{footnotesize}
\kapnum{3}Und es wurde aus demselben Brief vorgelesen:
\end{footnotesize}

Er lehrt, da� derselbe auch der einzig Wahrhaftige ist, indem er sagt: ">damit sie dich als den
einzig wahrhaftigen Gott erkennen"<, nicht so, als ob Gott allein einer w�re, sondern als ob nur ein
einzig wahrhaftiger Gott existierte mit der absolut notwendigen Hinzuf�gung des \frq wahrhaftig\flq.
Denn der Sohn ist auch selbst Gott, aber nicht wahrhaftiger Gott. Denn ein einziger ist auch einzig
wahrhaftiger Gott, weil er niemanden vor sich hat. Denn wenn auch der Sohn selbst wahrhaftiger
(Gott) ist, dann ist er doch wohl Gott wie ein Abbild des wahrhaftigen Gottes, da ">das
Wort auch Gott war"<, freilich nicht so wie der einzige, wahrhaftige Gott.

\begin{footnotesize}
\kapnum{4}Er erdreistete sich, das Wort von Gott zu trennen und das Wort einen
anderen Gott zu nennen, der im Blick auf sein Wesen und die Macht vom Vater unterschieden
ist. Und in welch gro�e Blasphemie er dabei verfiel, kann man klar und leicht an
den von ihm geschriebenen Worten lernen.
Denn er hat mit eben diesen Worten geschrieben:
\end{footnotesize}

Freilich wird das Bild und das, dessen Abbild es ist, nicht f�r ein und dasselbe gehalten,
sondern (es sind) zwei Wesen, zwei Dinge und zwei M�chte, wie es auch so viele Bezeichnungen gibt. 

\begin{footnotesize}
\kapnum{5}Da er den Heiland nur als Menschen erweisen wollte, sagte er folgendes, als ob
er uns das gr��te, unsagbare Geheimnis des Apostels aufdeckte:
\end{footnotesize}

Deswegen hat ganz sicher auch der g�ttliche Apostel uns die unsagbare und mystische Lehre �ber Gott
gegeben und rief und schrie ">Einer ist Gott"<, und danach sagt er, nach dem einen
Gott ist ">ein Mittler Gottes und der Menschen, der Mensch Christus Jesus"<.
% \section{Brief des Athanasius von Anazarbos an Alexander von Alexandrien}
\kapitel{Fragment eines Briefes des Athanasius von Anazarba an Alexander von Alexandrien (Urk.~11)}
\label{ch:11}

\begin{footnotesize}
Er schrieb an den Bischof Alexander und wagte folgenderma�en zu reden:
\end{footnotesize}

\kapnum{1}Was tadelst du Arius und seine Mitstreiter, wenn sie sagen, der Sohn Gottes wurde
aus Nichts als ein Gesch�pf gemacht und ist eines von allen? Unter den hundert Schafen, mit denen alle
Gesch�pfe im Gleichnis verglichen werden, ist n�mlich eines von ihnen auch der Sohn.

\kapnum{2}Wenn also die Hundert keine Gesch�pfe und nicht geworden sind oder wenn es noch
etwas gibt jenseits dieser Hundert, dann ist klar, da� der Sohn weder ein Gesch�pf noch eines
wie alle anderen sein soll. Wenn aber alle Hundert geworden sind und es nichts jenseits der Hundert
gibt au�er Gott allein, was sagen Arius und seine Mitstreiter Verkehrtes, wenn sie ihn als einen
von den Hundert ansehen und Christus dazurechnen und sagen: Er ist einer von ihnen allen!
%% Dokument 35 %%
\kapitel{Fragmente eines Briefes (?) des Athanasius von Anazarba}
% \label{ch:35}
\label{ch:AthAnaz}
\begin{praefatio}
  \begin{description}
  \item[Vor Nicaea 325?]Die Datierung dieser Fragmente ist v�llig
    unklar. Falls sie zu dem Fragment eines Briefes des Athanasius von
    Anazarba\index[namen]{Athanasius!Bischof von Anazarba} an
    Alexander von Alexandrien\index[namen]{Alexander!Bischof von
      Alexandrien} geh�ren, das Athanasius von
    Alexandrien\index[namen]{Athanasius!Bischof von Alexandrien} in
    syn. 17 zitiert (= Dok \ref{ch:11} [Urk. 11]), fallen sie in die
    Zeit vor der Synode von Nicaea\index[synoden]{Nicaea!a. 325},
    d.\,h. ungef�hr in das Jahr 322. Sie k�nnten aber auch aus einer
    anderen Schrift stammen, da sich Athanasius von
    Anazarba\index[namen]{Athanasius!Bischof von Anazarba}
    offensichtlich mehrfach literarisch zu Thesen des
    Arius\index[namen]{Arius!Presbyter in Alexandrien} ge�u�ert hat
    und z.\,B. wahrscheinlich auch der Autor einer Homilie ist, die
    unter dem Namen des Athanasius von
    Alexandrien\index[namen]{Athanasius!Bischof von Alexandrien}
    �berliefert wurde (\cite{Tetz:Homilie}). Von ihm ist nur bekannt,
    da� er Bischof von Anazarba in Cilicia war und sp�ter Lehrer des
    A"etius\index[namen]{A"etius!Diakon in Alexandrien} wurde
    (Philost., h.\,e. III 15); daraus lassen sich jedoch keine
    chronologischen Hinweise ableiten.
  \item[�berlieferung]Die drei Fragmente sind im Codex Vaticanus
    latinus 5750 (p. 275) �berliefert, der zusammen mit Codex
    Ambrosianus E 147 sup. urspr�nglich einen Band mit 792 Bl�ttern
    bildete und aus der Bibliothek von Bobbio stammte. Es handelt sich
    um ein Palimpsest aus dem 7. Jh., bei dem zwei Handschriften mit
    der Korrespondenz zwischen Fronto und Marc Aurel und mit Scholien
    zu den Reden Ciceros f�r eine Abschrift der Akten der Synode von
    Chalcedon wiederverwendet wurden. Vor Ende des 7. Jh.s wurden etwa
    140 Seiten mit ">arianischen"< Texten �berschrieben. Gryson hat in
    seiner kritischen Edition (Scripta Arriana Latina, CChr.SL
    87,229-265) die anonym und titellos �berlieferten ">arianischen"<
    Fragmente neu sortiert und nach inhaltlichen Kriterien zwei
    voneinander zu unterscheidenden Schriften zugewiesen, die er
    ">Adversus Orthodoxos et Macedonianos"< (Fragment 1--12) und
    ">Instructio Verae Fidei"< (Fragment 13--23) nennt. F�r die erste
    Schrift, die Bez�ge zur sog. ersten sirmischen Formel zeigt, aber
    die Auseinandersetzungen um die Pneumatologie voraussetzt, nimmt
    er illyrische Herkunft an und datiert sie nach 380. Der Anonymus
    zitiert aus einem Text des Athanasius von
    Anazarba\index[namen]{Athanasius!Bischof von Anazarba}, der sich
    wiederum auf Dionys von Alexandrien\index[namen]{Dionysius!Bischof
      von Alexandrien} beruft (Frgm. 4 [\editioncite[235]{Gry},
    basierend auf \cite{DeBruyne}]).

    Die drei Athanasius von Anazarba zuzuschreibenden Fragmente sind
    �bersetzungen aus dem Griechischen, die dem griechischen Text Wort
    f�r Wort zu folgen scheinen.\footnoteA{Vgl. die R�ck�bersetzungen
      ins Griechische bei \cite[51]{Opitz:Dionys} und
      \cite[258]{Abramowski:Dionys}.} Seit ihrer Entdeckung wird das
    zweite Fragment als Zitat des Dionysius von Alexandrien
    diskutiert, wobei der �berlieferungszusammenhang trotz der
    theologischen Ablehnung einer Identifikationstheologie, die
    inhaltlich zu Dionysius passen w�rde, hier keine eindeutige
    Entscheidung zul��t. 
\item[Fundstelle]Codex Vaticanus lat. 5750, p. 275 (Frgm. 4 [\editioncite[235]{Gry}])
  \end{description}
\end{praefatio}
%
\begin{pairs}
\begin{Leftside}
\beginnumbering
\pstart
\noindent\kap{1}\edtext{\abb{}}{\killnumber\Cfootnote{\hskip -1em\latintext Cod. Vat. lat.
5750}}\specialindex{quellen}{chapter}{Codices!Vaticanus lat. 5750!p. 275}
\dots\ provisor, omnium iudex et
\edtext{despensator}{\Dfootnote{dispensator \textit{coni. Mai}}}, deus qui omnia creavit et construxit, qui
fecit omnia ex nihilo.
\pend
\pstart
\kap{2}\footnotesize{Iterum idem ipse Athanasius\edindex[namen]{Athanasius!Bischof von Anazarba}
antiquorum profert memoriam ac Dionisi\edindex[namen]{Dionysius!Bischof von Alexandrien}
episcopi, ut ostendat ante esse patrem quam filius generaretur,
dicens:}
\pend
\pstart
Ita pater quidem pater et non filius, non quia factus est, sed quia est, non ex aliquo,
sed in se permanens; filius autem et non pater, non quia erat, sed quia factus est, non de
se, sed ex eo qui eum fecit filii dignitatem sortitus est. \pend
\pstart
\kap{3}\footnotesize{Deinde ipse Athanasius:}\edindex[namen]{Athanasius!Bischof von Anazarba}
\pend
\pstart
Non enim se erigit filius contra patrem neque putat \edtext{\abb{\edtext{\abb{parem}}{\Dfootnote{\textit{coni. Bauer} paria \textit{cod.}}} esse cum
deo}}{\Afootnote{\latintext vgl. Phil 2,6}}\edindex[bibel]{Philipper!2,6|textit}, cedit
autem patri suo et fatetur docens omnes quia pater \edtext{maior}{\Dfootnote{\latintext
maior, maior \textit{coni. Gryson} \hskip 1ex maior se est, maior \textit{coni. de
Bruyne}}\lemma{\abb{maior}}\Afootnote{\latintext vgl. Io 14,28}}\edindex[bibel]{Johannes!14,28|textit},
autem non vastitate neque 
\edtext{\abb{magnitudine}}{\Dfootnote{\textit{coni. Mai} magnitudinem \textit{cod.}}}, quae quidem corporum propria sunt, sed perpetuitate et
innarrabili eius paterna ac generandi 
\edtext{\abb{virtute}}{\Dfootnote{\textit{coni. Mai} virtutem \textit{cod.}}}, et quia ipse quidem sempiternus est et in
se plenitudinem habens et a nullo vitam habens.
\pend
% \endnumbering
\end{Leftside}
\begin{Rightside}
\begin{translatio}
\beginnumbering
\pstart
\noindent\dots\ Gott, der Lenker, Richter und Verwalter von allem, der alles erschaffen und
errichtet, der alles aus nichts geschaffen hat.
\pend
\pstart
\footnotesize{Dann wiederum erinnert derselbe Athanasius an die Alten und an den Bischof Dionys,
um zu zeigen, da� der Vater war, bevor der Sohn gezeugt wurde, n�mlich:}
\pend
\pstart
So ist der Vater also Vater und nicht Sohn, nicht weil er gemacht ist, sondern weil er
existiert; er ist nicht aus einem anderen, sondern bleibt immer in sich; der Sohn aber ist
nicht Vater, nicht weil er (schon) war, sondern weil er gemacht worden ist; nicht aus sich
heraus, sondern aus jenem, der ihn gemacht hat, erhielt er die
Sohnesw�rde.
\pend
\pstart
\footnotesize{Hierauf derselbe Athanasius:}
\pend
\pstart
Der Sohn erhebt sich n�mlich nicht gegen den Vater, auch meint er nicht, Gott gleich zu
sein, vielmehr tritt er hinter seinem Vater zur�ck und bekennt und lehrt alle, da� der
Vater gr��er sei, aber nicht dem Umfang oder den Ausma�en nach, was doch Eigenschaften von
K�rpern sind, sondern nach der Ewigkeit und seiner unaussprechlichen v�terlichen und
zeugenden Kraft, und weil er ja selbst ewig ist und F�lle in sich selbst hat und von
niemandem das Leben bekommen hat.
\pend
\endnumbering
\end{translatio}
\end{Rightside}
\Columns
\end{pairs}
% \thispagestyle{empty}
%%% Local Variables: 
%%% mode: latex
%%% TeX-master: "dokumente_master"
%%% End: 

% DOKUMENT 36 %%%
%%%% Input-Datei OHNE TeX-Pr�ambel %%%%
\kapitel[Fragmente zweier Briefe des Theognis von Nicaea]{Fragmente zweier Briefe des
Theognis von Nicaea}
% \label{ch:38}
\label{ch:TheognisNiz}
\begin{praefatio}
  \begin{description}
  \item[Vor Nicaea 325?] Im Text bietet sich kein
    Anhaltspunkt f�r eine Datierung. Es kann jedoch vermutet werden,
    da� die Fragmente der beiden Briefe in die Zeit vor der Synode von
    Nicaea\index[synoden]{Nicaea!a. 325} 325 geh�ren, als viele Briefe
    pro und contra Arius\index[namen]{Arius!Presbyter in Alexandrien}
    und dessen theologische Ideen an Alexander von
    Alexandrien\index[namen]{Alexander!Bischof von Alexandrien}
    verschickt wurden (vgl. Dok. \ref{ch:7} = Urk. 7; Dok. \ref{ch:9} =
    Urk. 9; Dok. \ref{ch:11} = Urk. 11; Dok. \ref{ch:12} = Urk. 12 und
    die Einleitung zur Chronologie), da es sich beim Adressaten
    (\textit{ad papam}) wohl um den alexandrinischen Bischof
    handelt. �ber Theognis\index[namen]{Theognis!Bischof von Nicaea}
    wissen wir ferner, da� er sich auf der Synode von Nicaea geweigert
    hatte, die Verurteilung des Arius\index[namen]{Arius!Presbyter in
      Alexandrien} zu unterzeichnen (Soz., h.\,e. I 21,3;
    Dok. \ref{ch:27} = Urk. 27; Dok. \ref{ch:31} = Urk. 31), und
    daraufhin selbst verbannt wurde. An der Synode von
    Tyrus\index[synoden]{Tyrus!a. 335}, die zur Absetzung des
    Athanasius\index[namen]{Athanasius!Bischof von Anazarba} f�hrte,
    war er als Mitglied der Mareotis-Kommission
    (vgl. \ref{sec:BriefJuliusII},32; \ref{sec:SerdicaRundbrief},6;
    Soz., h.\,e. II 25,19) ma�geblich beteiligt.
  \item[�berlieferung]Vgl. Dok. \ref{ch:AthAnaz}. Die beiden
    Briefexzerpte sind ebenfalls �bersetzungen aus dem
    Griechischen. Folgt man der Konjektur von \cite{DeBruyne}, so sind
    sie Theognis von Nicaea\index[namen]{Theognis!Bischof von Nicaea}
    zuzuordnen. Der letzte Satz des dritten Exzerpts geh�rt gegen
    Gryson noch zum Zitat, weil der Satz ebenfalls auf den Beginn
    eines Werkes verweist und weil das Zitat wahrscheinlich am Ende
    des Fragments abbricht, da noch gar keine Aussage zu dem
    angesprochenen Thema getroffen worden ist.
  \item[Fundstelle]Codex Vaticanus lat. 5750, p. 275~f. (Frgm. 4
    [\editioncite[235~f.]{Gry}])
  \end{description}
\end{praefatio}
\begin{pairs}
\selectlanguage{latin}
\begin{Leftside}
% \beginnumbering
\pstart
\noindent\kap{1}\edtext{\abb{}}{\killnumber\Cfootnote{\hskip -1em Cod. Vat. lat.
5750}}\specialindex{quellen}{chapter}{Codices!Vaticanus lat. 5750!p. 275~f.}%
\footnotesize{Similiter etiam Bithenus episcopus
\edtext{\abb{Teognius}}{\lemma{\abb{\normalfont{Teognius}}}\Dfootnote{\textit{coni. de
Bruyne} \normalfont{et cognius} \textit{cod.} \normalfont{et cognitus} \textit{coni.
Mai}}}\edindex[namen]{Theognis!Bischof von Nicaea} ad papam}:
\pend
\pstart
ergo filium genitum dicimus,
filius autem ingenitus numquam fieri potest; solum autem patrem scientes ingenitum de
sanctis \edtext{\abb{scripturis}}{\Dfootnote{+ illum solum adoramus \textit{add. de
Bruyne}}}: veneramur autem filium, quia apud nos certum est hanc eius gloriam ad patrem
ascendere.
\pend
\pstart
\kap{2}\footnotesize{et post pauca idem}:
\pend
\pstart
cum ergo \edtext{maiorem se
\edtext{\abb{patre\Ladd{m}}}{\Dfootnote{\textit{coni. de Bruyne} patre \textit{cod.} pater
\textit{coni. Mai}}}}{\Afootnote{vgl. Io 14,28}}\edindex[bibel]{Johannes!14,28|textit} ostendat,
certum esse \edtext{\abb{quia}}{\Dfootnote{+ pater est deus \textit{add. de
Bruyne}}} non solum \edtext{\abb{operatione\ladd{m}}}{\Dfootnote{\textit{coni. Wintjes}
secundum operationem \textit{coni. Gryson} ob rationem \textit{susp. Mai}}} creaturae, sed
quia ingenitus est.
\pend
\pstart
\kap{3}\footnotesize{similiter idem ipse in alia epistula}:
\pend
\pstart
de patre autem et \edtext{filio}{\Dfootnote{fili \textit{cod.}}} \edtext{dicere}{\Dfootnote{diceris \textit{cod.} dicere iusta \textit{coni. de Bruyne}}}, sicut scis, super \edtext{\abb{nubeculam}}{\Dfootnote{\textit{coni. de Bruyne} baculum \textit{cod.}}} est ambulare. ergo vere primum \edtext{deprecabor}{\Dfootnote{deprecamur \textit{coni. Mai}}} dominum, ut veniam mihi tribuat propter necessitatem, et ita de his incipiam non de plurimis quaestionibus, neque per circuitum, sed per compendium.
\pend
\endnumbering
\end{Leftside}
\begin{Rightside}
\begin{translatio}
\beginnumbering
\pstart
\noindent\kapR{1}\footnotesize{Ebenso wendet sich auch der bithynische Bischof Theognis an den Papas:}
\pend
\pstart
Also nennen wir den Sohn gezeugt, einen ungezeugten Sohn dagegen kann es niemals geben.
Wir wissen indes aus den heiligen Schriften, da� der Vater allein ungezeugt ist: Den Sohn
aber verehren wir, weil f�r uns feststeht, da� seine Herrlichkeit zum Vater aufsteigt.
\pend
\pstart
\kapR{2}\footnotesize{Und wenig sp�ter sagt derselbe:}
\pend
\pstart
Wenn er also zeigt, da� der Vater gr��er
ist als er, so steht fest, da� er es nicht nur auf Grund des Hervorbringens der Sch�pfung
ist, sondern auch weil er ungezeugt ist.
\pend
\pstart
\kapR{3}\footnotesize{Ebenso sagt derselbe selbst in einem anderen Brief:}
\pend
\pstart
Aber �ber den Vater und den
Sohn zu sprechen hei�t, wie du wei�t, �ber Wolken zu gehen. Also werde ich wahrlich zuerst
den Herrn bitten, mir wegen der Notwendigkeit zu verzeihen, und so werde ich beginnen, �ber
diese Fragen, nicht �ber die vielen anderen zu schreiben, und nicht mit vielen Umschweifen,
sondern kurz und knapp.
\pend
\endnumbering
\end{translatio}
\end{Rightside}
\Columns
\end{pairs}
% \thispagestyle{empty}
%%% Local Variables: 
%%% mode: latex
%%% TeX-master: "dokumente_master"
%%% End: 

\kapitel{Fragmente des Tomus des Alexander von Alexandrien an alle Bisch�fe (Urk.~15)}
\label{ch:15}
\begin{center}
Von Alexander aus Alexandrien
\end{center}
\begin{footnotesize}
  Aus dem Tomus, der von dem Papa, dem Erzbischof von Alexandrien,
  geschrieben wurde �ber den rechten Glauben an die gottliebenden
  Bisch�fe an allen Orten, welchem auch die gottliebenden Bisch�fe,
  die an der Zahl etwa 200 waren, zugestimmt haben, indem sie die Hand
  hoben, da� sie (ihn) so ann�hmen. Dasselbe Schreiben wurde nun vor
  allem geschrieben gegen die Ruchlosigkeit des Arius und jener, die
  sich mit ihm absonderten, sodann aber �ber den katholischen Glauben,
  so wie es jedermann ziemt, ihn festzuhalten, und da� die heilige
  Jungfrau Gottesgeb�rerin sei.
\end{footnotesize}
 
\kapnum{1}Meinen Herrn und Amtsbruder, den Geliebten meiner Seele, Melitius, und die �brigen
Bisch�fe der katholischen Kirche gr��t Alexander in Gott.
 
\begin{footnotesize}Und nach dem Beginn:\end{footnotesize}
 
\kapnum{2}Und zum rechten Glauben, dem �ber den Vater und den Sohn, gem�� dem, was uns die
Schriften lehren: Den einen heiligen Geist bekennen wir und die eine katholische Kirche
und die Auferstehung der Toten, dessen Anfang unser Herr und Heiland Jesus Christus
gewesen ist, indem er Fleisch anzog von der Gottesgeb�rerin Maria, da� er zu dem
Geschlecht der Menschen komme, indem er starb, auferstand aus der Unterwelt und aufstieg
in den Himmel und sich setzte zur Rechten der Majest�t.
 
\kapnum{3}Dies habe ich zum Teil brief"|lich niedergelegt, wobei ich es unterlie�, alles einzelne davon genau aufzuschreiben, damit ihr nicht eure g�ttliche Last zu tragen verge�t.
Dies haben wir gelehrt. Dies haben wir verk�ndigt. Dies sind die apostolischen
Lehren der Kirche. Zu denen gerieten die Anh�nger des Arius und
des Achillas und die, die sich mit ihnen von der Kirche abgespalten haben, in Gegensatz; denn sie lehren Fremdes im Vergleich zu unserer rechten Lehre, gem�� dem seligen Paulus, der sagte: ">Wenn jemand euch etwas anderes verk�ndigt, als das, was ihr empfangen habt, der sei verflucht."<
 
\begin{footnotesize}Und nach anderem.\end{footnotesize}
 
\kapnum{4}Denn auch diesem Wort gem��~-- n�mlich jenes ">Im Anfang war das Wort"<,
verleugnen sie, und jenes: ">Christus ist Gottes Kraft und Gottes Weisheit"< oder ">Er ist
das Wort und die Weisheit des Vaters"<, lehren sie nicht. Oder da� Gott nicht (schon)
immer die Weisheit und das Wort gezeugt habe, glauben sie. Dieses aber auch (nur) zu
denken, kommt einer Seele zu, die ungl�ubig und fern ist von den J�ngern Christi.
 
\begin{footnotesize}Und nach der Unterschrift derjenigen von ganz Aegyptus und der Theba"is und
Libyas und der Pentapolis und von den oberen Orten mit denen von Palaestina und von
Arabia und von Achaia und von Thracia und vom Hellespont und von Asia und Caria und
Lycia und Lydia und Phrygia und Pamphylia und Galatia und Pisidia und von Pontus und
Polemoniacus, und von Cappadocia und von Armenia unterschrieb auch Philogonius, der
Bischof von Antiochia in Syria [und alle vom Osten, die gottliebenden Bisch�fe von
Mesopotamia und von Augusto-Euphratesia und von Cilicia und von Isauria und Phoenice].
\end{footnotesize}
 
\kapnum{5}Ich, Philogonius, der Bischof der katholischen Kirche von Antiochia, indem ich
den Glauben, der im Schreiben meines Herrn und Seelenfreundes Alexander ist, �beraus
lobe und mit ihm und mit der �bereinkunft der heiligen Ordnung derjenigen �bereinstimme,
welche eines Sinnes sind, unterschrieb und alle diejenigen, welche im Osten sind, was oben
geschrieben steht.
% erstellt von AvS 22.11.04
% \section{Brief des Arius an Eusebius von Nikomedien}
\kapitel{Brief des Arius an Eusebius von Nikomedien (Urk.~1)}
\label{ch:1}
\kapnum{1}Dem hei�ersehnten Herrn, dem treuen Menschen Gottes, dem rechtgl�ubigen Eusebius entbietet
Arius, der vom Papas Alexander ungerechterweise um der Wahrheit willen, die alles besiegt und 
die auch du mit dem Schild deckst, verfolgt wird, im Herrn seinen Gru�.

\kapnum{2}Als mein Vater Ammonius im Begriff war, nach Nikomedien zu reisen, schien es mir vern�nftig und geziemend zu sein, dich durch ihn zu gr��en, zugleich aber
auch die dir angeborene Liebe und Gesinnung, die du den Br�dern gegen�ber um Gott und seines Christus willen hegst, daran zu erinnern, da� der Bischof uns arg zusetzt, verfolgt und alle
Hebel gegen uns in Bewegung setzt, so da� er uns aus der Stadt vertreibt wie gottlose Menschen, weil
wir nicht mit ihm �bereinstimmen, wenn er �ffentlich sagt: ">Immer Gott immer Sohn, zugleich Vater
zugleich Sohn, der Sohn existiert ungezeugt zusammen mit dem Vater, ewig-gezeugt, ungezeugt
geworden, weder durch einen Gedanken noch durch irgendeinen kleinsten Moment geht Gott dem
Sohn voran, ewig Gott ewig Sohn, aus Gott selbst ist der Sohn."<

\kapnum{3}Und da ja dein Bruder Eusebius in Caesarea, Theodotus, Paulinus, Athanasius, Gregorius,
A"etius und alle im Osten sagen, da� Gott vor dem Sohn ohne Anfang existiert, wurden sie
ausgeschlossen, abgesehen von Philogonius, Hellanicus und Macarius, h�retischen Menschen, die nicht einmal Taufunterricht empfangen haben, von denen die einen den Sohn ">R�lpser"< nennen, die anderen
">Hervorgeworfenes"<, wieder andere ihn aber zusammen mit dem Vater ungezeugt nennen.

\kapnum{4}Und diese Gottlosigkeiten k�nnten wir nicht einmal dann h�ren, wenn uns die H�retiker
unz�hlige Tode androhten. Wir aber, was sagen, denken, lehrten und lehren wir auch heute? Da� der Sohn nicht
ungezeugt und nicht Teil eines Ungezeugten auf irgendeine Art und Weise ist, noch da� er aus
irgendetwas Existierendem entstanden ist, sondern auf Grund von Wollen und Wunsch vor
Zeiten und Ewigkeiten, voll der Gnade und der Wahrheit, Gott, Eingeborener,
unver�nderlich.

\kapnum{5}Und bevor er gezeugt, geschaffen, beschlossen oder gegr�ndet wurde, war er nicht.
Denn er war nicht ungezeugt. Wir werden aber verfolgt, weil wir sagen, der Sohn hat einen Ursprung,
Gott aber ist ursprungslos. Deswegen werden wir verfolgt, weil wir auch sagen, er ist aus
nichts. Wir haben aber so gesprochen, weil er kein Teil Gottes ist und weil
er nicht aus irgendetwas Existierendem stammt. Deswegen werden wir verfolgt, das �brige wei�t 
Du.

Ich bete, da� es dir im Herrn wohl ergehen m�ge -- eingedenk unserer Betr�bnisse --, mein
Mit-Lukianist, wahrhaftig ">Frommer"<\footnote{Hier liegt ein Wortspiel mit dem Namen
Eusebius vor.}.
% erstellt von AvS 22.11.04
% \section{Brief Eusebius' von Nikomedien an Arius}
\kapitel{Fragment eines Briefes des Eusebius von Nikomedien an Arius (Urk.~2)}
\label{ch:2}

\begin{footnotesize}
Und Eusebius von Nikomedien schrieb �berfl�ssigerweise an Arius:
\end{footnotesize}

Da du recht denkst, bete, da� alle so denken! Denn einem jeden ist offensichtlich, da� das, was
gemacht worden ist, nicht existierte, bevor es wurde. Und das Gewordene hat einen Anfang
seines Seins.
% \section{Brief Alexanders von Alexandrien an Alexander von Byzanz}
\kapitel{Brief des Alexander von Alexandrien an Alexander von Byzanz (Urk.~14)}
\label{ch:14}

Den hochverehrten und gleichgesinnten Bruder Alexander\footnote{Alexander von Byzanz
und nicht Alexander von Thessalonike. Vgl. unten Dok. \ref{ch:18} und
\cite{Winkelmann:Byzanz}.} gr��t Alexander im Herrn!

\kapnum{1}Das herrschs�chtige und habgierige Ansinnen schlechter Menschen, die unter vielerlei derartigen
Vorw�nden die kirchliche Fr�mmigkeit angreifen, trachtet von Natur aus stets nach Di�zesen, die
gr��er zu sein scheinen. Angestachelt n�mlich von dem in ihnen wirksamen Teufel legen sie alle
Fr�mmigkeit ab, wenden sich zu dem vor ihnen liegenden Vergn�gen und treten dabei die Furcht vor dem Gericht
Gottes mit F��en.

\kapnum{2}�berdies sehe ich mich, der ich darunter zu leiden habe, gezwungen, eure Fr�mmigkeit zu
informieren, damit ihr euch vor derartigen Leuten in Acht nehmt, auf da� nicht irgendjemand von
ihnen es wagt, auch in euer Gebiet einzudringen, entweder pers�nlich (denn die Zauberer sind in der
Lage, bis hin zum Betrug zu heucheln) oder durch l�gnerisch herausgeputzte Briefe, durch die sie den vereinnahmen verm�gen, der sie mit einfachem und arglosem Glauben annimmt.

\kapnum{3}Arius und Achilleus n�mlich, seit kurzem zusammengerottet, haben herrschs�chtig wie
Colluthus gehandelt und dabei noch viel schlimmer als jener gew�tet. Jener beschwerte sich n�mlich
�ber gerade diese und nahm sie als Vorwand f�r seine eigene verwerf"|liche Absicht; diese aber sahen
sich, als sie dessen Handel mit Christus bemerkten, au�erstande, der Kirche noch l�nger gehorsam
zu bleiben, sondern bauten sich R�uberh�hlen, veranstalteten unabl�ssig Versammlungen in ihnen
und studierten Tag und Nacht Verleumdungen gegen Christus und gegen uns ein.

\kapnum{4}Indem sie die ganze gottesf�rchtige apostolische Lehre verklagten, suchten sie sich nach
j�discher Art eine christusbek�mpfende Rotte zusammen, verleugneten die Gottheit unseres Erl�sers
und verk�ndeten, er sei allen anderen gleich, und indem sie alle Aussagen �ber seine heilsame
Menschwerdung und f�r uns veranstaltete Erniedrigung ausw�hlen, versuchen sie daraus ihre gottlose Verk�ndigung zusammenzustellen, w�hrend sie die Worte �ber seine
anf�ngliche Gottheit und seine unaussprechliche Herrlichkeit beim Vater beiseiteschieben.

\kapnum{5}So verhelfen sie also der gottlosen Meinung der Heiden und Juden �ber Christus zur Macht
und rennen soviel wie m�glich dem Lob von diesen Leuten hinterher, indem sie n�mlich alles genau so
veranstalten, wie es bei jenen an uns verspottet wird, und ferner t�glich Aufst�nde und Verfolgungen
gegen uns anzetteln, au�erdem Gerichtsversammlungen zusammentrommeln aufgrund von Antr�gen
zuchtloser Frauen, die sie get�uscht haben, und auch noch das Christentum verspotten, da die sie
umgebenden jungen Frauen ohne Anstand auf allen Stra�en umherlaufen. Sogar auch das rei�feste Gewand
Christi, welches die Henker nicht zertrennen wollten, wagen sie zu zerteilen.

\kapnum{6}Wir aber haben sie nun, auch wenn wir wegen der Heimlichtuerei erst sp�t davon erfahren
haben, entsprechend ihrer Lebensf�hrung und ihrem unsittlichem Ansinnen einstimmig aus der Kirche, die die Gottheit Christi anbetet, ausgeschlossen.

\kapnum{7}Sie aber heizten die Stimmung gegen uns an, versuchten, sich an die gleichgesinnten
Mitdiener zu wenden, wobei sie zwar nach au�en hin eine Bitte um Frieden und Gemeinschaft
vort�uschten, in Wahrheit aber danach trachteten, einige von ihnen durch Sch�nf�rberei in ihre
eigene Krankheit zu verstricken, und sie baten sie um geschw�tzige Briefe, um sie den von ihnen
Get�uschten vorzutragen und so die Unverbesserlichen, die zur Gottlosigkeit verleitet wurden, dahin zu ziehen, wohin sie sie mit ihrer T�uschung bringen wollten, als h�tten sie Bisch�fe als Unterst�tzer und
Gleichgesinnte.

\kapnum{8}Nat�rlich geben sie vor ihnen nicht zu, was sie alles B�se bei uns gelehrt und
getan haben, weshalb sie ja auch ausgeschlossen wurden, sondern sie gehen dar�ber entweder mit
Schweigen hinweg oder vertuschen und verdunkeln es mit wirrem Gerede und Geschreibe.

\kapnum{9}Weil sie also ihre verderbliche Lehre hinter gewinnenden und schmeichlerischen Reden
verstecken, vereinnahmen sie den, der Betr�gereien auf"|liegt, und enthalten sich sogar nicht,
unsere Fr�mmigkeit bei allen zu verleumden. Daher passiert es sogar, da� einige ihre Briefe
unterschreiben und sie in die Kirche aufnehmen, wobei meiner Ansicht nach ein ganz schlechtes
Licht auf die Mitdiener f�llt, die derartiges wagen, nicht nur weil die apostolische Weisung dies
nicht zul��t, sondern weil sie auch die teuflische Stimmung gegen Christus bei jenen noch weiter
anheizen.

\kapnum{10}Deswegen will ich nun auch nicht mehr warten, Geliebte, und beginne daher, euch
den Unglauben dieser Leute vorzustellen, die sagen, da� es einmal war, als der Sohn Gottes nicht war, und
da� der, der zuvor nicht war, sp�ter wurde, und da� er, als er wurde, wann auch immer das war, aber
ein solcher wurde, wie auch jeder Mensch ist.

\kapnum{11}">Alles n�mlich"<, sagen sie, ">hat Gott aus nichts gemacht"<, wobei sie der Erschaffung
aller vern�nftigen und unvern�nftigen Wesen auch den Sohn Gottes einverleiben. Folglich sagen sie
auch, er habe eine wandelbare Natur, zu Tugendhaftem und auch Schlechtem in der Lage, und zusammen
mit der These ">aus nichts"< �bergehen sie die g�ttlichen Texte �ber sein immerw�hrendes Sein, die das
Unwandelbare des Wortes und die Gottheit der Weisheit des Wortes zeigen, was Christus ist. ">Auch
wir n�mlich"<, sagen die Halunken, ">k�nnen Gottes S�hne werden wie auch jener."< Denn es steht
dort: ">S�hne habe ich gezeugt und erh�ht."<

\kapnum{12}Werden sie aber auf den darauf"|folgenden Satz hingewiesen, der besagt: ">Sie haben mich
aber verworfen"<, was doch unnat�rlich ist f�r den Erl�ser, der eine unwandelbare Natur
besitzt, so lassen sie alle Scheu fallen und sagen, Gott habe aufgrund seines Vorherwissens und
seiner Vorausschau von ihm gewu�t, da� er ihn nicht verwerfen werde, und ihn deshalb vor allen
auserw�hlt.

\kapnum{13}Denn nicht von Natur aus oder weil er irgendeinen Vorzug vor den anderen S�hnen habe
(denn weder von Natur aus, so sagen sie, sei jemand Gottes Sohn, noch weil jemand irgendein
besonderes Verh�ltnis zu ihm habe), habe Gott ihn erw�hlt, sondern weil er, obwohl auch er eine
wandelbare Natur bes��e, sich dennoch nicht dem Schlechten
zugewandt habe, da er mit sittlichen Anstrengungen f�r seinen Lebenswandel Sorge tr�ge.

\kapnum{14}So w�rde sich n�mlich die Sohnschaft von Paulus und Petrus, wenn auch sie dasselbe erreicht
h�tten, nicht von seiner Sohnschaft unterscheiden. Zur Unterst�tzung f�r diese wahnsinnige
Lehre beleidigen sie die Schriften und verweisen auf die Aussage �ber Christus in den Psalmen, die
so lautet: ">Du hast die Gerechtigkeit geliebt und das Unrecht geha�t. Deswegen hat dich Gott, dein
Gott, mit dem �l der Freude vor deinen Mitstreitern gesalbt."<

\kapnum{15}Der Evangelist Johannes erkl�rt ausreichend, da� der Sohn Gottes weder aus nichts wurde
noch einmal nicht war, denn er schreibt folgenderma�en �ber ihn: ">der eingeborene Sohn, der im
Scho� des Vaters ist."< Weil n�mlich der g�ttliche Lehrer im Sinn hatte, auf zwei voneinander
ungetrennte Dinge, den Vater und den Sohn, hinzuweisen, nannte er ihn im Scho� des Vaters seiend.

\kapnum{16}Weil aber das Wort Gottes nicht zu den aus nichts Gewordenen hinzuzurechnen ist, sagt
derselbe Johannes, da� alles durch ihn geworden ist. Seine ganz besondere Hypostase erl�uterte er
mit den Worten: ">Im Anfang war das Wort, und das Wort war bei Gott, und Gott war das Wort. Alles
wurde durch es, und ohne es wurde nicht eines."<

\kapnum{17}Wenn n�mlich alles durch es wurde, wie sollte der, der den Gewordenen das Sein
verleiht, selbst einmal nicht gewesen sein? Es wird n�mlich nicht festgelegt, wie das schaffende Wort
dieselbe Natur hat wie die gewordenen Dinge, wenn doch er selbst am Anfang war, alles aber durch ihn
wurde [und er alles aus nichts machte].

\kapnum{18}Denn das Seiende scheint doch das Gegenteil von dem aus Nicht-Seienden zu sein und
ist davon sehr weit entfernt. Die eine Stelle belegt n�mlich, da� zwischen Vater und Sohn
kein Abstand ist, wobei freilich kein Mensch diese Vorstellung zu einer begriff"|lichen Definition bringen kann. Da� die Welt aus nichts geschaffen wurde, beinhaltet andererseits, da� nur sie eine j�ngere
Seinsweise und sp�tere Entstehung hat, da alles sein besonderes Wesen vom Vater durch den Sohn
empfangen hat.

\kapnum{19}Da also der so fromme Johannes das Sein des g�ttlichen Wortes als erhaben und die
Vorstellungskraft von uns Gewordenen �bersteigend ansah, vermied er es, sein Werden und Entstehen
zu beschreiben, er wagte es nicht einmal, das Schaffen mit gleichwertigen Worten zu benennen wie das Gewordene, nicht weil er (der Sohn) ungezeugt sei~-- denn der eine Ungezeugte ist der Vater~--,
sondern weil die nicht zu beschreibende Hypostase des eingeborenen Gottes jenseits der gesch�rften Wahrnehmung der Evangelisten, wahrscheinlich sogar auch der
Engel liegt. Ich meine nicht, da� man die, die es wagen, bis zu diesen Fragen vorzudringen, f�r
fromm halten sollte, da sie das Wort ">Was zu schwer ist f�r dich, untersuche nicht, und was zu hoch
f�r dich ist, erforsche nicht"< �berh�ren.

\kapnum{20}Denn wenn die Erkenntnis vieler anderer Dinge, und zwar unvergleichlich geringerer als
jenes, dem menschlichen Verstehen verborgen ist~-- wie bei Paulus: ">Was kein Auge gesehen und keiner
geh�rt hat und in kein menschliches Herz gekommen ist, was Gott denen, die ihn lieben, bereitet
hat"<, aber auch von den Sternen sagt Gott zu Abraham, da� er (Abraham) sie nicht z�hlen k�nne, und ferner
sagt er: ">Die Sandk�rner des Meeres und die Tropfen des Regens, wer wird sie z�hlen?"<~--, wie sollte
sich dann jemand �berfl�ssigerweise mit der Hypostase des g�ttlichen Wortes besch�ftigen, es sei
denn, er ist mit Schwermut belastet?

\kapnum{21}�ber diese Hypostase sagt der prophetische Geist: ">Ihre Entstehung, wer wird sie
ergr�nden?"< Auch unser Erl�ser selbst erwies diesbez�glich den S�ulen von allem in der Welt eine
Wohltat und sorgte daf�r, die Erkenntnis hier�ber von ihnen fernzuhalten, indem er n�mlich sagte,
da� ein Verstehen f�r sie alle unnat�rlich sei, allein dem Vater geb�hre ein Wissen �ber dieses rein
g�ttliche Geheimnis; ">denn niemand wei�, wer der Sohn ist"<, sagt er, ">au�er der Vater. Und
niemand kennt den Vater au�er der Sohn."< Davon handelt auch, meine ich, der Satz des Vaters: ">Mein
Geheimnis geh�rt mir."<

\kapnum{22}Da� es aber unangebracht ist, vom Sohn zu denken, er sei aus nichts entstanden, wird aus
dem ">aus nichts"< selbst erwiesen, da es eine zeitliche Vorstellung impliziert, auch wenn die
Unverst�ndigen das Verr�ckte ihrer Meinung nicht erkennen. Denn dieses ">er war nicht"< mu� doch
einem zeitlichen Rahmen angeh�ren oder irgendeinem Zwischenraum innerhalb der �onen.

\kapnum{23}Wenn es also wahr ist, da� alles durch ihn wurde, dann ist deutlich, da� auch jeder
Zeitraum, Zeitspanne und Zeitabschnitt und das ">einmal"<, in denen man das ">er war nicht"<
ansetzen k�nnte, durch ihn geworden ist; und ist es etwa nicht v�llig unglaubw�rdig zu sagen, da�
der, der die Zeiten und �onen und Zeitpunkte, in die das ">er war nicht"< einzuf�gen ist, gemacht
hat, einmal nicht gewesen ist? Es ist n�mlich undenkbar und v�llig unverst�ndlich zu sagen, da� der,
der Ursache f�r eine Sache ist, selbst erst nach der Entstehung jener Sache geworden sei.

\kapnum{24}F�r sie geht n�mlich der Weisheit Gottes, die das All erschaffen hat, jener
Zeitabschnitt voraus, in welchem, wie sie sagen, der Sohn noch nicht durch den Vater geworden war,
wodurch bei ihnen die Schrift l�gt, wenn sie ihn als Erstgeborenen vor aller Sch�pfung
bezeichnet.

\kapnum{25}Au�erdem stimmt diesem auch der redegewandte Paulus zu, der �ber ihn sagt: ">den er zum
Erben von allem eingesetzt hat, durch den er auch die �onen gemacht hat"<, aber auch: ">in ihm ist
alles geschaffen worden, das in den Himmeln und das auf Erden, das Sichtbare und das Unsichtbare,
seien es M�chte, Gewalten, Herrschaften oder Throne; alles ist durch ihn und zu ihm hin geschaffen
worden; und er selbst ist vor allem."<

\kapnum{26}Da also nun offensichtlich ist, da� die Ansicht, er wurde aus nichts, absolut gottlos
ist, so mu� der Vater immer Vater sein; er ist aber immer Vater, da der Sohn da ist, weswegen er
Vater genannt wird. Da aber der Sohn immer bei ihm ist, ist er immer vollkommener Vater, ohne Fehl in seiner G�te, da er den eingeborenen Sohn nicht zeitlich, auch nicht in einem
Zeitabschnitt noch auch aus nichts gezeugt hat.

\kapnum{27}Und jetzt? Ist es etwa nicht gottlos zu sagen, da� die Weisheit Gottes einmal nicht war,
die sagt: ">Ich war bei ihm und hielt Ordnung, ich war es, an der er sich erfreute"<, oder da� die
Kraft Gottes einmal nicht dagewesen ist, oder da� sein Wort einmal verst�mmelt gewesen sei, oder die
anderen Bezeichnungen, aus denen der Sohn erkannt und der Vater beschrieben wird? Denn zu sagen, da�
der Abglanz der Herrlichkeit nicht ist, l�scht auch die urspr�ngliche Lichtquelle, dessen Abglanz er
ist. Ferner, wenn auch das Bild Gottes nicht immer da war, ist deutlich, da� dann auch der nicht immer da
ist, dessen Bild er ist.

\kapnum{28}Au�erdem, wenn der Abdruck der Hypostase Gottes nicht da ist, wird zugleich damit auch
jener beseitigt, der in ihm vollkommen zum Abdruck kommt. Daraus ist es m�glich zu erkennen, da� die
Sohnschaft unseres Erl�sers keine Gemeinsamkeit mit der Sohnschaft der �brigen hat.

\kapnum{29}Auf dieselbe Art, wie erwiesenerma�en seine unaussprechliche Hypostase alle Dinge, denen
er das Sein verliehen hat, unvergleichlich �bertrifft, so unterscheidet sich auch seine Sohnschaft,
die auf nat�rliche Art der v�terlichen Gottheit entspringt, in unaussprechlichem �berma� von denen,
die durch ihn per Adoption zu S�hnen gemacht worden sind. Er hat n�mlich eine unwandelbare Natur,
die vollkommen und in jeder Hinsicht bed�rfnislos ist; sie aber ben�tigen dessen Hilfe, da sie
unstetem Wandel unterliegen.

\kapnum{30}Denn wie k�nnte wohl die Weisheit Gottes Fortschritte machen, oder wie k�nnte die
Wahrheit selbst noch zunehmen? Oder wie k�nnte sich wohl das Wort Gottes oder das Leben oder das
wahre Licht verbessern? Wenn dies aber so ist, dann ist es doch wohl erst recht unnat�rlich, wenn
die Weisheit zur Torheit f�hig w�re oder die Kraft Gottes zur Schwachheit neigen w�rde oder die
Vernunft durch Unvern�nftiges geschw�cht w�re oder wenn sich Finsternis mit dem wahren Licht
vermischt h�tte, da doch der Apostel selbst sagt: ">Was hat das Licht mit der Finsternis gemein, oder
was ist die Gemeinsamkeit zwischen Christus und Beliar?"<, und Salomo, da� es unm�glich sei, auch
nur in Gedanken die Wege der Schlange auf einem Felsen herauszufinden, der aber, nach Paulus,
Christus ist! Die Menschen und die Engel aber, seine Gesch�pfe, haben Gaben empfangen, durch das �ben von Tugend und durch die Beobachtung des Gesetzes fortzuschreiten und nicht mehr zu s�ndigen.

\kapnum{31}Daher wird doch unser Herr, der von Natur aus Sohn des Vaters ist, von allen angebetet.
Die aber, die den Geist der Knechtschaft abgelegt und nach Anstrengungen und Fortschritten den Geist
der Sohnschaft empfangen haben, werden die Hilfe dessen, der von Natur aus Sohn ist,  empfangen und durch Adoption auch selbst S�hne werden.

\kapnum{32}Seine echte, einzigartige, nat�rliche und auserw�hlte Sohnschaft hat Paulus
folgenderma�en dargestellt, als er �ber Gott redete: ">der aber seinen eigenen Sohn nicht verschont,
sondern ihn f�r uns offensichtlich nicht nat�rliche S�hne hingegeben hat."<

\kapnum{33}Denn im Gegensatz zu den nicht eigenen S�hnen sagte er, da� er der eigene Sohn sei. Und
im Evangelium steht: ">Dies ist mein geliebter Sohn, an dem ich Wohlgefallen habe."< Ferner sagt der
Erl�ser in den Psalmen: ">Der Herr sprach zu mir: Du bist mein Sohn!"< Indem er die Echtheit der
Sohnschaft vor Augen stellt, weist er darauf hin, da� er nicht noch andere echte S�hne neben diesem
hat.

\kapnum{34}Und was ist mit der Stelle ">Aus dem Scho� habe ich dich vor dem Morgenstern gezeugt"<?
Zeigt sie nicht geradewegs die nat�rliche Sohnschaft aus v�terlicher Entbindung, welche der Sohn
nicht durch �bung und eifriges Verbessern erreicht hat, sondern durch seine nat�rliche
Besonderheit? Daher hat der eingeborene Sohn des Vaters auch eine unver�nderliche Sohnschaft. Das
Wort wei� aber, da� die Sohnschaft der Vernunftwesen diesen nicht der Natur nach zukommt, sondern
aufgrund von Anstrengungen und der Gnadengabe Gottes ver�nderlich ist; ">denn als die S�hne Gottes
die T�chter der Menschen sahen, nahmen sie sich davon Frauen"< und so fort; und ">S�hne habe ich
gezeugt und erh�ht, sie aber haben mich verworfen"<, so werden wir von Jesaja belehrt, da� Gott
gesprochen habe.

\kapnum{35}Ich k�nnte, Geliebte, noch viele Aspekte erg�nzen, lasse sie aber weg, da ich meine, da�
es unangebracht ist, gleichgesinnten Lehrern so viel Bekanntes vor Augen zu halten. Denn ihr seid
selbst von Gott Belehrte und �berseht gewi� nicht, da� die k�rzlich innerhalb der kirchlichen
Fr�mmigkeit aufgekommene Lehre von Ebion und Artemas stammt und eine Kopie von Paulus von Samosata
aus Antiochien ist, der durch eine Synode und das Urteil von Bisch�fen, die von �berall her zusammengekommen waren, aus der Kirche ausgeschlossen wurde.

\kapnum{36}Und Lucian, der diesem nachfolgte, blieb w�hrend der vielen Jahre dreier Bisch�fe von
Versammlungen ausgeschlossen. Die aber, die jetzt unter uns ">aus nichts"< verbreiten, haben an dem
Most der H�resie jener geschl�rft und sind verborgene Spr��linge von jenen, und zwar Arius, Achillas
und die Gruppe, die mit ihnen zusammen �bel handelt.

\kapnum{37}Und ich wei� nicht, wie in Syria drei geweihte Bisch�fe diese durch ihre Zustimmung zu
noch schlimmeren Taten anstacheln; �ber diese soll das Urteil eurer Einsch�tzung unterliegen. Sie
haben die Schriftstellen �ber das erl�sende Leiden, die Erniedrigung und Ent�u�erung, �ber seine
sogenannte Armut und �ber die Dinge, die der Erl�ser f�r uns erst angenommen hat, im Kopf und legen sie vor, um gegen seine h�chste und von Anfang an bestehende Gottheit Einw�nde zu erheben, aber die Stellen �ber seine Herrlichkeit von Natur aus und seine edle Geburt und sein Verweilen beim Vater haben sie vergessen, wie zum Beispiel die Stelle ">ich und der Vater sind eins."<

\kapnum{38}Das aber sagt der Herr, nicht weil er sich selbst als Vater ausgeben will, auch nicht
weil er die der Hypostase nach zwei Naturen f�r eine erkl�ren will, sondern weil der Sohn des Vaters
die Gleichheit mit dem Vater genau auf nat�rliche Art bewahrt, indem er von Natur aus in jeder
Hinsicht eine Gleichheit mit ihm ausdr�ckt, ein unver�nderliches Bild des Vaters ist und ein vom
Urbild gepr�gtes Abbild.

\kapnum{39}Daher kl�rte der Herr bereitwillig den Philippus auf, als dieser ihn zu sehen verlangte, und
antwortete ihm, als dieser sagte ">Zeige mir den Vater!"<: ">Wer mich sieht, sieht den Vater"<, so
da� also der Vater wie durch einen unbefleckten und lebendigen Spiegel in seinem g�ttlichen Abbild
zu sehen ist.

\kapnum{40}Vergleichbares sagen die Heiligen in den Psalmen: ">In deinem Licht werden wir das Licht
schauen."< Deswegen ehrt der, der den Sohn ehrt, auch den Vater, und zwar zu Recht. Denn jedes
l�sterliche Wort, das gegen den Sohn zu �u�ern gewagt wird, bezieht sich auch auf den Vater.

\kapnum{41}Und was wundert das jetzt noch, was ich noch berichten mu�, Geliebte, wenn ich die
verlogenen Vorw�rfe gegen mich und gegen unser so gottesf�rchtiges Volk darlegen werde! Denn die,
die gegen die Gottheit des Gottessohnes zu Felde ziehen, lassen keine Gelegenheit aus, grobe Mi�t�ne
gegen uns anzustimmen. Sie aber halten es f�r unter ihrer W�rde, sich selbst mit irgendjemandem von den
Alten zu vergleichen, ertragen es auch nicht, den Lehrern, mit denen wir von klein auf Umgang
hatten, gleichgestellt zu werden, sondern sie halten keinen unter den Mitbisch�fen irgendwo auch nur
f�r m��ig weise, man findet sie sagen, da� sie allein weise seien und sich in der Lehre
auskennen und da� ihnen allein offenbart worden sei, was keinem anderen sonst unter der Sonne je in
den Sinn gekommen ist.

\kapnum{42}Oh, welch gottlose Blindheit und ma�lose Raserei, verbunden mit bodenloser,
tr�bsinniger Ruhmsucht und satanischer Gesinnung, die in ihren unseligen Seelen umhergeistert!

\kapnum{43}Es besch�mt sie nicht die gottliebende Klarheit der alten Schriften, auch der
�bereinstimmende Glaube der Mitdiener an Christus hat ihre Frechheit gegen ihn nicht verringert.
Sogar die D�monen, die sich zur�ckhalten, ein l�sterliches Wort gegen den Sohn Gottes zu sagen,
k�nnen deren Freveltaten nicht ertragen.

\kapnum{44}Dies also sei von uns nach m�glichen Kr�ften gegen die, die mit unerzogenem Gebell gegen
Christus anst�rmen und unseren frommen Glauben an ihn �ffentlich verleumden, bereitgestellt. Diese
Erfinder sinnlosen Geredes sagen n�mlich, da� wir, wenn wir uns von der gottlosen und schriftfernen
L�sterung gegen Christus ">aus nichts"< distanzieren, zwei Ungezeugte lehren, denn die Unbelehrten
sagen, da� nur eins von zweien zutreffen k�nne und er (der Sohn) entweder aus nichts zu denken oder
ansonsten von zwei Ungezeugten auszugehen sei. Die Unge�bten sind dar�ber im Unklaren, wie gro� der
Unterschied zwischen dem ungezeugten Vater und den von ihm aus nichts geschaffenen Dingen ist, sowohl
der vern�nftigen als auch der unvern�nftigen.

\kapnum{45}Und zwischen ihnen vermittelt die eingeborene Natur, durch welche der Vater des
Gott-Wortes das All aus nichts gemacht hat, und wird aus dem seienden Vater selbst gezeugt, wie es
der Herr selbst einmal bezeugte, als er sagte: ">Der, der den Vater liebt, liebt auch den Sohn, der
aus ihm gezeugt wurde."<

\kapnum{46}Unter diesen Voraussetzungen glauben wir so, wie es die apostolische Kirche f�r
richtig h�lt: an
einen allein ungezeugten Vater, der keine andere Ursache f�r sein Sein hat, unwandelbar und
unver�nderlich, der selbst immer ein und derselbe bleibt, der weder eine Verbesserung noch
eine Verschlechterung annimmt, den Urheber des Gesetzes, der Propheten und der Evangelien, Herr der
Patriarchen, der Apostel und aller Heiligen; und an einen Herrn Jesus Christus, Gottes eingeborenen
Sohn, gezeugt nicht aus nichts, sondern aus dem seienden Vater, nicht wie die k�rperlichen
Dinge durch Schnitte oder durch Emanation aufgrund von Trennungen, wie es Sabellius oder Valentinus meinen,
sondern auf unbeschreibliche und unfa�bare Art und Weise, gem�� dem, der, wie wir schon oben
angef�hrt haben, sagt: ">Sein Werden, wer wird es verstehen?"<, weil seine Hypostase f�r jede gewordene
Natur unergr�ndlich ist, wie auch der Vater selbst unergr�ndlich ist, da die Natur der Vernunftwesen
das Wissen um die v�terliche Abstammung nicht ergreifen kann.

\kapnum{47}Was aber die M�nner, die vom Geist der Wahrheit getrieben werden, nicht von mir lernen
m�ssen, da uns dar�ber bereits  �berdeutlich das Wort Christi belehrt, ist: ">Keiner wei�, wer der Vater
ist, au�er der Sohn; und keiner wei�, wer der Sohn ist, au�er der Vater."< Und wir haben gelernt,
da� der Sohn wie der Vater unwandelbar und unver�nderlich ist, sich selbst genug und vollkommen,
gleich wie der Vater, abgesehen vom Ungezeugtsein. Er ist n�mlich das genaueste und absolut
unver�nderliche Abbild des Vaters.

\kapnum{48}Es ist n�mlich deutlich, da� das Abbild mit allem ausgef�llt ist, worin die gr��ere �hnlichkeit besteht, wie es der Herr selbst lehrte: ">Mein Vater"<, sagte er, ">ist
gr��er als ich."< Und entsprechend glauben wir, da� der Sohn immer aus dem Vater ist, ">denn er ist
der Abglanz der Herrlichkeit und Abdruck der v�terlichen Hypostase."< Aber niemand m�ge bitte das
">immer"< auf die Vorstellung des ">Ungezeugtseins"< beziehen, wie es die glauben, deren
Wahrnehmungsf�higkeit der Seele abgestumpft ist. Denn keiner dieser Ausdr�cke, weder das ">er war"<, noch das
">immer"<, auch nicht das ">vor den �onen"< bedeutet dasselbe wie ">ungezeugt"<.

\kapnum{49}Aber welche Namen auch immer die menschliche Vorstellungskraft sich m�ht zu schaffen, das
Ungezeugte erkl�rt sie nicht (ich glaube, da� auch ihr so denkt, und setze mein Vertrauen auf eure
korrekte Entscheidung in allen Dingen), da ja diese Namen in keiner Hinsicht den Begriff des Ungezeugtseins
erl�utern.

\kapnum{50}Denn diese Namen scheinen gewisserma�en in die zeitliche Sp�hre hineinzureichen, denn
sie sind  nicht wirklich in der Lage, die Gottheit des Eingeborenen und sozusagen sein ">altes
Alter"< angemessen zu beschreiben, auch wenn die heiligen M�nner sie zwangsweise jeder nach seinem
Verm�gen gebrauchten, um auf das Geheimnis hinzuweisen, w�hrend sie gleichzeitig mit einer
vern�nftigen Entschuldigung um Verst�ndnis bei den Zuh�rern baten, n�mlich mit den Worten: ">soweit
wir herankommen"<.

\kapnum{51}Wenn aber diese M�nner einen irgendwie besseren Wortlaut als von menschlichen Lippen
erwarten und bekanntgeben, da� das von ihnen st�ckweise Erkannte zu vernichten ist, dann ist
deutlich, da� das ">er war"<, ">immer"< und ">vor den �onen"< ihre Erwartungen weit entt�uscht; aber
wie dem auch sei, die Worte sind nicht dasselbe wie ">ungezeugt"<.

\kapnum{52}Also mu� dem ungezeugten Vater die eigene W�rde bewahrt werden, indem man ihm keinen
anderen Urheber des Seins zuweist. Dem Sohn aber mu� die passende Ehre zugewiesen werden, indem man
auf seine anfangslose Zeugung aus dem Vater hinweist, und, wie oben gesagt, ihm die Ehre erweist
und nur gottesf�rchtig und wohlmeinend das ">er war"< und ">immer"< und ">vor den �onen"< auf ihn
bezieht und dabei wahrlich seine Gottheit nicht leugnet, sondern bei dem Abbild und Abdruck auf die
genaueste �hnlichkeit mit dem Vater in jeder Hinsicht hinweist, in der �berzeugung, da� das
Ungezeugte die Besonderheit ist, die allein dem Vater zukommt, weil ja auch der Erl�ser selbst
sagte: ">Der Vater ist gr��er als ich."<

\kapnum{53}Zus�tzlich aber zu dieser gottesf�rchtigen Ansicht �ber den Vater und den Sohn, wie ihn
uns die heiligen Schriften lehren, bekennen wir einen heiligen Geist, der sowohl die heiligen
Menschen des alten als auch die g�ttlichen Lehrer des sogenannten neuen Bundes bewegt hat; und
die eine und alleinige, katholische und apostolische Kirche, immer unangreifbar, auch wenn die ganze
Welt sie bek�mpfen wollte, die den Sieg davontr�gt gegen jeden gottlosen Aufstand der Irrgl�ubigen,
da ihr Hausherr uns Zuversicht einfl��te mit dem Ausruf: ">Seid zuversichtlich, ich habe die Welt
besiegt."<

\kapnum{54}Danach aber, wissen wir, kommt die Auferstehung der Toten, deren Erstling unser Herr
Jesus Christus ist, der wahrlich und nicht nur dem Anschein nach einen Leib aus der Gottesgeb�rerin
Maria getragen hat ">am Ende der Zeiten zur Vergebung der S�nden"<, der sich beim Geschlecht der
Menschen aufhielt, gekreuzigt wurde und gestorben ist, der aber  dadurch nicht in seiner G�ttlichkeit vermindert wurde, der von den Toten auferstanden ist, in den Himmel aufgenommen wurde und ">zur
Rechten des H�chsten sitzt."<

\kapnum{55}Ich habe dies im Brief nur zum Teil ausgef�hrt, da ich, wie gesagt, der Ansicht bin, da�
es nicht angemessen ist, die Details genauer auszuf�hren, da diese Dinge eurem heiligen Eifer nicht
verborgen sind. Dies lehren wir, dies predigen wir, dies sind die apostolischen Lehren der Kirche,
f�r die wir auch sterben, und wir machen uns weniger Gedanken um die, die uns zwingen, diesen Lehren
abzuschw�ren. Auch wenn sie uns mit Folter dazu zwingen, wenden wir uns nicht von der in diesen
Lehren liegenden Hoffnung ab.

\kapnum{56}Da sie im Widerspruch zu diesen Lehren standen, wurden die um Arius und Achillas und
die, die mit ihnen Feinde der Wahrheit wurden, aus der Kirche ausgeschlossen, da sie von unserer
gottesf�rchtigen Lehre entfremdet wurden, wie es der selige Paulus beschreibt: ">Wenn jemand euch
ein anderes Evangelium verk�ndet als das, welches ihr empfangen habt, der sei verdammt"<, auch wenn
er vorgibt, ein ">Engel vom Himmel"< zu sein, aber auch: ">Wenn jemand anders lehrt und sich nicht den
heilsamen Worten unseres Herrn Jesus Christus und der gottesf�rchtigen Lehre anschlie�t, so ist er
verblendet, auch wenn der nichts versteht"<, und so weiter.

\kapnum{57}Diese also, die von der Bruderschaft aus der Kirche ausgeschlossen worden sind, nehme bitte niemand von euch auf,
und auch das von ihnen Gesagte oder Geschriebene ertragt bitte nicht, denn die Zauberk�nstler l�gen
in allen Dingen und sie sagen gewi� nicht die Wahrheit.

\kapnum{58}Denn sie ziehen durch die St�dte zu keinem anderen Zweck, als unter Vort�uschung der
Freundschaft und im Namen des Friedens durch Heuchelei und Schmeichelei Briefe auszuteilen und anzunehmen, um dadurch einige von ihnen get�uschte ">Frauen voller S�nden"< (und so weiter) in die Irre zu f�hren.

\kapnum{59}Diese also, die so dreist gegen Christus aufzutreten wagen, die das Christentum einerseits
�ffentlich in den Schmutz ziehen, andererseits es mit Vergn�gen vor Gericht anzeigen, die, soweit
sie k�nnen, in Friedenszeiten eine Verfolgungsjagd gegen uns entfachen, die das unsagbare Geheimnis
der Erzeugung Christi entkr�ften, diese weist von euch, geliebte und gleichgesinnte Br�der, schlie�t
euch uns an gegen deren wahnsinnige Dreistigkeit, gleichwie unsere Mitbisch�fe, die sehr erbost
waren, mir gegen sie geschrieben und den Rundbrief mitunterschrieben haben, den ich auch euch
geschickt habe durch meinen Sohn, den Diakon Api~-- und zwar sind diese aus ganz Aegyptus und der
Theba"is, aus Libya, der Pentapolis, Syria, Lycia, Pamphylia, Asia, Cappadocia und den anderen
Nachbarregionen; ich hoffe, wie von diesen, so auch von euch zu h�ren.

\kapnum{60}Denn ich kann viele Mittel f�r die Gesch�digten bereitstellen, aber dies hat sich als
wirksamstes Mittel f�r das von ihnen get�uschte Volk erwiesen, weil man dem gemeinsamen Votum
unserer Mitbisch�fe gew�hnlich folgt und deswegen mit Eifer umkehren wird. Seid gegr��t miteinander,
zusammen mit den Br�dern unter euch. Ich bete, da� es euch im Herrn wohlergehe; es helfe mir eure
christusliebende Seele!

\begin{footnotesize}
Die aus der Kirche ausgeschlossenen H�retiker sind folgende: von den Presbytern Arius, Achillas, Aeithales, Sarmates, \Ladd{Carpones}, ein weiterer Arius; von den Diakonen aber \Ladd{Gaius}, Euzoius, Lucius, Julius, Menas, Helladius.
\end{footnotesize}\clearpage

% \section{Nachricht von einem Briefe Alexanders von Alexandrien an Silvester von Rom}
\kapitel{Regest eines Briefes des Alexander von Alexandrien an Silvester von Rom (Urk.~16)}
\label{ch:16}
\thispagestyle{empty}

Es existieren Briefe des damaligen Bischofs Alexander, adressiert an Silvester, seligen Angedenkens,
in denen er, noch vor der Ordination des Athanasius, bekanntgab, elf Leute, Presbyter sowohl als
auch Diakone, aus der Kirche ausgeschlossen zu haben, weil sie der H�resie des Arius folgten. Es
hei�t, da� einige von ihnen jetzt au�erhalb der katholischen Kirche eine Stellung haben und eigene
kleine Versammlungen organisieren, mit denen auch Georgius in Alexandrien bekanntlich in brief"|lichem
Kontakt steht.

% \section{Brief Kaiser Konstantins an Alexander von Alexandrien und Arius}
\kapitel{Brief des Kaisers Konstantin an Alexander von Alexandrien und Arius (Urk.~17)}
\label{ch:17}
\thispagestyle{empty}

Der Sieger Konstantin, der Gro�e, der Augustus, an Alexander und Arius!

\kapnum{1}Den Gott des Alls, den Helfer bei meinen Unternehmungen, den Erl�ser, mache ich 
zu meinem Zeugen daf�r, da� ich mir zwei Dinge vorgenommen habe, weswegen ich tats�chlich das Amt
�bernommen hatte. Zuerst wollte ich n�mlich die religi�se Gesinnung aller
V�lkern zu einer einheitlichen Gestalt vereinen, zweitens aber den Leib der gemeinsam bewohnten
Welt, die wie an einer schweren Wunde litt, wiederbeleben und vereinen. Da ich mir dies vorgenommen
hatte, plante ich das eine mit dem unaussprechlichen Auge des Geistes, das andere aber versuchte ich mit der Macht des bewaffneten Armes durchzusetzen, da ich wu�te, wenn ich unter allen
Gottesdienern zu meinen eigenen Gebeten eine gemeinsame Gesinnung aufrichte, da� dann auch der
Dienst an den �ffentlichen Angelegenheiten, begleitet durch die frommen Einstellungen aller, eine
Erneuerung gewinnen wird.

\kapnum{2}Als n�mlich ein unertr�glicher Wahnsinn wahrlich ganz Afrika ergriffen hatte, weil es
einige gewagt hatten, die Religion der V�lker in unbek�mmertem Leichtsinn in verschiedene H�resien
aufzuspalten, wollte ich diese Krankheit beruhigen, fand aber heraus, da� keine andere Therapie in
dieser Sache hilft, au�er da� ich einige von euch hinschicke, da� sie bei der Vermittlung zwischen
den miteinander Streitenden helfen, nachdem ich ja den gemeinsamen Feind der bewohnten Welt, der
euren heiligen Versammlungen seinen eigenen gesetzlosen Willen entgegensetzte, verjagt habe.

\kapnum{3}Da n�mlich die Kraft des Lichtes und das Gesetz der heiligen Religion durch die Wohltat
des H�chsten gleichsam aus dem Scho� des Ostens ausgeteilt worden und zugleich die ganze bewohnte Welt
von einem heiligen Leuchter erhellt war, bem�hte ich mich, euch mit einem Wink der Seele und der Kraft
der Augen zu suchen, weil ich glaubte, da� ihr gleichsam F�hrer zur Erl�sung der V�lker seid. Gleich bei dem gro�en Sieg also und dem wahren Triumph �ber die Feinde beeilte ich mich,
dies als erstes anzustreben, denn es schien mit das Dringlichste und Ehrenvollste von allen zu sein.

\kapnum{4}Oh, edelste und g�ttliche Vorsehung, welch gef�hrliche Verwundung trifft mein Geh�r,
mehr aber noch mein Herz! Sie zeigte, da� unter euch ein viel schwererer Streit entstanden war
als dort hinterlassen, und da� ihr, von denen ich hoffte, Heilung f�r die anderen zu
bekommen, eurerseits noch weitaus dringender selbst Heilung braucht. Als ich mir aber Gedanken �ber den Anla�
und das Thema dieser Auseinandersetzungen machte, da zeigte sich der Grund als doch sehr
geringf�gig und keineswegs ausreichend f�r einen derartigen Streit. Daher sehe ich mich zu diesem
Brief gezwungen und wende mich, nachdem ich die g�ttliche
Vorsehung als Helfer in dieser Sache angerufen habe, an euren einm�tigen Scharfsinn, und stelle mich zu Recht gleichsam als Friedensrichter mitten in den Streit bei euch untereinander.

\kapnum{5}Denn mit der Hilfe des H�chsten d�rfte es mir, falls es auch irgendeinen gewichtigen
Anla� f�r einen Streit g�be, wohl nicht schwer fallen, indem ich ein Wort an die frommen
�berzeugungen der H�rer richte, jeden zum Besseren zu wenden, in diesem Fall aber, wo der Anla�
winzig und sehr unbedeutend ist, der dem Ganzen im Weg steht, wird man gewi� einfacher und viel
leichter eine Kl�rung in dieser Angelegenheit erreichen.

\kapnum{6}Ich habe also in Erfahrung gebracht, da� der Ausbruch des gegenw�rtigen Streites
folgenderma�en geschah: Du, Alexander, hast bei den Presbytern nachgefragt, was jeder von
ihnen denn von irgendeiner Stelle in dem geschriebenen Gesetz, vielmehr von einem unwichtigen Aspekt
irgendeiner Frage hielte. Und du, Arius, hast voreilig etwas erwidert, was du entweder von Anfang an nicht h�ttest denken d�rfen oder zumindest sofort mit Schweigen h�ttest �bergehen m�ssen. Dadurch
wurde bei euch das Zerw�rfnis geweckt und die Gemeinschaft verleugnet, das geheiligte Volk aber in
zwei Teile zerrissen und aus der Verbindung mit dem gemeinsamen Leib gezerrt.

\kapnum{7}Daher sollte jeder von euch gleicherma�en eine Entschuldigung anbieten und das
akzeptieren, was euer Mitdiener euch zu Recht ans Herz legt. Was aber meine ich damit? Es w�re von
Anfang an angebracht gewesen, zu diesen Themen weder Fragen zu stellen noch auch auf Fragen zu
antworten. 

\kapnum{8}Denn derartige Fragen, f�r die keine gesetzliche Verpflichtung besteht, sondern die aus
nutzloser Faulheit die Streitsucht stellt, m�ssen wir, auch wenn sie im Rahmen philosophischer
�bungen vorkommen, im Verstand einschlie�en und nicht leichtfertig in �ffentliche Versammlungen
einbringen oder gar leichtsinnigerweise den Ohren des Volkes anvertrauen. Denn wer ist schon in der
Lage, die Macht so gro�er und sehr schwerer Themen genau zu durchschauen oder angemessen zu
interpretieren? Und wenn auch jemandem zugetraut wird, da� er das ohne weiteres k�nne, welchen Teil des Volkes
wird er �berzeugen? Oder wer kann derartig komplizierte Fragen behandeln, ohne auf gef�hrliches
Terrain zu geraten? In solchen Fragen ist also die Geschw�tzigkeit zur�ckzuhalten, da
entweder wir aufgrund unserer schwachen Natur das vorgelegte Thema nicht interpretieren k�nnen oder
da die Zuh�rer aufgrund ihres schwerf�lligeren Fassungsverm�gens nicht in der Lage sind, zu einem
genauen Verst�ndnis des Gesagten durchzudringen, damit das Volk sich nicht aus beiden Gr�nden
zwangsweise spaltet oder verl�stert. 

\kapnum{9}Daher sollen die gedankenlose Frage und die verantwortungslose Antwort f�r einander gleiches
gegenseitiges Verst�ndnis aufbringen. Denn der Anla� f�r diesen Streit hat sich weder an einer
zentralen Stelle der Ermahnungen an uns im Gesetz entz�ndet, noch wurde uns wegen der Gottesverehrung
irgendeine neue H�resie untergeschoben, sondern ihr habt ein und dieselbe Ansicht, so da� ihr zu
einer gemeinsamen �bereinkunft zusammenkommen k�nntet. Denn man sollte doch meinen, da� es weder
angemessen noch �berhaupt rechtens ist, das so gro�e Volk Gottes, welches unter euren Gebeten und Einsichten
gedeihen sollte, zu entzweien, nur weil ihr euch untereinander �ber kleine und so geringf�gige
Fragen zerstreitet.

\kapnum{10}Um aber eure Vernunft durch ein kleines Beispiel zu erinnern, denkt doch bitte auch an die
Philosophen, wie sie alle in einer Lehre verbunden sind, oft aber auch, wenn sie sich auch in
irgendeinem Teilbereich ihrer Aussagen unterscheiden und sich wegen der Gelehrsamkeit trennen,
dennoch aber wieder wegen der Einheit der Lehre zusammenfinden. Wenn dies aber so ist, m�ssen dann
nicht erst recht wir, die wir als Diener des gro�en Gottes eingesetzt wurden, in der Frage nach der
Gottesverehrung einm�tig sein? La�t uns aber mit besserem Wissen und gr��erer Einsicht das Gesagte
betrachten, ob es richtig ist, wenn wegen einer winzigen und nebens�chlichen �u�erung bei euch
Br�der gegen Br�der vorgehen und das hohe Gut der Gemeinschaft aufgrund gottloser Streitereien durch euch,
die ihr untereinander �ber so kleine und keineswegs notwendige Themen streitet, auseinandergerissen
wird. Dies pa�t doch wohl besser zu unvern�nftigen Kindern, als da� es zur Einsicht heiliger und
vern�nftiger M�nner geh�rt.

\kapnum{11}La�t uns doch freiwillig von der teuflischen Versuchung Abstand nehmen! Unser gro�er
Gott, der Erl�ser aller, hat den Lichtschein auf alle gemeinsam gerichtet. La�t mich durch seine
Vorsehung mein Bem�hen um die Verehrung des H�chsten zu Ende f�hren, damit ich dessen V�lker bei euch
durch meine Worte, Dienste und Ermahnungen zu einem gemeinschaftlichen Zusammenleben f�hre.

\kapnum{12}Denn da, wie ich sagte, unter euch ein Glaube und eine Meinung �ber die religi�se
Richtung ist, und da die Vorschrift des Gesetzes durch seine zwei Teile das Ganze zu einer
Entscheidung der Seele zusammenf�gt, deswegen darf dies, was unter euch einen kleinen Streit
verursacht hat, da es sich doch nicht auf die Macht des ganzen Gesetzes bezieht, keineswegs
irgendeine Spaltung oder Unruhe bei euch verursachen.

\kapnum{13}Ich sage dies nicht, um euch zu zwingen, in einer so l�cherlichen Frage, wie auch immer sie lauten mag, �bereinzustimmen. Es ist n�mlich m�glich, das hohe Gut der Gemeinschaft
reinlich unter euch zu bewahren und die Gemeinsamkeiten in allen Dingen zu erhalten, auch wenn bei
euch untereinander irgendeine Meinungsverschiedenheit �ber nebens�chliche Fragen, meistens nur in einem
Teilbereich, entsteht, da nie alle in allen Dingen dasselbe wollen und auch nie eine Natur oder eine
Meinung bei uns vorherrscht.

\kapnum{14}�ber die g�ttliche Vorsehung sollte es also bei euch einen Glauben, ein Verst�ndnis,
eine �bereinkunft �ber den H�chsten geben, was ihr aber �ber diese ganz nebens�chlichen Fragen
untereinander f�r Haarspaltereien veranstaltet, das mu�, auch wenn ihr nicht zu einer Meinung
zusammenfindet, im Geiste bleiben, bewahrt in unaussprechlichen Gedanken. Das Besondere der
gemeinsamen Freundschaft jedoch und der Glaube an die Wahrheit und die Ehrfurcht vor Gott und der
Verehrung des Gesetzes soll bei euch unersch�tterlich bleiben. Kehrt zur�ck zur Freundschaft und
Liebe untereinander, erstattet dem Volk �berall die vertrauten Umarmungen zur�ck, und nachdem ihr
gleichsam eure Seelen gereinigt habt, erkennt einander wieder an. Oft ist n�mlich die Freundschaft noch
herzlicher, gerade wenn sie nach der Aufhebung der Feindschaft zur Vers�hnung zur�ckkehrt.

\kapnum{15}Gebt mir also die ruhigen Tage und die sorgenfreien N�chte wieder zur�ck, damit auch mir
die Lust an dem reinen Licht und die Freude an einem ruhigen Leben bewahrt bleibt. Sonst mu� ich unweigerlich seufzen und bitterlich weinen und mein ganzes Leben lang nicht mehr
in Ruhe leben. Denn wenn die V�lker Gottes, ich meine meine Mitdiener, durch so ungerechte und
sch�dliche Streitsucht untereinander zerr�ttet sind, wie ist es mir da m�glich, weiterhin die Vernunft
zu bewahren? Damit ihr aber das Ausma� meiner Trauer dar�ber erkennt: Als ich vor kurzem in
Nikomedien war, wollte ich spontan in den Osten reisen. Als es mich schon zu euch zog und ich schon
teilweise bei euch war, da warf die Nachricht �ber diese Angelegenheit den Entschlu� wieder um,
damit meine Augen nicht gezwungen wurden, das zu sehen, was ich nicht einmal den Ohren zumuten
wollte. 
�ffnet mir doch wieder durch Eintracht bei euch den Weg in den Osten, den ihr mit eurer Streitsucht untereinander
versperrt habt, und erlaubt mir die Freude, euch bald wieder zusammen mit allen anderen V�lkern zu
sehen und dem H�chsten mit den richtigen Formulierungen von Gebeten geb�hrend zu danken f�r die gemeinsame
Eintracht aller und die Freiheit.
\clearpage
\kapitel{Brief der Synode von Antiochien des Jahres 325 an Alexander von Byzanz (Urk.~18)}
\label{ch:18}
\thispagestyle{empty}

\begin{footnotesize}
Abschrift des Schreibens der Synode von Antiochien an Alexander, den Bischof des Neuen Rom.
\end{footnotesize}

\kapnum{1}Den Heiligen und Seelenverwandten, den geliebten Bruder und Mitdiener, Alexander,
gr��en Ossius, Eustathius, Amphion, Bassianus, Zenobius, Piperius, Salamanes, Gregorius,
Magnus, Petrus, Longinus, Manicius, Mocimus, Agapius, Macedonius, Paulus, Bassianus,
Seleucus, Sopatrus, Antiochus, Macarius, Jacob, Hellanicus, Nicetas, Archelaus, Macrinus,
Germanus, Anatolius, Zo"ilus, Cyrillus, Paulinus, A"etius, Mose, Eustathius, Alexander,
Irenaius, Rabbula, Paulus, Lupus, Nicomachus, Philoxenus, Maximus, Marinus, Euphrantion,
Tarcondimantus, Irenicus, Petrus, Pegasius, Eupsychius, Asclepius, Alpheius, Bassus,
Gerontius, Hesychius, Avidius und Terentius im Herrn.

\kapnum{2}Da die katholische Kirche �berall ein Leib ist, auch wenn
es an verschiedenen Orten ein Zelt der Versammlung gibt, gleichsam als Glieder des ganzen
Leibes, ist es folgerichtig, da� auch deiner Liebe diese Angelegenheiten bekannt werden, die
sowohl von mir als auch von unseren heiligen, seelenverwandten Br�dern und Mitdienern verhandelt
wie auch verabschiedet wurden, damit auch du, der du gleichsam im Geiste gegenw�rtig
bist, zugleich mit uns sprichst und uns dieses gebietest, was von uns dann
als gesund und auch dem kirchlichen Gesetz gem�� festgesetzt und verabschiedet wurde. 
 
\kapnum{3}Als ich zu der Kirche der Antiochener kam und sah, da� sie sehr
verwirrt war vom Lolch in der Lehre und durch die Verwirrung von gewissen Leuten,
schien es mir gut zu sein, da� nicht von mir allein, was solcher Art ist, exkommuniziert
und verworfen werde, sondern da� es auch recht sei, unsere Seelenverwandten und die Amtsbr�der, besonders die in unserer Nachbarschaft, zu ermahnen wegen dieser
f�r die Br�der dr�ngenden und bedr�ckenden Angelegenheit, f�r die n�mlich, die aus
Palaestina, Arabia und Phoenice sind und aus Syria coelis, aus Cilicia und auch f�r
einige von denen, die in Cappadocia sind, so da� wir in gemeinsamer
�berlegung erkennen und pr�fen und schlie�lich die Angelegenheiten, die die Kirche angehen, gemeinsam
festlegen. Denn zugleich von vielen Gerechten wird jene Stadt bewohnt.

\kapnum{4}Da uns also die G�te Gottes allzumal versammelt hat in der Bleibe in
Antiochien, fanden wir, als wir nachdachten und die gemeinsamen und n�tzlichen Dinge
und, was der Kirche Gottes f�rderlich ist, behandelten, arge Verwirrung vor,
besonders deswegen, weil das kirchliche Gesetz und seine Canones mittlerweile in vieler
Hinsicht verdammt und verachtet wurden von gewissen Personen, die sich wie junge Leute
geb�rden, und so g�nzlich zum Schweigen gebracht waren.

\kapnum{5}Weil es verhindert wurde, da� sich eine Bischofssynode in den Orten
dieser Regionen versammelte, schien es gut, da� als erstes das untersucht w�rde,
was mehr als alles Gute und mehr als alles Erhabene ist, vielmehr, was
das ganze Geheimnis des Glaubens ist, der bei uns gilt. Ich spreche n�mlich vom
Erl�ser von uns allen, dem Sohn des lebendigen Gottes.

\kapnum{6}Weil n�mlich unser Bruder und Mitdiener, der ehrw�rdige und geliebte Alexander,
der Bischof von Alexandria, einige von den Presbytern, die zu Arius halten, aus der
Kirche ausgeschlossen hat wegen der Beleidigung, die sie gegen unseren Erl�ser
ge�u�ert haben, so da� auch in die Gemeinschaft einige von ihnen aufgenommen wurden, weil sie auch manche durch ihre gottlose Lehre zu t�uschen in der Lage
sind, schien es deshalb der heiligen Synode gut, da� dies zuerst untersucht werde, so da�, wenn so
die Hauptsache der Geheimnisse bei uns bewahrt bliebe, auch diese �brigen Dinge, jedes
einzelne f�r sich, behandelt werden k�nnten.

\kapnum{7}Und als wir uns nun gemeinsam in der Gegenwart auch von einigen
redegewandten Br�dern wegen des kirchlichen Glaubens, den wir von den Schriften und von
den Aposteln gelernt und von den V�tern empfangen haben, versammelt hatten, lie�en wir das Wort ausgehen. Und auch das, was von Alexander, dem Bischof von Alexandrien, behandelt wurde gegen die, die
zu Arius halten, haben wir ins Zentrum gestellt, damit, wenn einige dem entgegengesetzt
die Lehre zu verkehren scheinen, diese auch in der Kirche fremd seien, damit sie nicht etwa,
wenn sie in ihr blieben, einige der allzu Einf�ltigen hinwegrissen.

\kapnum{8}Der Glaube nun, der zuerst gleichsam von
geistlichen M�nnern, von denen man weiterhin nicht denken sollte,
da� sie im Fleische lebten oder sich berieten, sondern da� sie im Geiste die heiligen
Schriften der Bibel inspiriert vom Geist meditiert haben, folgender: an einen Gott zu
glauben, den Vater, den Allm�chtigen, den Unbegreif"|lichen, Unver�nderlichen,
 Unwandelbaren, den F�rsorger und All-Lenker, den Gerechten, den Guten, den Sch�pfer des
Himmels und der Erde und von allem, was darin ist, den Herrn des Gesetzes und der
Propheten und des neuen Bundes,

\kapnum{9}und an den einen Herrn Jesus Christus, den einzigen Sohn, der gezeugt wurde
nicht aus nichts, sondern vom Vater, nicht wie ein Gesch�pf, sondern als
wirklich Gezeugter. Er wurde aber gezeugt auf unsagbare und unaussprechliche Weise, weil
es der Vater allein war, der ihn gezeugt hat, und der Sohn, der gezeugt wurde, es erkannt
hat: ">Denn niemand kennt den Vater, es sei denn der Sohn, noch den Sohn, es sei denn der
Vater."< Er hat alle Zeit existiert und es stimmt nicht, da� er anfangs nicht existiert hat. 

\kapnum{10}Denn da� er Bild sei, er allein, haben wir von den g�ttlichen Schriften
gelernt. Nicht, da� er nicht gezeugt w�re~-- vom Vater versteht sich, nicht durch
Setzung; denn es ist etwas Frevelhaftes und eine Beleidigung, das zu sagen~--, sondern wirklich
und wahrhaft nennen ihn die Schriften Sohn, der gezeugt ist, wie wir auch glauben,
da� er unver�nderlich und unwandelbar ist. Nicht aber glauben wir, da� er durch Willen
oder durch Setzung gezeugt wurde oder gewesen ist, als ob er
aus nichts zu existieren scheine, sondern da� es angemessen ist, da� er gezeugt
wurde, und da� es also nicht statthaft ist, da� es verstanden werde gem�� der
Gleichheit und der Natur und der Mischung von allem diesem, was durch ihn geworden ist,

\kapnum{11}sondern weil er alles Verstehen und Denken und Reden �berschreitet, bekennen
wir, da� er vom ungezeugten Vater gezeugt wurde, Gott-Wort, Licht-Wahrheit,
Gerechtigkeit, Jesus der Christus, Herr von allem und Erl�ser. Denn er ist Bild 
nicht aus dem Willen noch von etwas anderem, sondern aus eben der v�terlichen Hypostase.
Er aber ist der Sohn, das Gott-Wort. Auch ist er im Fleische von der Gottesgeb�rerin Maria
geboren worden, hat Fleisch angezogen, gelitten, ist gestorben, auferstanden
von den Toten und aufgefahren in den Himmel. Er sitzt aber zur Rechten der erhabenen
Majest�t. Und er kommt, um die Lebenden und die Toten zu richten. 

\kapnum{12}So wie er unser Erl�ser ist, lehren die heiligen Schriften au�erdem auch, 
an den einen Geist zu glauben, an die eine katholische Kirche, die
Auferstehung der Toten und das Gericht der Vergeltung nach dem, was jemand im Leibe
getan hat, sei es also Gutes oder B�ses, 

\kapnum{13}Dabei schlie�en wir die aus der Kirche aus, die �ber den Sohn Gottes sagen und
glauben und verk�ndigen: ">Gesch�pf"< oder ">Geschaffenes"< oder ">Gemachtes"< und nicht,
da� er wahrhaft als ">Gezeugter"< existiert oder da� ">es einmal war, da� er
nicht existiert hat"<. Denn wir, wir glauben, da� er existiert hat und da� er existiert
und da� er Licht ist. 
Mit denen aber, die denken, da� er durch die Gnade des freien
Willens unver�nderlich ist, steht es wie mit denen, die die Zeugung aus nichts
anf�hren und sagen, da� er nicht wie der Vater von Natur aus unwandelbar existiert. Denn
als Abbild des Vaters n�mlich, wie in allen anderen Dingen, so besonders auch hierin, wird unser Erl�ser verk�ndigt.

\kapnum{14}Dieser Glaube also wurde festgelegt, und die ganze heilige Synode
hat �bereingestimmt und bekannt, da� dies die apostolische und erl�sende Botschaft sei;
und alle Mitdiener waren einer Meinung dar�ber. Allein Theodotus von Laodicea, Narcissus von 
Neronias und Eusebius von Caesarea in Palaestina erschienen jedoch wie Leute, die die Heiligen Schriften und die apostolischen
Lehren verga�en und die, obwohl sie mit viel T�cke versuchten, sich zu verh�llen und ihre
Torheiten durch die �berredung mit Worten und nicht durch die Wahrheit zu verbergen, dennoch
dem Entgegengesetztes einf�hrten. Denn auch auf Grund der Tatsachen und dessen, was sie gefragt wurden und fragten, sind sie dessen �berf�hrt
worden, da� auch bei ihnen diesselbe Auf"|fassung wie die bei den Anh�ngern des Arius
festzustellen ist und da� sie das Gegenteil von dem denken, was zuvor festgelegt wurde. Von nun an, da sie dabei blieben und sich nicht sch�mten vor der heiligen
Synode, die dar�ber M�he hatte und gekr�nkt war, urteilten wir alle, die Mitdiener,
die in der Synode sind, da� wir mit diesen keine Gemeinschaft haben, und da� sie des
Dienstes nicht w�rdig sind wegen ihres Glaubens, der der katholischen Kirche sch�dlich ist.

\kapnum{15}Und wie du wei�t, haben wir dir deswegen geschrieben, damit auch du deine Seele
bewahrst und dich h�test von der Gemeinschaft mit ihnen und davor, da� du an sie
schreibst oder von ihnen Briefe der Gemeinschaft empf�ngst. Auch dies aber wisse, da� wir 
wegen der gro�en Menschenliebe der Synode diesen Leuten die gro�e und heilige Synode in Ancyra als einen Ort zur Bu�e und zur Belehrung in der Wahrheit einger�umt
haben. Deshalb trage Sorge, da� du allen gleichgesinnten Br�dern dies sendest, damit
auch sie erkennen k�nnen, was diese betrifft und wer die sind, die sich von der Kirche
getrennt haben und nicht mit ihr �bereinstimmen. Gr��e alle Br�der, die mit euch sind
und bei euch sind. Dich, Bruder, gr��en, die mit uns sind, im Herrn.

\begin{footnotesize}
Ende des Briefs der Synode in Antiochien an Alexander,
den Bischof des neuen Rom, das Konstantinopel ist.
\end{footnotesize}
% \section{Brief des Narcissus von Neronias an Chrestus, Euphronius und Eusebius}
\kapitel[Markell von Ancyra �ber einen Brief des Narcissus von Neronias an Chrestus, Euphronius und Eusebius (Urk.~19)][Markell von Ancyra �ber einen Brief des Narcissus von Neronias (Urk.~19)]{Markell von Ancyra �ber einen Brief des Narcissus von Neronias an Chrestus, Euphronius und Eusebius (Urk.~19)}
\label{ch:19}

\kapnum{1}">Mir fiel n�mlich der Brief des Narcissus, des Vorstehers von Neronias, in die H�nde,
den er an einen gewissen Chrestus, an Euphronius und an Eusebius geschrieben hatte, in dem er schreibt, da� der Bischof Ossius ihn gefragt hatte, ob er wie Eusebius aus Palaestina sage, es gebe zwei Wesen; und daraus habe ich erfahren, da� er geantwortet hat, er glaube, es gebe drei Wesen."<

\kapnum{2} \dots\ Und danach wendet er (Markell) sich Narcissus zu und sagt:
">Auch wenn jemand das sagt und einen ersten und zweiten Gott einf�hrt, wie es Narcissus in seinen
Schriften formuliert hat (\,\dots), weil n�mlich er und sein Vater zwei seien, so haben wir zum Teil
schon geh�rt, was der Herr selbst und die heiligen Schriften (dar�ber) bezeugen. Wenn also Narcissus
deswegen das Wort der Kraft nach vom Vater abtrennen will, so soll er wissen, da� der Prophet, der
geschrieben hat, wie Gott sprach: \frq La�t uns den Menschen machen nach unserem Bild und unserer
Gleichheit\flq, selbst auch geschrieben hat: \frq Und Gott machte den Menschen\flq ."<
\clearpage
\kapitel{Brief des Kaisers Konstantin mit der Einberufung zur Synode von Nicaea (Urk.~20)}
\label{ch:20}
\thispagestyle{empty}

\begin{center}
Brief des Kaisers Konstantin, der die Bisch�fe von Ancyra nach Nicaea rief.\end{center}

Da� in meinen Augen nichts gewichtiger ist als die Furcht vor Gott, glaube ich, ist
jedermann offenbar. Weil aber zun�chst darin �bereinstimmung bestand, da� die Bischofssynode in Ancyra in Galatia stattfinden sollte, so scheint es uns jetzt aus vielen Gr�nden gut, da� es passend sei, da� sie sich in Nicaea in Bithynia versammle: Sowohl wegen
der Bisch�fe, die aus Italia und den �brigen Gegenden Europas sind, als auch wegen der
guten Mischung der Luft und damit ich als Augenzeuge und Teilnehmer dem Geschehen nahe sei. Deshalb, liebe Br�der, lasse ich euch wissen, da� ihr euch alle geflissentlich in jener genannten Stadt, d.\,h. in Nicaea, versammelt. Jeder einzelne von euch
also, wenn er denkt, da� dies vorz�glich sei, wie ich bereits gesagt habe, bem�he
sich, ohne irgendeinen Verzug schnell zu kommen, um aus der N�he in eigener Person Augenzeuge von dem zu werden, was geschieht.

Gott m�ge euch bewahren, liebe Br�der.


% \section{Fragment des Briefes des Euseb von Nikomedien an die Synode von Nic�a}
\kapitel{Fragment eines Briefes des Eusebius von Nikomedien (Urk.~21)}
\label{ch:21}
\thispagestyle{empty}

\begin{footnotesize}
Wie ihr Urheber, Eusebius von Nikomedien, in seinem Brief verr�t und schreibt:
\end{footnotesize}

Wenn wir wirklich den Sohn Gottes ebenfalls ">ungeworden"< nennen, dann k�nnen wir auch gleich anfangen, 
ihn als ">wesenseins"< zu bekennen.

\begin{footnotesize}
Als dieser Brief auf der Synode von Nicaea verlesen wurde, nahmen die V�ter dieses Wort in die
Glaubenserkl�rung auf, da sie sahen, da� es ein Schreckgespenst f�r ihre Gegner war.
\end{footnotesize}
% erstellt von UH
% \section{Brief des arianischen Euseb von C�sarea an Mitglieder aus seiner Gemeinde}
\kapitel{Brief des Eusebius von Caesarea an seine Kirche �ber die Synode von Nicaea (Urk.~22)}
\label{ch:22}
\thispagestyle{empty}

\kapnum{1}Von den Verhandlungen �ber den kirchlichen Glauben auf der
gro�en, in Nicaea versammelten Synode habt ihr, Geliebte, wahrscheinlich
schon von anderen erfahren, da gew�hnlich das Ger�cht dem genauen
Bericht �ber die Ereignisse vorauseilt. Aber damit euch die wahren Vorf�lle nicht durch
derartige Ger�chte verzerrt berichtet werden, sind wir gezwungen,
euch zun�chst den von uns vorgelegten Text �ber den Glauben und
anschlie�end den zweiten Text, den man nach Einf�gen von Zus�tzen in
unseren Vorschlag ver�ffentlicht hat, zuzuschicken.

\kapnum{2}Unser Schreiben also, das in Gegenwart unseres gottgeliebtesten
Kaisers vorgelesen und f�r gut und annehmbar befunden worden ist,
lautet folgenderma�en:

\kapnum{3}">Wie wir es von den Bisch�fen vor uns in der ersten Katechese
und damals, als wir die Taufe empfingen, �bernommen haben, wie
wir es aus den g�ttlichen Schriften gelernt und im Presbyteramt und
im Bischofsamt selbst geglaubt und gelehrt haben, so glauben wir auch
jetzt und empfehlen euch unseren Glauben; er ist aber folgender:

\kapnum{4}Wir glauben an einen Gott, den Vater, den Allm�chtigen, den
Sch�pfer aller sichtbaren und unsichtbaren Dinge, und an einen
Herrn Jesus Christus, Gottes Wort, Gott von Gott, Licht von Licht,
Leben von Leben, eingeborener Sohn, Erstgeborener vor aller Sch�pfung, 
vor allen �onen aus dem Vater gezeugt,
durch den auch alles geworden ist; der um unserer
Erl�sung willen Fleisch wurde, unter den Menschen wohnte und
litt und am dritten Tag auferstand und hinaufging zum Vater und
wiederkommen wird in Herrlichkeit zu richten die Lebenden und die
Toten. Wir glauben aber auch an den heiligen Geist.

\kapnum{5}Wir glauben, da� jeder von diesen da ist und existiert, der Vater
wahrhaftig als Vater, der Sohn wahrhaftig als Sohn und der heilige
Geist wahrhaftig als heiliger Geist, wie es auch unser Herr
sagte, als er seine J�nger zur Verk�ndigung
aussandte: ">Geht hin und lehrt alle V�lker und
tauft sie auf den Namen des Vaters und des Sohnes und des heiligen
Geistes."< Und in Bezug auf sie versichern wir, da� wir
es so halten und so denken und schon immer so gehalten haben und bis
zum Tod an diesem Glauben festhalten werden und entsprechend jede gottlose
H�resie verdammen.

\kapnum{6}Wir bezeugen, da� wir dies mit Herz und Seele schon immer gedacht
haben, seitdem wir unser selbst bewu�t sind, und auch jetzt in
Wahrheit so denken und reden �ber Gott, den Allm�chtigen, und unseren Herrn 
Jesus Christus, und wir k�nnen euch mit Beweisen zeigen
und davon �berzeugen, da� wir auch in vergangenen Zeiten so
geglaubt und verk�ndet haben"<.

\kapnum{7}Gegen\looseness=1\ den von uns vorgelegten Glauben gab es keinen Raum f�r
Widerspruch, sondern sogar unser gottgeliebtester Kaiser bezeugte
selbst als erster, da� er das Richtige enthalte. Er stimmt zu,
da� er selbst so denke, und befahl allen, diesem Glauben
beizupflichten, die Glaubenss�tze zu unterschreiben und sich darauf
zu einigen; nur das eine Wort ">wesenseins"<
solle hinzugef�gt werden, welches
sogar er selbst mit folgenden Worten erl�uterte: der Sohn werde nicht
">wesenseins"< genannt hinsichtlich eines k�rperlichen
Leidens, er entstehe weder durch Teilung noch durch irgendeine
Abtrennung vom Vater, denn es sei unm�glich, da� die
immaterielle, geistige und k�rperlose Natur, sondern es sei angemessen, diese
Dinge in g�ttlichen und unaussprechlichen Worten zu denken. Solcherma�en
philosophierte also unser weisester und fr�mmster Kaiser, sie aber,
unter dem Vorwand des Zusatzes ">wesenseins"<,
verfa�ten folgenden Text: 

\kapnum{8}">Wir glauben an einen Gott, den Vater, den Allm�chtigen, den Sch�pfer aller sichtbaren und unsichtbaren Dinge;
und an einen Herrn Jesus Christus, den Sohn Gottes, als Eingeborener gezeugt aus dem Vater, das
hei�t aus dem Wesen des Vaters, Gott von Gott, Licht von Licht, wahrer Gott von wahrem Gott, gezeugt
und nicht geschaffen, wesenseins mit dem Vater, durch den alles wurde, was im Himmel und auf Erden ist,
der f�r uns Menschen und um unseres Heils willen herabstieg und Fleisch wurde, der Mensch geworden ist,
litt und am dritten Tag auferstand, aufstieg in die Himmel, der kommen wird, um die Lebenden
und die Toten zu richten; und an den heiligen Geist.
Die aber sagen, \frq es war einmal, da� er nicht war\flq{} und \frq er war nicht, bevor er gezeugt
wurde\flq{} und \frq da� er aus nichts wurde\flq{} oder die behaupten, er sei aus einer anderen Hypostase oder
einem anderen Wesen, oder aber sagen, der Sohn Gottes sei geschaffen, wandelbar oder ver�nderlich, diese verdammt
die katholische und apostolische Kirche."<


\kapnum{9}Nachdem aber dieser Text von ihnen vorgetragen worden war, haben wir ihnen nicht ungepr�ft durchgehen lassen, wie das
">aus dem Wesen des Vaters"< und das
">mit dem Vater wesenseins"< von ihnen gemeint sei. Deswegen n�mlich
wurden Nachfragen und Antworten angesto�en, und in der Diskussion 
wurde der Sinn der Worte gepr�ft. Und schlie�lich haben sie sich darauf
verst�ndigt, da� ">aus dem Wesen"< darauf hinweise,
da� er zwar aus dem Vater sei, aber gewi� nicht als Teil des
Vaters existiere.

\kapnum{10}Aber auch uns schien es richtig zu sein, dieser Ansicht
zuzustimmen, da die fromme Lehre besagt, der Sohn sei aus dem Vater,
ohne sogleich ein Teil seines Wesens zu sein. Daher haben auch wir
dieser Ansicht zugestimmt und auch den Ausdruck
">wesenseins"< nicht abgelehnt, da wir einerseits das
Ziel des Friedens und andererseits das Ziel, nicht von der richtigen
Ansicht abzufallen, im Blick hatten.

\kapnum{11}Entsprechend haben wir aber auch das ">gezeugt und
nicht geschaffen"< akzeptiert, nachdem sie gesagt hatten, das
">geschaffen"< sei die Bezeichnung f�r die �brigen
Gesch�pfe, die durch den Sohn geworden sind, mit denen der Sohn
nichts gemein hat. Er sei doch kein Gesch�pf wie die durch ihn
Gewordenen, sondern er habe ein erhabeneres Wesen als jedes Gesch�pf,
das, wie es die g�ttlichen Schriften lehren, so aus dem Vater
gezeugt worden ist, da� die Art und Weise seiner Zeugung f�r
jede geschaffene Natur unaussprechlich und unbegreif"|lich sei.

\kapnum{12}So hat die Diskussion auch das ">der Sohn ist dem
Vater wesenseins"< gepr�ft und verstanden, und zwar nicht nach Art
und Weise der K�rper, auch nicht vergleichbar mit den sterblichen
Wesen, n�mlich nicht nach einer Teilung oder Abtrennung des Wesens,
aber auch nicht zufolge irgendeines Affektes oder eines Wandels oder
einer Ver�nderung des Wesens und der Kraft des Vaters. Denn allem
diesem sei die ungewordene Natur des Vaters fremd. 

\kapnum{13}Aber das ">wesenseins dem Vater"< zeige, da�
der Sohn Gottes keinerlei �hnlichkeit mit den gewordenen Gesch�pfen habe,
sondern allein dem Vater, dem Erzeuger, in jeder Hinsicht
gleiche und nicht aus irgendeiner
anderen Hypostase oder anderem Wesen, sondern aus dem Vater sei. Es schien uns richtig zu
sein, auch diesem Ausdruck zuzustimmen, nachdem er auf diese Art und
Weise ausgelegt worden war, insbesondere da wir von einigen �lteren
Gelehrten, ausgezeichneten Bisch�fen und Schreibern wu�ten,
da� sie in der Lehre �ber Gott in Bezug auf den Vater und den
Sohn den Begriff ">wesenseins"< verwendet hatten.

\kapnum{14}Dieses sei also zu der Glaubenserkl�rung angemerkt, der wir
alle nicht ohne Pr�fung zugestimmt haben, sondern entsprechend der
vorgelegten Erl�uterungen, welche in Anwesenheit des
gottgeliebtesten Kaisers selbst untersucht worden sind und vor dem Hintergrund
der genannten �berlegungen Zustimmung erfahren haben. 

\kapnum{15}Auch die Verurteilung, die von ihnen mit der Glaubenserkl�rung herausgegeben worden ist, haben wir f�r schadlos gehalten, da sie den
Gebrauch von schriftfremden Ausdr�cken ausschlie�t, weshalb
doch beinahe die ganze Verwirrung und Aufregung in der Kirche
entstanden ist. Da n�mlich keine von Gott inspirierte Schrift
">aus nichts"< und ">es war einmal,
da� er nicht war"< und die �brigen Beschreibungen verwendet,
schien es nicht vern�nftig, diese anzuwenden und zu lehren. Dem haben auch wir
zugestimmt, da es sinnvoll erschien, nachdem wir auch in der
fr�heren Zeit nicht gewohnt waren, diese W�rter zu gebrauchen. 

\kapnum{16}Ferner haben wir auch die Verurteilung von ">vor der
Zeugung war er nicht"< keineswegs f�r abwegig gehalten, weil von
allen das Sein des Gottessohnes vor seiner fleischlichen Zeugung
bekannt wird. Aber schon unser
gottgeliebtester Kaiser hat das Argument daf�r gebracht, da� er
hinsichtlich seiner innerg�ttlichen Zeugung auch vor allen �onen
ist, da er vor seiner wirksamen Zeugung der Macht nach ungezeugt in dem
Vater ist, da der Vater immer Vater ist sowie auch immer K�nig und
immer Erl�ser; alles ist er der Macht nach, und immer bleibt er
in jeder Hinsicht und in jeder Weise gleich.

\kapnum{17}Diese Ausf�hrungen haben wir euch notwendigerweise zugeschickt, Geliebte, 
und euch deutlich das Ergebnis unserer Untersuchung und unsere Zustimmung
vorgestellt, au�erdem, wie wir aus gutem Grund damals und bis zur letzten
Stunde Widerstand geleistet haben, solange uns der fremde Text
anst��ig war, wie wir dann aber ohne Streitsucht den nicht mehr
anst��igen Text akzeptiert haben, als uns der Sinn dieser Worte
nach unserer sorgf�ltigen Pr�fung mit unserem eigenen Bekenntnis in
der vorgelegten Glaubenserkl�rung �bereinzustimmen schien.
% \section{Das Schreiben der Synode von Nic�a an die �gypter}
\kapitel{Brief der Synode von Nicaea an die Bisch�fe �gyptens (Urk.~23)}
\label{ch:23}

\kapnum{1}Der durch Gottes Gnade heiligen und gro�en Kirche Alexandriens und den geliebten Br�dern
in Aegyptus, Libya und der Pentapolis schicken die Bisch�fe, die in Nicaea zusammenkamen und die
gro�e und heilige Synode abhielten, Gr��e im Herrn.

\kapnum{2}Nachdem die Gnade Gottes und der gottgeliebteste Kaiser Konstantin uns aus verschiedenen
Landesteilen und St�dten zusammengef�hrt und die gro�e und heilige Synode in Nicaea stattgefunden
hatte, erschien es uns durchaus geboten, auch euch einen Bericht von der heiligen Synode zu
schicken, damit ihr in der Lage seid zu wissen, was angesprochen und untersucht, ferner was
beschlossen und festgehalten wurde. Zuerst wurden also in Anwesenheit des
gottgeliebtesten Kaisers Konstantin die Probleme mit der gottlosen und gesetzeswidrigen Lehre des
Arius und seiner Anh�nger untersucht.

\kapnum{3}Und alle gemeinsam haben zugestimmt, seine gottlose Lehre und l�sterlichen Thesen und
Worte, die er benutzte, um Gottes Sohn zu beleidigen, zu verurteilen, n�mlich ">aus nichts ist er"<
und ">bevor er gezeugt wurde, war er nicht"< und ">es war einmal, da� er nicht war"<, au�erdem zu
sagen, der Sohn Gottes sei aufgrund seiner Willensfreiheit zu schlechten und tugendhaften Taten
f�hig, und zu vertreten, er sei ein Gesch�pf und ein Werk.

\kapnum{4}Das alles verdammte die heilige Synode und ertrug es nicht einmal in Ans�tzen, die
gottlose Lehre, den Irrsinn und die l�sterlichen Worte zu h�ren. Was aber mit jenem schlie�lich
geschah, habt ihr sicher entweder schon geh�rt oder werdet es noch erfahren, um nicht den Eindruck zu erwecken, da� wir den Mann zu mi�handelten, der f�r seine S�nde die entsprechende Strafe schon empfangen hat.

\kapnum{5}Seine Gottlosigkeit war aber schon so m�chtig geworden, da� auch Theonas von Marmarice
und Secundus von Ptolema"is Schaden nahmen, denn auch jene erhielten dieselben Urteile.
Nachdem aber die Gnade Gottes Aegyptus von jener schlechten Lehre und Blasphemie, auch
von den Personen, die es gewagt hatten, bei dem zuvor friedlichen Volk Aufstand und Zwietracht zu
verursachen, befreit hatte, verblieb noch das Problem der Dreistigkeit des Melitius und der von ihm
Geweihten. Und was auf der Synode �ber dieses Thema beschlossen wurde, wollen wir euch
bekanntmachen, geliebte Br�der.

\kapnum{6}Man beschlo� also, da die Synode von gro�er Menschenliebe ger�hrt war, denn genaugenommen h�tte er keine Milde verdient, da� Melitius in seiner eigenen Stadt bleiben darf, keine
Erlaubnis zu w�hlen oder zu weihen haben soll und da� er zu diesem Zweck in keiner anderen Region oder 
Stadt erscheinen, sondern blo� den Namen der Ehre behalten darf.

\kapnum{7}Die von ihm eingesetzten Personen wurden durch eine geheimnisvollere Handauflegung
best�tigt und in die Gemeinschaft aufgenommen, auf da� sie zwar das Amt innehaben und der Kirche
dienen, generell aber an zweiter Stelle stehen hinter allen, die in jeder Parochie und Kirche von
unserem ehrw�rdigsten Bruder und Mitdiener Alexander gepr�ft und fr�her geweiht worden sind, so da�
sie keinerlei Erlaubnis haben, die Personen, die ihnen gefallen, auszuw�hlen oder Namen
vorzuschlagen oder �berhaupt irgendetwas zu tun ohne Zustimmung des Bischofs der katholischen
Kirche, der von Alexander eingesetzt worden ist.

\kapnum{8}Die aber, von denen man mit Gottes Gnade und euren Gebeten herausfand, da� sie in
keinem Schisma stehen, sondern ohne Schaden in der katholischen und apostolischen Kirche verblieben
sind, haben das Recht, zu w�hlen und Namen vorzuschlagen aus den W�rdigen des Klerus, und
sie d�rfen �berhaupt alles im Rahmen der kirchlichen Gesetze und Satzungen tun.

\kapnum{9}Wenn aber der Fall eintritt, da� einer von denen in der Kirche zu leben aufh�rt, dann
sollen die bereits Wiederaufgenommenen das Amt des Verstorbenen �bernehmen, aber nur, wenn sie
w�rdig erscheinen und das Volk sie ausw�hlt und der alexandrinische Bischof zustimmt und es
besiegelt.

\kapnum{10}Diese Rechte wurden zwar allen anderen einger�umt, aber f�r die Person des Melitius wurde nicht
dasselbe beschlossen wegen seines wiederholten Ungehorsams und seiner leichfertigen und unbesonnenen
Ansichten, damit keinerlei Machtbefugnis oder Eigenst�ndigkeit einem Menschen gegeben werde, der
wieder dieselben Unruhen verursachen k�nnte.

\kapnum{11}Dies sind die Entscheidungen und Sonderbestimmungen f�r �gypten und die heiligste Kirche
Alexandriens; was aber noch anderes festgelegt und beschlossen wurde, zusammen mit unserem Herrn und
ehrw�rdigsten Mitdiener und Bruder Alexander, wird er selbst euch genauestens berichten, wenn er bei
euch sein wird, da er ja Vorsitzender und Teilnehmer der Verhandlungen gewesen ist.

\kapnum{12}Wir verk�nden euch aber auch mit Freude eine Einigung �ber das heilige Osterfest, da auch
diese Angelegenheit aufgrund eurer Gebete in Ordnung gebracht worden ist, so da� alle Br�der im
Osten, die fr�her mit den Juden das Fest begingen, mit den R�mern und euch und uns allen nun
�bereinstimmen, die wir uns von Anfang an bis jetzt daran gehalten haben, zusammen mit euch das
Osterfest zu begehen.

\kapnum{13}Freut euch also �ber die Regelungen, �ber den gemeinsamen Frieden und die Eintracht und
die Vertreibung jeder H�resie, nehmt mit sehr gro�er Verehrung und sehr viel Liebe unseren
Mitdiener, euren Bischof Alexander, auf, der uns mit seiner Anwesenheit erfreut und in so hohem
Alter so viele M�hsal auf sich genommen hat, um Frieden bei euch zu erreichen. 
Betet aber auch f�r uns alle, damit die guten Beschl�sse bestehen bleiben durch den allm�chtigen
Gott und unseren Herrn Jesus Christus im heiligen Geist, dem Ehre von Ewigkeit zu Ewigkeit zukommt.
Amen.
% \section{Symbol der Synode von Nic�a}
\kapitel{Theologische Erkl�rung der Synode von Nicaea (Urk.~24)}
\label{ch:24}

\begin{footnotesize}
Die in Nicaea zusammengekommenen Bisch�fe waren beinahe 300 an der Zahl, verurteilten
die arianische H�esie und verbannten Arius und seine Mitstreiter. Schlie�lich legten sie zur Widerlegung 
jeder H�resie den kirchlichen Glauben schriftlich fest.
\end{footnotesize}

Das Folgende ist in Nicaea beschlossen worden:

Wir glauben an einen Gott, den Vater, den Allm�chtigen, den Sch�pfer aller sichtbaren und unsichtbaren Dinge;
und an einen Herrn Jesus Christus, den Sohn Gottes, als Eingeborener gezeugt aus dem Vater, das
hei�t aus dem Wesen des Vaters, Gott von Gott, Licht von Licht, wahrer Gott von wahrem Gott, gezeugt
und nicht geschaffen, wesenseins mit dem Vater, durch den alles wurde, was im Himmel und auf Erden ist,
der f�r uns Menschen und um unseres Heils willen herabstieg und Fleisch wurde, der Mensch geworden ist,
litt und am dritten Tag auferstand, aufstieg in die Himmel, der kommen wird, um die Lebenden
und die Toten zu richten; und an den heiligen Geist.
Die aber sagen, ">es war einmal, da� er nicht war"< oder ">er war nicht, bevor er gezeugt
wurde"< oder ">aus dem Nichts wurde er"< oder die behaupten, er sei aus einer anderen Hypostase oder
einem anderen Wesen, oder aber sagen, der Sohn Gottes sei geschaffen, wandelbar oder ver�nderlich, diese verdammt
die katholische und apostolische Kirche.
\clearpage
% \section{Brief Kaiser Konstantins an Arius und Genossen}
\kapitel[Brief des Kaisers Konstantin an Arius und seine Anh�nger (Urk.~34)][Brief des Kaisers Konstantin an Arius und seine Anh�nger (Urk.~34)]{Brief des Kaisers Konstantin an Arius und seine Anh�nger (Urk.~34)\protect\footnote{Die Emendationen von \cite{Scheidweiler:Byz}, sind stillschweigend ber�cksichtigt, vgl. auch \cite[xcvii]{Brennecke2006}.}}
\label{ch:34}
\thispagestyle{empty}

Konstantin Augustus an Arius und seine Anh�nger, die Arianer!

\kapnum{1}Ein schlechter Interpret ist in Wahrheit ein Abbild und eine Statue des Teufels.
Genau wie n�mlich die geschickten Bildhauer jenen (Teufel) als t�uschenden Lockvogel
darstellen, indem sie ihn mit scheinbarer Sch�nheit versehen, obwohl er von Natur aus ganz
h��lich ist, um die Geplagten zu verderben und sie dem Irrtum auszuliefern, auf dieselbe
Art, meine ich, handelt der, der sich nur daf�r einzusetzen scheint, skrupellos das Gift
der gewohnten Dreistigkeit zu verbreiten.

\kapnum{2}Denn er f�hrt einen neuen und einen, seitdem es Menschen gibt, noch nie
dagewesenen ungl�ubigen Glauben ein; daher scheint jene Aussage, die schon lange zu den
g�ttlichen Spr�chen geh�rt, n�mlich da� sie dem Schlechten gegen�ber treu bleiben, nicht
die Wahrheit zu verfehlen.

\kapnum{3}Was soll man dazu sagen, wenn jemand die Gabe zur Selbstkritik verloren hat und
nicht mehr danach strebt, irgendeine Hilfe zur Unterst�tzung zu finden? Warum rufe ich
also: ">Christus! Christus! Herr! Herr!"<? Warum verwundet uns denn t�glich diese gottlose
R�uberbande? Es steht uns eine gewaltig r�cksichtslose Verwegenheit entgegen, sie br�llt
und knirscht mit den Z�hnen, entstellt vor Schande und verwundet von diversen Anklagen.

\kapnum{4}Diese Verwegenheit wird zwischen dem Gesetz und deiner Predigt wie zwischen
irgendwelchen zerst�rerischen St�rmen und Wellen hin- und hergeschleudert und gibt
verdorbene W�rter von sich; sie scheint aber Dinge zu schreiben, die nicht einmal du, der du mit dem Ewigen deiner eigenen Quelle, dem Vater,
zusammen existierst, in deiner Selbstreflexion definiert hast. �berhaupt tr�gt und bringt sie irgendwelche f�rchterlichen und
gesetzwidrigen Gottlosigkeiten zusammen, manchmal indem sie selbst gro�e Reden schwingt,
manchmal wiederum, indem sie sich auf dem Eifer der Ungl�cklichen emportragen l��t, die
sie selbst, wenn diese arglos anwesend sind, betr�gt und vernichtet.

\kapnum{5}Jetzt m�chte ich mich aber mit der Natur ihres Vorsitzenden befassen. Denn was
sagt er also? Er sagt: ">Entweder la�t uns festhalten, was wir schon
besitzen, oder es geschehe so, wie wir selbst es wollen."< Er ist gefallen, und er ist auch
bez�glich dieser Pl�ne gefallen, durch List, wie er sagt, und durch ein falsches Spiel
beseitigt. Das ist egal. Er sch�tzt sowieso nur das, was ihm in b�ser Absicht zuflie�t. ">Wir haben die Mehrheit"<, sagt er.

\kapnum{6}Da m�chte ich mich nun selbst ein wenig in die erste Reihe stellen, um
Zuschauer der Kriege der Besessenheit zu werden. ">Ich selbst"<, sagte ich, ">will
vortreten, der ich gewohnt bin, die Kriege der t�richten Leute zu beenden."< Los jetzt,
Ares Arius, Schilde tun Not! Nein, tu dies nicht, wir bitten dich! Der Umgang mit
Aphrodite m�ge dich bremsen! Ach da� es sich so f�r dich ziehmte, in der Fr�mmigkeit zu Christus zu bl�hen, wie du dich so hervorragend beim Volk einzubringen scheinst.

\kapnum{7}Siehe, ich komme jetzt noch einmal als Flehender und will nicht mit Waffen
k�mpfen, auch wenn ich die Macht dazu habe; ich will n�mlich, gewappnet mit dem Glauben an
Christus, dich heilen und die anderen gesund pflegen.

\kapnum{8}Warum bejahst du also, diese Dinge zu tun, die nicht deinen Gewohnheiten
entsprechen? Oder sage mir, mit welcher Ruhe oder mit welcher �berlegenheit bist du
umgeben, oder mehr noch, zu welcher Frechheit hat es dich hingetrieben? Oh, die
Verwegenheit sollte durch ein passendes Blitzgewitter ausgemerzt werden! Denn h�rt, was er
mir k�rzlich erkl�rte, wobei er mit einem gifttriefenden Stift schrieb: ">So glauben
wir"<, sagte er. Dann f�gte er, glaube ich, irgendwelche ich wei� nicht wie schw�lstige
und �beraus penible Ausf�hrungen hinzu, ging dann �ber zu noch abwegigeren Dingen und
verschwieg nichts von seinen ruchlosen Sachen, sondern �ffnete, so d�rfte man es wohl
bezeichnen, seinen ganzen Schatz an Wahnsinn. Er sagte: ">Wir werden vertrieben, und sie
heben die Erlaubnis auf, uns aufzunehmen."<

\kapnum{9}Aber dies tut nichts zur Sache; achtet aber auf das Folgende, ich werde n�mlich
seine Worte verwenden: ">Wir bitten dich, wenn der Bischof von Alexandrien bei seiner
Meinung bleibt, da� uns schlie�lich die Erlaubnis gegeben werde, nach der Vorschrift des
Gesetzes Gott die rechtm��ige und gebotene Verehrung entgegenzubringen."<

\kapnum{10}Oh, welches Ausma� an Schamlosigkeit, die man mit Eifer f�r die Wahrheit aufdecken
mu�. Denn was ihm im Moment Vergn�gen bereitet, das gibt er in kurzen S�tzen wieder. Was
redest du, Verr�ckter! Arbeitest du darauf hin, eine Spaltung, die uns gefallen w�rde, mit
dem Elend deiner kaputten Gedanken zu vertiefen, und trachtest du danach, die zu verderben,
die dir in deine Bosheit gefolgt sind?

\kapnum{11}Du sagst: ">Was soll ich machen, wenn keiner bef�rwortet, mich aufzunehmen?"<
Denn das rufst du oft aus der gottlosen Kehle heraus. Ich aber werde dir entgegenhalten:
">Wo zeigtest du einen klaren Hinweis und Beleg f�r deine eigene Ansicht?"< Diese h�ttest du
enth�llen und den g�ttlichen und menschlichen Wesen deutlich klarstellen m�ssen. Giftige Schlangen
werden besonders dann �berm��ig zornig, wenn sie merken, da� sie selbst in die hinterste
Ecke der H�hle getrieben werden.

\kapnum{12}Eine Sache aber ist in der Tat ziemlich gewitzt von ihm, da� er sehr gut und eifrig wie unter einer Maske der Scham Schweigen vort�uscht. Durch Schauspielerei gibst du
dich zwar als zahm und f�gsam, h�ltst aber vor der Menge die F�lle von
Bosheiten verborgen, die du insgeheim tausendfach im Schilde f�hrst. Oh Ungl�ck! Wie es der
B�se gewollt hat, so hat sich Arius f�r uns als Werkstatt von Gesetzwidrigkeiten bereitgestellt.

\kapnum{13}Komm also heraus und sage mir jetzt die Erkennungsmerkmale deines eigenen
Glaubens und verschweige sie nicht, oh du entstellte Fratze und zu Boshaftigkeit zu reizende Natur! Du sagst: ">ein Gott"<? Du hast meine Zustimmung, denke so! Du sagst: ">Das Wort ist in seinem Wesen anfangs- und endlos"<? Das gef�llt mir, glaube so!

\kapnum{14}Wenn du noch etwas dar�ber hinaus hinzuf�gst, so erkl�re ich das f�r null und
nichtig. Wenn du etwas in Bezug auf die gottlose Trennung zusammenschusterst, gestehe ich,
dies zu �bersehen und zu �berh�ren. Wenn du ">die Fremdheit des Leibes in Bezug auf den
Heilsplan der g�ttlichen Kr�fte"< �bernimmst, so weise ich das nicht zur�ck. Wenn du sagst:
">Der Geist der Ewigkeit ist in dem �berlegenen Wort geworden"<, akzeptiere ich das. Wer
hat den Vater gekannt, wenn nicht der, der vom Vater gekommen ist?
Wen hat der Vater gekannt, wenn nicht den, den er ewig und anfangslos aus sich gezeugt hat?
Du, der du ohne Zweifel falsch glaubst, meinst, man m�sse eine ">fremde
Hypostase"< unterordnen, ich aber wei�, da� die F�lle der einzigartigen und alles
durchdringenden Macht des Vaters und des Sohnes ein Wesen ist.

\kapnum{15}Wenn du also dem Sohn von jener F�lle wegnimmst, von der niemals etwas
losgetrennt werden kann, nicht einmal durch einen Gedanken der Streits�chtigen, dann
entwirfst du die Eigenschaften des Zuwachses  und steckst f�r jenen die Eckdaten der wissenschaftlichen Untersuchungen fest, der im
Ganzen aus sich heraus die Ewigkeit stiftet, den unverweslichen Gedanken, der durch sich
selbst den Glauben an die Unsterblichkeit und die Kirche ausgeteilt hat. Wirf es weg
jetzt, wirf weg dieses t�richte Unrecht, du Durchtriebener und Sch�nredner, du Verk�nder
der schlechten Ideen, die zum Unglauben der Unverst�ndigen f�hren!

\kapnum{16}Zu Recht hat dich also der B�se zu seiner eigenen Schlechtigkeit
bekehrt, und denen, denen derartiges vielleicht angenehm zu sein scheint (\dag), ist dieses �bel ganz und gar verderblich.

\kapnum{17}Los jetzt, nimm Abstand von der Besch�ftigung mit verkehrten Dingen und h�r
zu, du teuflischer Arius! Mit dir rede ich n�mlich. Bemerkst du nicht, da� du tats�chlich
aus der Kirche Gottes ausgeschlossen worden bist? Du bist verloren, verstehe es wohl, wenn
du nicht auf dich schaust und deinen jetzigen Wahnsinn verurteilst. Aber du wirst sagen,
da� der P�bel auf deiner Seite steht und den Kummer mittr�gt.

\kapnum{18}Nun h�re doch, leih uns dein Ohr, gottloser Arius, und erkenne deine
eigene Unvernunft. Du aber, du Besch�tzer von allen, Gott, sei wohlwollend gegen�ber dem
Gesagten, wenn es sich im Rahmen des Glaubens bewegt. Denn ich, dein Mensch, der ich von
dir gn�dige Unterst�tzung erfahre, werde aus uralten griechischen und r�mischen Texten
eindeutig aufzeigen, da� der Wahnsinn des Arius vor ungef�hr dreitausend Jahren von erythr�ischen (Sibylle) vorhergesagt und prophezeit worden ist.

\kapnum{19}Jene sagte n�mlich: ">Wehe dir, Libyen, am Meeresstrand gelegen. Denn es wird
f�r dich eine Zeit kommen, in der du gezwungen sein wirst, mit dem Volk und deinen eigenen
T�chtern in einen gewaltigen, grausamen und sehr schweren Kampf einzutreten, wodurch alle
vor eine Pr�fung des Glaubens und der Fr�mmigkeit gestellt werden, f�r dich aber wird es
sich als Endkatastrophe erweisen. Denn ihr habt es gewagt, das Lager der himmlischen
Bl�ten aufzubrechen, es mit Bissen zu zerrei�en und es wahrlich mit eisernen Z�hnen zu
zersto�en."<

\kapnum{20}Was nun, oh du Schurke? Wo auf der Erde bekennst du dich aufzuhalten? Doch
genau dort! Ich besitze n�mlich deinen Brief, den du mit einer Schreibfeder aus Wahnsinn
an mich geschrieben hast, in dem du sagst, da� das ganze libysche Volk dir in Bezug auf
die Erl�sung zustimme. Solltest du aber nicht gestehen, da� diese Dinge so liegen, so
werde ich, Gott sei mein Zeuge, wirklich das uralte Buch der Erythraia, in griechischer
Sprache verfa�t, nach Alexandrien schicken, so da� du bald auf verlorenem Posten stehen
wirst.

\kapnum{21}Bist du etwa unschuldig, du Dickkopf? Hast du etwa nicht auf der ganzen Linie
verloren, du Elender, umzingelt von einem so gewaltigen Spruch? Wir wissen es, wir wissen
von deinen Aktivit�ten. Welche Sorge, welche Furcht dich umgibt, das ist uns nicht
verborgen. Oh, du Elender und Geplagter, was f�r ein Stumpfsinn deines Verstandes, der du nicht einmal die
Krankheit und Unf�higkeit deiner eigenen Seele verjagst. Du Gottloser, der du die Wahrheit
mit vielerlei Reden untergr�bst und dich bei diesem Verhalten nicht sch�mst, uns zu tadeln
und einmal zu widerlegen, wie es dir gef�llt, und dann wieder zu ermahnen. Du Angeber im
Glauben und in Reden, dem die Elenden auch noch hinterherlaufen, um sich Hilfe zu
verschaffen.

\kapnum{22}Eigentlich sollte man sich nicht einmal mit einem derartigen Menschen
abgeben, ihn nicht einmal ansprechen, es sei denn, einer glaubt, in dessen heimt�ckischen
Worten l�ge f�r die Ma�vollen Hoffnung auf rechtes Leben.

\kapnum{23}Aber weit gefehlt, dies ist nicht das Wahre; oh, was f�r eine Torheitvon euch, die ihr Euch
mit diesem abgebt! Welcher Ingrimm hat Euch dazu gezwungen, dessen verbitterte Zunge und
Gesicht zu ertragen?

\kapnum{24}Sei es drum. Ich werde nun aber das Wort an dich selbst richten. Oh, du
t�richte Seele, du geschw�tzige Zunge, du ungl�ubiger Geist, bereite mit Worten kein,
wie ich sage, ausgedehntes, breites Gel�nde, sondern einen gut begrenzten, nicht morsch gewordenen, sondern standhaften und der Natur nach harten Kreis, du Gottloser, du Schurke und
Durchtriebener! Denn ich hebe an, dies zu sagen: Viel lieber noch werde ich eine
Schlinge um dich legen, dich mit Worten fesseln und in die Mitte stellen, so da� das ganze
Volk deine Schlechtigkeit erfahre.

\kapnum{25}Nun aber la�t uns die Sache selbst angehen. Die H�nde sind bereits gewaschen. La�t
uns zum Gebet gehen! Rufe Gott an! Besser, warte noch kurz, sage mir, du
Hinterlistiger, welchen Gott du zu Hilfe rufst? 

\kapnum{26}Doch ich kann nicht still halten: Oh, M�chtiger, der du Herrschaft �ber alles hast, Vater der einzigen Kraft,
Wegen dieses Unseligen erleidet deine Kirche Schmach, Striemen, auch Wunden und
Schmerzen. Arius w�hlt dir~-- und freilich sehr begabt~-- schon einen passenden Platz aus, auf dem sitzend, glaube ich, du dir deinen Christus, der aus dir ist und der der Anf�hrer unserer Hilfe ist, als Mitgott und Sohn durch Adaptionsrecht verschafft hast und nun besitzt.

\kapnum{27}Vernimm, ich bitte dich, seinen seltsamen Glauben. Er glaubt, Herr, da� du
eine r�umliche Bewegung vollziehst. Er wagt es, dich durch einen Kreis eines begrenzten Sitzes zu
beschr�nken. Wo fehlt denn deine Gegenwart? Oder wo nehmen denn nicht alle dein Wirken 
aus deinen sich �berall hin erstreckenden Gesetzen wahr? Du umfa�t n�mlich alles selbst, und es ist nicht rechtens, sich au�erhalb von dir einen Raum oder irgendetwas anderes
vorzustellen. Deine Macht ist mit ihrer Wirksamkeit in dieser Hinsicht unendlich.

\kapnum{28}H�re jetzt zu, Gott, ihr aber, das ganze Volk, pa�t auf! Denn dieser schamlose
und nutzlose Mensch sch�tzt Fr�mmigkeit vor, w�hrend er Verworfenheit und
Gesetzlosigkeit auf die Spitze treibt.

\kapnum{29}">Wohlan"<, sagt er, ">ich will nicht, da� Gott Leid durch Gewalt zu ertragen scheint.
Deswegen behaupte und erfinde ich f�r den Glauben Seltsames, da� Gott sich n�mlich eine
Hilfe besorgte, indem er sich ein neugewordenes, neuerschaffenes Wesen, Christus,
einrichtete, wie es mir scheint. Was du n�mlich von ihm wegnimmst, um diesen Teil machst
du ihn kleiner"<, sagt er.

\kapnum{30}Ist dies also f�r dich der Glaube, du Bedr�nger und Verderber? Du h�ltst in deiner Annahme auch den f�r ein Gebilde, der die Gebilde der Heiden
verurteilt? Du nennst den nachtr�glich Hinzugekommenen und gleichsam Diener f�r das, was zu tun ist, der ohne �berlegung
und Gedanken alles vollendet, dadurch, da� er mit der Ewigkeit des Vaters zusammen ist? F�ge jetzt also, wenn du es wagst, f�ge, sage ich, Gott auch noch das Sorgen um und das Hoffen auf den Ausgang hinzu, noch dazu das Nachdenken, das Reflektieren, das �u�ern seiner Meinung �ber das Untersuchte, das Darlegen und �berhaupt das Sich-Freuen, das Lachen und das Trauern.

\kapnum{31}Was sagst du jetzt, du gr��te Jammergestalt, du leibhaftiger Erreger des
B�sen? Verstehe endlich, falls du dazu in der Lage bist, da� du �berf�hrt bist, elend in
deine List verstrickt zu sein.

\kapnum{32}Er sagt: ">Christus hat f�r uns gelitten."< Aber ich habe oben schon gesagt,
da� er in Gestalt des Leibes geschickt worden ist. ">Ja"<, sagt er, ">aber es ist zu
f�rchten, da� wir ihn irgendwie zu erniedrigen scheinen."< Dann, du Mittler wilder
Tiere, bist du etwa nicht eindeutig verr�ckt und rasend, wenn du dies sagst? Denn
siehe doch, der Kosmos ist selbst eine Gestalt oder er ist doch ein K�rper, und die
Sterne sind freilich als Pr�gebilder davor angebracht, und �berhaupt ist der Geist dieses
kugelartigen Kreises ein Abbild des Seienden und gleichsam eine Gestalt. Und doch ist Gott �berall
da. Wo sind also in Gott die Mi�handlungen? Oder inwiefern wird Gott vermindert?

\kapnum{33}Oh, du Vaterm�rder des Schicklichen, �berlege bitte doch anhand seiner Beweise und
bedenke, ob dies S�nde zu sein scheint, da� Gott in Christus ist. Jener sah n�mlich die
Schande des Wortes und hat schnell eine Bestrafung veranla�t. Au�erdem gibt es t�glich
S�nden in der Welt. Und dennoch ist Gott da und l��t keinen Richterspruch aus. Wie wird er
also dadurch erniedrigt, wenn die Gr��e seiner Kraft �berall zu erkennen ist? Gar
nicht, glaube ich.

\kapnum{34}Denn der Weltgeist besteht durch Gott. Durch ihn bleibt alles erhalten.
Durch ihn gibt es jedes Gericht. Der Glaube an Christus ist ohne Anfang aus ihm. Christus
ist das ganze Gesetz Gottes; durch ihn ist es unbegrenzt und endlos.

\kapnum{35}Aber du scheinst nur deiner Logik zu folgen, oh du �berma� an Wahnsinn!
Wende dich jetzt deinem Verderben hin, du Schwert des Teufels! Seht doch her,
seht alle her, welch kl�gliche T�ne er festgehalten vom Bi� der Schlange
ausst��t, wie seine Adern und Muskeln vom Gift ergriffen wurden und sich in
gro�en Schmerzen hin- und herwerfen, wie sein K�rper dahinschwand, g�nzlich abgemagert! Er
ist voller Verwilderung, Dreck, Wehklagen, Bl�sse, Zittern und unz�hliger Bosheiten und
schwindet gr��lich dahin~-- wie scheu�lich und besudelt ist die Menge seines Haares, wie
beinahe halbtot und geschw�cht schon sein Blick, wie blutleer sein Gesicht und vor Sorge
gequ�lt, als wenn alles gleichzeitig auf ihn einst�rzte. Ingrimm, Wahnsinn und Torheit, sie haben dich
w�hrend der Zeit deines Leidens verroht und verwildert.

\kapnum{36}Weil er es im Moment nicht einmal bemerkt, wie schlecht es ihm geht, so ruft
er: ">Ich springe vor Freude und h�pfe aus Vergn�gen und fliege hoch!"< Dann sagt er
wieder ganz unreif: ">So sei es, dann sind wir eben verloren."<

\kapnum{37}Das ist allerdings wahr. Die Bosheit hat allein dir deinen Eifer f�r sich
gro�z�gig erstattet und dir das leichthin �bergeben, was sonst durch gro�en Einsatz
erworben wird. Wohlan jetzt, sage nun, wo sind deine feierlichen Ank�ndigungen? Wasch
dich doch im Nil, wenn es geht, oh du Mensch voller widersinnigem Stumpfsinn. Du hast in
der Tat erreicht, den ganzen Erdkreis durch deine Gottlosigkeit zu verwirren.

\kapnum{38}Siehst du denn etwa nicht ein, da� ich, ein Mensch Gottes, �ber alles genau
Bescheid wei�? Aber ich wei� nicht, ob es Not tut, zu bleiben oder fortzugehen. Denn ich kann
diesen nicht mehr sehen und scheue die S�nde, Arius, du Ares! Du hast uns zwar ins Licht
gezerrt, dich selbst aber in die Finsternis geworfen, du Elender! Das scheint das Ende deiner
M�hen.

\kapnum{39}Aber ich will noch einmal darauf zur�ckkommen. Du sagst, es g�be eine Menge von denen, die um dich herumschweifen. 
In Ordnung, denke ich, nimm diese nur auf, nimm sie, sage ich. Denn sie
haben sich selbst den W�lfen und L�wen zum Fra� hingegeben. Ferner ist jedem von diesen
die zehnfache Kopfsteuer auferlegt, und bedr�ngt von diesen Kosten wird jeder sogleich
sehr ins Schwitzen geraten, wenn er nicht auf dem schnellsten Weg zur heilsamen Kirche l�uft und zum
Frieden der Liebe durch den Liebestrank der Harmonie zur�ckkehrt.

\kapnum{40}Denn sie werden nicht mehr von dir, der von einem schlechten Gewissen abgeurteilt ist,
in die Irre gef�hrt werden, ferner werden sie es nicht mehr ertragen, verloren zu gehen, weil sie
noch deinen verbrecherischen �berlegungen anh�ngen. Klar und deutlich sichbar sind in
Zukunft allen Menschen deine T�uschungsman�ver. Und du wirst ohne Zweifel nichts mehr erreichen
k�nnen, sondern trittst umsonst in Erscheinung, spielst Milde und Sanftmut mit Worten vor
und umgibst dich nach au�en sozusagen mit einer Maske der Einfalt. Alle deine 
Kunstgriffe werden vergeblich sein! Denn sogleich wird dich die Wahrheit umstellen! Der
Regen der Macht wird sozusagen sofort deine Flammen l�schen.

\kapnum{41}Und die F�rsorge um die �ffentlichen Dienste greift f�rwahr nach deinen Gef�hrten, deinen Gesinnungsgenossen, die deinem Willen schon anheimgefallen sind, es sei denn, sie
entfliehen schnellstens deiner Gemeinschaft und tauschen sie gegen den unverderblichen Glauben
ein.

\kapnum{42}Du aber, du eisenharter Mann, gib mir einen Beweis deiner Gesinnung, wenn du
dir selbst vetraust, stark bist in einem festen Glauben und ein ganz und gar reines Gewissen
hast. Komm zu mir, komm, sage ich, zu einem Menschen Gottes. Vertraue, da� ich durch meine
eigenen Fragen die Geheimnisse deines Herzens erforsche. Auch falls n�mlich irgendetwas
Verr�cktes darinnen zu sein scheint, werde ich dich unter Anrufung der g�ttlichen Gnade
besser wie nur irgendein Beispiel heilen. Falls du aber in der Seele gesund zu sein scheinst, werde
ich das Licht der Wahrheit in dir wahrnehmen und Gottes Gnade sehen und zugleich auch
meine eigene Gottesfurcht erkennen.

\kapnum{43}\textit{Und mit anderer Hand:} Gott m�ge Euch beh�ten, geliebte Br�der.

\begin{footnotesize}
Durch die Beamten Synkletius und Gaudentius wurde dies �bermittelt und im Palast verlesen, als Paterius Eparch von
�gypten war.
\end{footnotesize}
\clearpage
% \section{Das Edikt gegen Arius}
\kapitel{Fragment eines Ediktes des Kaisers Konstantin gegen Arius und seine Anh�nger (Urk.~33)}
\label{ch:33}
\thispagestyle{empty}

Der Sieger Konstantin der Gro�e, Augustus, an die Bisch�fe und an das Volk!

\kapnum{1}Arius,\looseness=-1\ der Schlechte und Gottlose nachgeahmt hat, ertr�gt zu Recht dieselbe Schmach wie
jene. Wie n�mlich Porphyrius, der Feind der Gottesf�rchtigkeit, diverse gesetzwidrige Gesch�tze
gegen die Gottesverehrung aufgeboten und daf�r den entsprechenden Lohn gefunden hat, so da� er
von der Nachwelt verschm�ht wird und in �beraus schlechtem Ruf steht und seine gottlosen B�cher von
der Bildfl�che verschwunden sind, so gefiel es auch jetzt, Arius und seine Anh�nger ebenfalls
Porphyrianer zu nennen, damit sie ihren Namen nach denen tragen, deren Art sie nachgeahmt
haben.

\kapnum{2}Ferner aber, wenn irgendwelche von Arius verfa�ten Texte gefunden werden, sind diese dem Feuer
zu �bergeben, damit nicht nur das �bel seiner Lehre verschwindet, sondern auch �berhaupt keine
Erinnerung an ihn mehr verbleibt. Dar�berhinaus bestimme ich, da�, wenn jemand dabei ertappt wird,
wie er von Arius verfa�te Texte verbirgt und nicht sogleich hervorholt und ins Feuer wirft, diesen
die Todesstrafe treffen soll. Sobald er dessen �berf�hrt wird, soll er die Todesstrafe erleiden.

\textit{Und mit anderer Hand:} Gott m�ge euch bewahren, geliebte Br�der!
% \section{Brief Kaiser Konstantins an die alexandrinische Gemeinde}
\kapitel{Brief des Kaisers Konstantin an die Kirche von Alexandria (Urk.~25)}
\label{ch:25}
\thispagestyle{empty}

Konstantin Augustus an die katholische Kirche der Alexandriner!

\kapnum{1}Seid\looseness=1\ gegr��t, geliebte Br�der! Wir haben von der g�ttlichen Vorsehung vollkommene
Gnadengaben empfangen, damit wir jeden Irrtum von uns fernhalten und ein und denselben Glauben
anerkennen.

\kapnum{2}Schlie�lich vermag der Teufel nichts gegen uns auszurichten. Jede Bosheit, die er gegen
uns auszuhecken versuchte, wurde mit Stumpf und Stiel ausgerotten. Der Glanz der Wahrheit besiegte auf
den Befehl Gottes hin die Zwietracht, die Schismen, jene Tumulte und das, um es so auszudr�cken,
t�dliche Gift der Uneinigkeit. Wir alle verehren also einen Gott dem Namen nach und glauben, da� er einer ist.

\kapnum{3}Um dies aber zu erreichen, habe ich auf Grund der Ermahnung Gottes sehr viele Bisch�fe in die
Stadt Nicaea zusammengerufen, mit denen ich selbst, der ich mich �berm��ig freue, gleichsam einer von euch, eurer Mitdiener zu sein, die Untersuchung der Wahrheit auf mich nahm.

\kapnum{4}Alles was Zweifel oder Anla� f�r Streit zu erzeugen schienen, wurde genau
untersucht und widerlegt. Und die g�ttliche Majest�t m�ge dar�ber hinwegsehen, wie gro�e und
f�rchterliche Blasphemien einige �ber unseren Erl�ser, �ber unsere Hoffnung und unser Leben
vorbrachten und sogar zugaben und bekannten, im Widerspruch zu den g�ttlichen Schriften und dem
heiligen Glauben zu glauben.

\kapnum{5}W�hrend also �ber dreihundert Bisch�fe, bewundert f�r ihre Besonnenheit und Urteilsf�higkeit, ein
und denselben Glauben bekr�ftigten, der auch in Wahrheit ein Glaube genau nach g�ttlichem Gesetz ist, wurde
deutlich, da� allein Arius von der teuf"|lischen Kraft �berw�ltigt worden war und dieses �bel zuerst bei
euch, dann aber auch bei anderen mit gottlosen Hintergedanken verbreitet hatte.

\kapnum{6}La�t uns also die Erkenntnis annehmen, die der Allm�chtige gew�hrt hat. La�t uns zu
unseren geliebten Br�dern zur�ckkehren, die ein unversch�mter Diener des Teufels von uns weggelockt hat. 
La�t uns mit vollem Einsatz zum gemeinsamen Leib und unseren echten Gliedern hineilen.

\kapnum{7}Dies ist angemessen f�r euren Scharfsinn, f�r euren Glauben und f�r eure
Fr�mmigkeit, da� ihr zur g�ttlichen Gnade zur�ckkehrt, nachdem jener, von dem feststeht, da� er ein Feind der Wahrheit ist, seines Irrtums �berf�hrt worden ist.

\kapnum{8}Denn was den dreihundert Bisch�fen zugleich gefallen hat, das ist nichts anderes als der Wille
Gottes, besonders da doch wohl der heilige Geist hinter den Ansichten dieser so bedeutenden M�nner
steht und den g�ttlichen Willen aufleuchten l��t.

\kapnum{9}Daher soll niemand Zweifel hegen, niemand soll dar�ber hinweggehen, sondern kehrt alle
entschlossen auf den Weg der Wahrheit zur�ck, damit ich, wenn ich schon bald zu euch komme,
zusammen mit euch den geschuldeten Dank dem allm�chtigen Gott bekennen kann, weil er den reinen
Glauben gezeigt und uns die ersehnte Liebe zur�ckgegeben hat.

\textit{Und mit anderer Hand:} Gott m�ge euch bewahren, geliebte Br�der!
% \clearpage
% \section{Brief Kaiser Konstantins an die Gemeinde �ber die nic�nischen Beschl�sse zum
% Ostertermin}
\kapitel{Rundbrief des Kaisers Konstantin (Urk.~26)}
\label{ch:26}

Konstantin Augustus an die Kirchen.

\kapnum{1}Nachdem ich aus der Wohlfahrt des Gemeinwesens erfahren habe, wie gro�e Gnade der g�ttlichen
Macht erwachsen ist, entschlo� ich mich, vor allen Dingen daf�r zu sorgen, da� beim seligen Volk der
katholischen Kirche ein Glaube, reine Liebe und einm�tige Fr�mmigkeit gegen�ber dem allm�chtigen Gott
bewahrt werden.

\kapnum{2}Da es aber nicht m�glich war, eine unersch�tterliche und feste Ordnung zu errichten, es
sei denn alle oder doch die meisten Bisch�fe kommen an einem Ort zusammen und es kommt zu einer
Besprechung von jeder umstrittenen Frage �ber die heiligste Gottesverehrung, haben sich deswegen
die meisten Bisch�fe getroffen (auch ich selbst war wie einer von euch mit dabei; und ich
m�chte nicht leugnen, da� ich mich dar�ber sehr gefreut habe, euer Mitdiener zu sein), und jede
umstrittene Frage wurde so lange diskutiert, bis eine Meinung zu Tage gef�rdert wurde, die einer
einstimmigen �bereinkunft gen�gt und die Gott, dem H�ter des Alls, gef�llt, so da� nichts mehr
verblieb, was Zwietracht und Glaubenszweifel verursachen k�nnte.

\kapnum{3}In diesem Zusammenhang wurde auch �ber den heiligsten Tag des Osterfestes diskutiert, und
es schien im Sinn der allgemeinen Meinung zu sein, da� alle Christen das Osterfest �berall an einem Tag
begehen. Denn was kann es Sch�neres, was Ehrw�rdigeres geben, als da� dieser Festtag, von dem wir
die Hoffnung auf Auferstehung empfangen haben, von allen fehlerlos nach einer Ordnung und nach einer
klaren Berechnung eingehalten wird.
Erstens schien es uns n�mlich unw�rdig zu sein, jenes heiligste Fest im Anschlu� an die Gewohnheit
der Juden zu feiern, den Verruchten, die ihre eigenen H�nde mit frevelhaftem Vergehen beschmutzt und
zu Recht erblindete Seelen haben. Denn, nachdem jenes Volk verworfen worden ist, ist es m�glich, 
die Erf�llung dieses Abhaltens (sc. des Festes) auch in k�nftigen Zeiten nach einer eher wahrhaftigen Ordnung, 
die wir vom ersten Leidenstag an bis jetzt bewahrt haben, auszurichten.

\kapnum{4}Nichts sollen wir also mit dem feindseligen Volk der Juden gemein haben. Wir haben
n�mlich vom Erl�ser einen anderen Weg empfangen; f�r unsere heiligste Gottesverehrung gibt es einen Weg, der sowohl gesetzm��ig als auch angemessen ist. La�t uns diesen einm�tig entlanggehen und uns von jener
h��lichen Gewohnheit fernhalten, verehrte Br�der!

\kapnum{5}Denn es ist doch wirklich unm�glich, da� jene prahlen, wir seien nicht in der Lage, ohne
ihre Lehre dieses Fest auszurichten. Was aber k�nnen sie schon Richtiges denken, die nach
jenem Mord am Herrn und Vater den Verstand verloren haben und nicht mehr von irgendeiner
vern�nftigen Erw�gung, sondern von ma�loser Begierde dahin geleitet werden, wohin auch immer ihr 
angeborener Wahnsinn sie f�hrt! Daher erkennen sie n�mlich auch in dieser Frage nicht die Wahrheit,
so da� sie, die sich immer im �berma� irren, anstelle einer entsprechenden Anpassung das
Passafest in demselben Jahr zweimal feiern.

\kapnum{6}Weswegen sollen wir also diesen folgen, die bekanntlich an einem gro�en Irrtum
erkrankt sind? Denn zweimal in ein und demselben Jahr Passa zu feiern, werden wir niemals
annehmen. Aber auch wenn das nicht w�re, so w�re es notwendig, da� ihr euren
Scharfsinn allzeit mit Eifer und Gebeten daf�r gebraucht, da� sich die Reinheit eurer Seele in
keinem Bereich mit den Sitten so �bler Menschen zu �berschneiden scheint.

\kapnum{7}Zus�tzlich ist auch folgendes zu beachten, da� es in einer derart bedeutenden Sache
und bei der Feier eines so wichtigen Gottesdienstes nicht angebracht ist, verschiedener Meinung zu
sein.

\kapnum{8}Denn unser Erl�ser hat uns einen Tag f�r unsere Befreiung, d.\,h. einen Tag seines so
heiligen Leidens �berliefert. Er wollte, da� seine katholische Kirche eine sei, die, auch wenn die
meisten Teile an vielen und unterschiedlichen Orten zerstreut sind, dennoch von einem Geist, d.\,h.
einem g�ttlichen Willen, umsorgt wird.

\kapnum{9}Der Scharfsinn eurer Fr�mmigkeit m�ge ferner dar�ber nachdenken, wie schlimm und
unanst�ndig es ist, wenn an denselben Tagen die einen die Fastengebote befolgen, die anderen aber
gemeinsame Festmahle abhalten, und wenn nach den Ostertagen die einen sich in Feiern und Erholung
�ben, die anderen aber die vorgeschriebenen Fastentage einhalten. Aus diesem Grund will die
g�ttliche Vorsehung, da� diese Zust�nde eine angemessene Verbesserung erfahren und auf eine Linie
gebracht werden, was alle, wie ich meine, auch einsehen.

\kapnum{10}Da es also n�tig war, diese Angelegenheit so zu verbessern, da� keine Gemeinsamkeiten
mit jenen Vater- und Herrenm�rdern bestehen bleiben, ist eine Regelung angemessen, die
alle Kirchen aus den westlichen, s�dlichen und n�rdlichen Teilen der bewohnten Welt
sowie einige der �stlichen Orte einhalten, weswegen alle meinten, es w�re zum jetzigen Zeitpunkt gut, und weswegen ich selbst versprach,eurem Scharfsinn zu gen�gen, damit das, was in der Stadt Rom, in ganz Italien und Africa, in Aegyptus, Spania,
Gallia, Britannia, Libya, in ganz Griechenland, in den Di�zesen Asia und Pontus und in Cilicia
mit einer �bereinstimmenden Meinung eingehalten wird, auch freudig eure
Einsicht annimmt, wenn sie dar�ber nachdenkt, da� nicht
nur die Zahl der Kirchen an den vorgenannten Orten gr��er ist, sondern da� es auch ganz
au�erordentlich gottesf�rchtig ist, wenn alle gemeinsam das wollen, was auch eine genaue Berechnung
nahezulegen scheint, und wenn sie keine Gemeinsamkeiten haben mit dem Meineid der Juden.

\kapnum{11}Um schlie�lich das Wichtigste zusammenzufassend zu sagen: Es hat dem gemeinsamen Urteil aller
gefallen, das heiligste Osterfest an ein und demselben Tag zu begehen. Denn es ziemt sich nicht, in einer so
heiligen Angelegenheit irgendwelche Differenzen zu haben, und besser ist es, sich genau der Meinung
anzuschlie�en, der keinerlei fremder Irrtum und Verfehlung untergemischt ist.

\kapnum{12}Da die Dinge so liegen, nehmt ohne Starrsinn die himmlische Gnadengabe und das wahrhaft g�ttliche
Gebot an! Denn alles, was auch immer in den heiligen Besprechungen der Bisch�fe
verhandelt wird, hat Anhalt am g�ttlichen Willen. Deswegen seid ihr verpflichtet, nachdem ihr allen geliebten Br�dern das
oben Geschriebene verdeutlicht habt, nun die obige Entscheidung, also die Einhaltung des so heiligen
Festtages, zu �bernehmen und umzusetzen, damit ich, wenn ich zum von mir lang ersehnten Anblick eures Verfassung
komme, das heilige Fest mit euch an ein und demselben Tag feiern kann und Freude habe an euch in
jeder Hinsicht, da ich sehe, wie die teuf"|lische Grausamkeit von der g�ttlichen Kraft durch unsere
Taten beseitigt wurde, da unser Glaube und Frieden und Einm�tigkeit �berall in gro�er Kraft steht.

M�ge Gott euch bewahren, geliebte Br�der!
% \clearpage
% \section{Brief Kaiser Konstantins an die Gemeinde von Nikomedien}
\kapitel{Brief des Kaisers Konstantin an die Kirche von Nikomedien (Urk.~27)}
\label{ch:27}

Konstantin Augustus an die katholische Kirche von Nikomedien.

\kapnum{1}Ohne Zweifel wi�t ihr alle genau, geliebte Br�der, da� Gott, der Herrscher, und der
Erl�ser Christus Vater und Sohn sind; den Vater nenne ich anfangslos, ohne Ende, Ursprung seines
�ons, der Sohn aber ist der Wille des Vaters, der durch keine �berlegung erfa�t und der in der
Vollendung seiner Werke durch kein erforschtes Wesen begriffen werden kann. Denn
wer dies versteht oder verstehen will, der mu� f�r alle Art von
Peinigung unendliche Geduld haben.

\kapnum{2}Aber der Sohn Gottes, Christus, der Sch�pfer von allen Dingen und Anf�hrer der
Auferstehung selbst, wurde gezeugt, wie es dem Glauben, an dem wir festgehalten haben,
entspricht, er wurde gezeugt~-- besser gesagt, er selbst, der immer im Vater ist, kam selbst hervor,
um die von ihm geschaffenen Dinge zu ordnen~--; er wurde also gezeugt in einem unteilbaren Hervorgang.
Denn der Wille blieb zugleich auch in seiner eigenen Wohnung und organisiert und verwaltet das, was
unterschiedlichster Pflege bedarf, nach der jeweiligen Beschaffenheit.

\kapnum{3}Was also ist zwischen Gott, dem Vater, und dem Sohn? Offensichtlich nichts! Diese F�lle der Dinge n�mlich hat durch Wahrnehmung den Befehl des Willens empfangen und nicht den aus dem Wesen des Vaters abgeteilten Willen abgetrennt.

\kapnum{4}Daraus folgt also: Wer ist es, der mehr aus Scham als aus Torheit das Leiden meines Herrn
Christus f�rchtete? Leidet etwa nun das G�ttliche, wenn die Wohnung des ehrw�rdigen Leibes zur
Erkenntnis ihrer eigenen Heiligkeit hinf�hrt, oder unterliegt das einer Ber�hrung, was vom Leib
getrennt ist? Macht nicht gerade dies den Unterschied, was sich der
Niedrigkeit des Leibes entzieht? Leben wir nicht, auch wenn der Ruhm der Seele den Leib in den Tod
ruft?

\kapnum{5}Was f�r einen Spielraum f�r Zweifel l��t also demzufolge der unverletzte und reine
Glaube? Oder siehst du nicht, da� Gott einen besonders ehrw�rdigen Leib ausgew�hlt hat, durch den er
ein Zeugnis f�r den Glauben und ein Beispiel f�r seine eigene Tugend zeigen wollte; auch wollte er
die Vernichtung des Menschengeschlechts, verursacht durch sch�dlichen Irrtum, absch�tteln,
eine neue Lehre der Gottesverehrung geben und durch eine beispielhafte Reinigung die unw�rdigen
Taten des Geistes l�utern, schlie�lich die Todesqual auf"|l�sen und den Siegespreis der
Unsterblichkeit ausrufen!

\kapnum{6}Aber ihr, die ihr schlie�lich zu Recht von mir aufgrund der gemeinsamen Liebe Br�der
genannt werdet, �berseht bitte nicht, da� ich euer Mitdiener bin, �berseht bitte nicht
die Festung eurer Erl�sung, deren Pflege ich ehrlich auf mich genommen habe und durch die wir nicht
nur die Waffen unserer Feinde niedergestreckt, sondern auch die noch lebenden Seelen eingeschlossen
haben, um den wahren Glauben der Menschenliebe zu zeigen.

\kapnum{7}Ich freute mich aber �ber diese G�ter, besonders wegen der Erneuerung des Erdkreises. 
Und es war bewundernswert, damit gew�rdigt zu werden, so viele V�lker wahrlich zur Eintracht zu
f�hren, von denen es vor kurzem noch hie�, sie kennten Gott nicht. Aber was werden diese Heiden
kennenlernen, die sich noch keine Gedanken �ber Streitereien gemacht haben? Was glaubt ihr also,
geliebte Br�der, wessen ich euch anschuldigen mu�? Christen sind wir und zerstreiten uns in kl�glichen
Haarspaltereien!

\kapnum{8}Ist dies also unser Glaube, ist dies die Lehre des heiligsten Gesetzes? Aber was ist der
Grund, durch den das Verderben des gegenw�rtigen �bels entstanden ist? Oh, dieses �berma� an
Widerborstigkeit, an Ha�, das jedes �rgernis �bertrifft! Welch Ausma� eines Verbrechens zeigt sich,
wenn geleugnet wird, da� der Sohn des Vaters aus dem ungeteilten Wesen des Vaters hervorgegangen
ist! Ist Gott etwa nicht �berall, obwohl wir wahrnehmen, da� er immer bei uns ist?
Besteht etwa nicht durch seine Kraft die gute Ordnung des Alls, obwohl ihm der Abstand der Trennung
mangelt?

\kapnum{9}Was soll also f�r euch gemacht werden? Geliebte Br�der, begreift endlich, ich bitte
euch, die Qualen des gegenw�rtigen Schmerzes. Ihr habt angek�ndigt, den zu bekennen, dessen Sein
ihr leugnet, da euch dieser verdorbene Lehrer dazu �berredet. Ich flehe euch an: Wer ist es,
der dies die so arglosen Menge gelehrt hat? Eusebius, offensichtlich ein Miteingeweihter in die
tyrannische Grausamkeit. Denn da� er �berall ein Anh�nger des Tyrannen war, das kann man aus allem
erkennen. Dies bezeugen die Opfer unter den Bisch�fen, unter den wahrhaftigen Bisch�fen,
und dies schreit ausdr�cklich die bislang schwerste Verfolgung der Christen heraus.

\kapnum{10}Ich will jetzt nichts �ber die gegen mich ver�bten Anschl�ge berichten, durch die er zu
dem Zeitpunkt, als die Mitl�ufer der Gegenseite besonders aktiv waren, Spione gegen
mich losschickte; nur bewaffnete Hilfe leistete er dem Tyrannen nicht.

\kapnum{11}Niemand soll glauben, ich w�re f�r einen Beweis in dieser Sache nicht ger�stet.
Denn ein klarer Beweis ist die Tatsache, da� die Presbyter und Diakone, die sich dem Eusebius
angeschlossen haben, offensichtlich von mir festgenommen worden sind. Aber dies �bergehe ich, was nicht
aufgrund einer Ver�rgerung, sondern um jene zu besch�men von mir jetzt vorgebracht worden ist.
Allein jenes f�rchte ich, allein dar�ber mache ich mir Gedanken, da� ich sehe, wie ihr zur Gruppe
der Schuldigen dazugerechnet werdet. Denn unter der F�hrung und Verkehrung des Eusebius habt ihr euer
Gewissen von der Wahrheit entfremdet.

\kapnum{12}Aber eine Heilung ist nicht schwer, wenn ihr jetzt wirklich einen frommen und reinen
Bischof annehmt und zu Gott aufblickt, was im Moment an euch liegt und schon l�ngst durch eure
Entscheidung erreicht worden w�re, wenn nicht der besagte Eusebius mit einem Sturm seiner Anh�nger
hierher gekommen w�re und die geordneten Zust�nde schamlos durcheinander gebracht h�tte.

\kapnum{13}Aber nachdem eurer Liebe einiges �ber diesen Eusebius mitgeteilt werden mu�te, m�ge sich eure
Geduld an die in der Stadt Nicaea veranstaltete Synode erinnern, bei der auch ich dabei war, wie es
der Fr�mmigkeit meines Gewissens angemessen ist, wobei ich nichts anderes wollte, als Eintracht unter
allen zu bewirken und haupts�chlich diese Angelegenheit aufzudecken und zu Fall zu bringen, die
ihren Anfang in dem Irrsinn des Alexandriners Arius genommen, sofort aber an Macht gewonnen
hatte durch den unm�glichen und sch�dlichen Eifer des Eusebius.

\kapnum{14}Aber dieser Eusebius, Geliebte und Verehrte, was glaubt ihr, mit welcher Menschenmenge, da
er doch seinem eigenen Gewissen unterlegen ist, mit welcher Schande er sich f�r die �berall
verurteilte Falschlehre stark gemacht hat. Er schickte mir n�mlich die verschiedensten Leute, die sich
f�r ihn aussprachen, und erbat von mir gewisserma�en Kampfgenossenschaft, damit er nicht aus seinem
gegenw�rtigen Amt entlassen wird, falls er eines so gro�en Fehlers �berf�hrt w�rde. Gott selbst, der
mir und auch euch gegen�ber freundlich gesinnt bleiben m�ge, ist mein Zeuge daf�r, denn auch mir hat jener geschadet und mich �bervorteilt, was auch ihr wissen sollt. Alles wurde n�mlich damals so
gemacht, wie er es wollte, wobei er alles m�gliche Schlechte in seinem eigenen Sinn verborgen hielt.

\kapnum{15}Aber jetzt, um den Rest seiner Torheit zu �bergehen, bitte ich euch, vernehmt, was
er besonders mit Theognis angestellt hat, den er als Teilhaber seines Unsinnes hat. Ich hatte einigen
Alexandrinern, die sich von unserem Glauben losgesagt hatten, befohlen, dorthin zu kommen, weil durch
deren Unterst�tzung die Flamme der Zwietracht erwacht war.

\kapnum{16}Aber diese hervorragenden und guten Bisch�fe, denen die Wahrheit der Synode ein f�r
allemal Bu�e auferlegt hat, nahmen diese nicht nur auf und brachten sie bei sich in Sicherheit,
sondern verbanden sich sogar mit deren schlechter Lebensart. Daher beschlo� ich, mit diesen
undankbaren Menschen folgenderma�en zu verfahren: Ich befahl, da� sie nach ihrer Festnahme  m�glichst weit weg verbannt werden sollten.

\kapnum{17}Jetzt ist es eure Sache, in jenem Glauben, den es schon immer gegeben hat und den es
auch immer geben mu�, auf Gott zu schauen, und sich so zu verhalten, da� wir uns freuen, heilige,
rechtgl�ubige und menschenfreundliche Bisch�fe zu haben. Falls es aber jemand unbedachterweise wagen
sollte, jene verdorbenen Menschen ins Ged�chtnis zu rufen oder ihnen Lob entgegenzubringen, so soll er von seiner
Unverfrorenheit sofort durch die Macht des Dieners Gottes, also durch meine Macht, abgedr�ngt
werden.

Gott m�ge euch bewahren, geliebte Br�der!
% \section{Brief Kaiser Konstantins an Theodot von Laodicea}
\kapitel{Brief des Kaisers Konstantin an Theodot von Laodicea (Urk.~28)}
\label{ch:28}

Der Sieger Konstantin Augustus an Theodot.

\kapnum{1}Wie gro� die Gewalt des g�ttlichen Zorns ist, kannst auch du aus dem lernen, was Eusebius
und Theognis erlitten haben, die sich bei der heiligen Gottesverehrung wie Betrunkene aufgef�hrt und
den Namen Gottes, des Erl�sers, auch nachdem ihnen verziehen wurde, mit der Bildung einer eigenen Bande
besudelt haben. Als es n�mlich nach der einm�tigen �bereinkunft auf der Synode besonders wichtig gewesen
w�re, den fr�heren Irrtum zurechtzur�cken, da wurden sie �berf�hrt, wie sie an ihren unm�glichen
Positionen festhielten.

\kapnum{2}Deswegen hat also die g�ttliche Vorsehung sie aus ihrem eigenen Volk ausgesto�en.
Da sie es auch nicht ertrug mitanzusehen, wie der Unsinn von wenigen unschuldige Seelen verdarb,
forderte sie von ihnen f�r den gegenw�rtigen Zeitpunkt eine angemessene Strafe; eine gr��ere
Bestrafung f�r die Zukunft die ganze Ewigkeit hindurch steht aber noch aus.

\kapnum{3}Ich war der Meinung, da� dies deinem Scharfsinn mitgeteilt werden mu�, damit du darauf
achtest, falls irgendein schlechter Rat von diesen, was ich aber nicht glaube, vor deine
Entscheidung gebracht wird, diesen von der Seele fernzuhalten und dir, wie es sich geziemt, eine
reine Gesinnung und einen makellosen, heiligen und unbefleckten Glauben an Gott, den Erl�ser, zu
bewahren. Und so mu� jemand reagieren, der sich entschlie�t, den unversehrten Kampfpreis des ewigen
Lebens zu verdienen.

\textit{Und mit anderer Hand:}
M�ge Gott dich, geliebter Bruder, bewahren.
% \section{Brief Kaiser Konstantins an Arius}
\kapitel{Brief des Kaisers Konstantin an Arius (Urk.~29)}
\label{ch:29}

Der Sieger Konstantin der Gro�e, Augustus, an Arius.

Vor l�ngerer Zeit schon wurde deiner Standhaftigkeit mitgeteilt, da� du zu unserem Hof kommen m�gest, um
unsere Gegenwart genie�en zu k�nnen. Ich wundere mich aber sehr, da� du dies nicht sofort getan
hast. Daher besteige jetzt einen �ffentlichen Wagen und komm schnell an unseren Hof, damit du unser
Wohlwollen und unser Zusammensein erf�hrst und dann in deine Vaterstadt zur�ckkehren kannst.
Gott m�ge dich bewahren, Geliebter.

Ausgegeben am 27. November.
% \clearpage
% \section{Brief der Presbyter Arius und Euzoius an Kaiser Konstantin}
\kapitel{Brief des Arius und des Euzoius an Kaiser Konstantin (Urk.~30)}
\label{ch:30}

Unseren ehrw�rdigsten und gottgeliebtesten Herrscher, Kaiser Konstantin, gr��en Arius und Euzoius.

\kapnum{1}Wie deine gottgeliebte Fr�mmigkeit geboten hat, Herr Kaiser, legen wir unseren
pers�nlichen Glauben vor und bekennen schriftlich vor Gott, da� wir und alle unsere Mitstreiter
folgenderma�en glauben:

\kapnum{2}Wir glauben an einen Gott, den Vater, den Allm�chtigen, und an den Herrn Jesus Christus, seinen
eingeborenen Sohn, der aus ihm vor allen Zeiten geboren worden ist, Gott, Wort, durch den alles
wurde, das im Himmel und das auf Erden, der herabstieg und Fleisch annahm, litt, auferstand und
hinaufstieg in den Himmel und der wieder kommen wird die Lebenden und die Toten zu richten.

\kapnum{3}Und an den heiligen Geist, die Auferstehung des Fleisches, an ein Leben im kommenden �on,
an das Himmelreich, an eine katholische Kirche Gottes vom Anfang bis zum Ende der Erde.

\kapnum{4}Diesen Glauben aber haben wir aus den heiligen Evangelien empfangen, wie es der Herr
seinen eigenen J�ngern sagte: ">Geht und lehrt alle V�lker, tauft sie auf den Namen des Vaters und
des Sohnes und des heiligen Geistes."< Falls wir aber dies nicht so glauben und nicht in Wahrheit
den Vater und den Sohn und den heiligen Geist annehmen, wie es die ganze katholische Kirche und die
Schriften lehren, an die wir in jeder Hinsicht glauben, so ist Gott unser Richter, sowohl jetzt als
auch an dem kommenden Tag.

\kapnum{5}Daher bitten wir deine Fr�mmigkeit, gottgeliebtester Kaiser, uns, die wir Anh�nger der
Kirche sind und am Glauben und Denken der Kirche und der heiligen Schriften festhalten, durch deine
friedensstiftende und gottesf�rchtige Fr�mmigkeit mit unserer Mutter, der Kirche also, wieder zu
vereinigen, da die Differenzen und die daraus erwachsenden Wortgefechte aufgehoben worden sind, so
da� wir und die Kirche nun miteinander Frieden halten und alle gemeinsam die gewohnten Gebete f�r
dein friedliches und frommes Kaisertum und f�r dein ganzes Geschlecht verrichten werden.
%% DOKUMENT 37 %%%
%% Erstellt von uh
%% �nderungen:
%%%% 13.5.2004 Layout-Korrekturen (avs) %%%%
\kapitel{Berichte �ber Synoden in Antiochien des Jahres 327}
\label{ch:Antiochien326}
\section{Bericht �ber Verhandlungen gegen Asclepas von Gaza}
% \label{sec:36.1} 
\label{sec:Asclepas} 
\begin{praefatio}
  \begin{description}
  \item[327]Die Datierung ist allein aus einer Angabe in dem Schreiben
    der �stlichen Teilsynode von
    Serdica\index[synoden]{Serdica!a. 343} (Dok.
    \ref{sec:RundbriefSerdikaOst}) herzuleiten: \textit{dicimus autem
      Asclepan, qui ante decem et septem annos episcopos honore
      discinctus est}
    (\edpageref{Asclepas:Serdica},\lineref{Asclepas:Serdica}--\lineref{Asclepas2:Serdica}). Ist
    diese Angabe korrekt, so k�me man bei einer Datierung der Synode
    von Serdica in das Jahr 343 auf das Jahr 327. Da dieses Regest von
    einer Verurteilung des Asclepas\index[namen]{Asclepas!Bischof von
      Gaza} auf einer Synode in
    Antiochien\index[synoden]{Antiochien!a. 327} unter der Leitung des
    Eusebius von Caesarea\index[namen]{Eusebius!Bischof von Caesarea}
    ausgeht, ist die Absetzung des dortigen Bischofs
    Eustathius\index[namen]{Eustathius!Bischof von Antiochien}
    (s. folgendes Regest) wohl vorauszusetzen. Der Hinweis im
    �stlichen Synodalschreiben von Serdica (Dok.
    \ref{sec:RundbriefSerdikaOst},14), da�
    Athanasius\index[namen]{Athanasius!Bischof von Anazarba} dem
    zugestimmt habe, l��t eventuell ein sp�teres Datum vermuten, da
    Athanasius\index[namen]{Athanasius!Bischof von Anazarba} erst 328
    Bischof wurde, andererseits kann die Formulierung auch so gedeutet
    werden, da� Athanasius\index[namen]{Athanasius!Bischof von
      Anazarba} sp�ter die schon erfolgte Verurteilung
    best�tigte. Geh�ren die in Dok. \ref{sec:SerdicaRundbrief},6
    erw�hnten ">gef�lschten"< Briefe des Theognis von
    Nicaea\index[namen]{Theognis!Bischof von Nicaea} gegen
    Asclepas\index[namen]{Asclepas!Bischof von Gaza} (sowie gegen
    Athanasius\index[namen]{Athanasius!Bischof von Anazarba} und
    Markell\index[namen]{Markell!Bischof von Ancyra}) in diesen
    Zusammenhang, wird die Datierung problematisch, da
    Theognis\index[namen]{Theognis!Bischof von Nicaea} als in
    Nicaea\index[synoden]{Nicaea!a. 325} Verurteilter 327 nicht in der
    Position gewesen sein d�rfte, diese Verurteilung
    voranzutreiben. Eventuell geh�ren diese Briefe aber auch zu einer
    zweiten Verurteilung des Asclepas\index[namen]{Asclepas!Bischof
      von Gaza} aufgrund von Tumulten nach seiner R�ckkehr im Jahr 337
    (vgl. auch Dok.  \ref{sec:RundbriefSerdikaOst},10; Ath., fug. 3,3;
    h.\,Ar. 5,2; Soz., h.\,e. III 8,1; Socr., h.\,e.  II 15,2). Der
    Hinweis bei Thdt., h.\,e. I 29,7, da�
    Asclepas\index[namen]{Asclepas!Bischof von Gaza} auf der Synode in
    Tyrus\index[synoden]{Tyrus!a. 335} wegen falscher Lehre angeklagt
    worden sei, l��t sich mit den �brigen Regesten nicht in
    �bereinstimmung bringen.
  \item[�berlieferung] Vgl. die Einleitung zu
    Dok. \ref{sec:SerdicaRundbrief}.
  \item[Fundstelle] Dok. \ref{sec:SerdicaRundbrief},12 (Thdt.,
    h.\,e. II 8,26 [\editioncite[108,15--18]{Hansen:Thdt}]; Hil.,
    coll.\,antiar. B II 1,6 [\editioncite[118,3--6]{Hil:coll}]; Cod.\,Ver. LX f. 84b; Ath., apol.\,sec. 45,2 [\editioncite[122,3--6]{Opitz1935}])
  \end{description}
\end{praefatio}
\begin{pairs}
\begin{Leftside}
\beginnumbering
\selectlanguage{polutonikogreek}
\pstart
%%%%%%%%%%%%% aus Dok. 43.1, 12 %%%%%%%%%%%%%%%%%%%%%%%%%%%%%%%%%
%%%%%%%%%%%%% aktuellen Stand eingef�gt 1.5.2007 AvS %%%%%%%%%%%%%
\hskip -1,25em\edtext{\abb{}}{\killnumber\Cfootnote{\hskip -1em\latintext Thdt.(BAN+GS(s)=r LF=z T) Hil. Cod.\,Ver. Ath.(BKO  RE)}}\specialindex{quellen}{section}{Theodoret!h.\,e.!II 8,26}\specialindex{quellen}{section}{Hilarius!coll.\,antiar.!B II 1}\specialindex{quellen}{section}{Codices!Veronensis LX!f.
81a--86a}\specialindex{quellen}{section}{Athanasius!apol.\,sec.!42--47}
\edtext{\abb{ka`i}}{\Dfootnote{\latintext > Hil. \textit{etiam} Cod.\,Ver.}}
\edtext{>Asklhp~ac}{\Dfootnote{\latintext \textit{Asclepius} Hil.
\textit{Asclepas} Cod.\,Ver.}}\edindex[namen]{Asclepas!Bischof von Gaza}
\edtext{d`e}{\Dfootnote{\latintext \textit{sed} Hil. > Cod.\,Ver.}}
\edtext{<o sulleitourg`oc}{\Dfootnote{\latintext \textit{quoepiscopus noster}
Hil. \textit{conminister} Cod.\,Ver.}}
\edtext{pro'hnegken}{\Dfootnote{pros'hnegken \latintext Thdt.(BAszT)
\textit{protulit} Hil. Cod.\,Ver.}}
\edtext{<upomn'hmata}{\Dfootnote{\latintext \textit{acta} Hil. \textit{gesta}
Cod.\,Ver.}}
\edtext{gegenhm'ena}{\Dfootnote{gen'omena \latintext Ath. \textit{confecta} Cod.\,Ver. \textit{quae confecta sunt} Hil.}}
\edtext{>en >Antioqe'ia|}{\Dfootnote{>enant'ia \latintext Thdt.(s) \textit{apud
Anthiociam} Hil. \textit{aput Anthiochiam} Cod.\,Ver.}}\edindex[synoden]{Antiochien!a. 327},
\edtext{\edtext{par'ontwn}{\Dfootnote{par'ontwn ka`i \latintext Thdt.(A)
\greektext par`a \latintext Thdt.(BN)}}
\edtext{\abb{t~wn}}{\Dfootnote{\latintext > Thdt.(sFT)}}
kathg'orwn}{\Dfootnote{\latintext ] \textit{praesentibus adversariis} Hil.
\textit{praesentibus accusatoribus} Cod.\,Ver.}} ka`i E>useb'iou\edindex[namen]{Eusebius!Bischof von Caesarea} to~u >ap`o
Kaisare'iac; ka`i >ek t~wn >apof'asewn t~wn
\edtext{dikas'antwn}{\Dfootnote{dikast~wn \latintext Thdt.(B)
\textit{iudicandum} Cod.\,Ver. \textit{iudicatum} Hil.}}
\edtext{>episk'opwn}{\Dfootnote{\latintext \textit{episcopum} Hil.}}
\edtext{\edtext{>'edeixen}{\Dfootnote{\latintext ostendisse Hil.}}
<eaut`on}{\Dfootnote{>'edeixan a>ut`on \latintext Thdt.(s)}}
\edtext{>aj~won}{\Dfootnote{\latintext \textit{inreprehensibilem} Hil.
\textit{innocentem} Cod.\,Ver.}} e>~inai.
\pend
\endnumbering
\end{Leftside}
\begin{Rightside}
\begin{translatio}
\beginnumbering
\pstart
\noindent Auch der Mitdiener Asklepas brachte Aufzeichnungen herbei, die in
Antiochia in Gegenwart seiner Ankl�ger und des Eusebius von Caesarea verfa�t
worden waren.\footnoteA{Dies ist die
einzige Erw�hnung von Verhandlungen gegen Asclepas in Antiochien unter der Leitung des
Eusebius von Caesarea. Der zugrundeliegende Anla� bleibt im Dunkeln, insbesondere wenn die
Bemerkung im �stlichen Synodalschreiben von Serdica zutreffen sollte, da� Athanasius
seinerseits damals dieser Absetzung zugestimmt habe (s.\,o. Einleitung). Eventuelle
theologische Motive sind unbekannt, �berliefert ist nur, da� er sp�ter mit Paulus von
Konstantinopel Kontakt pflegte (vgl. Dok. \ref{sec:RundbriefSerdikaOst},21) und da� sich
die westliche Teilsynode von Serdica f�r eine Kirchengemeinschaft mit ihm ausgesprochen
hat gegen den ausdr�cklichen Protest aus dem Osten (vgl. Dok. \ref{sec:RundbriefSerdikaOst},25;
Dok. \ref{sec:SerdicaRundbrief},15). Asclepas konnte anscheinend nach seiner zweiten
R�ckkehr ca. 347 noch eine zeitlang unbehelligt Bischof bleiben (Soz., h.\,e. II 24; Socr.,
h.\,e. II 23).} Und er konnte anhand
der Aussagen der urteilenden Bisch�fe aufzeigen, da� er unschuldig war.
\pend
\endnumbering
\end{translatio}
\end{Rightside}
\Columns
\end{pairs}
%% Erstellt von uh
%% �nderungen:
%%%% 13.5.2004 Layout-Korrekturen (avs) %%%%
% \cleardoublepage
% \thispagestyle{empty}
\section{Bericht �ber Verhandlungen gegen Eustathius von Antiochia}
% \label{sec:36.2} 
\label{sec:Eustathius} 
\begin{praefatio}
  \begin{description}
  \item[327]Die Datierung der Absetzung des
    Eustathius\index[namen]{Eustathius!Bischof von Antiochien} ist
    problematisch. Geht sie der Absetzung des
    Asclepas\index[namen]{Asclepas!Bischof von Gaza} voraus und wird
    diese fr�h datiert, so wurde auch
    Eustathius\index[namen]{Eustathius!Bischof von Antiochien} bald
    nach Nicaea\index[synoden]{Nicaea!a. 325} exkommuniziert. Die
    Tatsache, da� Eusebius von Caesarea\index[namen]{Eusebius!Bischof
      von Caesarea} und nicht Eusebius von
    Nikomedien\index[namen]{Eusebius!Bischof von Nikomedien} (der
    Bericht bei Theodoret ist nicht glaubw�rdig, s.\,u. zur
    �berlieferung) die Verantwortung daf�r trug, scheint f�r diese
    fr�he Datierung in die Zeit vor der R�ckkehr des Nikomediers aus
    dem Exil 328 zu
    sprechen. Athanasius\index[namen]{Athanasius!Bischof von Anazarba}
    verbindet sp�ter die Absetzung des
    Eustathius\index[namen]{Eustathius!Bischof von Antiochien} mit der
    Anklage, er habe die Kaisermutter Helena beleidigt (h.\,Ar. 4,1),
    was w�hrend der Pilgerreise Helenas nach Palaestina (Eusebius,
    v.\,C. III 42--46; Ambr., ob.\,Theod. 41) h�tte geschehen sein
    k�nnen (\cite[224--226]{Schwartz:GesIII}).  Da aber das Datum
    dieser Reise nicht eindeutig ist, hilft dieser wohl auch
    tendenzielle Hinweis bei Ath.  nicht f�r eine Datierung der
    Absetzung des Eustathius\index[namen]{Eustathius!Bischof von
      Antiochien}. Athanasius erw�hnt die Absetzung des
    Eustathius\index[namen]{Eustathius!Bischof von Antiochien}
    zusammen mit weiteren Absetzungen (h.\,Ar. 5,2; fug. 3,3:
    Euphration von Balanea\index[namen]{Euphration!Bischof von
      Balanea}, Cymatius von Paltus\index[namen]{Cymatius!Bischof von
      Paltus}, Eutropius von Adrianopel\index[namen]{Eutropius!Bischof
      von Adrianopel}, Carterius von
    Antaradus\index[namen]{Carterius!Bischof von Antaradus}, Asclepas
    von Gaza\index[namen]{Asclepas!Bischof von Gaza}, Cyrus von
    Beroea\index[namen]{Cyrus!Bischof von Beroea}, Diodor von
    Tenedus\index[namen]{Diodorus!Bischof von Tenedus}, Domnion von
    Sirmium\index[namen]{Domnion!Bischof von Sirmium}, Hellanicus von
    Tripolis\index[namen]{Hellanicus!Bischof von Tripolis}), obwohl
    diese sicher nicht auf ein und derselben Synode veranla�t worden
    waren. Bei einer fr�hen Datierung bleibt nat�rlich nur eine kurze
    Zeitspanne �brig, in der
    Eustathius\index[namen]{Eustathius!Bischof von Antiochien} Bischof
    von Antiochien war, da er erst kurz vor der Synode in
    Antiochien\index[synoden]{Antiochien!a. 325} 325
    (vgl. Dok. \ref{ch:15} = Urk. 15,5 und Dok. \ref{ch:18} =
    Urk. 18,1) vom Bischofssitz Beroea\index[namen]{Beroea} nach
    Antiochien\index[namen]{Antiochien} gewechselt war.
    Eustathius\index[namen]{Eustathius!Bischof von Antiochien} wird
    bald nach seiner Exilierung gestorben sein, da es keine weiteren
    Nachrichten �ber ihn und keinerlei Hinweise auf eine R�ckkehr aus
    dem Exil gibt.
  \item[�berlieferung]Socrates beruft sich hier einerseits auf
    pers�nliche Kenntnisnahme von Briefen zwischen Eustathius von
    Antiochien\index[namen]{Eustathius!Bischof von Antiochien} und
    Eusebius von Caesarea\index[namen]{Eusebius!Bischof von Caesarea}
    im Vorfeld dieser Synode, andererseits scheinen ihm Synodalakten
    vorgelegen zu haben, die neben dem Vorwurf des Sabellianismus
    moralische Verfehlungen angeben. Zus�tzlich referiert er aus einem
    Enkomion des Georg von Laodicea\index[namen]{Georg!Bischof von
      Laodicea} auf Eusebius von Emesa\index[namen]{Eusebius!Bischof
      von Emesa}, worin auch dieser berichtet habe, da�
    Eustathius\index[namen]{Eustathius!Bischof von Antiochien}
    aufgrund eines Sabellianismus-Vorwurfs, den sein Nachfolger, Cyrus
    von Beroea\index[namen]{Cyrus!Bischof von Beroea}, vorgebracht
    hatte, abgesetzt worden sei. Auf dem Bericht des Socrates beruht
    die k�rzere Schilderung in Soz., h.\,e. II 18~f.  Nach Thdt.,
    h.\,e. I 21,3--9 (von Theodoret abh�ngig ist Anon.Cyz., h.\,e. III
    16,8--29) wurde Eustathius wegen eines unehelichen Kindes
    abgesetzt. Diesselbe legendarische �berlieferung findet sich auch
    bei Philost., h.\,e. II 7. Da sowohl Theodoret als auch
    Philostorgius den nur in wenigen identifizierbaren Fragmenten
    �berlieferten anonymen hom�ischen Historiker benutzt haben, der
    mit einiger Sicherheit im antiochenischen Milieu zu lokalisieren
    ist, k�nnte der anonyme hom�ische Historiker hier als Quelle
    sowohl f�r Theodoret als auch f�r Philostorgius in Frage kommen.
    Der Bericht bei Eusebius inklusive der zitierten Kaiserbriefe,
    v.\,C. III 59--62 (vgl. h.\,e. X 1) betrifft nur die sich
    anschlie�enden Differenzen um den Bischofssitz in
    Antiochien\index[namen]{Antiochien}, die hier au�er acht bleiben
    k�nnen.
  \item[Fundstelle]Socr., h.\,e. I 23,8--24,9 (\editioncite[70,7--71,21]{Hansen:Socr})
  \end{description}
\end{praefatio}
\begin{pairs}
\selectlanguage{polutonikogreek}
\begin{Leftside}
\beginnumbering
\pstart
\hskip -1.25em\edtext{\abb{}}{\killnumber\Cfootnote{\hskip -1em\latintext MF=b A
Arm.}}\specialindex{quellen}{section}{Socrates!h.\,e.!I 23,8--24,9}
% \kap{1} <wc d`e <hme~ic >ek diaf'orwn >epistol~wn e<ur'hkamen, <`ac met`a t`hn sun'odon
% o<i >ep'iskopoi pr`oc >all'hlouc >'egrafon, <h to~u <omoous'iou l'exic tin`ac diet'aratte,
% per`i <`hn katatrib'omenoi ka`i >akribologo'umenoi t`on kat'' >all'hlwn p'olemon
% >'hgeiran; \dots\
% \pend
% \pstart
\kap{1}ka`i E>ust'ajioc\edindex[namen]{Eustathius!Bischof von Antiochien} m`en
<o >Antioqe'iac >ep'iskopoc dias'urei t`on Pamf'ilou
E>us'ebion\edindex[namen]{Eusebius!Bischof von Caesarea} <wc t`hn >en
Nika'ia\edindex[synoden]{Nicaea!a. 325}| p'istin paraqar'attonta,
E>us'ebioc\edindex[namen]{Eusebius!Bischof von Caesarea}
\edtext{\abb{d`e}}{\Dfootnote{+ <o Pamf'ilou \latintext Arm. Soz.}} t`hn m`en >en
Nika'ia|\edindex[synoden]{Nicaea!a. 325}
p'istin o>'u fhsi paraba'inein, diab'allei d`e \edtext{\abb{t`on}}{\Dfootnote{\latintext >
b}} E>ust'ajion\edindex[namen]{Eustathius!Bischof von Antiochien} <wc t`hn
Sabell'iou\edindex[namen]{Sabellius} d'oxan e>isv\-'agon\-ta. di`a
\edtext{ta~uta}{\Dfootnote{ta~uj> \latintext A}} <'ekastoi
\edtext{\abb{<wc}}{\Dfootnote{\latintext > Arm.}} kat`a >antip'alwn to`uc l'ogouc
\edtext{sun'egrafon}{\Dfootnote{sunegr'afonto \latintext b}}, >amf'otero'i
\edtext{\abb{te}}{\Dfootnote{\latintext > Arm.}} l'egontec >enup'ostat'on te ka`i
\edtext{\abb{sun\-up\-'ar\-qon\-ta}}{\Dfootnote{\latintext coni. Hansen \greektext >enup'arqonta
\latintext bA}} t`on u<i`on e>~inai to~u jeo~u, \edtext{<'ena te je`on}{\Dfootnote{m'ian
te je'othta \latintext Arm.}} >en tris`in <upost'asesin e>~inai <omologo~untec, >all'hloic
o>uk o>~id'' <'opwc  sumfwn~hsai o>uk >'isquon ka`i di`a ta~uta <hsuq'azein o>uden`i
tr'opw| >hne'iqonto. 
\pend
\pstart
\kap{2}S'unodon o~>un >en >Antioqe'ia\edindex[synoden]{Antiochien!a. 327}| poi'hsantec
kajairo~usin E>ust'ajion\edindex[namen]{Eustathius!Bischof von Antiochien} <wc
t`a
Sabell'iou\edindex[namen]{Sabellius}
\edtext{\abb{m~allon}}{\Dfootnote{\latintext > Arm.}} frono~unta \edtext{>`h
<'aper}{\Dfootnote{e>'iper <`a \latintext A \greektext <'aper ka`i \latintext Arm.}} <h
>en Nika'ia|\edindex[synoden]{Nicaea!a. 325} s'unodoc >edogm'atisen, <wc m`en o>~un
tin'ec fasin
\Ladd{\edtext{\abb{ka`i}}{\Dfootnote{\latintext add. Hansen ex Arm.}}} di'' >'allac o>uk >agaj`ac
a>it'iac; \edtext{faner~wc g`ar}{\Dfootnote{<`ac faner~wc \latintext Arm.}} o>uk
e>ir'hkasi. \edtext{to~uto}{\Dfootnote{ta~uta \latintext A}} d`e >ep`i p'antwn e>i'wjasi
t~wn kajairoum'enwn poie~in o<i >ep'iskopoi,
\edtext{kakhgoro~untec}{\Dfootnote{kathgoro~untec \latintext Arm.}} m`en ka`i >asebe~in
l'egontec, t`ac d`e a>it'iac t~hc >asebe'iac o>u prostij'entec.
\pend
\pstart 
\kap{3}<'oti m'entoi <wc sabell'izonta\edindex[namen]{Sabellianer} kaje~ilon
E>ust'ajion\edindex[namen]{Eustathius!Bischof von Antiochien},
\edindex[namen]{Cyrus!Bischof von
Beroea}\edtext{K'urou to~u Bero'iac
>episk'opou \edtext{kathgoro~untoc}{\Dfootnote{kakhgoro~untoc \latintext
A}}}{\Dfootnote{K'urion t`on Bero'iac >ep'iskopon kathgoro~unta \latintext
M\textsuperscript{r}}} a>uto~u, Ge'wrgioc <o Laodike'iac\edindex[namen]{Georg!Bischof
von Laodicea} t~hc >en Sur'ia| >ep'iskopoc,
e<~ic >`wn t~wn miso'untwn t`o <omoo'usion, >en t~w| >egkwm'iw|, \edtext{\abb{<`o
e>ic}}{\Dfootnote{\latintext Arm. \greektext t~w e>ic \latintext b \greektext je`ic t~w
\latintext A}} E>us'ebion t`on \edindex[namen]{Eusebius!Bischof von
Emesa}\edtext{>Emeshn`on}{\Dfootnote{>Emishn`on \latintext
M\textsuperscript{1}FA Arm.}} >'egrayen, \edtext{\abb{e>'irhken}}{\Dfootnote{\latintext
Arm. \greektext e>irhk'enai \latintext bA}}. 
\pend
\pstart
\kap{4}ka`i per`i m`en to~u \edtext{>Emeshno~u}{\Dfootnote{>Emishno~u \latintext
M\textsuperscript{1}FA}} E>useb'iou\edindex[namen]{Eusebius!Bischof von Emesa}
kat`a q'wran >ero~umen; Ge'wrgioc\edindex[namen]{Georg!Bischof von Laodicea}
\edtext{d`e}{\Dfootnote{te \latintext A}} per`i
E>ustaj'iou\edindex[namen]{Eustathius!Bischof von Antiochien}
\Ladd{\edtext{\abb{>ap'ijana}}{\Dfootnote{\latintext add. Hansen ex Arm}}}
gr'afei. f'askwn g`ar E>ust'ajion\edindex[namen]{Eustathius!Bischof von
Antiochien} <up`o K'urou\edindex[namen]{Cyrus!Bischof von Beroea}
kathgore~isjai <wc sabell'izonta\edindex[namen]{Sabellianer}, a>~ujic
t`on \Ladd{\edtext{\abb{a>ut`on}}{\Dfootnote{\latintext add. Hansen ex Arm.}}}
K~uron\edindex[namen]{Cyrus!Bischof von Beroea} >ep`i to~ic
a>uto~ic <al'onta kajh|r~hsja'i fhsin. 
\pend
\pstart
\kap{5}ka`i p~wc o<~i'on te K~uron\edindex[namen]{Cyrus!Bischof von
Beroea} t`a Sabell'iou\edindex[namen]{Sabellianer} frono~unta kathgore~in
\edindex[namen]{Eustathius!Bischof von
Antiochien}\edtext{E>ustaj'iou}{\Dfootnote{E>ust'ajion \latintext M\textsuperscript{1}}}
<wc
sabell'izontoc? >'eoiken o>~un \edindex[namen]{Eustathius!Bischof von
Antiochien}\edtext{E>ust'ajioc}{\Dfootnote{E>ust'ajion \latintext b}}
di'' <et'erac kajh|r~hsjai prof'aseic. 
\pend
\pstart
\kap{6}t'ote d`e >en t~h| >Antioqe'ia|\edindex[synoden]{Antiochien!a. 327}
\edtext{dein`h}{\Dfootnote{di`a
to~uto \latintext Arm.}} st'asic >ep`i t~h| a>uto~u kajair'esei geg'enhtai, ka`i
\edtext{met`a}{\Dfootnote{met`a t`a\latintext? M\textsuperscript{1}}} ta~uta poll'akic
per`i >epilog~hc >episk'opou toso~utoc >ex'hfjh purs'oc, <wc mikro~u de~hsai t`hn p~asan
\Ladd{\edtext{\abb{>ek b'ajrwn}}{\Dfootnote{\latintext Arm. Eus. > bA}}}
>anatrap~hnai p'olin, e>ic d'uo tm'hmata diairej'entoc to~u \Ladd{\edtext{\abb{t~hc
>ekklhs'iac}}{\Dfootnote{\latintext Arm. Eus. > bA}}} lao~u, t~wn m`en
E>us'ebion\edindex[namen]{Eusebius!Bischof von Caesarea} t`on Pamf'ilou \edtext{\abb{>ek t~hc >en
Palaist'inh| Kaisare'iac}}{\Dfootnote{\latintext > Arm.}} metaf'erein filoneiko'untwn
\mbox{>ep`i} t`hn >Anti'oqeian\edindex[synoden]{Antiochien!a. 327}, t~wn d`e speud'ontwn
\edtext{>epanagage~in}{\Dfootnote{>epan'agein \latintext susp. Hansen}}
E>ust'ajion\edindex[namen]{Eustathius!Bischof von Antiochien}.
\pend
\pstart
\skipnumbering
\pend
\pstart
\kap{7}sunelamb'aneto \edtext{\abb{d`e}}{\Dfootnote{+ >en \latintext A}} <ekat'erw|
m'eri ka`i t`o koin`on t~hc p'olewc, ka`i stratiwtik`h qe`ir <wc kat`a polem'iwn
kek'inhto, <wc ka`i xif~wn m'ellein <'aptesjai, e>i m`h <o je'oc te ka`i <o par`a to~u
basil'ewc f'oboc t`ac <orm`ac \edtext{\abb{to~u pl'hjouc}}{\Dfootnote{\latintext > Arm.}}
>an'esteilen. 
\kap{8}<o m`en g`ar basile`uc di'' >epistol~wn \edtext{\abb{t`hn
gegenhm'enhn st'asin}}{\Dfootnote{\latintext Arm. Syr. \greektext t`a gegenhm'ena
ka`i t`ac st'aseic \latintext bA}} kat'epausen, E>us'ebioc\edindex[namen]{Eusebius!Bischof
von Caesarea} d`e
\edtext{\abb{paraiths'amenoc}}{\Dfootnote{+ t`hn >Antioqe'iac >episkop`hn >an~hljen e>ic
Kais'areian >ep`i t`on <eauto~u jr'onon \latintext Arm., vgl. Eus., v.\,C. III 61,2}}; >ef''
<~w| ka`i jaum'asac a>ut`on <o basile`uc gr'afei te a>ut~w|
ka`i t`hn pr'ojesin a>uto~u >epain'esac mak'arion >apokale~i,
\edtext{\abb{<'oti}}{\Dfootnote{\latintext M\textsuperscript{1}FA Arm. \greektext ka`i
<'oti \latintext M\textsuperscript{r}}} o>u mi~ac p'olewc, >all`a p'ashc <apl~wc
t~hc o>ikoum'enhc \edtext{>ep'iskopoc >'axioc e~>inai}{\Dfootnote{e>i >ep'iskopoc >'axioc
~>hn \latintext M\textsuperscript{r}}} \edtext{>ekr'ijh}{\Dfootnote{>ekl'hjh  \latintext ? M\textsuperscript{1} \greektext gr. >ekl'hjh\latintext
F\mg}}.
\pend
\pstart 
\kap{9}>efex~hc o>~un >ep`i >'eth >okt`w l'egetai t`on >en
>Antioqe'ia|\edindex[namen]{\latintext Antiochien} jr'onon t~hc
>ekklhs'iac sqol'asai; >oy`e d'e pote spoud~h| t~wn t`hn >en
Nika'ia|\edindex[synoden]{Nicaea!a. 325} p'istin paratr'epein
spoudaz'ontwn qeirotone~itai E>ufr'onioc\edindex[namen]{Euphronius!Bischof von
Antiochien}. 
\pend
\pstart
\kap{10}tosa~uta m`en \ladd{\edtext{\abb{ka`i}}{\Dfootnote{\latintext del. Hansen > Arm.}}} per`i t~hc
sun'odou, \edtext{<`h}{\Dfootnote{>`h \latintext A}} \edtext{\abb{kat`a}}{\Dfootnote{+
t`hn \latintext b}} >Anti'oqeian\edindex[synoden]{Antiochien!a. 327} di''
E>ust'ajion\edindex[namen]{Eustathius!Bischof von Antiochien} g'egonen,
<istore'isjw. 
\pend
\endnumbering
\end{Leftside}
\begin{Rightside}
\begin{translatio}
% \selectlanguage{german}
\beginnumbering
\pstart
% \kapR{23,6}Wie wir aber aus diversen Briefen herausgefunden haben, die die Bisch�fe nach
% der Synode einander geschrieben hatten, verwirrte einige der Begriff ">homousios"<, �ber
% den sie sich derart aufrieben und den Kopf zerbrachen, da� sie Feindseligkeiten
% gegeneinander anzettelten. \dots\ 
% \pend
% \pstart
\noindent\kapR{23,8}Und Eustathius, der Bischof von Antiochien, beschuldigte Eusebius, den Sohn
des Pamphilus, da� er den Glauben von Nicaea verf�lsche; Eusebius dagegen sagte, er w�rde
den Glauben von Nicaea nicht verlassen, warf aber Eustathius vor, er f�hre die Lehre des
Sabellius\footnoteA{\protect\label{fn:Sabell}Dieser Vorwurf gegen Eustathius bezieht
sich auf
die Vorstellung, den Vater und den Sohn so zu einer Person zusammenzufassen, da� diese mal
Vater, mal Sohn sei (\griech{u<iop'atwr}, vgl. Dok. \ref{ch:6} = Urk. 6,3 mit Anm.; Eus., e.\,th. I 5; 20,44; Hil., ad Const. II 9; Ath., Dion. 5,1~f.; Dok. \ref{sec:RundbriefSerdikaOst},27), oft kombiniert mit dem
Vorwurf, da� dann auch der Vater leidensf�hig sei (vgl. Ath., Ar. III 4; Dion. 5,1;
Dok. \ref{ch:Makrostichos},11 mit Anm.). Auch dem Markell warf Eusebius von Caesarea
Sabellianismus vor (vgl. z.\,B. e.\,th. I 5; 14; 20,14.42--44.101 u.\,�.), obwohl
dieser sich selber davon distanziert hatte (fr. 69 Seibt/Vinzent). Diese H�resie geht auf
Sabellius zur�ck, der um 220 n.\,Chr. in Rom verurteilt wurde (Hipp., ref. IX 11~f.; Dok.
\ref{sec:RundbriefSerdikaOst},26; Dok. \ref{ch:Makrostichos},11), �ber den aber keine
genaueren Informationen vorliegen (vgl. \cite{Bienert93}), so da� sp�ter alle
Varianten des modalistischen Monarchianismus als Sabellianismus kritisiert wurden.
Eustathius zog diesen Vorwurf offensichtlich wegen seiner Lehre der einen Hypostase (vgl.
fr. 88; 136; 142 Declerck) auf sich.} ein. Deswegen verfa�ten beide Texte dazu wie gegen
Feinde, und w�hrend beide doch meinten, der Sohn Gottes sei in einer Hypostase und
Existenz, und bekannten, der eine Gott sei in drei Hypostasen, konnten sie einander, ich
wei� nicht warum, nicht zustimmen\footnoteA{Die theologische Diskussionslage verschiebt
sich in den Jahren nach Nicaea durch diesen Streit zwischen Eustathius, Markell und
Eusebius auf die Frage, ob und wie den trinitarischen Personen der Begriff Hypostase
zugewiesen werden kann (Ein- oder Dreihypostasentheologie). Das Unverst�ndnis des Socrates
gegen�ber diesen Differenzen ergibt sich allein aus der Sicht des historisch sp�teren
Zeugen, als theologische und terminologische Kl�rungen durch den sog. Neuniz�nismus
bereits gefunden worden waren.}; und deswegen vermochten sie auf keinen Fall Frieden zu
halten.
\pend
\pstart
\kapR{24,1}Sie veranstalteten also eine Synode in Antiochien\footnoteA{Sozomenus (h.\,e. II
19,1) nennt neben Eusebius von Caesarea auch Paulinus von Tyrus (vgl. Dok. \ref{ch:8} = Urk. 8; Dok. \ref{ch:9} = Urk. 9) und
Patrophilus von Scythopolis (vgl. Dok. \ref{sec:BerichteAntiochien341},2,5 Anm.) als
diejenigen, die, von Eustathius beschuldigt, seine Verurteilung vorangetrieben h�tten.}
und klagten Eustathius an, er w�rde eher die Lehren des Sabellius vertreten als das, was die
Synode in Nicaea festgelegt hatte; wie aber manche behaupten, auch wegen anderer, nicht sch�ner
Anschuldigungen, die sie n�mlich nicht offen ausgesprochen  haben. So aber handeln
Bisch�fe gew�hnlich bei allen, die abgesetzt werden, da sie jene zwar beschimpfen und
ihnen Gottlosigkeit vorwerfen, die Hintergr�nde der Gottlosigkeit aber nicht auf den Tisch
legen. 
\pend
\pstart
\kapR{2}Da� tats�chlich Eustathius als Sabellianer abgesetzt worden ist, weil Cyrus, der
Bischof von Beroea\footnoteA{Eustathius war sein Amtsvorg�nger (s. Einleitung); vgl. auch
Ath., fug. 3,3; h.\,Ar. 5,2.}, ihn angeklagt hatte, berichtete Georg, der Bischof von
Laodicea\footnoteA{Vgl. zu Georg von Laodicea Dok. \ref{sec:BerichteAntiochien341},2,5 Anm.
Georg war Presbyter in Antiochien und wurde von Eustathius nach seiner R�ckkehr von der
Synode von Nicaea der Stadt verwiesen (Ath., h.\,Ar. 4).} in Syrien, einer derjenigen, die
den Begriff ">homousios"< hassen, in seinem Enkomion auf Eusebius von
Emesa\footnoteA{\protect\label{fn:Emesa}Vgl. Socr., h.\,e. II 9; Soz., h.\,e. III 6,6; Dok.
\ref{sec:BerichteAntiochien341},1,9 Anm.}.
\pend
\pstart
\kapR{3}Auch �ber Eusebius von Emesa werde ich an gegebener
Stelle sprechen. Georg aber schreibt �ber Eustathius unglaubw�rdige Dinge, denn w�hrend er einerseits berichtet, da�
Eustathius von Cyrus als Sabellianer beschuldigt worden sei, sagt er andererseits zugleich, da� derselbe Cyrus
bei denselben Lehren ertappt und angeklagt worden sei. 
\pend
\pstart
\kapR{4}Wie aber ist es m�glich, da� Cyrus den Lehren des Sabellius anh�ngt und zugleich
Eustathius als Sabellianer anklagt? Also scheint Eustathius aus anderen Gr�nden angeklagt
worden zu sein. 
\pend
\pstart
\kapR{5}Es entstand damals aber f�rchterliche Zwietracht in Antiochien aufgrund seiner
Verurteilung; und seitdem wurde oft bei der Wahl des Bischofs ein derartiges
Feuer entfacht, da� es wenig bedurfte, die ganze Stadt von Grund auf in Aufruhr zu
versetzen\footnoteA{Dies ist der Beginn des jahrelangen sog. antiochenischen Schismas,
vgl. dazu Ath., tom. und ep.\,Iov. mit den dazugeh�rigen Anmerkungen.}, da das Kirchenvolk in zwei
H�lften zerfallen war; die einen erstrebten, Eusebius, den Sohn des Pamphilus, aus Caesarea
in Palaestina nach Antiochien zu versetzen, die anderen dagegen wollten Eustathius
wieder einsetzen. 
\pend
\pstart
\skipnumbering
\pend
\pstart
\kapR{6}Auf beiden Seiten aber wurde sogar die B�rgerwehr aufgestellt und
milit�rische Macht demonstriert wie gegen Feinde, so da� sie beinahe die Klingen gekreuzt
h�tten, wenn nicht Gott und die Furcht vor dem Kaiser den Angriff der Menge verhindert
h�tte. 
\kapR{7}Denn der Kaiser beruhigte den entstandenen Aufruhr mit
Briefen, Eusebius aber, indem er
h�f"|lich absagte. Deswegen bewunderte ihn der Kaiser, schrieb ihm, lobte seinen
Entschlu� und nannte ihn gl�ckselig, da er f�r w�rdig gehalten werde, Bischof nicht einer Stadt,
sondern der ganzen bewohnten Welt zu sein.\footnoteA{Vgl. dazu Eus., v.\,C. III 59--62.}
\pend
\pstart
\kapR{8}Es wird berichtet, da� acht Jahre hintereinander der Bischofsstuhl der Kirche
von Antiochien unbesetzt blieb\footnoteA{Nach Eusebius (Marcell. I 4,2) folgt aber Paulinus von
Tyrus Eustathius auf den antiochenischen Bischofsstuhl, wenn auch nur f�r ein halbes Jahr
(Philost., h.\,e. III 15); Theodoret erw�hnt einen Eulalius (h.\,e. I 22,1) als Vorg�nger vor
Euphronius. Diese Unklarheiten belegen auf ihre Weise die Streitereien in Antiochien.};
sp�ter aber, nach langem Einsatz derer, die den Glauben von Nicaea verdrehen wollten, wurde
dann Euphronius gew�hlt. 
\pend
\pstart
\kapR{9}Soviel sei �ber die Synode berichtet, die in Antiochien wegen Eustathius abgehalten
worden ist. 
\pend
\endnumbering
\end{translatio}
\end{Rightside}
\Columns
\end{pairs}
\selectlanguage{german}
%%% Local Variables: 
%%% mode: latex
%%% TeX-master: "dokumente_master"
%%% End: 

% \section{Brief des Euseb von Nikomedien und Theognis von Niz�a an die zweite Synode von
% Nic�a}
\kapitel{Brief des Eusebius von Nikomedien und des Theognis von Nicaea an eine Synode (Urk.~31)}
\label{ch:31}


\kapnum{1}Als bereits Verurteilte vor dem Gericht eurer Fr�mmigkeit mu�ten wir bei dem Urteil eurer
heiligen Entscheidung Ruhe bewahren. Da es aber unangemessen ist, durch Schweigen gegen sich selbst
den Verleumdern einen Beweis in die H�nde zu spielen, deswegen geben wir bekannt, da� wir mit dem
Glauben �bereingestimmt und nach einer Pr�fung des Sinnes von ">wesenseins"< ganz und gar den Frieden
unterst�tzt haben; wir sind n�mlich keineswegs Anh�nger der H�resie gewesen.

\kapnum{2}Nachdem wir zur Sicherheit der Kirchen noch einmal dargelegt haben, wie unsere
Gedankeng�nge verlaufen, und die �berzeugt haben, die von uns �berzeugt werden mu�ten, haben wir
dem Glauben zugestimmt, die Verurteilung haben wir aber nicht unterschrieben,
nicht als ob wir damit den Glauben tadelten, sondern da wir nicht der Ansicht sind, derartiges gelte
f�r den Angeklagten, von dem �berzeugt, was er selbst uns mit Briefen oder in pers�nlichen Gespr�chen
vorbrachte, da� derartiges nicht auf ihn zutreffe.

\kapnum{3}Wenn aber eure heilige Synode davon �berzeugt war, werden wir uns nicht widersetzen,
sondern unterst�tzen die Entscheidungen von euch und geben durch dieses Schreiben unsere Zustimmung
bekannt, nicht als ob wir schwer an der Verbannung tragen, sondern weil wir den Verdacht der H�resie
absch�tteln wollen.

\kapnum{4}Denn wenn ihr euch jetzt pers�nlich dazu entschlie�t, uns wieder aufzunehmen, werdet ihr
welche gewinnen, die euch in allem unterst�tzen und sich euren Entschl�ssen anschlie�en,
insbesondere da es eurer Fr�mmigkeit gefiel, dem in diesen Angelegenheiten Beschuldigten mit
Freundlichkeit zu begegnen und ihn zur�ckzurufen. Es ist aber unangemessen, da� wir, wenn der, der
schuldig zu sein schien, zur�ckgerufen worden ist und sich gerechtfertigt hat in den Dingen, derer
er beschuldigt wurde, im Schweigen verharren und so Beweise gegen uns selbst liefern.

\kapnum{5}Entschlie�t euch also, wie es eurer christusliebenden Fr�mmigkeit entspricht, den
gottgeliebtesten Kaiser zu erinnern, unsere Bitten auszuh�ndigen und baldigst einen Entschlu� �ber
uns zu fassen, der euch gef�llt.

% \section{Brief Kaiser Konstantins an Alexander von Alexandrien}
\kapitel{Brief des Kaisers Konstantin an Alexander von Alexandrien (Urk.~32)}
\label{ch:32}

Der Sieger Konstantin der Gro�e, Augustus, an den Vater Alexander, den Bischof.

\kapnum{1}Und jetzt wird also der sch�dliche Neid mit gottlosen Spitzfindigkeiten f�r einen
Aufschub zur�ckbellen. Was bedeuten also die gegenw�rtigen Verh�ltnisse? Beschlie�en wir andere Dinge,
als wie es vom heiligen Geist durch euch beschlossen worden ist, geliebter Bruder?

\kapnum{2}Ich sage, da� Arius, jener Arius, zu mir, dem Augustus, kommen soll, auf Zuraten von
vielen hin, da er versprach, das von unserem katholischen Glauben zu denken, was von euch auf der
Synode in Nicaea bestimmt und festgelegt worden ist, wo auch ich dabei war und mitbestimmt und mich
mit euch um eure Angelegenheiten gek�mmert habe.

\kapnum{3}Sofort darauf kam er also zu uns, zusammen mit Euzoius, sobald sie vom Befehl des
kaiserlichen Willens erfahren hatten. Ich erforschte also mit ihnen in  Anwesenheit von vielen das
Wort des Lebens. Ich bin jener Mensch, der seinen Sinn mit reinem Glauben auf Gott ausgerichtet hat.
Ich bin es, der euch unterst�tzt, der ich mein ganzes Streben auf unseren Frieden und Einm�tigkeit
gerichtet habe.

\begin{footnotesize}
Und nach anderem:
\end{footnotesize}

\kapnum{4}Ich schickte sie also los und erinnerte euch nicht nur, sondern bat euch auch, diese
bittenden Menschen wieder aufzunehmen. Wenn ihr also herausfindet, da� sie nach dem in Nicaea
festgelegten, rechten, immer lebendigen, apostolischen Glauben streben~-- sie haben n�mlich bei uns
bekr�ftigt, so zu denken~-- , dann sorgt euch um alle, ich bitte euch. Denn wenn ihr euch
in F�rsorge f�r diese �bt, dann werdet ihr wohl den Ha� durch Eintracht besiegen.

\kapnum{5}Ich bitte euch also, helft der Eintracht, bringt die G�ter der Freundschaft zu denen, die
die Angelegenheiten des Glaubens nicht durchschauen, bringt mir die Dinge zu Ohren, die ich will und
erstrebe, Frieden und Eintracht unter euch allen.

Gott m�ge dich beh�ten, verehrtester Vater!
\clearpage

%% DOKUMENT 38 %%%
\kapitel[Bericht des Athanasius �ber Konflikte um die Wiederherstellung der Kircheneinheit in �gypten][Konflikte um die Wiederherstellung der Kircheneinheit in �gypten]{Bericht des Athanasius �ber Konflikte um die Wiederherstellung der Kircheneinheit in �gypten}
\thispagestyle{empty}
% \label{ch:37}
\label{ch:Arius}
\begin{praefatio}
  \begin{description}
  \item[328--335]Der Brief Konstantins\index[namen]{Konstantin,
      Kaiser} kann nur ungef�hr in die Zeit zwischen der Ernennung des
    Athanasius\index[namen]{Athanasius!Bischof von Alexandrien} zum
    Bischof (ind.\,ep.\,fest. 3: 27.4.328) und den Verhandlungen gegen
    Athanasius\index[namen]{Athanasius!Bischof von Alexandrien} auf
    den Synoden von Caesarea\index[synoden]{Caesarea!a. 334} und
    Tyrus\index[synoden]{Tyrus!a. 335} (334/335) datiert
    werden. Athanasius\index[namen]{Athanasius!Bischof von
      Alexandrien} zitiert hier aus einem Brief des Kaisers, der ihn
    laut seiner eigenen Einf�hrung dazu aufgefordert haben soll, die
    Anh�nger des Arius\index[namen]{Arius!Presbyter in Alexandrien}
    wieder in die Kirche aufzunehmen.  Es ist aber zu beachten, da�
    der zitierte Kaiserbrief ">die um Arius"< nicht namentlich erw�hnt
    und sich der Bezug auf die Arianer nur aus dem Kontext bei
    Athanasius\index[namen]{Athanasius!Bischof von Alexandrien} (und
    ihm folgend wohl bei Socrates) ergibt. Das Fragment ist vielleicht
    eher ein Teil eines Briefes, in dem
    Konstantin\index[namen]{Konstantin, Kaiser}
    Athanasius\index[namen]{Athanasius!Bischof von Alexandrien} die
    bereitwilligere Aufnahme ehemaliger Melitianer (vgl. den Plural im
    Text: \griech{to~ic boulom'enois, tinas a>ut~wn} und die
    Umschreibung: \griech{to~ic boulom'enois e>ic t`hn >ekklhs'ian
      e>iselje~in}) befahl, wie es auch der Kontext bei Sozomenus
    nahelegt. Zur Frage, ob Arius selbst zu dieser Zeit noch lebte
    oder nicht, vgl. die einleitenden Bemerkungen zur Chronologie.
  \item[�berlieferung]Socr., Soz., Anon.\,Cyz. und v.\,Ath., die nur
    das Brieffragment zitieren, gehen auf eine Quelle zur�ck und sind
    wohl von Athanasius abh�ngig.
  \item[Fundstelle] Ath., apol.\,sec. 59,4--60,1
    (\editioncite[139,18--140,15]{Opitz1935}; \S 2: Socr., h.\,e. I
    27,4 (\editioncite[75,14--18]{Hansen:Socr}), Soz., h.\,e. II
    22,5 (\editioncite[79,11--15]{Hansen:Soz}), Anon.\,Cyz. III
    14 (\editioncite[136,8--12]{Hansen:Gel}) und v.\,Ath. (BHG 185) 2
    (\cite[CCXXV A/B]{Athanasius:PG})
  \end{description}
\end{praefatio}

\begin{pairs}
\selectlanguage{polutonikogreek}
\begin{Leftside}
\beginnumbering
\pstart 
\hskip -1.25em\edtext{\abb{}}{\killnumber\Cfootnote{\hskip -1em\latintext
    Ath.(BKO RE)}}\specialindex{quellen}{chapter}{Athanasius!apol.\,sec.!59,4--60,1}
\kap{1}E>us'ebioc\edindex[namen]{Eusebius!Bischof von Nikomedien}
to'inun to~uto maj`wn ka`i proist'amenoc t~hc >areian~hc a<ir'esewc
p'empei ka`i >wne~itai to`uc Melitiano`uc\edindex[namen]{Melitianer}
>ep`i polla~ic >epaggel'iaic. ka`i g'inetai m`en a>ut~wn kr'ufa
f'iloc, sunt'attetai d`e a>uto~ic e>ic <`on >ebo'uleto kair'on. t`hn
m`en o>~un >arq`hn pros'epempe protr'epwn d'exasja'i me \edtext{to`uc
  per`i}{\Dfootnote{t`on \latintext Ath.(B)}}
>'Areion\edindex[namen]{Arianer}, ka`i >agr'afwc m`en >hpe'ilei,
gr'afwn d`e >hx'iou.  >epeid`h d`e >ant\-'ele\-gon m`h qr~hnai f'askwn
deqj~hnai to`uc a<'iresin >efeur'ontac kat`a t~hc >alhje'iac ka`i
>anajematisj'entac par`a t~hc o>ikoumenik~hc sun'odou poie~i ka`i
basil'ea moi gr'ayai t`on makar'ithn
Kwnstant~inon\edindex[namen]{Konstantin, Kaiser} >apeil`hn >'eq\-on\-ta,
e>i m`h l'aboimi to`uc per`i >'Areion\edindex[namen]{Arianer}, ta~ut'a
me paje~in, <`a pr'oteron ka`i n~un p'eponja. t`o to'inun m'eroc t~hc
>epistol~hc >esti to~uto ka`i palat~inoi
Sugkl'htioc\edindex[namen]{Syncletius!Agens in rebus} ka`i
Gaud'entioc\edindex[namen]{Gaudentius!Agens in rebus} >~hsan o<i
kom'isantec t`a gr'ammata; 
\pend 
\pstart
\edtext{\abb{}}{\xxref{3}{2}\Cfootnote{\latintext
    v.\,Ath.}}\specialindex{quellen}{chapter}{Vita Athanasii!BHG 185!2
  (PG 25, CCXXV A/B)}
\kap{2}\edtext{\textit{M'eroc}}{\lemma{\abb{}}\Dfootnote{M'eroc + t~hc
    \dt{Ath.(KE)}}}\edlabel{3} \textit{>epistol~hc}
\edtext{\textit{to~u}}{\lemma{\abb{}}\Dfootnote{to~u \dt{> Ath.(E)}}}
\textit{basil'ewc Kwnstant'inou}\edindex[namen]{Konstantin,
  Kaiser}\edindex[namen]{Constantinus, Kaiser|see{Konstantin, Kaiser}}
\pend
\pstart
\edtext{\abb{}}{\xxref{1}{2}\Cfootnote{\latintext  Socr.(MF=b AT Arm.) Soz.(BC=b) Anon.\,Cyz.(A)}}\specialindex{quellen}{chapter}{Socrates!h.\,e.!I 27,4}\specialindex{quellen}{chapter}{Sozomenus!h.\,e.!II 22,5}\specialindex{quellen}{chapter}{Anonymus Cyzicenus!h.\,e.!III 14}
>'Eqwn\edlabel{1}  to'inun t~hc >em~hc
\edtext{boul'hsewc}{\Dfootnote{boul~hc \latintext Socr.(bArm.) Anon.\,Cyz. v.\,Ath.}} t`o
gn'wrisma \edtext{<'apasi}{\Dfootnote{p~asi \latintext Socr. Soz. Anon.\,Cyz. v.\,Ath.}} to~ic
boulom'enoic e>ic t`hn >ekklhs'ian e>iselje~in >ak'wluton
\edtext{par'asqou}{\Dfootnote{par'asqe \latintext Anon.\,Cyz.}} t`hn e>'isodon. >e`an g`ar
\edtext{\abb{gn~w}}{\Dfootnote{\latintext > v.\,Ath.}}
\edtext{\abb{<wc}}{\Dfootnote{\latintext Ath.(R\corr)  Socr.(bA Arm.) Soz. Anon.\,Cyz.
\greektext + e>i \latintext Ath.(BKOR*E*) v.\,Ath. \greektext <wse`i >'h \latintext
Ath.(E\corr) Socr.(T)}} kek'wluk'ac \edtext{tinac
\edtext{\abb{a>ut~wn}}{\Dfootnote{\latintext >
Anon.\,Cyz.}}}{\lemma{\abb{}}\Dfootnote{\responsio\ a>ut~wn tinac \latintext Soz.}} t~hc
\edtext{>ekklhs'iac}{\Dfootnote{>ekklhsiastik~hc \latintext Anon.\,Cyz.}}
\edtext{\abb{metapoioum'enouc}}{\Dfootnote{+ p'istewc \latintext Anon.\,Cyz.}} >`h
\edtext{\abb{>ape~irxac}}{\Dfootnote{+ to`uc toio'utouc \latintext Anon.\,Cyz.}} t~hc
e>is'odou, \edtext{>apostel~w}{\Dfootnote{>apost'ellw \latintext
Socr.(M\textsuperscript{1}Arm.) \greektext >apostell~w \latintext Socr.(A)}}
\edtext{paraqr~hma}{\Dfootnote{paraut'ika \latintext Ath.(B)}} t`on
\edtext{\abb{ka`i}}{\Dfootnote{\latintext > Socr.(bA Arm.) Soz. Anon.\,Cyz.}} kajair'hsont'a
se >ex >em~hc kele'usewc ka`i \edtext{t~wn t'opwn}{\Dfootnote{t`on t'opon \latintext
Ath.(K) Socr.(M\textsuperscript{1}) v.\,Ath. \greektext to~u t'opou \latintext
Socr.(M\textsuperscript{r}) \greektext to~u t'opou to'utou \latintext Socr.(Arm.) +
\greektext se \latintext v.\,Ath.}} \edtext{metast'hsonta}{\Dfootnote{metast'hsanta
\latintext Ath.(E)}}.\edlabel{2}
\pend
\pstart
\kap{3}>Epeid`h to'inun \edtext{\abb{ka`i}}{\Dfootnote{\latintext > Ath.(E)}} basil'ea
gr'afwn >'epeijon mhdem'ian \edtext{e>~inai
koinwn'ian}{\lemma{\abb{}}\Dfootnote{\responsio\ koinwn'ian e~>inai \latintext Ath.(KO)}}
t~h| qristom'aqw| a<ir'esei pr`oc t`hn kajolik`hn >ekklhs'ian, t'ote loip`on E>us'ebioc\edindex[namen]{Eusebius!Bischof von Nikomedien}
t`on kair`on <`on sunef'wnhse met`a t~wn Melitian~wn\edindex[namen]{Melitianer} prof'erwn gr'afei ka`i pe'ijei
\edtext{to'utouc}{\Dfootnote{to'utoic \latintext Ath.(E)}} pl'asasjai pr'ofasin, <'in'',
<'wsper kat`a P'etrou\edindex[namen]{Petrus!Bischof von Alexandrien} ka`i >Aqill~a\edindex[namen]{Achillas!Bischof von Alexandrien} ka`i >Alex'androu\edindex[namen]{Alexander!Bischof von Alexandrien} memelet'hkasin, o<'utw ka`i kaj''
<hm~wn >epino'hswsi ka`i jrul'hswsi.
\pend
\endnumbering
\end{Leftside}
\begin{Rightside}
\begin{translatio}
\beginnumbering
\pstart
\noindent\kapR{1}Als Eusebius (von Nicomedia), der auch die arianische
H�resie anf�hrte, also dies erfuhr\footnoteA{Gemeint sind erneute Proteste der Melitianer
(vgl. Dok. \ref{ch:23} = Urk. 23) in Alexandrien, die sich wohl auf die Bischofswahl des Athanasius bezogen
haben d�rften (Soz., h.\,e. II 17,4; Philost., h.\,e. II 11; Epiph., haer. 68,7,3~f.; 69,11,4; vgl. Ath., apol.\,sec. 6,4).}, schickte er nach den Melitianern und kaufte sie mit vielen Versprechungen.
Und er wurde heimlich ihr Freund und verband sich mit ihnen so lange, wie es ihm
gefiel. Als erstes nun wandte er sich an mich und legte mir nahe, die um Arius
aufzunehmen, das hei�t m�ndlich drohte er mir, schriftlich bat er mich. Da ich aber
widersprach und erkl�rte, da� man nicht die aufnehmen darf, die im Widerspruch zur
Wahrheit eine H�resie erfinden und von der �kumenischen Synode verurteilt wurden, erwirkte
er, da� auch der Kaiser, der selige Konstantin, mir schrieb und damit drohte, da� ich
dieses erleiden werde, was ich zuvor erlitten habe und auch jetzt noch erleide\footnoteA{Athanasius
blickt auf seine wiederholten Vertreibungen aus Alexandrien und jahrelangen Aufenthalte im
Exil zur�ck. Da nicht eindeutig ist, wann Athanasius die apol.\,sec. verfa�t oder auch
erg�nzt hat, bleibt unklar, auf welchen Zeitpunkt sich dieses ">jetzt"< bezieht (2. oder
3. Exil?).}, falls ich die um Arius nicht aufnehme. Das Folgende ist ein Auszug aus
diesem Brief, und Syncletius und Gaudentius, die Palatini\footnoteA{Syncletius und
Gaudentius sind auch die �berbringer von Dok. \ref{ch:33} = Urk. 33 und Dok. \ref{ch:34} = Urk. 34; dort werden sie als
Magistrianoi (\textit{agentes in rebus}) bezeichnet.}, waren es, die
den Brief �berbrachten.
\pend
\pstart
\kapR{2}\textit{Auszug aus einem Brief Kaiser Konstantins:}
\pend
\pstart
Da du also Kenntnis von meinem Willen hast, gew�hre ungehinderten
Zutritt allen, die in die Kirche hinein wollen. Denn wenn ich erfahre, da� du welche von ihnen, die
nach der Kirchengemeinschaft streben, daran gehindert oder es verweigert hast, sie
einzulassen, werde ich sofort jemanden schicken, der dich auf meinen Befehl hin absetzen
und von deinem Sitz vertreiben wird.
\pend
\pstart
\kapR{3}Als ich daraufhin auch dem Kaiser schrieb\footnoteA{Ein entsprechender Brief ist
nicht �berliefert; vgl. die einleitenden Bemerkungen.} und ihn davon �berzeugte, da� die
christusfeindliche H�resie auf keinen Fall mit der katholischen Kirche Gemeinschaft habe,
da ergriff Eusebius also die Gelegenheit zu dem Zeitpunkt, den er mit den
Melitianern ausgemacht hatte, schrieb ihnen und
�berzeugte sie, einen Vorwand zu erfinden, um so, wie sie gegen Petrus,
Achillas und Alexander vorgegangen waren, auch gegen uns Pl�ne zu schmieden und Ger�chte zu
verbreiten.
\pend
\endnumbering
\end{translatio}
\end{Rightside}
\Columns
\end{pairs}
% \thispagestyle{empty}
%%% Local Variables: 
%%% mode: latex
%%% TeX-master: "dokumente_master"
%%% End: 


%% DOKUMENT 39 %%%
\input{39.1.tex}
% \chapter[Die Synode von Jerusalem]{Die Synode von Jerusalem}
\chapter{Rundbrief der Synode von Jerusalem des Jahres 335}
\thispagestyle{empty}
% \label{ch:39}
\label{ch:Jerusalem335}
% \section{Schreiben der Synode von Jerusalem an die �gypter und die gesamte Kirche}
% \label{sec:39.2}
\label{sec:BriefJerusalem}
\begin{praefatio}
  \begin{description}
  \item[September 335]Der Kaiser lud die Anfang September noch in
    Tyrus\index[synoden]{Tyrus!a. 335} tagenden Bisch�fe nach
    Jerusalem\index[synoden]{Jerusalem!a. 335} zur Weihe der Anastasis
    ein (vgl. Eus., v.\,C. IV 43,1~f.; Soz., h.\,e. II 26,1; Socr.,
    h.\,e. I 33,1), die anl��lich seines drei�igj�hrigen
    Regierungsjubil�ums (nach chron. pasch. ad annum 335 [531 Dindorf]
    am 25. Juli 335) stattfinden sollte. Als Datum der Einweihung ist
    der 13. September 335 anzunehmen
    (\cite[222,9--12]{Renoux:CodexII},
    \cite[Nr. 1234]{Tarchnischvili:LectionnaireII}; chron. pasch. ad
    annum 334 [531 Dindorf] nennt demgegen�ber den 17. September 334),
    sp�ter wurde die j�hrlich wiederkehrende Feier durch die Feier der
    Kreuzauffindung am 14. September erg�nzt und mit einer Festwoche
    begangen (Itinerarium Egeriae 48~f.; Theodosius, De situ terrae
    sanctae 120).

    Wie sich aus dem Brief ergibt, befahl der Kaiser offenbar
    nachdr�cklich, im Rahmen dieser Feierlichkeiten endg�ltig die
    Einheit der Kirche wiederherzustellen. Besonders angesprochen sind
    Presbyter um Arius (siehe � 5), die zuvor schon beim Kaiser
    gewesen waren und ihm ein nicht mehr �berliefertes schriftliches
    Zeugnis ihres rechten Glaubens vorgelegt hatten. Gem�� dieses
    Auftrags wurden die Anh�nger des Arius von den in Jerusalem
    versammelten Bisch�fen wieder aufgenommen, und mit diesem Brief
    wird ein entsprechendes Verhalten den �gyptern nahegelegt. Es
    scheint das Anliegen des Kaisers gewesen zu sein, nun, einige
    Jahre nach dem Tod des Arius, in �gypten die Gemeinschaft zwischen
    den Anh�ngern des Athanasius und denjenigen des Arius
    wiederherzustellen. Zum Irrtum der Kirchenhistoriker Socrates und
    Sozomenus, die Rehabilitierung des Arius und Euzoius
    (Dok. \ref{ch:29} = Urk. 29; Dok. \ref{ch:30} = Urk. 30) mit
    dieser Synode in Verbindung zu bringen, vgl. die einleitenden
    Bemerkungen zur Chronologie. Auch der Brief des Kaisers, den
    Sozomenus h.\,e. II 27,13 zusammenfa�t, geh�rt offenbar zur
    fr�heren Rehabilitierung des Arius und ist somit chronologisch
    nach Dok. \ref{ch:30} = Urk. 30 einzuordnen (s. Anm. zu � 2).
  \item[�berlieferung]Athanasius zitiert in apol.\,sec. nur den Beginn
    des Briefes, der f�r seine Argumentation dort einschl�gig
    ist. Au�erdem ist der in apol.\,sec. �berlieferte Text durch
    kleinere Auslassungen gekennzeichnet und an einigen Stellen
    verderbt, so da� der �berlieferung in syn. der Vorzug zu geben
    ist.
  \item[Fundstelle]Ath., syn. 21,2--7 (\editioncite[247,22--248,17]{Opitz1935});
    \refpassage{Jm-Anf}{>orjodox'ian} Ath., apol.\,sec. 84,2--4
    (\editioncite[162,28--163,10]{Opitz1935})
  \end{description}
\end{praefatio}
\begin{pairs}
\selectlanguage{polutonikogreek}
\begin{Leftside}
\beginnumbering
\pstart
\hskip -1.25em\edtext{\abb{}}{\killnumber\Cfootnote{\hskip -1.1em\latintext syn.(BKPOR)
apol.\,sec.(BOR)}}\edlabel{Jm-Anf}
\specialindex{quellen}{chapter}{Athanasius!syn.!21,2--7}\specialindex{quellen}{chapter}{Athanasius!apol.\,sec.!84,2--4}
\kap{1}<H <ag'ia s'unodoc <h >en <Ierosol'umoic\edindex[synoden]{Jerusalem!a. 335} jeo~u
q'ariti sunaqje~isa t~h| >ekklhs'ia| to~u jeo~u t~h| >en
>Alexandre'ia|\edindex[namen]{Alexandrien} ka`i \edtext{to~ic}{\Dfootnote{ta~ic \latintext
syn.(R)}} kat`a p~asan t`hn A>'igupton\edindex[namen]{Aegyptus} ka`i
Jhba'ida\edindex[namen]{Theba"is} ka`i Lib'uhn\edindex[namen]{Libya} {ka`i}
Pent'apolin\edindex[namen]{Pentapolis} ka`i to~ic kat`a t`hn o>ikoum'enhn >episk'opoic
ka`i presbut'eroic ka`i diak'onoic >en kur'iw| qa'irein.
\pend
\pstart
\kap{2}p~asi m`en <hm~in to~ic >ep`i t`o a>ut`o suneljo~usin >ex >eparqi~wn diaf'orwn
pr`oc t~h| meg'alh| panhg'urei, <`hn >ep`i \edtext{\abb{t~h|}}{\Dfootnote{\latintext >
apol.\,sec.(B)}} >afier'wsei to~u swthr'iou martur'iou spoud~h| to~u jeofilest'atou
basil'ewc Kwnstant'inou\edindex[namen]{Konstantin, Kaiser} t~w| p'antwn
basile~i je~w| ka`i t~w| Qrist~w| a>uto~u
kataskeuasj'entoc >epetel'esamen, ple'iona jumhd'ian <h to~u
\edtext{Qristo~u}{\Dfootnote{jeo~u \latintext apol.\,sec.}} q'aric
\edtext{\abb{par'esqen}}{\Dfootnote{\latintext > apol.\,sec.}}, \edtext{<`hn
>epo'ihsen}{\Dfootnote{>enepo'ihsen \latintext apol.\,sec.}} a>ut'oc te <o jeofil'estatoc
basile`uc di`a gramm'atwn o>ike'iwn to~uj'', <'op\-er >eqr~hn, parorm'hsac p'anta m`en
\edtext{\abb{>exor'isai}}{\Dfootnote{\latintext coni. Opitz \greektext >exor'isac \latintext
syn. apol.\,sec.}} t~hc >ekklhs'iac to~u jeo~u fj'onon ka`i p~asan makr`an
\edtext{\abb{>apel'asai}}{\Dfootnote{\latintext coni. Opitz \greektext >apel'asac \latintext
syn. apol.\,sec.}} baskan'ian, di'' <~hc t`a to~u \edtext{Qristo~u}{\Dfootnote{jeo~u
\latintext apol.\,sec.}} \edtext{\abb{m'elh}}{\Afootnote{\latintext vgl. 1Cor 12,12}}
\edindex[bibel]{Korinther I!12,12|textit}
\edtext{\abb{p'alai}}{\Dfootnote{\latintext > apol.\,sec.(BO)}} pr'oteron dieist'hkei,
<hplwm'enh| d`e ka`i e>irhna'ia| yuq~h| d'exasjai  to`uc per`i
>'Areion\edindex[namen]{Arianer}, o<`uc pr'oc tina
kair`on <o mis'okaloc fj'onoc >'exw gen'esjai t~hc >ekklhs'iac e>irg'asato.
\kap{3}>emart'urei d`e to~ic >andr'asin <o jeofil'estatoc basile`uc di`a t~hc >epistol~hc
p'istewc >orjotom'ian, <`hn par'' a>ut~wn puj'omenoc a>ut'oc te di'' <eauto~u par`a z'wshc
fwn~hc \edtext{\abb{a>ut~wn}}{\Dfootnote{\latintext > apol.\,sec.(BO)}} >ako'usac
>aped'exato <hm~in te \edtext{faner`an}{\Dfootnote{faner`on \latintext syn.(P)}}
katest'hsato, <upot'axac to~ic <eauto~u gr'ammasin >'eggrafon t`hn t~wn >andr~wn
\edtext{\abb{>orjodox'ian}}{\Cfootnote{\dt{des. apol.\,sec.(BOR)}}}\edlabel{>orjodox'ian},
<`hn >ep'egnwmen o<i p'antec <ugi~h te o>~usan ka`i >ekklhsiastik'hn.
\pend
\pstart
\kap{4}ka`i e>ik'otwc parek'alei to`uc >'andrac <upodeqj~hnai ka`i <enwj~hnai t~h|
>ekklhs'ia| to~u jeo~u, <'wsper o>~un ka`i a>uto`i >ek t~wn >isot'upwn e>'isesje t~hc
a>ut~hc >epistol~hc, <~hc pr`oc t`hn <umet'eran e>ul'abeian diepemy'ameja.
\kap{5}piste'uomen <'oti ka`i \edtext{<hm~in}{\Dfootnote{<hm~in ka`i \latintext syn.(R)}}
a>uto~ic, <wc >`an t`a o>ike~ia m'elh to~u <umet'erou s'wmatoc
\edtext{>apolamb'anousi}{\Dfootnote {>apolamb'anwsi \latintext syn.(B)}}, meg'alh qar`a ka`i
e>ufros'unh gen'hsetai t`a <eaut~wn spl'agqna ka`i to`uc <eaut~wn >adelfo'uc te ka`i
pat'erac gnwr'izous'i te ka`i >apolamb'anousin, o>u \edtext{m'onon}{\Dfootnote{m'onwn
\latintext syn.(R)}} t~wn presbut'erwn t~wn per`i >'Ar\-eion\edindex[namen]{Arianer}
>apodoj'entwn <um~in, >all`a ka`i
pant`oc to~u lao~u ka`i t~hc plhj'uoc <ap'ashc, <`h prof'asei t~wn e>irhm'enwn >andr~wn
makr~w| qr'onw| par'' <um~in dieist'hkei.
\kap{6}ka`i pr'epei ge >alhj~wc gn'ontac <um~ac t`a pepragm'ena, ka`i <wc >ekoin'wnhsan
o<i >'andrec pared'eqjhs'an te <up`o t~hc tosa'uthc <ag\-'iac sun'odou, projum'otata ka`i
a>uto`uc >asp'asasjai t`hn pr`oc t`a o>ike~ia m'elh sun\-'af\-ei\-'an te ka`i e>ir'hnhn, <'oti
m'alista t`a t~hc >ekteje'ishc <up'' a>ut~wn p'istewc >anamf'hriston s'wzei t`hn par`a
to~ic p~asin <omologoum'enhn >apostolik`hn par'ados'in te ka`i didaskal'ian.
\pend
\endnumbering
\end{Leftside}
\begin{Rightside}
\begin{translatio}
\beginnumbering
\pstart
\noindent\kapR{1}Die heilige Synode, die durch die Gnade Gottes in Jerusalem zusammengef�hrt
worden ist, gr��t im Herrn die Kirche Gottes in
Alexandrien und die Bisch�fe, Presbyter und Diakone in ganz Aegyptus, der Theba"is, Libya,
der Pentapolis und der ganzen Welt.
\pend
\pstart
\kapR{2}Uns, die wir uns alle hier am selben Ort aus den verschiedenen
Provinzen\footnoteA{Vgl. Eus., v.\,C. IV 43: Hier findet sich eine Liste der Provinzen,
deren Bisch�fe an der Synode teilnehmen.} zu dem gro�en Fest versammelt haben, das wir
anl��lich der Weihe des Martyrions des Erl�sers begingen, das durch den Eifer des
gottgef�lligsten Kaisers Konstantin f�r den Gott und K�nig aller und seinen Christus
erbaut wurde,~-- uns gew�hrte die Gnade Christi eine noch gr��ere Herzensfreude. Diese
bereitete unser gottgeliebtester Kaiser selbst, indem er uns mit seinem eigenen
Brief\footnoteA{Dieser Brief selbst ist nicht erhalten. Das Regest eines Schreibens
Konstantins an eine Jerusalemer Synode �ber Verhandlungen zur Wiederaufnahme des Arius und
Euzoius bei Soz., h.\,e. II 27,13 geh�rt aufgrund einiger Widerspr�che nicht zu dieser
Jerusalemer Synode, sondern zu der fr�heren Rehabilitierung dieser beiden und damit zu 
Dok. \ref{ch:29} = Urk. 29; Dok. \ref{ch:30} = Urk. 30.
Hier im Schreiben der Jerusalemer Synode von 335 hei�t es, da� der Kaiser den
Glauben der Presbyter um Arius schon �berpr�ft und f�r rechtm��ig beurteilt habe, so da�
er hier nur noch eine synodale Best�tigung einfordert. In Soz., h.\,e. II 27,13 dagegen wird
berichtet, da� der Kaiser eine Glaubenspr�fung erst von der Synode erw�nscht, deren
Verhandlungen demzufolge im Ergebnis offen bleiben, auch wenn der Kaiser ein mildes Urteil
empfiehlt.} dr�ngte, das zu tun, was getan werden mu�te, n�mlich einerseits alle Mi�gunst
aus der Kirche Gottes zu beseitigen und die ganze, weitverbreitete Bosheit zu beenden,
durch die die Glieder Christi lange zuvor voneinander getrennt worden waren, und
andererseits mit einf�ltigem und friedlichem Herzen die Anh�nger des Arius aufzunehmen, 
die der das Gute hassende Neid f�r eine
gewisse Zeit au�erhalb der Kirche stehen lie�.
\kapR{3}Unser gottgeliebtester Kaiser legte durch seinen Brief f�r diese M�nner Zeugnis �ber 
die Rechtm��igkeit ihres Glaubens ab. Nachdem er ihn von ihnen erfragt und selbst h�chstpers�nlich
aus ihrem eigenen Mund geh�rt hatte, hie� er diesen Glauben gut und tat ihn uns kund, indem er 
an seinen eigenen Brief das schriftliche Zeugnis der Rechtgl�ubigkeit der M�nner 
anh�ngte\footnoteA{Dieses Bekenntnis ist nicht �berliefert.}, das wir dann alle als richtig und 
mit der Lehre der Kirche �bereinstimmend anerkannten.
\pend
\pstart
\kapR{4}Zu Recht bat er, da� die M�nner aufgenommen und mit der Kirche Gottes vereint
werden, wie ihr auch selbst aus der Kopie des n�mlichen Briefes erfahren werdet, den wir
an euere Hochw�rden geschickt haben.
\kapR{5}Wir glauben, da� auch ihr selbst, als ob ihr die eigenen Glieder eueres K�rpers
wiederaufnehmen w�rdet, gro�e Freude und Gl�ck empfinden werdet, wenn ihr die eigenen
Glieder, die eigenen Br�der und V�ter anerkennt und wiederaufnehmt, da euch nicht nur die
Presbyter um Arius zur�ckgegeben worden sind, sondern auch das ganze Volk und die ganze Menge,
die aufgrund besagter M�nner f�r lange Zeit von euch getrennt worden war.
\kapR{6}Und es ziemt sich wahrlich, da� ihr, die ihr um das Geschehene wi�t~-- und da die
M�nner mit uns Gemeinschaft hatten und von einer so gro�en, heiligen Synode
wiederaufgenommen wurden~-- da� auch ihr sie absolut bereitwillig willkommen hei�t in der
Vereinigung mit den eigenen Gliedern und in Frieden, da insbesondere die Positionen ihres
Glaubens, der von ihnen bekannt wurde, ohne Zweifel die apostolische �berlieferung und
Lehre, in der alle �bereinstimmen, bewahren.
\pend
\endnumbering
\end{translatio}
\end{Rightside}
\Columns
\end{pairs}
%%% Local Variables: 
%%% mode: latex
%%% TeX-master: "dokumente_master"
%%% End: 


%% DOKUMENT 40 %%%
% \renewcommand*{\goalfraction}{.7}
%% Erstellt von uh
%% �nderungen:
%%%% 13.5.2004 Layout-Korrekturen (avs) %%%%
\chapter{Berichte �ber die Synode in Konstantinopel gegen Markell von Ancyra}
\thispagestyle{empty}
% \label{ch:40}
\label{ch:Konstantinopel336}
\begin{praefatio}
  \begin{description}
  \item[336?]Die Datierung ist umstritten, je nachdem, ob das Fehlen
    Markells\index[namen]{Markell!Bischof von Ancyra} auf der
    Kirchweihsynode von Jerusalem\index[synoden]{Jerusalem!a. 335}
    (Dok. \ref{ch:Konstantinopel336},3,2) als Beleg f�r seine schon erfolgte
    Absetzung angesehen wird oder nicht. Da aber von einer
    Ablehnung der Beschl�sse durch
    Markell\index[namen]{Markell!Bischof von Ancyra} selbst die Rede ist,
    wird man dem chronologischen Zusammenhang bei Socrates und
    Sozomenus folgen und die
    Konstantinopler\index[synoden]{Konstantinopel!a. 336?} Synode zur
    Absetzung Markells\index[namen]{Markell!Bischof von Ancyra} an das
    Ende der Regierungszeit Konstantins\index[namen]{Konstantin,
      Kaiser} setzen k�nnen. Allerdings ist die Synode entgegen der
    Darstellung bei Socrates (h.\,e. I 36,1) von der Versammlung
    einiger Synodaler aus Tyrus\index[synoden]{Tyrus!a. 335} in
    Konstantinopel\index[synoden]{Konstantinopel!a. 336?} aufgrund der
    Differenzen um Athanasius\index[namen]{Athanasius!Bischof von
      Alexandrien} zu unterscheiden, da sich gegen
    Markell\index[namen]{Markell!Bischof von Ancyra} eine andere Teilnehmerschar
    zusammenfand. Die bei Eusebius �berlieferten Fragmente
    aus Markells\index[namen]{Markell!Bischof von Ancyra} Schrift
    gegen Asterius\index[namen]{Asterius!Sophist}, die den Anla� zu
    seiner Verurteilung lieferten und nach Beschlu� der Synode
    aufgesp�rt und vernichtet werden sollten, geben zwar keinen Hinweis auf eine
    Datierung, aber Eusebius bald erfolgte Widerlegung \textit{Contra
      Marcellum} scheint kurz nach dem Tod des Kaisers
    Konstantin\index[namen]{Konstantin, Kaiser} geschrieben worden zu
    sein (\cite[242~f.]{Seibt94}), was die Datierung der Synode gegen
    Markell\index[namen]{Markell!Bischof von Ancyra} an das Ende der
    Regierungszeit des Kaisers zu best�tigen scheint.
  \item[�berlieferung]Den Ereignissen am n�chsten steht der Hinweis
    bei Eusebius (Marcell. II 4,29), der sicher selbst bei der
    Verurteilung von Markell\index[namen]{Markell!Bischof von Ancyra}
    anwesend gewesen ist. Der zweite Bericht ist im Synodalschreiben
    der �stlichen Teilsynode von
    Serdica\index[synoden]{Serdica!a. 343} �berliefert (Hil.,
    coll.\,antiar. A IV 1,3; zum Text vgl. unten
    Dok. \ref{sec:RundbriefSerdikaOst}), da dort gegen die r�mische
    Wiederaufnahme Markells\index[namen]{Markell!Bischof von Ancyra}
    in die Kirchengemeinschaft protestiert wird. Ferner liefert der
    Kirchenhistoriker Sozomenus (h.\,e. II 33) die ausf�hrlichste
    Darstellung; ihm scheinen Synodalakten vorgelegen zu haben (Socr.,
    h.\,e. I 36 ist wesentlich k�rzer). Die Erw�hnungen bei Athanasius
    bringen keinen inhaltlichen Gewinn �ber die Information hinaus,
    da� Markell\index[namen]{Markell!Bischof von Ancyra}
    ">ungerechterweise"< verurteilt worden sei (Ath., h.\,Ar. 6;
    fug. 3).
  \item[Fundstelle]\refpassage{40}{41} Eus., Marcell. II 4,29
    (\editioncite[58,4--11]{Klostermann}); \refpassage{42}{43}
    Dok. \ref{sec:RundbriefSerdikaOst},4
    (\refpassage{43.11:42}{43.11:43} = Hil., coll.\,antiar. A IV 1,3
    [\editioncite[50~f.]{Hil:coll}]); \refpassage{44}{45} Soz., h.\,e. II 33
    (\editioncite[98,12--99,8]{Hansen:Soz})
  \end{description}
\end{praefatio}
\autor{Der Bericht des Eusebius}
\begin{pairs}
\selectlanguage{polutonikogreek}
\begin{Leftside}
\beginnumbering
\pstart
\hskip -1.25em\edtext{\abb{}}{\xxref{40}{41}\Cfootnote{\latintext
Eus.(V)}}\specialindex{quellen}{chapter}{Eusebius von Caesarea!Marcell.!II 4,29}
\kap{1}e>ik'otwc\edlabel{40} >'ara ta~uta basil'ea t`on <wc >alhj~wc jeofil~h ka`i
trismak'arion kat`a to~u >andr`oc >ek'inei, ka'itoi mur'ia kolake'usantoc ka`i poll`a
basil'ewc >egk'wmia a>uto~u >en suggr'ammati dielj'ontoc. ta~uta d`e ka`i t`hn <ag'ian
s'unodon >en t~h| basilik~h| suneljo~usan p'olei >ex >eparqi~wn diaf'orwn
P'ontou\edindex[namen]{Pontus} ka`i Kappadok'iac\edindex[namen]{Cappadocia}
>As'iac\edindex[namen]{Asia} te ka`i Frug'iac\edindex[namen]{Phrygia} ka`i
Bijun'iac\edindex[namen]{Bithynia} Jr'a|khc\edindex[namen]{Thracia} te ka`i t~wn
>ep\-'ek\-ei\-na mer~wn sthlite'uein t`on >'andra di`a t~hc kat'' a>uto~u graf~hc ka`i m`h
j'elousan >exebi'azeto.\edlabel{41}
\pend
\end{Leftside}
\begin{Rightside}
\begin{translatio}
\beginnumbering
\pstart
% \autor{Eusebius}
\noindent Mit Recht also brachte dies (sc. Markells Lehren) den wahrhaft von Gott geliebten und dreimal
gesegneten Kaiser gegen diesen Mann auf, auch wenn er zahllose Schmeicheleien und viele
Lobeshymnen auf den Kaiser in seiner Schrift\footnoteA{Von dieser Schrift, von
Markell gegen die ">Eusebianer"< und insbesondere gegen Asterius verfa�t und dem Kaiser
Konstantin gewidmet, sind nur noch die Fragmente �berliefert, die Eusebius von Caesarea in
seiner Widerlegung (\textit{Contra Marcellum} und \textit{De ecclesiastica theologia})
anf�hrt (Seibt/Vinzent). Die Fragmente belegen diese ">Schmeicheleien"< nicht,
nur eine direkte Anrede an den Kaiser in fr. 2, 6, 16 (Seibt/Vinzent).
Eusebius berichtet aber ferner, da� Markell seine Schrift dem Kaiser pers�nlich �berbracht
und um sein Urteil gebeten habe (Marcell. II 30~f.), wohl zu dem Zweck, da� der Kaiser gegen
die von Markell theologisch kritisierten Bisch�fe einschreite. F�r weitere theologische
Hinweise vgl. bes. Dok. \ref{sec:MarkellJulius}.} einstreute. Dies aber zwang auch die
heilige Synode, die in der kaiserlichen Stadt aus diversen Provinzen, n�mlich aus Pontus, Cappadocia,
Asia, Phrygia, Bithynia, Thracia und jenseitigen Teilen zusammengekommen war, diesen
Menschen wegen seiner gegen ihn gerichteten Schrift zu verwerfen, auch wenn sie es nicht
wollte.
\pend
\end{translatio}
\end{Rightside}
\Columns
\end{pairs}
\clearpage
\autor{Der Bericht in Dok. \ref{sec:RundbriefSerdikaOst},4}
\begin{pairs}
\selectlanguage{latin}
\begin{Leftside}
\pstart
\hskip -1.25em\edtext{\abb{}}{\xxref{42}{43}\Cfootnote{Hil.(A)}}\specialindex{quellen}{chapter}{Hilarius!coll.\,antiar.!A IV 1,3} %% Bezeugungsleiste
\kap{2,1}Magna\edlabel{42} autem fuit
\edtext{parentibus nostris atque maioribus}{\lemma{\abb{}}
\Dfootnote{\responsio\ parentibus atque maioribus nostris \textit{coni. Faber}}}
sollicitudo de supradicta praedicatione
sacrilega.
\edtext{\abb{condicitur}}{\Dfootnote{\textit{coni. edd.} conditur \textit{A}}} namque in
Constantinopolim\edindex[namen]{Konstantinopel} civitatem
sub praesentia beatissimae memoriae
Constantini\edindex[namen]{Konstantin, Kaiser} imperatoris concilium
\edtext{\abb{episcoporum}}{\lemma{\abb{}} \Dfootnote{episcoporum qui \textit{coni. Faber}}}.
\pend
\pstart
\kap{2}ex multis Orientalium provinciis advenerunt, ut
hominem malis rebus inbutum salubri consilio reformarent et ille ammonitione,
sanctissima
\edtext{\abb{correptione}}{\Dfootnote{\textit{del. Faber}}}, correptus a
\edtext{sacrilega praedicatione}{\lemma{\abb{}}\Dfootnote{\responsio\
praedicatione sacrilega \textit{coni. C}}} discederet. quique increpantes illum et exprobantes
necnon
\edtext{\abb{etiam}}{\Dfootnote{\textit{del. C}}} caritatis affectu postulantes
multo tempore nec quicquam proficiebant.
\pend
\pstart
\kap{3}namque post unam
et secundam multasque correptiones cum nihil proficere potuissent -- perdurabat
enim et
contradicebat rectae fidei et contentione maligna ecclesiae catholicae
resistebat --,
\edtext{exinde}{\Dfootnote{et inde \textit{coni. C*}}} illum
omnes
\edtext{\abb{horrere}}{\Dfootnote{\textit{coni. C vel edd.} horrore \textit{A}}} ac vitare
\edtext{\abb{coeperunt}}{\Dfootnote{\textit{coni. C vel edd.} ceperunt \textit{A}}}. et videntes,
quoniam
subversus est
\edtext{et}{\Dfootnote{a \textit{coni. Faber}}}
\edtext{\abb{peccato}}{\Dfootnote{vel peccatis
\textit{coni. Faber} peccata \textit{A}}} et est a semet ipso
damnatus, actis eum ecclesiasticis damnaverunt, ne ulterius oves Christi
pestiferis contactibus
malis macularet.
\pend
\pstart
\kap{4}tunc namque etiam eius aliquos pravissimos sensus contra fidem
rectam ecclesiamque
sanctissimam propter memoriam posterorum cautelamque suis sanctissimis
scripturis in archivo
ecclesiae condiderunt.
\pend
\pstart
\kap{5}sed haec quidem secundum impietatem Marcelli\edindex[namen]{Markell!Bischof von Ancyra} haeretici
prima fuerunt; peiora
sunt deinde subsecuta. nam quis fidelium credat aut patiatur ea, quae ab ipso
male gesta atque
conscripta sunt quaeque digne anathematizata sunt
\edtext{\abb{iam}}{\Dfootnote{\textit{coni. Feder} nam \textit{A} \textit{del. vel lac.
susp. Coustant}}} cum ipso Marcello\edindex[namen]{Markell!Bischof von Ancyra} a parentibus nostris in
\edtext{\abb{Constantinopoli civitate}}{\Dfootnote{\textit{coni. C vel edd.} Constantinopolim
civitatem \textit{A}}}\edindex[namen]{Konstantinopel}? namque liber sententiarum extat in
ipsum ab episcopis
conscriptus, in
quo libro etiam isti qui nunc cum ipso sunt Marcello\edindex[namen]{Markell!Bischof von Ancyra} atque illi favent, id est
Protogenes\edindex[namen]{Protogenes!Bischof von Serdica}
\edtext{Sardicae}{\Dfootnote{Serdicae \textit{coni. Coustant}}}
civitatis episcopus et
\edtext{\abb{Cyriacus a Naiso}}{\Dfootnote{\textit{coni. Feder} Cyriacusanais \textit{A}
Siriacusanais \textit{coni. T} Siracusanus \textit{coni. Faber} \latintext
Ariminensis \textit{susp. Coustant}}}\edindex[namen]{Cyriacus!Bischof von Na"issus},
\edtext{\abb{in ipsum sententiam}}{\Dfootnote{\textit{coni. Coustant} in ipsa
sententia \textit{A}}} manu propria conscripserunt.
\pend
\pstart
\kap{6}quorum manus
\edtext{valens}{\Dfootnote{Valens \textit{coni. C\corr}}} testatur fidem sanctissimam nullo
genere
\edtext{\abb{mutandam}}{\Dfootnote{\textit{coni. C vel edd.} mutanda \textit{A}}} nec ecclesiam
sanctam
praedicatione
falsa
\edtext{\abb{subvertendam}}{\Dfootnote{\textit{coni. C vel edd.} subvertenda \textit{A}}}, ne hoc
modo
\edtext{\abb{pestis}}{\Dfootnote{\textit{coni. C vel edd.} pestes \textit{A}}} ac lues animarum
\edtext{hominibus}{\Dfootnote{omnibus \textit{coni. Faber}}} gravissima
importetur Paulo dicente:
\edtext{sive nos sive angelus de caelo aliter annuntiaverit vobis quam quod
accepistis, anathema sit.}{\lemma{\abb{}}\Afootnote{Gal 1,8-9}}\edindex[bibel]{Galater!1,8--9}\edindex[bibel]{Galater!1,8--9}\edlabel{43}
%% Hil., coll.antiar. A IV 1,3 %%%
\pend
\end{Leftside}
\begin{Rightside}
\begin{translatio}
\pstart
% \autor{Hilarius}
\noindent Dennoch herrschte gro�e Sorge bei unseren V�tern und Vorg�ngern �ber die obengenannte,
gottlose Verk�ndigung. Man berief n�mlich nach Konstantinopel eine Bischofssynode
in Anwesenheit Kaiser Konstantins seligen Angedenkens ein.
\pend
\pstart
 Sie kamen aus vielen �stlichen
Provinzen, damit sie diesen von vielen �beln getr�nkten Menschen durch einen heilsamen
Beschlu� besserten und jener durch Ermahnung und heiligsten Tadel gescholten werde und von der
gottlosen Verk�ndigung ablasse. Und sie nahmen ihn in die Mangel, pr�ften ihn und stellten
gewi� auch lange Zeit ihre Forderungen mit wohlwollender Liebe, aber erreichten absolut nichts.
\pend
\pstart
Denn da sie nach einer, der zweiten und etlichen Ermahnungen nicht vorankommen konnten~-- er blieb n�mlich stur und widersprach dem rechten Glauben und widersetzte sich
in b�sartigem Streit der katholischen Kirche~--, begannen hierauf alle vor jenem
zur�ckzuschrecken und ihn zu meiden. Und als sie sahen, da� er von der S�nde zu
Fall gebracht und durch sich selbst verurteilt worden war, haben sie ihn durch kirchliche
Beschl�sse verurteilt, damit er nicht noch dar�ber hinaus die Schafe Christi durch
todbringende, �ble Ansteckungen beflecke.
\pend
\pstart
 Damals haben sie n�mlich auch einige seiner
krummsten Ansichten gegen den rechten Glauben und die heiligste Kirche zur Erinnerung
f�r die Sp�teren und zum Schutz f�r ihre heiligsten Schriften im Archiv der Kirche
hinterlegt.
\pend
\pstart
Aber dies waren nur die ersten Erlebnisse mit der Gottlosigkeit
des H�retikers Markell. Schlimmere sind hierauf gefolgt. Denn welcher Gl�ubige 
glaubt oder ertr�gt schon das, was von jenem eigens �bles angestellt und zusammengeschrieben wurde
und was zusammen mit seiner Person zu Recht schon von unseren V�tern in Konstantinopel 
mit dem Anathema belegt worden ist? Es liegt n�mlich ein gegen ihn von den Bisch�fen verfa�ter
\textit{Liber sententiarum}\footnoteA{Entsprechendes ist nicht �berliefert, d�rfte aber den Erkl�rungen
mit Unterschriften, die z.B. dem \textit{Tomus ad Antiochenus} (tom. 9,3--11,3) angeh�ngt wurden, vergleichbar sein.} vor, in dem auch die, die nun auf der Seite von Markell stehen und
ihn unterst�tzen, dies sind Protogenes, Bischof von Serdica,\footnoteA{Vgl. Dok. \ref{ch:SerdicaEinl} (Teilnehmerliste der ">westlichen"< Synode von Serdica) und den Brief Dok. \ref{sec:BriefOssiusProtogenes}.} und Cyriacus von Na"issus\footnoteA{In den Listen der Teilnehmer der ">westlichen"< Synode von Serdica
(vgl. Dok. \ref{ch:SerdicaEinl}) taucht Gaudentius von Na"issus auf, wohl sein
unmittelbarer Amtsnachfolger; vgl. Soz., h.\,e. III 11,8.}, gegen ihn mit eigener Hand Stellung
bezogen haben.
\pend
\pstart
Von deren Hand wird nachdr�cklich bezeugt, da� der heiligste Glaube in
keiner Weise ver�ndert werden und die heilige Kirche durch keine falsche Verk�ndigung
umgest�rzt werden darf, damit nicht auf diese Weise Verderben und f�r die Seelen schlimmstes Gift
in die Menschen hineingetragen werde, wie Paulus sagt: ">Seien es wir oder ein Engel
vom Himmel, der euch etwas anders verk�ndet hat, als ihr empfangen habt, er sei
ausgeschlossen."<
\pend
\end{translatio}
\end{Rightside}
\Columns
\end{pairs}

\autor{Der Bericht des Sozomenus}
\begin{pairs}
\selectlanguage{polutonikogreek}
\begin{Leftside}
\pstart
% \selectlanguage{polutonikogreek}
\hskip -1.25em\edtext{\abb{}}{\xxref{44}{45}\Cfootnote{\latintext Soz.(BC=b
Cod.Nic)}}\specialindex{quellen}{chapter}{Sozomenus!h.\,e.!II 33} %% Bezeugungsleiste
\kap{3,1}>En\edlabel{44} d`e t~w| t'ote ka`i M'arkellon t`on >Agk'urac
>ep'iskopon\edindex[namen]{Markell!Bischof von Ancyra} t~hc Galat~wn, <wc kain~wn
dogm'atwn e>ishght`hn ka`i t`on u<i`on to~u jeo~u l'egonta >ek Mar'iac t`hn >arq`hn
e>ilhf'enai ka`i t'eloc <'exein t`hn a>uto~u basile'ian, ka`i graf'hn tina per`i to'utou
sunt'axanta, sunelj'ontec >en Kwnstantinoup'olei\edindex[synoden]{Konstantinopel!a. 336?}
kaje~ilon ka`i t~hc >ekklhs'iac >ex'ebalon; ka`i
Basile'iw|\edindex[namen]{Basilius!Bischof von Ancyra} dein~w| l'egein ka`i >ep`i
paide'usei <upeilhmm'enw| >epitr'epousi t`hn >episkop`hn t~hc
Galat~wn\edindex[namen]{Galatia} paroik'iac. ka`i ta~ic a>ut'oji >ekklhs'iaic >'egrayan
>anazht~hsai t`hn Mark'ellou\edindex[namen]{Markell!Bischof von Ancyra} b'iblon ka`i
>exafan'isai ka`i to`uc t`a a>ut`a frono~untac, e>'i tinac e<'uroien, metab'allein.
\kap{2}ka`i di`a m`en t`o pol'ustiqon t~hc graf~hc m`h t`o p~an <upo\-t'ax\-ai bibl'ion
>ed'hlwsan, <rht`a d'e tina >en'ejhkan t~h| a>ut~wn >epistol~h| pr`oc >'elegqon to~u
dox'azein a>ut`on t'ade. >el'egeto d`e pr'oc tinwn ta~uta <wc >en zht'hsei e>ir~hsjai
Mark'ellw|\edindex[namen]{Markell!Bischof von Ancyra} ka`i <wc <wmologhm'ena diabebl~hsjai
ka`i a>ut~w| t~w| basile~i par`a t~wn >amf`i t`on E>us'ebion\edindex[namen]{Eusebianer},
kaj'oti <uperfu~wc >eqal'epainon \edtext{\abb{a>ut~w|}}{\Dfootnote{+ o~<ia \latintext
Cod.Nic.}} m'hte >en t~h| kat`a Foin'ikhn\edindex[synoden]{Tyrus!a. 335} sun'odw|
suntejeim'enw| to~ic <up'' a>ut~wn \edtext{\abb{<orisje~isi}}{\Dfootnote{+ kat`a
>Ajanas'iou \latintext Cod.Nic}} \edtext{\abb{m'hte}}{\Dfootnote{+ to~ic \latintext
Cod.Nic.}} >en <Iero\-so\-l'u\-moic\edindex[synoden]{Jerusalem!a. 335} >ep`i
>Are'iw|\edindex[namen]{Arius!Presbyter in Alexandrien} m'hte t~hc >af\-ier\-'wse\-wc to~u
meg'alou \edtext{martur'iou}{\Dfootnote{marture'iou \latintext Cod.Nic.}} metasq'onti di''
>apofug`hn t~hc pr`oc a>uto`uc koinwn'iac.
\pend
\pstart
>Am'elei\kap{3} toi gr'afontec per`i a>uto~u t~w| basile~i ka`i ta~uta e>ic diabol`hn
pro>'uferon, <wc ka`i a>uto~u <ubrism'enou <up'' a>uto~u m`h >ax\-i'w\-san\-toc t`hn >afi'erwsin
tim~hsai to~u >en <Ierosol'umoic\edindex[namen]{Jerusalem} ne`w o>ikodomhj'entoc.
\kap{4}pr'ofasic d`e g'egone
Mark'ellw|\edindex[namen]{Markell!Bischof von Ancyra} ta'uthc t~hc graf~hc
>Ast'eri'oc\edindex[namen]{Asterius!Sophist} \edtext{\abb{tic}}{\Dfootnote{\dt{coni. Hansen e
Socr.} te \dt{b}}} >ek Kappadok'iac\edindex[namen]{Cappadocia} sofist'hc, <`oc
ka`i per`i to~u d'ogmatoc l'ogouc
suggr'afwn t~h| >Are'iou\edindex[namen]{Arius!Presbyter in Alexandrien}
d'oxh| sumferom'enouc perii`wn t`ac p'oleic >epede'iknuto ka`i
\edtext{\abb{to~ic}}{\Dfootnote{+ ginom'enoic \latintext B}} >episk'opoic ka`i ta~ic
ginom'enaic sun\-'od\-oic <wc >ep'ipan \edtext{pareg'ineto}{\Dfootnote{pareg'eneto \latintext
B}}. >antil'egwn g`ar a>ut~w| M'arkelloc\edindex[namen]{Markell!Bischof von
Ancyra} >`h <ek`wn >`h o>uq o<'utw no'hsac e>ic t`hn
Pa'ulou to~u Samosat'ewc\edindex[namen]{Paulus von Samosata}
>exekul'isjh d'oxan. >all'' <o m`en
\Ladd{\edtext{\abb{>en}}{\Dfootnote{\latintext add. Bidez}}} t~h| >en Sardo~i sun'odw|
<'usteron t`hn >episkop`hn >ane'ilhfe m`h frone~in <~wde logis'amenoc.\edlabel{45}
%% Sozomenos, h.e. 33,1-3 %%%
\pend
\endnumbering
\end{Leftside}
\begin{Rightside}
\begin{translatio}
\pstart
% \autor{Sozomenus}
\noindent\kapR{II 33,1}Damals aber versammelten sie\footnoteA{Die um Eusebius von Nikomedien und
Theognis von Nicaea.} sich in Konstantinopel und verurteilten auch Markell, den Bischof von
Ancyra in Galatia, da er neue Lehren eingef�hrt und gesagt habe, da� der Sohn Gottes aus
Maria seinen Anfang genommen habe und da� seine Herrschaft ein Ende haben werde, und sogar
eine Schrift dar�ber verfa�t hatte, und schlossen ihn aus der Kirche aus. Und sie
�bertrugen Basilius\footnoteA{Basilius von Ancyra, der in den f�nfziger Jahren zeitweise
zu den engen kirchenpolitischen Beratern des Kaisers z�hlte, spielt auf den Synoden jener
Zeit eine f�hrende Rolle (auf der Synode in Sirmium gegen Photinus und beim Protest gegen
die Wahl des Eudoxius zum Bischof von Antiochien; vgl. Epiph., haer. 73), wurde im Januar
360 selbst auf der Konstantinopler Synode abgesetzt (Socr., h.\,e. II 42; Soz., h.\,e. IV
24,3).}, der gut reden konnte und gebildet war, die bisch�fliche Aufsicht �ber die galatische
Provinz. Sie schrieben auch an die dortigen Kirchen, da� sie nach dem Buch des Markell
fahnden, es verschwinden lassen und die bekehren sollten, die dasselbe denken, falls
sie welche f�nden.
\kapR{2}Sie teilten mit, da� sie wegen der L�nge der Schrift nicht das
ganze Buch angef�gt h�tten, aber einige Aussagen h�tten sie ihrem Brief eingeschoben zum Beweis, da�
er wirklich solches lehre.\footnoteA{Wie �blich verschickte die Synode einen Rundbrief an
die umliegenden, betroffenen Gemeinden, diesmal mit Ausz�gen aus Markells verurteiltem
Werk, au�erdem sandte die Synode einen Brief an den Kaiser Konstantin (s.\,u. \S 3), worin zur
Begr�ndung der Verurteilung ebenfalls auf das beleidigende Fernbleiben Markells von der
Weihe der Jerusalemer Kirche hingewiesen wurde.} Manche berichten aber, da� diese
Aussagen von Markell wie in einer Diskussion vorgetragen und von denen um
Eusebius\footnoteA{Diese schon bei Athanasius vorkommende ungenaue Beschreibung l��t offen,
welcher Eusebius gemeint ist. Als Kontrahent Markells kommt Eusebius von Caesarea in Frage, als
f�hrende Pers�nlichkeit in Kleinasien dagegen Eusebius von Nikomedien.} gleichsam als seine
Bekenntnisaussagen auch vor dem Kaiser selbst in verleumderischer Absicht ausgegeben wurden, weil sie sehr zornig auf ihn
waren, da er (Markell) weder auf der Synode in Ph�nizien\footnoteA{Gemeint ist die Synode
von Tyrus, die im Herbst 335 zur Absetzung des Athanasius einberufen worden war.} ihren
Beschl�ssen zugestimmt hatte noch in Jerusalem\footnoteA{Socrates (h.\,e. I 36) berichtet,
da� Markells Buch bereits in Jerusalem (siehe voriges Dokument) kritisiert worden sei und
Markell sich der Auf"|lage, seine Schrift zu verbrennen, widersetzt habe, weswegen die
Versammlung in Konstantinopel erst n�tig geworden sei.} (ihren Beschl�ssen) �ber
Arius\footnoteA{Vgl. die Bemerkungen zu Dok. \ref{ch:Jerusalem335}.} und auch nicht an der
Weihe des gro�en Martyrions teilgenommen hatte, um die Gemeinschaft mit
ihnen zu vermeiden.
\pend
\pstart
\kapR{3}In der Tat, als sie �ber ihn an den Kaiser schrieben, brachten sie auch dies als
Anklage vor, da� auch er (der Kaiser) von ihm verh�hnt worden sei, insofern jener es f�r unter
seiner W�rde gehalten hatte, die Weihe der in Jerusalem neu erbauten Kirche zu beehren.
\kapR{4}Den Anla� f�r diese Schrift lieferte Markell aber ein gewisser
Asterius,\footnoteA{Von Asterius sind nur Fragmente �berliefert (\cite{Vinzent:Asterius}),
einerseits von einem ">Syntagmation"< (vor Nicaea), andererseits von seiner Schrift (nach
Nicaea) zur Verteidigung eines Briefes des Eusebius von Nikomedien an Paulinus von Tyrus (Dok. \ref{ch:8} = Urk. 8; Dok. \ref{ch:9} = Urk. 9), in denen er sich als systematischer Denker einer subordinatianischen Theologie
erweist. Athanasius vers�umt es nicht zu erw�hnen, da� Asterius ">geopfert habe"< (decr.
8,1), sicherlich w�hrend der diokletianischen Verfolgung; weitere biographische Details sind 
unbekannt. Eusebius nennt neben
Asterius aber auch Eusebius von Nikomedien, Paulinus von Tyrus, Narcissus von Neronias,
Origenes und sich selbst als Markells Gegner (Marcell. I 4,1--3).} ein Sophist aus
Cappadocia, der auch Reden �ber die Glaubenslehre verfa�t hatte, die den Ansichten des
Arius zustimmen, und die er auf seinem Weg durch die St�dte vor den Bisch�fen und auf Synoden, 
die dort gew�hnlich stattfanden,
herumzeigte. Und weil er ihm widersprach, verfiel Markell, sei es wissentlich oder
unwissentlich, der Lehre des Paulus von Samosata.\footnoteA{\protect\label{fn:Samosata}Der Vorwurf, den im 3. Jh.
verurteilten Irrlehren des Paulus von Samosata zu verfallen, ist ein �bliches Element der
antih�retischen Polemik. Urheber ist in diesem Fall Eusebius von Caesarea, der in seiner
Schrift \textit{De ecclesiastica theologia} Markell vorwirft, er bezeichne wie Paulus von
Samosata den Sohn Gottes als blo�en Menschen und lehre, da� Gott Vater, Sohn und heiliger
Geist ein und dieselbe Person seien (e.\,th. I 14,2; 20,43; III 6,4); vgl. auch die Distanzierung von
Markell und von Paulus in der sog. 3. antiochenischen Formel (Dok. \ref{sec:AntIII}) und die Erw�hnung in Dok.
\ref{ch:Makrostichos},8. Zum historischen Paulus von Samosata vgl. Eus., h.\,e. VII 27--30 und Dok. \ref{ch:Makrostichos},8 Anm.}
Aber er konnte sp�ter auf der Synode von Serdica\footnoteA{Vgl. dazu
Dok. \ref{ch:SerdicaEinl}.} sein Bischofsamt wieder aufnehmen, da er bewies, nicht so zu
denken.
\pend
\endnumbering
\end{translatio}
\end{Rightside}
\Columns
\end{pairs}
\selectlanguage{german}
% \renewcommand*{\goalfraction}{.8}

% DOKUMENT 41 %%%
%% Erstellt von uh
%% �nderungen:
%%%% 13.5.2004 Layout-Korrekturen (avs) %%%%
\chapter{Briefwechsel und Synoden in Rom und Antiochien der Jahre 340 und 341}
\thispagestyle{empty}
\section{Regest eines Briefes des Julius von Rom an die �stlichen Bisch�fe}
% \label{sec:41.1}
\label{sec:BriefJulius}
\begin{praefatio}
  \begin{description}
  \item[Fr�hjahr 340]Hintergrund sind die seit
    Nicaea\index[synoden]{Nicaea!a. 325} anhaltenden Differenzen um
    Personalien und Bischofssitze, womit sich aufgrund gegenseitiger
    H�resievorw�rfe zunehmend auch theologische Fragen verkn�pften,
    insbesondere nach der Aufnahme
    Markells\index[namen]{Markell!Bischof von Ancyra} in die
    Kirchengemeinschaft durch den Westen, der zuvor wegen H�resie im
    Osten verurteilt worden war
    (vgl. Dok. \ref{ch:Konstantinopel336}). Gegen die R�ckkehr des
    Athanasius\index[namen]{Athanasius!Bischof von Alexandrien} aus
    seinem ersten Exil nach Alexandrien\index[namen]{Alexandrien} im
    November 337 (ind.\,ep.\,fest. 10 [a. 338]) protestierten die
    Bisch�fe des Ostens, die ihn in auf der Synode von
    Tyrus\index[synoden]{Tyrus!a. 335} im Jahr 335 abgesetzt
    hatten. F�r sie war Pistus\index[namen]{Pistus!Bischof von
      Alexandrien} der rechtm��ige alexandrinische Bischof (er war von
    Alexander von Alexandrien\index[namen]{Alexander!Bischof von
      Alexandrien} und auf der Synode von
    Nicaea\index[synoden]{Nicaea!a. 325} 325 exkommuniziert und von
    Secundus von Ptolema"is\index[namen]{Secundus!Bischof von
      Ptolema"is} ordiniert worden [vgl. Dok. \ref{sec:4a} = Urk. 4a;
    Dok. \ref{ch:6} = Urk. 6; Dok. \ref{sec:BriefJuliusII},16;
    ep.\,encycl. 6,2; Epiph., haer. 69,8,5]). Sie sandten Schreiben
    sowohl an Julius von Rom\index[namen]{Julius!Bischof von Rom} als
    auch an die Kaiser zusammen mit den Akten der Synode von
    Tyrus\index[synoden]{Tyrus!a. 335} (h.\,Ar. 9,1; apol.\,sec. 19,3;
    20,1; 83,4; Dok. \ref{sec:BriefJuliusII},8~f.; Socr., h.\,e. II
    17,4). In Reaktion darauf berief
    Athanasius\index[namen]{Athanasius!Bischof von Alexandrien} eine
    Synode ein (338), und die ca. 80 in
    Alexandrien\index[synoden]{Alexandrien!a. 338} versammelten
    �gyptischen Bisch�fe verfa�ten einen apologetischen Brief
    (apol.\,sec. 3--19) und verschickten ihn ebenfalls. In
    Rom\index[namen]{Rom} trafen nun die alexandrinischen Gesandten
    auf die aus Antiochien\index[namen]{Antiochien} (Socr., h.\,e. II
    17,5~f.; Ath., apol.\,sec. 20,1;
    Dok. \ref{sec:BriefJuliusII},8~f.; ep.\,encycl. 7,2).  Angesichts
    der tiefgreifenden theologischen Differenzen zwischen den Kirchen
    des Ostens und denen des Westens, die in den von den westlichen
    Kirchen nicht rezipierten Exkommunikationen von Athanasius und
    Markell ihren Niederschlag fanden, entstand die Notwendigkeit
    einer synodalen Kl�rung der Lage.  Auf einen entsprechenden Brief
    des Julius\index[namen]{Julius!Bischof von Rom} reagierten die
    Antiochener jedoch mit einer eigenen
    Synode\index[synoden]{Antiochien!a. 339}, auf der
    Gregor\index[namen]{Gregor!Bischof von Alexandrien} anstelle des
    umstrittenen Pistus\index[namen]{Pistus!Bischof von Alexandrien}
    als Bischof f�r Alexandrien bestimmt wurde
    (Dok. \ref{sec:BriefJuliusII},41--45;
    Dok. \ref{sec:SerdicaRundbrief},15 mit Anm.), der dann kurz vor
    Ostern 339 (ind.\,ep.\,fest. 11) mit milit�rischer Hilfe in
    Alexandrien\index[namen]{Alexandrien} einziehen konnte.
    Athanasius\index[namen]{Athanasius!Bischof von Alexandrien}
    protestierte gegen diese Ereignisse in seiner ep.\,encycl. und
    wandte sich nun nach Rom\index[namen]{Rom}, einerseits wegen der
    traditionell guten Beziehungen zwischen
    Alexandrien\index[namen]{Alexandrien} und Rom\index[namen]{Rom},
    andererseits auch, um dem Herrschaftsbereich des
    Constantius\index[namen]{Constantius, Kaiser} zu entfliehen
    (apol.\,Const. 4,1--4; das genaue Datum ist unbekannt, da es
    schwierig zu beurteilen ist, ob
    Athanasius\index[namen]{Athanasius!Bischof von Alexandrien}
    tats�chlich direkt, wie er es in apol.\,Const. beschreibt, nach
    Rom\index[namen]{Rom} reiste oder auf Umwegen, wie es seine Gegner
    behaupteten [Dok. \ref{sec:RundbriefSerdikaOst},11]). In
    Rom\index[namen]{Rom} fanden sich weitere aus dem Osten exilierte
    Bisch�fe ein wie Markell von Ancyra\index[namen]{Markell!Bischof
      von Ancyra} (vgl. Dok. \ref{sec:MarkellJulius}), Paulus von
    Konstantinopel\index[namen]{Paulus!Bischof von Konstantinopel} und
    Asclepas von Gaza\index[namen]{Asclepas!Bischof von Gaza}
    (vgl. Dok. \ref{sec:BriefJuliusII},52; Dok.
    \ref{sec:RundbriefSerdikaOst},10.15.21.28; Ath.,
    apol.\,Const. 4,1; Socr., h.\,e. II 15,1~f.; Soz., h.\,e. III
    8,1). Daraufhin lud der r�mische Bischof
    Julius\index[namen]{Julius!Bischof von Rom} mit einem Schreiben,
    von dem nur das vorliegende Regest erhalten ist, erneut zu einer
    Synode nach Rom\index[synoden]{Rom!a. 339} ein und setzte zugleich
    schon einen Termin fest, zu dem eine Delegation aus dem Osten
    erscheinen sollte.
  \item[�berlieferung]Der Brief des
    Julius\index[namen]{Julius!Bischof von Rom} ist nicht
    �berliefert. Neben dem Regest bei Soz., h.\,e. III
    8,3~f. (vgl. Socr., h.\,e. II 15,3) gibt es nur eine Erw�hnung
    durch Athanasius in h.\,Ar. 11,1 (vgl. apol.\,sec. 20,1).
  \item[Fundstelle] \refpassage{soz1}{soz2} Soz., h.\,e. III
    8,3~f. (\editioncite[110,27--111,4]{Hansen:Soz}); \refpassage{ath1}{ath2} Ath.,
    h.\,Ar. 11,1 (\editioncite[188,30--32]{Opitz1935})
  \end{description}
\end{praefatio}
\autor{Das Regest bei Sozomenus}
\begin{pairs}
\selectlanguage{polutonikogreek}
\begin{Leftside}
\beginnumbering
\pstart
\hskip -1.25em\edtext{\abb{}}{\xxref{soz1}{soz2}\Cfootnote{\latintext Soz.
(BC=b)}}\specialindex{quellen}{section}{Sozomenus!h.\,e.!III 8,3~f.}
\kap{1}ka`i\edlabel{soz1} to~ic >an`a t`hn <'ew >episk'opoic >'egraye memf'omenoc <wc
o>uk >orj~wc bouleusam'enoic per`i to`uc >'andrac ka`i t`ac >ekklhs'iac tar'attousi t~w|
m`h >emm'enein to~ic >en Nika'ia|\edindex[synoden]{Nicaea!a. 325} d'oxasin. >ol'igouc
d`e >ek p'antwn e>ic <rht`hn
<hm'eran pare~inai >ek'eleuse diel'egxontac dika'ian >ep'' a>uto~ic >enhnoq'enai t`hn
y~hfon; >`h to~u loipo~u o>uk >an'exesjai >hpe'ilhsen, e>i m`h pa'usointo
newter'izontec.
\pend
\pstart
Ka`i <o m`en t'ade >'egrayen.\edlabel{soz2}
\pend
\end{Leftside}
\begin{Rightside}
\begin{translatio}
\beginnumbering
\pstart
% \autor{Sozomenus}
\noindent Und er (Julius) schrieb an die Bisch�fe aus dem Osten und machte ihnen Vorw�rfe, da� sie nicht
korrekt mit den M�nnern verfahren w�ren und die Kirchen dadurch verwirren w�rden, da� sie sich
nicht an die Beschl�sse von Nicaea hielten. Also ordnete er an, da� einige stellvertretend
f�r alle an dem festgesetzten Tag erscheinen sollten, um klarzustellen, ob die Entscheidung
bei ihnen zu Recht gefallen sei; andernfalls drohte er, sie nicht l�nger ertragen zu
k�nnen, wenn sie nicht mit den Neuerungen aufh�rten.
\pend
\pstart
Derart fiel also sein Brief
aus\footnoteA{Das Antwortschreiben der Antiochener nach Rom best�tigt (vgl. Dok.
\ref{sec:BriefSynode341}), da� Julius in anscheinend barschem Tonfall nicht nur zu einer
Synode eingeladen, sondern den Antiochenern auch ">Arianismus"< vorgeworfen hatte. Zur
Frage nach der Berechtigung des Julius, zu einer Synode einzuladen, um die bereits
gefallenen Synodalurteile erneut aufzurollen, vgl. Dok. \ref{sec:BriefSynode341} und
\ref{sec:BriefJuliusII}.}.
\pend
\endnumbering
\end{translatio}
\end{Rightside}
\Columns
\end{pairs}
\autor{Die Erw�hnung bei Athanasius}
\begin{pairs}
\selectlanguage{polutonikogreek}
\begin{Leftside}
\pstart
\hskip -1.25em\edtext{\abb{}}{\xxref{ath1}{ath2}\Cfootnote{\latintext Ath. (BKPO
R)}}\specialindex{quellen}{section}{Athanasius!h.\,Ar.!11,1} %%
\kap{2}<o\edlabel{ath1} d`e >Io'ulioc\edindex[namen]{Julius!Bischof von
Rom} gr'afei ka`i p'empei presv\-bu\-t'e\-rouc,
>Elp'idion\edindex[namen]{Elpidius!Presbyter in Rom} ka`i
Fil'oxenon\edindex[namen]{Philoxenus!Presbyter in Rom}, <or'isac ka`i
\edtext{projesm'ian}{\Dfootnote{projum'ian
\latintext B}}, <'ina >`h >'eljwsin >`h gin'wskoien <eauto`uc <up'optouc e>~inai kat`a
p'anta.\edlabel{ath2}%% Ath., h.Ar. 11,1%%%
\pend
% \endnumbering
\end{Leftside}
\begin{Rightside}
\begin{translatio}
\beginnumbering
\pstart
% \autor{Athanasius}
\noindent Und Julius schrieb einen Brief\footnoteA{Sowohl an dieser Stelle als auch in apol.sec.
20,1 erw�hnt Athanasius diesen Brief des Julius, um seine Flucht nach Rom 339/340 als Reise
zur Synode darzustellen.} und schickte Presbyter, Elpidius und Philoxenus, und setzte dabei auch eine Frist, damit sie entweder k�men oder erkannten, da� sie selbst in jeder Hinsicht
verd�chtig sind.
\pend
\endnumbering
\end{translatio}
\end{Rightside}
\Columns
\end{pairs}
% \thispagestyle{empty}
%%% Local Variables: 
%%% mode: latex
%%% TeX-master: "dokumente_master"
%%% End: 

%% Erstellt von uh
%% �nderungen:
%%%% 13.5.2004 Layout-Korrekturen (avs) %%%%
% \renewcommand*{\goalfraction}{.8}
% \cleartooddpage
\section{Berichte �ber die Kirchweihsynode von Antiochien des Jahres 341}
% \label{sec:41.2}
\label{sec:BerichteAntiochien341}
\begin{praefatio}
  \begin{description}
  \item[Anfang 341]Der Zeitpunkt der antiochenischen Synode und deren
    Beziehung zur r�mischen Synode ist nicht einfach zu bestimmen;
    weder ist die absolute Datierung beider Synoden eindeutig noch
    deren relative Beziehung zueinander. �berliefert ist, da� die
    r�mischen Gesandten im Januar zur�ckgeschickt wurden
    (Dok. \ref{sec:BriefJuliusII},22) und da� die antiochenische
    Synode zwischen 1.9.340 und 31.8.341 stattfand (Socr.,
    h.\,e. II 8,2 [s.\,u.]; Ath., syn. 25,1); dazu pa�t es, wenn
    Eltester den 6.1. als Datum f�r die Kirchweihe in
    Antiochien\index[namen]{Antiochien} wahrscheinlich macht
    (\cite[254~f.]{Eltester:Antiochia}), und wenn es stimmt, da�
    Athanasius\index[namen]{Athanasius!Bischof von Alexandrien} ein
    Jahr und sechs Monate auf die Delegation aus dem Osten gewartet
    hat (vgl. Dok. \ref{sec:BriefJuliusII},40). Wurde fr�her die
    r�mische Synode vor der antiochenischen angesetzt
    (\cite{Schwartz:GesIII}), so wird nun das Verh�ltnis seit den
    Arbeiten von \cite{Schneemelcher:Kirchweih}, \cite{Tetz:Kirchweih}
    und \cite[7~f.]{Brennecke:Hilarius} umgekehrt. Eindeutig ist, da�
    der Brief des Julius\index[namen]{Julius!Bischof von Rom}
    (Dok. \ref{sec:BriefJuliusII}) auf eine Absage aus
    Antiochien\index[namen]{Antiochien}
    (Dok. \ref{sec:BriefSynode341}) antwortet. Da diese Absage von
    einer Synode stammt und nicht das Werk eines einzelnen ist
    (vgl. die Adressaten der Antwort des Julius in
    Dok. \ref{sec:BriefJuliusII}), mu� man entweder zwei Synoden in
    Antiochien annehmen (antiochenische Synode mit der Absage an
    Julius~-- r�mische Synode~-- antiochenische Synode mit den
    Glaubenserkl�rungen; von einer permanenten Synode in Antiochien
    ist nicht auszugehen) oder besser die r�mische der antiochenischen
    folgen lassen. Eine gewisse zeitliche �berschneidung ist ebenfalls
    anzunehmen, da die r�mische Synode schon vor dem Eintreffen der
    Absage begonnen hatte (s. Dok. \ref{sec:MarkellJulius} und
    \ref{sec:BriefJuliusII}). Auf der Synode von
    Antiochien\index[synoden]{Antiochien!a. 341} war ebenfalls Kaiser
    Constantius\index[namen]{Constantius, Kaiser} anwesend (Soz.,
    h.\,e. III 9,1), au�erdem 90 (so Ath., syn. 25,1; Socr., h.\,e. II
    8,3 [siehe Text]) oder 97 (so Hil., syn. 28 [502A]; Soz.,
    h.\,e. III 5,2~f. [siehe Text 2,2]) Bisch�fe.
  \item[�berlieferung]Von den Ereignissen auf der
    Kirchweihsynode in Antiochien\index[synoden]{Antiochien!a. 341}
    sind keine Akten �berliefert, sondern nur Beschreibungen bei den
    Kirchenhistorikern. Hier werden zun�chst allgemeine Berichte �ber
    die Synode vorgestellt (Socr., h.\,e. II 7,3--8,7;
    10,19~f.; Soz., h.\,e. III 5,1--4;
    5,10--6,1.7~f.),
    wobei zu beachten ist, da� sowohl Socrates als auch, ihm folgend,
    Sozomenus die Ereignisse dieser Synode mit derjenigen in
    Antiochien\index[synoden]{Antiochien!a. 338} aus dem Jahr 338
    verbinden, auf der gegen die R�ckkehr des
    Athanasius\index[namen]{Athanasius!Bischof von Alexandrien} aus
    seinem ersten Exil protestiert wurde
    (s. Dok. \ref{sec:BriefJulius} Einleitung). Ferner sind mit dieser
    Synode der �berlieferung nach theologische Erkl�rungen verbunden (Dok. \ref{sec:AntIII},
    \ref{sec:AntII} und \ref{sec:AntI}), au�erdem gibt es Berichte
    �ber einen Antwortbrief der Antiochener nach Rom\index[namen]{Rom}
    (Dok. \ref{sec:BriefSynode341}). Grundlage dieser Berichte der
    Kirchenhistoriker sind einerseits die Schriften des Athanasius,
    insbesondere seine Schrift \textit{De synodis}, dar�berhinaus
    nat�rlich auch die Synodalaktensammlung des Sabinus von Heraclea,
    den Socrates (h.\,e. II 15; 17) gerade in diesem Zusammenhang mit
    kritischen Bemerkungen �ber seine anti-athanasianische und
    anti-nicaenische Tendenz erw�hnt. Ferner d�rften Dokumente dieser
    Synode auch zu den Unterlagen der sog. Doppelsynode von
    Rimini/Seleucia\index[synoden]{Rimini!a. 359}\index[synoden]{Seleucia!a. 359}
    geh�rt haben, da dort wieder auf die in
    Antiochien\index[synoden]{Antiochien!a. 341} formulierte
    Glaubenserkl�rung zur�ckgegriffen wurde (Soz., h.\,e. IV
    22,6).
  \item[Fundstelle]\refpassage{sok3}{sok4} Socr., h.\,e. II 7,3--8,7;
    10,19~f. (\editioncite[97,5--98,7; 101,23--26]{Hansen:Socr}); \refpassage{soz3}{soz4}
    Soz., h.\,e. III 5,1--4; 5,10--6,1.7~f. (\editioncite[105,15--106,8; 107,4--14;
    108,14--18]{Hansen:Soz})
  \end{description}
\end{praefatio}
\clearpage
\autor{Der Bericht des Socrates}
\begin{pairs}
\selectlanguage{polutonikogreek}
\begin{Leftside}
% \beginnumbering
\pstart
\hskip -1.25em\edtext{\abb{}}{\xxref{sok3}{sok4}\Cfootnote{\latintext Socr. (MF=b AT
Arm.)}}\specialindex{quellen}{section}{Socrates!h.\,e.!II 7,3--8,7;10,19~f.} %%
\kap{1,1}ka`i\edlabel{sok3} <o m`en basile`uc ta~uta pr'axac
\Ladd{\edtext{\abb{e>u\-j'e\-wc}}{\Dfootnote{\latintext Arm.}}} >ep`i t`hn
>Anti'oqeian\edindex[namen]{Antiochien}
<'wrmhsen,
\kap{2}E>us'ebioc\edindex[namen]{Eusebius!Bischof von Nikomedien} d`e
o>uden`i tr'opw| <hsuq'azein >hbo'uleto, >all`a (t`o
to~u l'ogou) p'anta l'ijon >ek'inei, <'opwc >`an <`o pro'ejeto
\edtext{\abb{katerg'ashtai}}{\Cfootnote{\dt{des. T}}}.
\kap{3}Kataskeu'azei o>~un s'unodon >en
>Antioqe'ia|\edindex[synoden]{Antiochien!a. 341} t~hc Sur'iac gen'esjai, prof'asei m`en
t~wn >egkain'iwn t~hc >ekklhs'iac,
<`hn <o pat`hr m`en t~wn A>ugo'ustwn kataskeu'azein >'hrxato,
\edtext{\abb{met`a}}{\Dfootnote{+ t`hn \latintext A}} teleut`hn d`e a>uto~u <o u<i`oc
Kwnst'antioc\edindex[namen]{Constantius, Kaiser} dek'atw| >'etei >ap`o t~hc
jemeli'wsewc sunet'elesen, t`o d`e >alhj`ec >ep`i
t~h| >anatrop~h| ka`i kajair'esei t~hc <omoous'iou \edtext{\abb{p'istewc}}{\Dfootnote{+
per`i Max'imou <Ierosol'umwn \latintext A}}.
\pend
\pstart
\kap{4}>en ta'uth| d`e t~h| sun'odw| sun~hljon >ek diaf'orwn p'olewn >ep'iskopoi
>enen'hkonta. \edindex[namen]{Maximus!Bischof von
Jerusalem}\edtext{\abb{M'aximoc}}{\Cfootnote{\dt{it. inc. T}}}
\edtext{m'entoi}{\Dfootnote{d`e \latintext T\textsuperscript{*}}} <o
\edtext{\abb{t~wn}}{\Dfootnote{\latintext > T}} <Ierosol'umwn >ep'iskopoc, <`oc
Mak'arion\edindex[namen]{Macarius!Bischof von Jerusalem}
died'exato, o>u pareg'eneto >en a>ut~h|, \edtext{>epilogis'amenoc}{\Dfootnote{logis'amenoc
\latintext T}} <wc e>'ih sunarpage`ic ka`i t~h| kajair'esei <upogr'ayac
>Ajanas'iou\edindex[namen]{Athanasius!Bischof von Alexandrien}.
\kap{5}>all`a m`hn o>ud`e >Io'ulioc\edindex[namen]{Julius!Bischof von Rom}
par~hn <o t~hc meg'isthc <R'wmhc
>ep'iskopoc, o>ud`e m`hn \edtext{\abb{e>ic}}{\Dfootnote{+ t`on \latintext bA}} t'opon
a>uto~u >apest'alkei tin'a, ka'itoi kan'onoc >ekklhsiastiko~u kele'uontoc m`h de~in
\edtext{\abb{par`a}}{\Dfootnote{+ t`hn \latintext bA}} gn'wmhn to~u >episk'opou
\edtext{\abb{t~hc}}{\Dfootnote{\latintext > bA}} <R'wmhc t`ac >ekklhs'iac
\edtext{kanon'izesjai}{\Dfootnote{kanon'izein \latintext bA >Arm.}}. \pend
\pstart
\kap{6}sugkrote~itai \edtext{\abb{o~>un}}{\Dfootnote{\latintext > T}} a<'uth <h s'unodoc
>en \edtext{\abb{t~h|}}{\Dfootnote{\latintext > A}}
>Antioqe'ia|\edindex[synoden]{Antiochien!a. 341}
\edtext{par'ontoc}{\Dfootnote{parous'ia \latintext bA \greektext + <wc >'efhn \latintext
T}} \edtext{\abb{Kwnstant'iou}}{\Dfootnote{\latintext > Arm.}}\edindex[namen]{Constantius,
Kaiser} \edtext{\abb{to~u
basil'ewc}}{\Dfootnote{\latintext > T}} >en <upate'ia|
\edtext{Markell'inou}{\Dfootnote{Markelliano~u \latintext Arm. \griech{Mark'ellou}
\latintext b}}\edindex[namen]{Marcellinus!Konsul} ka`i
\edtext{Prob'inou}{\Dfootnote{probiano~u
\latintext T}}\edindex[namen]{Probinus!Konsul}; >~hn d`e
p'empton >'etoc to~uto \edtext{\abb{>ap`o}}{\Dfootnote{\latintext > T}} t~hc teleut~hc
to~u t~wn \edtext{\abb{A>ugo'ustwn}}{\Cfootnote{\dt{it. des. T}}} patr`oc
Kwnstant'inou\edindex[namen]{Konstantin, Kaiser};
proeist'hkei d`e t'ote t~hc >en >Antioqe'ia|
>ekklhs'iac
\edtext{Fl'akilloc}{\Dfootnote{Pl'akitoc \latintext FA}}\edindex[namen]{Flacillus!Bischof
von Antiochien}\edlabel{Flacillus} diadex'amenoc E>ufr'onion\edindex[namen]{Euphronius!Bischof von
Antiochien}.
\kap{7}o<i per`i E>us'ebion\edindex[namen]{Eusebianer} o>~un >'ergon
\edtext{t'ijentai}{\Dfootnote{tij'entec
M\textsuperscript{1}}} \edtext{\abb{prohgoum'enwc}}{\Dfootnote{\latintext > Arm.}}
\edtext{>Ajan'asion diab'allein}{\lemma{\abb{}}\Dfootnote{\responsio\ diab'allein
>ajan'asion \latintext A Arm.}}\edindex[namen]{Athanasius!Bischof von
Alexandrien}, pr~wton m`en <wc par`a kan'ona
\edtext{\abb{pr'axanta}}{\Cfootnote{\dt{it. inc. T}}}, <`on a>uto`i <'wrisan
\edtext{\abb{t'ote}}{\Dfootnote{+ ka`i \latintext Arm.}}, <'oti m`h gn'wmh| koino~u
sunedr'iou t~wn >episk'opwn t`hn t'axin t~hc <ierws'unhc >an'elaben,
\edtext{>all''}{\Dfootnote{>epanelj`wn g`ar >ap`o t~hc >exor'iac \latintext FA}} <eaut~w|
>epitr'eyac e>ic t`hn >ekklhs'ian e>isep'hdhse;
\kap{8}\edtext{>'epeita d`e <wc
>ep`i}{\Dfootnote{ka`i <'oti >en \latintext F}} t~h| e>is'odw| a>uto~u
\edtext{\abb{taraq~hc}}{\Cfootnote{\dt{it des. T}}} genom'enhc pollo`i >en t~h| st'asei
>ap'ejanon, ka`i <'wc tinwn a>ikisj'entwn <up`o
>Ajanas'iou\edindex[namen]{Athanasius!Bischof von Alexandrien}, tin~wn d`e dikasthr'ioic
\edtext{\abb{paradoj'entwn}}{\Dfootnote{+ o>u m`hn >all`a ka`i t`a >en t'urw pepragm'ena
kat`a >ajanas'iou (>ajan'asion \latintext A) \griech{e>ic m'eson ~>hgon} FA}}.
\pend
\pstart
\noindent\dots
\pend
\pstart
\kap{9}Toia~utai m`en a<i t~wn >en >Antioqe'ia|\edindex[synoden]{Antiochien!a. 341}
t'ote sunelj'ontwn per`i t~hc p'istewc
>ekj'eseic >eg'enonto, a<~ic ka`i Grhg'orioc\edindex[namen]{Gregor!Bischof von
Alexandrien} m'hpw t~hc >Alexandre'iac\edindex[namen]{Alexandrien} >epib`ac <wc
>ep'iskopoc a>ut~hc kajup'egrayen.
\kap{10}ka`i <h m`en >eke~i t'ote genom'enh s'unodoc
ta~uta pr'axasa ka`i >'alla tin`a nomojet'hsasa diel'ujh.\edlabel{sok4}
\pend
\end{Leftside}
\begin{Rightside}
\begin{translatio}
\beginnumbering
\pstart
% \autor{Sokrates}
\noindent\kapR{7,3}Und als der Kaiser dies\footnoteA{Zuvor wird berichtet, da� Constantius Paulus
als Bischof von Konstantinopel ab- und Eusebius aus Nikomedien an seiner Stelle einsetzte.}
veranla�t hatte, brach er sogleich nach Antiochien auf; \kapR{8,1} Eusebius aber wollte auf
keinen Fall Ruhe halten, sondern lie�, wie es hei�t, keinen Stein auf dem anderen, damit er
erreichte, was er wollte. \kapR{2} Er veranstaltete also im syrischen Antiochien eine
Synode unter dem Vorwand der Weihe der Kirche\footnoteA{H�ufig wurde zu einer Synode
anl��lich einer Kirchweihe geladen, wie z.\,B. in Jerusalem 335 (s. Dok.
\ref{ch:Jerusalem335}). Zum Datum dieser Kirchweihe vgl. \cite{Eltester:Antiochia}.},
die der Vater der Kaiser zu errichten begonnen hatte und die nach seinem Tod schlie�lich im zehnten Jahr nach
der Grundsteinlegung sein Sohn Constantius fertigstellte, in Wahrheit aber, um den
Glauben an das ">wesenseins"< zu zerst�ren und zu vernichten.
\pend
\pstart
\kapR{3}Zu dieser Synode kamen aus verschiedenen St�dten 90 Bisch�fe zusammen.
Maximus\footnoteA{\protect\label{fn:Maximus}Soz., h.\,e. II 20 berichtet ausf�hrlicher
�ber Maximus als Confessor und Nachfolger des Macarius. Als  Athanasius aus
seinem Exil 346 zur�ckkehrte und durch Jerusalem reiste, hat Maximus ihn mit einer Synode und Begleitbrief
unterst�tzt (Socr., h.\,e. II 24; d.\,h. er l��t die Beschl�sse von der ">westlichen"< Synode von Serdica, Dok.
\ref{ch:SerdicaEinl}, best�tigen), wurde sp�ter aber selbst vom Bischofssitz vertrieben
(Socr., h.\,e. II 38; Nachfolger ist Cyrill). Der hier erw�hnte Widerstand des Maximus gegen
eine Verurteilung des Athanasius bezieht sich auf die Ereignisse in Tyrus (vgl. Soz., h.\,e.
II 25,20).} jedoch, der Bischof von Jerusalem, der Macarius nachfolgte, erschien nicht auf
ihr und erkl�rte, da� er unter Zwang die Verurteilung des Athanasius unterschrieben h�tte.
\kapR{4}Aber auch Julius, der Bischof des gr��ten Rom, war nicht anwesend, auch hatte
er niemanden an seiner Statt geschickt, obwohl der kirchliche Kanon vorsieht, da� die Kirchen keine
Beschl�sse gegen die Ansicht des Bischofs von Rom fassen sollen.\footnoteA{Es gibt keine
Belege f�r ein derartiges kirchliches Gesetz in jener Zeit (vgl. aber Canon 3 der Synode von 
Serdica). Sozomenus l��t die
entsprechende Bemerkung aus. Auch Julius von Rom selbst erw�hnt Entsprechendes nicht,
weist allein darauf hin, da� es kirchliche Tradition sei, bei Entscheidungen �ber
Alexandrien die r�mische Kirche miteinzubeziehen (s. Dok. \ref{sec:BriefJuliusII},63.)
und da� es deswegen besser gewesen w�re, ein gesamtkirchliches Urteil �ber Athanasius und
Markell zusammen mit der westlichen Kirche anzustreben.}
\pend
\pstart
\kapR{5}Es versammelte sich also unter Anwesenheit des Kaisers Constantius diese Synode
in Antiochien zur Zeit des Konsulats von Marcellinus und Probinus\footnoteA{Daraus (auch
Ath., syn. 25,1) ergibt sich eine zeitliche Eingrenzung von September 340 bis August
341.}; es war aber das f�nfte Jahr nach dem Tod Konstantins, des Vaters der
Augusti. Damals aber stand Flacillus\footnoteA{Der Name ist bei den verschiedenen Autoren in unterschiedlicher Form (neben der Form Flacillus finden sich auch noch die Varianten Placitus und Flacitus) �berliefert, ohne da� eine Entscheidung gef�llt werden k�nnte. Vgl. zu den Vorg�ngern Dok.
\ref{sec:Eustathius}; Nachfolger ist Stephanus (Socr., h.\,e. II 26; Soz., h.\,e. III 20), der erstmals als Bischof von Antiochien als Teilnehmer der Synode von Serdica bezeugt ist, vgl. Dok. \ref{sec:NominaepiscSerdikaOst}, Nr. 1.}, der Nachfolger des Euphronius,
der Kirche von Antiochien vor. \kapR{6}Die Eusebianer besch�ftigten sich\footnoteA{Das hier in
1,6~f. Berichtete geh�rt tats�chlich zu einer fr�heren Synode in Antiochien im Jahr
338.} nun vornehmlich damit, Athanasius zu verwerfen, da er erstens gegen den Kanon
gehandelt h�tte, den sie damals selbst aufgestellt hatten, da er nicht auf Beschlu� einer
gemeinsamen Synode der Bisch�fe die heiligen Amtsgesch�fte wieder aufgenommen habe,
sondern nach eigenem Gutd�nken zur Kirche zur�ckgekehrt sei; \kapR{7}dann aber auch, weil bei
seinem Einzug Unruhe entstanden und viele bei dem Aufruhr umgekommen seien, und
einige von Athanasius mi�handelt, andere vor Gericht gezogen worden seien.
\pend
\pstart
\noindent\dots
\pend
\pstart
\kapR{10,19}Derart sind also die Erkl�rungen �ber den Glauben von denen, die damals in
Antiochien zusammengekommen waren, ausgefallen,\footnoteA{Socr., h.\,e. II 9,1--10,18 
(der ausgelassene Abschnitt; vgl. Soz., h.\,e. III 6,6) enth�lt, da Socrates die 
antiochenischen Synoden verwechselt, einen l�ngeren Bericht zun�chst �ber Eusebius von Emesa, 
da er als Bischof f�r Alexandrien anstelle Athanasius vorgesehen war, und anschlie�end �ber 
die Wahl Gregors, da Eusebius die Wahl ablehnte; es folgen dann die zwei antiochenischen Formeln (vgl. Dok.
\ref{sec:AntII}; \ref{sec:AntI}).} die auch Gregor unterschrieben hatte, als er noch nicht 
gleichsam als dortiger Bischof nach Alexandrien gegangen war. \kapR{20}Und als die
damals dort tagende Synode dies getan und auch einige andere Gesetze beschlossen hatte,\footnoteA{Traditionell wurden dieser Synode die in der kanonistischen �berlieferung bezeugten antiochenischen Kanones (\cite[100--126]{Joannou:Fonti}) zugewiesen.} wurde sie aufgel�st.
\pend
\end{translatio}
\end{Rightside}
\Columns
\end{pairs}
\autor{Der Bericht des Sozomenus}
\begin{pairs}
\selectlanguage{polutonikogreek}
\begin{Leftside}
\pstart
\hskip -1.25em\edtext{\abb{}}{\xxref{soz3}{soz4}\Cfootnote{\latintext Soz. (BC=b T (bis
6,1))}}\specialindex{quellen}{section}{Sozomenus!h.\,e.!III 5,1--4; 5,10--6,1.7~f.}
\kap{2,1}Ka`i\edlabel{soz3} <o m`en t'ade pr'axac \edtext{e>ic
>Anti'oqeian t~hc Sur'iac
~<hken}{\Dfootnote{e>ic t`hn >Anti'oqeian t~hc Sur'iac <orm'hsantoc \latintext
T*}}\edindex[namen]{Antiochien}. >'hdh
d`e >exergasje'ishc t~hc >enj'ade >ekklhs'iac, <`hn meg'ejei ka`i k'allei <uperfu~a >'eti
peri`wn Kwnstant~inoc\edindex[namen]{Konstantin, Kaiser} <upourg~w| qrhs'amenoc
Kwnstant'iw|\edindex[namen]{Constantius, Kaiser} t~w| paid`i o>ikodome~in
>'hrxato, e>ic kair`on >'edoxe to~ic >amf`i t`on E>us'ebion\edindex[namen]{Eusebianer}
\edtext{p'alai}{\Dfootnote{p'alin \latintext B}} to~uto spoud'azousi s'unodon gen'esjai.
\kap{2}o<`i d`h t'ote ka`i <'eteroi t~wn t`a a>ut`a frono'untwn a>uto~ic e>ic
>enen'hkonta ka`i <ept`a \edtext{telo~untec >episk'opouc}{\Dfootnote{telo'untwn \latintext
T}} pollaq'ojen e>ic >Anti'oqeian\edindex[namen]{Antiochien} sun~hljon,
\edtext{prof'asei}{\Dfootnote{pr'ofasin \latintext b}} \edtext{\abb{m`en}}{\Dfootnote{+
<wc \latintext b}} >ep`i >afier'wsei t~hc \edtext{neourgo~u}{\Dfootnote{neourgo~uc
\latintext C}} >ekklhs'iac, <wc d`e t`o >apob`an >'edeixen, >ep`i metapoi'hsei t~wn >en
Nika'ia|\edindex[synoden]{Nicaea!a. 325} dox'antwn. <hge~ito d`e
\edtext{\abb{thnika~uta}}{\Dfootnote{\latintext > T}} t~hc >Antioq'ewn >ekklhs'iac
\edtext{Pl'akhtoc}{\Dfootnote{pl'akitoc \latintext T; cf.
Socr.}}\edindex[namen]{Flacillus!Bischof von Antiochien}\edindex[namen]{Placitus!Bischof von Antiochien|see{Flacillus!Bischof von Antiochien}} met`a
E>ufr'onion\edindex[namen]{Euphronius!Bischof von Antiochien}, p'empton d`e >'etoc
>hn'ueto >ap`o t~hc Kwnstant'inou\edindex[namen]{Konstantin, Kaiser} \edtext{\abb{to~u
meg'alou}}{\Dfootnote{\latintext > T}} teleut~hc.
\kap{3}>epe`i d`e p'antec o<i >ep'iskopoi \edtext{sun~hljon}{\Dfootnote{par~hsan
\latintext T*}}, par~hn d`e ka`i <o basile`uc Kwnst'antioc\edindex[namen]{Constantius,
Kaiser}, \edtext{\abb{>hgan'aktoun}}{\Dfootnote{+ d`e \latintext T*}} o<i ple'iouc ka`i
dein~wc >Ajan'asion\edindex[namen]{Athanasius!Bischof von Alexandrien}
\edtext{>eph|ti~wnto <wc <ieratik`on <uperid'onta jesm'on, <`on}{\lemma{>eph|ti~wnto
\dots\ <`on}\Dfootnote{>epith~wnto <wc <ieratik~wn <uperid'onta jesm~wn <`on \latintext
T}} a>uto`i >'ejento, ka`i \edtext{pr`in >epitrap~hnai par`a sun'odou t`hn >Alexandr'ewn
>ekklhs'ian >apolab'onta}{\lemma{\abb{pr`in \dots\ >apolab'onta}}\Dfootnote{\latintext >
T}}\edindex[namen]{Alexandrien}. >ek to'utou d`e ka`i jan'atou polit~wn
a>'ition a>ut`on >'elegon, <wc <hn'ika e>ic t`hn p'olin e>is'h|ei st'asewc kinhje'ishc
ka`i poll~wn m`en >anairej'entwn, t~wn d`e dikasthr'ioic paradoj'entwn.
\kap{4}meg'isthc \edtext{\abb{te}}{\Dfootnote{\latintext > T*}} diabol~hc <up`o toio'utwn
l'ogwn kat`a >Ajanas'iou\edindex[namen]{Athanasius!Bischof von Alexandrien} <ufanje'ishc
>eyhf'isanto Grhg'orion\edindex[namen]{Gregor!Bischof von Alexandrien} t~hc >Alexandr'ewn
>ekklhs'iac prostate~in. 
\pend
\pstart
\noindent\dots
\pend
\pstart
\kap{5}met'esqon d`e ta'uthc t~hc sun'odou o>u m'onon
E>us'ebioc\edindex[namen]{Eusebius!Bischof von Nikomedien}, <`oc met`a
Pa~ulon\edindex[namen]{Paulus!Bischof von Konstantinopel}
>ekbeblhm'enon >ek Nikomhde'iac metast`ac t`on
Kwnstantinoup'olewc\edindex[namen]{Konstantinopel} \edtext{e~>iqe
jr'onon}{\lemma{\abb{}}\Dfootnote{\responsio\ jr'onon e~>iqe \latintext T}},
\edtext{>all`a ka`i >Ak'akioc <o}{\Dfootnote{>Ak'aki'oc te \latintext
T}}\edindex[namen]{Acacius!Bischof von Caesarea}
E>useb'iou\edindex[namen]{Eusebius!Bischof von Caesarea} to~u
Pamf'ilou di'adoqoc ka`i Patr'ofiloc <o Skujop'olewc\edindex[namen]{Patrophilus!Bischof
von Scythopolis} ka`i Je'odwroc <o <Hrakle'iac\edindex[namen]{Theodorus!Bischof
von Heraclea} t~hc
\edtext{\abb{pr`in}}{\Dfootnote{\latintext > T}} Per'injou >onomazom'enhc,
E>ud'oxi'oc\edindex[namen]{Eudoxius!Bischof von Germanicia} te
\edtext{\abb{<o}}{\Dfootnote{\latintext > B Cod.Nic.}}
\edtext{Germanike'iac}{\Dfootnote{Germanik'iac \latintext B}}, <`oc <'usteron met`a
Maked'onion\edindex[namen]{Macedonius!Bischof von Konstantinopel} \edtext{t`hn
Kwnstantinoup'olewc >epetr'aph}{\Dfootnote{t~hc
Kwnstantinoup'olewc >egqeir'izetai t`hn \latintext T}}\edindex[namen]{Konstantinopel}
>ekklhs'ian, ka`i Grhg'orioc\edindex[namen]{Gregor!Bischof von
Alexandrien} <o
t~hc >Alexandr'ewn \edtext{\abb{>ekklhs'iac}}{\Dfootnote{\latintext > T}} a<ireje`ic
prostate~in, o<`i d`h t'ote t`a a>ut`a frone~in >all'hloic <wmol'oghnto, >all`a g`ar ka`i
Di'anioc\edindex[namen]{Dianius!Bischof von Caesarea} <o
\edtext{\abb{t~hc}}{\Dfootnote{\latintext > b}} par`a Kappad'okaic Kaisare'iac
>ep'iskopoc ka`i Ge'wrgioc <o Laodike'iac\edindex[namen]{Georg!Bischof von
Laodicea} t~hc par`a S'uroic, >'alloi te
\edtext{\abb{pollo`i}}{\Dfootnote{\latintext > T}} mhtropolitik`ac ka`i >'allwc
>epis'hmouc >ekklhs'iac >episkopo~untec. s`un auto~ic d`e ka`i
Eus'ebioc\edindex[namen]{Eusebius!Bischof von Emesa} <o >ep'iklhn
\edtext{>Emesv\-hn\-'oc}{\Dfootnote{>Emesin'oc \latintext T \greektext >Em'eshc \latintext A
Cod.Nic.}};
\pend
\pstart
\noindent\dots
\pend
\pstart
\kap{6}>en d`e t~w| n~un t`a a>ut`a to~ic >en >Antioqe'ia|\edindex[synoden]{Antiochien!a. 341} suneljo~usin >eyhf'isato.
M'aximon\edindex[namen]{Maximus!Bischof von Jerusalem} m'entoi
\edtext{t`on}{\Dfootnote{t~wn \latintext C}} <Ierosol'umwn >ep'iskopon
>ep'ithdec l'egetai ta'uthn >apofuge~in t`hn s'unodon, metamelhj'enta kaj'oti >apathje`ic
s'umyhfoc >eg'eneto to~ic >Ajan'asion\edindex[namen]{Athanasius!Bischof von
Alexandrien} kajelo~usin. o>u m`hn o>ud`e <o
\edtext{t`on}{\Dfootnote{t~wn \latintext C}} <Rwma'iwn di'epwn jr'onon o>ud`e t~wn >'allwn
>Ital~wn >`h t~wn >ep'ekeina <Rwma'iwn o>ude`ic >enj'ade sun~hljen.\edlabel{soz4}
\pend
% \endnumbering
\end{Leftside}
\begin{Rightside}
\begin{translatio}
% \beginnumbering
\pstart
% \autor{Sozomenus}
\noindent\kapR{III 5,1}Nachdem er das geregelt hatte, kam er (der Kaiser) ins
syrische Antiochien. Da der dortige Kirchenbau gerade fertiggestellt
war, der an Gr��e und Sch�nheit alles �berragt, und den Konstantin
seinerzeit noch, unterst�tzt von seinem Sohn Constantius, zu errichten
begonnen hatte, da schien den Eusebianern der geeignete Zeitpunkt
gekommen zu sein, eine Synode zu versammeln, was sie schon lange
anstrebten.  
\kapR{2}Sie also und andere, die mit ihnen einig waren,
97 Bisch�fe zusammengenommen, kamen damals aus vielen Richtungen nach
Antiochien zusammen, angeblich um die neuerbaute Kirche zu weihen,
aber, wie der Ablauf zeigte, zur �nderung der Beschl�sse von Nicaea.
Der Kirche von Antiochien stand damals Placitus\footnoteA{Vgl. oben die Anm. zu \edpageref{Flacillus},\lineref{Flacillus}.}, der Nachfolger des
Euphronius, vor, und man z�hlte das f�nfte Jahr nach dem Tode
Konstantins des Gro�en.  
\kapR{3}Nachdem aber alle Bisch�fe versammelt
waren und sogar Kaiser Constantius erschienen war, f�hrte die Mehrheit
Beschwerde gegen Athanasius und beschuldigten ihn
heftigst\footnoteA{Das hier in 2,3~f. Berichtete geh�rt tats�chlich zu
  einer fr�heren Synode in Antiochien im Jahr 338; die Einsetzung
  Gregors geht der Flucht des Athanasius nach Rom voraus.}, da� er eine
kirchliche Vorschrift �bertreten habe, die sie selbst erst beschlossen
hatten, da� er n�mlich die Kirche der Alexandriner �bernommen h�tte,
noch bevor eine Synode dies gestattet habe. Infolgedessen sagten
sie, er trage auch die Schuld am Tod von B�rgern (Alexandriens), da sein
Einzug in die Stadt einen Aufstand ausgel�st habe und viele
umgekommen, andere aber den Gerichten �bergeben worden seien. 
\kapR{4}Als nun wegen derartigen Geredes eine sehr gro�e Anklage gegen
Athanasius geschmiedet worden war, beschlossen sie, da� Gregor der
Kirche von Alexandrien vorstehen solle\footnoteA{Es folgt ein Referat
  �ber die theologischen Beschl�ssen der Synode, vgl. die folgenden
  Dokumente.}.  
\pend
\pstart
\noindent\dots
\pend
\pstart
\kapR{III 5,10}An dieser Synode aber nahm nicht nur Eusebius teil, der nach der Vertreibung
des Paulus aus Nikomedien auf den Thron von Konstantinopel gewechselt hatte\footnoteA{Eusebius
von Nikomedien; Kritik am Wechsel des Bischofssitzes �u�erte schon Alexander von
Alexandrien, vgl. Dok. \ref{sec:4b} = Urk. 4b,4; Can. 15 und 16 von Nicaea.}, sondern auch
Acacius\footnoteA{Acacius von Caesarea, seit ca. 340 Nachfolger des Eusebius und wie dieser ein
Hauptgegner Markells (Socr., h.\,e. II 4; Soz., h.\,e. III 2,9). Unter Umst�nden gehen auf ihn Formulierungen der sogenannten
zweiten antiochenischen Formel zur�ck; 359 war er an der Formulierung der theologischen Erkl�rung der Synode von Seleucia beteiligt. Sein Tod f�llt in das Jahr
365.}, der Nachfolger des Eusebius, des Sohnes des Pamphilus, Patrophilus von
Scythopolis\footnoteA{\protect\label{fn:Patrophilus}Patrophilus von Scythopolis, war
schon Adressat von Briefen des Arius, als jener Unterst�tzung gegen seine Vertreibung aus
Alexandrien suchte (Dok. \ref{ch:10} = Urk. 10), akzeptierte dann die Entscheidung von Nicaea 325
(Soz., h.\,e. I 21,2), war aber 335 mitverantwortlich f�r die erste Exilierung des
Athanasius (Socr., h.\,e. I 35); Acacius und Patrophilus setzten sp�ter Cyrill anstelle von Maximus
als Bischof von Jerusalem ein (Socr., h.\,e. II 38,1).}, Theodorus von
Heraclea\footnoteA{Theodorus von Heraclea, ebenfalls verantwortlich f�r die Verurteilung
des Athanasius in Tyrus (Socr., h.\,e. I 31), wurde sp�ter mit einer theologischen
Kompromi�formel im Anschlu� an die Synode von Antiochien nach Trier gesandt (Socr., h.\,e.
II 18; Soz., h.\,e. III 10; vgl. Dok. \ref{sec:BriefJuliusII},32; Dok. \ref{ch:AntIV}); er verstarb vor der Synode von
Rimini/Seleucia.}, das fr�her Perinthus hie�, Eudoxius von Germanicia\footnoteA{Eudoxius,
Bischof von Germanicia (Ath., h.\,Ar. 4,2; syn. 1,3; Soz., h.\,e. III 14,42), geh�rt zu der
Delegation, die 344 die sog. Ekthesis makrostichos nach Mailand �bermittelte (s. Dok.
\ref{ch:Makrostichos} und Ath., syn. 26,1), wurde 357/58 Bischof von Antiochien (Ath.,
syn. 12,2; Socr., h.\,e. II 37,7--9; Soz., h.\,e. IV 12,4), 360 Bischof von Konstantinopel
(Socr., h.\,e. II 42; Soz., h.\,e. IV 26,1) und war als solcher mitverantwortlich f�r die hom�ische Kirchenpolitik
des Kaisers Valens.}, der sich sp�ter nach Macedonius der Kirche von Konstantinopel zuwandte,
und Gregor\footnoteA{Gregor, vgl. Dok. \ref{sec:BriefJulius}, Einleitung.}, der ausgesucht
worden war, der Kirche von Alexandrien vorzustehen~-- sie alle aber bekannten damals
einander, dieselben Ansichten zu vertreten~--, aber auch Dianius\footnoteA{Dianius,
Bischof von Caesarea in Cappadocia; vgl. auch die Unterschriftenliste der ">�stlichen"<
Synode von Serdica (Dok. \ref{sec:NominaepiscSerdikaOst}, Nr. 17; und Basilius, ep.
51).}, der Bischof von Caesarea in Cappadocia, Georg von Laodicea\footnoteA{Georg von
Laodicea, alexandrinischer Presbyter, schon von Alexander als Anh�nger des Arius
vertrieben (apol.\,sec. 8,3), schrieb aus Antiochien an Arius und Alexander (vgl. Dok. \ref{ch:12} = Urk. 12; Dok. \ref{ch:13} = Urk. 13 und Dok. \ref{ch:16} = Urk. 16), wurde um 330 Bischof von Laodicea (Ath., h.\,Ar. 4,2) und von der ">westlichen"< Synode in Serdica
343 (vgl. Dok. \ref{ch:SerdicaEinl}) in Abwesenheit verurteilt (Dok.
\ref{sec:SerdicaRundbrief},14; 16; fug. 26,4; Soz., h.\,e. III 12,3), distanzierte sich sp�ter
von der hom�ischen Richtung des Eudoxius (Brief bei Soz., h.\,e. IV 13,2~f.), geh�rte mit
Basilius von Ancyra zu den sog. Hom�usianern (vgl. Epiph., haer. 73) und ist Verfasser
eines heute verlorenen Enkomiums auf Eusebius von Emesa und eines ebenfalls verlorenen Traktats gegen die Manich�er (Epiphan., haer. 66,21,3).} in Syrien
und viele andere, die Bisch�fe von Kirchen in Metropolen oder anderweitig bekannten
St�dten waren. \kapR{6,1}Unter ihnen war auch Eusebius, der ">der Emesener"< genannt wurde.
\pend
\pstart
\noindent\dots
\pend
\pstart
\kapR{6,7}Damals aber stimmte er (Eusebius von Emesa) zusammen mit den in Antiochien
Versammelten f�r dieselben Entscheidungen. \kapR{6,8} Maximus jedoch\footnoteA{Zu Maximus
vgl. die Anm. auf S. \pageref{fn:Maximus}.}, der Bischof von
Jerusalem, soll angeblich bewu�t dieser Synode ferngeblieben seien, da es ihn gereute, da� er
get�uscht worden war und daher die unterst�tzt hatte, die den Athanasius abgesetzt hatten. Freilich reiste auch nicht der Vorsteher des r�mischen Stuhls oder irgendein anderer der Italier oder der R�mer jenseits davon an.
\pend
\endnumbering
\end{translatio}
\end{Rightside}
\Columns
\end{pairs}
% \thispagestyle{empty}
\selectlanguage{german}
% \renewcommand*{\goalfraction}{.9}

%%% Local Variables: 
%%% mode: latex
%%% TeX-master: "dokumente_master"
%%% End: 

%% Erstellt von uh
%% �nderungen:
% \cleartooddpage
\section[Theologische Erkl�rung des Theophronius von Tyana (3. antiochenische Formel)][Theologische Erkl�rung des Theophronius von Tyana]{Theologische Erkl�rung des Theophronius von Tyana (3. antiochenische Formel)}
% \label{sec:41.3}
\label{sec:AntIII}
\begin{praefatio}
  \begin{description}
  \item[Anfang 341]Vgl. oben
    S. \pageref{sec:BerichteAntiochien341}. Diese rechtfertigende
    Erkl�rung hat der sonst unbekannte Bischof Theophronius von Tyana\index[namen]{Theophronius!Bischof
      von Tyana} wahrscheinlich zu Beginn der antiochenischen Synode vorgelegt,
    da die Frage der Rechtgl�ubigkeit und damit der synodalen
    Gemeinschaft zu kl�ren war. In �hnlicher Weise mu�te im Jahr 325
    auf der Synode von Nicaea\index[synoden]{Nicaea!a. 325} Eusebius von
    Caesarea\index[namen]{Eusebius!Bischof von Caesarea} seinen
    pers�nlichen Glauben erkl�ren (vgl. Dok. \ref{ch:22} = Urk. 22) wie auch Markell von
    Ancyra\index[namen]{Markell!Bischof von Ancyra} in
    Rom\index[namen]{Rom} (vgl. Dok. \ref{sec:MarkellJulius}). So ist
    die Reihenfolge der Texte, wie sie
    Athanasius\index[namen]{Athanasius!Bischof von Alexandrien} in
    seiner Schrift \textit{De synodis} bietet, nicht durch die Chronologie,
    sondern durch die logische Reihung der Akten der antiochenischen
    Synode, wie sie Athanasius offenbar vorlagen, bestimmt. �ber die
    Person des Theophronius\index[namen]{Theophronius!Bischof von
      Tyana} ist sonst nichts weiter bekannt.
  \item[�berlieferung]Die �berlieferung dieser Erkl�rung ist
    der Absicht des Athanasius zu verdanken, in seiner Schrift De
    synodis die wankelm�tige Theologie seiner Gegner, der Eusebianer,
    mit einer Vielfalt von Bekenntnissen zu belegen, so da� er auch
    diesen Text zitiert. Vielleicht kannte auch Hilarius von
    Poitiers\index[namen]{Hilarius!Bischof von Poitiers} diesen Text
    und spielt vor seinem Zitat der sog. zweiten antiochenischen
    Formel (vgl. Dok. \ref{sec:AntII}) darauf an: \textit{expositio
      ecclesiasticae fidei, quae exposita est in synodo habita per
      encaenias Antiochenae ecclesiae consummatae. exposuerunt qui
      adferunt episcopi nonaginta septem, cum in suspicionem venisset
      unus ex episcopi quod praua sentiret} (Hil., syn. 29 [\cite[502\,A]{Hil:Syn}]). Athanasius\index[namen]{Athanasius!Bischof von
      Alexandrien} konnte offenbar auf eine Aktensammlung der
    antiochenischen Synode zur�ckgreifen, die die Synodalen damals
    nach Rom\index[namen]{Rom} geschickt hatten, einen Brief, eine
    allgemeine Glaubenserkl�rung der Synode und diese pers�nliche
    Erkl�rung enthaltend. Vielleicht wurde sie als vorbildliches
    Beispiel angeh�ngt, da die Antiochener Gleiches von
    Markell\index[namen]{Markell!Bischof von Ancyra} selbst
    erwarteten.
  \item[Fundstelle]Ath., syn. 24,2--5 (\editioncite[250,8--21]{Opitz1935})
  \end{description}
\end{praefatio}
\begin{pairs}
\selectlanguage{polutonikogreek}
\begin{Leftside}
% \beginnumbering
\pstart
\hskip -.68em\edtext{\abb{}}{\killnumber\Cfootnote{\hskip -1.1em\latintext Ath.(BKPO
R)}}\specialindex{quellen}{section}{Athanasius!syn.!24,2--4} %%
\begin{footnotesize}
\kap{1}\hskip -1em Ka`i Jeofr'onioc d'e tic >ep'iskopoc Tu'anwn\edindex[namen]{Theophronius!Bischof
von Tyana} sunje`ic ka`i a>ut`oc
>ex'ejeto t`hn p'istin ta'uthn >'emprosjen t~wn p'antwn, <~h| ka`i p'antec <up'egrayan
>apodex'amenoi t`hn to~u >anjr'wpou p'istin;
\end{footnotesize}
\pend
\pstart
\kap{2}o>~iden <o je'oc, <`on m'artura kal~w >ep`i t`hn >em`hn yuq'hn, <'oti o<'utwc
piste'uw; e>ic je`on pat'era \edtext{\abb{pantokr'atora}}{\Afootnote{\latintext vgl. Apc
1,8 u.\,�.}}\edindex[bibel]{Offenbarung!1,8|textit}, t`on t~wn <'olwn kt'isthn ka`i
poiht'hn, \edtext{\abb{>ex o~<u t`a p'anta}}{\Afootnote{\latintext vgl. 1Cor
8,6}}\edindex[bibel]{Korinther I!8,6|textit},
\pend
\pstart
\kap{3}ka`i e>ic t`on u<i`on a>uto~u t`on
\edtext{\abb{monogen~h}}{\Afootnote{\latintext Io 1,14.18; 3,16; 1Io
4,9}},\edindex[bibel]{Johannes!1,14}\edindex[bibel]{Johannes!1,18}\edindex[bibel]{Johannes!3,16}\edindex[bibel]{Johannes I!4,9} je'on,
\edtext{\abb{l'ogon, d'unamin ka`i
sof'ian}}{\Afootnote{\latintext vgl. Io 1,1; 1Cor
1,24}}\edindex[bibel]{Johannes!1,1|textit}\edindex[bibel]{Korinther I!1,24|textit}, t`on
k'urion <hm~wn >Ihso~un Qrist'on, \edtext{\abb{di'' o~<u t`a p'an\-ta}}{\Afootnote{\latintext vgl.
Io 1,3; 1Cor 8,6; Col 1,16; Hebr
1,2}},\edindex[bibel]{Johannes!1,3|textit}
\edindex[bibel]{Korinther I!8,6|textit}
\edindex[bibel]{Kolosser!1,16|textit}\edindex[bibel]{Hebraeer!1,2|textit}t`on gennhj'enta >ek to~u patr`oc \edtext{pr`o \edtext{\abb{t~wn}}{\Dfootnote{\latintext >
K}} a>i'wnwn}{\Afootnote{\latintext vgl. 1Cor 2,7}}\edindex[bibel]{Korinther I!2,7|textit}, je`on t'eleion >ek jeo~u tele'iou ka`i >'onta
\edtext{\abb{pr`oc t`on
je`on}}{\Afootnote{\latintext vgl. Io 1,2}}\edindex[bibel]{Johannes!1,2|textit} >en
<upost'asei, \edtext{\abb{>ep'' >esq'atwn d`e t~wn <hmer~wn}}{\Afootnote{\latintext vgl.
Hebr 1,2}}\edindex[bibel]{Hebraeer!1,2|textit}
\edtext{\abb{katelj'onta}}{\Afootnote{\latintext vgl. Iac
3,15}}\edindex[bibel]{Jakobus!3,15|textit} ka`i \edtext{\abb{gennhj'enta >ek t~hc
parj'enou}}{\Afootnote{\latintext vgl. Mt 1,23; Lc
1,27.34~f.}}\edindex[bibel]{Matthaeus!1,23|textit}\edindex[bibel]{Lukas!1,27|textit}\edindex[bibel]{Lukas!1,34~f.|textit} kat`a t`ac graf'ac,
\edtext{\abb{>enanjrwp'hsanta}}{\Afootnote{\latintext vgl. 1Cor
15,47}}\edindex[bibel]{Korinther I!15,47|textit},
\edtext{\abb{paj'onta}}{\Afootnote{\latintext vgl. 1Petr 2,21; auch 1Cor
15,3}}\edindex[bibel]{Petrus I!2,21|textit}\edindex[bibel]{Korinther I!15,3|textit} ka`i
\edtext{\abb{>anast'anta}}{\Afootnote{\dt{vgl. Mc 8,31par; 9,9~f.par; Io 20,9; Act
10,41; 1Thess
4,14}}}\edindex[bibel]{Markus!8,31par.|textit}\edindex[bibel]{Markus!9,9~f.par|textit}\edindex[bibel]{Johannes!20,9|textit}\edindex[bibel]{Apostelgeschichte!10,41|textit}\edindex[bibel]{Thessalonicher I!4,14|textit} >ap`o t~wn nekr~wn ka`i
\edtext{\abb{>anelj'onta e>ic to`uc o>urano`uc}}{\Afootnote{\latintext vgl. Mc 16,19; Act
1,2; 1Petr
3,22}}\edindex[bibel]{Markus!16,19|textit}\edindex[bibel]{Apostelgeschichte!1,2|textit}\edindex[bibel]{Petrus I!3,22|textit} ka`i \edtext{\abb{kajesj'enta >ek dexi~wn to~u
patr`oc a>uto~u}}{\Afootnote{\latintext vgl. Ps 110,1; Eph 1,20; Col 3,1; 1Petr 3,22; Hebr
1,3; Mc 16,19}}\edindex[bibel]{Psalmen!110,1|textit}\edindex[bibel]{Epheser!1,20|textit}\edindex[bibel]{Kolosser!3,1|textit}\edindex[bibel]{Petrus I!3,22|textit}\edindex[bibel]{Hebraeer!1,3|textit}\edindex[bibel]{Markus!16,19|textit}
ka`i p'alin \edtext{\abb{>erq'omenon met`a d'oxhc ka`i dun'amewc}}{\Afootnote{\latintext
vgl. Mt 24,30; auch Mt 10,23par; Mc 8,38par; Act 1,11; 1Cor
4,5}}\edindex[bibel]{Matthaeus!24,30|textit}\edindex[bibel]{Matthaeus!10,23par|textit}\edindex[bibel]{Markus!8,38par|textit}\edindex[bibel]{Apostelgeschichte!1,11|textit}\edindex[bibel]{Korinther I!4,5|textit} \edtext{\abb{kr~inai z~wntac ka`i
nekro`uc}}{\Afootnote{\latintext 2Tim 4,1; 1Petr 4,5; vgl. Io 5,22; Act
10,42}}\edindex[bibel]{Timotheus II!4,1}\edindex[bibel]{Petrus I!4,5}\edindex[bibel]{Johannes!5,22|textit}
\edindex[bibel]{Apostelgeschichte!10,42|textit}
ka`i m'enonta e>ic to`uc a>i~wnac,
\pend
\pstart
\kap{4}ka`i e>ic t`o pne~uma t`o <'agion, t`on
\edtext{\abb{par'aklhton}}{\Afootnote{\latintext Io 14,16.26; 15,26;
16,7}}\edindex[bibel]{Johannes!14,16}\edindex[bibel]{Johannes!14,26}\edindex[bibel]{Johannes!15,26}\edindex[bibel]{Johannes!16,7}, t`o \edtext{\abb{pne~uma t~hc
>alhje'iac}}{\Afootnote{\latintext Io 14,17; 15,26; 16,13; 1Io
4,6}}\edindex[bibel]{Johannes!14,17}\edindex[bibel]{Johannes!15,26}\edindex[bibel]{Johannes!16,13}\edindex[bibel]{Johannes I!4,6}, <`o ka`i di`a to~u
prof'htou >ephgge'ilato <o je`oc \edtext{\abb{>ekq'eein}}{\Afootnote{\latintext vgl. Act
2,17-21.33; Io 15,26; Tit
3,6}}\edindex[bibel]{Apostelgeschichte!2,17--21|textit}\edindex[bibel]{Apostelgeschichte!2,33|textit}\edindex[bibel]{Johannes!15,26|textit}\edindex[bibel]{Titus!3,6|textit} >ep`i
to`uc <eauto~u do'ulouc ka`i <o k'urioc >ephgge'ilato p'emyai to~ic <eauto~u majhta~ic,
<`o ka`i >'epemyen, <wc a<i Pr'axeic t~wn >Apost'olwn marturo~usin.
\pend
\pstart
\kap{5}e>i d'e tic par`a ta'uthn t`hn p'istin did'askei >`h >'eqei >en <eaut~w|,
\edtext{\abb{>an'ajema >'estw}}{\Dfootnote{\latintext BKPO R \greektext  <`a m'ajhm'a
>esti \latintext coni. Tetz del. Scheidweiler}} \edtext{\abb{ka`i}}{\Dfootnote{\latintext del.
Scheidweiler}} \Ladd{\edtext{\abb{e>'i tic did'askei t`a}}{\Dfootnote{\latintext coni.
Stockhausen \griech{t`a} coni. Scheidweiler \griech{kat`a} coni. Opitz}}} Mark'ellou
to~u >Agk'urac\edindex[namen]{Markell!Bischof von Ancyra} >`h
Sabell'iou\edindex[namen]{Sabellius} >`h Pa'ulou to~u Samosat'ewc\edindex[namen]{Paulus
von Samosata}, >an'ajema >'estw ka`i a>ut`oc ka`i p'antec o<i koinwno~untec a>ut~w|.
\pend
% \endnumbering
\end{Leftside}
\begin{Rightside}
\begin{translatio}
\beginnumbering
\pstart
\begin{footnotesize}
\noindent\kapR{24,1}Auch ein gewisser Theophronius, Bischof von Tyana, stellte folgende
Glaubenserkl�rung zusammen und legte sie selbst allen vor, die auch alle unterschrieben
und damit den Glauben dieses Menschen akzeptierten:
\end{footnotesize}
\pend
\pstart
\kapR{2}">Gott, den ich als Zeugen f�r meine Seele anrufe, wei�, da� ich folgenderma�en
glaube: an Gott, den Vater, den Allm�chtigen\footnoteA{Im Unterschied zum Nicaenum (Dok. \ref{ch:24} = Urk.
24) und zur 1. und 2. antiochenischen Formel (Dok. \ref{sec:AntII}; \ref{sec:AntI}) ohne das Attribut
\griech{e~<ic.}}, den Sch�pfer und Erschaffer des Alles, aus dem alles ist,
\pend
\pstart
\kapR{3}und an seinen eingeborenen\footnoteA{Dieses Attribut wurde
  bevorzugt von Markell dem Pr�existenten zugewiesen im Unterschied zu
  \griech{prwt'otokoc}, das dem Inkarnierten vorbehalten bleibe,
  vgl. Markell, fr. 10,13--15 (Seibt/Vinzent).} Sohn, Gott, Wort,
Kraft und Weisheit, unseren Herrn Jesus Christus, durch den alles
ist\footnoteA{Parallel formuliert zum ersten Artikel: \griech{>ex o~<u
    t`a p'anta.}}, der aus dem Vater vor den Zeiten gezeugt
wurde\footnoteA{Die Rede von der pr�existenten Zeugung des Sohnes aus
  dem Vater wurde von Markell kritisiert, vgl. Markell, fr. 71
  (Seibt/Vinzent).}, vollkommener Gott aus vollkommenem
Gott\footnoteA{Erweiterung der Aussage des Nicaenums \griech{je`on >ek
    jeo~u}, vgl. die gr��ere Ausschm�ckung bei der 2. antiochenischen
  Formel (Dok. \ref{sec:AntII},1,2).}, und der bei Gott in einer
Hypostase\footnoteA{In der 2. antiochenischen Formel
  (Dok. \ref{sec:AntII}) fehlt Entsprechendes, aber Eusebius von Caesarea
  hatte Markell vorgeworfen, er raube dem Pr�existenten die Hypostase
  bzw. die Eigenexistenz (e.\,th. I 10,4; II 12,3~ff.). Eventuell w�hlt Theophronius bewu�t die Formulierung \griech{>en <upost'asei}, ohne sich auf eine eindeutige Option in der Hypostasenfrage festzulegen.} ist, der in den
letzten Tagen\footnoteA{Dieser folgende heilsgeschichtliche Abschnitt
  findet sich in der 2. antiochenischen Formel (Dok. \ref{sec:AntII},1,4)
  wieder, wo zus�tzlich eine plerophore Umschreibung des Erl�sers als
  Mittler zwischen Gott und den Menschen, der den v�terlichen Willen
  erf�llt, eingeschoben ist. In Teilen sind gleiche Formulierungen
  schon im Nicaenum (Dok. \ref{ch:24} = Urk. 24) zu finden, wobei das nicaenische
  ">Fleischwerden"< nun mit ">Zeugung aus der Jungfrau nach der
  Schrift"< umschrieben wird.} herabgestiegen und aus der Jungfrau
gem�� den Schriften geboren worden ist, Mensch wurde, der gelitten hat und auferstanden ist von den Toten und
hinaufgestiegen ist in den Himmel und sich zur Rechten seines Vaters
gesetzt hat und der wiederum kommt mit Herrlichkeit und Kraft, um die
Lebenden und die Toten zu richten, und der f�r alle Zeiten
bleibt\footnoteA{Dieser Zusatz bedeutet eine Distanzierung von
  Markell.},
\pend
\pstart
\kapR{4}und an den heiligen Geist\footnoteA{Dieser dritte Artikel ist
  sehr ausf�hrlich, wie es sich weder im Nicaenum noch bei Markell oder
  in der 1. antiochenischen Formel (Dok. \ref{sec:AntI}) findet. Die
  etwas ausgeweitete entsprechende Passage in der zweiten
  antiochenischen Formel (Dok. \ref{sec:AntII},1,5~f.) f�llt wieder anders
  aus und bietet durch die Einarbeitung des Taufbefehls aus Mt 28,19
  eine �berleitung zur Betonung der drei Hypostasen.}, den Tr�ster,
den Geist der Wahrheit, den Gott auch durch den Propheten verhei�en
hat, �ber seine eigenen Knechte auszugie�en, und den der Herr
verhei�en hat, seinen J�ngern zu schicken, den er auch geschickt hat,
wie es die Apostelgeschichte bezeugt.
\pend
\pstart
\kapR{5}Wenn aber jemand gegen diesen Glauben lehrt oder es im Sinn
hat, sei er verflucht, und wenn einer das lehrt, was Markell von
Ancyra lehrt oder Sabellius oder Paulus von
Samosata,\footnoteA{Theophronius mu�te sich offensichtlich von Markell
  bzw. von den Irrlehren des Sabellius und Paulus, die Markell erneuere,
  distanzieren. Zu Sabellius vgl. die Anm. auf
  S. \pageref{fn:Sabell}; zu Paulus von Samosata vgl. die Anm. auf S. \pageref{fn:Samosata}.} sei er selbst verflucht
samt allen, die mit ihm Gemeinschaft halten."<
\pend
\endnumbering
\end{translatio}
\end{Rightside}
\Columns
\end{pairs}
% \thispagestyle{empty}
\selectlanguage{german}
%%% Local Variables:
%%% mode: latex
%%% TeX-master: "dokumente_master"
%%% End:

%% Erstellt von uh
%% �nderungen:
%%%% 13.5.2004 Layout-Korrekturen (avs) %%%%
% \section[Ant. I/II]{Briefe und Glaubenserkl�rung der Kirchweihsynode von Antiochien}
% \cleartooddpage
% \label{sec:41.4}
\section[Theologische Erkl�rung der Synode von Antiochien des Jahres 341 (2. antiochenische
  Formel)][Theologische Erkl�rung der Synode von Antiochien des Jahres 341]{Theologische Erkl�rung der Synode von Antiochien des Jahres 341\\(2. antiochenische
  Formel)}
\label{sec:AntII}
\begin{praefatio}
  \begin{description}
  \item[Anfang 341]Zur Datierung
    s.\,o. S. \pageref{sec:BerichteAntiochien341}. Alle
    �berlieferungstr�ger weisen diese Erkl�rung der Kirchweihsynode
    von Antiochien\index[synoden]{Antiochien!a. 341} zu. Ferner ist zu
    beachten, da� diese Erkl�rung urspr�nglich zu einem Brief geh�rte,
    �ber dessen Inhalt nichts weiter bekannt ist als die wenigen
    Hinweise, die Sozomenus in seinem Regest bietet (h.\,e. III
    5,8~f.; Text s.\,u.). Nach der Einleitung des Hilarius (syn. 29
    [\cite[502\,A]{Hil:Syn}]) wurde diese Erkl�rung dadurch veranla�t, da�
    einer der Bisch�fe in den Verdacht der H�resie geraten sei:
    \textit{Exposuerunt qui adferunt episcopi nonaginta septem, cum in
      suspicionem venisset unus ex episcopis quod prava
      sentiret}. Eventuell ist damit Theophronius von
    Tyana\index[namen]{Theophronius!Bischof von Tyana} gemeint
    (vgl. Dok. \ref{sec:AntIII}), und die antimarkellische Tendenz der sogenannten zweiten antiochenischen Formel
    scheint dies zu best�tigen. Somit d�rfte dieser Text als Teil des
    Synodalbriefes der Antiochener an die Bisch�fe des Ostens zu
    verstehen sein, der verfa�t wurde, um die theologischen Positionen
    zu konsolidieren und ">sabellianische"< (markellische) Tendenzen
    auszuschlie�en.
  \item[�berlieferung]Diese Erkl�rung ist mehrfach �berliefert,
    zweimal in griechischer Fassung bei Athanasius und Socrates und
    einmal in lateinischer bei Hilarius.  Da sowohl die Fassung des
    Socrates als auch die des Hilarius Auslassungen vorweisen, die die
    jeweils anderen beiden aber �berliefern, liegt dieser Edition
    der Text des Athanasius zugrunde (vgl. auch
    \cite[72--74]{Durst:Epistula}). An einigen Stellen treten in der
    lateinischen Fassung des Hilarius auch gravierende
    �bertragungsfehler auf (\griech{>'anjrwpon} wird mit
    \textit{agnus}, \griech{>ap'ostol'on} mit \textit{praedestinatus}
    wiedergegeben, vgl. \cite[72--74]{Durst:Epistula}). Es ist nicht m�glich, die �berlieferten Fassungen
    Sabinus\index[namen]{Sabinus von Heraclea} oder der Aktensammlung
    zuzuweisen, die bei den Verhandlungen von
    Rimini\index[synoden]{Rimini!a. 359}""/""Seleucia\index[synoden]{Seleucia!a. 359}""/""Konstantinopel\index[synoden]{Konstantinopel!a. 360}
    benutzt wurde (Soz., h.\,e. IV 22,6). Sozomenus bietet kein Zitat,
    sondern ein Referat (h.\,e. III 5,8~f.), wobei er jedoch au�erdem auf Aktenmaterial der
    Synode zur�ckgriff, wie die zus�tzlichen Informationen zeigen.
  \item[Fundstelle]Ath., syn. 23,2--10 (\editioncite[249,11--250,4]{Opitz1935}); Socr.,
    h.\,e. II 10,10--18 (\editioncite[100,16--101,22]{Hansen:Socr}); Hil., syn. 29~f. (\cite[502\,A--504\,A]{Hil:Syn}). \refpassage{sozreg1}{sozreg2} Soz.,
    h.\,e. III 5,8~f. (\editioncite[106,23--107,3]{Hansen:Soz})
  \end{description}
\end{praefatio}
\autor{Das Bekenntnis}
\begin{pairs}
\selectlanguage{polutonikogreek}
\begin{Leftside}
% \beginnumbering
\pstart 
\hskip -1.4em\edtext{\abb{}}{\killnumber\Cfootnote{\hskip -1em\latintext Socr. (MF=b A T) Ath. (BKPO
R) Hil.}}\specialindex{quellen}{section}{Socrates!h.\,e.!II 10,10--18}\specialindex{quellen}{section}{Athanasius!syn.!23,2--10}
\kap{1,1}Piste'uomen >akolo'ujwc t~h| \edtext{e>uaggelik~h| ka`i
>apostolik~h|}{\lemma{\abb{}}\Dfootnote{\responsio\ >apostolik~h ka`i e>uaggelik~h
\latintext Socr.(T)}} parad'osei e>ic \edtext{\abb{<'ena je`on}}{\Afootnote{\latintext
vgl. 1Cor 8,6; Eph 4,6}}\edindex[bibel]{Korinther I!8,6|textit}\edindex[bibel]{Epheser!4,6|textit} pat'era
\edtext{\abb{pantokr'atora}}{\Afootnote{\latintext Apc 1,8
u.\,�.}}\edindex[bibel]{Offenbarung!1,8|textit}, t`on \edtext{t~wn
<'olwn}{\Dfootnote{\latintext
\textit{cunctorum quae sunt} Hil.}} dhmiourg'on \edtext{\abb{te}}{\Dfootnote{\latintext >
Hil.}} ka`i \edtext{poiht`hn ka`i pronoht'hn}{\lemma{\abb{}}\Dfootnote{\responsio\
pronoht'hn ka`i poiht`hn \latintext Socr.(T) \griech{poiht'hn} Socr.(bA) \textit{factorem
et provisorem} Hil.}}, \edtext{\abb{>ex o~<u t`a p'anta}}{\Afootnote{\latintext vgl. 1Cor
8,6}\Dfootnote{\latintext > Socr.}}\edindex[bibel]{Korinther I!8,6|textit},
\pend
\pstart
\kap{2}ka`i \edtext{\abb{e>ic}}{\Dfootnote{\latintext > Socr.(A)}} <'ena k'urion
\edtext{>Ihso~un Qrist'on}{\lemma{\abb{>Ihso~un Qrist`on}}\Afootnote{\latintext 1Cor
8,6}}\edindex[bibel]{Korinther I!8,6}, t`on u<i`on a>uto~u, t`on
\edtext{\abb{monogen~h}}{\Afootnote{\latintext Io 1,14.18; 3,16; 1Io
4,9}}\edindex[bibel]{Johannes!1,14}\edindex[bibel]{Johannes!1,18}\edindex[bibel]{Johannes!3,16}\edindex[bibel]{Johannes I!4,9} je'on, \edtext{di'' o~<u t`a
\edtext{\abb{p'anta}}{\Dfootnote{+ >eg'eneto \latintext Socr.(bA)}}}{\Afootnote{\latintext
vgl. Io 1,3; 1Cor 8,6; Col 1,16; Hebr
1,2}}\edindex[bibel]{Johannes!1,3|textit}\edindex[bibel]{Korinther I!8,6|textit}\edindex[bibel]{Kolosser!1,16|textit}\edindex[bibel]{Hebraeer!1,2|textit},
t`on gennhj'enta \edtext{\abb{\edtext{\abb{pr`o}}{\Dfootnote{+ p'antwn \latintext Socr.}}
t~wn \edtext{\abb{a>i'wnwn}}{\Afootnote{\latintext vgl. 1Cor 2,7}}}}{\Dfootnote{\latintext
> Hil.}}\edindex[bibel]{Korinther I!2,7|textit} >ek to~u patr'oc, je`on
\edtext{>ek}{\Dfootnote{\latintext \textit{de} Hil.}} jeo~u, <'olon >ex <'olou, m'onon >ek
m'onou, t'eleion \edtext{>ek}{\lemma{>ek\ts{2}}\Dfootnote{\latintext \textit{de} Hil.}} tele'iou, basil'ea
\edtext{>ek}{\Dfootnote{\latintext \textit{de} Hil. }} basil'ewc, k'urion
\edtext{>ap`o}{\Dfootnote{>ek \latintext Socr.(T) Ath.(B) \textit{de} Hil.}} kur'iou,
\edtext{l'ogon \edtext{\abb{z~wnta}}{\Dfootnote{\latintext > Hil.}}}{\Afootnote{\latintext
vgl. Io 1,4; 1Io 1,1}}\edindex[bibel]{Johannes!1,4|textit}\edindex[bibel]{Johannes I!1,1|textit},
\edtext{\edtext{sof'ian}{\lemma{\abb{sof'ian}}\Afootnote{\latintext vgl. 1Cor 1,24.30}}
z~wsan}{\Dfootnote{sof'ian, zw`hn \latintext Socr. Hil.}}\edindex[bibel]{Korinther I!1,24|textit}\edindex[bibel]{Korinther I!1,30|textit},
\edtext{\abb{f~wc >alhjin'on}}{\Afootnote{\latintext Io 1,9; 1Io
2,8}}\edindex[bibel]{Johannes!1,9}\edindex[bibel]{Johannes I!2,8}, \edtext{\abb{<od'on,
>al'hjeian}}{\Afootnote{\latintext vgl. Io 14,6} \lemma{<od'on,
>al'hjeian}\Dfootnote{<od'on >alhje'iac \latintext Socr. \textit{viam veram}
Hil.}}\edindex[bibel]{Johannes!14,6|textit},
\edtext{\abb{>an'astasin}}{\Afootnote{\latintext Io
11,25}}\edindex[bibel]{Johannes!11,25}, \edtext{\abb{poim'ena}}{\Afootnote{\latintext Io
10,11.14; vgl. Hebr 13,20; 1Petr
2,25}}\edindex[bibel]{Johannes!10,11}\edindex[bibel]{Johannes!10,14}\edindex[bibel]{Hebraeer!13,20|textit}\edindex[bibel]{Petrus I!2,25|textit},
\edtext{\abb{j'uran}}{\Afootnote{\latintext Io
10,7.9}}\edindex[bibel]{Johannes!10,7}\edindex[bibel]{Johannes!10,9}, >'atrept'on 
\edtext{\abb{te}}{\Dfootnote{\dt{> Hil.}}} ka`i
>anallo'iwton, \edtext{\abb{t~hc}}{\Dfootnote{\latintext dupl. Socr.(bA)}} je'othtoc
o>us'iac te ka`i \edtext{\edtext{\abb{boul~hc ka`i}}{\Dfootnote{\latintext > Hil.}}
dun'amewc}{\lemma{\abb{}}\Dfootnote{\responsio\ dun'amewc ka`i boul~hc \latintext Socr.}}
ka`i d'oxhc \edtext{\abb{to~u patr`oc}}{\Dfootnote{\latintext > Hil}} >apar'allakton
\edtext{\abb{e>ik'ona}}{\Afootnote{\latintext vgl. Col
1,15}}\edindex[bibel]{Kolosser!1,15|textit}, 
\edtext{t`on
\edtext{\abb{prwt'otokon}}{\Afootnote{\latintext vgl. Col
1,15}}}{\lemma{prwt'otokon}\Dfootnote{\latintext \textit{primum editum} Hil.}}\edindex[bibel]{Kolosser!1,15|textit}
\edtext{\abb{p'ashc}}{\Dfootnote{+ t~hc
\latintext Ath.(BP)}} kt'isewc,
\kap{3}t`on \edtext{>'onta}{\Dfootnote {\latintext \textit{semper fuit} Hil.}}
\edtext{>en >arq~h| pr`oc t`on je'on}{\lemma{\abb{>en \dots\ je'on}}\Afootnote{\latintext
Io 1,2}}\edindex[bibel]{Johannes!1,2}, \edtext{l'ogon \edtext{je`on}{\Dfootnote{jeo~u
\latintext Socr.(M\textsuperscript{1})}}}{\lemma{\abb{}}\Dfootnote{\responsio\ je`on
l'ogon \latintext Ath.(B)}} \edtext{kat`a t`o e>irhm'enon}{\Dfootnote{\latintext
\textit{iuxta quae dictum est} Hil.}} >en t~w| e>uaggel'iw|; \edtext{((ka`i je`oc >~hn <o
l'ogoc)), di'' o<~u t`a p'anta >eg'eneto}{\lemma{\abb{ka`i je`oc \dots\
>eg'eneto}}\Afootnote{\latintext vgl. Io 1,1.3; Col
1,16}}\edindex[bibel]{Johannes!1,1|textit}\edindex[bibel]{Johannes!1,3|textit}\edindex[bibel]{Kolosser!1,16|textit},
ka`i \edtext{\abb{>en <~w| t`a p'anta sun'esthke}}{\Afootnote{\latintext vgl. Col
1,17}}\edindex[bibel]{Kolosser!1,17|textit}, 
\kap{4}t`on \edtext{\abb{>ep'' >esq'atwn t~wn
<hmer~wn}}{\Afootnote{\latintext vgl. Hebr 1,2}}\edindex[bibel]{Hebraeer!1,2|textit}
\edtext{\abb{katelj'onta >'anwjen}}{\Afootnote{\latintext vgl. Iac
3,15}}\edindex[bibel]{Jakobus!3,15|textit} ka`i \edtext{\abb{gennhj'enta >ek
parj'enou}}{\Afootnote{\dt{vgl. Mt 1,23; Lc
1,27.34~f.}}}\edindex[bibel]{Matthaeus!1,23|textit}\edindex[bibel]{Lukas!1,27|textit}\edindex[bibel]{Lukas!1,34~f.|textit} kat`a t`ac graf`ac ka`i
\edtext{\edtext{>'anjrwpon}{\Dfootnote {\latintext
\textit{agnus} Hil.}} gen'omenon}{\lemma{\abb{>'anjrwpon gen'omenon}}\Afootnote{\latintext
vgl. Io 1,14; 1Cor 15,47}}\edindex[bibel]{Johannes!1,14|textit}\edindex[bibel]{Korinther I!15,47|textit},
\edtext{\abb{mes'ithn jeo~u ka`i >anjr'wpwn}}{\Afootnote{\latintext 1Tim
2,5}}\edindex[bibel]{Timotheus I!2,5|textit}
\edtext{\abb{>ap'ostol'on}}{\Afootnote{\latintext vgl.
Hebr 3,1}\lemma{>ap'ostol'on}\Dfootnote{\latintext
\textit{praedestinatus} Hil.}}\edindex[bibel]{Hebraeer!3,1|textit} 
\edtext{\abb{te}}{\Dfootnote{\latintext > Hil}} t~hc p'istewc <hm~wn ka`i \edtext{\abb{>arqhg`on t~hc
zw~hc}}{\Afootnote{\latintext Act 3,15}}\edindex[bibel]{Apostelgeschichte!3,15},
\edtext{<'wc fhsi <'oti}{\Dfootnote{\latintext \textit{dixit quippe} Hil.}}
((\edlabel{jo638-1}\edtext{\edtext{katab'ebhka}{\Dfootnote{katab'ebh \latintext Socr.(T)}}
>ek to~u o>urano~u, \edtext{\abb{o>uq}}{\Dfootnote{\latintext + \textit{enim }Hil.}} <'ina
poi~w t`o j'elhma t`o >em'on, >all`a \edtext{\abb{t`o j'elhma}}{\Dfootnote{\latintext >
Socr.(T)}} to~u p'emyant'oc me}{\lemma{\abb{}}\Afootnote{\latintext
Io 6,38}}\edindex[bibel]{Johannes!6,38}))\edlabel{jo638-2}, t`on \edtext{paj'onta
\edtext{\abb{<up`er <hm~wn}}{\Dfootnote{\latintext > Hil.}}}{\lemma{\abb{paj'onta \dots\
<hm~wn}}\Afootnote{\latintext vgl. 1Petr 2,21; auch 1Cor 15,3}}\edindex[bibel]{Petrus I!2,21|textit}\edindex[bibel]{Korinther I!15,3|textit}, ka`i
\edtext{\edtext{\abb{>anast'anta}}{\Dfootnote{\latintext + \textit{pro nobis} Hil.}} t~h|
tr'ith| <hm'era|}{\lemma{\abb{>anast'anta \dots\ <hm'era|}}\Afootnote{\latintext vgl. 1Cor
15,4; Eph 1,20 und Mt 17,9par; 20,19par; 1Thess 4,14}}\edindex[bibel]{Korinther I!15,4|textit}\edindex[bibel]{Epheser!1,20|textit}\edindex[bibel]{Matthaeus!17,9par|textit}
\edindex[bibel]{Matthaeus!20,19par|textit}\edindex[bibel]{Thessalonicher I!4,14|textit} ka`i \edtext{>anelj'onta
\edtext{\edtext{\abb{e>ic}}{\Dfootnote{+ to`uc \latintext Ath.(K)}}
o>urano'uc}{\Dfootnote{\latintext ]\textit{in caelis} Hil.}}}{\Afootnote{\latintext vgl. Mc 16,19; Act 1,2; 1Petr
3,22}}\edindex[bibel]{Markus!16,19|textit}\edindex[bibel]{Apostelgeschichte!1,2|textit}\edindex[bibel]{Petrus I!3,22|textit}, ka`i
\edtext{\edtext{kajesj'enta}{\Dfootnote{kajez'omenon \latintext Socr.(T)}}
>en dexi\aci\ to~u patr`oc}{\lemma{\abb{kajesj'enta}}\Afootnote{\latintext vgl. Ps 110,1;
Eph 1,20; Col 3,1; 1Petr 3,22; Hebr 1,3; Mc
16,19}}\edindex[bibel]{Psalmen!110,1|textit}\edindex[bibel]{Epheser!1,20|textit}\edindex[bibel]{Kolosser!3,1|textit}\edindex[bibel]{Petrus I!3,22|textit}\edindex[bibel]{Hebraeer!1,3|textit}\edindex[bibel]{Markus!16,19|textit} ka`i p'alin \edtext{>erq'omenon met`a d'oxhc
\edtext{\abb{ka`i
dun'amewc}}{\Dfootnote{\latintext > Hil.}}}{\Afootnote{\latintext vgl. Mt 24,30; auch Mt
10,23par; Mc 8,38par; Act 1,11; 1Cor 4,5}}\edindex[bibel]{Matthaeus!24,30|textit}\edindex[bibel]{Matthaeus!10,23par|textit}\edindex[bibel]{Markus!8,38par|textit}\edindex[bibel]{Apostelgeschichte!1,11|textit}\edindex[bibel]{Korinther I!4,5|textit}
\edtext{\abb{kr~inai
z~wntac ka`i
nekro'uc}}{\Afootnote{\latintext 2Tim 4,1; 1Petr 4,5; vgl. Io 5,22; Act
10,42}}\edindex[bibel]{Timotheus II!4,1}\edindex[bibel]{Petrus I!4,5}\edindex[bibel]{Johannes!5,22|textit}\edindex[bibel]{Apostelgeschichte!10,42|textit}.
\pend
\pstart
\kap{5}ka`i e>ic t`o pne~uma t`o <'agion, t`o e>ic
\edtext{\abb{par'aklhsin}}{\Afootnote{\latintext Io 14,16.26; 15,26; 16,7}
\lemma{par'aklhsin}\Dfootnote{kl~hsin \latintext
Socr.(T)}}\edindex[bibel]{Johannes!14,16}\edindex[bibel]{Johannes!14,26}\edindex[bibel]{Johannes!15,26}\edindex[bibel]{Johannes!16,7} ka`i
\edtext{\abb{<agiasm`on}}{\Afootnote{\latintext vgl. Rom 1,4; 1Petr
1,2}}\edindex[bibel]{Roemer!1,4|textit}\edindex[bibel]{Petrus I!1,2|textit}
\edtext{\abb{ka`i}}{\lemma{ka`i\ts{2}}\Dfootnote{+
e>ic \latintext Socr.}} tele'iwsin to~ic piste'uousi did'omenon,
\edtext{kaj`wc}{\Dfootnote{\latintext \textit{iuxta quod} Hil.}} ka`i <o k'urioc
\edtext{\abb{<hm~wn}}{\Dfootnote{\latintext > Hil.}} >Ihso~uc Qrist`oc diet'axato to~ic
majhta~ic l'egwn ((\edtext{poreuj'entec majhte'usate p'anta t`a >'ejnh bapt'izontec
a>uto`uc e>ic t`o >'onoma to~u patr`oc ka`i to~u u<io~u ka`i to~u <ag'iou
pne'umatoc}{\lemma{\abb{}}\Afootnote{\latintext Mt
28,19}}\edindex[bibel]{Matthaeus!28,19})),
\kap{6}\edtext{dhlon'oti}{\Dfootnote{\latintext \textit{manifeste utique} Hil.}} patr`oc,
>alhj~wc \edtext{patr`oc \edtext{\abb{>'ontoc}}{\Dfootnote{\latintext >
Hil.}}}{\lemma{\abb{}}\Dfootnote{\responsio\ >'ontoc \dt{(+} to~u \latintext Socr.(T))
\griech{patr`oc} Socr.}}, \edtext{u<io~u d`e}{\Dfootnote{ka`i u<io~u \latintext Socr.(bT)
\greektext ka`i u<io~u d`e \latintext Socr.(A) \textit{certumque fili} Hil.}} >alhj~wc
\edtext{\abb{\edtext{u<io~u >'ontoc}{\lemma{\abb{}}\Dfootnote{\responsio\ >'ontoc u<io~u \latintext
Socr.(T)}}, \edtext{to~u d`e}{\Dfootnote{ka`i
\latintext Socr.(bT) \griech{ka`i to~u} Socr.(A)}} \edtext{<ag'iou
pne'umatoc}{\lemma{\abb{}}\Dfootnote{\responsio\ pne'umatoc <ag'iou \latintext Socr.(bT)}}
\edtext{\abb{>alhj~wc}}{\Dfootnote{\latintext > Socr.(M\ts{1})}}}}{\Dfootnote{\latintext > Hil., \textit{fili et spiritus sancti vere} add. edd. e textum graecum}} 
\edtext{\edtext{\abb{<ag\-'iou pne'umatoc}}{\Dfootnote{\latintext > Socr.(T)}} 
\edtext{\abb{>'ontoc}}{\Dfootnote{\latintext > Socr.(M\ts{1}) Hil.}}}{\Dfootnote{\responsio\ >'ontoc pne'umatoc <ag'iou}}, \edtext{t~wn
>onom'atwn}{\Dfootnote{\latintext \textit{hisque
nominibus} Hil.}} o>uq <apl~wc o>ud`e \edtext{>arg~wn}{\Dfootnote{>arg~wc \latintext
Socr.(T)}} keim'enwn, >all`a shmain'ontwn >akrib~wc t`hn
\edtext{o>ike'ian}{\Dfootnote{>id'ian \latintext Socr. \textit{propriam} Hil.}}
\edtext{<ek'astou}{\Dfootnote{<ek'astw| \latintext Ath.(B)}} t~wn >onomazom'enwn
<up'ostas'in \edtext{\abb{te}}{\Dfootnote{\latintext > Hil.}} ka`i t'axin ka`i d'oxan, <wc
e>~inai t~h| m`en <upost'asei tr'ia, t~h| d`e \edtext{sumfwn'ia|}{\Dfootnote{sumfu'ia
\latintext Socr.(A)}} <'en.
\pend
\pstart
\kap{7}ta'uthn o>~un >'eqontec t`hn p'istin \edtext{\abb{ka`i >ex >arq~hc ka`i m'eqri
t'elouc >'eqontec}}{\Dfootnote{\latintext > Socr.}} >en'wpion to~u jeo~u ka`i to~u
Qristo~u p~asan a<iretik`hn \edtext{\edtext{kakodox'ian}{\Dfootnote{\latintext \textit{pravam sectam} Hil.}} >anajemat'izomen}{\lemma{\abb{}}\Dfootnote{\responsio\
>anajemat'izomen kakodox'ian \latintext Socr.(bA)}}. 
\kap{8}ka`i e>'i tic par`a t`hn <ugi~h t~wn graf~wn
\edtext{\edtext{>orj`hn}{\Dfootnote{\latintext \textit{et rectam} Hil.}}
p'istin}{\lemma{\abb{}}\Dfootnote{\responsio\ p'istin >orj`hn \latintext Socr.(T)}}
did'askei l'egwn \edtext{\abb{>`h qr'onon}}{\Dfootnote{\latintext > Socr.}} >`h kair`on
>`h a>i~wna \edtext{\abb{>`h}}{\lemma{>`h\ts{1}}\Dfootnote{\latintext > Socr.}} e>~inai >`h gegon'enai pr`o
to~u \edtext{gennhj~hnai t`on u<i'on}{\Dfootnote{t`on u<i`on to~u jeo~u \latintext
Socr.(bA) \textit{generatur filius} Hil.}}, >an'ajema >'estw. ka`i e>'i tic
l'egei t`on u<i`on kt'isma <wc <`en t~wn ktism'atwn \edtext{\abb{>`h g'ennhma \edtext{<wc
<`en t~wn gennhm'atwn}{\Dfootnote{\latintext \textit{sicut sunt nativitates}
Hil.}}}}{\Dfootnote{\latintext > Ath.(B)}} \edtext{\abb{>`h po'ihma \edtext{<wc <`en t~wn
poihm'atwn}{\Dfootnote{\latintext \textit{sicut sunt facturae}
Hil.}}}}{\Dfootnote{\latintext > Ath.(B) Socr.}} ka`i m`h <wc a<i je~iai grafa`i
\edtext{parad'edwkan}{\Dfootnote{paraded'wkasi \latintext Socr.(M\textsuperscript{c}FA)}}
t~wn proeirhm'enwn \edtext{<'ekaston >af> <ek'astou}{\Dfootnote{<'ekaston \latintext
Socr.(T) \greektext <'ekasta \latintext Socr.(bA) \textit{singula quaeque} Hil.}}, 
\edtext{>`h}{\Dfootnote{\latintext \textit{et} Hil.}}
e>'i 
\edtext{tic}{\Dfootnote{ti \latintext Socr. \textit{quis} Hil.}} \edtext{>'allo}{\Dfootnote{>'alla \latintext Socr.(T)}} did'askei >`h
e>uaggel'izetai par'' <`o parel'abomen, >an'ajema >'estw. 
\kap{9}<hme~ic g`ar p~asi to~ic >ek t~wn je'iwn graf~wn paradedom'enoic
\edtext{<up'o}{\Dfootnote{>ap'o \latintext Ath.(K)}} 
\edtext{\abb{te}}{\Dfootnote{+ t~wn \latintext Socr.(bA) > Hil.}} profht~wn ka`i >apo\-st'o\-lwn
\edtext{>alhjin~wc}{\Dfootnote{>alhj~wc \latintext Socr.(T)}} 
\edtext{\abb{te}}{\Dfootnote{\latintext > Hil.}} ka`i
\edtext{>emf'obwc}{\Dfootnote{>emfan~wc \latintext Socr.(bA) \textit{cum timore} Hil.}}
ka`i piste'uomen ka`i \edtext{>akoloujo~umen}{\Dfootnote{>akolouj~wmen \latintext
Ath.(R)}}. 
\pend
\end{Leftside}
\begin{Rightside}
\begin{translatio}
\beginnumbering
\pstart
% \autor{Bekenntnis} 
\noindent\kapR{1}Wir glauben im Anschlu� an die evangelische und apostolische �berlieferung an einen
Gott\footnoteA{Diese Erkl�rung beginnt �hnlich wie die des Theophronius, hier ist
zus�tzlich das Attribut ">einer"< wie im Nicaenum eingef�gt, ferner die Beschreibung als
Demiurg.}, den Vater, den Allm�chtigen, den Sch�pfer, Erschaffer und Bewahrer des Alls, aus dem
alles ist,
\pend
\pstart
\kapR{2}und an einen Herrn Jesus Christus\footnoteA{Dieser Abschnitt beginnt
wie das Nicaenum, erweitert um das pointierte Attribut ">eingeborener Gott"<. Abschlie�end
werden viele Varianten des schon bei Eusebius (Dok. \ref{ch:22} = Urk. 22,4) und im Nicaenum (Dok. \ref{ch:24} = Urk. 24) zu
lesenden Ausdrucks ">Gott aus Gott"< geboten, erweitert um die christologischen
Hoheitstitel aus dem Johannesevangelium (letztere wollte Markell nach fr. 7 und 3
[Seibt/Vinzent] erst dem Inkarnierten zuweisen).}, seinen Sohn, den eingeborenen Gott,
durch den alle Dinge sind, der vor den Zeiten aus dem Vater gezeugt worden ist, Gott aus
Gott, Ganzer aus Ganzem, Einziger aus Einzigem, Vollkommener aus Vollkommenem, K�nig vom K�nig,
Herr vom Herrn, lebendiges Wort, lebendige Weisheit, wahres Licht, Weg, Wahrheit,
Auferstehung, Hirte, T�r, unver�nderlich und unwandelbar\footnoteA{Vgl. die Attribute im
Nicaenum.}, genaues Abbild\footnoteA{Zentrales Attribut bei Alexander von Alexandrien (Dok. \ref{ch:14} = Urk. 14,38~f.47~f.52), Acacius (Epiph., haer.
72,7,3--10,3) und Asterius (fr. 10 und 13).} des Wesens der Gottheit und des Willens, der
Kraft und der Herrlichkeit des Vaters, den Erstgeborenen\footnoteA{Vgl. Eusebius von
Caesarea, Dok. \ref{ch:22} = Urk. 22,4 (43,11 Opitz).} aller Sch�pfung, 
\kapR{3}der am Anfang bei Gott war,
Gott-Wort, wie es im Evangelium gesagt ist: ">und Gott war das Wort"<, durch den alles
wurde und in dem alles besteht, 
\kapR{4}der in den letzten Tagen von oben herabstieg und
gem�� der Schrift aus einer Jungfrau geboren worden ist und Mensch wurde, Mittler zwischen
Gott und den Menschen, Apostel unseres Glaubens und Anf�hrer des Lebens, wie er sagt:
">Ich bin vom Himmel herabgekommen, nicht um meinen Willen\footnoteA{Vgl. Dok.
\ref{sec:AntI},4 (\refpassage{jo434-1}{jo434-2}).}
zu tun, sondern den Willen dessen, der mich gesandt hat"<, der f�r uns litt und am dritten Tag auferstand
und hinaufstieg in den Himmel und zur Rechten des Vaters sitzt und wiederkommen wird mit
Herrlichkeit und Kraft, um die Lebenden und die Toten zu richten. 
\pend
\pstart
\kapR{5}und an den heiligen Geist\footnoteA{Der dritte Artikel ist im Vergleich zum Nicaenum ausf�hrlicher formuliert, hinf�hrend
zum Taufbefehl aus Mt 28,19, um damit zu Ausf�hrungen �ber die hypostatische Dreiheit von
Vater, Sohn und heiligem Geist �berzuleiten.}, der den
Glaubenden zur Tr�stung, Heiligung und Vervollkommnung gegeben ist, wie auch unser Herr Jesus
Christus den J�ngern befahl, als er sagte: ">Geht, lehrt alle V�lker und tauft sie auf den
Namen des Vaters, des Sohnes und des heiligen Geistes"<, \kapR{6}also eines Vaters, der
wahrhaft Vater ist, und eines Sohnes, der wahrhaft Sohn ist, und des heiligen Geistes, der
wahrhaft heiliger Geist ist, deren Namen nicht einfach so oder ohne Sinn bestehen,
sondern genau die eigene Hypostase, den Rang und die Herrlichkeit eines jeden der Benannten bezeichnen,
so da� sie der Hypostase nach drei, der Eintracht nach aber eins sind. 
\pend
\pstart
\kapR{7}Da wir also diesen Glauben haben und zwar sowohl von Anfang an als auch bis zum
Ende, verurteilen wir vor Gott und Christus jeden h�retischen Irrglauben. \kapR{8}Und wenn
jemand\footnoteA{Es folgen Anathematismen, in der Formulierung anders als im Nicaenum, aber scheinbar ebenfalls
gegen die Positionen gerichtet, die Arius zugeschrieben worden waren; vgl. Dok. \ref{ch:24} = Urk. 24 (\editioncite[52,2--5]{Opitz:Urk}); Dok. \ref{ch:6} = Urk. 6,2 (\editioncite[12,9~f.]{Opitz:Urk}).} gegen den gesunden und rechten Glauben der Schriften lehrt und sagt, da� es eine Zeitspanne, einen Zeitpunkt oder ein
Zeitalter gebe oder gegeben habe vor der Zeugung des Sohnes, sei er verdammt. Und wenn
jemand den Sohn ein Geschaffenes wie eines der Geschaffenen oder ein Gesch�pf wie eines der
Gesch�pfe oder ein Werk wie eines der Werke nennt und nicht so wie die g�ttlichen Schriften
jede der gerade benannten Aussagen �berliefert, oder wenn er anderes als das, was wir �berliefert bekommen
haben, lehrt oder predigt, sei er verdammt. \kapR{9}Denn wir glauben und folgen
wahrlich und ehrfurchtsvoll allem, was aus den g�ttlichen Schriften durch die
Propheten und die Apostel �berliefert ist. 
\pend
\endnumbering
\end{translatio}
\end{Rightside}
\Columns
\end{pairs}
\clearpage
\autor{Das Regest bei Sozomenus}
\begin{pairs}
\selectlanguage{polutonikogreek}
\begin{Leftside}
\pstart
\hskip -1.4em\edtext{\abb{}}{\xxref{sozreg1}{sozreg2}\Cfootnote{\hskip -1em\latintext Soz.
(BC=b)}}\specialindex{quellen}{section}{Sozomenus!h.\,e.!IV 22,6} %% Bezeugungsleiste
\kap{2,1}metamelhj'entec\edlabel{sozreg1} d`e <wc >'eoiken >ep`i ta'uth| t~h| graf~h|, p'alin <et'eran
par`a ta'uthn >ex'ejento, t`a m`en >'alla, <wc o>~imai, sun'a|dousan t~w| d'ogmati t~wn
>en Nika'ia|\edindex[synoden]{Nicaea!a. 325} sunelj'ontwn, e>i m'h tic >emo`i
>'adhloc di'anoia to~ic <rhto~ic >afan~wc
>'egkeitai; o>uk o>~ida d`e >anj> <'otou ((<omoo'usion)) e>ipe~in t`on u<i`on paraiths'amenoi
((>'atrept'on te ka`i >anallo'iwton t~hc je'othtoc)) >apef'hnanto, ((o>us'iac te ka`i boul~hc
ka`i dun'amewc ka`i d'oxhc >apar'allakton e>ik'ona ka`i prwt'otokon p'ashc kt'isewc.))
\kap{2}\edtext{>'elegon}{\Cfootnote{\dt{inc. T}}} d`e ta'uthn t`hn p'istin
\edtext{<ol'ografon e<urhk'enai}{\Dfootnote{e~>inai ka`i e<urek'enai <ol'ografon
\latintext T}} Loukiano~u\edindex[namen]{Lucian von Antiochien} to~u
>en Nikomhde'ia|\edindex[namen]{Nikomedien} martur'hsantoc, >andr`oc t'a te
>'alla e>udokimwt'atou ka`i t`ac <ier`ac graf`ac e>ic >'akron >hkribwk'otoc; p'oteron d`e
>alhj~wc ta~uta >'efasan >`h t`hn >id'ian graf`hn semnopoio~untec t~w| >axi'wmati to~u
m'arturoc, l'egein o>uk >'eqw.\edlabel{sozreg2}
\pend
% \endnumbering
\end{Leftside}
\begin{Rightside}
\begin{translatio}
\beginnumbering
\pstart
% \autor{Regest}
\noindent\kapR{III 5,8}Allem Anschein nach aber bereuten sie diesen Text und formulierten neben ihm 
wiederum einen weiteren, der im �brigen aber, wie ich meine, mit der Glaubenserkl�rung der
in Nicaea Versammelten �bereinstimmt, es sei denn, es liegt ein mir verborgener Sinn hinter
den Worten versteckt. Ich wei� aber nicht, warum sie es vermieden, den Sohn
">wesenseins"< zu nennen und ihn stattdessen als ">unwandelbar und unver�nderlich der
Gottheit nach"< bezeichneten und als ">unver�nderliches Abbild des Wesens, des Willens,
der Kraft und der Herrlichkeit und Erstgeborener aller Sch�pfung."< \kapR{9}Sie wiesen
darauf hin, da� sie diese Glaubenserkl�rung wortw�rtlich bei Lucianus\footnoteA{Lucianus,
Presbyter aus Antiochien, erlitt am 7.1.312 in Nikomedien den M�rtyrertod (Vita, GCS
Philost., Anh. VI; Eus., h.\,e. VIII 13,2; IX 6,3). Durch die M�rtyrerverehrung in
Drepanon/Helenopolis, gef�rdert besonders durch Eusebius von Nikomedien (vgl. Dok. \ref{ch:1} = Urk. 1,5, wo
Eusebius als ">\griech{sulloukianist'a}"< bezeichnet wird), gewann Lucianus an Bedeutung, die
auch sp�ter in hom�ischen Kreisen fortbestand. Die Zuschreibung dieser theologischen Erkl�rung
an Lucianus wird erstmals fa�bar, als die Hom�usianer auf der Doppelsynode
von Rimini/Seleucia darauf zur�ckgriffen, da Athanasius und Hilarius sie noch
nicht zu kennen scheinen, Sozomenus Entsprechendes aber wohl in der Synodalaktensammlung
des Sabinus vorfand. Aus der antiochenischen Erkl�rung kann deswegen kein Glaubens-
oder Taufbekenntnis des Lucianus rekonstruiert werden.}, dem M�rtyrer aus Nikomedien,
gefunden h�tten, einem auch im �brigen ber�hmten Mann, einem Meister der scharfsinnigen
Analyse der heiligen Schriften. Ob sie das aber wahrheitsgem�� sagten oder nur ihren
eigenen Text mit dem Ruhm des M�rtyrers ehren wollten, kann ich nicht sagen. 
\pend
\endnumbering
\end{translatio}
\end{Rightside}
\Columns
\end{pairs}
%%% Local Variables: 
%%% mode: latex
%%% TeX-master: "dokumente_master"
%%% End: 
%%%%%%%%%%%%%% Antiochenum I %%%%%%%%%%%%
\selectlanguage{german}
% \cleartooddpage
\section[Fragment eines Rundbriefes der Synode von Antiochien des Jahres 341 (1. antiochenische Formel)][Fragment eines Rundbriefes der Synode von Antiochien des Jahres 341]{Fragment eines Rundbriefes der Synode von Antiochien des Jahres 341\\(1. antiochenische Formel)}
\label{sec:AntI}
\begin{praefatio}
  \begin{description}
  \item[Anfang 341]Zur Datierung vgl. die Bemerkungen zu
    Dok. \ref{sec:BerichteAntiochien341}.  Dieses Textst�ck geh�rt zu
    einem Rundbrief, den die Synode an alle Bisch�fe verschickt hat,
    wie sowohl Socr., h.\,e. II 10,9 (\griech{ta~uta m`en >en t~h|
      pr'wth| >epistol~h| gr'ayantec to~ic kat`a p'olin >'epempon}
    [\editioncite[100,13]{Hansen:Socr}]) als auch Soz., h.\,e. III 5,5 (\griech{gr'ammata
      d`e diep'emyanto to~ic kat`a p'olin >episk'opoic} [\editioncite[196,10~f.]{Hansen:Soz}]) berichten. Hauptthema des Fragments ist die Abwehr
    des Vorwurfs, Arianer\index[namen]{Arianer} zu sein. Genau dieses
    hatte Athanasius\index[namen]{Athanasius!Bischof von Alexandrien}
    den �stlichen Bisch�fen unterstellt, zuletzt in seiner
    ep.\,encycl. (2,2~f. [\editioncite[171,3--7]{Opitz1935}]; 7,1 [\editioncite[176,11--13]{Opitz1935}]) von 339; so
    k�nnte dieser Rundbrief aus Antiochien\index[namen]{Antiochien}
    auch als Reaktion und Gegenst�ck zur \textit{Epistula encyclica} des
    Athanasius zu verstehen sein. Wahrscheinlich wurde dieser Brief
    auch den r�mischen Gesandten in
    Antiochien\index[namen]{Antiochien} zus�tzlich zur
    Antwort der Synode auf die Aufforderung des Julius mitgegeben,\index[namen]{Julius!Bischof
      von Rom} zu einer r�mischen Synode
    (vgl. Dok. \ref{sec:BriefJulius}und
    \ref{sec:BriefSynode341}) zu erscheinen. Nach Dok. \ref{sec:BriefJuliusII},22
    reisten diese Gesandten Januar 341 nach Rom\index[namen]{Rom} zur�ck. Die
    theologische Erkl�rung erweist sich z.\,T. als Kurzfassung der sogenannten zweiten
    antiochenischen Formel.
  \item[�berlieferung]Das Fragment des Rundbriefes �berliefern
    Athanasius und Socrates; Sozomenus bietet h.\,e. III 5,5--7 nur
    ein Regest. Die Quelle der Kirchenhistoriker d�rfte �ber
    Athanasius hinaus Sabinus von Heraclea sein, da sich nur bei Socr. und Soz. der Hinweis
    findet, da� es sich um einen Rundbrief handelt. Ferner k�nnte man
    aus dem Referat des Sozomenus (h.\,e. III 5,5~f.:
    \foreignlanguage{polutonikogreek}{piste'uein d`e sf~ac kat`a t`hn
      >ex >arq~hc paradoje~isan p'istin. e>~inai d`e ta'uthn <`hn
      <up'etaxan t~h| a>ut~wn >epistol~h|}\dots
    [\editioncite[106,13~f.]{Hansen:Soz}]) schlie�en, da� die eigentliche
    Glaubenserkl�rung der Synode (Dok. \ref{sec:AntII}), die bei Athanasius und Socrates ohne
    Unterbrechung gleich folgt (ab Z. \lineref{Par3}), ein Anhang zum
    Brief gewesen ist und erst von Athanasius und ihm folgend von
    Socrates an diese Stelle eingef�gt wurde. Wie in
    Dok. \ref{sec:AntII} ist auch hier die Fassung des Athanasius
    vorzuziehen.
  \item[Fundstelle]Ath., syn. 22,3--7 (\editioncite[248,30--249,8]{Opitz1935}); Socr.,
    h.\,e. II 10,4--8 (\editioncite[99,20--100,12]{Hansen:Socr})
  \end{description}
\end{praefatio}
\begin{pairs}
\selectlanguage{polutonikogreek}
\begin{Leftside}
% \beginnumbering
\pstart
\hskip -1.2em\edtext{\abb{}}{\killnumber\Cfootnote{\hskip -1em\latintext Ath.(BKPO R) Socr. (MF=b
AT)}}\specialindex{quellen}{section}{Athanasius!syn.!22,3--7}\specialindex{quellen}{section}{Socrates!h.\,e.!II 10,4--8}
\kap{1}\looseness=-1\ <Hme~ic o>'ute >ak'oloujoi >Are'iou\edindex[namen]{Arius!Presbyter in Alexandrien}
geg'onamen -- p~wc
g`ar >ep'iskopoi >'ontec \edtext{>akoloujo~umen} {\Dfootnote{>akolouj'hsomen \latintext
Socr.}} presbut'erw|? -- o>'ute >'allhn tin`a p'istin par`a t`hn >ex >arq~hc
\edtext{paradoje~isan}{\Dfootnote{>ekteje~isan \latintext Socr.(bA)}}
\edtext{>edex'ameja}{\Dfootnote{>exedex'ameja \latintext Socr.(T)}}, 
\kap{2}\edtext{>all`a
ka`i a>uto`i}{\Dfootnote{>all'' a>uto`i \latintext Socr.(T) \greektext >all`a ka`i <hme~ic
\latintext Socr.(bA)}} >exetasta`i ka`i dokimasta`i t~hc p'istewc a>uto~u gen'omenoi
m~allon a>ut`on proshk'ameja \edtext{>'hper}{\Dfootnote{e>'iper \latintext Socr.(T)}}
>hkolouj'hsamen; \edtext{gn'wsesje d`e}{\Dfootnote{ka`i gn'wsesje \latintext Socr.(bA)}} 
>ap`o t~wn legom'enwn.
\pend
\pstart
\kap{3}\edlabel{Par3}memaj'hkamen g`ar \edtext{\abb{>ex >arq~hc}}{\Dfootnote{\latintext >
Socr.(T)}} e>ic \edtext{\abb{<'ena}}{\Afootnote{\latintext vgl. 1Cor 8,6; Eph
4,6}}\edindex[bibel]{Korinther I!8,6|textit}\edindex[bibel]{Epheser!4,6|textit}
\edtext{\abb{je`on
t`on}}{\Dfootnote {\latintext > Socr.(T)}} t~wn <'olwn je`on piste'uein, t`on
\edtext{\abb{p'antwn}}{\Dfootnote{+ t~wn \latintext Socr.(T)}} noht~wn te ka`i a>isjht~wn
dhmiourg'on te ka`i pronoht'hn, 
\kap{4}ka`i e>ic <'ena u<i`on to~u jeo~u
\edtext{\abb{monogen~h}}{\Afootnote{\latintext Io 1,14.18; 3,16; 1Io
4,9}}\edindex[bibel]{Johannes!1,14}\edindex[bibel]{Johannes!1,18}\edindex[bibel]{Johannes!3,16}\edindex[bibel]{Johannes I!4,9}, pr`o
\edtext{\abb{p'antwn}}{\Dfootnote{+ t~wn \latintext
Socr.}} \edtext{\abb{a>i'wnwn}}{\Afootnote{\latintext vgl. 1Cor
2,7}}\edindex[bibel]{Korinther I!2,7|textit} <up'arqonta ka`i sun'onta t~w| gegennhk'oti
a>ut`on
patr'i, \edtext{\abb{di> o~<u \edtext{\abb{ka`i}}{\Dfootnote{\latintext > Socr.(T)}} t`a
p'anta >eg'eneto}}{\Afootnote{\latintext vgl. Io 1,3; 1Cor 8,6; Col 1,16; Hebr
1,2}}\edindex[bibel]{Johannes!1,3|textit}\edindex[bibel]{Korinther I!8,6|textit}\edindex[bibel]{Kolosser!1,16|textit}\edindex[bibel]{Hebraeer!1,2|textit}, t`a
\edtext{\abb{te}}{\Dfootnote{\latintext > Socr.(bA)}} \edtext{\abb{<orat`a ka`i t`a
>a'orata}}{\Afootnote{\latintext Col 1,16}}\edindex[bibel]{Kolosser!1,16}, t`on ka`i
\edtext{\abb{>ep> \edtext{\abb{>esq'atwn}}{\Dfootnote{+ t~wn \latintext Socr.}}
<hmer~wn}}{\Afootnote{\latintext vgl. Hebr 1,2}}\edindex[bibel]{Hebraeer!1,2|textit} kat''
e>udok'ian to~u patr`oc katelj'onta ka`i \edtext{\abb{s'arka}}{\Afootnote{\latintext vgl.
Io 1,14; Rom 1,3;
8,3}}\edindex[bibel]{Johannes!1,14|textit}\edindex[bibel]{Roemer!1,3|textit}\edindex[bibel]{Roemer!8,3|textit} >ek
\edtext{\abb{t~hc}}{\Dfootnote{+ <ag'iac \latintext Socr.(bA)}}
\edtext{\abb{parj'enou}}{\Afootnote{\latintext vgl. Mt
1,23; Lc 1,27.34~f.}}\edindex[bibel]{Matthaeus!1,23|textit}\edindex[bibel]{Lukas!1,27|textit}
\edindex[bibel]{Lukas!1,34~f.|textit}
>aneilhf'ota ka`i \edlabel{jo434-1}p~asan t`hn patrik`hn a>uto~u
\edtext{bo'ulhsin}{\Dfootnote{boul`hn \latintext Socr.}}
\edtext{\abb{sunekpeplhrwk'ota}}{\Afootnote{\latintext vgl. Io
4,34}}\edindex[bibel]{Johannes!4,34|textit}\edlabel{jo434-2},
\edtext{\abb{peponj'enai}}{\Afootnote{\latintext vgl. 1Petr 2,21; 3,18; Act
3,18}}\edindex[bibel]{Petrus I!2,21|textit}\edindex[bibel]{Petrus I!3,18|textit}\edindex[bibel]{Apostelgeschichte!3,18|textit} ka`i
\edtext{\abb{>eghg'erjai}}{\Afootnote{\latintext vgl. Mt 16,21par; 1Cor 15,4; Rom 4,24; Eph
1,20}}\edindex[bibel]{Matthaeus!16,21par|textit}\edindex[bibel]{Korinther I!15,4|textit}\edindex[bibel]{Roemer!4,24|textit}\edindex[bibel]{Epheser!1,20|textit} ka`i
\edtext{\abb{e>ic o>urano`uc
>anelhluj'enai}}{\Afootnote{\latintext vgl. Mc 16,19; Act 1,2; 1Petr
3,22}}\edindex[bibel]{Markus!16,19|textit}\edindex[bibel]{Apostelgeschichte!1,2|textit}
\edindex[bibel]{Petrus I!3,22|textit} ka`i \edtext{\abb{>en dexi\aci\  to~u patr`oc
kaj'ezesjai}}{\Afootnote{\latintext
vgl. Ps 110,1; Eph 1,20; Col 3,1; 1Petr 3,22; Hebr 1,3; Mc
16,19}}\edindex[bibel]{Psalmen!110,1|textit}\edindex[bibel]{Epheser!1,20|textit}\edindex[bibel]{Kolosser!3,1|textit}\edindex[bibel]{Petrus I!3,22|textit}\edindex[bibel]{Hebraeer!1,3|textit}\edindex[bibel]{Markus!16,19|textit} ka`i
\edtext{\abb{p'alin}}{\Dfootnote{\latintext > Socr.(bA)}} >erq'omenon
\edtext{\abb{kr~inai z~wntac ka`i nekro`uc}}{\Afootnote{\latintext 2Tim 4,1; 1Petr 4,5;
vgl. Io 5,22; Act 10,42}}\edindex[bibel]{Timotheus II!4,1}\edindex[bibel]{Petrus I!4,5}\edindex[bibel]{Johannes!5,22|textit}\edindex[bibel]{Apostelgeschichte!10,42|textit}, ka`i
diam'enonta basil'ea ka`i je`on e>ic to`uc a>i~wnac.
\pend
\pstart
\kap{5}piste'uomen \edtext{\abb{d`e}}{\Dfootnote{\latintext > Socr.}} ka`i e>ic t`o
<'agion pne~uma; e>i \edtext{\abb{d`e}}{\Dfootnote{\latintext > Socr.(T)}} de~i
prosje~inai piste'uomen ka`i per`i sark`oc >anast'asewc ka`i \edtext{\abb{zw~hc
a>iwn'iou}}{\Afootnote{\latintext vgl. Gal 6,8; 1Tim 1,16; Jud
21}}\edindex[bibel]{Galater!6,8|textit}\edindex[bibel]{Timotheus I!1,16|textit}\edindex[bibel]{Judas!21|textit}.
\pend
% \endnumbering
\end{Leftside}
\begin{Rightside}
\begin{translatio}
\beginnumbering
\pstart
\noindent\kapR{1}Wir sind weder Gefolgsleute des Arius geworden~-- denn wie k�nnten wir als
Bisch�fe einem Presbyter folgen?~-- noch haben wir einen anderen Glauben als den angenommen, der von
Anfang an �berliefert ist, 
\kapR{2}sondern es haben, da wir auch selbst Pr�fer und Gutachter
seines Glaubens gewesen sind, vielmehr wir ihn zugelassen,\footnoteA{Das bezieht sich
zur�ck auf die Rehabilitierung des Arius, vgl. Dok. \ref{ch:29} = Urk. 29; Dok. \ref{ch:30} = Urk. 30; Dok. \ref{ch:32} = Urk. 32; Dok. \ref{ch:Arius} und die einleitenden Bemerkungen zur Chronologie.} als da� wir ihm
gefolgt sind; ihr werdet es aber aus dem Gesagten erkennen.
\pend
\pstart
\kapR{3}Denn wir haben von Anfang an gelernt, an einen Gott, den Gott des Alls\footnoteA{Die
sonst �blichen Attribute ">Vater"<, ">Allm�chtiger"< fehlen.}, zu glauben, den Sch�pfer
und Bewahrer\footnoteA{Diese beiden Attribute stehen auch in der sog. 2. antiochenischen Formel (Dok. \ref{sec:AntII},1,1, dort ist
zus�tzlich die �bliche Beschreibung \griech{poiht'hn} eingef�gt).} aller vorstellbaren und wahrnehmbaren Dinge,
\kapR{4}und an einen eingeborenen Sohn Gottes, der vor allen Zeiten existiert und mit
seinem Erzeuger, dem Vater, zusammen ist\footnoteA{Eine Kurzfassung der antimarkellischen
Aussagen, wie sie in der sog. 2. antiochenischen Formel (Dok. \ref{sec:AntII},1,6) ausf�hrlich angeh�ngt werden, die so nat�rlich auch eine Abgrenzung zum Arianismus bedeuten; vgl. auch den entsprechenden Passus bei
Theophronius \griech{ka`i >'onta pr`oc t`on je`on >en <upost'asei} in Dok. \ref{sec:AntIII},3.},
durch den auch alles wurde, das Sichtbare und das Unsichtbare, der in den letzten Tagen
nach dem Willen des Vaters herabkam und Fleisch aus der Jungfrau annahm und den ganzen
v�terlichen Willen erf�llt hat,\footnoteA{Vgl. Dok. \ref{sec:AntII},4,1,4
(\refpassage{jo638-1}{jo638-2}).} der gelitten hat, auferweckt
worden ist, in den Himmel aufgestiegen ist, zur Rechten des Vaters sitzt und wieder kommt, um die Lebenden und die Toten zu richten, und der Herrscher und Gott auf ewig bleibt.
\pend
\pstart
\kapR{5}Wir glauben aber auch an den heiligen Geist; wenn aber noch etwas hinzuzuf�gen
ist, glauben wir auch an die Auferstehung des Fleisches und das ewige
Leben\footnoteA{Diese Aussagen, hier etwas salopp nachgereicht, finden sich nur im
Bekenntnis des Markell in seinem Brief (Dok. \ref{sec:MarkellJulius},11) und in dem
Bekenntnis, das Arius zu seiner Rehabilitierung vorlegte (Dok. \ref{ch:30} = Urk. 30,3).}.
\pend
\endnumbering
\end{translatio}
\end{Rightside}
\Columns
\end{pairs}
% \thispagestyle{empty}
\selectlanguage{german}
% \clearpage
%% Erstellt von uh
%% �nderungen:
%%%% 13.5.2004 Layout-Korrekturen (avs) %%%%
% \cleartooddpage
\section{Berichte �ber einen Brief der Synode von Antiochien im Jahr 341 an Julius von Rom}
% \label{sec:41.5}
\label{sec:BriefSynode341}
\begin{praefatio}
  \begin{description}
  \item[Anfang 341]Nach Dok. \ref{sec:BriefJuliusII},22 wurden die
    r�mischen Gesandten gezwungen, bis Januar (341) in
    Antiochien\index[namen]{Antiochien} zu bleiben (zum Datum
    vgl. Einleitung zu Dok. \ref{sec:BerichteAntiochien341}). Diesen
    Gesandten wurde eine Antwort an den r�mischen Bischof mitgegeben,
    wie sich aus dem anschlie�enden Schreiben des
    Julius\index[namen]{Julius!Bischof von Rom}
    (Dok. \ref{sec:BriefJuliusII}) ergibt. Die dort genannten
    Adressaten Dianius\index[namen]{Dianius!Bischof von Caesarea},
    Flacillus\index[namen]{Flacillus!Bischof von Antiochien},
    Narcissus\index[namen]{Narcissus!Bischof von Neronias},
    Eusebius\index[namen]{Eusebius!Bischof von Nikomedien},
    Maris\index[namen]{Maris!Bischof von Chalcedon},
    Macedonius\index[namen]{Macedonius!Bischof von Mopsuestia} sind
    hier als Absender anzunehmen. Dianius\index[namen]{Dianius!Bischof
      von Caesarea}, Flacillus\index[namen]{Flacillus!Bischof von
      Antiochien}, Eusebius\index[namen]{Eusebius!Bischof von
      Nikomedien} und Theodorus\index[namen]{Theodorus!Bischof von
      Heraclea} werden neben anderen als wichtige Teilnehmer der
    Synode von Antiochien\index[synoden]{Antiochien!a. 341} auch in
    den Berichten bei den Kirchenhistorikern Socrates und Sozomenus
    erw�hnt (vgl. Dok.  \ref{sec:BerichteAntiochien341}).
  \item[�berlieferung]Es gibt Referate �ber ein Schriftst�ck der
    antiochenischen Synode an Julius von
    Rom\index[namen]{Julius!Bischof von Rom} bei den
    Kirchenhistorikern Socrates (h.\,e. II 15,5~f.)
    und Sozomenus (h.\,e. III 8,4--8). Ihnen
    d�rfte dieser Text aus der Sammlung des Sabinus von
    Heraclea\index[namen]{Sabinus von Heraclea} vorgelegen haben, wie
    Socrates berichtet: ">Sabinus\index[namen]{Sabinus von
      Heraclea} dagegen, Anh�nger der H�resie des
    Macedonius\index[namen]{Macedonius!Bischof von Konstantinopel},
    den ich bereits oben erw�hnt habe, bietet die Briefe des Julius
    \index[namen]{Julius!Bischof von Rom}in seiner Synodalsammlung
    nicht, obwohl er den Brief der Antiochener an
    Julius\index[namen]{Julius!Bischof von Rom} nicht ausl��t."<
    (\griech{Sab~inoc m'entoi <o t~hc Makedon'iou a<ir'esewc, o~<u
      ka`i >'hdh pr'oteron >emnhmone'usamen, t`ac par`a >Ioul'iou
      >epistol`ac >en t~h| Sunagwg~h| t~wn sunodik~wn o>uk >en'ejhken
      ka'itoi t`hn par`a t~wn >en >Antioqeia| pr`oc >Io'ulion o>u
      paral'ipwn.} Socr., h.\,e. II 17,10 [\editioncite[110,16--19]{Hansen:Socr}]) Aus der
    Antwort des Julius\index[namen]{Julius!Bischof von Rom} auf diesen
    Brief (Dok. \ref{sec:BriefJuliusII}; besonders � 11; 20--24; 40; 48;
    56; 58; 62~f.)  lassen sich diese Regesten best�tigen. Da Soz. in
    seinem Referat (� 2,5) berichtet, da� in diesem Brief selbst �ber
    dogmatische Dinge nicht verhandelt wurde, ist davon auszugehen,
    da� die sogenannte erste antiochenische Formel (Dok. \ref{sec:AntI}) nicht
    direkt zu diesem Brief geh�rte, wurde aber eventuell den r�mischen
    Gesandten gesondert mitgegeben.
  \item[Fundstelle]\refpassage{socrber1}{socrber2} Socr., h.\,e. II
    15,5~f. (\editioncite[106,4--11]{Hansen:Socr}); \refpassage{sozber1}{sozber2} Soz.,
    h.\,e. III 8,4--8 (\editioncite[111,4--26]{Hansen:Soz})
  \end{description}
\end{praefatio}
\autor{Der Bericht des Socrates}
\begin{pairs}
\selectlanguage{polutonikogreek}
\begin{Leftside}
% \beginnumbering
\pstart
\hskip -1.3em\edtext{\abb{}}{\xxref{socrber1}{socrber2}\Cfootnote{\latintext Socr. (MF=b AT
Arm.)}}\specialindex{quellen}{section}{Socrates!h.\,e.!II 15,5~f.} %% Bezeugungsleiste
\kap{1}O<i\edlabel{socrber1} \edtext{\abb{d`e}}{\Dfootnote{\latintext > T}} dex'amenoi <'ubrin >epoio~unto
t`hn >ep'iplhxin, ka`i s'unodon >en t~h| >Antioqe'ia|\edindex[synoden]{Antiochien!a. 341}
khr'uxantec, \edtext{\abb{sunelj'ontec}}{\Dfootnote{+ te \latintext A}} >en a>ut~h|
\edtext{koin~h| gn'wmh| di> >epistol~hc sfodr'oteron}{\lemma{\abb{}}\Dfootnote{\responsio\
gn'wmh| koin~h| sfodr'oteron di> >epistol~hc \latintext bA}} >antegkalo~usi t~w|
>Ioul'iw|\edindex[namen]{Julius!Bischof von Rom}, dhlo~untec m`h de~in kanon'izesjai par>
a>uto~u, e>i bo'ulointo >exela'unein tin`ac \edtext{t~wn >ekklhsi~wn}{\Dfootnote{t~hc
>ekklhs'iac \latintext Arm.}}; \edtext{mhd`e}{\Dfootnote{mhd`en \latintext
M\textsuperscript{1}}} g`ar a>uto`uc >anteipe~in, <'ote
\edtext{\abb{Na'uaton}}{\Dfootnote{\latintext Cod.Nic. \greektext Nau'ation \latintext A
\griech{Nauatian~wn} M\textsuperscript{1} \griech{Nau'aton} M\textsuperscript{r}F
\griech{Nabatiano`uc} T \griech{Nauatiano`i} Arm.}}\edindex[namen]{Novatian!Presbyter in
Rom} t~hc >ekklhs'iac \edtext{>'hlaunon}{\Dfootnote{>'hlaunen \dt{M\textsuperscript{1} AT}
>el'ash| \dt{Arm. +} <o t~w >eke~ise kat`a kairo`uc t~hc >episkop~hc <hgo~umenoc \dt{T}}}.
Ta~uta m`en o<i t~hc <e'w|ac >ep'iskopoi \edtext{t~w| t~hc <R'wmhc
>episk'opw|}{\Dfootnote{t~w| >episk'opw| <R'wmhc \latintext bA > Arm.}}
>Ioul'iw|\edindex[namen]{Julius!Bischof von Rom} diep'emponto.\edlabel{socrber2}
\pend
\end{Leftside}
\begin{Rightside}
\begin{translatio}
\beginnumbering
\pstart
% \autor{Sokrates} 
\noindent\kapR{II 15,5}Die Empf�nger aber hielten den Tadel f�r eine Kr�nkung und riefen eine
Synode in Antiochien aus, und als sie dann auf ihr zusammengekommen waren, machten sie einm�tig 
wiederum Julius ziemlich heftig in einem Brief Vorw�rfe und zeigten auf, da� es nicht an ihm sei, dar�ber zu befinden, wenn sie Leute aus den Kirchen ausschlie�en wollten; denn auch sie selbst h�tten
nicht widersprochen, als man Novatian\footnoteA{\protect\label{fn:Novatian}Novatian,
r�mischer Presbyter, wurde im sogenannten Streit um die \textit{lapsi} wegen seiner h�rteren Haltung
zu Bu�ma�nahmen vom r�mischen Bischof und einer Synode exkommuniziert (Eus., h.\,e. VI 43;
Hier., vir.\,ill. 70). Vgl. Dok. \ref{sec:BriefJuliusII},20.} aus der Kirche ausschlo�.
\kapR{6}Dies also schickten die Bisch�fe des Ostens an Julius, den Bischof von Rom.
\pend
\endnumbering
\end{translatio}
\end{Rightside}
\Columns
\end{pairs}
\autor{Der Bericht des Sozomenus}
\begin{pairs}
\selectlanguage{polutonikogreek}
\begin{Leftside}
\pstart
\hskip -1.6em\edtext{\abb{}}{\xxref{sozber1}{sozber2}\Cfootnote{\latintext Soz. (BC =
b)}}\specialindex{quellen}{section}{Sozomenus!h.\,e.!III 8,4--8} %% Bezeugungsleiste
\kap{2,1}o<i\edlabel{sozber1} d`e >ep`i ta'utaic qalep~wc >'hnegkan ka`i sulleg'entec >en
>Antioqe'ia|\index[synoden]{Antiochien!a. 341}
>ant'egrayan \hbox{>Ioul'iw|}\edindex[namen]{Julius!Bischof von Rom} kekalliephm'enhn
tin`a ka`i dikanik~wc suntetagm'enhn >epistol'hn,
e>irwne'iac te poll~hc >an'aplewn ka`i >apeil~hc o>uk >amoiro~usan deinot'athc.
f'erein \kap{2}m`en \edtext{\abb{g`ar}}{\Dfootnote{+ par`a \latintext coni. Schwartz}}
p~asi filotim'ian t`hn <Rwma'iwn\edindex[namen]{Rom} >ekklhs'ian >en to~ic
gr'ammasin <wmol'ogoun, <wc
>apost'olwn frontist'hrion ka`i e>usebe'iac mhtr'opolin >ex >arq~hc gegenhm'enhn, e>i ka`i
>ek t~hc <'ew >ened'hmhsan \edtext{a>ut~h|}{\Dfootnote{>auto`i \dt{B}}} o<i to~u d'ogmatoc
e>ishghta'i. o>u par`a to~uto d`e t`a deutere~ia f'erein >hx'ioun, <'oti m`h meg'ejei >`h
pl'hjei \edtext{\abb{>ekklhsi~wn}}{\Dfootnote{\latintext coni. Primmer \greektext >ekklhs'iac \latintext b}} pleonekto~usin, <wc
>aret~h| ka`i proair'esei nik~wntec. 
\pend
\pstart
\kap{3}e>ic >egkl'hmata d`e prof'erontec >Ioul'iw|\edindex[namen]{Julius!Bischof von Rom}
t`o koinwn~hsai to~ic >amf`i t`on
>Ajan'asion\edindex[namen]{Athanasius!Bischof von Anazarba} >eqal'epainon <wc
<ubrism'enhc a>ut~wn t~hc sun'odou ka`i t~hc >apof'asewc
>anaireje'ishc; ka`i t`o gen'omenon <wc >'adikon ka`i >ekklhsiastiko~u jesmo~u >ap\aci don
di'eballon.
\kap{4}>ep`i to'utoic d`e <wd'ipwc memy'amenoi ka`i dein`a peponj'enai
martur'amenoi, deqom'enw| m`en >Ioul'iw|\edindex[namen]{Julius!Bischof von Rom} t`hn
kaja'iresin t~wn pr`oc
\edtext{a>ut~wn}{\Dfootnote{>aut`on \latintext b (corr. C\textsuperscript{1})}}
>elhlam'enwn ka`i t`hn kat'astasin t~wn >ant'' a>ut~wn qeirotonhj'entwn e>ir'hnhn ka`i
koinwn'ian >ephgg'ellonto, \edtext{>anjistam'enw|}{\Dfootnote{>anjistam'enwn \latintext
B}} d`e to~ic dedogm'enoic \edtext{t>anant'ia}{\Dfootnote{t`a >enant'ia \latintext C}}
prohg'oreusan; >epe`i ka`i to`uc pr`o a>ut~wn >an`a t`hn <'ew <ier'eac o>ud`en >anteipe~in
>isqur'izonto, <hn'ika \edtext{Nau'atoc}{\Dfootnote{Nauatian`oc \latintext b (corr.
R\textsuperscript{c})}}\edindex[namen]{Novatian!Presbyter in Rom} t~hc
<Rwma'iwn\edindex[namen]{Rom} >ekklhs'iac >hl'ajh. 
\pend
\pstart
\kap{5}per`i d`e t~wn pepragm'enwn par`a t`a d'oxanta to~ic >en
Nika'ia|\edindex[synoden]{Nicaea!a. 325} suneljo~usin
o>ud`en a>ut~w| >ant'egrayan, poll`ac m`en a>it'iac >'eqein e>ic para'ithsin >anagka'ian
t~wn gegenhm'enwn dhl'wsantec, >apologe~isjai d`e n~un <up`er to'utwn peritt`on e>ip'ontec
<wc <'apax <omo~u >ep`i p~asin >adike~in <uponohj'entec.\edlabel{sozber2}
\pend
% \endnumbering
\end{Leftside}
\begin{Rightside}
\begin{translatio}
\beginnumbering
\pstart
% \autor{Sozomenus} 
\noindent\kapR{III 8,4}Diese aber waren ungehalten dar�ber\footnoteA{Das hei�t �ber das Schreiben
des Julius von Rom, vgl. Dok. \ref{sec:BriefJulius}}, versammelten sich in
Antiochien und schrieben Julius einen wohlklingend und rechtskundig verfa�ten Brief
zur�ck\footnoteA{Es ging im Wesentlichen um das kirchenrechtliche Problem, welche
Kompetenzen welche Synode hat und wieweit synodale Urteile g�ltig oder revidierbar sind:
Hat eine r�mische Synode das Recht, das Urteil einer antiochenischen Synode aufzuheben?
Die Antiochener beharrten auf ihren gleichwertigen Rang und drohten Julius sogar mit
Exkommunikation, falls er die Abgesetzten aufnehme. Julius hielt daraufhin den Antiochenern
vor, es liege nicht in deren Machtbereich, Bisch�fe in Alexandrien ein- und
abzusetzen, au�erdem w�rden sie selbst durch die Wiederaufnahme von ">Arianer"< die
Beschl�sse der Synode von Nicaea 325 n.\,Chr. revidieren (Dok. \ref{sec:BriefJuliusII}). In
diesem Zusammenhang wird bereits eine starke Ost-West-Spannung deutlich, die sich auf der
Synode von Serdica verfestigt hat (vgl. Dok. \ref{ch:SerdicaEinl}).}, gespickt mit lauter
Doppeldeutigkeiten und keineswegs ohne heftigste Drohung. \kapR{5}Sie gaben zwar in ihrem
Schreiben zu, da� die Kirche von Rom von allen in Ehren gehalten werde, da sie von Anfang
an Studierzimmer der Apostel und Hauptstadt der Gottesfurcht gewesen sei, wenn auch
die Urheber des Glaubens aus dem Osten nach Rom gekommen seien. Sie forderten aber, nicht
dadurch an zweiter Stelle zu stehen, da� sie nicht in Gr��e oder Menge der Kirchen Rom
�berragten, da sie n�mlich durch Tugend und Gesinnung siegen w�rden. 
\pend
\pstart
\kapR{6}Sie machten aber auch Julius Vorw�rfe, da� er mit denen um Athanasius in
Gemeinschaft stehe, und beklagten sich, da� somit ihre Synode entehrt und ihr 
Beschlu� aufgehoben worden sei. Die Geschehnisse verd�chtigten sie als rechtlos und abweichend 
vom kirchlichen Gesetz. \kapR{7}Nachdem sie also derartige Vorw�rfe vorgebracht und auf
die grobe Mi�handlung hingewiesen hatten, boten sie Julius Frieden und Gemeinschaft an,
wenn er die Absetzung der von ihnen Vertriebenen und die Einsetzung der an ihrer Stelle
Gew�hlten �bernehme; wenn er den Beschl�ssen aber Widerstand entgegensetze, so k�ndigten
sie Gegenteiliges an. Denn auch die �stlichen Bisch�fe vor ihnen h�tten sich nicht dazu
erm�chtigt zu widersprechen, als Novatian\footnoteA{Vgl. oben Anm. \ref{fn:Novatian} 
auf S. \pageref{fn:Novatian}.} aus der Kirche von Rom ausgeschlossen wurde. 
\pend
\pstart
\kapR{8}�ber die Vorf�lle im Widerspruch zu den Beschl�ssen der in Nicaea Versammelten
schrieben sie ihm nichts zur�ck, nachdem sie er�ffnet hatten, da� sie viele Gr�nde f�r
eine erzwungene Ablehnung der Geschehnisse h�tten, und f�gten hinzu, es sei
�berfl�ssig, sich jetzt in diesen Sachen zu verteidigen, da ihnen unterstellt werde, immer
und in allem im Unrecht zu sein. 
\pend
\endnumbering
\end{translatio}
\end{Rightside}
\Columns
\end{pairs}
% \thispagestyle{empty}
\selectlanguage{german}
% \cleartooddpage
% \renewcommand*{\goalfraction}{.6}
%% Erstellt von uh
%% �nderungen:
%%%%
\section{Brief des Markell von Ancyra an Julius von Rom}
% \label{sec:41.6}
\label{sec:MarkellJulius}
\begin{praefatio}
  \begin{description}
  \item[Fr�hjahr 341]Nach seiner ersten Verurteilung vor 337
    (vgl. Dok.  \ref{ch:Konstantinopel336}) konnte
    Markell\index[namen]{Markell!Bischof von Ancyra} nach
    Konstantins\index[namen]{Konstantin, Kaiser} Tod auf seinen
    Bischofssitz Ancyra zur�ckkehren
    (vgl. Dok. \ref{sec:RundbriefSerdikaOst},10), wurde aber erneut
    vertrieben und floh um 339/340 nach Rom\index[namen]{Rom}, um seine theologische
    Rehabilitierung zu erreichen. Als jedoch nach langem Warten
    (Markell nennt ein Jahr und drei Monate) auf eine Delegation der
    Antiochener nichts geschah und sogar eine Absage eintraf
    (vgl. Dok.  \ref{sec:BriefSynode341}), verfa�te
    Markell\index[namen]{Markell!Bischof von Ancyra} diesen Brief, da er
    abreisen wollte (das Ziel ist unbekannt). Er bat
    Julius\index[namen]{Julius!Bischof von Rom}, seinen Brief dem
    Synodalschreiben beizuf�gen. Ein Abschnitt aus dem Brief des
    Julius\index[namen]{Julius!Bischof von Rom}
    (s. Dok. \ref{sec:BriefJuliusII},48~f.; vgl.  auch h.\,Ar. 6)
    best�tigt, da� auf der r�mischen Synode �ber
    Markell\index[namen]{Markell!Bischof von Ancyra} verhandelt wurde,
    er seinen Glauben erkl�ren mu�te und daraufhin rehabilitiert
    wurde. Es ist daher wenig wahrscheinlich, da�
    Markell\index[namen]{Markell!Bischof von Ancyra} diesen Brief noch
    vor der r�mischen Synode schrieb. Daher d�rfte dieser Text in das
    Fr�hjahr 341 zu datieren sein.
  \item[�berlieferung]Der Text des Briefes ist nur durch Epiphanius
    �berliefert; der Brief wird erw�hnt Ath., h.\,Ar. 6,2
    (\editioncite[186,4~f.]{Opitz1935}).
  \item[Fundstelle]Epiph., haer. 72,2,1--3,5 (\editioncite[256,13--259,3]{Epi}) [~= Marcell., fr. 129 (\editioncite[214,12--215,39]{Klostermann}; \editioncite[124--128]{Vinzent:Markell}]
  \end{description}
\end{praefatio}
\begin{pairs}
\selectlanguage{polutonikogreek}
\begin{Leftside}
% \beginnumbering
\pstart
\hskip -1.5em\edtext{\abb{}}{\killnumber\Cfootnote{\hskip -1em\latintext Epiph.
(J)}}\specialindex{quellen}{section}{Epiphanius!haer.!72,2,1--3,5} %% Bezeugungsleiste
\noindent\kap{1}T~w| makariwt'atw| sulleitourg~w| >Ioul'iw|\edindex[namen]{Julius!Bischof von Rom} M'arkelloc\edindex[namen]{Markell!Bischof von Ancyra} >en Qrist~w| qa'irein. 
\pend
\pstart
\kap{2}>Epeid'h tinec t~wn katagnwsj'entwn pr'oteron >ep`i t~w| m`h >orj~wc piste'uein,
o<`uc >eg`w >en t~h| kat`a N'ikaian\edindex[synoden]{Nicaea!a. 325} sun'odw| di'hlegxa,
kat'' >emo~u gr'ayai t~h|
jeosebe'ia| sou >et'olmhsan, <wc >`an >emo~u m`h >orj~wc \edtext{\abb{mhd`e}}{\Dfootnote{\dt{coni. Dindorf} m'hte \dt{J}}} >ekklhsiastik~wc
frono~untoc, t`o <eaut~wn >'egklhma e>ic >em`e metatej~hnai spoud'azontec,
\kap{3}to'utou <'eneken >anagka~ion <hghs'amhn >apant'hsac e>ic t`hn
<R'wmhn\edindex[namen]{Rom} <upomn~hsa'i
se, <'ina to`uc kat'' >emo~u gr'ayantac metaste'ilh| <up`er to~u
\edtext{\abb{>apant'hsantac}}{\Dfootnote{\dt{coni. Cornarius} >apant'hsantoc \dt{J}}}
a>uto`uc >ep'' >amfot'eroic >elegqj~hnai <up'' >emo~u, <'oti \edtext{\abb{te}}{\Dfootnote{\dt{del. Schwartz}}} ka`i <`a gegr'afasi kat''
>emo~u yeud~h >'onta tugq'anei ka`i \edtext{\abb{<'oti}}{\Dfootnote{\dt{del. Schwartz}}} >'eti ka`i n~un >epim'enousi t~h| <eaut~wn
prot'era| pl'anh| ka`i dein`a kat'a te t~wn to~u jeo~u >ekklhsi~wn ka`i <hm~wn t~wn
proest'wtwn a>ut~wn tetolm'hkasin.
>epe`i\kap{4} to'inun >apant~hsai o>uk >hboul'hjhsan, >aposte'ilant'oc sou presbut'erouc
pr`oc a>uto`uc ka`i ta~uta >emo~u >eniaut`on ka`i tre~ic <'olouc m~hnac >en t~h| <R'wmh|
pepoihk'otoc, >anagka~ion <hghs'amhn, m'ellwn >ente~ujen >exi'enai, >'eggraf'on soi t`hn
>emauto~u p'istin met`a p'ashc >alhje'iac t~h| >emauto~u qeir`i gr'ayac >epido~unai, <`hn
>'emajon >'ek te t~wn je'iwn graf~wn >edid'aqjhn, ka`i t~wn kak~wc <up'' a>ut~wn
legom'enwn <upomn~hsa'i se, <'ina gn~w|c o<~ic qr'wmenoi pr`oc >ap'athn t~wn >akou'ontwn
l'ogoic t`hn >al'hjeian kr'uptein bo'ulontai.
\pend
\pstart
\kap{5}fas`i g`ar m`h >'idion ka`i >alhjin`on l'ogon e>~inai to~u
\edtext{\abb{pantokr'atoroc}}{\Afootnote{\latintext vgl. Apc 1,8
u.�.}}\edindex[bibel]{Offenbarung!1,8|textit} jeo~u t`on u<i`on t`on \edtext{\abb{k'urion
<hm~wn
>Ihso~un Qrist'on}}{\Afootnote{\latintext vgl. Rom 1,4; 1Cor
8,6}}\edindex[bibel]{Roemer!1,4|textit}\edindex[bibel]{Korinther I!8,6|textit}, >all''
<'eteron a>uto~u
l'ogon e>~inai ka`i <et'eran \edtext{\abb{sof'ian ka`i d'unamin}}{\Afootnote{\latintext
vgl. 1Cor 1,24}}\edindex[bibel]{Korinther I!1,24|textit}. to~uton gen'omenon <up'' a>uto~u
>wnom'asjai \edtext{\abb{l'ogon}}{\Afootnote{\latintext vgl. Io
1,1}}\edindex[bibel]{Johannes!1,1|textit} ka`i sof'ian ka`i d'unamin, ka`i di`a t`o
o<'utwc
a>uto`uc frone~in >'allhn <up'ostasin diest~wsan to~u patr`oc e>~ina'i fasin. 
\kap{6}>'eti \edtext{m'entoi}{\Dfootnote{m`en \dt{coni. Rettberg} m`hn \dt{coni. Klostermann}}} ka`i
pro\"{u}p'arqein to~u u<io~u t`on pat'era di'' <~wn gr'afousin >apofa'inontai
\Ladd{\edtext{\abb{ka`i}}{\Dfootnote{\latintext add. Cornarius}}} m`h e>~inai
a>ut`on >alhj~wc u<i`on >ek to~u jeo~u; >all`a k>`an
\edtext{\abb{l'egwsin}}{\Dfootnote{\latintext coni. Dindorf \greektext l'egousin
\latintext J}} >ek to~u jeo~u, o<'utwc l'egousin <wc \edtext{\abb{ka`i t`a
p'anta}}{\Dfootnote{\latintext coni. Rettberg \greektext kat`a p'anta \latintext J}}. >'eti m`hn
ka`i <'oti >~hn pote <'ote o>uk >~hn l'egein tolm~wsi ka`i kt'isma a>ut`on ka`i po'ihma
e>~inai, dior'izontec a>ut`on >ap`o to~u patr'oc. to`uc o>~un ta~uta l'egontac
>allotr'iouc t~hc kajolik~hc >ekklhs'iac e>~inai pep'isteumai.
\pend
\pstart
\kap{7}piste'uw d'e, <ep'omenoc ta~ic je'iaic grafa~ic, <'oti \edtext{\abb{e<~ic
je`oc}}{\Afootnote{\latintext vgl. 1Cor 8,6; Eph 4,6}}
\edindex[bibel]{Korinther I!8,6|textit}\edindex[bibel]{Epheser!4,6|textit} ka`i <o to'utou
\edtext{\abb{monogen`hc}}{\Afootnote{\latintext vgl. Io 1,14.18; 3,16; 1Io
4,9}}\edindex[bibel]{Johannes!1,14|textit}\edindex[bibel]{Johannes!1,18|textit}
\edindex[bibel]{Johannes!3,16|textit}\edindex[bibel]{Johannes I!4,9|textit} u<i`oc l'ogoc,
<o >ae`i sunup'arqwn t~w| patr`i ka`i
mhdep'wpote >arq`hn to~u e>~inai >esqhk'wc, >alhj~wc >ek to~u jeo~u <up'arqwn, o>u
ktisje'ic, o>u poihje'ic, >all`a >ae`i >'wn, >ae`i sumbasile'uwn t~w| je~w| ka`i patr'i,
\edtext{((o<~u t~hc basile'iac)), kat`a t`hn to~u >apost'olou martur'ian, ((o>uk >'estai
t'eloc))}{\lemma{\abb{}}\Afootnote{\latintext Lc
1,33}}\edindex[bibel]{Lukas!1,33}. 
\kap{8}o<~utoc u<i'oc, \edtext{\abb{o<~utoc d'unamic,
o<~utoc sof'ia}}{\Afootnote{\latintext vgl. 1Cor 1,24}},
\edindex[bibel]{Korinther I!1,24|textit}
o<~utoc >'idioc ka`i >alhj`hc to~u jeo~u l'ogoc, <o \edtext{\abb{k'urioc <hm~wn >Ihso~uc
Qrist'oc}}{\Afootnote{\latintext vgl. Rom 1,4; 1Cor
8,6}}\edindex[bibel]{Roemer!1,4|textit}\edindex[bibel]{Korinther I!8,6|textit},
>adia'iretoc d'unamic to~u
jeo~u, di'' o<~u \edtext{t`a}{\lemma{\abb{t`a\textsuperscript{1}}}\Dfootnote{\latintext del. Klostermann}} p'anta t`a
gen'omena g'egone, kaj`wc t`o e>uagg'elion marture~i
\edtext{\abb{l'egon}}{\Dfootnote{\dt{coni. Holl} l'egwn \dt{J}}} \edtext{((>en >arq~h|
>~hn <o l'ogoc, ka`i <o l'ogoc >~hn pr`oc t`on je'on, ka`i je`oc >~hn <o l'ogoc. p'anta
di'' a>uto~u >eg'eneto, ka`i qwr`ic a>uto~u >eg'eneto o>ud`e <'en.))}{\lemma {\abb{}}\Afootnote{\latintext Io 1,1--3; vgl. 1Cor 8,6; Col 1,16; Hebr
1,2}}\edindex[bibel]{Johannes!1,1--3}
\edindex[bibel]{Korinther I!8,6|textit}
\edindex[bibel]{Kolosser!1,16|textit}\edindex[bibel]{Hebraeer!1,2|textit} 
\kap{9}o<~ut'oc >estin <o
l'ogoc, per`i o<~u ka`i Louk~ac <o e>uaggelist`hc marture~i l'egwn \edtext{((kaj`wc
par'edwkan <hm~in o<i >ap'' >arq~hc a>ut'optai ka`i <uphr'etai gen'omenoi to~u
l'ogou))}{\lemma{\abb{}}\Afootnote{\latintext Lc
1,2}}\edindex[bibel]{Lukas!1,2}; per`i to'utou ka`i Dau`id >'efh ((\edtext{>exhre'uxato <h
kard'ia mou l'ogon >agaj'on}{\lemma{\abb{}}\Afootnote{\latintext Ps 44,2}}\edindex[bibel]{Psalmen!44,2}.)) 
\kap{10}o<'utw
ka`i <o k'urioc <hm~wn >Ihso~uc Qrist`oc <hm~ac did'askei di`a to~u e>uaggel'iou, l'egwn
((\edtext{>eg`w >ek to~u patr`oc >ex~hljon ka`i <'hkw}{\lemma{\abb{}}\Afootnote{\latintext Io 8,42}}\edindex[bibel]{Johannes!8,42})); o<~utoc
\edtext{\abb{>ep'' >esq'atwn t~wn <hmer~wn}}{\Afootnote{\latintext Hebr
1,2}}\edindex[bibel]{Hebraeer!1,2} \edtext{\abb{katelj`wn}}{\Afootnote{\latintext vgl. Iac
3,15}}\edindex[bibel]{Jakobus!3,15|textit} di`a t`hn <hmet'eran swthr'ian ka`i
\edtext{\abb{>ek
t~hc parj'enou Mar'iac gennhje`ic}}{\Afootnote{\latintext vgl. Mt 1,23; Lc
1,27.34~f.}}\edindex[bibel]{Matthaeus!1,23|textit}\edindex[bibel]{Lukas!1,27|textit}
\edindex[bibel]{Lukas!1,34~f.|textit} \edtext{\abb{t`on >'anjrwpon
\edtext{>'elabe}{\Dfootnote{>an'elabe \latintext
coni. Klostermann}}}}{\Afootnote{\latintext vgl. Phil 2,7; auch Io 1,14; 1Cor
15,47}}.\edindex[bibel]{Philipper!2,7|textit}\edindex[bibel]{Johannes!1,14|textit}
\edindex[bibel]{Korinther I!15,47|textit}
\pend
\pstart
\kap{11}Piste'uw o>~un
e>ic je`on \edtext{\abb{pantokr'atora}}{\Afootnote{\latintext vgl. Apc 1,8
u.�.}}\edindex[bibel]{Offenbarung!1,8|textit} 
% \pend
% \pstart
ka`i e>ic Qrist`on >Ihso~un t`on u<i`on a>uto~u t`on
\edtext{\abb{monogen~h}}{\Afootnote{\latintext Io 1,14.18; 3,16; 1Io
4,9}},\edindex[bibel]{Johannes!1,14}\edindex[bibel]{Johannes!1,18}\edindex[bibel]{Johannes!3,16}
\edindex[bibel]{Johannes I!4,9} t`on \edtext{\abb{k'urion <hm~wn}}{\Afootnote{\latintext
vgl. Rom 1,4}}\edindex[bibel]{Roemer!1,4|textit}
% \pend
% \pstart
t`on gennhj'enta \edtext{\abb{>ek pne'umatoc <ag'iou}}{\Afootnote{\latintext vgl. Mt
1,20}}\edindex[bibel]{Matthaeus!1,20|textit} ka`i Mar'iac t~hc
\edtext{\abb{parj'enou}}{\Afootnote{\latintext vgl. Mt 1,23; Lc
1,27.34~f.}},\edindex[bibel]{Matthaeus!1,23|textit}\edindex[bibel]{Lukas!1,27|textit}
\edindex[bibel]{Lukas!1,34~f.|textit} 
% \pend
% \pstart
t`on \edtext{\abb{>ep`i Pont'iou Pil'atou}}{\Afootnote{\latintext 1Tim
6,13}}\edindex[bibel]{Timotheus I!6,13} \edtext{\abb{staurwj'enta}}{\Afootnote{\latintext
1Cor 1,23; 2,2.8; 2Cor 13,4; Gal 3,1}}\edindex[bibel]{Korinther I!1,23}\edindex[bibel]{Korinther I!2,2}\edindex[bibel]{Korinther I!2,8}\edindex[bibel]{Korinther II!13,4}\edindex[bibel]{Galater!3,1} ka`i
\edtext{\abb{taf'enta}}{\Afootnote{\latintext vgl.
Mt 27,60par}}\edindex[bibel]{Matthaeus!27,60par|textit} 
% \pend
% \pstart
ka`i \edtext{\abb{t~h| tr'ith| <hm'era| >anast'anta}}{\Afootnote{\latintext vgl. Lc 24,46;
1Cor 15,4; 1Thess 4,14; Eph 1,20; und Mt 17,9par;
20,19par}}\edindex[bibel]{Lukas!24,46|textit}\edindex[bibel]{Korinther I!15,4|textit}\edindex[bibel]{Thessalonicher I!4,14|textit}\edindex[bibel]{Epheser!1,20|textit}\edindex[bibel]{Matthaeus!17,9par|textit}\edindex[bibel]{Matthaeus!20,19par|textit} >ek t~wn
nekr~wn, 
% \pend
% \pstart
\edtext{\abb{>anab'anta e>ic to`uc o>urano`uc}}{\Afootnote{\latintext vgl. Io 3,13; Apc
11,12; Act 2,34}}\edindex[bibel]{Johannes!3,13|textit}\edindex[bibel]{Offenbarung!11,12|textit}\edindex[bibel]{Apostelgeschichte!2,34|textit} 
% \pend
% \pstart
ka`i \edtext{\abb{kaj'hmenon >en dexi\aci\ to~u patr'oc}}{\Afootnote{\latintext vgl. Ps
110,1; Eph 1,20; Col 3,1; 1Petr 3,22; Hebr 1,3; Mc
16,19}}\edindex[bibel]{Psalmen!110,1|textit}\edindex[bibel]{Epheser!1,20|textit}\edindex[bibel]{Kolosser!3,1|textit}\edindex[bibel]{Petrus I!3,22|textit}\edindex[bibel]{Hebraeer!1,3|textit}\edindex[bibel]{Markus!16,19|textit}, <'ojen
>'erqetai \edtext{\abb{kr'inein z~wntac ka`i nekro'uc}}{\Afootnote{\latintext 2Tim 4,1;
1Petr 4,5; vgl. Io 5,22; Act 10,42}}\edindex[bibel]{Timotheus II!4,1}\edindex[bibel]{Petrus I!4,5}\edindex[bibel]{Johannes!5,22|textit}\edindex[bibel]{Apostelgeschichte!10,42|textit};
% \pend
% \pstart
ka`i e>ic t`o <'agion pne~uma,
% \pend
% \pstart
\edtext{\abb{<ag'ian >ekklhs'ian}}{\Afootnote{\latintext Hermas 1,6; 3,4}},
\edtext{\abb{>'afesin <amarti~wn}}{\Afootnote{\latintext Mt 26,28; Mc 1,4; Lc 1,77; 3,3;
24,27}}\edindex[bibel]{Matthaeus!26,28}\edindex[bibel]{Markus!1,4}\edindex[bibel]{Lukas!1,77}\edindex[bibel]{Lukas!3,3}\edindex[bibel]{Lukas!24,27}, 
% \pend
% \pstart
sark`oc >an'astasin, \edtext{\abb{zw`hn a>i'wnion}}{\Afootnote{\latintext vgl. Gal 6,8;
1Tim 1,16; Jud 21}}\edindex[bibel]{Galater!6,8|textit}\edindex[bibel]{Timotheus I!1,16|textit}\edindex[bibel]{Judas!21|textit}. 
\pend
\pstart
\kap{12}>Adia'ireton e>~inai t`hn je'othta to~u patr`oc ka`i to~u u<io~u par`a t~wn je'iwn
memaj'hkamen graf~wn. e>i g'ar tic qwr'izei t`on u<i`on tout'esti t`on l'ogon to~u
pantokr'atoroc jeo~u, >an'agkh a>ut`on >`h d'uo jeo`uc e>~inai nom'izein, <'oper
>all'otrion t~hc je'iac didaskal'iac e>~inai nen'omistai, >`h t`on l'ogon m`h e>~inai
je`on <omologe~in, <'oper ka`i a>ut`o >all'otrion t~hc >orj~hc p'istewc \edtext{\abb{e>~inai}}{\Dfootnote{\dt{del. Rettberg}}} fa'inetai,
to~u e>uaggelisto~u l'egontoc ((\edtext{ka`i je`oc >~hn <o l'ogoc}{\lemma{\abb{}}\Afootnote{\latintext Io 1,1}}\edindex[bibel]{Johannes!1,1}.)) 
\kap{13}>eg`w d`e >akrib~wc mem'ajhka <'oti >adia'iretoc ka`i >aq'wrist'oc >estin
\edtext{<h d'unamic}{\Dfootnote{<~h| d'unamic \dt{coni. Rettberg}}} to~u
patr'oc, <o u<i'oc. a>ut`oc g`ar <o swt`hr <o k'urioc <hm~wn >Ihso~uc Qrist'oc fhsin
((\edtext{>en >emo`i <o pat`hr k>ag`w >en t~w| patr'i}{\lemma{\abb{}}\Afootnote{\latintext Io
10,38}}\edindex[bibel]{Johannes!10,38})) ka`i ((\edtext{>eg`w ka`i <o pat`hr <'en
>esmen}{\lemma{\abb{}}\Afootnote{\latintext Io
10,30}}\edindex[bibel]{Johannes!10,30})) ka`i ((\edtext{<o >em`e <ewrak`wc <e'wraken t`on
pat'era}{\lemma{\abb{}}\Afootnote{\latintext Io
14,9}}\edindex[bibel]{Johannes!14,9}.))
\kap{14}ta'uthn ka`i par`a t~wn je'iwn graf~wn e>ilhf`wc t`hn p'istin ka`i par`a t~wn
kat`a je`on prog'onwn didaqje`ic >'en te t~h| to~u jeo~u >ekklhs'ia| khr'uttw ka`i pr`oc
s`e n~un g'egrafa, t`o >ant'igrafon to'utou par'' >emaut~w| katasq'wn.
\kap{15}ka`i >axi~w t`o >ant'itup'on se to'utou t~h| pr`oc to`uc >episk'opouc >epistol~h|
>eggr'ayai, <'ina m'h tinec t~wn >akrib~wc m`h e>id'otwn <hm~ac
\edtext{k>ake'inouc}{\Dfootnote{k>ake'inoic \dt{coni. Petavius}}} to~ic <up'' a>ut~wn
grafe~isi pros'eqontec >apathj~wsin. >'errwsje. 
\pend
% \endnumbering
\end{Leftside}
\begin{Rightside}
\begin{translatio}
\beginnumbering
\pstart
\noindent\kapR{72,2,1}Den seligsten Mitdiener Julius gr��t Markell in Christus.
\pend
\pstart
Nachdem einige von denen, die fr�her schon verurteilt worden waren\footnoteA{Zur
Verurteilung z.\,B. des Eusebius von Caesarea vor der Synode von Nicaea vgl. Dok. \ref{ch:22}~= Urk. 22 und die Einleitung zur Chronologie; zur
Verurteilung des Eusebius von Nikomedien und des Theognis von Nicaea vgl. Dok. \ref{ch:27}~= Urk. 27; Dok. \ref{ch:31} = Urk. 31 und Dok.
\ref{sec:BriefJuliusII},32 Anm.}, weil sie nicht richtig glauben, und die ich auf der Synode
von Nicaea �berf�hrt hatte\footnoteA{�ber Markells Rolle auf der Synode von Nicaea gibt es nur
den allgemeinen Hinweis, er habe Arius und seinen Anh�ngern widersprochen (Brief des
Julius, Dok. \ref{sec:BriefJuliusII},14; 49).}, es gewagt haben, gegen mich an deine
Gottesfurcht zu schreiben, da� ich nicht richtig und nicht kirchlich\footnoteA{Vgl. den
Titel des zweiten Hauptwerkes des Eusebius von Caesarea gegen Markell ">�ber die kirchliche
Theologie"<.} d�chte, wobei sie sich beeilen, ihr eigenes Vergehen auf mich zu �bertragen,
\kapR{2,2}sah ich mich deswegen gezwungen, nach Rom zu reisen, um dich daran zu
erinnern, da� du die, die gegen mich geschrieben haben, rufen l��t, auf da� sie nach
ihrer Ankunft von mir in zweierlei Dingen �berf�hrt werden, n�mlich da� sich das, was sie gegen
mich schreiben, als L�ge erweist, und da� sie auch jetzt noch in ihrem fr�heren
Irrtum verbleiben und sich fortw�hrend f�rchterliche Dinge gegen die Kirchen Gottes und gegen
uns, deren Vorsteher, erlauben. \kapR{2,3}Da sie sich jedoch nicht einfinden wollten, nachdem
du Presbyter zu ihnen geschickt hattest und nachdem ich diese Angelegenheit ein Jahr lang
und ganze drei Monate in Rom vorangetrieben hatte\footnoteA{Vgl. die Absage in Dok.
\ref{sec:BriefSynode341} und die einleitenden Bemerkungen zu Dok. \ref{sec:BriefJulius}
und \ref{sec:BriefJuliusII}.}, sah ich mich, da ich von hier fortgehen m�chte, gezwungen,
dir eine schriftliche Fassung, die ich der Wahrheit gem�� eigenh�ndig aufgeschrieben habe,
meines eigenen Glaubens, den ich gelernt habe und �ber den ich aus den g�ttlichen Schriften belehrt wurde,
zu �bergeben und dich an ihre schlimmen Aussagen zu erinnern, damit du erkennst, welche Worte sie zur T�uschung der Zuh�rer verwenden und dadurch die Wahrheit verbergen wollen.
\pend
\pstart
\kapR{2,4}Sie sagen n�mlich, da� unser Herr Jesus Christus, der Sohn\footnoteA{Die Eusebianer
warfen Markell vor, er lehne die Bezeichnung ">Sohn"< f�r den Pr�existenten ab (Eus.,
e.\,th. II 14,11--15; 16; 19,4 u.\,�.), wogegen Markell sich hier verteidigt.} des allm�chtigen Gottes, nicht
sein eigenes und wahres Wort sei\footnoteA{Markell betont, da� der Pr�existente
das \textit{wahre Wort} Gottes ist, um die N�he und Einheit mit Gott zu beschreiben (vgl.
fr. 65 Seibt/Vinzent), Eusebius von Caesarea dagegen legt Wert darauf, da� der Sohn Gottes
\textit{wahrer Sohn} ist (vgl. z.\,B. Dok. \ref{ch:22}~= Urk. 22,5 und Dok. \ref{sec:AntII},1,6; auch Asterius,
fr. 60), um die selbst�ndige Existenz des Sohnes neben Gott herauszustellen.}, sondern ein
anderes\footnoteA{Markell kritisiert hiermit (und mit \griech{>'allhn <up'ostasin}) vor
allem die Lehre eines doppelten Logos (vgl. Marcell., fr. 117
Seibt/Vinzent), die er bei Asterius erkennt (vgl. Asterius, fr. 11, 12, 52, 54, 55,
56, 61, 62, auch 64, 65, 71, 74) und die auch Ps.-Ath., Ar. IV 1--4 diskutiert wird. Vgl. auch
\griech{>en <upost'asei} in der Erkl�rung des Theophronius (Dok. \ref{sec:AntIII},3).}
Wort und eine andere Weisheit und eine andere Kraft als er. Sie sagen, da� dieser, der durch ihn wurde, Wort, Weisheit und Kraft genannt werde, und dadurch, da� sie so denken, sagen sie, da� er eine andere Hypostase sei, die sich von der des Vaters unterscheide.\footnoteA{Hier wird deutlich, da� sich die theologische Diskussion seit Nicaea auf die Frage verschoben hat, ob der Vater und der Sohn eine oder
zwei Hypostasen seien (vgl. Dok. \ref{sec:AntII},1,6). Markell unterstellt mit seinen
anschlie�enden Bemerkungen (vgl. auch fr. 50; 51; 85; 86 Seibt/Vinzent), die Eusebianer lehrten damit dasselbe wie die ">Arianer"<.} 
\kapR{2,5}Ferner zeigen sie auch durch das, was sie schreiben, da� der Vater vor dem Sohn existiere und da� dieser nicht wahrer Sohn aus Gott sei; aber selbst wenn sie sagen ">aus Gott"<, so meinen sie es so, wie auch alles �brige aus Gott ist. Au�erdem wagen sie sogar zu sagen, da� es einmal war, als er nicht war, und da� er Gesch�pf und Werk sei, wobei sie ihn vom Vater abtrennen. Ich glaube also, da� die, die dieses
sagen, der katholischen Kirche fern stehen. 
\pend
\pstart
\kapR{2,6}Ich glaube\footnoteA{Bemerkenswert ist der Gebrauch der ersten Person singular
(wiederholt in 11 und auch bei Theophronius Dok. \ref{sec:AntIII}).} aber und folge dabei 
den g�ttlichen Schriften, da� Gott einer ist und da� sein eingeborener Sohn Wort
ist\footnoteA{Markell formuliert diesen Abschnitt (� 7--11), oft in Entsprechung zum
Vorherigen, um die Vorstellung von zwei Logoi abzuwehren, und weist diesem (s. die
h�ufigen Demonstrativpronomen) einen Logos, der ungetrennt bei Gott ist, die
Heilstatsachen zu.}, der immer zusammen mit dem Vater existiert\footnoteA{Vgl. im Dok.
\ref{sec:AntI},4: \griech{pr`o p'antwn a>i'wnwn <up'arqonta ka`i sun'onta t~w|
gegenhk'oti a>ut`on patr'i.}} und niemals einen Anfang des Seins hatte, da� er
wahrhaft aus Gott ist, da� er nicht geschaffen, nicht gemacht ist, sondern immer existiert, immer
zusammen mit Gott und dem Vater herrscht, ">dessen Herrschaft"< nach dem Zeugnis der Apostel
">kein Ende haben wird"<\footnoteA{Markell wurde vorgeworfen, er lehre ein Ende der Herrschaft des
Sohnes (vgl. fr. 101--105 Seibt/Vinzent), was er hiermit zu korrigieren sucht.}.
\kapR{2,7}Dieser ist Sohn, dieser ist Kraft, dieser ist Weisheit, dieser ist eigenes und
wahres Wort Gottes, unser Herr Jesus Christus, ungetrennte Kraft Gottes, durch den
alles Gewordene geworden ist, wie es das Evangelium bezeugt, wenn es sagt: ">Am Anfang war
das Wort, und das Wort war bei Gott, und Gott war das Wort. Alles ist durch ihn geworden,
und ohne ihn ist nicht eines geworden."< \kapR{2,8}Dieser ist das Wort, das auch
der Evangelist Lukas bezeugt, wenn er sagt: ">wie uns die �berliefert haben, die von
Anfang an Augenzeugen und Diener des Wortes gewesen sind"<; �ber diesen sagt auch David:
">Mein Herz �u�erte ein gutes Wort."< \kapR{2,9}So lehrt uns auch unser Herr Jesus
Christus durch das Evangelium, wenn er sagt: ">Ich bin aus dem Vater ausgegangen und
gekommen."< Dieser ist in den letzten Tagen um unseres Heiles willen herabgestiegen, ist aus
der Jungfrau Maria geboren und hat den Menschen angenommen. 
\pend
\pstart
\kapR{3,1}Ich glaube also an einen
allm�chtigen Gott\footnoteA{Der erste Artikel ger�t sehr kurz, die sonst �blichen
Attribute ">Vater"< und ">eins"< fehlen, vgl. aber � 7; 12.} 
% \pend
% \pstart
und an Jesus Christus, seinen eingeborenen Sohn, unseren Herrn, 
% \pend
% \pstart
der geboren wurde aus dem heiligen Geist und der Jungfrau Maria\footnoteA{Vgl. in Dok.
\ref{sec:AntIII}; \ref{sec:AntII}: \griech{gennhj'enta >ek (t~hc) parj'enou}.}, 
% \pend
% \pstart
der unter Pontius Pilatus gekreuzigt und begraben wurde\footnoteA{Eine ausf�hrlichere Beschreibung
des sonst in den Formeln genannten ">Leidens"<.} 
% \pend
% \pstart
und am dritten Tage auferstanden ist\footnoteA{Entsprechendes auch im Dok. \ref{sec:AntII} und
bei Eusebius, Dok. \ref{ch:22}~= Urk. 22,4.} von den Toten\footnoteA{Bei Theophronius Dok. \ref{sec:AntIII}:
\griech{>anast'anta >ap`o t~wn nekr~wn.}}, 
% \pend
% \pstart
aufgefahren ist in den Himmel 
% \pend
% \pstart
und der zur Rechten des Vaters sitzt, 
von wo er kommt, um die Lebenden und die Toten zu richten; 
% \pend
% \pstart
und an den heiligen Geist\footnoteA{Dieser Artikel bleibt so kurz wie schon im Nicaenum (Dok. \ref{ch:24}~= Urk. 24).}, 
% \pend
% \pstart
die heilige Kirche\footnoteA{Eine Aussage �ber die Kirche in einem Bekenntnis dieser Zeit
ist ungew�hnlich, findet sich aber interessanterweise auch in der Erkl�rung von Antiochien
(Dok. \ref{ch:18}~= Urk. 18,12) und bei Arius, Dok. \ref{ch:30}~= Urk. 30,3: \griech{e>ic m'ian kajolik`hn >ekklhs'ian to~u
jeo~u.} (64,11 Opitz).}, Vergebung der S�nden\footnoteA{Vgl. Eus.,
e.\,th. III 5,14.}, 
% \pend
% \pstart
Auferstehung des Fleisches und das ewige Leben\footnoteA{Vgl. wieder das von Arius Formulierte
Dok. \ref{ch:30}~= Urk. 30,3: \griech{ka`i e>ic sark`oc >an'astasin ka`i e>ic zw`hn to~u m'ellontoc a>i~wnoc}
(64,10 Opitz) und im Antiochenum I (Dok. \ref{sec:AntI}) den Schlu�satz: \griech{e>i d`e de~i
prosje~inai piste'uomen ka`i per`i sark`oc >anast'asewc ka`i zw~hc a>iwn'iou}; vgl. auch
Dok. \ref{ch:18}~= Urk. 18,12.}.\footnoteA{Markell setzt hier in � 11 nach � 7 zum zweiten Mal ein mit
">ich glaube"<. Dieser Abschnitt �hnelt sehr dem sogenannten \textit{Romanum}, der Vorform des
sp�teren \textit{Apostolicums}, f�r dessen Rekonstruktion gew�hnlich au�er dieser Textpassage bei Markell noch Rufin., symb. und Ambr., symb. (vgl. Ambr., ep. 15 extra coll.) herangezogen werden
(vgl. au�erdem noch Aug., serm. 212--215; Ps-Aug., serm. 240; 244 (Caes.\,Arel.);  Fulg.\,Rusp., c. Fab. frg. 36; Max.\,Taur., serm. 83; Nicet., symb.; Petr.\,Chrys., serm. 57--62; Quodv., symb.; Priscill., tract. 2; Cod. Laudianus gr. 35
[= Codex E der Apg], der neben der Apostelgeschichte auf der letzten Seite ein
Glaubensbekenntnis [lateinisch] �berliefert). Der inhaltlich vergleichbare Text 
Traditio apostolica 21 ist schwer zu datieren, au�erdem
�berliefern die verschiedenen Versionen der Traditio apostolica entsprechend
unterschiedliche Varianten der Tauf"|fragen; zus�tzlich ist die M�glichkeit einer
�berarbeitung dieser Tauf"|fragen nach dem \textit{Romanum} in Rechnung zu stellen. Es
stellt sich also die Frage, ob Markell hier zustimmend ein schon existierendes r�misches
Bekenntnis zitiert, um seine Rechtgl�ubigkeit best�tigt zu bekommen, ob es sich um ein
Bekenntnis aus seiner Heimatgemeinde handelt oder ob diese Erkl�rung ebenso wie die vielen
Bekenntnisse des vierten Jahrhunderts (vgl. z.\,B. Dok. \ref{ch:6}~= Urk. 6 und Dok. \ref{ch:30}~= Urk 30: Erkl�rung des Arius; Dok. \ref{ch:22}~= Urk.
22: Erkl�rung des Eusebius von Caesarea; Dok. \ref{sec:AntIII}: Erkl�rung des Theophronius)
ein individuelles ">Theologenbekenntnis"< ist, das in die aktuelle Diskussion geh�rt,
aber aufgrund der Akzeptanz Markells im Westen gr��ere Wirkung entfalten konnte. Eine
bislang noch nicht erwogene M�glichkeit ist, da� Markell hier eine Formel
zitiert, die kurz zuvor auf der r�mischen Synode verfa�t wurde. Darauf k�nnte deuten, da� Markell hier zweifellos etwas
 zitiert, da� das Bekenntnis von relativ schlichtem Charakter ist, da� es typisch westliche Formulierungen aufnimmt und von der
Diskussion im Osten noch relativ unber�hrt ist. Vgl. \cite{Kelly:Glaubensbekenntnisse};
\cite[v.\,a. 57--74]{Markschies:Traditio} und \cite{Vinzent:Entstehung}.}
\pend
\pstart
\kapR{3,2}Da� die Gottheit des Vaters und des Sohnes ungeteilt ist, haben wir von den
g�ttlichen Schriften gelernt. Wenn n�mlich jemand den Sohn, d.\,h. das Wort vom allm�chtigen
Gott trennt, so mu� er entweder glauben, da� es zwei G�tter\footnoteA{Zum Vorwurf des
Polytheismus vgl. Marcell., fr. 91, 97, 121, 122 (Seibt/Vinzent) und Asterius, fr. 38--42.}
gebe, was aber als der g�ttlichen Lehre fremd erachtet wird, oder er mu� bekennen, da�
das Wort nicht Gott sei, was aber ebenfalls dem rechten Glauben fremd zu sein scheint, da
der Evangelist sagt: ">und das Wort war Gott."< \kapR{3,3}Ich aber wei� genau, da� der
Sohn die ungeteilte und ungetrennte Kraft des Vaters ist. Denn der Erl�ser, unser Herr Jesus
Christus, sagt selbst: ">in mir ist der Vater, und ich bin in dem Vater"< und ">ich und
der Vater sind eins"< und ">wer mich gesehen hat, hat den Vater gesehen"<. 
\kapR{3,4}Diesen Glauben, den ich von den g�ttlichen Schriften empfangen und von den gottgem��en Vorg�ngern gelernt habe, verk�ndige ich in der Kirche Gottes und habe ich jetzt an dich geschrieben; eine Abschrift davon behalte ich bei mir selbst. 
\kapR{3,5}Und ich bitte dich, die Zweitschrift davon dem Brief an die Bisch�fe beizuf�gen, damit nicht einige von denjenigen, die uns und jene nicht genau kennen, auf die Schriftst�cke von
ihnen achten und so get�uscht werden. Seid gegr��t! 
\pend
\endnumbering
\end{translatio}
\end{Rightside}
\Columns
\end{pairs}
% \thispagestyle{empty}
\selectlanguage{german}
% \renewcommand*{\goalfraction}{.9}
% \cleartooddpage
%% Erstellt von uh
%% �nderungen:
\section{Brief des Julius von Rom an in Antiochien versammelte Bisch�fe}
% \label{sec:41.7}
\label{sec:BriefJuliusII}
\begin{praefatio}
  \begin{description}
  \item[Fr�hjahr 341]Die r�mische Synode hat nach � 26 an dem
    angesagten Termin (s. Dok.  \ref{sec:BriefJulius}) begonnen: gut
    50 Bisch�fe versammelten sich in der Kirche des Vitus (h.\,Ar. 15,1; apol.\,sec. 1,2; 20,3). Athanasius\index[namen]{Athanasius!Bischof von
      Alexandrien} hielt sich bis dahin schon ein Jahr und sechs
    Monate in Rom\index[namen]{Rom} auf (� 40;
    s. Dok. \ref{sec:BriefJulius}). Die Verhandlungen um
    Markell\index[namen]{Markell!Bischof von Ancyra} liegen bereits
    zur�ck, und eine Absage aus Antiochien\index[namen]{Antiochien} ist
    eingetroffen. Somit d�rfte dieser Brief am Schlu� der Synode im Fr�hjahr 341
    verfa�t worden sein.
  \item[�berlieferung]Den vollst�ndigen Brief �berliefert allein
    Athanasius (vgl. Socr., h.\,e. II 17,7--9; Ath., h.\,Ar. 15,1;
    apol.\,sec. 20,3). Zus�tzlich ging auch ein Brief der Synode an
    Constans\index[namen]{Constans, Kaiser} (h.\,Ar. 15,2), der jedoch
    nicht �berliefert ist.
  \item[Fundstelle]Ath., apol.\,sec. 21--35 (\editioncite[102,13--113,25]{Opitz1935})
  \end{description}
\end{praefatio}
\begin{pairs}
\selectlanguage{polutonikogreek}
\begin{Leftside}
% \beginnumbering
\pstart
\hskip -1.3em\edtext{\abb{}}{\killnumber\Cfootnote{\hskip -.6em\latintext\dt{BKO
RE}}}\specialindex{quellen}{section}{Athanasius!apol.\,sec.!21--35} %% Bezeugungsleiste
\kap{1}>Io'ulioc\edindex[namen]{Julius!Bischof von Rom}
\edtext{\abb{D\Ladd{i}an'iw|}}{\Dfootnote{\latintext coni. Opitz \greektext Dan'iw|
\latintext BKO RE}}\edindex[namen]{Dianius!Bischof von Caesarea} ka`i
Flak'illw|\edindex[namen]{Flacillus!Bischof von Antiochien}
Nark'issw|\edindex[namen]{Narcissus!Bischof von Neronias}
E>useb'iw|\edindex[namen]{Eusebius!Bischof von Nikomedien}
M'ari\edindex[namen]{Maris!Bischof von Chalcedon}
Makedon'iw|\edindex[namen]{Macedonius!Bischof von Mopsuestia}
Jeod'wrw|\edindex[namen]{Theodorus!Bischof von Heraclea} ka`i to~ic s`un a>uto~ic to~ic
>ap`o >Antioqe'iac\edindex[synoden]{Antiochien!a. 341} gr'ayasin <hm~in, >agaphto~ic
>adelfo~ic, >en kur'iw| qa'irein. 
\pend
\pstart
\kap{2}>An'egnwn t`a gr'ammata t`a di`a t~wn presbut'erwn mou
>Elpid'iou\edindex[namen]{Elpidius!Presbyter in Rom} ka`i
Filox'enou\edindex[namen]{Philoxenus!Presbyter in Rom}
\edtext{>apokomisj'enta}{\Dfootnote{komisj'enta \latintext RE}} ka`i >eja'umasa, p~wc
<hme~ic \edtext{m`en >ag'aph|}{\lemma{\abb{}}\Dfootnote{\responsio\ >ag'aph| m`en
\latintext E}} ka`i suneid'hsei >alhje'iac >egr'ayamen, <ume~ic d`e met`a filoneik'iac,
ka`i o>uq <wc >'eprepen, >epeste'ilate. <uperoy'ia g`ar ka`i >alazone'ia t~wn gray'antwn
di`a t~hc >epistol~hc >ede'iknuto. ta~uta d`e >all'otria t~hc >en Qrist~w| p'iste'wc
>estin.
\kap{3}>'edei g`ar t`a met`a >ag'aphc graf'enta >amoib~hc t~hc >'ishc met`a >ag'aphc
tuqe~in ka`i m`h met`a filoneik'iac. >`h o>uq`i >ag'aphc >est`i gn'wrisma presbut'erouc
>aposte~ilai, sumpaje~in to~ic p'asqousi, protr'eyasjai to`uc gr'ayantac >elje~in, <'ina
p'anta j~atton l'usin lab'onta diorjwj~hnai dunhj~h|, ka`i mhk'eti m'hte o<i >adelfo`i
<hm~wn p'asqwsi m'hte \edtext{<um~ac}{\Dfootnote{<hm~ac \latintext E}} tinec diab'allwsin?
>all'' o>uk o>~ida, t'i t`o d'oxan o<'utwc \edtext{<um~ac}{\Dfootnote{<hm~ac \latintext
BKO}} diatej~hnai, <'wste ka`i <hm~ac poi~hsai log'izesjai <'oti ka`i >en o<~ic >ed'oxate
<r'hmasin <hm~ac tim~an, ta~uta \edtext{metasqhmatiz'omenoi}{\Dfootnote{sqhmatiz'omenoi
\latintext RE}} met`a e>irwne'iac tin`oc e>ir'hkate. 
\kap{4}ka`i g`ar ka`i o<i
presb'uteroi o<i >apostal'entec, o<`uc >'edei met`a qar~ac >epanelje~in, to>unant'ion
lupo'umenoi >epan~hljon, >ef'' o<~ic <ewr'akasin >eke~i ginom'enoic. ka`i >'egwge to~ic
gr'ammasin >entuq`wn poll`a logis'amenoc kat'esqon par'' >emaut~w| t`hn >epistol'hn,
nom'izwn <'omwc <'hxein tin`ac ka`i m`h qre'ian e>~inai t~hc >epistol~hc, <'ina m`h ka`i
>ec faner`on >eljo~usa pollo`uc t~wn >enta~uja lup'hsh|. >epeid`h d`e mhden`oc >elj'ontoc
>an'agkh g'egonen a>ut`hn prokomisj~hnai, <omolog~w <um~in, p'antec >eja'umasan ka`i
>egg`uc >apist'iac geg'onasin, e>i <'olwc par> <um~wn toia~uta >egr'afh; filoneik'iac g`ar
m~allon ka`i o>uk >ag'aphc >~hn <h >epistol'h. 
\kap{5}e>i m`en o>~un filotim'iac l'ogwn <'eneken <o <upagore'usac >'egrayen, >'allwn t`o
toio~uton >epit'hdeuma. >en g`ar to~ic >ekklhsiastiko~ic o>u l'ogwn >ep'ideix'ic >estin,
>all`a kan'onec >apostoliko`i ka`i spoud`h to~u m`h skandal'izein <'ena t~wn mikr~wn t~wn
>en t~h| >ekklhs'ia|. sumf'erei g`ar kat`a t`on >ekklhsiastik`on l'ogon
((\edtext{\abb{m'ulon >onik`on kremasj~hnai e>ic t`on tr'aqhlon ka`i katapontisj~hnai >`h
skandal'isai k>`an <'ena t~wn mikr~wn.}}{\lemma{\abb{}}\Afootnote{\latintext Mt 18,6}}\edindex[bibel]{Matthaeus!18,6})) e>i d`e <'wc
tinwn leluphm'enwn di`a t`hn pr`oc >all'hlouc mikroyuq'ian -- o>u g`ar >`an e>'ipoimi
p'antwn e>~inai ta'uthn gn'wmhn -- toia'uth g'egonen <h >epistol'h, >'eprepe m`en mhd`e
<'olwc luphj~hnai mhd`e ((\edtext{\abb{>epid~unai t`on <'hlion >ep`i t~h|
l'uph|}}{\lemma{\abb{}}\Afootnote{\latintext Eph
4,26}}\edindex[bibel]{Epheser!4,26})), >'edei d`e <'omwc m`h e>ic toso~uton proaqj~hnai,
<'wste ka`i >'eggrafon a>ut`hn >epide'ixasjai.
\pend
\pstart
\kap{6}T'i g`ar ka`i g'egonen >'axion l'uphc >`h >en t'ini >~hn >'axion
\edtext{luphj~hnai <um~ac}{\lemma{\abb{}}\Dfootnote{\responsio\ <um~ac luphj~hnai
\latintext E}} o<~ic ka`i >egr'ayamen? >`h <'oti proetrey'ameja e>ic s'unodon >apant~hsai?
>all`a to~uto m~allon >'edei met`a qar~ac d'exasjai. o<i g`ar parrhs'ian >'eqontec, >ef''
o<~ic pepoi'hkasi ka'i, <wc a>uto`i l'egousi, kekr'ikasin, o>uk >aganakto~usin, e>i par''
<et'erwn >exet'azoito <h kr'isic, >all`a jarro~usin, <'oti <`a dika'iwc >'ekrinan, ta~uta
>'adika o>uk >'an pote g'enoito.
\kap{7}di`a to~uto ka`i o<i >en t~h| kat`a N'ikaian\edindex[synoden]{Nicaea!a. 325}
meg'alh| sun'odw| sunelj'ontec
>ep'iskopoi o>uk >'aneu jeo~u boul'hsewc suneq'wrhsan >en <et'era| sun'odw| t`a t~hc
prot'erac >exet'azesjai, 
\edtext{\abb{<'ina}}{\Dfootnote{\latintext dupl. O}} ka`i o<i kr'inontec pr`o >ofjalm~wn >'eqontec t`hn
>esom'enhn deut'eran kr'isin met`a p'ashc >asfale'iac >exet'azwsi ka`i o<i krin'omenoi
piste'uwsi, m`h kat'' >'eqjran t~wn prot'erwn, >all`a kat`a t`o d'ikaion <eauto`uc
kr'inesjai. e>i d`e t`o toio~uton >'ejoc palai`on tugq'anon, mnhmoneuj`en d`e ka`i graf`en
>en t~h| meg'alh| sun'odw|, <ume~ic to~uto par'' <um~in >isq'uein o>u j'elete, >aprep`hc
m`en <h toia'uth para'ithsic; t`o g`ar <'apax sun'hjeian >esqhk`oc >en t~h| >ekklhs'ia|
ka`i <up`o sun'odwn bebaiwj`en o>uk e>'ulogon \edtext{\abb{<up`o
>ol'igwn}}{\Dfootnote{\latintext > K}} paral'uesjai, >'allwc te
\Ladd{\edtext{\abb{e>i}}{\Dfootnote{\latintext add. Scheidweiler}}} o>ud`e >en to'utw|
dika'iwc >`an fane~ien luphj'entec.
\kap{8}o<i g`ar par> <um~wn t~wn per`i E>us'ebion\edindex[namen]{Eusebianer}
>apostal'entec met`a gramm'atwn, l'egw d`h Mak'arioc
\edindex[namen]{Macarius!Presbyter in Alexandrien} <o presb'uteroc ka`i
Mart'urioc\edindex[namen]{Martyrius!Diakon in Antiochien} ka`i
<Hs'uqioc\edindex[namen]{Hesychius!Diakon in Antiochien} o<i di'akonoi,
>apant'hsantec >enta~uja, <wc o>uk >hdun'hjhsan pr`oc to`uc >elj'ontac
>Ajanas'iou\edindex[namen]{Athanasius!Bischof von Alexandrien} presbut'erouc
>antist~hnai, >all'' >en p~asi dietr'eponto ka`i dihl'egqonto, t`o thnika~uta >hx'iwsan
<hm~ac <'wste s'unodon sugkrot~hsai ka`i gr'ayai ka`i
>Ajanas'iw|\edindex[namen]{Athanasius!Bischof von Alexandrien} t~w| >episk'opw| e>ic
>Alex'andreian, gr'ayai d`e ka`i to~ic per`i E>us'ebion\edindex[namen]{Eusebianer}, <'ina
>ep`i parous'ia| p'antwn <h dika'ia kr'isic >exeneqj~hnai dunhj~h|; t'ote g`ar ka`i
>apodeikn'unai p'anta t`a kat`a >Ajan'asion
\edindex[namen]{Athanasius!Bischof von Alexandrien} >ephgge'ilanto.
\kap{9}koin~h| g`ar <uf'' <hm~wn dihl'egqjhsan o<i per`i
Mart'urion\edindex[namen]{Martyrius!Diakon in Antiochien} ka`i
<Hs'uqion\edindex[namen]{Hesychius!Diakon in Antiochien} ka`i o<i
>Ajanas'iou\edindex[namen]{Athanasius!Bischof von Alexandrien} to~u
\edtext{\abb{>episk'opou}}{\Dfootnote{+ t~hc >Alexandre'iac \latintext RE}} presb'uteroi
met`a pepoij'hsewc >anj'istanto, o<i d`e per`i 
Mart'urion\edindex[namen]{Martyrius!Diakon in Antiochien}, e>i de~i t>alhj`ec
e>ipe~in, >en p~asi dietr'eponto, <'ojen ka`i >hx'iwsan s'unodon gen'esjai.
\kap{10}e>i to'inun \edtext{mhd`e}{\Dfootnote{mhd`en \latintext K}} t~wn per`i
Mart'urion\edindex[namen]{Martyrius!Diakon in Antiochien} ka`i
<Hs'uqion\edindex[namen]{Hesychius!Diakon in Antiochien} >axiws'antwn
gen'esjai s'unodon protrey'amenoc \edtext{>'hmhn
>eg`w}{\lemma{\abb{}}\Dfootnote{\responsio\ >eg`w >'hmhn \latintext K}} sk~ulai to`uc
gr'ayantac <'eneken t~wn >adelf~wn <hm~wn t~wn a>itiwm'enwn >adik'ian peponj'enai, ka`i
o<'utwc e>'ulogoc >~hn ka`i dika'ia <h protrop'h, >'esti g`ar >ekklhsiastik`h ka`i je~w|
>ar'eskousa; <'ote d`e o<`uc <ume~ic a>uto`i 
o<i per`i E>us'ebion\edindex[namen]{Eusebianer}
>axiop'istouc <hg'hsasje, ka`i o<~utoi >hx'iwsan <hm~ac sugkal'esai, >ak'oloujon >~hn
to`uc klhj'entac m`h luphj~hnai, >all`a m~allon proj'umwc >apant~hsai.
\kap{11}o>uko~un <h m`en d'oxasa
>agan'akthsic t~wn luphj'entwn propet'hc, <h d`e para'ithsic t~wn m`h jelhs'antwn
>apant~hsai >aprep`hc ka`i <'upoptoc >ek to'utwn de'iknutai. a>iti~ata'i tic <`a pr'attwn
a>ut`oc >apod'eqetai, e>i par'' <et'erou gin'omena \edtext{bl'epoi}{\Dfootnote{bl'epei
\latintext K}}? e>i g'ar, <wc gr'afete, >as'aleuton >'eqei t`hn >isq`un <ek'asth s'unodoc,
ka`i >atim'azetai <o kr'inac >e`an par'' <et'erwn <h kr'isic
\edtext{>exet'azhtai}{\Dfootnote{>erg'azhtai \latintext B}}, skope~ite, >agaphto'i, t'inec
e>is`in o<i s'unodon >atim'azontec ka`i t'inec t`a t~wn fjas'antwn kr~inai dial'uousi.
ka`i <'ina m`h t`a kaj'' <'ekaston n~un >exet'azwn >epibare~in tinac dok~w, >all`a t'o ge
teleuta~ion gen'omenon, >ef'' <~w| ka`i fr'ixeien >'an tic >ako'uwn, >arke~i pr`oc
>ap'odeixin p'antwn t~wn paraleifj'entwn.
\pend
\pstart
\kap{12}O<i >Areiano`i\edindex[namen]{Arianer} o<i
>ap`o to~u t~hc makar'iac mn'hmhc 
>Alex'androu\edindex[namen]{Alexander!Bischof von Alexandrien} to~u
genom'enou >episk'opou \edtext{\abb{t~hc}}{\Dfootnote{\latintext > K}} >Alexandre'iac
>ep`i >asebe'ia| >ekblhj'entec o>u m'onon <up`o t~wn kaj'' <ek'asthn p'olin
>apekhr'uqjhsan, >all`a ka`i <up`o p'antwn t~wn koin~h| sunelj'ontwn >en t~h| kat`a
N'ikaian\edindex[synoden]{Nicaea!a. 325} meg'alh| sun'odw| >anejemat'isjhsan. o>u g`ar
>~hn a>ut~wn t`o tuq`on
plhmm'elhma, o>ud`e e>ic >'anjrwpon >~hsan <amart'hsantec, >all'' e>ic a>ut`on t`on
k'urion <hm~wn >Ihso~un Qrist`on t`on u<i`on to~u jeo~u to~u z~wntoc.
\kap{13}ka`i <'omwc o<i <up`o p'ashc t~hc o>ikoum'enhc >apokhruqj'entec ka`i kat`a
p~asan >ekklhs'ian sthliteuj'entec n~un l'egontai ded'eqjai, >ef'' <~w| ka`i <um~ac
>ako'usantac qalepa'inein d'ikaion <hgo~umai. t'inec o>~un e>isin o<i s'unodon
>atim'azontec? o>uq o<i t~wn triakos'iwn t`ac y'hfouc par'' o>ud`en j'emenoi ka`i
>as'ebeian e>usebe'iac prokr'inantec?
\kap{14}<h m`en g`ar t~wn >Areiomanit~wn a<'iresic <up`o p'antwn t~wn <apantaqo~u
>episk'opwn kategn'wsjh ka`i >apekhr'uqjh, 
>Ajan'asioc\edindex[namen]{Athanasius!Bischof von Alexandrien} d`e ka`i
M'arkelloc\edindex[namen]{Markell!Bischof von Ancyra} o<i >ep'iskopoi
ple'ionac >'eqousi to`uc <up`er <eaut~wn l'egontac ka`i gr'afontac.
M'arkelloc\edindex[namen]{Markell!Bischof von Ancyra} m`en g`ar
>emartur'hjh <hm~in ka`i >en t~h| kat`a N'ikaian\edindex[synoden]{Nicaea!a. 325}
sun'odw| to~ic t`a >Are'iou frono~usin
>anteirhk'wc, >Ajan'asioc\edindex[namen]{Athanasius!Bischof von Alexandrien}
d`e >emartur'hjh mhd`e >en T'urw|\edindex[synoden]{Tyrus!a. 335} katagnwsje'ic, >en
d`e t~w|
Mare'wth|\edindex[namen]{Mareotis} m`h pare~inai, >'enja t`a <upomn'hmata kat''
a>uto~u gegen~hsjai l'egetai.
o>'idate d'e, >agaphto'i, <'oti \edtext{\abb{t`a}}{\Dfootnote{\latintext > E}} kat`a
monom'ereian >isq`un o>uk >'eqei, >all'' <'upopta tugq'anei.
\kap{15}ka`i <'omwc to'utwn >'ontwn <hme~ic <up`er >akribe'iac o>'ute <um~in o>'ute
to~ic <up`er a>ut~wn gr'ayasi pr'okrima poio~untec proetrey'ameja to`uc gr'ayantac
>elje~in, <'in'' >epeid`h ple'iouc e>is`in o<i <up`er a>ut~wn gr'ayantec, >ep`i sun'odou
p'anta >exetasj~h| pr`oc t`o mhd`e t`on >ana'ition katakrij~hnai m'hte t`on <upe'ujunon
<wc kajar`on logisj~hnai. o>uko~un o>u par'' <hm~wn >atim'azetai s'unodoc, >all`a par''
>eke'inwn t~wn <apl~wc ka`i <wc >'etuqe to`uc par`a p'antwn katakrij'entac >Areiano`uc
ka`i par`a gn'wmhn t~wn krin'antwn dexam'enwn. o<i g`ar ple'ionec >'hdh
\edtext{\abb{>anal'usant'ec e>isi s`un Qrist~w|}}{\Afootnote{\latintext vgl. Phil
1,23}},\edindex[bibel]{Philipper!1,23|textit} o<i d`e >'eti ka`i n~un >en t~w| b'iw|
to'utw| >exet'azontai >aganakto~untec <'oti t`hn a>ut~wn kr'isin >'elus'an tinec.
\pend
\pstart
\kap{16}To~uto d`e ka`i >ap`o t~wn genom'enwn m`en >en t~h|
>Alexandre'ia|\edindex[namen]{Alexandrien} >'egnwmen (ka`i g`ar ka`i
Karp'wnhc\edindex[namen]{Carpones!Presbyter in Alexandrien} tic >ekblhje`ic <up`o
>Alex'androu\edindex[namen]{Alexander!Bischof von Alexandrien} di`a t`hn
>Are'iou\edindex[namen]{Arius!Presbyter in Alexandrien} a<'iresin met'a tinwn ka`i
a>ut~wn >ekblhj'entwn di`a t`hn a>ut`hn a<'iresin >elhl'ujasin >enta~uja >apostal'entec
par`a Grhgor'iou\edindex[namen]{Gregor!Bischof von Alexandrien} tin'oc),
<'omwc d`e >em'ajomen ka`i par`a 
Makar'iou\edindex[namen]{Macarius!Presbyter in Alexandrien} to~u presbut'erou ka`i
Martur'iou\edindex[namen]{Martyrius!Diakon in Antiochien} ka`i
<Hsuq'iou\edindex[namen]{Hesychius!Diakon in Antiochien} t~wn diak'onwn.
pr`o \edtext{to~u g`ar}{\lemma{\abb{}}\Dfootnote{\responsio\ g`ar to~u \latintext KO}}
>apant~hsai to`uc >Ajanas'iou\edindex[namen]{Athanasius!Bischof von Alexandrien}
presbut'erouc proetr'eponto <hm~ac gr'afein >en 
>Alexandre'ia|\edindex[namen]{Alexandrien} Pist~w|
\edindex[namen]{Pistus!Bischof von Alexandrien} tini, <hn'ika ka`i 
>Ajan'asioc\edindex[namen]{Athanasius!Bischof von Alexandrien} <o >ep'iskopoc >en
>Alexandre'ia| >~hn.
\kap{17}to~uton d`e t`on 
Pist`on\edindex[namen]{Pistus!Bischof von Alexandrien} o<i
>Ajanas'iou\edindex[namen]{Athanasius!Bischof von Alexandrien} to~u
>episk'opou presb'uteroi paragen'omenoi >ap'edeixan e>~inai
>Areian`on\edindex[namen]{Arius!Presbyter in Alexandrien} >ekblhj'enta m`en
<up`o >Alex'androu\edindex[namen]{Alexander!Bischof von Alexandrien} to~u
>episk'opou ka`i t~hc kat`a N'ikaian\edindex[synoden]{Nicaea!a. 325} sun'odou,
katastaj'enta d`e <up`o 
Seko'undou\edindex[namen]{Secundus!Bischof von Ptolema"is} tin'oc, <`on <h meg'alh
s'unodoc >Areian`on >'onta
>ex'ebale. to~uto d`e o>ud`e
a>uto`i o<i per`i Mart'urion\edindex[namen]{Martyrius!Diakon in Antiochien}
>ant'elegon o>ud`e >hrno~unto t`on Pist`on\edindex[namen]{Pistus!Bischof von Alexandrien}
<up`o Seko'undou\edindex[namen]{Secundus!Bischof von Ptolema"is} >esqhk'enai
t`hn kat'astasin. skope~ite to'inun ka`i >ek to'utwn, t'inec >`an <up`o m'emyin dika'iwc
g'enointo, o<i m`h peisj'entec <hme~ic, <'wste 
Pist~w|\edindex[namen]{Pistus!Bischof von Alexandrien} t~w|
\edtext{>Areian~w|}{\Dfootnote{>Are'iw| \latintext B}} \edtext{gr'ayai}{\Dfootnote{gr'afai
\latintext O}}, >`h o<i sumboule'uontec >atim'asai t`hn meg'alhn s'unodon ka`i to~ic
>aseb'esin <wc e>useb'esi gr'ayai?
\kap{18}ka`i g`ar ka`i Mak'arioc\edindex[namen]{Macarius!Presbyter in Alexandrien} <o
presb'uteroc <o par`a E>useb'iou\edindex[namen]{Eusebius!Bischof von Nikomedien} met`a
t~wn per`i Mart'urion\edindex[namen]{Martyrius!Diakon in Antiochien}
>apostale`ic <wc >'hkousen >epist'antac to`uc
presbut'erouc >Ajanas'iou\edindex[namen]{Athanasius!Bischof von Alexandrien},
>ekdeqom'enwn <hm~wn t`hn parous'ian a>uto~u met`a t~wn per`i
Mart'urion\edindex[namen]{Martyrius!Diakon in Antiochien} ka`i
<Hs'uqion\edindex[namen]{Hesychius!Diakon in Antiochien} >aped'hmhse
nukt`oc ka'itoi nos~wn t~w| s'wmati, <wc >ek
to'utou loip`on <hm~ac >akolo'ujwc stoq'azesjai <'oti a>isqun'omenoc t`on kat`a 
Pisto~u\edindex[namen]{Pistus!Bischof von Alexandrien}
>'elegqon >aneq'wrhsen.
\kap{19}>ad'unaton g`ar t`hn kat'astasin 
Seko'undou\edindex[namen]{Secundus!Bischof von Ptolema"is} to~u
>Areiano~u >en t~h| kajolik~h| >ekklhs'ia| >isq'usai. >atim'ia g`ar >alhj~wc a<'uth kat`a
t~hc sun'odou ka`i t~wn >en a>ut~h| sunelj'ontwn >episk'opwn, >e`an t`a met`a tosa'uthc
spoud~hc ka`i e>ulabe'iac <wc jeo~u par'ontoc gen'omena >ant`i mhden`oc luj~h|.
\pend
\pstart
\kap{20}E>'iper o>~un, <wc gr'afete, >ek to~u kat`a
\edtext{Nob'aton}{\Dfootnote{Nau'aton \latintext KO}}\edindex[namen]{Novatian! } ka`i t`on Samosat'ea Pa~ulon\edindex[namen]{Paulus von Samosata}
parade'igmatoc t`a t~wn sun'odwn >isq'uein d'ogmata qr'h, >'edei m~allon m`h luj~hnai t~wn
triakos'iwn t`hn y~hfon, >'edei t`hn kajolik`hn s'unodon <up`o t~wn >ol'igwn m`h
>atimasj~hnai. a<iretiko`i g`ar o<i >Areiano`i <'wsper k>ake~inoi ka`i <'omoiai a<i kat`a
to'utwn y~hfoi ta~ic kat'' >eke'inwn. to'utwn d`e tolmhj'entwn t'inec e>is`in o<i fl'oga
diqono'iac >an'ayantec? <hm~ac g`ar to~uto pepoihk'enai gr'ayantec >em'emyasje. >~ar''
o>~un <hme~ic diqono'iac e>irgas'ameja o<i sunalgo~untec to~ic p'asqousin >adelfo~ic ka`i
kat`a kan'ona p'anta pepoihk'otec? >`h o<i filone'ikwc ka`i par`a kan'ona \edtext{t~wn
triakos'iwn t`hn y~hfon}{\lemma{\abb{}}\Dfootnote{\responsio\ t`hn y~hfon t~wn triakos'iwn
\latintext K}} l'usantec ka`i kat`a p'anta t`hn s'unodon >atim'asantec?
\kap{21}o>u g`ar
m'onon o<i >Areiano`i >ed'eqjhsan, >all`a ka`i >ap`o t'opou e>ic t'opon memelet'hkasin
\edtext{>ep'iskopoi diaba'inein}{\lemma{\abb{}}\Dfootnote{\responsio\ diaba'inein
>ep'iskopoi \latintext K}}. e>i o>~un >alhj~wc >'ishn ka`i t`hn a>ut`hn <hge~isje tim`hn
t~wn >episk'opwn ka`i m`h >ek to~u meg'ejouc t~wn p'olewn, <wc gr'afete, kr'inete to`uc
>episk'opouc, >'edei t`on pepisteum'enon mikr`an m'enein >en t~h| pisteuje'ish| ka`i m`h
>exoujene~in m`en \edtext{t`o}{\Dfootnote{t`on \latintext B}} pepisteum'enon, metaba'inein
d`e e>ic t`hn m`h >egqeirisje~isan, <'ina t~hc m`en par`a jeo~u doje'ishc katafron~h|,
t`hn d`e t~wn >anjr'wpwn kenodox'ian >agap'hsh|.
\kap{22}o>uko~un, >agaphto'i, >'edei
>apant~hsai ka`i m`h parait'hsasjai, <'ina ka`i t'eloc l'abh| t`o pr~agma; to~uto g`ar <o
l'ogoc >apaite~i. >all'' >'iswc <h projesm'ia >enep'odise; gr'ayantec g`ar >em'emyasje,
<'oti sten`hn t`hn projesm'ian t~hc sun'odou <wr'isamen, >all`a ka`i to~uto, >agaphto'i,
pr'ofas'ic >estin. e>i m`en g`ar >erqom'enouc tin`ac sun'ekleisen <h <hm'era, sten`on >`an
t`o di'asthma t~hc projesm'iac >hl'egqjh, e>i d`e o<i >elje~in m`h boul'omenoi
katesq'hkasi ka`i to`uc presbut'erouc <'ewc a>uto~u to~u >Iannouar'iou mhn'oc, m`h
jarro'untwn >est`in <h pr'ofasic.
\kap{23}>~hljon g`ar >'an, <wc proe~ipon, e>i
>ej'arroun o>u pr`oc t`o di'asthma t~hc <odo~u skopo~untec o>ud`e pr`oc t`hn projesm'ian
<or~wntec, >all'' >ep`i to~ic dika'ioic ka`i to~ic e>ul'ogoic parrhsiaz'omenoi. >all''
>'iswc di`a t`on kair`on o>uk >ap'hnthsan. to~uto g`ar gr'afontec p'alin >edhl'wsate, <wc
>'ara >'edei <hm~ac skop'hsantac t`on >ep`i t~hc <E'w|ac kair`on m`h protr'eyasjai <um~ac
>apant~hsai. e>i m`en o>~un di`a t`o toio~uton e>~inai t`on kair`on o>uk >aphnt'hsate,
<'wc fate, >'edei prot'erouc 
\edtext{\abb{<um~ac}}{\Dfootnote{\dt{B\corr} <hm~ac \dt{B*}}} t`on kair`on to~uton skop'hsantac m`h a>it'iouc
sq'ismatoc mhd`e >ololug~hc ka`i jr'hnwn >en ta~ic >ekklhs'iaic gen'esjai. n~un d`e o<i
ta~uta pepoihk'otec >'edeixan m`h t`on kair`on a>'ition, >all`a t`hn proa'iresin t~wn m`h
jelhs'antwn >apant~hsai.
\pend
\pstart
\kap{24}Jaum'azw d`e k>ake~ino t`o m'eroc t~hc >epistol~hc, p~wc <'olwc k>`an >egr'afh
par'' <um~wn <'oti d`h to~ic per`i E>us'ebion\edindex[namen]{Eusebianer} m'onoic ka`i o>u
p~asin <um~in m'onoc
>'egraya; e>uq'ereian g`ar t~wn memyam'enwn m~allon >'an tic e<'uroi >'hper >al'hjeian.
>eg`w g`ar o>uk >'allojen lab`wn kat`a >Ajanas'iou\edindex[namen]{Athanasius!Bischof von 
Alexandrien} gr'ammata >`h di`a t~wn per`i
Mart'urion\edindex[namen]{Martyrius!Diakon in Antiochien} ka`i
<Hs'uqion\edindex[namen]{Hesychius!Diakon in Antiochien} >eke'inoic
>an'agkh| >'egraya to~ic ka`i gr'ayasi kat'' a>uto~u.
>'edei to'inun >`h to`uc per`i E>us'ebion\edindex[namen]{Eusebianer} m`h m'onouc qwr`ic
p'antwn <um~wn gr'ayai >`h
<um~ac, o<~ic m`h >'egraya, m`h \edtext{lupe~isjai}{\Dfootnote{lupe~isje \latintext B}},
e>i >eke'inoic >egr'afh to~ic ka`i gr'ayasin. e>i g`ar >eqr~hn ka`i p~asin <um~in
>episte~ilai, >'edei ka`i <um~ac s`un >eke'inoic gr'ayai.
\kap{25}n~un d`e t`o >ak'oloujon
skopo~untec >eke'inoic >egr'ayamen to~ic ka`i dhl'wsasin <hm~in ka`i to~ic >aposte'ilasi
pr`oc <hm~ac. e>i d`e ka`i t`o m'onon >em`e gegraf'enai >eke'inoic >ek'inhsen <um~ac,
>ak'olouj'on >estin <um~ac >aganakte~in, <'oti ka`i m'onw| >emo`i >'egrayan. >all`a ka`i
>en to'utw| pijan`h m`en ka`i o>uk \edtext{\abb{>'alogoc}}{\Dfootnote{\latintext coni.
Montfaucon \greektext e>'ulogoc \latintext codd.}} <h pr'ofasic, >agaphto'i, <'omwc d`e
gnwr'isai <um~in >anagka~ion <'oti, e>i ka`i m'onoc >'egraya, >all'' o>uk >emo~u m'onou
>est`in a<'uth \edtext{\abb{<h}}{\Dfootnote{\latintext > RE}} gn'wmh >all`a ka`i p'antwn
t~wn kat`a t`hn >Ital'ian\edindex[namen]{\latintext Italien} ka`i t~wn >en to'utoic to~ic m'eresin >episk'opwn. ka`i >'egwge
to`uc p'antac o>uk >hj'elhsa poi~hsai gr'ayai, <'ina m`h par`a poll~wn t`o b'aroc
>'eqwsin.
\kap{26}>am'elei ka`i n~un t~h| <orisje'ish| projesm'ia| sun~hljon >ep'iskopoi
ka`i ta'uthc t~hc gn'wmhc geg'onasin, <`hn p'alin gr'afwn <um~in shma'inw, <'wste,
>agaphto'i, e>i ka`i m'onoc >epist'ellw, >all`a p'antwn \edtext{gn'wmhn e~>inai
ta'uthn}{\lemma{\abb{}}\Dfootnote{\responsio\ e~>inai gn'wmhn ta'uthn \latintext B
\responsio\ \griech{gn'wmhn ta'uthn e~>inai} R}} gin'wskete. ta~uta m`en o>~un per`i to~u
m`h e>ul'ogwc tin`ac >ex <um~wn prof'aseic, >all`a >ad'ikouc ka`i <up'optouc por'isasjai.
\pend
\pstart
\kap{27}Per`i d`e to~u m`h e>uqer~wc mhd`e >ad'ikwc \edtext{<hm~ac}{\Dfootnote{<um~ac
\latintext RE}} \edtext{<upoded'eqjai}{\Dfootnote{<upod'eqesjai \latintext K}} e>ic
koinwn'ian to`uc sunepisk'opouc <hm~wn >Ajan'asion\edindex[namen]{Athanasius!Bischof von 
Alexandrien} ka`i M'arkellon\edindex[namen]{Markell!Bischof von Ancyra}, e>i
ka`i t`a proleqj'enta <ikan'a, <'omwc di`a braq'ewn e>'ulogon <um~in de~ixai.
\edtext{\abb{>'egrayan}}{\Dfootnote{\latintext coni. Commelin \greektext >egr'ayate
\latintext codd.}} o<i per`i E>us'ebion\edindex[namen]{Eusebianer} pr'oteron kat`a t~wn
per`i >Ajan'asion\edindex[namen]{Athanasius!Bischof von Alexandrien},
>egr'ayate
d`e ka`i n~un <ume~ic, >'egrayan d`e ka`i ple~istoi >ep'iskopoi >ap`o t~hc
A>ig'uptou\edindex[namen]{Aegyptus} ka`i
>ex >'allwn >eparqi~wn <up`er 
>Ajanas'iou\edindex[namen]{Athanasius!Bischof von Alexandrien}.
\kap{28}pr~wton m`en o>~un t`a kat'' a>uto~u
gr'ammata par'' <um~wn m'aqetai pr`oc <eaut`a ka`i o>udem'ian sumfwn'ian >'eqei t`a
de'utera pr`oc t`a pr~wta, >all'' >en pollo~ic t`a pr~wta <up`o t~wn deut'erwn l'uetai
ka`i t`a de'utera <up`o t~wn pr'wtwn diab'alletai; >asumf'wnwn d`e >'ontwn t~wn gramm'atwn
o>udem'ia p'istic per`i t~wn legom'enwn >est'in. >'epeita, e>i to~ic par'' <um~wn
grafe~isin >axio~ute piste'uein, >ak'olouj'on >esti ka`i to~ic <up`er a>uto~u gr'ayasi m`h
>apist~hsai, ka`i m'alista <'oti <ume~ic m`en p'orrwjen, >eke~inoi d`e >en to~ic t'opoic
>'ontec ka`i e>id'otec t`on >'andra ka`i t`a >eke~i gin'omena pr'agmata gr'afousi
marturo~untec a>uto~u t~w| b'iw| ka`i diabebaio'umenoi >en p~asin a>ut`on suskeu`hn
peponj'enai.
\kap{29}ka`i p'alin >Ars'eni'oc\edindex[namen]{Arsenius!Bischof von Hypsele} tic
>ep'iskopoc >el'eqjh pot`e <wc
>anaireje`ic par`a >Ajanas'iou\edindex[namen]{Athanasius!Bischof von Alexandrien}, >all`a
to~uton >em'ajomen z~hn,
\edtext{\abb{>all`a}}{\Dfootnote{\latintext > E}} ka`i fil'ian >'eqein pr`oc
a>ut'on.
\kap{30}t`a <upomn'hmata t`a >en Mare'wth|\edindex[namen]{Mareotis} gen'omena
diebebai'wsato kat`a
monom'ereian gegen~hsjai; m'hte g`ar >eke~i pare~inai
Mak'arion\edindex[namen]{Macarius!Presbyter in Alexandrien} t`on presb'uteron t`on
kathgoro'umenon m'hte a>ut`on t`on >ep'iskopon a>uto~u
>Ajan'asion\edindex[namen]{Athanasius!Bischof von Alexandrien}. ka`i to~uto
o>u m'onon
>ek t~wn a>uto~u l'ogwn, >all`a ka`i >ek t~wn <upomnhm'atwn, <~wn >ek'omisan <hm~in o<i
per`i Mart'urion\edindex[namen]{Martyrius!Diakon in Antiochien} ka`i
<Hs'uqion\edindex[namen]{Hesychius!Diakon in Antiochien}, >'egnwmen.
>anagn'ontec g`ar e<'uromen <'oti <o m`en
kat'hgoroc >Isq'urac\edindex[namen]{Ischyras!Presbyter in der Mareotis} >eke~i par~hn, o>'ute d`e
Mak'arioc\edindex[namen]{Macarius!Presbyter in Alexandrien} o>'ute <o
>ep'iskopoc
>Ajan'asioc\edindex[namen]{Athanasius!Bischof von Alexandrien}; >all`a ka`i
\edtext{\abb{to`uc}}{\Dfootnote{\latintext > B}} presbut'erouc
>Ajanas'iou\edindex[namen]{Athanasius!Bischof von Alexandrien} >axio~untac
pare~inai ka`i m`h sugqwrhj'entac.
\kap{31}>'edei d`e,
>agaphto'i, e>'iper met`a >alhje'iac \edtext{>eg'ineto}{\Dfootnote{>eg'eneto \latintext
RE}} t`o krit'hrion m`h m'onon t`on kat'hgoron, >all`a ka`i t`on
\edtext{kathgoro'umenon}{\Dfootnote{kathgouro'umenon \latintext B}} pare~inai. <'wsper
g`ar >en t~h| T'urw|\edindex[synoden]{Tyrus!a. 335}
Mak'arioc\edindex[namen]{Macarius!Presbyter in Alexandrien} par~hn <o kathgoro'umenoc
ka`i >Isq'urac\edindex[namen]{Ischyras!Presbyter in der Mareotis} <o kat'hgoroc,
ka`i o>ud`en >ede'iqjh, o<'utwc >'edei ka`i >en t~w| Mare'wth|\edindex[namen]{Mareotis}
m`h m'onon >apelje~in t`on
kat'hgoron, >all`a ka`i t`on kathgoro'umenon, <'ina par`wn >`h >elegqj~h| >`h m`h
>elegqje`ic de'ixh| t`hn sukofant'ian. n~un d`e to'utou m`h
\edtext{genom'enou}{\Dfootnote{ginom'enou \latintext K}}, >all`a m'onou >apelj'ontoc to~u
kathg'orou, mej'' <~wn parh|t'hsato >Ajan'asioc\edindex[namen]{Athanasius!Bischof von
Alexandrien}, <'upopta t`a pr'agmata fa'inetai.
\pend
\pstart
\kap{32}>Hiti~ato\looseness=-1\ d`e ka`i to`uc >apelj'ontac e>ic t`on
Mare'wthn\edindex[namen]{Mareotis} par`a gn'wmhn a>uto~u
>apelhluj'enai. >'elege g`ar <'oti Je'ognion\edindex[namen]{Theognis!Bischof von Nicaea}
ka`i M'arin\edindex[namen]{Maris!Bischof von Chalcedon} ka`i
Je'odwron\edindex[namen]{Theodorus!Bischof von Heraclea},
\edtext{\abb{O>urs'akion}}{\Dfootnote{te
\dt{coni. Schwartz}}}\edindex[namen]{Ursacius!Bischof von Singidunum} ka`i
O>u'alenta\edindex[namen]{Valens!Bischof von Mursa} ka`i
Maked'onion\edindex[namen]{Macedonius!Bischof von Mopsuestia} <up'optouc
>'ontac >ap'esteilan. ka`i to~uto o>uk >ek t~wn l'ogwn a>uto~u
\edtext{m'onon}{\Dfootnote{m'onwn \latintext RE}}, >all`a ka`i >ek t~hc >epistol~hc
>Alex'androu\edindex[namen]{Alexander!Bischof von Thessalonike} to~u genom'enou
>episk'opou t~hc Jessalon'ikhc >ede'iknue. pro'hnegke g`ar
>epistol`hn \edtext{\abb{a>uto~u}}{\Dfootnote{\latintext > E}} grafe~isan pr`oc
Dion'usion\edindex[namen]{Dionysius!Comes}
t`on >en t~h| sun'odw| k'omhta, >en <~h| dhlo~i faner`an suskeu`hn gen'esjai kat`a
>Ajanas'iou\edindex[namen]{Athanasius!Bischof von Alexandrien}.
\kap{33}ka`i a>uto~u d`e to~u kathg'orou >Isq'ura\edindex[namen]{Ischyras!Presbyter in der Mareotis} proek'omise
qe~ira <ol'ografon
a>ujentik'hn, >en <~h| m'artura \edtext{t`on}{\lemma{\abb{t`on\ts{1}}}\Dfootnote{\latintext > E}} je`on t`on
pantokr'atora >epikalo'umenoc m'hte pothr'iou kl'asin m'hte trap'ezhc >anatrop`hn
gegen~hsjai >'elegen, >all'' <upobebl~hsjai <eaut`on <up'o tinwn pl'asasjai ta'uthn t`hn
kathgor'ian. >apant'hsantec d`e ka`i presb'uteroi to~u Mare'wtou\edindex[namen]{Mareotis}
diebebai'wsanto m'hte
t`on >Isq'uran\edindex[namen]{Ischyras!Presbyter in der Mareotis} presb'uteron e>~inai t~hc kajolik~hc >ekklhs'iac
m'hte ti toio~uton
peplhmmelhk'enai Mak'arion\edindex[namen]{Macarius!Presbyter in Alexandrien}, <opo~ion
>eke~inoc kathg'orhsen.
\kap{34}o<i presb'uteroi d`e
ka`i o<i di'akonoi >apant'hsantec >enta~uja o>uk >ol'iga, >all`a ka`i poll`a >emart'urhsan
<up`er to~u >episk'opou >Ajanas'iou\edindex[namen]{Athanasius!Bischof von Alexandrien}
diabebaio'umenoi mhd`en >alhj`ec e>~inai t~wn kat''
a>uto~u legom'enwn, suskeu`hn d`e a>ut`on peponj'enai. ka`i o<i >ap`o t~hc
A>ig'uptou\edindex[namen]{Aegyptus} d`e
ka`i Lib'uhc\edindex[namen]{Libya} p'antec >ep'iskopoi gr'afontec diebebai'wsanto ka`i
t`hn kat'astasin a>uto~u
>'ennomon ka`i >ekklhsiastik`hn gegon'enai ka`i p'anta t`a par'' <um~wn leg'omena kat''
a>uto~u e>~inai yeud~h; o>'ute g`ar f'onon gegen~hsjai o>'ute tin`ac >anairej~hnai di''
a>ut`on o>'ute kl'asin pothr'iou gegen~hsjai, >all`a p'anta e>~inai yeud~h.
\kap{35}ka`i >ek t~wn <upomnhm'atwn d`e t~wn >en Mare'wth|\edindex[namen]{Mareotis} kat`a
monom'ereian
\edtext{genom'enwn}{\Dfootnote{gegenhm'enwn \latintext B}} >ede'iknuen <o >ep'iskopoc
>Ajan'asioc\edindex[namen]{Athanasius!Bischof von Alexandrien} <'ena kathqo'umenon
>exetasj'enta ka`i e>ip'onta >'endon e>~inai met`a
>Isq'ura\edindex[namen]{Ischyras!Presbyter in der Mareotis}, <'ote 
Mak'arioc\edindex[namen]{Macarius!Presbyter in Alexandrien} <o presb'uteroc
>Ajanas'iou\edindex[namen]{Athanasius!Bischof von Alexandrien}, <wc l'egousin, >ep'esth
t~w| t'opw|, ka`i >'allouc d`e >exetasj'entac ka`i e>ip'ontac t`on m`en >en kell'iw|
mikr~w|, t`on d`e >'opisjen t~hc j'urac katake~isjai t`on
>Isq'uran\edindex[namen]{Ischyras!Presbyter in der Mareotis} t'ote noso~unta, <'ote
Mak'arion\edindex[namen]{Macarius!Presbyter in Alexandrien} l'egousin >aphnthk'enai
>eke~i.
\kap{36}>ap`o d`h to'utwn <~wn >'elege, ka`i <hme~ic >akolo'ujwc stoqaz'omeja <'oti p~wc
o<~i'on te t`on >'opisjen t~hc j'urac n'osw| katake'imenon t'ote <esthk'enai ka`i
leitourge~in ka`i prosf'erein? >`h p~wc o<~i'on te >~hn prosfor`an proke~isjai >'endon
>'ontwn t~wn kathqoum'enwn? e>i g`ar >'endon >~hsan o<i kathqo'umenoi, o>'upw >~hn <o
kair`oc t~hc prosfor~ac.
\kap{37}ta~uta, <'wsper e>'irhtai, >'elegen <o >ep'iskopoc
>Ajan'asioc\edindex[namen]{Athanasius!Bischof von Alexandrien} ka`i >ek t~wn
<upomnhm'atwn >ede'iknue, diabebaioum'enwn ka`i t~wn s`un a>ut~w| mhd`e <'olwc a>ut`on
presb'uteron gegen~hsja'i pote >en t~h| kajolik~h| >ekklhs'ia| mhd`e
\edtext{sesun~hqjai}{\Dfootnote{sunaqj~hnai \latintext KO}} a>ut'on pote >en t~h|
>ekklhs'ia| <wc presb'uteron; o>ud`e g`ar \edtext{o>ud`e}{\lemma{\abb{o>ud`e\ts{2}}}\Dfootnote{\dt{del.
Opitz}}} <'ote >Al'exandroc\edindex[namen]{Alexander!Bischof von Alexandrien} >ed'eqeto
kat`a filanjrwp'ian t~hc meg'alhc sun'odou to`uc
>ap`o \edtext{\abb{to~u}}{\Dfootnote{\latintext > B}} sq'ismatoc
\edtext{Melit'iou}{\Dfootnote{Melet'iou \latintext E*}}\edindex[namen]{Melitius!Bischof von Lycopolis}, >wnom'asjai
a>ut`on <up`o 
\edtext{Melit'iou}{\Dfootnote{Melet'iou \latintext E*}}\edindex[namen]{Melitius!Bischof von Lycopolis} met`a t~wn a>uto~u
diebebaio~unto; <`o ka`i m'egist'on >esti
tekm'hrion m`h e>~inai a>ut`on m'hte 
\edtext{Melit'iou}{\Dfootnote{Melet'iou \latintext E*}}\edindex[namen]{Melitius!Bischof von Lycopolis}. e>i g`ar >~hn, p'antwc ka`i
a>ut`oc a>uto~ic \edtext{sunhrijme~ito}{\Dfootnote{sunhrijm'hjh \latintext K}}.
\kap{38}>'allwc te ka`i >en >'alloic <o >Isq'urac\edindex[namen]{Ischyras!Presbyter in der Mareotis} yeus'amenoc
>ede'iknuto <up`o
>Ajanas'iou\edindex[namen]{Athanasius!Bischof von Alexandrien} >ek t~wn <upomnhm'atwn;
kathgor'hsac g`ar <wc bibl'iwn kekaum'enwn <'ote, <wc
l'egousi, Mak'arioc\edindex[namen]{Macarius!Presbyter in Alexandrien} >ep'esth,
>hl'egqjh <uf'' <~wn a>ut`oc >'hnegke mart'urwn yeus'amenoc.
\pend
\pstart
\kap{39}To'utwn to'inun o<'utwc legom'enwn ka`i toso'utwn m`en
\edtext{\abb{>'ontwn}}{\Dfootnote{\latintext > E}} t~wn mart'urwn t~wn <up`er a>uto~u,
toso'utwn d`e dikaiwm'atwn proferom'enwn <up'' a>uto~u, t'i >'edei poie~in <hm~ac? >`h t'i
<o >ekklhsiastik`oc kan`wn >apaite~i >`h m`h katagn~wnai to~u >andr'oc, >all`a m~allon
>apod'exasjai ka`i >'eqein a>ut`on >ep'iskopon, <'wsper ka`i e>'iqomen?
\kap{40}ka`i g`ar
pr`oc to'utoic p~asi par'emeinen >enta~uja >eniaut`on ka`i <`ex m~hnac >ekdeq'omenoc t`hn
parous'ian <um~wn >`h t~wn boulom'enwn >elje~in; t~h| d`e parous'ia| >edus'wpei p'antac,
<'oti o>uk >`an par~hn e>i m`h >ej'arrei. ka`i g`ar o>uk >af'' <eauto~u >el'hlujen, >all`a
klhje`ic ka`i lab`wn gr'ammata par'' <hm~wn, kaj'aper ka`i <um~in >egr'ayamen.
\kap{41}ka`i <'omwc met`a tosa~uta <ume~ic <wc par`a kan'onac poi'hsantac <hm~ac >em'emyasje.
skope~ite \edtext{\abb{to'inun}}{\Dfootnote{\latintext > E}}, t'inec e>is`in o<i par`a
\edtext{kan'onac}{\Dfootnote{kan'ona \latintext KORE}} pr'axantec, <hme~ic o<i met`a
toso'utwn >apode'ixewn t`on >'andra 
\edtext{dex'amenoi}{\Dfootnote{>apodex'amenoi \latintext E*}} >`h o<i >ap`o tri'akonta ka`i <`ex mon~wn
>en >Antioqe'ia|\edindex[namen]{Antiochien} >onom'asant'ec tina <wc >ep'iskopon x'enon
ka`i >aposte'ilantec e>ic
\edtext{\abb{t`hn}}{\Dfootnote{\latintext > B}} >Alex'andreian\edindex[namen]{Alexandrien}
met`a stratiwtik~hc
>exous'iac? <'oper o>u g'egonen o>ud`e e>ic Gall'iac a>uto~u >apostal'entoc; >egeg'onei
g`ar >`an ka`i t'ote, e>i >'ontwc >~hn katagnwsje'ic. >am'elei
\edtext{\abb{>epanelj`wn}}{\Dfootnote{\latintext > E}} sqol'azousan ka`i >ekdeqom'enhn
a>ut`on t`hn >ekklhs'ian e<~uren.
\pend
\pstart
\kap{42}>All`a n~un o>uk o>~ida, po'iw| tr'opw| g'egone t`a gen'omena. pr~wton m`en
g'ar, e>i de~i t>alhj`ec e>ipe~in, o>uk >'edei gray'antwn <hm~wn s'unodon gen'esjai
prolabe~in tinac t`hn >ek t~hc sun'odou kr'isin. >'epeita o>uk >'edei toia'uthn
kainotom'ian kat`a t~hc >ekklhs'iac gen'esjai. po~ioc g`ar kan`wn >ekklhsiastik`oc >`h
po'ia par'adosic >apostolik`h toia'uth, <'wste e>irhneuo'ushc >ekklhs'iac ka`i >episk'opwn
toso'utwn <om'onoian >eq'ontwn pr`oc t`on >ep'iskopon t~hc
>Alexandre'iac >Ajan'asion\edindex[namen]{Athanasius!Bischof von Alexandrien}
>apostal~hnai Grhg'orion\edindex[namen]{Gregor!Bischof von Alexandrien}, x'enon m`en
t~hc p'olewc, m'hte >eke~i baptisj'enta m'hte
ginwsk'omenon to~ic pollo~ic, m`h a>ithj'enta par`a presbut'erwn, m`h par'' >episk'opwn,
m`h par`a la~wn, >all`a katastaj~hnai m`en >en >Antioqe'ia|\edindex[namen]{Antiochien},
>apostal~hnai d`e e>ic t`hn
>Alex'andreian\edindex[namen]{Alexandrien}, o>u met`a presbut'erwn, o>u met`a diak'onwn
t~hc p'olewc, o>u met`a
>episk'opwn t~hc A>ig'uptou\edindex[namen]{Aegyptus}, >all`a met`a stratiwt~wn?
\kap{43}to~uto g`ar >'elegon ka`i
>h|ti~wnto o<i >enta~uja >elj'ontec. e>i g`ar ka`i met`a t`hn s'unodon <upe'ujunoc >~hn
e<ureje`ic \edtext{\abb{<o}}{\Dfootnote{\latintext > K}}
>Ajan'asioc\edindex[namen]{Athanasius!Bischof von Alexandrien}, o>uk >'edei t`hn
kat'astasin o<'utwc paran'omwc ka`i par`a t`on >ekklhsiastik`on kan'ona gen'esjai, >all''
>ep'' a>ut~hc t~hc >ekklhs'iac, >ap'' a>uto~u to~u <ierate'iou, >ap'' a>uto~u to~u kl'hrou
to`uc >en t~h| >eparq'ia| >episk'opouc >'edei katast~hsai ka`i m`h n~un to`uc >ap`o t~wn
>apost'olwn kan'onac paral'uesjai. >~ara g'ar, e>i kaj'' <en`oc <um~wn >egeg'onei t`o
toio~uton, o>uk >`an >ebo'hsate, o>uk >`an >hxi'wsate <wc paralelum'enwn t~wn kan'onwn
>ekdikhj~hnai?
\kap{44}>agaphto'i, <wc jeo~u par'ontoc met`a >alhje'iac fjegg'omeja ka`i
\edtext{l'egomen}{\Dfootnote{l'egwmen \latintext B*}}; o>uk >'esti to~uto e>useb`ec o>ud`e n'omimon o>ud`e >ekklhsiastik'on. ka`i g`ar
ka`i t`a leg'omena gegen~hsjai par`a 
Grhgor'iou\edindex[namen]{Gregor!Bischof von Alexandrien} >en t~h| e>is'odw| a>uto~u
de'iknusi t`hn
t~hc katast'asewc t'axin. >en g`ar toio'utoic e>irhniko~ic kairo~ic, <wc a>uto`i
\edtext{o<i >elj'ontec >ap`o t~hc >Alexandre'iac}{\lemma{\abb{}}\Dfootnote{\responsio\ o<i
>ap`o t~hc >Aleqandre'iac >elj'ontec \latintext RE}}\edindex[namen]{Alexandrien}
>ap'hggeilan, kaj`wc ka`i o<i
>ep'iskopoi >'egrayan, <h >ekklhs'ia >emprhsm`on <up'emeine, parj'enoi >egumn'wjhsan,
mon'azontec katepat'hjhsan, presb'uteroi ka`i pollo`i to~u lao~u >h|k'isjhsan \edtext{ka`i
b'ian pep'onjasin, >ep'iskopoi >efulak'isjhsan}{\lemma{\abb{ka`i \dots\
>efulak'isjhsan}}\Dfootnote{\latintext > E}}, peries'urhsan pollo'i, t`a <'agia 
\edtext{\abb{must'hria}}{\Dfootnote{\latintext > E}},
>ef'' o<~ic >h|ti~wnto Mak'arion\edindex[namen]{Macarius!Presbyter in Alexandrien} t`on
presb'uteron, <up`o >ejnik~wn dihrp'azeto ka`i e>ic
g~hn 
\edtext{>eb'alleto}{\Dfootnote{kateb'alleto \latintext B*}}, <'ina t`hn Grhgor'iou\edindex[namen]{Gregor!Bischof von Alexandrien}
tin`ec kat'astasin d'exwntai.
\kap{45}t`a d`e
toia~uta de'iknusi to`uc paral'uontac to`uc kan'onac. e>i g`ar nom'imh >~hn <h
kat'astasic, o>uk >`an di`a paranom'iac >hn'agkaze pe'ijesjai to`uc nom'imwc >apeijo~untac
a>ut~w|. ka`i <'omwc toio'utwn genom'enwn gr'afete e>ir'hnhn meg'alhn gegen~hsjai >en
\edtext{\abb{t~h|}}{\Dfootnote{\latintext > BKO}}
>Alexandre'ia|\edindex[namen]{Alexandrien} ka`i t~h| A>ig'uptw|\edindex[namen]{Aegyptus},
>ekt`oc e>i m`h \edtext{>antimetab'eblhtai}{\Dfootnote{>antib'eblhtai \latintext K}} t`o
>'ergon t~hc e>ir'hnhc ka`i t`a toia~uta e>ir'hnhn >onom'azete.
\pend
\pstart
\kap{46}K>ake~ino\looseness=-1\ d`e <um~in >anagka~ion >en'omisa dhl~wsai <'oti
>Ajan'asioc\edindex[namen]{Athanasius!Bischof von Alexandrien}
diebebaio~uto Mak'arion\edindex[namen]{Macarius!Presbyter in Alexandrien} >en
T'urw|\edindex[synoden]{Tyrus!a. 335} <up`o strati'wtac gegen~hsjai ka`i m'onon t`on
kat'hgoron >apelhluj'enai met`a t~wn >apelj'ontwn e>ic t`on
Mare'wthn\edindex[namen]{Mareotis}, ka`i to`uc m`en
presbut'erouc >axio~untac pare~inai >en t~h| >exet'asei m`h sugkeqwr~hsjai, t`hn d`e
>ex'etasin gegen~hsjai per`i pothr'iou ka`i trap'ezhc >ep`i parous'ia| to~u >ep'arqou ka`i
t~hc t'axewc a>uto~u par'ontwn >ejnik~wn ka`i >Iouda'iwn. to~uto d`e kat`a t`ac >arq`ac
>'apiston >~hn, e>i m`h ka`i >ek t~wn <upomnhm'atwn >ede'iknuto, >ef'' <~w| ka`i
\edtext{>ejaum'asamen}{\Dfootnote{>epijaum'azein \latintext coni. Commelin}}, nom'izw d`e
ka`i <um~ac >'eti jaum'azein, >agaphto'i.
\kap{47}presb'uteroi m`en o>uk >epitr'epontai
pare~inai o<i ka`i t~wn musthr'iwn leitourgo`i tugq'anontec, >ep`i d`e >exwtiko~u
dikasto~u par'ontwn kathqoum'enwn ka`i t'o ge qe'iriston, >ep`i >ejnik~wn ka`i >Iouda'iwn
t~wn diabeblhm'enwn per`i t`on Qristianism`on >ex'etasic per`i a<'imatoc Qristo~u ka`i
s'wmatoc Qristo~u g'inetai. e>i g`ar ka`i <'olwc 
\edtext{>egeg'onei}{\Dfootnote{g'inetai \dt{O\corr} g'egone \dt{O* (?)}}} ti plhmm'elhma, >'edei >en
t~h| >ekklhs'ia| <up`o klhrik~wn \edtext{nom'imwn}{\Dfootnote{nom'imwc \latintext coni.
Commelin}} >exet'azesjai t`a toia~uta \edtext{ka`i m`h <up`o >ejnik~wn t~wn t`on l'ogon
bdelussom'enwn ka`i m`h e>id'otwn t`hn >al'hjeian. to~uto d`e t`o <am'arthma <hl'ikon ka`i
<opo~i'on >esti, sunor~an ka`i <um~ac ka`i p'antac pep'isteuka. per`i m`en >Ajanas'iou
toia~uta}{\lemma{\abb{ka`i \dots\ toia~uta}}\Dfootnote{\latintext >
BKO}}\edindex[namen]{Athanasius!Bischof von Alexandrien}.
\pend
\pstart
\kap{48}Per`i d`e Mark'ellou\edindex[namen]{Markell!Bischof von Ancyra}, >epeid`h ka`i
per`i a>uto~u <wc >asebo~untoc e>ic t`on
Qrist`on >egr'ayate, dhl~wsai <um~in >espo'udasa <'oti >enta~uja 
\edtext{gen'omenoc}{\Dfootnote{gin'omenoc \latintext R*}} diebebai'wsato
m`en m`h e>~inai >alhj~h t`a per`i a>uto~u graf'enta par'' <um~wn, <'omwc d`e
>apaito'umenoc par'' <hm~wn e>ipe~in per`i t~hc p'istewc o<'utwc met`a parrhs'iac
>apekr'inato di'' <eauto~u <wc >epign~wnai m`en <hm~ac <'oti mhd`en >'exwjen t~hc
>alhje'iac <omologe~i.
\kap{49}o<'utwc g`ar e>useb~wc per`i to~u kur'iou ka`i swt~hroc
<hm~wn >Ihso~u Qristo~u <wmol'oghse frone~in, <'wsper ka`i <h kajolik`h >ekklhs'ia
frone~i; ka`i o>u n~un ta~uta pefronhk'enai diebebai'wsato, >all`a ka`i >'ekpalai,
<wspero~un ka`i o<i <hm'eteroi presb'uteroi t'ote >en t~h| kat`a
N'ikaian\edindex[synoden]{Nicaea!a. 325} sun'odw|
gen'omenoi >emart'urhsan a>uto~u t~h| >orjodox'ia|. ka`i g`ar ka`i t'ote ka`i n~un kat`a
t~hc a<ir'esewc t~wn >Areian~wn\edindex[namen]{Arianer}
pefronhk'enai diisqur'isato, >ef'' <~w| ka`i <um~ac
<upomn~hsai d'ikai'on >estin, <'ina mhde`ic t`hn toia'uthn a<'iresin >apod'eqhtai, >all`a
bdel'utthtai <wc >allotr'ian t~hc <ugiaino'ushc didaskal'iac.
\kap{50}>orj`a to'inun
a>ut`on frono~unta ka`i >ep`i >orjodox'ia| marturo'umenon t'i p'alin ka`i >ep`i
\edtext{to'utou}{\Dfootnote{to'utoic \latintext E}} >'edei poie~in <hm~ac >`h >'eqein
a>ut'on, <'wsper ka`i e>'iqomen, >ep'iskopon ka`i m`h >apob'allein t~hc
koinwn'iac?
\kap{51}ta~uta m`en o>~un >eg`w o>uq <wc <uperapologo'umenoc a>ut~wn g'egrafa,
>all'' <'eneka to~u piste~usai <um~ac <'oti dika'iwc ka`i kanonik~wc >edex'ameja to`uc
>'andrac, ka`i m'athn filoneike~ite, <um~ac d`e d'ikai'on >esti spoud'asai ka`i p'anta
tr'opon kame~in, <'ina t`a m`en par`a kan'ona gen'omena diorj'wsewc t'uqh|, a<i d`e
>ekklhs'iai e>ir'hnhn >'eqwsi pr`oc t`o t`hn to~u kur'iou e>ir'hnhn t`hn doje~isan <hm~in
parame~inai ka`i m`h sq'izesjai t`ac >ekklhs'iac mhd`e <um~ac <wc a>it'iouc sq'ismatoc
m'emyin <upome~inai. <omolog~w g`ar <um~in t`a gen'omena o>uk e>ir'hnhc, >all`a sq'ismatoc
prof'aseic e>is'in.
\pend
\pstart
\kap{52}O>u g`ar m'onon o<i per`i 
>Ajan'asion\edindex[namen]{Athanasius!Bischof von Alexandrien} ka`i
M'arkellon\edindex[namen]{Markell!Bischof von Ancyra} o<i >ep'iskopoi
>elhl'ujasin >enta~uja \edtext{a>iti'wmenoi kat'' a>ut~wn >adik'ian
\edtext{gegon'enai}{\Dfootnote{genom'enhn \latintext
KOR}}}{\lemma{\abb{}}\Dfootnote{\responsio\ >adik'ian genom'enhn kat'' a>ut~wn a>iti'wmenoi
\latintext K}}, >all`a ka`i ple~istoi >'alloi >ep'iskopoi >ap`o
Jr\aai khc\edindex[namen]{Thracia}, >ap`o Ko'ilhc
Sur'iac\edindex[namen]{Syria coele}, >ap`o Foin'ikhc\edindex[namen]{Phoenice}
ka`i Palaist'inhc\edindex[namen]{Palaestina}. ka`i presb'uteroi m`en o>uk
>ol'igoi, >'alloi
d'e tinec >ap`o >Alexandre'iac\edindex[namen]{Alexandrien} ka`i >'alloi >ex >'allwn
mer~wn, >ap'hnthsan e>ic t`hn
>enta~uja s'unodon, >ep`i p'antwn t~wn sunelj'ontwn >episk'opwn pr`oc to~ic >'alloic
\edtext{\abb{o~<ic}}{\Dfootnote{\latintext > E}} >'elegon
\edtext{\abb{>'eti}}{\Dfootnote{\latintext coni. Montfaucon \greektext <'oti \latintext
codd.}} ka`i >apwd'uronto b'ian ka`i >adik'ian \edtext{t`ac}{\Dfootnote{t~hc \latintext B}}
>ekklhs'iac peponj'enai, ka`i <'omoia t~wn kat`a
>Alex'andreian\edindex[namen]{Alexandrien} ka`i t`ac <eaut~wn ka`i
>en >'allaic >ekklhs'iaic gegen~hsjai diebebaio~unto o>u l'ogw| m'onon, >all`a ka`i
pr'agmasi.
\kap{53}ka`i >ex A>ig'uptou\edindex[namen]{Aegyptus} d`e ka`i t~hc
>Alexandre'iac\edindex[namen]{Alexandrien} p'alin >elj'ontec n~un
presb'uteroi met`a gramm'atwn >apwd'uronto <'oti pollo`i >ep'iskopoi ka`i presb'uteroi
j'elontec >elje~in e>ic t`hn s'unodon >ekwl'ujhsan. m'eqri g`ar n~un ka`i met`a t`hn to~u
>episk'opou >Ajanas'iou\edindex[namen]{Athanasius!Bischof von Alexandrien} >apodhm'ian
>'elegon >episk'opouc <omologht`ac katak'optesjai
plhga~ic ka`i >'allouc fulak'izesjai, >'hdh d`e ka`i >arqa'iouc, ple~iston <'oson qr'onon
>'eqontac >en t~h| >episkop~h|, e>ic leitourg'iac dhmos'iac parad'idosjai ka`i sqed`on
p'antac to`uc t~hc kajolik~hc >ekklhs'iac klhriko`uc ka`i lao`uc >epiboule'uesjai ka`i
di'wkesjai. ka`i g`ar ka'i tinac >episk'opouc ka'i tinac >adelfo`uc <uperor'iouc >'elegon
gegen~hsjai di'' o>ud`en <'eteron, >`h <'ina ka`i >'akontec
\edtext{>anagk'azwntai}{\Dfootnote{>anagk'azontai \latintext R}} koinwne~in
Grhgor'iw|\edindex[namen]{Gregor!Bischof von Alexandrien}
ka`i to~ic s`un a>ut~w| >Areiano~ic\edindex[namen]{Arianer}.
\kap{54}ka`i >en >Agk'ura|\edindex[namen]{Ancyra} d`e t~hc
Galat'iac o>uk
>ol'iga gegen~hsjai, >all`a t`a a>ut`a p'alin to~ic kat`a
>Alex'andreian\edindex[namen]{Alexandrien} genom'enoic
>hko'usamen ka`i par'' <et'erwn, ka`i 
M'arkelloc\edindex[namen]{Markell!Bischof von Ancyra} d`e <o >ep'iskopoc diemart'urato.
pr`oc d`e to'utoic ka`i toia'utac kathgor'iac ka`i o<'utw dein`ac kat'a tinwn >ex <um~wn,
<'ina m`h l'egw t`a >on'omata, o<i >apant'hsantec e>ir'hkasin, <`ac >eg`w m`en gr'ayai
parh|ths'amhn; >'iswc \edtext{\abb{d`e}}{\Dfootnote{\latintext > B}} ka`i <ume~ic par''
<et'erwn >akhk'oate.
\kap{55}di`a to~uto g`ar m'alista ka`i >'egraya protrep'omenoc <um~ac
>elje~in, <'ina ka`i par'ontec >ako'ushte ka`i p'anta diorjwj~hnai ka`i jerapeuj~hnai
dunhj~h|. >ep`i d`e to'utw| \edtext{projumot'erwc}{\Dfootnote{projumot'erouc \latintext
O}} >'edei to`uc klhj'entac >apant~hsai ka`i m`h parait'hsasjai, <'ina \edtext{\abb{m'hte
<wc}}{\Dfootnote{\latintext coni. Opitz \greektext m`h t'ewc \latintext codd.}} <'upoptoi
per`i t`a leqj'enta, >e`an m`h >apant'hswsi, nomisj~wsi
\edtext{\abb{m'hte}}{\Dfootnote{\latintext del. Commelin}} <wc m`h dun'amenoi >el'egxai
<`a >'egrayan.
\pend
\pstart
\kap{56}To'utwn to'inun o<'utwc legom'enwn ka`i o<'utwc t~wn >ekklhsi~wn pasqous~wn ka`i
>epibouleuom'enwn, <wc o<i >apagg'ellontec diebebaio~unto, t'inec e>is`in o<i fl'oga
diqono'iac >an'ayantec? <hme~ic o<i lupo'umenoi >ep`i to~ic toio'utoic ka`i sump'asqontec
to~ic p'asqousin >adelfo~ic >`h o<i t`a toia~uta >ergas'amenoi? jaum'azw g'ar, p~wc
toia'uthc ka`i tosa'uthc >eke~i >akatastas'iac kaj'' <ek'asthn >ekklhs'ian o>'ushc, di''
<`hn ka`i o<i >apant'hsantec >~hljon >enta~uja, gr'afete <ume~ic <om'onoian gegen~hsjai
>en ta~ic >ekklhs'iaic. t`a d`e toia~uta o>uk >ep`i o>ikodom~h| t~hc >ekklhs'iac, >all''
>ep`i kajair'esei ta'uthc g'inetai. ka`i o<i >en to'utoic d`e qa'irontec o>'uk e>isin
e>ir'hnhc u<io'i, >all`a >akatastas'iac; <o d`e je`oc <hm~wn ((\edtext{o>uk >'estin
>akatastas'iac >all'' e>ir'hnhc}{\lemma{\abb{}}\Afootnote{\latintext 1Cor
14,33}}\edindex[bibel]{Korinther I!14,33})).
\kap{57}di'oper, <wc o>~iden <o je`oc ka`i pat`hr to~u kur'iou <hm~wn >Ihso~u Qristo~u,
khd'omenoc m`en ka`i t~hc \edtext{<um~wn}{\Dfootnote{<hm~wn \latintext B}} <upol'hyewc,
e>uq'omenoc d`e ka`i 
\edtext{t`ac}{\Dfootnote{t~hc \latintext B*}} >ekklhs'iac m`h >en >akatastas'ia| e>~inai, >all`a diam'enein
<'wsper <up`o t~wn >apost'olwn >ekanon'isjh, gr'ayai <um~in ta~uta >anagka~ion <hghs'amhn,
<'ina >'hdh pot`e duswp'hshte to`uc di`a t`hn pr`oc >all'hlouc >ap'eqjeian o<'utw
diajem'enouc t`ac >ekklhs'iac. >'hkousa g`ar <'oti tin'ec e>isin >ol'igoi o<i to'utwn
p'antwn a>'itioi tugq'anontec.
\kap{58}ka`i \edtext{spoud'asate}{\Dfootnote{spoud'ashte \latintext KORE}} <wc spl'agqna
>'eqontec \edtext{o>iktirmo~u}{\Dfootnote{o>iktirm~wn \latintext BK}} diorj'wsasjai, <wc
proe~ipon, t`a par`a kan'ona gen'omena, <'ina, \edtext{e>i
ka'i}{\lemma{\abb{}}\Dfootnote{\responsio\ ka`i e>i \latintext E}} ti proel'hfjh, to~uto
t~h| <umet'era| spoud~h| jerapeuj~h|. ka`i m`h gr'afhte <'oti t`hn
Mark'ellou\edindex[namen]{Markell!Bischof von Ancyra} ka`i
>Ajanas'iou\edindex[namen]{Athanasius!Bischof von Alexandrien} >`h t`hn <hm~wn <elo~u
koinwn'ian, o>u g`ar e>ir'hnhc, >all`a filoneik'iac
ka`i misadelf'iac t`a toia~uta gnwr'ismata. di`a to~uto o>~un k>ag`w t`a proeirhm'ena
>'egraya, <'ina maj'ontec <'oti o>uk >ad'ikwc >edex'ameja a>uto'uc, pa'ushsje t~hc
toia'uthc >'eridoc.
\kap{59}e>i m`en g`ar >elj'ontwn \edtext{<um~wn}{\Dfootnote{<hm~wn \latintext E}} >~hsan
katagnwsj'entec, e>i m`h e>ul'ogouc >apode'ixeic >'eqein >efa'inonto <up`er <eaut~wn,
kal~wc >`an t`a toia~uta gegraf'hkeite; >epeid`h d'e, <wc proe~ipon, kanonik~wc ka`i o>uk
>ad'ikwc t`hn pr`oc a>uto`uc >'esqomen koinwn'ian, parakal~w <up`er Qristo~u, m`h
>epitr'eyhte diasqisj~hnai t`a \edtext{m'elh}{\Dfootnote{m'erh \latintext B}} to~u
Qristo~u mhd`e to~ic prol'hmmasi piste'ushte, >all`a t`hn to~u kur'iou e>ir'hnhn
protim'hsate. 
\kap{60}o>u g`ar <'osion o>ud`e d'ikaion di'' >ol'igwn mikroyuq'ian to`uc m`h
katagnwsj'entac >aporr'iptein ka`i \edtext{\abb{lupe~in >en to'utw| t`o
pne~uma}}{\Afootnote{\latintext vgl. Eph 4,30}}\edindex[bibel]{Epheser!4,30|textit}; e>i d`e
nom'izete d'unasja'i tina kat'' a>ut~wn >apode~ixai ka`i e>ic pr'oswpon a>uto`uc
>el'egxai, >elj'etwsan o<i boul'omenoi. <eto'imouc g`ar <eauto`uc ka`i a>uto`i e>~inai
>ephgge'ilanto, <'wste \edtext{ka`i}{\lemma{\abb{ka`i\ts{1}}}\Dfootnote{\latintext > E}} >apode~ixai ka`i
diel'egxai, per`i <~wn <hm~in >ap'hggeilan.
\pend
\pstart
\kap{61}Shm'anate o>~un <hm~in, >agaphto'i, per`i to'utou, <'ina k>ake'inoic gr'aywmen
ka`i to~ic >ofe'ilousi p'alin sunelje~in >episk'opoic pr`oc t`o p'antwn par'ontwn to`uc
<upeuj'unouc katagnwsj~hnai ka`i mhk'eti >akatastas'ian >en ta~ic >ekklhs'iaic
\edtext{gen'esjai}{\Dfootnote{g'inesjai \latintext E}}. >arke~i g`ar t`a gen'omena;
>arke~i <'oti par'ontwn >episk'opwn >ep'iskopoi >exwr'izonto. per`i o<~u o>ud`e
makrhgore~in de~i, <'ina m`h bare~isjai o<i par'ontec t'ote dok~wsin. e>i g`ar
\edtext{de~i}{\Dfootnote{d`h \latintext R}} t>alhj`ec e>ipe~in, o>uk >'edei m'eqri to'utwn
fj'asai o>ud`e e>ic toso~uton >elje~in t`ac mikroyuq'iac.
\kap{62}>'estw d`e >Ajan'asioc\edindex[namen]{Athanasius!Bischof von Alexandrien} ka`i
M'arkelloc\edindex[namen]{Markell!Bischof von Ancyra}, <wc gr'afete,
metet'ejhsan >ap`o t~wn
>id'iwn t'opwn, t'i ka`i per`i t~wn >'allwn >'an tic e>'ipoi t~wn >ek diaf'orwn t'opwn,
\edtext{\abb{<wc proe~ipon}}{\Dfootnote{\latintext > K}}, >elj'ontwn >enta~uja >episk'opwn
ka`i presbut'erwn? ka`i a>uto`i g`ar p'alin <hrp'asjai <eauto`uc ka`i toia~uta peponj'enai
>'elegon.
\kap{63}>~w >agaphto'i, o>uk'eti kat`a t`o e>uagg'elion, >all`a loip`on >ep`i >exorism~w|
ka`i jan'atw| a<i kr'iseic t~hc >ekklhs'iac e>is'in. e>i g`ar ka`i <'olwc, <'wc fate,
g'egon'e ti e>ic a>uto`uc <am'arthma, >'edei kat`a t`on >ekklhsiastik`on kan'ona ka`i m`h
o<'utwc gegen~hsjai t`hn kr'isin, >'edei graf~hnai p~asin <hm~in, <'ina o<'utwc par`a
p'antwn <orisj~h| t`o d'ikaion. >ep'iskopoi g`ar >~hsan o<i p'asqontec ka`i o>uq a<i
tuqo~usai >ekklhs'iai a<i p'asqousai, >all'' <~wn a>uto`i \edtext{o<i >ap'ostoloi di''
<eaut~wn}{\lemma{\abb{}}\Dfootnote{\responsio\ di'' <eaut~wn o<i >ap'ostoloi \latintext
RE}} kajhg'hsanto.
\kap{64}\edlabel{64}di`a t'i d`e per`i t~hc >Alexandr'ewn\edindex[namen]{Alexandrien}
\edtext{\edtext{>ekklhs'iac}{\Dfootnote{p'olewc \latintext BKO}}
m'alista}{\lemma{\abb{}}\Dfootnote{\responsio\ m'alista >ekklhs'iac \latintext E}} o>uk >egr'afeto <hm~in? >`h
>agnoe~ite, <'oti to~uto >'ejoc >~hn, pr'oteron gr'afesjai <hm~in ka`i o<'utwc >'enjen
<or'izesjai t`a d'ikaia? e>i m`en o>~un \edtext{\abb{ti}}{\Dfootnote{\latintext > K}}
toio~uton >~hn <upopteuj`en e>ic t`on >ep'iskopon t`on >eke~i, >'edei pr`oc t`hn >enta~uja
>ekklhs'ian graf~hnai, n~un d`e o<i <hm~ac \edtext{\abb{m`h}}{\Dfootnote{\latintext >
KE*}} plhrofor'hsantec, pr'axantec d`e a>uto`i <wc >hj'elhsan, loip`on ka`i <hm~ac o>u
katagn'ontac bo'ulontai sumy'hfouc e>~inai.
\kap{65}o>uq o<'utwc a<i Pa'ulou diat'axeic, o>uq o<'utwc o<i pat'erec paraded'wkasin;
>'alloc t'upoc >est`in o<~utoc ka`i kain`on t`o >epit'hdeuma. parakal~w, met`a
\edtext{makrojum'iac}{\Dfootnote{projum'iac \latintext B}} >en'egkate; <up`er to~u
\edtext{koin~h|}{\Dfootnote{koino~u \latintext RE}} sumf'eront'oc >estin <`a gr'afw. <`a
g`ar pareil'hfamen par`a to~u makar'iou P'etrou to~u >apost'olou, ta~uta ka`i <um~in
dhl~w. ka`i o>uk >`an >'egraya faner`a <hgo'umenoc e>~inai ta~uta par`a p~asin, e>i m`h
t`a gen'omena <hm~ac >et'araxen.
\kap{66}>ep'iskopoi <arp'azontai ka`i >ektop'izontai, >'alloi d`e >allaq'ojen
>antit'ijentai ka`i >'alloi >epiboule'uontai, <'wste >ep`i m`en to~ic <arpasje~isin
\edtext{a>uto`uc}{\Dfootnote{to`uc lao`uc \latintext coni. St�hlin}} penje~in, >ep`i d`e
to~ic pempom'enoic >anagk'azesjai, <'ina, o<`uc m`en j'elousi, m`h >epizht~wsin, o<`uc d`e
m`h \edtext{bo'ulontai}{\Dfootnote{bo'ulwntai \latintext R*E}},
\edtext{d'eqwntai}{\Dfootnote{d'eqontai \latintext ORE}}.
\kap{67}>axi~w <um~ac mhk'eti toia~uta g'inesjai, gr'ayate d`e m~allon kat`a t~wn t`a
toia~uta >epiqeiro'untwn, <'ina mhk'eti toia~uta p'asqwsin a<i >ekklhs'iai mhd'e tic
>ep'iskopoc >`h presb'uteroc <'ubrin 
\edtext{p'asqh|}{\Dfootnote{par'asqh| \dt{E} gr'' p'asqh| \dt{E\mg}}} >`h par`a gn'wmhn, <'wsper >ed'hlwsan <hm~in, >anagk'azhta'i tic poie~in, <'ina m`h ka`i par`a to~ic >'ejnesi g'elwta >ofl'hswmen, ka`i
pr'o ge p'antwn, \edtext{\abb{<'ina m`h}}{\Dfootnote{\latintext > BKO}} t`on je`on
parox'unwmen. <'ekastoc g`ar <hm~wn \edtext{\abb{>apod'wsei l'ogon >en <hm'era|
kr'isewc}}{\lemma{\abb{>apod'wsei \dots\ kr'isewc}}\Afootnote{\latintext vgl. Mt
12,36}}\edindex[bibel]{Matthaeus!12,36}, per`i <~wn >enta~uja >'epraxe.
\kap{68}g'enoito d`e
p'antac kat`a je`on fron~hsai, <'ina ka`i a<i >ekklhs'iai to`uc >episk'opouc a>ut~wn
>apolabo~usai qa'irwsi di`a pant`oc >en Qrist~w| >Ihso~u t~w| kur'iw| <hm~wn, di'' o<~u
t~w| patr`i <h d'oxa e>ic to`uc a>i~wnac t~wn a>i'wnwn. >am'hn. >err~wsjai <um~ac >en
kur'iw| e>'uqomai, >agaphto`i ka`i pojein'otatoi >adelfo'i.
\pend
\endnumbering
\end{Leftside}
\begin{Rightside}
\begin{translatio}
\beginnumbering
\pstart
\noindent\kapR{21,1}Julius an Dianius, Flacillus, Narcissus, Eusebius, Maris, Macedonius, Theodorus
samt denen, die aus Antiochien an uns geschrieben haben, den geliebten Br�dern, Gru� im
Herrn! 
\pend
\pstart
\kapR{2}Ich\looseness=-1\ habe den Brief\footnoteA{Vgl. Dok. \ref{sec:BriefSynode341}.}, der durch meine
Presbyter Elpidius und Philoxenus �berbracht wurde, gelesen und mich dar�ber gewundert,
da� wir zwar mit Liebe und einem Bewu�tsein von der Wahrheit geschrieben hatten, ihr aber
mit Streitsucht und keineswegs so, wie es sich geziemt, zur�ckgeschrieben habt. In dem 
Brief zeigte sich n�mlich Hochmut und Prahlsucht der Schreiber. Dies ist aber dem Glauben in Christus fremd. 
\kapR{3}Denn das mit Liebe Geschriebene h�tte eine
gleichfalls mit Liebe geschriebene Antwort bekommen m�ssen und nicht eine mit Streitsucht geschriebene.
Oder ist es nicht ein Erweis der Liebe, Presbyter zu senden\footnoteA{Vgl. Dok.
\ref{sec:BriefJulius}.}, den Leidenden Mitleid zu spenden, denen, die geschrieben haben,
nahezulegen zu kommen, damit alle Probleme bald gel�st werden und in Ordnung gebracht werden k�nnen und
unsere Br�der nicht mehr leiden noch irgendwelche euch verleumden? Aber ich wei� nicht,
was eure Meinung so verstimmt hat, da� wir sogar anfangen zu glauben, da� ihr auch die
Worte, mit denen ihr uns zu ehren scheint, verdreht und mit gewisser Ironie vorgebracht
habt. 
\kapR{4}Denn auch die abgesandten Presbyter, die mit Freude h�tten zur�ckkommen sollen,
kehrten im Gegenteil betr�bt �ber das, was sie dort gesehen haben, zur�ck. Auch ich selbst
habe, nachdem ich den Brief empfangen habe, lange �berlegt und den Brief f�r mich
behalten, da ich glaubte, da� dennoch einige kommen w�rden und ich den Brief nicht
brauchen w�rde, damit er nicht, wenn er ver�ffentlicht w�rde, viele der Hiesigen verletzte.
Als er aber, da niemand erschien, gezeigt werden mu�te, wunderten sich alle, mu� ich euch
zugeben, und waren beinahe fassungslos, da� �berhaupt von euch Derartiges geschrieben
worden war. Denn der Brief enthielt mehr Streitsucht als Liebe.
\kapR{5}Falls also der Verfasser nur aus Liebe um Worte geschrieben hat, so ist ein
derartiges Verhalten eine Angelegenheit anderer Leute. Denn bei den kirchlichen
Angelegenheiten geht es nicht um den Nachweis von Redegewandtheit, sondern um die
apostolischen Gesetze und die Sorge, keinen von den Kleinen in der Kirche zu ver�rgern.
Denn es ziemt sich nach dem kirchlichen Wort, ">eher einen M�hlstein um den Hals gebunden
zu bekommen und ertr�nkt zu werden, als auch nur einen von den Kleinen zu ver�rgern"<. Wenn aber
dieser Brief geschrieben wurde, weil einige betr�bt waren aufgrund gegenseitiger
Engherzigkeit~-- denn ich m�chte nicht behaupten, da� alle diese Ansicht haben~-- so w�re es
angebracht gewesen, nicht ganz und gar Tr�bsal zu blasen oder gar ">die Sonne �ber der Tr�bsal
untergehen zu lassen"<; man h�tte wenigstens nicht soweit gehen sollen, diese
Tr�bsal schriftlich zu zeigen. 
\pend
\pstart
\kapR{22,1}Denn was ist geschehen, das Tr�bsal verdient, oder was unter den Worten, die
wir geschrieben haben, war dazu angetan, euch zu betr�ben? Etwa, da� wir (euch)
aufgefordert haben, zu einer Synode zu erscheinen? Aber dies h�ttet ihr besser mit Freude
aufnehmen sollen. Denn die, die Selbstvertrauen haben in das, was sie getan und, wie sie
sagen, entschieden haben, sind nicht unwirsch, wenn die Entscheidung von anderen gepr�ft
wird, sondern sind zuversichtlich, da sich das, was sie mit Recht entschieden haben, doch wohl niemals
als Unrecht erweisen d�rfte. 
\kapR{2}Deswegen haben auch die in Nicaea auf der gro�en Synode
versammelten Bisch�fe nicht ohne Ratschluss Gottes zugestanden, da� auf einer weiteren
Synode die Entscheidungen der vorangehenden gepr�ft werden\footnoteA{Evtl. liegt hier ein Bezug auf Can. 5 der Synode von Nicaea vor; vgl. auch unten S. \edpageref{64}, � 64.}, damit auch die Entscheidungstr�ger, den
Beschluss einer zweiten Synode vor Augen, mit gr��ter Gr�ndlichkeit untersuchen, und damit
die, �ber die geurteilt wird, glauben, da� sie nicht nach Ha�gef�hlen der fr�heren
Richter, sondern nach Gerechtigkeit beurteilt werden. Wenn aber dies eine alte Tradition
ist, die auf der gro�en Synode in Erinnerung gebracht und festgehalten wurde, ihr aber
nicht wollt, da� sie bei euch gelte, so ist ein derartiges Ansinnen unangebracht.
Denn was einmal in der Kirche zur Gewohnheit geworden und synodal best�tigt worden ist, darf rechtens nicht von einigen wenigen aufgel�st zu werden, zumal sie auch offensichtlich kein Recht haben, dar�ber
gekr�nkt zu sein.
\kapR{3}Denn als die von euch, den Eusebianern, mit Schriftst�cken losgeschickt
wurden\footnoteA{Zu diesem R�ckblick vgl. die Einleitung zu Dok. \ref{sec:BriefJulius}.},
ich rede von Macarius, dem Presbyter, und Martyrius und Hesychius, den Diakonen, die
hierher gekommen waren, und den Presbytern des Athanasius, die gekommen waren, nicht
standhalten konnten, sondern in allem widerlegt und �berf�hrt wurden,~-- in dieser
Situation baten sie uns, eine Synode zusammenzurufen und sowohl dem Bischof Athanasius nach Alexandrien zu schreiben als auch den Anh�ngern des Eusebius, damit vor aller Augen das
gerechte Urteil gef�llt werden k�nne. Denn damals erkl�rten sie sich auch bereit, alle
Anklagen gegen Athanasius zu belegen. 
\kapR{4}Denn von uns gemeinsam wurden die um Martyrius und Hesychius widerlegt, und die
Presbyter des Bischofs Athanasius widerstanden durch ihre Zuversicht, die aber um Martyrius,
um die Wahrheit zu sagen, wurden in allem widerlegt, weswegen sie auch um eine Synode
baten. 
\kapR{5}H�tte ich also, wenn die um Martyrius und Hesychius nicht um eine Synode
gebeten h�tten, dazu aufgefordert, wegen unserer Br�der, die klagen, Unrecht zu erleiden,
die Briefschreiber nicht durchzulassen, dann w�re auch so die Auf"|forderung berechtigt und
rechtens gewesen, denn sie ist kirchlich und Gott wohlgef�llig. Da aber auch die, die ihr um
Eusebius selbst f�r glaubw�rdig hieltet, uns baten, (eine Synode) zusammenzurufen, w�re
es angemessen gewesen, da� die Gerufenen nicht ver�rgert gewesen, sondern vielmehr
bereitwillig herangereist w�ren. 
\kapR{6}Folglich erweist sich daraus, da� die scheinbare Unwilligkeit der Ver�rgerten
voreilig und die Entschuldigung derer, die nicht kommen wollten, infolgedessen unpassend und verd�chtig ist. Wird man
etwa das, was man, wenn man es selbst tut, guthei�t, verurteilen, wenn man es einen
anderen tun sieht? Wenn n�mlich, wie ihr schreibt, jede Synode unantastbare Vollmacht hat und
der Richter entehrt wird, wenn das Urteil von anderen gepr�ft wird, dann beachtet,
Geliebte, welche es sind, die die Synode entehren, und welche es sind, die die Urteile der
Vorg�nger auf"|l�sen. Und damit es nicht so scheint, als ob ich jemanden bel�stige, indem ich jetzt 
alle Einzelf�lle untersuche, so gen�gen freilich als Beispiel f�r alles �brige die j�ngsten
Ereignisse, �ber die man entsetzt sein d�rfte, wenn man davon h�rt.
\pend
\pstart
\kapR{23,1}Die Arianer, die von Alexander, dem ehemaligen Bischof Alexandriens, seligen
Angedenkens, wegen Gottlosigkeit exkommuniziert worden waren\footnoteA{Vgl. Dok. \ref{ch:1} = Urk. 1; Dok. \ref{sec:4b} = Urk. 4b.}, wurden
nicht nur von denen in allen St�dten blo�gestellt, sondern auch von allen, die gemeinsam auf der
gro�en Synode in Nicaea zusammenkamen, verurteilt. Denn ihr Fehler war kein gew�hnlicher,
auch haben sie nicht an einem Menschen ges�ndigt, sondern an unserm Herrn Jesus Christus
selbst, dem Sohn des lebendigen Gottes. 
\kapR{2}Und dennoch hei�t es, da� die, die von
der ganzen Welt verurteilt und von der ganzen Kirche gebrandmarkt wurden, nun aufgenommen
werden\footnoteA{Vgl. Dok. \ref{ch:Jerusalem335}.}, wor�ber meiner Meinung nach auch ihr
euch, wenn ihr davon erfahrt, zu Recht beschweren m��tet. Welche sind es also, die die Synode
entehren? Etwa nicht die, die die Abstimmung der 300 Personen f�r nichtig erachten und die
Gottlosigkeit der Gottesfurcht vorziehen?
\kapR{3}Denn die H�resie der Ariomaniten wurde von allen Bisch�fen �berall verurteilt und
verboten, die Bisch�fe Athanasius und Markell\footnoteA{Vgl. Dok.
\ref{sec:MarkellJulius}.} aber haben ziemlich viele, die f�r sie aussagen und schreiben. Denn �ber
Markell wurde uns bezeugt, da� er auch auf der Synode in Nicaea denen, die arianisch dachten,
widerstand\footnoteA{Vgl. Dok. \ref{sec:MarkellJulius},2 Anm.}, von Athanasius aber wurde
uns bezeugt, da� er auch in Tyrus nicht verurteilt wurde, in der Mareotis aber nicht
anwesend war, wo das belastende Material gegen ihn herstammt, wie es hei�t. Ihr wi�t aber,
Geliebte, da� einseitiges Material keine Beweiskraft hat, sondern verd�chtig erscheint.
\kapR{4}Und obwohl es sich so verh�lt, haben wir dennoch aus Sorgfalt weder euch noch denen, die
f�r sie geschrieben haben, den Vorrang gegeben und die, die geschrieben haben,
aufgefordert zu kommen, da es ja ziemlich viele sind, die f�r sie geschrieben haben, damit auf einer
Synode alles untersucht wird, auf da� weder der Unschuldige verurteilt noch der
Verd�chtigte f�r unschuldig gehalten werde. Nicht von uns wird also eine Synode entehrt,
sondern von denen, die ohne Grund und nach Belieben die Arianer, die von allen verurteilt
worden sind, auch gegen das Urteil der Richtenden aufnehmen. Denn die meisten (der Richter)
sind schon mit Christus erl�st, die anderen aber werden auch jetzt noch in diesem Leben
gepr�ft und sind unwillig dar�ber, da� einige ihr Urteil aufgehoben haben. 
\pend
\pstart
\kapR{24,1}Dies wissen wir aber auch von denen, die in Alexandrien waren (es ist n�mlich
auch ein gewisser Carpones\footnoteA{Vgl. Dok. \ref{sec:4b} = Urk. 4b,6.}, der von Alexander wegen der
arianischen H�resie exkommuniziert wurde, zusammen mit einigen, die auch selbst wegen derselben
H�resie exkommuniziert wurden, hierhergekommen, da ein gewisser Gregor sie geschickt hatte), und
gleiches haben wir von dem Presbyter Macarius und den Diakonen Martyrius und Hesychius
erfahren. Bevor n�mlich die Presbyter des Bischofs Athanasius herkamen, forderten sie uns auf,
einem Pistus\footnoteA{Pistus wurde von Alexander exkommuniziert, vgl. Dok. \ref{sec:4a} = Urk. 4a,2; seine
Einsetzung bezeugt auch ep.\,encycl. 6; vgl. Dok. \ref{sec:BriefJulius}, Einleitung.} in
Alexandrien zu schreiben, w�hrend doch Athanasius der Bischof in Alexandrien war.
\kapR{2}Die gegenw�rtigen Presbyter des Athanasius wiesen aber darauf hin, da� dieser
Pistus ein vom Bischof Alexander und der nicaenischen Synode exkommunizierter Arianer sei,
der aber von einem gewissen Secundus\footnoteA{Secundus von Ptolema"is, ein in Nicaea exkommunizierter
Anh�nger des Arius; vgl. Dok. \ref{ch:23} = Urk. 23,5; Ath., syn. 12,3; h.\,Ar. 65,3; 71,4; ep.\,Aeg.\,Lib. 19,8.},
den die gro�e Synode als Arianer exkommunizierte, eingesetzt worden sei. In diesem Punkt
widersprachen aber nicht einmal die um Martyrius selbst, auch leugneten sie nicht, da� Pistus von
Secundus die Weihe empfangen habe. �berlegt also auch vor diesem Hintergrund, wer wohl
zu Recht getadelt werden d�rfte, wir, die nicht �berredet wurden, dem Arianer Pistus zu
schreiben, oder die, die beschlossen hatten, die gro�e Synode zu entehren und den
Gottlosen wie Gottesf�rchtigen zu schreiben? 
\kapR{3}Denn auch der Presbyter Macarius,
der von Eusebius samt denen um Martyrius geschickt worden war, reiste nachts, obwohl er
krank war, zusammen mit denen um Martyrius und Hesychius ab, als er h�rte, da� die
Presbyter des Athanasius herkamen, obwohl wir seine Ankunft erwarteten, so da� wir daraus
schlie�lich zu Recht vermutet haben, da� er sich entfernte, da er sich �ber die
Widerlegung des Pistus sch�mte.
\kapR{4}Es ist n�mlich unm�glich, da� die Weihe des Arianers Secundus in der katholischen
Kirche G�ltigkeit besitzt. Denn dies ist wahrlich eine Entehrung der Synode und der auf
ihr versammelten Bisch�fe, wenn das, was mit so gro�em Eifer und Fr�mmigkeit gleichsam in
Gegenwart Gottes geschah, als nichtig aufgel�st wird.
\pend
\pstart
\kapR{25,1}Wenn es also, wie ihr schreibt, n�tig ist, da� die Beschl�sse der Synoden wie
bei Novatian\footnoteA{Vgl. Dok. \ref{sec:BriefSynode341},1.} und Paulus von
Samosata\footnoteA{Vgl. Dok. \ref{ch:Konstantinopel336},3,4.} bestehen bleiben, so h�tte
vielmehr die Abstimmung der Dreihundert nicht aufgel�st und die katholische Synode
nicht von so wenigen entehrt werden d�rfen. Denn genau wie jene sind die Arianer H�retiker, 
und die Beschl�sse gegen diese sind denen gegen jene vergleichbar. Da aber dies gewagt
worden ist, wer sind die, die den Brand der Streitereien wieder anz�ndeten? Denn ihr habt
uns geschrieben und getadelt, da� wir Derartiges getan haben. Haben wir also Streitereien
angezettelt, weil wir Mitleid haben mit den leidenden Br�dern und in allem nach dem Gesetz
gehandelt haben? Oder die, die den Streit lieben und gegen das Gesetz den Beschlu� der
Dreihundert aufhoben und in allem die Synode entehrten?
\kapR{2}Denn es wurden nicht nur die Arianer aufgenommen, sondern die Bisch�fe haben sich
auch daran gew�hnt, von Ort zu Ort zu wechseln. Wenn ihr also wirklich die Ehre der
Bisch�fe f�r ein und dieselbe haltet und die Bisch�fe nicht nach der Gr��e der St�dte
beurteilt, wie ihr schreibt, so h�tte der, dem eine kleine anvertraut worden ist, in der
ihm anvertrauten (Stadt) bleiben sollen und nicht die anvertraute Stadt verachten und zu
einer ihm nicht �bergebenen wechseln sollen, wodurch er die von Gott �bergebene verachtet, aber
die menschliche Ruhmsucht liebt\footnoteA{Das zielt auf Eusebius, der von Nikomedien nach
Konstantinopel wechselte (vgl. apol.\,sec. 6,6 und schon Dok. \ref{sec:4b} = Urk. 4b,4, vgl. auch Kanon 15 von Nicaea).}. 
\kapR{3}Ihr h�ttet
also, Geliebte, herkommen und nicht ausweichen sollen, damit die Sache ein Ende nimmt.
Denn das gebietet die Vernunft. Aber vielleicht hinderte euch der Termin. Denn ihr habt in
dem Brief getadelt, da� wir einen kurzfristigen Termin f�r die Synode festgelegt haben,
aber auch dies ist, Geliebte, eine Ausrede. Denn wenn der Tag Leute, die kommen, gehindert h�tte, so w�re dies ein Beweis f�r die zu kurze Zeitspanne; wenn aber die, die
nicht kommen wollten, sogar die Presbyter bis zum Januar selbst festgehalten haben, so handelt es
sich um eine Ausrede derer, die keine Zuversicht haben.
\kapR{4}Denn sie w�ren gekommen, wie gesagt, wenn sie Zuversicht gehabt h�tten, ohne auf
den langen Weg zu achten oder auf den Termin zu schauen, sondern im Vertrauen auf gerechte und vern�nftige Leute. Aber vielleicht reisten sie nicht wegen der
Zeitumst�nde her. Denn auch dies habt ihr geschrieben und aufgezeigt, da� wir nicht von
euch verlangen sollten herzureisen, da ihr die Lage im Osten beachten m��tet. Wenn ihr
aber wegen der derartigen Umst�nde nicht gekommen seid, wie ihr sagt, so h�ttet ihr
deswegen zuerst darauf achten sollen, da� ihr keine Spaltung, Jammern oder Klagen in den Kirchen verursacht. 
\kap{5}Jetzt aber haben die, die so gehandelt haben, gezeigt, da�
nicht die Zeitumst�nde, sondern der Entschlu� der Grund ist, da� sie nicht herreisen
wollten.
\pend
\pstart
\kapR{26,1}Ich wundere mich aber auch �ber jene Stelle des Briefes, worin es hei�t, da� ich
nur denen um Eusebius und nicht euch allen geschrieben habe, obwohl er doch ganz und gar f�r euch geschrieben wurde. Daran ersieht man doch wohl eher
den Leichtsinn der Tadelnden als deren Wahrhaftigkeit. Da ich n�mlich nicht anderswoher 
Briefe gegen Athanasius empfangen hatte, sondern nur durch die um Martyrius und Hesychius,
habe ich gezwungenerma�en jenen geschrieben, die auch gegen ihn geschrieben hatten. Es
h�tten also entweder die um Eusebius nicht alleine ohne euch alle schreiben sollen, oder ihr,
denen ich nicht geschrieben habe, h�ttet euch nicht dar�ber �rgern sollen, da� ich jenen
geschrieben habe, die auch schrieben. Wenn es n�mlich erforderlich gewesen w�re, euch alle
anzuschreiben, h�ttet auch ihr zusammen mit jenen schreiben sollen. 
\kapR{2}Nun aber
haben wir konsequenterweise jene ber�cksichtigt und ihnen geschrieben, die sich uns vorgestellt haben und an uns
herangetreten sind. Wenn es euch aber gereizt hat, da� ich nur jenen geschrieben habe, so
m��t ihr euch folglich auch dar�ber beschweren, da� sie auch an mich allein geschrieben
haben. Aber auch diesbez�glich gibt es einen hinreichenden und vern�nftigen Anla�,
Geliebte, da ich euch darauf hinweisen mu�, da�, auch wenn nur ich geschrieben habe, diese
Ansicht nicht nur die meinige ist, sondern die aller Bisch�fe Italiens und in diesen
Gegenden. Und ich wollte nun nicht alle veranlassen zu schreiben, damit sie nicht von
so viel bel�stigt werden.
\kapR{3}Die Bisch�fe sind �brigens genau zum festgesetzten Termin zusammengekommen und
unterst�tzen diese Position, die wiederum ich euch schriftlich mitteile, so da� ihr,
Geliebte, auch wenn nur ich schreibe, erkennt, da� diese Meinung die aller ist. Soweit also
dazu, da� einige von euch nicht vern�nftige, sondern ungerechtfertigte und
verd�chtige Entschuldigungen vorbringen.
\pend
\pstart
\kapR{27,1}Da� wir aber keineswegs leichtfertig und zu Unrecht unsere Mitbisch�fe
Athanasius und Markell in die Gemeinschaft aufgenommen haben, zeigen wir euch, auch wenn
das zuvor Gesagte ausreicht, dennoch berechtigterweise mit wenigen Worten auf. Zuerst haben die um
Eusebius gegen die um Athanasius geschrieben, jetzt aber habt auch ihr geschrieben; es haben
aber auch sehr viele Bisch�fe aus �gypten und den anderen Provinzen f�r Athanasius geschrieben.
\kapR{2}Erstens also widersprechen sich eure Schriftst�cke gegen ihn untereinander, und
das zweite stimmt nicht mit dem ersten �berein, sondern das erste wird in vielen Belangen
vom zweiten aufgehoben und das zweite vom ersten verleumdet. Wenn aber die Schriftst�cke
nicht �bereinstimmen, so besteht kein Vertrauen in die Aussagen. Wenn ihr ferner fordert,
da� man euren Schreiben glaubt, so darf man folglich auch denen, die f�r ihn geschrieben
haben, nicht mi�trauen, und zwar besonders da ihr weit weg seid, jene aber vor Ort sind,
den Mann kennen und �ber die dortigen Ereignisse schreiben, Zeugnis �ber sein Leben
ablegen und bekr�ftigen, da� er in allem verleumdet worden ist.
\kapR{3}Und es wurde einmal von einem gewissen Bischof Arsenius\footnoteA{Zu dem Fall des
Arsenius vgl. Soz., h.\,e. II 25,12: ein Athanasius nahestehender Bischof Plusanius habe
Arsenius mi�handelt und dessen Haus niedergebrannt. Die Melitianer wollten Athanasius mit
diesem Vorfall schaden, aber nachdem der totgeglaubte Arsenius lebendig wiedergefunden
wurde, blieb die Klage gegenstandslos. Vgl. dazu auch Ath., apol.\,sec. 8,4 (aus dem
Schreiben des Synode der Alexandriner 338 n.\,Chr. an Julius von Rom �ber Arsenius); 63,4;
65; 69 (Brief des Arsenius an Athanasius).} berichtet, da� er von Athanasius get�tet
worden sei, aber wir erfuhren, da� dieser lebt und sogar mit ihm befreundet ist.
\kapR{4}Er best�tigte, da� die Unterlagen, die in der Mareotis entstanden, einseitig
seien; denn weder der angeklagte Presbyter Macarius\footnoteA{Zu Macarius vgl. Anm. zu
� 33.} noch sein Bischof Athanasius seien dort selbst anwesend gewesen. Und dies wissen
wir nicht nur durch seine Worte, sondern auch durch die Unterlagen, die uns die um
Martyrius und Hesychius gebracht haben. W�hrend wir sie lasen, fanden wir n�mlich heraus,
da� zwar der Ankl�ger Ischyras dort war, weder aber Macarius noch der Bischof Athanasius,
und obwohl sogar die Presbyter des Athanasius darum baten, dabei zu sein, wurden sie nicht zugelassen.
\kapR{5}Es h�tte aber, Geliebte, wenn der Wahrheit entsprechend Recht gesprochen werden
sollte, nicht alleine der Ankl�ger, sondern auch der Beklagte anwesend sein m�ssen. Denn wie 
in Tyrus der Beklagte Macarius und der Ankl�ger Ischyras anwesend waren, und nichts
bewiesen wurde, so h�tte auch in die Mareotis nicht nur der Kl�ger reisen sollen, sondern
auch der Beklagte, damit er entweder in Anwesenheit �berf�hrt w�rde oder sich, wenn er nicht
�berf�hrt wird, die Verleumdung erwiese. Nun ist dies aber nicht geschehen, sondern allein
der Ankl�ger zusammen mit denen, die Athanasius ablehnen, sind abgereist, so da� die
Angelegenheit Verdacht erweckt.
\pend
\pstart
\kapR{28,1}Er beschwerte sich aber auch dar�ber, da� die, die in die Mareotis gereist
waren, gegen seine Meinung fortgereist seien. Er berichtete n�mlich, da� sie Theognis,
Maris, Theodorus, Ursacius, Valens und Macedonius\footnoteA{Theognis von Nicaea, Maris von
Chalcedon, Theodorus von Heraclea in Thrakien, Ursacius von Singidunum, Valens von Mursa
und Macedonius von Mopsuestia. Es handelt sich um die sog. Mareotis-Kommission (vgl. Ath.,
apol.\,sec. 13,2; 72,4; 73; 74~f.; 76,2; 77), die von der Synode in Tyrus zur Untersuchung
der Anklagen gegen Athanasius nach �gypten geschickt wurde. Theognis verweigerte wie Eusebius
von Nikomedien die Unterschrift unter das Nicaenum und wurde von Konstantin nach Gallien
verbannt, konnte nach nachtr�glicher Zustimmung zum Nicaenum (ohne
Anathematismen) zur�ckkehren (Dok. \ref{ch:31} = Urk. 31), blieb aber Gegner des Athanasius und wirkte an
seiner Verurteilung mit (vgl. Dok. \ref{ch:TheognisNiz}; \ref{sec:SerdicaRundbrief},6).
Auch Maris unterst�tzte vor Nicaea Arius gegen dessen Ausschlu� aus Alexandrien (Ath., syn.
17,1) und geh�rte sp�ter zu der Delegation, die die sog. vierte antiochenische Formel nach
Gallien �bermittelte (syn. 25,1; vgl. Dok. \ref{ch:AntIV}), ist ferner auch Adressat
dieses Schreibens von Julius. Dieser sp�teren Delegation nach Gallien geh�rte auch Theodorus
an, der schlie�lich von der ">westlichen"< Synode von Serdica exkommuniziert werden sollte
(vgl. Dok. \ref{sec:SerdicaRundbrief},14; 16), aber noch 351 an der Synode in Sirmium
teilnahm (vgl. Hil., coll.\,antiar. B VII 9; auch Soz., h.\,e. IV 8). Auch Theodorus ist
Adressat dieses Schreibens des Julius und wird in den Berichten �ber die antiochenische
Synode als prominenter Teilnehmer genannt (Dok. \ref{sec:BerichteAntiochien341},2,5).
Macedonius, ebenfalls Adressat des Juliusbriefes, �bermittelte 344 die sog. Ekthesis
makrostichos (vgl. Dok. \ref{ch:Makrostichos} Einleitung). Zu Valens und Ursacius vgl.
Dok. \ref{sec:SerdicaWestBekenntnis},2.}, die unter Verdacht standen, losgeschickt hatten.
Und dies belegte er nicht nur mit seinen Worten, sondern auch mit einem Brief des
Alexander, der Bischof von Thessalonike war\footnoteA{Brief in Ath., apol.\,sec. 80; vgl.
auch apol.\,sec. 16,1; 17,1.}. Er legte n�mlich einen Brief von ihm vor, den er an
Dionysius, den Comes bei der Synode,\footnoteA{Der Comes Flavius Dionysius war nach Eus., v.\,C. IV 42,3 (= Thdt., h.\,e. I 29); Ath., apol.\,sec. 71,2 und Socr., h.\,e. I 28.31 von staatlicher Seite f�r die Durchf�hrung der Synode zust�ndig. Er ist Empf�nger der Briefe apol.\,sec. 78, 79 und 80 und Verfasser des Schreibens apol.\,sec. 81.} geschrieben hatte, in dem er klarstellt, da�
offensichtlich eine Verleumdung gegen Athanasius in Gange sei. 
\kapR{2}Er brachte sogar einen echten, handgeschriebenen Brief des Ankl�gers Ischyras selbst,\footnoteA{Der Fall des Ischyras und sein Konflikt mit dem Presbyter des
Athanasius Macarius war ein Hauptanklagepunkt, der zur Absendung der Mareotis-Kommission
und zur Absetzung des Athanasius in Tyrus 335 n.\,Chr. f�hrte. Der erw�hnte Brief steht in
Ath., apol.\,sec. 64; zur Sache vgl. Soz., h.\,e. II 25; Ath., apol.\,sec. 63,2--5 und die
Dokumente in apol.\,sec. 73--85.} mit, in dem er den allm�chtigen Gott als Zeugen anruft und
sagt, da� weder ein Kelch zerbrochen noch ein Tisch umgeworfen worden sei, sondern er
selbst sei von einigen dazu angestiftet worden, diese Anklage zu erfinden. Aber auch
Presbyter der Mareotis, die hergereist waren, versicherten, da� weder Ischyras ein
Presbyter der katholischen Kirche sei noch Macarius Derartiges angestellt habe, wie jener
unterstellt habe.
\kapR{3}Die Priester aber und die Diakone, die hierher gekommen waren, bezeugten nicht
wenig, sondern sehr viel f�r den Bischof Athanasius und versicherten, da� nichts von dem,
was gegen ihn vorgebracht wird, wahr sei, sondern er verleumdet worden sei. Und alle Bisch�fe aus �gypten und Libyen best�tigten schriftlich,
da� seine Weihe rechtens und kirchlich geschehen sei und da� alles, was von euch gegen ihn
vorgetragen wurde, eine L�ge sei. Denn weder sei ein Mord geschehen oder seinetwegen
gemordet worden, noch sei ein Kelch zerbrochen worden, sondern alles sei eine L�ge.
\kapR{4}Und aus den in der Mareotis einseitig gesammelten Unterlagen zeigte der Bischof
Athanasius, da� ein Katechumene befragt wurde und aussagte, da� er mit Ischyras drinnen
gewesen sei, als der Presbyter des Athanasius, Macarius, wie sie sagen, an dem Ort war. Auch 
andere aber wurden befragt und einer sagte, er sei in einem kleinen Zimmer gewesen, ein
anderer, da� Ischyras hinter der T�r lag und damals krank gewesen sei, als sich Macarius,
wie sie sagen, dort aufhielt.
\kapR{5}Wegen dieser Angaben, die er mitteilte, stellten konsequenterweise auch wir
�berlegungen an: Wie war es m�glich, da� einer, der krank hinter der T�r lag,
damals stehen, die Liturgie halten und opfern konnte? Oder wie war es m�glich, da�
geopfert wurde, w�hrend sich drinnen die Katechumenen aufhielten? Denn wenn die
Katechumenen drinnen waren, war noch nicht der Zeitpunkt f�r das Opfer.
\kapR{6}Dies trug, wie gesagt, der Bischof Athanasius vor und belegte aus den Unterlagen,
auch unterst�tzt von seinen Begleitern, da� jener �berhaupt niemals in der katholischen Kirche Presbyter gewesen sei noch sich jemals als Presbyter an einer kirchlichen
Versammlung beteiligt habe. Denn auch nicht, als Alexander der Menschenliebe der
gro�en Synode entsprechend die Anh�nger des Schismatikers Melitius aufnahm, sei er von Melitius zu den
Seinigen gerechnet worden, so bekr�ftigten sie. Und dies ist der gr��te Beweis, da� er
nicht einmal zu Melitius geh�rte. Denn wenn es so w�re, dann h�tte er freilich auch selbst dazugerechnet werden
m�ssen. Dar�berhinaus wurden von Athanasius auch in anderen Sachen durch die Unterlagen
die L�gen des Ischyras nachgewiesen. Denn da er geklagt hatte, da� B�cher verbrannt
wurden, als, wie sie sagen, Macarius anwesend gewesen sei, wurde er als L�gner �berf�hrt
von denen, die er selbst als Zeugen vorbrachte.
\pend
\pstart
\kapR{29,1}Da also diese Dinge so ausgesprochen wurden und es so viele Zeugen f�r ihn gab
und so vieles Rechtfertigendes von ihm vorgebracht wurde, was h�tten wir tun sollen? Oder
was schreibt das kirchliche Gesetz vor, au�er den Mann nicht zu verurteilen, sondern ihn
vielmehr aufzunehmen und als Bischof zu behandeln, wie wir es auch taten? 
\kapR{2}Denn
dar�ber hinaus blieb er hier ein Jahr und sechs Monate\footnoteA{Vgl. die Einleitung zu
Dok. \ref{sec:BriefJulius}.} und wartete auf eure Ankunft oder die derer, die kommen wollten.
Durch seine Anwesenheit aber besch�mte er alle, denn er w�re nicht da, h�tte er keine
Zuversicht. Denn er ist nicht von selbst gekommen, sondern weil er von uns gerufen wurde
und Briefe empfangen hatte, wie wir auch euch geschrieben haben.
\kapR{3}Und dennoch habt ihr nach alledem uns getadelt, als ob wir gegen die Gesetze
versto�en h�tten. Betrachtet jetzt, wer es ist, der gegen die Gesetze versto�en hat, wir,
die wir nach so vielen Beweisen den Mann aufgenommen haben, oder die, die 36 Stationen
entfernt in Antiochien einen Fremden gleichsam zum Bischof ernannten und ihn mit milit�rischer
Vollmacht nach Alexandrien schickten?\footnoteA{Gemeint ist die Einsetzung des Gregor, s.\,u.
� 42 und Einleitung zu Dok. \ref{sec:BriefJulius} und Dok.
\ref{sec:BriefSerdikaAlexandrien},16.} Dies geschah nicht einmal, als er nach Gallien
geschickt wurde.\footnoteA{Gemeint ist das erste Exil des Athanasius in Trier, als kein anderer Bischof in Alexandrien eingesetzt worden war.} Denn es w�re auch damals passiert, wenn er tats�chlich verurteilt worden
w�re. Er fand in der Tat bei seiner R�ckkehr die Kirche einsam vor, die auf ihn wartete.
\pend
\pstart
\kapR{30,1}Jetzt aber wei� ich nicht, wie die Ereignisse abgelaufen sind. Erstens
n�mlich, wenn ich die Wahrheit sagen soll, h�tte man nicht, w�hrend wir zu einer Synode
einluden, das Urteil der Synode vorwegnehmen sollen. Ferner h�tte nicht eine derartige
Neuerung im Widerspruch zur Kirche geschehen d�rfen. Denn was f�r ein kirchliches Gesetz
oder was f�r eine apostolische �berlieferung ist das, da� man, obwohl die Kirche Frieden
hat und so viele Bisch�fe in Harmonie mit Athanasius, dem Bischof von Alexandrien, leben, Gregor schickt, 
der Stadt fremd, weder dort getauft noch den
meisten bekannt, nicht hergebeten von den Presbytern, nicht von den Bisch�fen, nicht vom
Volk, sondern in Antiochien geweiht, nach Alexandrien geschickt, aber nicht mit
Presbytern, nicht mit Diakonen der Stadt, nicht mit Bisch�fen aus �gypten, sondern mit
Soldaten? 
\kapR{2}Das erz�hlten und beklagten n�mlich die, die hierhergekommen waren.
Denn auch wenn man nach der Synode den Athanasius f�r schuldig befunden h�tte, so h�tte
die Weihe nicht derartig gesetzwidrig und im Widerspruch zu den kirchlichen Gesetzen
geschehen d�rfen, sondern die Bisch�fe in der Provinz h�tten aus der eigenen Kirche, aus
der eigenen Priesterschaft, aus dem eigenen Klerus die Weihe vollziehen m�ssen,\footnoteA{Vgl. Kanon 4 von Nicaea.} und zwar
ohne nun die Gesetze der Apostel zu �bertreten. Wenn n�mlich gegen einen von euch so
vorgegangen worden w�re, h�ttet ihr nicht protestiert, h�ttet ihr nicht eine
Bestrafung verlangt, da die Gesetze �bertreten wurden?
\kapR{3}Geliebte, wie vor Gott erkl�ren wir mit Wahrheit und sagen: Das ist nicht
gottesf�rchtig, auch nicht gesetzlich oder kirchlich! Denn auch die Berichte dar�ber, was
von Gregor bei seinem Einzug getan worden ist, zeigen die Qualit�t seiner Weihe. Denn
w�hrend derart friedlicher Zeiten, wie es die selbst berichteten, die aus Alexandrien kamen, wie
es auch die Bisch�fe schrieben, erlebte die Kirche einen Brandsturm, wurden Jungfrauen
entbl��t, Einsiedler niedergetrampelt, Presbyter und viele aus dem Volk wurden mi�handelt
und erlitten Gewalt, Bisch�fe wurden gefangengenommen, viele wurden herumgezerrt, die
heiligen Mysterien, weswegen sie den Presbyter Macarius beschuldigten, wurden von den
Nichtchristen geraubt und auf die Erde geworfen, damit einige die Weihe Gregors
annehmen.
\kapR{4}Derartige Ereignisse aber weisen auf die, die die Gesetze �bertreten. Denn w�re
die Weihe rechtens gewesen, h�tte er nicht die, die ihm zu Recht nicht folgten, mit
rechtswidrigen Mitteln gezwungen zu folgen. Und obwohl diese Ereignisse geschehen sind,
schreibt ihr, da� gro�er Friede in Alexandrien und �gypten geherrscht habe, es sei denn,
da� das Werk des Friedens ge�ndert worden ist und ihr derartige Zust�nde Frieden nennt.
\pend
\pstart
\kapR{31,1}Auch jenes meine ich euch klarstellen zu m�ssen, da� Athanasius bekr�ftigt
hat, Macarius sei von Soldaten in Tyrus bewacht worden und nur der Ankl�ger sei mit der
Gesandtschaft in die Mareotis gereist, und den Presbytern, die baten, bei der Untersuchung
dabei zu sein, wurde dies verweigert, und die Untersuchung �ber Kelch und Tisch
habe vor dem Statthalter und seiner Truppe stattgefunden, in Gegenwart von Heiden und Juden.
Das w�re von Anfang an unglaubw�rdig gewesen, wenn es nicht auch aus den Unterlagen belegt
worden w�re, wor�ber auch ich mich sehr gewundert habe und auch ihr euch meiner Meinung
nach noch wundern werdet, Geliebte. 
\kapR{2}Den Presbytern wird nicht erlaubt, dabei zu
sein, obwohl doch sie die Liturgen der Mysterien sind, aber vor einem fremden Richter
unter Anwesenheit von Katechumenen und, was das Schlimmste ist, vor Heiden und Juden, die
das Christentum verleumden, findet die Untersuchung �ber Blut und Leib Christi
statt. Denn wenn �berhaupt irgendein Vergehen passiert w�re, h�tten diese Dinge in der
Kirche von rechtm��igen Klerikern untersucht werden m�ssen und nicht von Heiden, die das
Wort verabscheuen und die Wahrheit nicht kennen. Dies ist aber die so gro�e und derartige
S�nde, und ich bin �berzeugt, auch ihr alle seht dies ein. Soviel also zu
Athanasius.
\pend
\pstart
\kapR{32,1}�ber Markell\footnoteA{Zu Markell vgl. Dok. \ref{sec:MarkellJulius}.} aber, da
ihr auch �ber ihn geschrieben habt, als ob er gottlos �ber Christus d�chte, ist es mein
Anliegen, euch klarzustellen, da� er, als er hier war, versicherte, da� die Dinge, die
von euch �ber ihn geschrieben wurden, nicht der Wahrheit entspr�chen; gleichwohl wurde
aber, als er von uns gebeten wurde, �ber seinen Glauben zu sprechen, von ihm mit einer
derartigen Freim�tigkeit geantwortet, da� wir erkannten, da� er nichts au�erhalb der
Wahrheit bekannte. 
\kapR{2}Denn er bekannte, so gottesf�rchtig �ber den Herrn und unseren
Erl�ser Jesus Christus zu denken, wie es auch die katholische Kirche tut. Und er
best�tigte, nicht erst jetzt dieses zu denken, sondern schon lange; entsprechend bezeugten auch
unsere Presbyter, die damals auf der Synode von Nicaea waren, seine Rechtgl�ubigkeit. Er
bekr�ftigte n�mlich, sowohl damals als auch jetzt die H�resie der Arianer
zu verachten, weshalb es berechtigt sei, auch euch daran zu erinnern, damit
niemand die derartige H�resie annehme, sondern sie verabscheue als einen Fremdk�rper in
der gesunden Lehre.
\kapR{3}Da er also richtig denkt und als rechtgl�ubig bezeugt worden ist, wie h�tten wir
ihn anders behandeln sollen, denn als Bischof, wie wir es auch taten, und
ihn nicht von der Gemeinschaft ausschlie�en?
\kapR{4}Dies habe ich nun nicht geschrieben, um sie zu verteidigen, sondern
damit ihr glaubt, da� wir die M�nner nach Recht und Gesetz aufgenommen haben und da� ihr
umsonst streitet; es ist aber angemessen, da� ihr euch bem�ht und alle Hebel in Bewegung
setzt, damit die ungesetzlichen Ereignisse eine Korrektur erfahren, die Kirchen aber Frieden
haben, auf da� der Friede des Herrn, der uns �bergeben worden ist, erhalten bleibt, die
Kirchen sich nicht entzweien und euch nicht der Tadel verbleibt, ihr habet die
Entzweiung verursacht. Ich bekenne euch n�mlich, da� die Ereignisse keine Veranlassung zum
Frieden, sondern zur Entzweiung bieten.
\pend
\pstart
\kapR{33,1}Denn nicht nur die Bisch�fe um Athanasius und Markell sind hierhergekommen
und haben sich dar�ber beschwert, da� ihnen Unrecht widerfahren sei, sondern auch viele
andere Bisch�fe aus Thracia, Syria coele, Phoenice und Palaestina.\footnoteA{Thracia:
Lucius von Adrianopel (Socr., h.\,e. II 15,2; Soz., h.\,e. III 8,1; Dok. \ref{sec:RundbriefSerdikaOst},10). Syria coele: Cyrus von Beroea (Socr., h.\,e. I 24,3; Ath., h.\,Ar.
5,2); Euphration von Balaneae (Dok. \ref{ch:3} = Urk. 3; Ath., h.\,Ar. 5,2). Phoenice: Hellanicus von Tripolis (h.\,Ar.
5,2). Palaestina: Asclepas von Gaza (Socr., h.\,e. II 15,2; Soz., h.\,e. III 8,1; Dok.
\ref{sec:Asclepas}).} Auch nicht wenige Presbyter, einige aus Alexandrien, andere aus
anderen Gegenden, sind hierher zur Synode gekommen und beklagten sich vor allen
anwesenden Bisch�fen dar�ber auch noch zus�tzlich zu anderen Dingen, die sie vortrugen, da� die
Kirchen Gewalt und Unrecht erlitten, und best�tigten nicht nur mit Worten, sondern auch
mit Tatsachen, da� �hnliche Geschehnisse wie in Alexandrien auch in ihren und in anderen
Kirchen vorgefallen seien.
\kapR{2}Auch aus �gypten und Alexandrien kamen jetzt nochmals Presbyter mit
Schriftst�cken und beklagten, da� viele Bisch�fe und Presbyter, die zur Synode kommen
wollten, daran gehindert worden waren. Sie berichteten n�mlich, da� bis jetzt, auch nach
der Abreise des Bischofs Athanasius, bekennende Bisch�fe geschlagen, andere verhaftet
wurden, und sogar auch alte, die schon sehr lange Jahre im Bischofsamt waren, zu �ffentlichen
Diensten verpflichtet und fast alle Kleriker und Laien der katholischen Kirche bedroht und
verfolgt wurden. Denn sie sagten, da� auch einige Bisch�fe und Br�der ausgesto�en wurden aus
keinem anderen Grund, als um sie gegen ihren Willen zu zwingen, mit Gregor und den
Arianern bei ihm Gemeinschaft zu halten. 
\kapR{3}Und da� in Ancyra in Galatia nicht
Geringf�giges, sondern genau dasselbe geschehen sei, h�rten wir wiederum von denen, die aus Alexandrien
herkamen, und von andern; auch der Bischof Markell bezeugte es. Zus�tzlich dazu haben
die, die hergereist waren, gegen einige von euch, um nicht ihre Namen zu nennen, derartige
und so f�rchterliche Anklagen erhoben, die ich nicht niederschreiben mag. Aber vielleicht
habt ihr sie auch von anderen geh�rt.
\kapR{4}Deshalb habe ich euch n�mlich extra geschrieben und gebeten zu kommen, damit ihr es
pers�nlich h�rt und alles gerichtet und geheilt werden kann. Deswegen h�tten die
Gerufenen bereitwillig herkommen und nicht ablehnen sollen, damit niemand glaubt, sie seien 
verd�chtig mit ihren Aussagen, und damit niemand glaubt, sie k�nnten nicht beweisen,
was sie schreiben, wenn sie nicht kommen.
\pend
\pstart
\kapR{34,1}Da diese Dinge also in dieser Form berichtet wurden und die Kirchen derart
litten und bedroht wurden, wie die Gesandten versicherten~-- wer hat da den Brandherd der
Zwietracht entz�ndet? Wir, die wir dar�ber ver�rgert sind und mit den leidenden Br�dern
mitleiden, oder die, die derartiges angestellt haben? Ich wundere mich n�mlich, weshalb ihr
schreibt, in den Kirchen herrsche Eintracht, obwohl dort in jeder Kirche ein derartiger und
so gro�er Aufruhr herrscht, weshalb auch die, die hergereist sind, hierher kamen. Derartiges
geschieht aber nicht zur Erbauung der Kirche, sondern zu ihrer Zerst�rung. Und die sich
dar�ber freuen, sind nicht S�hne des Friedens, sondern des Aufruhrs. Unser Gott ">ist aber
nicht einer des Aufruhrs, sondern des Friedens."< 
\kapR{2}Deshalb, wie der Gott und Vater
unseres Herrn Jesus Christus wei�, weil ich mich um eure Reaktion gesorgt habe, aber auch
f�r die Kirchen gebetet habe, da� sie nicht in Aufruhr stecken, sondern so bleiben, wie es
von den Aposteln eingerichtet worden ist, habe ich geglaubt, es sei notwendig, euch dies
zu schreiben, damit ihr endlich einmal die verachtet, die durch ihre gegenseitige
Feindschaft die Kirchen in diese Lage gebracht haben. Ich habe n�mlich geh�rt, da� es nur
wenige sind, die alle diese Ereignisse verursacht haben.
\kapR{3}Bem�ht euch, die ihr ja ein Herz des Mitleids habt, die ungesetzlichen Ereignisse in
Ordnung zu bringen, wie gesagt, damit, auch wenn etwas �bergangen wurde, dies durch euren
Einsatz wieder gutgemacht wird. Und schreibt nicht: ">Du hast die Gemeinschaft mit
Markell und Athanasius der mit uns vorgezogen"<, denn solcher Art sind die Kennzeichen nicht des
Friedens, sondern der Streitsucht und des Bruderhasses. Deshalb habe auch ich euch nun das
obige geschrieben, damit ihr wi�t, da� wir sie nicht zu Unrecht aufgenommen haben, und damit
ihr mit einem solchen Streit aufh�rt.
\kapR{4}Wenn ihr n�mlich hergekommen w�ret und sie verurteilt worden w�ren, wenn sie
augenscheinlich keine guten Beweise f�r sich h�tten vorlegen k�nnen, h�ttet ihr derartiges
zu Recht geschrieben. Da wir aber, wie gesagt, korrekterweise und nicht zu Unrecht mit
ihnen Gemeinschaft haben, bitte ich euch um Christi Willen, la�t euch nicht dazu herab,
die Glieder Christi zu zerrei�en, und glaubt nicht den Vorurteilen, sondern ehrt vor allem
den Frieden des Herrn. 
\kapR{5}Denn es ist nicht angebracht und auch nicht rechtens,
wegen der Engstirnigkeit einiger die nicht Verurteilten hinauszuwerfen und damit den Geist
zu betr�ben. Wenn ihr aber meint, jemand k�nnte etwas gegen sie vorweisen und sie von Angesicht zu Angesicht �berf�hren, so sollen die, die das wollen, herkommen. Denn sie erkl�rten sich auch selbst bereit, das aufzuzeigen und zu widerlegen, wor�ber sie uns berichtet haben.
\pend
\pstart
\kapR{35,1}Kl�rt uns also dar�ber auf, Geliebte, damit wir jenen schreiben und auch an
die Bisch�fe, die wieder zusammenkommen m�ssen, damit in Anwesenheit aller die Schuldigen
verurteilt werden und kein Aufruhr mehr in den Kirchen herrscht. Es ist n�mlich genug
geschehen. Es gen�gt, da� in Anwesenheit von Bisch�fen Bisch�fe herausgeworfen wurden.
Dar�ber sollte man nicht viele Worte verlieren, damit es nicht so scheint, als belaste man
die damals Anwesenden. Denn wenn man die Wahrheit sagen soll, so h�tte man es gar nicht
erst soweit kommen lassen und die Engstirnigkeit nicht soweit treiben lassen sollen.
\kapR{2}Es wurden also, wie ihr schreibt, Athanasius und Markell von ihren eigenen Orten
entfernt; was soll man auch zu den anderen sagen, Bisch�fen und Presbytern, die, wie gesagt, aus
unterschiedlichen Gegenden hierhergekommen waren? Auch sie berichteten n�mlich wiederum, da�
sie verschleppt worden seien und Derartiges erlitten h�tten.
\kapR{3}Oh Geliebte, nicht mehr f�r das Evangelium, sondern nur zur Verbannung und zum
T�ten gibt es die kirchlichen Gerichte. Denn wenn �berhaupt, wie ihr sagt, bei ihnen eine S�nde
vorgefallen ist, dann h�tte nach kirchlichem Gesetz und nicht auf diese Weise das Urteil
gef�llt werden m�ssen, es h�tte an uns alle geschrieben werden m�ssen, damit auf diese
Weise von allen eine gerechte Entscheidung gefunden worden w�re. Denn es waren Bisch�fe, denen Leid
zugef�gt wurde, und nicht irgendwelche Kirchen erfuhren Leid, sondern welche, denen die
Apostel selbst vorstanden.
\kapR{4}Warum wurde uns aber gerade �ber die Kirche Alexandriens nicht geschrieben? Oder
wi�t ihr nicht, da� es �blich war, zuerst uns
zu schreiben und so von hier aus Recht zu sprechen?\footnoteA{Julius beansprucht, da� Rom viel eher als
Antiochien die entscheidende Instanz sei, um �ber Angelegenheiten in Alexandrien zu
entscheiden, kann aber eigentlich nur auf die traditionell engen Beziehungen zwischen Rom
und Alexandrien hinweisen. Ein entsprechender Rechtsstatus ist nicht gegeben.} Wenn also eine derartige Verd�chtigung
gegen den dortigen Bischof erhoben wurde, h�tte an die hiesige Kirche geschrieben werden
m�ssen; jetzt aber wollen die, die uns nicht �berzeugt haben, selbst aber handeln, wie sie
wollen, da� nunmehr auch wir, die wir nichts bemerkt haben, ihnen zustimmen.
\kapR{5}So waren
nicht die Anordnungen des Paulus, so haben es uns nicht die V�ter �berliefert; dies ist
eine andere Art und eine neue Sitte. Ich bitte euch, nehmt es langm�tig auf; es dient dem
gemeinsamen Nutzen, was ich schreibe. Denn was wir vom seligen Apostel Petrus �berliefert
haben, das erkl�re ich auch euch. Und ich w�rde es nicht schreiben, da ich der �berzeugung bin,
dies ist bei allen bekannt, wenn nicht die Ereignisse uns verwirrt h�tten.
\kapR{6}Bisch�fe werden geraubt und vertrieben, andere aber werden von anderswoher an die
Stelle gesetzt und andere werden bedroht, so da� sie �ber die Geraubten trauern, von den
Geschickten aber in die Enge getrieben werden, damit sie die, die sie wollen, nicht
suchen, die aber, die sie nicht wollen, aufnehmen.
\kapR{7}Ich bitte euch, da� Derartiges nicht mehr geschieht, schreibt besser gegen die,
die Derartiges anstreben, damit die Kirchen nicht mehr Derartiges erleiden und nicht
irgendein Bischof oder Presbyter eine Mi�handlung erleidet oder jemand gegen seinen
Willen, wie sie uns erkl�rten, gezwungen wird zu handeln, damit wir nicht bei den Heiden
Gel�chter ernten, und vor allem, damit wir nicht Gott erz�rnen. Denn jeder von uns wird am
Tag des Gerichts Rechenschaft ablegen dar�ber, was er hier getan hat. 
\kapR{8}Es m�gen
aber alle wie Gott denken, damit auch die Kirchen ihre Bisch�fe wieder in Empfang nehmen und
sich in allem in Christus Jesus, unserem Herrn, freuen, durch den dem Vater die
Herrlichkeit sei von Ewigkeit zu Ewigkeit. Amen. Ich bitte um alles Gute im Herrn f�r euch, geliebte
und ersehnte Br�der!
\pend
\endnumbering
\end{translatio}
\end{Rightside}
\Columns
\end{pairs}
% \thispagestyle{empty}

%% DOKUMENT 42 %%%
% \cleartooddpage
\chapter[Theologische Erkl�rung einer Synode von Antiochien (4. antiochenische Formel)][Theologische Erkl�rung einer Synode von Antiochien (4. antiochenische Formel)]{Theologische Erkl�rung einer Synode von Antiochien\\(4. antiochenische Formel)}
\thispagestyle{empty}
% \label{sec:41.8}
\label{ch:AntIV}
\begin{praefatio}
  \begin{description}
  \item[Fr�hjahr/Sommer? 341]Diese theologische Erkl�rung wurde einige
    Monate nach den vorausgehenden antiochenischen Formeln (Ath.,
    syn. 25,1) durch eine Delegation einer weiteren
    antiochenischen Synode (\editioncite[250,27]{Opitz1935}) an den Hof des
    Constans\index[namen]{Constans, Kaiser} nach
    Trier\index[namen]{Trier} �berbracht.  Vorauszusetzen sind
    anscheinend Bem�hungen des Constans, bei seinem Bruder
    Constantius\index[namen]{Constantius, Kaiser} im Osten auf eine
    reichsweite Synode zu dr�ngen (Ath., h.\,Ar. 15,2;
    apol.\,sec. 20,3; 36,1; Socr., h.\,e. II 18,1; Soz., h.\,e. III 10,3~f.). Angeregt zu
    dieser Idee haben ihn sicherlich die aus dem Osten exilierten
    Bisch�fe wie Athanasius\index[namen]{Athanasius!Bischof von
      Alexandrien}, Markell\index[namen]{Markell!Bischof von Ancyra}
    und Paulus von Konstantinopel\index[namen]{Paulus!Bischof von
      Konstantinopel} (vgl. Socr., h.\,e. II 20; Soz., h.\,e. III 10,3;
    Thdt., h.\,e. II 4,4--6), die damit ihre R�ckkehr erreichen
    wollten, auch wenn Athanasius\index[namen]{Athanasius!Bischof von
      Alexandrien} selbst sp�ter entsprechende Aktivit�ten bestreitet
    (apol.\,Const. 4), ebenso wie die f�hrenden Bisch�fe des Westens
    wie Julius von Rom\index[namen]{Julius!Bischof von Rom}, Ossius
    von Cordoba\index[namen]{Ossius!Bischof von Cordoba} und Maximinus
    von Trier\index[namen]{Maximinus!Bischof von Trier}
    (Dok. \ref{sec:RundbriefSerdikaOst},15).  Kaiser
    Constans\index[namen]{Constans, Kaiser}, seit 340 Alleinherrscher
    im Westen, wird sich dieses Ansinnen auch aus politischen Motiven zu
    eigen gemacht haben. Die Delegation aus dem Osten (Narcissus von
    Neronias\index[namen]{Narcissus!Bischof von Neronias} [nach Ath.,
    syn. 17,1 schon vor der Synode von
    Nicaea\index[synoden]{Nicaea!a. 325} 325 am Streit um
    Arius\index[namen]{Arius!Presbyter in Alexandrien} beteiligt, nach
    Dok. \ref{ch:18} = Urk. 18 verurteilt auf der Synode
    von Antiochien\index[namen]{Antiochien} 325, nach Ath.,
    apol.\,sec. 77,2; 78,2; 79,1 beteiligt an der Synode von
    Tyrus\index[synoden]{Tyrus!a. 335} 335; Adressat des
    Julius-Briefes Dok. \ref{sec:BriefJuliusII}; von der ">westlichen"<
    Synode von Serdica\index[synoden]{Serdica!a. 343} verurteilt
    nach Dok.  \ref{sec:SerdicaRundbrief},14; 16; \ref{sec:B},6, aber
    nach Soz., h.\,e. IV 8,4 noch Ende der 40er Jahre an einer
    antiochenischen Synode beteiligt, die
    Georg\index[namen]{Georg!Bischof von Alexandrien} als
    Gegenbischof an Stelle von Athanasius\index[namen]{Athanasius!Bischof von
      Alexandrien} einsetzt]; Maris von
    Chalcedon\index[namen]{Maris!Bischof von Chalcedon} [vgl. Dok
    \ref{sec:BriefJuliusII},32], Theodorus von
    Heraclea\index[namen]{Theodorus!Bischof von Heraclea} [vgl. Dok.
    \ref{sec:BriefJuliusII},32]; Marcus von
    Arethusa\index[namen]{Marcus!Bischof von Arethusa} [ist nach
    Socr., h.\,e. II 29; Soz., h.\,e. IV 6,4 sp�ter an der sirmischen
    Synode 351\index[synoden]{Sirmium!a. 351} gegen
    Photinus\index[namen]{Photinus!Bischof von Sirmium} und nach Soz.,
    h.\,e. IV 16,19; 22,22 an der hom�ischen Synode von
    Seleucia\index[synoden]{Seleucia!a. 359} 359 beteiligt, erleidet
    nach Soz., h.\,e. V 10,8--14 unter Kaiser
    Julian\index[namen]{Julianus!Kaiser} den M�rtyrertod]) versuchte
    mit dieser Erkl�rung eine theologische Einigung, um so eine
    reichsweite Synode im Westen abzuwenden (nach Socr., h.\,e. II 18
    und Soz. h.\,e. III 10,5 hatte Constans\index[namen]{Constans,
      Kaiser} diese Delegation zur Glaubenspr�fung durch ihn selbst
    angefordert). Sie wurde jedoch nicht einmal empfangen, noch wurde
    ihr Anliegen zur Kenntnis genommen (vgl. Soz., h.\,e. III 10,6 und
    die Beschwerde der ">�stlichen"< Bisch�fe dar�ber in ihrem Schreiben
    zur Synode von Serdica\index[synoden]{Serdica!a. 343} in Dok.
    \ref{sec:RundbriefSerdikaOst},28). Anschlie�end fand die reichsweite
    Synode dennoch statt, zementierte aber nur die schon bestehende
    Spaltung (vgl. Dok. \ref{ch:SerdicaEinl}). Diese theologische
    Erkl�rung f�llt viel k�rzer aus als die sogenannte zweite antiochenische Formel\index[synoden]{Antiochien!a. 341} und ist in vielen
    Formulierungen auch formal dem Nicaenum �hnlich, inhaltlich vertritt sie eine Dreihypostasentheologie. Sie gewann gro�e
    Bedeutung, da sie von den Orientalen in ihrer Erkl�rung von
    Serdica\index[synoden]{Serdica!a. 343}
    (vgl. Dok. \ref{sec:BekenntnisSerdikaOst}), in der sogenannten Ekthesis
    makrostichos (Dok. \ref{ch:Makrostichos}) und in der
    theologischen Erkl�rung von Sirmium\index[synoden]{Sirmium!a. 351} (351
    n.\,Chr., s.\,u. �berlieferung) wieder jeweils mit
    weiteren Erg�nzungen aufgegriffen wurde.
  \item[�berlieferung]Dieser Text ist
    bei Athanasius und bei Socrates �berliefert. F�r die textkritische
    Beurteilung sind aber die sp�teren Formen (Erkl�rung der
    ">�stlichen"< Synode von
    Serdica\index[synoden]{Serdica!a. 343}:
    Dok. \ref{sec:BekenntnisSerdikaOst}; Ekthesis macrostichos:
    Dok. \ref{ch:Makrostichos}; Erkl�rung von
    Sirmium\index[synoden]{Sirmium!a. 351} 351: Ath., syn. 27,2~f.,
    Socr., h.\,e. II 30,3--5 und Hil., syn. 38) miteinzubeziehen, so
    da� elf Textfassungen mit kleineren Unterschieden zu
    ber�cksichtigen sind. In der Regel best�tigt die Mehrheit der
    Zeugen eine Fassung gegen�ber singul�ren Abweichungen, manchmal
    lassen sich aber auch Unterschiede ausfindig machen, die eventuell
    auch darauf zur�ckzuf�hren sind, da� die Formel in der
    sp�teren Rezeptionen minimal ver�ndert bzw. erweitert (aus
    \griech{>~hn pote qr'onoc} wurde z.\,B. \griech{>~hn pote qr'onoc
      >`h a>i'wn} und aus \griech{<h kajolik`h >ekklhs'ia} wurde
    \griech{<h kajolik`h ka`i <ag'ia >ekklhs'ia}) worden ist.
  \item[Fundstelle]Ath., syn. 25,2--5 (\editioncite[251,1--16]{Opitz1935}); Socr.,
    h.\,e. II 18,3--6 (\editioncite[111,10--112,3]{Hansen:Socr})
  \end{description}
\end{praefatio}
\begin{pairs}
\selectlanguage{polutonikogreek}
\begin{Leftside}
\beginnumbering
\pstart
% \autor{Athanasius, Sokrates}
\hskip -1.5em\edtext{\abb{}}{\killnumber\Cfootnote{\hskip -1em\latintext Ath. (BKPO R) Socr. (MF=b
AT)}}\specialindex{quellen}{chapter}{Athanasius!syn.!25,2--5}\specialindex{quellen}{chapter}{Socrates!h.\,e.!II 18,3--6}
\kap{1}Piste'uomen e>ic \edtext{\abb{\edtext{\abb{<'ena}}{\Dfootnote{\latintext >
Socr.(M\textsuperscript{1})}} je'on}}{\Afootnote{\latintext vgl. 1Cor 8,6; Eph
4,6}}\edindex[bibel]{Korinther I!8,6|textit}\edindex[bibel]{Epheser!4,6|textit}, pat'era, 
\edtext{\abb{pantokr'atora}}{\Afootnote{\latintext Apc 1,8
u.�.}}\edindex[bibel]{Offenbarung!1,8}, \edtext{\abb{kt'isthn}}{\Afootnote{\latintext vgl.
1Petr 4,19}}\edindex[bibel]{Petrus I!4,19|textit} ka`i poiht`hn t~wn
\edtext{p'antwn}{\Dfootnote{<ap'antwn \latintext Socr.(AT) \greektext >'ontwn \latintext
Ath.(BKPO)}}, \edtext{((>ex o<~u p~asa patri`a >en \edtext{o>urano~ic}{\Dfootnote{o>uran~w|
\latintext Socr.(AT)}} ka`i >ep`i g~hc >onom'azetai))}{\lemma{\abb{}}\Afootnote{\latintext Eph 3,15}}\edindex[bibel]{Epheser!3,15}. \pend
\pstart
\skipnumbering
\pend
\pstart
\kap{2}ka`i e>ic t`on \edtext{\abb{monogen~h}}{\Afootnote{\latintext Io 1,14.18; 3,16;
1Io 4,9}}\edindex[bibel]{Johannes!1,14}\edindex[bibel]{Johannes!1,18}
\edindex[bibel]{Johannes!3,16}\edindex[bibel]{Johannes I!4,9} a>uto~u u<i`on t`on k'urion
<hm~wn >Ihso~un Qrist'on, t`on
\edtext{\abb{pr`o p'antwn \edtext{\abb{t~wn}}{\Dfootnote{\latintext > Ath.(K) Socr.(T)}}
a>i'wnwn}}{\Afootnote{\latintext vgl. 1Cor 2,7}}\edindex[bibel]{Korinther I!2,7|textit}
>ek to~u
patr`oc gennhj'enta, je`on >ek jeo~u, \edtext{\abb{f~wc}}{\Afootnote{\dt{vgl. Io
1,4~f.9; 1Io 2,8}}}
\edindex[bibel]{Johannes!1,4~f.|textit}\edindex[bibel]{Johannes!1,9|textit}
\edindex[bibel]{Johannes I!2,8|textit}
>ek fwt'oc, \edtext{\abb{di'' o<~u >eg'eneto t`a \edtext{\abb{p'anta}}{\Dfootnote{+ t`a
\latintext Socr.(AT)}}}}{\Afootnote{\latintext vgl. Io 1,3; 1Cor 8,6; Col 1,16; Hebr
1,2}}\edindex[bibel]{Johannes!1,3|textit}
\edindex[bibel]{Korinther I!8,6|textit}\edindex[bibel]{Kolosser!1,16|textit}
\edindex[bibel]{Hebraeer!1,2|textit} \edtext{\abb{>en to~ic
o>urano~ic \edtext{\abb{ka`i}}{\Dfootnote{+ t`a \latintext Socr.(AT)}} >ep`i t~hc
g~hc}}{\Afootnote{\latintext Col 1,16; Eph
1,10}}\edindex[bibel]{Kolosser!1,16}\edindex[bibel]{Epheser!1,10},
\edtext{\abb{\edtext{t'a}{\lemma{\abb{t'a}}\Dfootnote{+ te \latintext Socr.(bA)}} <orat`a ka`i t`a
>a'orata}}{\Afootnote{\latintext Col 1,16}}\edindex[bibel]{Kolosser!1,16},
\edtext{\abb{l'ogon}}{\Afootnote{\latintext Io 1,1}}\edindex[bibel]{Johannes!1,1} >'onta
ka`i \edtext{\abb{sof'ian ka`i d'unamin}}{\Afootnote{\latintext 1Cor
1,24}}\edindex[bibel]{Korinther I!1,24} \edtext{ka`i}{\lemma{\abb{ka`i\ts{2}}}\Dfootnote{\latintext >
Socr.(T)}} \edtext{\abb{zw`hn}}{\Afootnote{\latintext Io 1,4; 11,25;
14,6}}\edindex[bibel]{Johannes!1,4}\edindex[bibel]{Johannes!11,25}
\edindex[bibel]{Johannes!14,6}
ka`i \edtext{\abb{\edtext{f~wc}{\Dfootnote{je`on \latintext Socr.(A)}}
>alhjin`on}}{\Afootnote{\latintext Io 1,9; 1Io
2,8}}\edindex[bibel]{Johannes!1,9}\edindex[bibel]{Johannes I!2,8},
\kap{3}t`on \edtext{\abb{>ep'' \edtext{>esq'atwn}{\Dfootnote{>esq'atou \latintext
Socr.(bA)}} t~wn <hmer~wn}}{\Afootnote{\latintext vgl. Hebr
1,2}}\edindex[bibel]{Hebraeer!1,2|textit} di'' <hm~ac
\edtext{\abb{>enanjrwp'hsanta}}{\Afootnote{\latintext vgl. 1Cor
15,47}}\edindex[bibel]{Korinther I!15,47|textit} ka`i \edtext{\abb{gennhj'enta >ek t~hc
<ag'iac \edtext{\abb{parj'enou}}{\Dfootnote{+ Mar'iac \latintext
Socr.(T)}}}}{\Afootnote{\latintext vgl. Mt 1,23; Lc
1,27.34~f.}},\edindex[bibel]{Matthaeus!1,23|textit}\edindex[bibel]{Lukas!1,27|textit}
\edindex[bibel]{Lukas!1,34~f.|textit} t`on
\edtext{\abb{staurwj'enta}}{\Afootnote{\latintext vgl. Mt 20,19; 27,22 u.�.;
1Cor 1,23; 2,2.8; 2Cor 13,4; Gal
3,1}}\edindex[bibel]{Matthaeus!20,19|textit}\edindex[bibel]{Matthaeus!27,22|textit}
\edindex[bibel]{Korinther I!1,23|textit}\edindex[bibel]{Korinther I!2,2|textit}
\edindex[bibel]{Korinther I!2,8|textit}
\edindex[bibel]{Korinther II!13,4|textit}\edindex[bibel]{Galater!3,1|textit}
ka`i \edtext{\abb{>apojan'onta}}{\Afootnote{\latintext
1Cor 15,3}\Dfootnote{paj'onta \latintext Socr.(T)}}\edindex[bibel]{Korinther I!15,3} ka`i
\edtext{\abb{taf'enta}}{\Afootnote{\latintext vgl. 1Cor 15,4}}\edindex[bibel]{Korinther I!15,4|textit} ka`i
\edtext{\abb{>anast'anta}}{\Afootnote{\latintext vgl. Mt 17,9par; 20,19par; 1Cor 15,4;
1Thess 4,14; auch Eph
1,20}}\edindex[bibel]{Matthaeus!17,9par|textit}\edindex[bibel]{Matthaeus!20,19par|textit}
\edindex[bibel]{Korinther I!15,4|textit}
\edindex[bibel]{Thessalonicher I!4,14|textit}\edindex[bibel]{Epheser!1,20|textit} >ek
nekr~wn t~h| tr'ith| <hm'era| ka`i \edtext{\abb{>analhfj'enta e>ic
o>uran`on}}{\Afootnote{\latintext Mc 16,19; Act 1,2}\Dfootnote{>anelhluj'ota e>ic to`uc
o>urano`uc \latintext
Socr.(bA)}}\edindex[bibel]{Markus!16,19}\edindex[bibel]{Apostelgeschichte!1,2}
\edtext{ka`i
\edtext{kajesj'enta}{\Dfootnote{kajez'omenon \latintext Ath.(B)}} >en dexi~a| to~u
patr`oc}{\lemma{\abb{ka`i \dots\ patr`oc}}\Afootnote{\latintext vgl. Ps 110,1; Eph 1,20;
Col 3,1; 1Petr 3,22; Hebr 1,3; Mc 16,19}\Dfootnote{\latintext >
Socr.(T)}}\edindex[bibel]{Psalmen!110,1|textit}\edindex[bibel]{Epheser!1,20|textit}
\edindex[bibel]{Kolosser!3,1|textit}\edindex[bibel]{Petrus I!3,22|textit}
\edindex[bibel]{Hebraeer!1,3|textit}\edindex[bibel]{Markus!16,19|textit} ka`i
>erq'omenon >ep`i \edtext{\abb{suntele'ia|}}{\Afootnote{\latintext vgl. Mt 13,39; 24,3;
28,20}}\edindex[bibel]{Matthaeus!13,39|textit}\edindex[bibel]{Matthaeus!24,3|textit}
\edindex[bibel]{Matthaeus!28,20|textit} \edtext{to~u a>i~wnoc}{\Dfootnote{t~wn a>i'wnwn
\latintext Socr.(bA)}} 
\edtext{\abb{kr~inai z~wntac \edtext{\abb{ka`i nekro`uc}}{\Dfootnote{\latintext >
Socr.(F)}}}}{\Afootnote{\latintext 2Tim 4,1; 1Petr 4,5; vgl. Io 5,22; Act
10,42}}\edindex[bibel]{Timotheus II!4,1}
\edindex[bibel]{Petrus I!4,5}\edindex[bibel]{Johannes!5,22|textit}
\edindex[bibel]{Apostelgeschichte!10,42|textit} ka`i
\edtext{\abb{>apodo~unai <ek'astw| kat`a t`a >'erga a>uto~u}}{\Afootnote{\latintext Rom
2,6; 2Tim 4,14; Apc 22,12}},\edindex[bibel]{Roemer!2,6}
\edindex[bibel]{Timotheus II!4,14}\edindex[bibel]{Offenbarung!22,12} o<~u <h
basile'ia \edtext{>akat'apaustoc}{\Dfootnote{>akat'alutoc \latintext Ath.}} o>~usa
diamene~i e>ic
to`uc >ape'irouc a>i~wnac; >'estai g`ar kajez'omenoc >en dexi~a| to~u patr`oc o>u m'onon
>en t~w| a>i~wni to'utw|, >all`a ka`i >en t~w|
\edtext{\abb{m'ellonti}}{\Afootnote{\latintext Eph 1,21}}\edindex[bibel]{Epheser!1,21}.
\pend
\pstart
\kap{4}ka`i e>ic t`o \edtext{pne~uma t`o <'agion}{\Dfootnote{<'agion pne~uma \latintext
Ath.}}, tout'esti t`on \edtext{\abb{par'aklhton}}{\Afootnote{\latintext Io 14,16.26;
15,26; 16,7}},\edindex[bibel]{Johannes!14,16}\edindex[bibel]{Johannes!14,26}
\edindex[bibel]{Johannes!15,26}\edindex[bibel]{Johannes!16,7} <'oper \edtext{\abb{>epaggeil'amenoc to~ic
>apost'oloic}}{\Afootnote{\latintext vgl. Io 14,26; 15,26; Act 2,17--21.33; Tit
3,6}}\edindex[bibel]{Johannes!14,26|textit}\edindex[bibel]{Apostelgeschichte!2,17--21|textit}
\edindex[bibel]{Apostelgeschichte!2,33|textit}\edindex[bibel]{Johannes!15,26|textit}
\edindex[bibel]{Titus!3,6|textit}
met`a t`hn e>ic o>urano`uc a>uto~u >'anodon >ap'esteile did'axai
\edtext{\abb{a>uto`uc}}{\Dfootnote{\latintext > Socr.(bA)}} ka`i <upomn~hsai p'anta, di''
o<~u ka`i \edtext{\abb{<agiasj'hsontai}}{\Afootnote{\latintext vgl. Rom 15,16; 1Cor
6,11}}\edindex[bibel]{Roemer!15,16|textit}\edindex[bibel]{Korinther I!6,11|textit} a<i
t~wn e>ilikrin~wc e>ic a>ut`on pepisteuk'otwn yuqa'i.
\pend
\pstart
\kap{5}to`uc d`e l'egontac >ex o>uk >'ontwn t`on u<i`on >`h >ex <et'erac <upost'asewc
ka`i m`h >ek to~u jeo~u \edtext{\abb{ka`i}}{\Dfootnote{+ t`o \latintext Socr.(T)}} >~hn
pote qr'onoc <'ote o>uk >~hn >allotr'iouc o>~iden <h kajolik`h >ekklhs'ia. 
\pend
\endnumbering
\end{Leftside}
\begin{Rightside}
\begin{translatio}
\beginnumbering
\pstart
\noindent\kapR{1}Wir glauben an einen Gott, Vater, Allm�chtigen, Sch�pfer und Erschaffer
von allem, nach dem alle Vaterschaft im Himmel und auf Erden benannt wird. 
\pend
\pstart

\pend
\pstart
\kapR{2}Und an seinen eingeborenen Sohn\footnoteA{Dieser Artikel f�llt im Vergleich zur
Dok. \ref{sec:AntII},1 wesentlich k�rzer aus.}, unsern Herrn Jesus Christus, der vor allen
Zeiten aus dem Vater gezeugt worden ist, Gott aus Gott, Licht aus Licht,\footnoteA{Vgl. Entsprechendes im
Nicaenum (Dok. \ref{ch:24} = Urk. 24), aber hier fehlen selbstverst�ndlich die in Nicaenum anschlie�enden Aussagen, der Sohn sei
gezeugt, nicht geschaffen, dem Vater wesenseins.} durch den alles im
Himmel und auf Erden wurde, das Sichtbare und das Unsichtbare, der Wort\footnoteA{Hier nun
in Anlehnung an bisherige antiochenische Tradition formuliert, aber k�rzer und ohne die
Aussage, der Sohn sei unver�nderliches Abbild des Vaters.}, Weisheit, Kraft, Leben und
wahres Licht ist, 
\kapR{3}der in den letzten Tagen f�r uns Mensch und geboren wurde aus
der heiligen Jungfrau, der gekreuzigt wurde, starb, begraben wurde und am dritten Tag von
den Toten auferstand und in den Himmel hinaufgenommen wurde und sich zur Rechten des
Vaters setzte, und der wiederkommen wird am Ende der Zeit zu richten die Lebenden und
die Toten und jedem nach seinen Werken zu vergelten, dessen Herrschaft unauf"|l�slich ist
und f�r alle Zeiten bestehen bleiben wird; er wird n�mlich zur Rechten des Vaters sitzen nicht
nur in dieser Zeit, sondern auch in der k�nftigen\footnoteA{Eine eindeutig gegen Markell
gerichtete Aussage, dem vorgeworfen wurde, er lehre ein Ende der Herrschaft des Sohnes,
wenn er sich dem Vater unterordne (vgl. Dok. \ref{ch:Konstantinopel336} und
\ref{sec:MarkellJulius}).}. 
\pend
\pstart
\kapR{4}Und (wir glauben) an den heiligen Geist, das hei�t an den Tr�ster, den er den
Aposteln versprochen hatte und nach seinem Aufstieg in den Himmel sandte, damit er sie
lehre und an alles erinnere, wodurch auch die Seelen derer, die aufrichtig an ihn geglaubt
haben, geheiligt werden. 
\pend
\pstart
\kapR{5}Die aber sagen\footnoteA{Zu den folgenden Verwerfungen vgl. die Anathematismen des
Nicaenums, Dok. \ref{ch:24} = Urk. 24, und auch die der zweiten antiochenischen Formel angeh�ngten Aussagen
(Dok. \ref{sec:AntII},1).}, der Sohn sei aus nichts oder aus einer anderen
Hypostase und nicht aus Gott, und da� es einmal eine Zeit gab, in der er nicht war, die
sieht die katholische Kirche als Fremde an. 
\pend
\endnumbering
\end{translatio}
\end{Rightside}
\Columns
\end{pairs}

%% DOKUMENT 43 %%%
\chapter{Die Synode von Serdica}
\thispagestyle{empty}
% \label{ch:43}
\label{ch:SerdicaEinl}
\begin{praefatio}
  \begin{description}
  \item[Vorgeschichte]Vorbereitungen f�r eine zweite reichsweite
    Synode wurden bereits seit l�ngerer Zeit getroffen (vgl. die
    Einleitungen zu Dok. 41--42). Nachdem der Versuch, in Rom eine
    gemeinsame Synode abzuhalten, gescheitert war, traten die
    westlichen Bisch�fe wohl �ber Maximinus von
    Trier\index[namen]{Maximinus!Bischof von Trier}, den
    Residenzbischof, an Kaiser Constans\index[namen]{Constans, Kaiser}
    mit ihrer Bitte heran, eine Synode beider Reichsteile
    einzuberufen. In dieser Sache sind offenbar mehrmals Briefe von
    Constans\index[namen]{Constans, Kaiser} an
    Constantius\index[namen]{Constantius, Kaiser} verschickt worden
    (Ath., h.\,Ar. 15,2; apol.\,sec. 36,1; Socr., h.\,e. II 18,1;
    Soz., h.\,e. III 10,3~f.). M�glicherweise war eine Auf"|forderung des
    Constans\index[namen]{Constans, Kaiser} an seinen Bruder
    Constantius\index[namen]{Constantius, Kaiser}, eine solche Synode
    einzuberufen, bereits 341 mit dem Antwortschreiben des
    Julius\index[namen]{Julius!Bischof von Rom}
    (Dok. \ref{sec:BriefJuliusII}) auf die Absage der �stlichen
    Bisch�fe, zur r�mischen Synode zu erscheinen, nach
    Antiochien\index[namen]{Antiochien} geschickt
    worden. Athanasius\index[namen]{Athanasius!Bischof von
      Alexandrien} selbst gibt an (apol.\,Const. 4,3), da� er drei
    Jahre nach seiner Abreise aus
    Alexandrien\index[namen]{Alexandrien} (339), also wohl 342,
    w�hrend seines Aufenthalts in Rom\index[namen]{Rom} von
    Constans\index[namen]{Constans, Kaiser} an den Mail�nder Hof
    geladen worden sei, wo er nach seiner Aussage erstmals von den
    Pl�nen zu einer reichsweiten Synode erfahren habe. Das ist aber
    wenig glaubw�rdig, da Athanasius sich in dieser Apologie dagegen
    verteidigt, an Intrigen gegen
    Constantius\index[namen]{Constantius, Kaiser} beteiligt gewesen zu
    sein. Im Anschlu� an diese Mail�nder\index[namen]{Mailand}
    Begegnung reiste Athanasius\index[namen]{Athanasius!Bischof von
      Alexandrien} zu einem zweiten Treffen mit dem Kaiser nach
    Trier\index[namen]{Trier} (die Anwesenheit des Kaisers
    Constans\index[namen]{Constans, Kaiser} in
    Trier\index[namen]{Trier} ist f�r Juni 343 belegt,
    s. apol.\,Const. 4,4 Anm.), wo sich auch einige Bisch�fe des
    Westens (der Ortsbischof Maximinus\index[namen]{Maximinus!Bischof
      von Trier} und Ossius von Cordoba\index[namen]{Ossius!Bischof
      von Cordoba}; apol.\,Const. 4,4) trafen, um die Synode
    vorzubereiten und dann von dort nach Serdica\index[namen]{Serdica}
    zu reisen. Wohl in diese Zeit f�llt die Reise der Delegation aus
    Antiochien\index[namen]{Antiochien} nach
    Trier\index[namen]{Trier}, die erfolglos umkehren mu�te
    (s. Dok. \ref{ch:AntIV}). Erneut schrieb
    Constans\index[namen]{Constans, Kaiser} an seinen Bruder
    (apol.\,Const. 4,4) und konnte ihn offenbar zu einem gemeinsamen
    Vorgehen �berreden; beide Kaiser forderten nun gemeinsam die
    Bisch�fe auf, zu einer reichsweiten Synode in
    Serdica\index[synoden]{Serdica!a. 343} zu erscheinen (Ath.,
    h.\,Ar. 15,2; Socr., h.\,e. II 20,1--3; Soz., h.\,e. III 11,3). Sie
    gaben sogar gemeinsam drei Tagesordnungspunkte vor, die die Synode zu
    behandeln hatte: Glaubensfragen, Personalfragen und
    kirchenrechtliche Fragen (vgl. Dok. \ref{sec:B},3).
  \item[Datierung]Die Synode fand aller Wahrscheinlichkeit nach im
    Herbst des Jahres 343 statt. Die Belege daf�r sind zwar im
    Einzelnen nicht unumstritten, deuten aber in ihrer Gesamtheit auf
    dieses Datum (vgl. auch die zusammenfassende Darstellung bei
    \cite[210--217]{Parvis:Marcellus}): Nach ind.\,ep.\,fest. 15
    (a. 343) fand die Synode im Jahr 343 statt, was auch durch das
    weitere Zeugnis der Indices zu den Osterfestbriefen best�tigt
    wird, wenn Athanasius ind.\,ep.\,fest. 16 (a. 344) und
    apol.\,Const. 4,5 zufolge das Osterfest nach der Synode in
    Na"issus verbracht hat und vor seiner R�ckkehr nach Alexandrien im
    Oktober 346 drei Festbriefe (d.\,h. 344, 345 und 346) nach der
    Synode verfa�t hat.  Auch der von Athanasius h.\,Ar. 16,2 genannte
    Triumph gegen die Perser (zur Datierung
    vgl. \cite[243~f.]{Burgess:Studies}) und die innere Chronologie der
    Darstellung des Athanasius\index[namen]{Athanasius!Bischof von
      Alexandrien} in apol.\,Const. 4 (\editioncite[282~f. mit Anm.]{Brennecke2006}) f�gen sich in diese Datierung, ebenso die
    weiteren Ereignisse nach der Synode: Die westliche Gruppe sandte
    eine Delegation nach Antiochien\index[namen]{Antiochien}, wo sie
    den Kaiser Constantius\index[namen]{Constantius, Kaiser} kurz vor
    Ostern 344 traf (h.\,Ar. 20,2~f.), und dieser schickte einen Brief
    nach Alexandrien\index[namen]{Alexandrien}, wo zehn Monate sp�ter
    am 26.6.345 der dortige Gegenbischof
    Gregor\index[namen]{Gregor!Bischof von Alexandrien} starb (Ath.,
    h.\,Ar. 21; ind.\,ep.\,fest. 17).  Au�erdem wurde nach
    ind.\,ep.\,fest. 15 auf der Synode von
    Serdica\index[synoden]{Serdica!a. 343} f�r 50 Jahre ein
    gemeinsamer Termin f�r das Osterfest in Rom\index[namen]{Rom} und
    Alexandrien\index[namen]{Alexandrien} beschlossen: Da im Jahr 343
    diese Termine aber noch auseinanderliegen
    (Alexandrien\index[namen]{Alexandrien}: 27. M�rz
    [ind.\,ep.\,fest. 15]; Rom\index[namen]{Rom}: 3. April
    [Chron.\,min. I 63]) und auch eine Liste der Daten des j�dischen
    Osterfestes (Cyclus Paschalis, EOMIA I 2,4, 641--643), die
    offenbar f�r die Berechnung herangezogen und zur Synode
    mitgebracht worden ist, die Jahre 328 bis 343 umfa�t, spricht auch
    dies eher f�r das Jahr 343.  Das scheinbar st�rkste Argument gegen
    das Jahr 343 und f�r die Datierung der Synode in das Jahr 342 ist
    der seit \cite[11, 56, 326]{Schwartz:GesIII} immer wieder
    herangezogene Text Codex Veronensis LX, f. 71b (EOMIA I 2,4, 637):
    Schwartz konjizierte die dort angegebene Datierung
    ">consolat. Constantini et Constantini"< zu ">consolat. Constantii
    III et Constantis II"<, was mit dem Jahr 342 gleichzusetzen w�re. (Damit
    emendierte er im Prinzip die Ballerinische Emendation auf das Jahr
    346 = ">consolat. Constantii IV et Constantis III"< in PL 56, 146,
    vgl. \cite[242 Anm. 170]{Burgess:Studies}.) Wie nun aber
    \cite[242~f.]{Burgess:Studies} gezeigt hat, ist die Angabe im
    Cod.\,Ver. entweder als ">Constantino VI et Constantino Caes."<
    (d.\,h. a. 320) oder als ">Constantino VIII et Constantino
    Caes. IV"< (d.\,h. 329) aufzul�sen. Da nun das Jahr 329 ebenfalls
    wie das Jahr 343 das 2. Jahr einer Indiktion war, kann Burgess
    sehr wahrscheinlich machen, da� bei der �bersetzung der
    griechische Vorlage des Codex Veronensis ins Lateinische und der
    wohl in diesem Zusammenhang erfolgten �bertragung der im
    lateinischen Sprachraum ungew�hnlichen Datierung nach Indiktionen
    in die �blichere nach Consulaten eine Verwechslung stattgefunden
    hatte und so f�lschlicherweise das zweite Jahr des vorangehenden
    Indiktionszyklus gew�hlt worden war. Damit w�rde schlie�lich auch
    der Codex Veronensis auf das Jahr 343 weisen.

    Falsch ist in jedem Fall die Datierung ins Jahr 347, dem Konsulat
    des Vulcacius Rufinus und des Flavius Eusebius, durch Socr.,
    h.\,e. II 20,4 und Soz., h.\,e. III 12,7, die hier beide wohl
    Sabinus von Heraclea folgen.
  
  \item[Ablauf]Aus den erhaltenen Dokumenten und Berichten l��t sich
    folgender Ablauf der Synode rekonstruieren: Die westlichen
    Bisch�fe trafen eher in Serdica ein als die �stlichen und nahmen
    Athanasius\index[namen]{Athanasius!Bischof von Alexandrien} und
    Markell\index[namen]{Markell!Bischof von Ancyra} wieder in die
    Gemeinschaft auf, ohne die Ankunft der �stlichen Bisch�fe
    abzuwarten. Diese protestierten gegen dieses Vorgehen bei ihrem
    Eintreffen und forderten die R�cknahme dieser Entscheidung. Da die
    westlichen Bisch�fe dies verweigerten, lehnte die �stliche
    Delegation gemeinsame Verhandlungen ab. Alle Versuche, einen
    Kompromi� zu finden, scheiterten
    (vgl. \ref{sec:RundbriefSerdikaOst}).  Beide Seiten tagten in
    Serdica\index[namen]{Serdica} weiter. Die �stlichen Bisch�fe zogen
    sich daf�r in das kaiserliche Palatium zur�ck (Ath.,
    h.\,Ar. 15,4).  Die Nachricht, da� die Orientalen in
    Philippopolis\index[namen]{Philippopolis} tagten
    (ind.\,ep.\,fest. 15 [a. 343]; Socr., h.\,e. II 20,9), beruht
    demnach auf einem Irrtum. Beide Teilsynoden exkommunizierten die
    f�hrenden Bisch�fe der Gegenpartei und verfa�ten mehrere
    Schriftst�cke. Von der ">westlichen"< Synode sind uns neben dem
    Rundbrief und der theologischen Erkl�rung mehrere Briefe,
    z.\,T. mit Unterschriftenlisten erhalten, von der ">�stlichen"<
    nur deren Rundbrief und die theologische Erkl�rung samt der
    Unterschriftenliste.

  \item[Teilnehmer]Aus diesen Unterschriftenlisten lassen sich auch
    die Anzahl und die Herkunft der Teilnehmer der Synode bestimmen.
    Athanasius\index[namen]{Athanasius!Bischof von Alexandrien} gibt
    die Gesamtzahl der Teilnehmer beider Synoden mit 170 an
    (h.\,Ar. 15,3). Da die �stliche Synode nach eigener Angabe 80
    Mitglieder z�hlte (vgl. Dok.  \ref{sec:RundbriefSerdikaOst}) und
    durch die Unterschriftenliste (Dok. \ref{sec:NominaepiscSerdikaOst}) 73 Bisch�fe
    bezeugt sind, bleiben f�r die ">westliche"< Synode zwischen 90 und 97
    Bisch�fe, die ungef�hr erreicht werden, wenn man die Namen der
    Athanasiusliste durch die Listen des Cod.\,Ver. erg�nzt.  Die
    beiden Kaiser waren auf der Synode nicht anwesend, die ">westliche"<
    hat nach Athanasius\index[namen]{Athanasius!Bischof von
      Alexandrien} sogar v�llig ohne staatliche Beamte getagt
    (h.\,Ar. 15,3). Daher verfa�ten sicherlich beide Teilsynoden
    Briefe an die Kaiser, wovon nur der Brief der ">westlichen"<
    Teilsynode an Constantius\index[namen]{Constantius, Kaiser}
    �berliefert ist (Dok. \ref{sec:BriefSerdikaConstantius}).

    Die ">westliche"< Synode hatte nach den in ihren Briefen gemachten
    Angaben Teilnehmer aus den folgenden Provinzen:
    \begin{longtable}[c]{p{1.5cm}p{1.5cm}p{1.5cm}p{1.5cm}p{1.5cm}}
      \toprule aus Namenslisten & Dok. \ref{sec:BriefSerdikaAlexandrien},1 &
      apol.\,sec. 1,2; h.\,Ar. 28,2 &
      Dok. \ref{sec:SerdicaRundbrief},1 (Cod.\,Ver.)  &
      Dok. \ref{sec:SerdicaRundbrief},1 (Thdt.)
      \\
      \midrule
      \endhead
      \midrule
      \endfoot
      \bottomrule
%       & \\
%       \caption*{�berblick �ber die Provinzen, aus denen Teilnehmer der
%         ">westlichen"< Synode kamen}
      \endlastfoot
      Achaia & Achaia & Achaia\footnote{Athanasius nennt noch folgende
        Provinzen, die aber auf eine nachtr\"agliche
        Unterschriftenliste zur�ckzuf�hren sein d�rften: Britannia,
        Brittia, Corsica, Cyprus, Dalmatia, Dacia, Isauria, Libya,
        Lycia, Pamphylia, Pentapolis, Theba"is, Picenus, Sicilia.} &
      Achaia & Achaia\footnote{Zu den Provinzen, die Theodoret
        zus�tzlich nennt, vgl. Anm. zu
        Dok. \ref{sec:SerdicaRundbrief}.}
      \\
      Africa & Africa & Africa & Africa & Africa
      \\
      Aegyptus & Aegyptus & Aegyptus & & Aegyptus
      \\
      Apulia & Apulia & Apulia & &
      \\
      Arabia & Arabia & Arabia & Arabia & Arabia
      \\
      Asia & & & & Asia
      \\
      & Calabria & Calabria & Calabria & Calabria
      \\
      Campania & Campania & Campania & Campania & Campania
      \\
      Creta & Creta & Creta & &
      \\
      Dacia & Dacia & Dacia & Dacia & Dacia
      \\
      Dacia rip.  & Dacia rip.  & & Dacia rip.  & Dacia rip.
      \\
      Dardania & Dardania & Dardania & Dardania & Dardania
      \\
      Epirus & Epirus & Epirus & Epirus & Epirus
      \\
      Galatia & & Galatia & & Galatia
      \\
      Gallia & Gallia & Gallia & Gallia & Gallia
      \\
      Italia & Italia & Italia & Italia & Italia
      \\
      Kykladen & & & & Kykladen
      \\
      Macedonia & Macedonia & Macedonia & Macedonia & Macedonia
      \\
      Moesia & Moesia & Moesia & Moesia & Moesia
      \\
      & Noricum & Noricum & &
      \\
      Palaestina & Palaestina & Palaestina & Palaestina & Palaestina
      \\
      Pannonia & Pannonia & Pannonia & Pannonia & Pannonia
      \\
      Rhodope & Rhodope & & Rhodope & Rhodope
      \\
      Roma & Roma & Roma & Roma & Roma
      \\
      & Sardinia & Sardinia & Sardinia & Sardinia
      \\
      Savia & Savia=Siscia & Savia & &
      \\
      Spania & Spania & Spania & Spania & Spania
      \\
      Thessalia & Thessalia & Thessalia & Thessalia & Thessalia
      \\
      Thracia & Thracia & Thracia & Thracia & Thracia
      \\
      Tuscia & & Tuscia & &
      \\
    \end{longtable}

    Die meisten der in den Listen genannten Provinzen lassen sich auch
    durch Namen abdecken. Aus den Provinzlisten der Teilnehmer der
    ">westlichen"< Synode m\"ussen gewi{\ss} noch Calabria und
    Sardinia hinzugenommen werden, auch wenn diesen keine bestimmten
    Namen zugewiesen werden k\"onnen.  Dies k\"onnte auch f\"ur
    Noricum zutreffen, aber diese Provinz k\"onnte auch zu denjenigen
    geh\"oren, aus der zwar kein Teilnehmer nach Serdica gereist war,
    aber nachtr\"aglich Bisch\"ofe unterschrieben haben.

    In der folgenden Tabelle sind die Namen der teilnehmenden Bisch�fe
    alphabetisch aufgef�hrt.  Fragezeichen hinter den Namen weisen
    darauf hin, da� zwar durch andere Listen eindeutig mehrere Tr�ger
    dieses Namens belegt sind, aber die konkrete Zuweisung unklar
    bleibt. Sind die Nummern in der ersten Spalte in eckige Klammern
    gesetzt, so beziehen sie sich auf die sp�ter angeh�ngten
    Unterschriften (vgl. dazu die Einleitung zu
    Dok. \ref{sec:SerdikaUnterschriften}).

      \begin{longtable}[c]{p{0.5cm}p{4.7cm}p{1.5cm}p{1cm}p{1cm}p{1cm}p{1.8cm}}
    \toprule Nr.  & Name & Dok. \ref{sec:SerdikaUnterschriften} & Dok. \ref{sec:B} &
    Dok. \ref{sec:BriefAthMareotis} & Dok. \ref{sec:BriefSerdikaMareotis} & Provinz
    \\
    \midrule
    \endhead
    \midrule
    \endfoot
    \bottomrule
	& & & & & & \\
    \caption*{�bersicht �ber die Namenslisten}
    \endlastfoot
    1 & Adolius\index[namen]{Adolius!Bischof} & 77 & -- & -- & -- & ?
    \\
    2 & Aelianus\index[namen]{Aelianus!Bischof von Gortyna} von
    Gortyna & 68 & -- & 56 & -- & (Creta)
    \\
    3 & A"etius von Thessalonike\index[namen]{A"etius!Bischof von
      Thessalonike} & 40 & 27 & -- & 8 & Macedonia
    \\
    4 & Alexander von Cyparissia\index[namen]{Alexander!Bischof von
      Cyparissia} & 28? 66?  & 57 & 45 & 21?  & Achaia
    \\
    5 & Alexander von Corone\index[namen]{Alexander!Bischof von
      Corone} & 28? 66?  & 60 & -- & 21?  & Achaia
    \\
    6 & Alexander von Larisa\index[namen]{Alexander!Bischof von
      Larisa} & 28? 66?  & 26 & -- & 21?  & Thessalia
    \\
    7 & Alypius von Megara\index[namen]{Alypius!Bischof von Megara} &
    38 & 34 & -- & 13 & Achaia
    \\
    8 & Amantius von Viminacium\index[namen]{Amantius!Bischof von
      Viminacium} (per presbyterum
    Maximum\index[namen]{Maximus!Presbyter in Viminacium}) & 60 & -- &
    44 & -- & Moesia
    \\
    % 8 & Ammianus von Castellum\index[namen]{Ammianus!Bischof von
    %   Castellum} & -- & -- & 23 & -- & Pannonia
    % \\
    9 & Ammonius\index[namen]{Ammonius!Bischof} & -- & -- & 4 & -- & ?
    \\
    10 & Annianus von Castellona\index[namen]{Annianus!Bischof von
      Castellona} & 51 & 2 & 23\footnote{�berliefert Ammianus de
      Castello Pannonie.} & -- & Spania
    \\
    11 & Antigonus von Pallene\index[namen]{Antigonus!Bischof von
      Pallene} & 67 & -- & 48 & -- & Macedonia
    \\
    12 & Appianus\index[namen]{Appianus!Bischof} & -- & -- & 11 & -- &
    ?
    \\
    13 & Aprianus von Poetovio\index[namen]{Aprianus!Bischof von
      Poetovio} & 43 & -- & 5?/47 & -- & Pannonia
    \\
    % 14 & Aprianus\index[namen]{Aprianus!Bischof} & 43?  & -- & 5 &
    % -- & ?
    % \\
    14 & Arius von Petra\index[namen]{Arius!Bischof von Petra} & 61 &
    41 & -- & 10 & Palaestina
    \\
    15 & Asclepas von Gaza\index[namen]{Asclepas!Bischof von Gaza} &
    62 & 12 & 10 & -- & Palaestina
    \\
    16 & Asterius\index[namen]{Asterius!Bischof in Arabien} & 54 & 42
    & -- & 16 & Arabia
    \\
    17 & Athanasius von Alexandrien\index[namen]{Athanasius!Bischof
      von Alexandrien} & 58 & 31 & -- & 2 & Aegyptus
    \\
    18 & Athenodorus von Elatia\index[namen]{Athenodorus!Bischof von
      Elatia} & 35 & 24 & 2 & 12 & Achaia
    \\
    19 & Bassus von Diocletianopolis\index[namen]{Bassus!Bischof von
      Diocletianopolis} & 14 & 8 & -- & 17 & Macedonia
    \\
    20 & Calepodius von Neapolis\index[namen]{Calepodius!Bischof von
      Neapolis} & 12 & 45 & -- & 20 & Campania
    \\
    21 & Calvus von Castra Martis\index[namen]{Calvus!Bischof von
      Castra Martis} & 19 & 36 & 18?/37 & -- & Dacia ripensis
    \\
    22 & Castus von Caesaraugusta\index[namen]{Castus!Bischof von
      Caesaraugusta} & 47 & 5 & 27 & -- & Spania
    \\
    23 & Cydonius von Cydonia\index[namen]{Cydonius!Bischof von
      Cydonia} & -- & -- & 60 & -- & Creta
    \\
    24 & Diodorus von Tenedus\index[namen]{Diodorus!Bischof von
      Tenedus} & 31 & 25 & 14 & -- & Asia
    \\
    25 & Dionysius von Elis\index[namen]{Dionysius!Bischof von Elis} &
    63 & 48 & -- & 6 & Achaia
    \\
    26 & Dioscurus von Therasia\index[namen]{Dioscurus!Bischof von
      Therasia} & 24 & 17 & -- & 18 & (Cyclades)
    \\
    27 & Dometianus von Acaria Constantias (?)\footnote{">Acaria
      Constantia"< ist nicht zu identifizieren; eventuell ist
      Dometianus mit dem folgenden Domitianus von Asturica
      identisch.}\index[namen]{Dometianus!Bischof von Acaria
      Constantias (?)}  & 48 & -- & 49 & 19 & ?
    \\
    28 & Domitianus von Asturica\index[namen]{Domitianus!Bischof von
      Asturica} & 18 & 4 & -- & -- & Spania
    \\
    29 & Heliodorus von Nicopolis\index[namen]{Heliodorus!Bischof von
      Nicopolis} & 52 & 39 & -- & 3 & Epirus
    \\
    30 & Eucarpus von Opus\index[namen]{Eucarpus!Bischof von Opus} &
    34?  & -- & 16?/54 & -- & Achaia
    \\
    % 33 & Eucarpus\index[namen]{Eucarpus!Bischof} & 34?  & -- & 16 &
    % -- & ?
    % \\
    31 & Eucissus von Cissamus\index[namen]{Eucissus!Bischof von
      Cissamus} & -- & -- & 59 & -- & Creta
    \\
    % 35 & Eugenius\index[namen]{Eugenius!Bischof} & 10?  & -- & 13 &
    % -- & ?
    % \\
    32 & Eugenius von Heraclea Lyncestis\index[namen]{Eugenius!Bischof
      von Heraclea Lyncestis} & 10?  & 21\footnote{�berliefert ist der
      Name als Euagrius.}  & 13?/31 & -- & Macedonia
    \\
    33 & Eulogius\index[namen]{Eulogius!Bischof} & 22 & -- & 12 & -- &
    ?
    \\
    (34 & Euphrates von Colonia
    Agrippinensis\index[namen]{Euphrates!Bischof von
      K�ln}\footnote{Genannt Ath., h.Ar. 20,2; Thdt., h.\,e. II 8,54;
      9,5.}  & -- & -- & -- & -- & Germania II)
    \\
    35 & Eutherius\index[namen]{Eutherius!Bischof in Pannonien} & 33?
    & 40 & -- & -- & Pannonia
    \\
    36 & Eutherius von Ganus\index[namen]{Eutherius!Bischof von Ganus}
    & 33?  & 11 & -- & -- & Thracia
    \\
    37 & Eutychius von Methone\index[namen]{Eutychius!Bischof von
      Methone} & 29 & 58/59 & 46 & -- & Achaia
    \\
    38 & Eutychus\index[namen]{Eutychus!Bischof} &
    72\footnote{Evtl. mit Nr. 34 identisch.}  & -- & -- & -- & ?
    \\
    39 & Florentius von Emerita
    Augusta\index[namen]{Florentius!Bischof von Emerita Augusta} & 13
    & 3 & 22 & -- & Spania
    \\
    40 & Fortunatianus von Aquileia\index[namen]{Fortunatianus!Bischof
      von Aquileia} & 49 & 37 & -- & -- & Italia
    \\
    41 & Gaudentius von Na"issus\index[namen]{Gaudentius!Bischof von
      Na"issus} & 4 & 32 & 21 & -- & Dacia
    \\
    42 & Gerontius von Beroea\index[namen]{Gerontius!Bischof von
      Beroea} & 20 & 56 & 7 & 14 & Macedonia
    \\
    43 & Gratus von Carthago\index[namen]{Gratus!Bischof von
      Carthago}\footnote{Genannt Serdica, Can. 7; Carthago 348,
      Can. 5.}  & [114] & -- & -- & -- & Africa
    \\
    44 & Hermogenes von Sicyon\index[namen]{Hermogenes!Bischof von
      Sicyon} & 46 & -- & 34 & -- & Achaia
    \\
    45 & Hymenaeus von Hypata\index[namen]{Hymenaeus!Bischof von
      Hypata} & 57 & 18 & 26 & -- & Thessalia
    \\
    46 & Januarius von Beneventum\index[namen]{Januarius!Bischof von
      Beneventum} & 26 & 15 & 24 & -- & Campania
    \\
    47 & Johannes\index[namen]{Johannes!Bischof} & -- & -- & -- & 4 &
    ?
    \\
    48 & Jonas von Parthicopolis\index[namen]{Jonas!Bischof von
      Parthicopolis} & 39 & 33 & -- & 5 & Macedonia
    \\
    49 & Irenaeus von Scyrus\index[namen]{Irenaeus!Bischof von Scyrus}
    & 36 & 46 & 38 & -- & Achaia
    \\
    50 & Julianus von Thebae Heptapylus\index[namen]{Julianus!Bischof
      von Thebae Heptapylus} & 37 & 22 & 3?/29 & -- & Achaia
    \\
    % 55 & Julianus\index[namen]{Julianus!Bischof} & 37?  & -- & 3 &
    % -- & ?
    % \\
    51 & Julius von Rom \index[namen]{Julius!Bischof von Rom}(per
    Arcidamum\index[namen]{Archidamus!Presbyter in Rom} et
    Philoxenum\index[namen]{Philoxenus!Presbyter in Rom} presbyteros
    et Leonem\index[namen]{Leo!Diakon in Rom} diaconum) & 2 & -- & 20
    & -- & Roma
    \\
    52 & Lucius von Verona\footnote{Ath., apol.Const. 3,6 mit Verweis
      auf Prosop.chr. II/2,
      1332f. Lucillus}\index[namen]{Lucius!Bischof von Verona} & 9?
    59?  & 20 & 30 & 15?  & Italia
    \\
    53 & Lucius von Hadrianopolis\index[namen]{Lucius!Bischof von
      Hadrianopolis} & 9? 59?  & 19 & 17 & 15?  & Thracia
    \\
    54 & Macedonius von Ulpiana\index[namen]{Macedonius!Bischof von
      Ulpiana} & 5 & 35 & 39 & -- & Dardania
    \\
    55 & Marcellus von Ankyra\index[namen]{Markell!Bischof von Ancyra}
    & 42 & 10 & 6 & -- & Galatia
    \\
    56 & Marcus von Siscia\index[namen]{Marcus!Bischof von Siscia} &
    50 & 52 & 53 & -- & Savia (Pannonia)
    \\
    57 & Martyrius von Naupactus\index[namen]{Martyrius!Bischof von
      Naupactus} & 32?  & 47 & 15?/40 & -- & Achaia
    \\
    % 63 & Martyrius\index[namen]{Martyrius!Bischof} & 32?  & -- & 15
    % & -- & ?
    % \\
    58 & Maximus von Luca\index[namen]{Maximus!Bischof von Luca} & 64
    & 7 & -- & -- & Tuscia
    \\
%    59 & Maximinus von Trier\index[namen]{Maximinus!Bischof von Trier}
%    & [112] & -- & [19\footnote{Maximinus war aber offensichtlich
%      nicht anwesend, sondern hat nur brieflich zugestimmt.}] & -- &
%    Gallia
%    \\
    59 & Musaeus von Thebae Phthiotides\index[namen]{Musaeus!Bischof
      von Thebae Phthiotides} & 53 & 13 & -- & -- & Thessalia
    \\
    60 & Musonius von Heraclium\index[namen]{Musonius!Bischof von
      Heraclium} & 71 & -- & 58 & -- & Creta
    \\
    61 & Ossius von Cordoba\index[namen]{Ossius!Bischof von Cordoba} &
    1 & 1 & -- & 1 & Spania
    \\
    62 & Olympius von Aenus\index[namen]{Olympius!Bischof von Aenus} &
    -- & -- & 50 & -- & Rhodope
    \\
    63 & Palladius von Dium\index[namen]{Palladius!Bischof von Dium} &
    17 & 55 & 41 & -- & Macedonia
    \\
    64 & Paregorius von Scupi\index[namen]{Paregorius!Bischof von
      Scupi} & 55 & 29 & 36 & 7 & Dardania
    \\
    65 & Patricius\index[namen]{Patricius!Bischof} & 76 & -- & -- & --
    & ?
    \\
    66 & Petrus\index[namen]{Petrus!Bischof} & 69 & -- & -- & -- & ?
    \\
    67 & Philologius\index[namen]{Philologius!Bischof} & 73 & -- & --
    & -- & ?
    \\
    68 & Plutarchus von Patras\index[namen]{Plutarchus!Bischof von
      Patras} & 56 & 38 & -- & 22 & Achaia
    \\
    69 & Porphyrius von Philippi\index[namen]{Porphyrius!Bischof von
      Philippi} & 23 & 9 & 8 & 11 & Macedonia
    \\
    70 & Praetextatus von Barcilona\index[namen]{Praetextatus!Bischof
      von Barcilona} & 7 & 6 & 25 & -- & Spania
    \\
    71 & Protasius von Mailand\index[namen]{Protasius!Bischof von
      Mailand} & 21 & 51 & 52 & -- & Italia
    \\
    72 & Protogenes von Serdica\index[namen]{Protogenes!Bischof von
      Serdica} & 3 & 16 & 1 & -- & Dacia
    \\
    73 & Restutus\index[namen]{Restutus!Bischof} & 41 & -- & -- & 26 &
    \\
    74 & Rheginus von Scopelus\index[namen]{Rheginus!Bischof von
      Scopelus}\footnote{Genannt ASS Febr. III, 495.}  & -- & -- & --
    & -- & Insulae
    \\
    75 & Sapricius\index[namen]{Sapricius!Bischof} & 78 & -- & -- & --
    & ?
    \\
    76 & Severus von Ravenna\index[namen]{Severus!Bischof von Ravenna}
    & [249?] 6?  & 49 & -- & 25?  & Italia
    \\
    77 & Severus von Chalcis\index[namen]{Severus!Bischof von Chalcis}
    & 6?  & -- & 28 & 25?  & Thessalia
    \\
    78 & Socras von Phoebia am Asopus\index[namen]{Socras!Bischof von
      Phoebia am Asopus} (von Phoebia am Asopus) & 30 & 43 & -- & -- &
    Achaia
    \\
    79 & Spudasius\index[namen]{Spudasius!Bischof} & 74 & -- & -- & --
    & ?
    \\
    80 & Stercorius von Canusium\index[namen]{Stercorius!Bischof von
      Canusium} & 16 & 44 & 33\footnote{�berliefert ist der Name als
      Sunosius.}  & -- & Apulia
    \\
    81 & Symphorus von Hierapytna\index[namen]{Symphorus!Bischof von
      Hierapytna} & 70 & -- & 57 & -- & Creta
    \\
    82 & Tryphon von Macaria\index[namen]{Tryphon!Bischof von Macaria}
    & 65 & 30 & 35 & -- & Achaia
    \\
    83 & Ursacius von Brixia\index[namen]{Ursacius!Bischof von Brixia}
    & 8 & 50 & 43 & -- & Italia
    \\
    84 & Valens von Oescus\index[namen]{Valens!Bischof von Oescus} &
    45 & 54 & -- & 9 & Dacia ripensis
    \\
    85 & Verissimus von Lugdunum\index[namen]{Verissimus!Bischof von
      Lugdunum} & [80] & 53 & 42 & -- & Gallia
    \\
    86 & Vincentius von Capua\index[namen]{Vincentius!Bischof von
      Capua} & 15 & 14 & -- & 23 & Campania
    \\
    87 & Vincentius\index[namen]{Vincentius!Bischof} & -- & -- & -- &
    27 & ?
    \\
    88 & Vitalis von Aquae\index[namen]{Vitalis!Bischof von Aquae} &
    11?  44?  & 28 & -- & 24?  & Dacia ripensis
    \\
    89 & Vitalis ">Vertaresis"<\index[namen]{Vitalis!Bischof
      ">Vertaresis"<}\footnote{Der Ort ist nicht zu identifizieren.}
    & 11? 44?  & -- & 55 & 24?  & Africa
    \\
    90 & Zosimus von Lychnidus\index[namen]{Zosimus!Bischof von
      Lychnidus} & 25? 27? 75?  & 23 & 9?/32 & -- & Macedonia
    \\
    91 & Zosimus von Horreum Margi\index[namen]{Zosimus!Bischof von
      Horreum Margi} & 25? 27? 75?  & -- & 9?/51 & -- & Moesia
    \\
    92 & Zosimus\index[namen]{Zosimus!Bischof|dub} & 25? 27? 75?  & --
    & 9?  & -- & ?
    \\
  \end{longtable}


%%% Local Variables: 
%%% mode: latex
%%% TeX-master: "dokumente_master"
%%% End: 


    \begin{figure}
	\label{SerdicaKarte}
      \begin{center}
        \includegraphics[scale=.8]{karte_serdica.pdf}
      \end{center}
      % \caption*{Die Herkunftsorte der Teilnehmer der Synode von
      %   Serdica}
    \end{figure}

    Die Teilnahme von Bisch�fen an der ">westlichen"< Synode verteilt auf Provinzen stellt sich
    somit wie folgt dar:

      \begin{longtable}[c]{p{2cm}p{11cm}}
        \toprule Provinz/Di�zese & Personen
        \\
        \midrule
        \endhead
        \midrule
        \endfoot
        \bottomrule
	& \\
        \caption*{Zuordnung der Bisch�fe der ">westlichen"< Synode
          zu Provinzen}
        \endlastfoot
        Spania & Domitianus von Asturica, Praetextatus von Barcino,
        Castus von Caesaraugusta, Annianus von Castellona, Ossius von
        Cordoba, Florentius von Emerita Augusta
        \\
        Africa & Vitalis ">Vertaresis"<
        \\
        Gallia & Verissimus von Lugdunum
        \\
        Italia & Fortunatianus von Aquileia, Ursacius von Brixia,
        Protasius von Mediolanum, Severus von Ravenna, Lucius von
        Verona
        \\
        Tuscia & Maximus von Luca
        \\
        Rom & Julius von Rom
        \\
        Campania & Januarius von Beneventum, Vincentius von Capua,
        Calepodius von Neapolis
        \\
        Apulia & Stercorius von Canusium
        \\
        Pannonia & Aprianus von Poetovio, Eutherius
        \\
        Savia & Marcus von Siscia
        \\
        Moesia & Zosimus von Horreum Margi, Amantius von Viminacium
        \\
        Dacia Ripensis & Vitalis von Aquae, Calvus von Castra Martis,
        Valens von Oescus
        \\
        Dacia & Protogenes von Serdica
        \\
        Dardania & Gaudentius von Na"issus, Paregorius von Scupi,
        Macedonius von Ulpiana
        \\
        Macedonia & A"etius von Thessalonike, Gerontius von Beroea,
        Bassus von Diocletianopolis, Palladius von Dium, Eugenius von
        Heraclea Lyncestis, Zosimus von Lychnidus, Antigonus von
        Pallene, Jonas von Parthicopolis, Porphyrius von Philippi
        \\
        Rodope & Olympius von Aenus
        \\
        Europe & Eutherius von Ganus
        \\
        Haemimons & Lucius von Hadrianopolis
        \\
        Asia & Theodorus von Tenedus
        \\
        Thessalia & Severus von Chalcis, Hymenaeus von Hypata,
        Alexander von Larisa, Musaeus von Thebae Phthiotides
        \\
        Epirus & Heliodorus von Nicopolis
        \\
        Cyclades & Dioscurus von Therasia
        \\
        Creta & Eucissus von Cissamus, Cydonius von Cydonia, Aelianus
        von Gortyna, Musonius von Heraclium, Symphorus von Hierapytna
        \\
        Achaia & Alexander von Corone, Alexander von Cyparissia,
        Athenodorus von Elatia, Dionysius von Elis, Tryphon von
        Macaria, Alypius von Megara, Eutychius von Methone, Martyrius
        von Naupactus, Eucarpus von Opus, Plutarchus von Patras,
        Socras von Phoebia am Asopus, Irenaeus von Scyrus, Hermogenes
        von Sicyon
        \\
        Aegyptus & Athanasius von Alexandria
        \\
        Galatia & Marcellus von Ancyra
        \\
        Palaestina & Asclepas von Gaza, Arius von Petra
        \\
        Arabia & Asterius
        \\
      \end{longtable}

      Betrachtet man diese Listen und die Karte auf S. \pageref{SerdicaKarte}, so ist augenf�llig,
      da� �berdurchschnittlich viele Bisch�fe, n�mlich rund ein Drittel, aus den Provinzen Achaia (13), Macedonia (9), Creta (5) und Thessalia (4) kamen, w�hrend viele andere Provinzen nur durch einen oder zwei Bisch�fe und die Bisch�fe des eigentlichen Westens (aus den Provinzen Italiens, Galliens, Spaniens und Afrikas) �berhaupt nur sehr schwach vertreten waren. Die �berwiegende Mehrheit der Teilnehmer der ">westlichen"< Synode stammte somit aus dem griechisch-sprachigen Bereich.

Demgegen�ber weist die Zusammensetzung der ">�stlichen"< Synode eine viel gleichm��igere Verteilung auf die Provinzen auf, wie die folgende Tabelle zeigt:

      \begin{longtable}[c]{p{2cm}p{11cm}}
        \toprule Provinz & Personen
        \\
        \midrule
        \endhead
        \midrule
        \endfoot
        \bottomrule
        & \\
        \caption*{Zuordnung der Bisch�fe der ">�stlichen"< Synode
          zu Provinzen}
        \endlastfoot
        Pannonia & Valens von Mursa
        \\
        Moesia & Ursacius von Singidunum
        \\
        Thracia & Demophilus von Beroea, Euticius von Philippopolis
        \\
        Haemimons & Timotheus von Anchialus
        \\
        Europe & Theodorus von Heraclea
        \\
        Hellespontus & Leucadas von Ilium, Niconeus von Troas
        \\
        Insulae & Bassus von Carpathus, Edesius von Cos, Agapius von
        Tinus
        \\
        Bithynia & Thelafius? von Chalcedon, Adamantius von Cios
        \\
        Asia & Menophantus von Ephesus, Eusbius von Magnesia, Eusebius
        von Pergamum
        \\
        Caria & Ambracius von Milet
        \\
        Lydia & Florentius von Ancyra, Pantagathus von Attalia
        \\
        Pamphilia & Eugenius von Lysinia, Sissinius von Perge
        \\
        Phrygia I & Antonius von Docimion, Eusebius von Dorylaeum
        \\
        Phrygia II & Sabinianus von Chadimena?, Flaccus von Hierapolis
        \\
        Paphlagonia & Philetus von Cratia
        \\
        Pisidia & Nonnius von Laodicea
        \\
        Galatia & Basilius von Ancyra, Carterius von Aspona, Philetus
        von Iuliopolis, Pison von Trocnada
        \\
        Pontus & Eulalius von Amasia, Theodulus von Neocaesarea,
        Prohaeresius von Sinope, Bitynicus von Zela
        \\
        Cappadocia & Dianius von Caesarea, Pancratius von Parnassus
        \\
        Cilicia & Dionysius von Alexandria, Pison von Adana, Narcissus
        von Neronias, Macedonius von Mopsuestia, Cyrotus von Rhosus
        \\
        Syria Coele & Marcus von Arethusa, Stephanus von Antiochia,
        Olympius von Doliche, Eustathius von Epiphania, Eudoxius von
        Germanicia, Gerontius von Raphanie, Antonius von Zeugma
        \\
        Phoenice & Macedonius von Berytus, Dominicus von Polidiane?,
        Vitalis von Tyrus
        \\
        Palaestina & Acacius von Caesarea, Quintianus von Gaza
        \\
        Arabia & Antonius von Bostra, Quirius von Philadelphia
        \\
        Aegyptus & Isaac von Letopolis, Callinicus von Pelusium,
        Eudaemon von Tanis, Timotheus
        \\
        Theba"is & Lucius von Antinous, Theogenes von Lycia
        \\
      \end{longtable}
    \end{description}
  \end{praefatio}
%%% Local Variables: 
%%% mode: latex
%%% TeX-master: ``dokumente_master''
%%% End: 

\count\Dfootins=907
%%%% Input-Datei OHNE TeX-Pr�ambel %%%%
% \renewcommand*{\goalfraction}{.7}
% \renewcommand*{\footfudgefiddle}{69}
\section{Rundbrief der ">westlichen"< Synode}
% \label{sec:43.1}
\label{sec:SerdicaRundbrief}
\begin{praefatio}
  \begin{description}
  \item[Herbst 343]Zur Datierung vgl. oben
    Dok. \ref{ch:SerdicaEinl}. Der Rundbrief rechtfertigt die
    Entscheidung der westlichen Teilsynode von
    Serdica\index[synoden]{Serdica!a. 343}, die im Osten
    exkommunizierten Bisch�fe, besonders
    Athanasius\index[namen]{Athanasius!Bischof von Alexandrien},
    Markell\index[namen]{Markell!Bischof von Ancyra} und
    Asclepas\index[namen]{Asclepas!Bischof von Gaza}, in die
    Kirchengemeinschaft aufgenommen und umgekehrt f�hrende Bisch�fe
    des Ostens wie Theodorus von
    Heraclea\index[namen]{Theodorus!Bischof von Heraclea}, Narcissus
    von Neronias\index[namen]{Narcissus!Bischof von Neronias},
    Stephanus von Antiochien\index[namen]{Stephanus!Bischof von
      Antiochien}, Georg von Laodicea\index[namen]{Georg!Bischof von
      Laodicea}, Acacius von Caesarea\index[namen]{Acacius!Bischof von
      Caesarea}, Menophantus von
    Ephesus\index[namen]{Menophantus!Bischof von Ephesus}, Ursacius
    von Singidunum\index[namen]{Ursacius!Bischof von Singidunum},
    Valens von Mursa\index[namen]{Valens!Bischof von Mursa} und deren
    Anh�nger abgesetzt zu haben. Hauptargument neben dem
    H�resievorwurf (� 2; 13; 16~f.) ist die Kritik an der
    Verweigerungshaltung, schon damals in
    Rom\index[synoden]{Rom!a. 341} (� 4;
    vgl. Dok. \ref{sec:BriefSynode341}) und auch jetzt in
    Serdica\index[synoden]{Serdica!a. 343} nicht zu gemeinsamen
    Verhandlungen erschienen zu sein, sondern im Gegenteil sogar die
    Flucht ergriffen zu haben (� 5--8; 13). Zudem habe die �berpr�fung
    der Vorw�rfe gegen Athanasius\index[namen]{Athanasius!Bischof von
      Alexandrien} (� 9~f.), Markell\index[namen]{Markell!Bischof von
      Ancyra} und Asclepas\index[namen]{Asclepas!Bischof von Gaza} (�
    11~f.) deren Unschuld ergeben. Wie eine Gegendarstellung l��t sich
    der Rundbrief der �stlichen Teilsynode von Serdica lesen
    (Dok. \ref{sec:RundbriefSerdikaOst}).
  \item[�berlieferung]
    % Sprachliche Untersuchungen zur Ekthesis der westlichen Synode
    % ergeben, da� die Glaubenserkl�rung und damit auch der westliche
    % Synodalbrief urspr�nglich in lateinischer Sprache abgefa�t waren
    % (Vgl. Dok \ref{sec:SerdicaWestBekenntnis} Anm. zu 3). Vermutlich
    % erfolgte aber eine �bersetzung ins Griechische noch vor Ort.
    % Sowohl die �berlieferung bei Hilarius als auch im Codex
    % Veronensis LX erweist sich als R�ck�bersetzung aus dem
    % Griechischen, w�hrend der Text Theodorets sich als �bersetzung
    % aus dem Lateinischen erweist. In der Fassung bei Athanasius
    % lassen sich keine sprachlichen Hinweise darauf finden, da� sie
    % aus dem Lateinischen �bersetzt w�re.
    Die uns erhaltenen Zeugnisse des Rundbriefes der ">westlichen"<
    Synode von Serdica lassen sich gerade auch wegen der gro�en
    �bereinstimmung der Fassungen bei Theodoret und Athanasius alle
    auf eine griechische Fassung zur�ckf�hren. Die lateinischen
    Versionen bei Hilarius und im Codex Veronensis sind �bersetzungen
    dieser griechischen Fassung. Da sich bei Theodoret aber noch
    Spuren einer lateinischen Vorlage fassen lassen, mu� man
    davon ausgehen, da� der Text urspr�nglich lateinisch abgefa�t
    war. Bei Athanasius ist der Nachweis einer lateinischen Vorlage
    zwar nur schwer zu erbringen, doch mu� man aufgrund der
    insgesamt gro�en �bereinstimmung mit Theodoret annehmen, da� beide
    letztlich auf eine autorisierte �bersetzung des lateinischen
    Originals zur�ckgehen, die vermutlich noch auf der Synode selbst
    angefertigt worden war. Folgende Nachweise k�nnen f�r diese
    Rekonstruktion der �berlieferung erbracht werden:

    Sowohl die �berlieferung bei Hilarius als auch im Codex Veronensis
    LX erweist sich als R�ck�bersetzung aus dem Griechischen. Der
    Begriff \textit{inultas} bei Hilarius (� 15) ist am ehesten durch
    eine �bersetzung aus einer griechischen Vorlage zu
    erkl�ren; entweder wurde hierbei das richtige
    \griech{>anekdihg'htouc} bei der �bersetzung f�lschlich zu
    \griech{>anekdik'htouc} verlesen oder war bereits in der Vorlage
    verschrieben. Auch an anderen Stellen l��t die oft umst�ndliche
    Syntax bei Hilarius eine �bersetzung vermuten. Der im Codex
    Veronensis LX �berlieferte Text des Bekenntnisses
    (Dok. \ref{sec:SerdicaWestBekenntnis}) kann durch die Verwendung
    von \textit{sol} in Dok. \ref{sec:SerdicaWestBekenntnis},11, was
    auf eine Verlesung von \griech{HLOGOS} zu \griech{HLIOS}
    zur�ckzuf�hren ist, eindeutig als �bersetzung aus dem Griechischen
    nachgewiesen werden. Da Rundbrief und Bekenntnis hier gemeinsam
    �berliefert sind und beide Texte der �berlieferung bei Theodoret
    sehr nahe stehen, ist auch f�r den Rundbrief im Codex Veronensis
    LX von einer R�ck�bersetzung aus dem Griechischen auszugehen.

    Theodoret l��t eine urspr�nglich lateinische Fassung erkennen.
    Dies ergibt sich aus der falschen Wiedergabe eines vermutlich
    originalen \textit{decessit} mit \griech{>an'esth} in � 6, da
    vom Sinn her an dieser Stelle ein Verb mit der Bedeutung
    ">sterben"< notwendig ist. Schwierig ist der Nachweis, da� auch
    der Text des Athanasius auf eine lateinische Vorlage zur�ckgeht.
    Ein Hinweis darauf ist die Formulierung \griech{>hsq'unonto m`en \dots\ d`e} in � 7
    in der Version des Athanasius. Die Varianten der einzelnen Autoren
    sind am besten durch einen urspr�nglich lateinischen Satz mit
    Hauptverb zu erkl�ren, der als Begr�ndung zum vorhergehenden Satz
    zu ziehen war, aber gedanklich auch mit dem nachfolgenden Satz
    verbunden werden konnte. Die griechische Vorlage, auf die
    Theodoret und Hilarius zur�ckgehen, verwandelte daher das
    Hauptverb zu einem Partizip und zog dieses mit dem folgenden Satz
    zusammen. Dadurch kommt es bei Hilarius und Theodoret zu einer
    Spannung im Text, da man erwartet, da� der folgende Satz mit einem
    ">aber"< angeschlossen wird, das aber in der Vorlage offenbar
    fehlte. Auch dem Verfasser des Textes im Codex Veronensis LX
    entging dieser Bruch nicht. Er verwandelte daher das Partizip in
    ein Hauptverb zur�ck und leitete den Satz mit \textit{et}
    ein. Athanasius, der wohl nicht auf dieselbe Vorlage wie Hilarius
    und Theodoret zur�ckgeht, blieb~-- vermutlich ausgehend von einer
    lateinischen Vorlage~-- bei dem urspr�nglichen Hauptverb und f�gte
    \griech{m`en \dots\ d`e} als Verbindungspartikel ein. Ein Beweis
    f�r eine direkte �bersetzung aus dem Lateinischen fehlt aber, da
    durch dieses Beispiel nur wahrscheinlich gemacht ist, da� alle uns
    erhaltenen Fassungen auf einen lateinischen Text zur�ckgehen.  Es
    ist aber zu vermuten, da� noch vor Ort in Serdica eine griechische
    Version des Textes erstellt wurde. In der uns noch erhaltenen
    �berlieferung des Textes kommen somit die griechischen Textzeugen,
    auch wenn es sich um �bersetzungen handelt, dem Ursprungstext
    n�her als die lateinischen Textzeugen, die sich als
    R�ck�bersetzungen griechischer �bersetzungen erweisen.

    In dem nur von Theodoret und dem Codex Veronensis LX �berlieferten
    Bekenntnis gibt es ein Indiz daf�r, da� der Text urspr�nglich
    lateinisch verfa�t ist. Der Einschub in
    Dok. \ref{sec:SerdicaWestBekenntnis},3, in dem klargestellt wird,
    da� die Griechen \griech{<up'ostasic} mit \griech{o>us'ia}
    wiedergeben, ergibt nur Sinn, wenn es an dieser Stelle um die
    griechische �bersetzung f�r \textit{substantia} geht.

    Obwohl sich Theodoret und der Codex Veronensis LX sehr nahe stehen
    und der Codex Veronensis LX den Text Theodorets bis in Details
    hinein nachahmt (vgl. � 6 \textit{subterfugiunt}), verbietet sich
    die Annahme, da� der Codex Veronensis LX eine direkte �bersetzung
    des Theodoret-Textes ist. Ein \textit{bonitatem} in � 2 ist nicht
    aus \griech{>ako`hn} zu erkl�ren. Schwieriger ist es, aufgrund des
    geringen Textumfanges von Synodalbrief und Bekenntnis das
    Verh�ltnis der �berlieferung bei Athanasius und Hilarius zu
    kl�ren. In � 8 hat Athanasius mit \griech{>ekoin'wnhsan} einen
    Fehler gegen alle �brigen Textzeugen. Er trifft mit \griech{di''
      a>uto~u} in � 3 jedoch das Richtige gegen die Haupt�berlieferung
    der drei anderen Textzeugen. Wenn einige Handschriften in der
    Theodoret�berlieferung eine �hnliche Variante wie Athanasius
    aufweisen, spricht dies f�r eine nachtr�gliche Kontaminierung mit
    der �berlieferung des Athanasius.

    \begin{figure}[h] % Stemma
      \begin{scriptsize}
	\begin{center}
          \begin{tikzpicture}
            \node (LatT) at (0,0) {\griech{W}}; \node (Lat) at (0,-.3)
            {(wahrscheinlich lateinisch)}; \node (Gr) at (0,-1.2) {};
            \node (GrT) at (0,-1) {Griechische �bersetzung}; \node
            (Ath) at (-1.5,-2.5) {Ath.}; \node (Gamma) at (1.5,-2.5)
            {\griech{g}}; \node (Alpha) at (1,-3.5) {\griech{a}};
            \node (Hil) at (2.5,-3.5) {Hil.}; \node (Thdt) at
            (0.5,-4.5) {Thdt.}; \node (Cod) at (2,-4.5) {Cod.Ver.};
            \draw (Lat) -- (GrT); \draw (Gr) -- (Ath); \draw (Gr) --
            (Gamma); \draw (Gamma) -- (Hil) node[midway,sloped,above]
            {�bers.}; \draw (Gamma) -- (Alpha); \draw (Alpha) --
            (Thdt); \draw (Alpha) -- (Cod) node[midway,sloped,above]
            {�bers.}; \draw[dotted] (Ath) -- (Thdt);
          \end{tikzpicture}
          \legend{Stemma f�r Dok. \ref{sec:SerdicaRundbrief}}
        \end{center}
      \end{scriptsize}
    \end{figure}


    Zu den \textit{Collectanea antiariana Parisina}: P. Pithou entdeckte 1590 eine
    bis dahin unbekannte Handschrift aus dem 15. Jh. (= T). Sie
    enthielt u.\,a. kirchenpolitisch relevante Dokumente des 4. und
    5. Jh., die z.\,T. von einem Autor des 4. Jh. kommentiert worden
    waren. Auf den ersten Blick l��t die Sammlung der Einzelst�cke
    keinen inneren Zusammenhang erkennen. Der erste Teil tr�gt weder
    Titel noch �berschrift, der zweite ist aufgrund des Incipit und
    Explicit Hilarius von Poitiers zuzuschreiben. Die von Pithou
    begonnene Erstedition konnte erst  1598 von N. Faber
    fertiggestellt werden. Allerdings konnte erst A. Feder in seiner
    kritischen Ausgabe von 1916 den Archetyp der bis zu diesem
    Zeitpunkt bekannten Handschriften, den Cod. Parisinus Armamentarii
    lat. 483 aus dem 9. Jh. (= A) einarbeiten. Feder bel��t in seiner
    Ausgabe die Anordnung der in den Handschriften �berlieferten
    Reihenfolge. Die titellose Serie bezeichnet er als Serie A, die
    eindeutig Hilarius zuzuordnende als Serie B und gibt dem gesamten
    Werk den Titel ">Collectanea antiariana Parisina"<. Inhaltlich kreisen
    die Dokumente um drei Themenkomplexe: Die Synode von Serdica und
    der Fall des Athanasius (1), der Fall der Liberius (2) und die
    Doppelsynode von Rimini und Seleucia (3).  Seit Coustant (1693),
    der sich auf Zeugnisse des Hieronymus und des Rufin st�tzte (Hier,
    vir.\,ill. 100 und Ruf., adult.\,libr.\,Orig. 11), galten die Fragmente
    als Teile eines verlorenen Werkes \textit{Adversus Valentem et
    Ursacium}, das Hilarius 359/360 in Konstantinopel verfa�t
    und dem er sp�ter einige Dokumente beigef�gt
    hatte. Marx\footcite{Marx:Zeugen} und Wilmart\footcite{Wilmart:Constantium}
    konnten jedoch nachweisen, da� Phoebadius das Werk bereits 358
    gekannt hat. Wilmart\footcite{Wilmart:Constantium} hat zudem
    aufgezeigt, da� auch der Liber I ad Constantium zusammen mit dem
    sich daran anschlie�enden erz�hlenden Text des Hilarius zu diesem
    historischen Werk geh�rt. Feder bezieht sich in seiner Ausgabe
    weitgehend auf die Ergebnisse von Wilmart. Demnach sind die
    Collectanea antiariana Parisina Teile des von Hieronymus bezeugten
    \textit{Liber adversus Valentem et Ursacium}, der sich aber aus
    vermutlich 3 urspr�nglich unabh�ngig voneinander entstandenen
    B�chern zusammensetzte. Der erste Band sei um 356 noch vor dem
    Exil des Hilarius entstanden, der zweite nach der Synode von
    Seleukia und der dritte nach
    367. H.\,C. Brennecke\footcite[301--312]{Brennecke:Hilarius} kommt
    bei seiner Interpretation der Collectanea antiariana
    Parisina zu einem anderen Ergebnis als Wilmart und Feder. Danach
    l��t sich kein Reflex auf die Synode von B�ziers
    ausmachen. Zudem war Hilarius das Nicaenum vor seinem Exil
    unbekannt, das aber in den Collectanea antiariana
    Parisina im Zusammenhang mit den Texten von Serdica �berliefert
    wird und damit wohl dem ersten Band, d.\,h. dem \textit{Liber primus
      adversus Valentem et Ursacium} zuzuordnen ist. Es ist daher davon auszugehen, da� 
    der \textit{Liber primus adversus Valentem et Ursacium} nach
    der Synode von Sirmium 357 im Exil im Osten entstanden ist und somit auch die
    Liberiusbriefe
    enthielt. Die Sammlung der uns in den Collectanea antiariana Parisina
    vorliegenden Bruchst�cke aus dem \textit{Liber adversus Valentem
      et Ursacium} geht auf einen uns unbekannten Exzerptor des
    5. Jh. zur�ck.

    % \frage{Feder hat die Fragmente der Collectanea antiariana
    %   Parisina in zwei Serien (A und B) unterteilt. Series B l��t
    %   sich eindeutig als Werk des Hilarius bezeichnet, die Fragmente
    %   von Series A sind nur aus dem Kontext, in dem sie �berliefert
    %   sind, Hilarius zuordnen. Andr�
    %   Wilmart\footcite{Wilmart:Constantium} konnte nachweisen, da�
    %   auch der Liber I ad Constantium zusammen mit dem sich daran
    %   anschlie�enden erz�hlenden Text des Hilarius zu diesem
    %   historischen Werk geh�rt, das vielleicht den Titel ">Gegen
    %   Valens und Ursacius"< trug und nach Lionel
    %   Wickham\footcite[XIV--XVI]{Wickham:Hilary} in drei Etappen von
    %   356 bis 366 entstanden ist.}


    Der \textit{Codex Veronensis LX (58)} umfa�t 126 Folia. Rein �u�ere
    Unterschiede zeigen, da� der Codex sekund�r aus zwei verschiedenen
    Werken zusammengef�gt ist. Obwohl sich beide Teile hinsichtlich
    Zeilenz�hlung und Lagenbezeichnungen unterscheiden, sind sie
    jedoch aufgrund der Schrift einem einzigen Schreiber
    zuzuordnen. Die Schrift selbst ist nach der Untersuchung von Field
    eine r�mische Unciale aus dem sp�ten
    8. Jh.\footcite[56--58]{Field:Communion} Damit weist er dem Codex
    einen r�mischen Ursprung zu und verwirft Telfers These von einer
    Entstehung in Bobbio.\footcite[220--222]{Telfer:CodVer} Eine
    Lokalisierung des Codex in Verona ist erst ab dem 10. Jh. aufgrund
    der Randbemerkungen m�glich, die von der Schrift her oftmals
    Bischof Ratherius von Verona zugeschrieben werden.  Der erste Teil
    des Codex (ff. 1--35) umfa�t die Akten des Konzils von Karthago
    von 419 sowie die Briefe an Papst Bonifatius (424) und Papst
    Coelestin (425). Der zweite Teil (ff. 37--126) enth�lt
    verschiedene v.\,a. kirchenrechtlich relevante Texte wie
    Kanonessammlungen und Synodalschreiben, die sich z.\,T. als
    �bersetzungen �stlicher Quellen erweisen. Das letzte Folio endet
    mit der Unterschrift eines Diakons Theodosius, weshalb in der
    Vergangenheit die gesamte Sammlung des Codex nach ihm benannt
    wurde. Tats�chlich kann man jedoch davon ausgehen, da� die
    Unterschrift ebenfalls aus der Vorlage �bernommen wurde und
    zum letzten Text der Sammlung geh�rt.  Folio 36, aus
    d�nnerem Material und gr��er als die �brigen Bl�tter, wurde vom
    Buchbinder zwischen die beiden Teile eingebunden und enth�lt
    eine Kopie des Gedichtes \textit{Dalmatiane, jugi, Caesar} aus dem
    Jahr 474. Die Folia 94--96 und 99 mit den Unterschriften der
    ">westlichen"< Synode von Serdica wurden von Bischof
    Ratherius ausradiert und u.\,a. mit den Kanones der ">westlichen"< Synode von Serdica
    �berschrieben. Er f�gte dabei auch ff. 97 und 98 ein.  Was zur
    Vereinigung der beiden Teile gef�hrt hat, ist nicht mehr
    auszumachen; ob die \textit{causa Apiarii}, die auf dem Konzil von
    Karthago 419 verhandelt wurde, das geschichtliche Ereignis ist,
    das zur Entstehung des Codex gef�hrt hat, ist eher zu
    bezweifeln.\footnote{Vgl. die Diskussion bei
      \cite[64--82]{Field:Communion}.}
    % \LitNil\ Satz? \LitNil\ so da� uns die griechischen Textzeugen
    % dennoch ein besseres Bild vom Originaltext liefern.  Da� es sich
    % bei Hilarius um eine �bersetzung aus dem Griechischen handelt,
    % zeigt die oftmals umst�ndliche Satzkonstruktion. Zum Teil wird
    % die Konstruktion der griechischen Vorlage �berhaupt nicht
    % verstanden. F�r den Codex Veronensis LX als R�ck�bersetzung vgl.
    % Dok. \ref{sec:SerdicaWestBekenntnis} Anm. zu 10. Aus den
    % Untersuchungen zur Text�berlieferung ergibt sich, da� Athanasius
    % als eigener �berlieferungsstrang Theodoret, dem Codex Veronensis
    % und Hilarius gegen�bersteht. Der Codex Veronensis LX stimmt
    % h�ufig w�rtlich mit Theodoret �berein. Dennoch gibt es auch
    % einige Gemeinsamkeiten von Hilarius und Theodoret dem Codex
    % Veronensis LX gegen�ber, so da� der Codex Veronensis LX wohl
    % nicht in direkter Abh�ngigkeit zu Theodoret steht, sondern
    % Hilarius, Theodoret und der Codex Veronensis LX auf einen
    % gemeinsamen Archetypus zur�ckzuf�hren sind. Cassiodor, der von
    % Theodoret abh�ngig ist, wird i.\,d.\,R. nicht ber�cksichtigt.

    Bei der Erstellung des Textes wurde nach folgenden Richtlinien
    vorgegangen: Bei den Varianten zwischen Athanasius und Theodoret,
    bei denen kein Bedeutungsunterschied festzustellen ist, wurde die
    Variante bevorzugt, die grammatikalisch �blicher ist, womit im
    allgemeinen Theodoret der Vorzug gegeben wird.
%   Bei der
%     Einarbeitung der lateinischen Textzeugen wurden Abweichungen, die
%     sich durch die R�ck�bersetzung erkl�ren, nicht
%     ber�cksichtigt. Dazu geh�ren Umstellungen im Satzbau, kleinere
%     grammatikalische Abweichungen (verschiedene Tempora, verschiedene
%     M�glichkeiten der Auf"|l�sung von Partizipien, z.\,T. Wechsel von
%     Singular und Plural, Wechsel von Konjunktionalsatz und AcI) und
%     verschiedene W�rter bei gleicher Semantik.
% 
%     Es werden dagegen folgende Varianten angegeben: Varianten bei
%     Hilarius und im Codex Veronensis LX, die f�r die Abh�ngigkeit der
%     Texte relevant sind; semantisch abweichende W�rter; Erg�nzungen
%     und Auslassungen; grammatikalische Abweichungen, die eine
%     wirkliche Sinn�nderung bedeuten. Wenn bereits in der griechischen
%     �berlieferung kleinere Varianten vorhanden sind, werden zum
%     Vergleich auch Hil. und der Cod.Ver. angegeben.

    Hier wurde f�r den Hilariustext
    der Collectanea antiariana Parisina nur auf den Codex A
    zur�ckgegriffen und der �bersichtlichkeit wegen auf eine weitere
    Differenzierung der �berlieferung verzichtet. Mit Hil. ist somit
    in Dok.  \ref{sec:SerdicaRundbrief} immer Hil.(A) gemeint. Diese
    Vorgehensweise l��t sich dadurch rechtfertigen, da� die Codices C
    und T ohnehin Abschriften von A sind und selbst in der Ausgabe von
    Feder nur bei gewichtigen Abweichungen in den Apparat aufgenommen
    sind. Der Codex S ist zwar von A unabh�ngig, doch sind uns aus
    diesem auch nur noch einige Varianten bei Coustant und Hardouin
    erhalten, die aber f�r dieses Dokument mit A �bereinstimmen.


    % Der Codex Veronensis LX wurde um 700 in Verona von einem
    % ungebildeten Schreiber zusammengestellt. Der erste Teil der
    % Handschrift (35 Bl�tter) ist nicht komplett erhalten und mit dem
    % zweiten nur durch die Buchbindung in einem Codex vereinigt. Der
    % zweite vollst�ndig erhaltene Teil dieser Handschrift enth�lt
    % verschiedene Kanones, Akten und historische Berichte,
    % u.a. einige Dokumente zur �stlichen und westlichen Synode von
    % Serdica\index[synoden]{Serdica!a. 343}. Wenngleich der Kodex
    % einige lateinische Originale enth�lt, ist er gr��tenteils eine
    % R�ck�bersetzung aus dem Griechischen. Er tr�gt die Unterschrift
    % eines Diakons Theodosius. Da dem Verfasser des Codex die
    % Unterschrift stilistisch nicht zuzutrauen ist, mu� man davon
    % ausgehen, da� auf eine �ltere Handschrift zur�ckgegriffen
    % wurde. Dies wirft wiederum die Frage auf, welche Texte diese
    % Handschrift bereits enthielt. Einige sp�tere Zus�tze lassen sich
    % klar abgrenzen.  (Verweis auf Editionen?)

%     F�r den Codex Veronensis LX wurde, sofern von den bisherigen
%     Herausgebern nicht eindeutige Schreibfehler verbessert wurden, auf
%     den Text des Codex selbst zur�ckgegriffen. Dabei wurden die vom
%     heutigen Gebrauch abweichenden Schreibweisen z.T. beibehalten.

%     Generell gilt:
% 
%     F�r den normalen Text werden alle Varianten angegeben, soweit sie
%     anhand der vorliegenden kritischen Textausgaben nachzuvollziehen
%     sind. Ist der Text bei mehreren Autoren �berliefert und es w�rde
%     zu un�bersichtlich, alle Varianten der verschiedenen Textzeugen
%     anzugeben, so wird auf eine genauere Aufschl�sselung der
%     �berlieferung verzichtet. Dann werden auch irrelevante
%     Abweichungen der Textzeugen voneinander wie Schreibvarianten nicht
%     ber�cksichtigt.
% 
%     Bei Namen werden grunds�tzlich folgende Varianten nicht
%     ber�cksichtigt: b/v/f, b/p, g/c, t/d, ti/ci, c/ch, t/th, ph/f,
%     ae/e, Verdopplungen von Buchstaben.

  \item[Fundstelle]Thdt., h.\,e. II 8,1--36 (\editioncite[101,4--112,15]{Hansen:Thdt}); Hil., coll.\,antiar. B II 1 (\editioncite[103,1--126,3]{Hil:coll}); Cod.Ver. LX f. 81a--86a; Ath., apol.\,sec. 42--47
    (\editioncite[119,4--123,25]{Opitz1935})

    % \begin{itemize}
    % \item Ath. steht als eigener �berlieferungsstrang Thdt.,
    %   Hil. und dem Cod.Ver.  gegen�ber. (wollen -- k�nnen;
    %   Kaisertitulaturen, Tod des Theodul, propter ecclesias)
    % \item Der Cod.Ver. ist eine sehr w�rtliche �bersetzung des
    %   Thdt.-Textes. Dies f�hrt z.T. zu einer schematischen
    %   �bersetzung, in der z.B. Komposita exakt in ihren einzelnen
    %   Wortbestandteilen �bertragen werden.
    % \item Dennoch l��t sich nicht ausmachen, ob Cod.Ver. und
    %   Thdt. gegen�ber Hil.  enger zusammengeh�ren oder ob die
    %   Abweichungen des Hil. auf seine mangelnden
    %   �bersetzungsf�higkeiten zur�ckzuf�hren sind. Im letzteren Fall
    %   w�ren Hil., Thdt.  und der Cod.Ver. auf eine gemeinsame
    %   Vorlage zur�ckzuf�hren. Daf�r sprechen auch einige
    %   Gemeinsamkeiten von Hil. und Thdt. gegen den Cod.Ver (propter
    %   id, quod / ipsos / inultas).
    % \item Der Cod.Ver. ist jedenfalls nicht direkt von
    %   Thdt. abh�ngig (bonitas- akoen/ time-reverentia ob ea, quae).
    % \item Die reverentia-Stelle k�nnte bei Thdt. nach
    %   Ath. korrigiert worden sein.  D.h. Thdt. hatte nur time. Aidos
    %   wurde aus Ath. erg�nzt. So k�nnte bei einem Teil der
    %   �berlieferung Thdt.s bestraft gem�� Ath. in umkommentiert
    %   ge�ndert worden sein.
    % \end{itemize}
%
    % Es werden i.d.R. nicht angegeben, weil sie sich durch die
    % �bersetzung erkl�ren:
%
%
    % Es wird entgegen dieser generellen Regel angegeben: Der Apparat
    % f�r Hilarius bei Feder gibt Konjekturen neuzeitlicher
    % Textausgaben und echte Varianten ohne deutliche Kennzeichnung
    % wieder. Diese Konjekturen werden hier i.d.R nicht
    % ber�cksichtigt.  F�r den Cod.Ver. wird die Textfassung bei
    % Turner zugrundegelegt, ohne die Verbesserungen zu
    % ber�cksichtigen, die diese bereits enth�lt.  Bei den Varianten
    % zwischen Ath. und Thdt., bei denen kein Bedeutungsunterschied
    % festzustellen ist, wurde die Variante bevorzugt, die
    % grammatikalisch �blicher ist. Da sowohl Ath. als auch
    % Thdt. �bersetzungen aus dem Lateinischen sind, sind diese
    % Varianten vermtlich auf einen unterschiedlichen
    % �bersetzungsstil zur�ckzuf�hren.

  \end{description}
\end{praefatio}
\begin{pairs}
\selectlanguage{polutonikogreek}
\begin{Leftside}
\beginnumbering
\pstart
\hskip -1.5em\edtext{\abb{}}{\killnumber\Cfootnote{\hskip -1.1em\dt{Thdt.(B
A H N(n)+GS(s)=r LF=z T) Hil. Cod.\,Ver. Ath.(BKO RE)}}}\specialindex{quellen}{section}{Theodoret!h.\,e.!II 8,1--36}\specialindex{quellen}{section}{Hilarius!coll.\,antiar.!B
II 1}\specialindex{quellen}{section}{Codices!Veronensis LX!f. 81a--86a}\specialindex{quellen}{section}{Athanasius!apol.\,sec.!42--47}
\kap{1}\edtext{<H <ag'ia s'unodoc <h kat`a
\edtext{\edtext{jeo~u}{\Dfootnote{q\oline{u} \dt{Thdt.(B)}}}
q'arin}{\lemma{\abb{}}\Dfootnote{\responsio\ q'arin jeo~u \dt{Ath.(K)}}}
\edtext{\edtext{>en Sardik~h|}{\Dfootnote{\dt{\textit{aput Sarsacam}
Cod.\,Ver.}}} sunaqje~isa}{\lemma{\abb{}}\Dfootnote{\responsio\ sunaqje~isa >en Sardik~h|
\dt{Ath.(BK) +} >ap'o te <R'wmhc ka`i Span'iwn ka`i Gall'iwn, >Ital'iac, Kampan'iac, Kalabr'iac, >Afrik~hc, Sardan'iac, Pannon'iac, Mus'iac, Dak'iac, Dardan'iac, >'allhc Dak'iac, Makedon'iac, Jessal'iac, >Aqa'iac, >Hpe'irwn, Jr\aai khc,
<Rod'ophc, >As'iac, Kar'iac, Bijun'iac, <Ellhsp'ontou, Frug'iac, Pisid'iac, Kappadok'iac, P'ontou, Kilik'iac, Frug'iac >'allhc, Pamful'iac, Lud'iac, n'hswn Kukl'adwn, A>ig'uptou,
Jhba\Ad{i}doc, Lib'uhc, Galat'iac, Palaist'inhc, >Arab'iac \dt{Thdt. (Varianten innerhalb der Thdt.-�berlieferung sind nicht vermerkt)
+ \textit{ex Roma, Hispaniis, Galliis, Italiae, Campaniae, Calabriae, Africa, Sardinia, Pannonia, Moesia, Dacia, Dardania, altera Dacia, Macedonia, Thessalia, Achaia, Epero, Thracia, Europe, Palestina, Arbia} Cod.\,Ver.
}}}
% \edtext{[%
% \edtext{\abb{>ap'o
% % \edtext{\abb{te}{\Dfootnote{\latintext > Cod.\,Ver.}} <R'wmhc
% % \edtext{ka`i Span'iwn ka`i Gall'iwn}{\Dfootnote{ka`i Span'iac ka`i Gall'iac
% % \latintext Thdt.(LT) \textit{Hispaniis, Galliis} Cod.\,Ver.}},
% % >Ital'iac,
% % \edtext{Kampan'iac}{\Dfootnote{Kappan'iac \latintext Thdt.(A)}},
% % Kalabr'iac, >Afrik~hc,
% % \edtext{Sardan'iac}{\Dfootnote{Sard'iac \latintext
% % Thdt.(n)}},
% % \edtext{Pannon'iac,}
% % \edtext{Mus'iac, Dak'iac, Dardan'iac, >'allhc}{\lemma{Mus'iac \dots\ >'allhc}
% % \Dfootnote{Dardan'iac >'allhc Mus'iac \latintext
% % Thdt.(T)}}}{\lemma{\abb{Pannon'iac \dots\ >'allhc}}\Dfootnote{\latintext >
% % Thdt.(L) \greektext Panon'iac \dots\ Dardan'iac \latintext Thdt.(G\mg) >
% % Thdt.(F)}} Dak'iac,
% % Makedon'iac,
% % \edtext{\abb{Jessal'iac}}{\Dfootnote{+ Pannon'iac Mus'iac, (+
% % >'allhc \latintext Thdt.(L)) \greektext Dak'iac, Dardan'iac \latintext
% % Thdt.(z)}}, >Aqa'iac, >Hpe'irwn, Jr\aai
% % khc,
% % \edtext{<Rod'ophc}{\Dfootnote{\latintext \textit{Europe\LitNil}
% % Cod.\,Ver.}},
% % \edtext{\abb{>As'iac, Kar'iac, Bijun'iac, <Ellhsp'ontou, Frug'iac,
% % \edtext{Pisid'iac, Kappadok'iac, P'ontou, Kilik'iac, Frug'iac
% % >'allhc}{\lemma{Pisid'iac \dots\ >'allhc}\Dfootnote{Pisid'iac \dots\ Frug'iac
% % \latintext > Thdt.(r) || \greektext Kilik'iac Frug'iac >'allhc \latintext Thdt.(BT)
% % \greektext Frug'iac >'allhc Kilik'iac \latintext Thdt.(A) \greektext Kilik'iac
% % >'allhc \latintext Thdt.(z)}}, Pamful'iac, Lud'iac, n'hswn Kukl'adwn,
% % \edtext{\abb{A>ig'uptou}}{\Dfootnote{\latintext > Thdt.(T)}}, Jhba\Ad{i}doc,
% % Lib'uhc, Galat'iac,}}{\lemma{\abb{>ap'o te \dots\
% % >Arab'iac}}\Dfootnote{\latintext > Ath.}}]}{\lemma{>ap`o \dots\
% % >Arab'iac}\Dfootnote{\dt{delevimus}}}{\lemma{>As'iac
% % \dots\ Galat'iac} \Dfootnote{\latintext
% % > Cod.\,Ver.}} Palaist'inhc,
% >Arab'iac}}
to~ic
\edtext{pantaqo~u}{\Dfootnote{<apantaqo~u \dt{Ath. \textit{ubique}
Cod.\,Ver.}}} >episk'opoic ka`i sulleitourgo~ic t~hc
\edtext{\abb{kajolik~hc}}{\Dfootnote{+ ka`i >apostolik~hc \dt{Thdt. Cod.\,Ver.}}}
>ekklhs'iac,
\edtext{>agaphto~ic}{\Dfootnote{\dt{\textit{dilectissimis} Cod.\,Ver.}}}
>adelfo~ic
\edtext{\abb{>en \edtext{kur'iw|}{\Dfootnote{q\oline{w} \dt{Thdt.(B)}}}
qa'irein}}{\Dfootnote{\dt{> Cod.\,Ver.}}}.}{\lemma{\abb{<H <ag'ia s'unodoc \dots\ qa'irein}}\Dfootnote{\dt{> Hil.}}}
% \index[namen]{\latintext Asien}\index[namen]{\latintext Karien}\index[namen]{\latintext
% Bithynien}\index[namen]{\latintext Hellespont}\index[namen]{\latintext
% Phrygien}\index[namen]{\latintext Pisidien}\index[namen]{\latintext
% Kappadokien}\index[namen]{\latintext Pontos}\index[namen]{\latintext
% Kilikien}\index[namen]{\latintext Phrygien}\index[namen]{\latintext
% Pamphylien}\index[namen]{\latintext Lydien}\index[namen]{\latintext
% Kykladen}\index[namen]{\latintext "Agypten}\index[namen]{\latintext
% Thebais}\index[namen]{\latintext Libyen}\index[namen]{\latintext
% Galatien}\index[namen]{\latintext Pal"astina}\index[namen]{\latintext
% Arabien}\index[namen]{\latintext Epirus}\index[namen]{\latintext
% Kalabrien}\index[namen]{\latintext Gallien}\index[namen]{\latintext
% Rom}\index[namen]{\latintext Spanien}\index[namen]{\latintext
% Kampanien}\index[namen]{\latintext Pannonien}\index[namen]{\latintext
% Dakien}\index[namen]{\latintext Thessalien}\index[namen]{\latintext
% Moesien}\index[namen]{\latintext
% Dakien}\index[namen]{\latintext Dardanien}\index[namen]{\latintext
% Makedonien}\index[namen]{\latintext Achaia}\index[namen]{\latintext
% Thrakien}\index[namen]{\latintext Rhodope}\index[namen]{\latintext
% Afrika}\index[namen]{\latintext Italien}\index[namen]{\latintext
% Sardinien}\index[namen]{\latintext Serdica}
\pend
\pstart
\kap{2}poll`a m`en ka`i
\edtext{poll'akic}{\Dfootnote{\dt{\textit{sepius} Cod.\,Ver. \textit{frequenter} Hil.}}}
\edtext{>et'olmhsan o<i
\edtext{>Areioman~itai}{\Dfootnote{\dt{\textit{Arriani heretici} Hil.
\textit{Arriani} Cod.\,Ver.}}}}{\lemma{\abb{}}\Dfootnote{\responsio\ o<i
>Areioman~itai >et'olmhsan \dt{Thdt.(A)}}}\edindex[namen]{Arianer} kat`a t~wn do'ulwn to~u jeo~u
t~wn t`hn p'istin fulatt'ontwn
\edtext{\abb{t`hn >orj'hn}}{\Dfootnote{\dt{+ \textit{et catholicam} Hil.}}}.
\edtext{n'ojon}{\Dfootnote{n'ojhn \dt{Ath.(B\corr) e \greektext n'ojon \latintext Ath.(B*) \textit{adulterinam} Hil. Cod.\,Ver.}}}
\edtext{g`ar}{\Dfootnote{\dt{\textit{enim} Cod.\,Ver. > Hil.}}}
\edtext{<upob'allontec}{\Dfootnote{<upob'alontec \dt{Thdt.(A)
\textit{inserentes} Hil. \textit{submittentes} Cod.\,Ver.}}} didaskal'ian to`uc
\edtext{>orjod'oxouc}{\Dfootnote{\dt{\textit{ordoxos} Hil.}}}
>ela'unein >epeir'ajhsan;
\edtext{toso~uton}{\Dfootnote{t'osou \latintext Thdt.(B)}}
\edtext{d`e}{\Dfootnote{\latintext \textit{autem} Hil. \textit{enim} Cod.\,Ver.}}
\edtext{\abb{loip`on}}{\Dfootnote{\latintext > Cod.\,Ver. \textit{nunc} Hil.}}
\edtext{katepan'esthsan}{\Dfootnote{ka`i >epan'esthsan \latintext Thdt.(B)}}
kat`a t~hc p'istewc
\edtext{\abb{<wc}}{\Dfootnote{\latintext \textit{> Hil.}}}
\edtext{mhd`e}{\Dfootnote{\latintext \textit{non} Hil. \textit{etiam} \dots\
\textit{non} Cod.\,Ver.}} t`hn
\edtext{e>us'ebeian}{\Dfootnote{>ako`hn \latintext Thdt. \greektext e>ul'abeian
\latintext Ath.(E) \textit{religiosam pietatem} Hil. \textit{bonitatem}
Cod.\,Ver.}}
\edtext{\abb{t~wn}}{\lemma{\abb{}} \Dfootnote{+ \latintext \textit{in} ante \greektext t~wn \latintext Hil.}}
\edtext{\abb{e>usebest'atwn}}{\Dfootnote{\latintext Ath.(KORE) \greektext
e>ulabest'atwn \latintext Ath.(B) \greektext jeofilest'atwn \latintext Thdt.
\textit{clementissimorum} Hil. \textit{piissimorum} Cod.\,Ver.}} basil'ewn
\edtext{laje~in}{\Dfootnote{\latintext \textit{latere} Hil.}}. toigaro~un
\edtext{t~hc q'aritoc
\edtext{\abb{to~u}}{\Dfootnote{\latintext > Thdt.(T)}} jeo~u
sunergo'ushc}{\Dfootnote{\latintext ] \textit{gratia dei subveniente} Cod.\,Ver.
\textit{gratiam adiubentes} Hil}},
\edtext{\abb{ka`i}}{\Dfootnote{\latintext > Cod.\,Ver.}} a>uto`i o<i
\edtext{e>useb'estatoi}{\Dfootnote{jeofil'estatoi \latintext Thdt.(A)
\textit{clementissimi} Hil. \textit{piissimi} Cod.\,Ver.}} basile~ic sun'hgagon
\edtext{\abb{<hm~ac}}{\Dfootnote{\latintext > Hil.}}
\edtext{>ek diaf'orwn}{\Dfootnote{\latintext \textit{ex diversis} Cod.\,Ver.}} >ep\-arqi\-~wn
ka`i p'olewn
\edtext{ka`i}{\lemma{\abb{ka`i\ts{2}}}\Dfootnote{\latintext > Hil.}}
t`hn <ag'ian
\edtext{ta'uthn}{\Dfootnote{\latintext \textit{hanc} Cod.\,Ver. \textit{ipsam}
Hil.}} s'unodon
\edtext{>ep`i t`hn}{\Dfootnote{>en t~h| \latintext Ath. \textit{in} Hil.
\textit{aput} Cod.\,Ver.}}
\edtext{\abb{Serd~wn}}{\Dfootnote{\latintext Ath.(BRE) \greektext Sard~wn
\latintext Ath.(KO) \greektext Sard'ewn \latintext Thdt.(AnF*) \greektext
S'ardewn \latintext Thdt.(SF\corr) \greektext Sarda'iwn \latintext Thdt.(GT)
\greektext Sardik`hn \latintext Thdt.(L) \greektext Sardikhs'iwn \latintext
Thdt.(B) \textit{Serdicensium} Hil. \textit{Sardicam} Cod.\,Ver.}}\edindex[synoden]{Serdica!a. 343}
\edtext{p'olin}{\Dfootnote{p'olei \latintext Ath. \textit{civitatem} Hil.
\latintext > Cod.\,Ver.}}
\edtext{gen'esjai}{\Dfootnote{gegen~hsjai \latintext Thdt.(r)}} ded'wkasin,
<'ina p~asa
\edtext{m`en}{\Dfootnote{\latintext \textit{quidem} Hil. > Cod.\,Ver.}} diq'onoia
\edtext{periairej~h|}{\Dfootnote{diairej~h \latintext Thdt.(T)
\textit{amputaretur} Hil. \textit{pertollatur} Cod.\,Ver.}},
\edtext{p'ashc d`e kakopist'iac
\edtext{>exelaje'ishc}{\Dfootnote{>exelasje'ishc \latintext Thdt.(B)
Ath.(RE)}}}{\Dfootnote{\latintext \textit{tota vero mala fide expulsa} Cod.\,Ver.
\textit{inqu\oline{a} autem doctrina penitus expulsa} Hil.}}
\edtext{<h e>ic
\edtext{Qrist`on}{\Dfootnote{t`on Qrist`on \latintext Thdt.}} e>us'ebeia
m'onh}{\Dfootnote{\latintext \textit{pietas sola, quae est in Christo} Hil.
\textit{in \oline{Cro} religio sola} Cod.\,Ver.}}
\edtext{par`a p~asi}{\Dfootnote{par`a p'antwn \latintext Ath. \textit{hominibus}
Hil. \textit{aput omnes} Cod.\,Ver.}}
\edtext{ful'atthtai}{\Dfootnote{ful'attetai \latintext Thdt.(B)}}.
\pend
\pstart
\kap{3}>~hljon
\edtext{g`ar ka`i o<i}{\Dfootnote{ka`i o<i \latintext > Thdt. \textit{enim} Hil.
\textit{enim et} Cod.\,Ver.}}
\edtext{>ap`o t~hc <e'w|ac >ep'iskopoi}{\lemma{\abb{}}\Dfootnote{\responsio\
>ep'iskopoi >ap`o t~hc <e'w|ac \latintext Thdt.(A)}}
\edtext{protrap'entec}{\Dfootnote{\latintext \textit{cohortatione excitati} Hil.
\textit{cohortati} Cod.\,Ver.}} \edtext{\abb{ka`i a>uto`i}}{\Dfootnote{\latintext
> Hil.}}
\edtext{par`a t~wn e>usebest'atwn basil'ewn}{\Dfootnote{\latintext
\textit{piissimorum imperatorum} Hil. \textit{a religiosissimis imperatoribus}
Cod.\,Ver.}}, m'alista
\edtext{\abb{di'' <'aper}}{\Dfootnote{\latintext Ath. \greektext di'' <'oper
\latintext Thdt.(BrzT) \greektext d'' <'oper \latintext Thdt.(A) \textit{propter
id, quod} Hil. \textit{ob ea, quae} Cod.\,Ver.}}
\edtext{>ejr'uloun poll'akic}{\Dfootnote{\latintext \textit{frequenter in
murmuratione fuit, id est} Hil. \textit{rumigerabantur frequenter} Cod.\,Ver.}}
\edtext{\abb{per`i}}{\Dfootnote{\latintext > Hil. Cod.\,Ver.}} t~wn
\edtext{>agapht~wn}{\Dfootnote{\latintext \textit{dilectissimis} Cod.\,Ver. Hil.}}
\edtext{>adelf~wn
\edtext{\abb{<hm~wn}}{\Dfootnote{\latintext > Ath.(E)}} ka`i
\edtext{sulleitourg~wn}{\Dfootnote{\latintext \textit{conministris} Cod.\,Ver. \textit{coepiscopis} Hil.}}}{\lemma{>adelf~wn <hm~wn ka`i sulleitourg~wn}\Dfootnote{<hm~wn >adelf~wn ka`i
sulleitourg~wn \latintext Ath.(K)
\greektext >adelf~wn ka`i sulleitourg~wn <hm~wn \latintext Thdt.(Arz)}}
>Ajanas'iou\edindex[namen]{Athanasius!Bischof von Alexandrien}
\edtext{>episk'opou}{\Dfootnote{to~u >episk'opou \latintext Thdt.(BArz)}}
\edtext{\abb{t~hc}}{\Dfootnote{\latintext > Thdt.(BT)}} >Alexandre'iac ka`i
Mark'ellou\index[namen]{Markell!Bischof von Ancyra}
\edtext{>episk'opou}{\Dfootnote{to~u >episk'opou \latintext Thdt. (Arz)}} t~hc
\edtext{>Agkurogalat'iac}{\Dfootnote{\latintext \textit{Ancyrogalatiae} Hil.
\textit{Ancyrae Galatiae} Cod.\,Ver. + \greektext ka`i
>Asklhp~a to~u G'azhc \latintext Thdt. + \textit{et Asclepa} Cod.\,Ver.}}.
\edtext{>'iswc g`ar ka`i}{\Dfootnote{\latintext \textit{arbitramus autem etiam}
Hil. \textit{forte enim} Cod.\,Ver.}} e>ic
\edtext{\abb{<um~ac}}{\Dfootnote{+ a>uto`uc \latintext Thdt. Hil.}}
\edtext{>'efjasan}{\Dfootnote{\latintext \textit{pervenisse} Hil.
\textit{pervenerunt} Cod.\,Ver.}}
\edtext{a>ut~wn a<i diabola'i}{\Dfootnote{a>ut~wn \latintext > Thdt.
\textit{eorum calumnias} Hil. \textit{de his accusationes} Cod.\,Ver.}},
\edtext{>'iswc ka`i}{\Dfootnote{<wc ka`i \latintext Thdt. \textit{forte et}
Cod.\,Ver. \textit{et}\dots\ \textit{non dubitavimus} Hil.}} t`ac
\edtext{<umet'erac}{\Dfootnote{<hmet'erac \latintext Thdt.(BrzT)}} >ako`ac
\edtext{>epeqe'irhsan}{\Dfootnote{\latintext \textit{temtasse eos} Hil.
\textit{temptaverunt} Cod.\,Ver.}} parasale~usai,
\edtext{<'ina
\edtext{kat`a m`en}{\Dfootnote{t`a m`en kat`a \latintext Thdt.(T)}} t~wn >aj'wwn
<`a l'egousi \edtext{piste'ushte}{\Dfootnote{\latintext \textit{credatis}
Hil. \greektext piste'ushtai \latintext Ath.(R) \greektext piste'uhtai \latintext Thdt. Cod.\,Ver.}}}{\Dfootnote{\latintext \textit{ut ea,
quae adversus innocentes dicunt
maledici, credantur} Cod.\,Ver.}},
\edtext{\abb{t`hn}}{\Dfootnote{\dt{Ath.(O\corr)} t~hc \dt{Ath.(O*)}}}
\edtext{d`e}{\Dfootnote{\latintext \textit{haec} Cod.\,Ver.}} t~hc
\edtext{moqjhr~ac}{\Dfootnote{\latintext \textit{sceleratae} Hil.
\textit{molestissimae} Cod.\,Ver.}}
\edtext{a>ut~wn
\edtext{a<ir'esewc}{\Dfootnote{\latintext \textit{hereses} Hil.}}}{\lemma{\abb{}}\Dfootnote{\responsio\ a<ir'esewc
a>ut~wn \latintext  Thdt.(rzT)}}
\edtext{<up'onoian
\edtext{>epikr'uywsin}{\Dfootnote{<upokr'uywsin \latintext Ath.(E*), \greektext >epikr'uywsin \latintext in mg. Ath.(E\corr)}}}{\Dfootnote{<upokr'uywsin \latintext Thdt.(n)}}. >all'' o>uk
\edtext{>ep`i pol`u}{\Dfootnote{\latintext \textit{diutius} Hil. \textit{diu}
Cod.\,Ver.}} ta~uta
\edtext{poie~in}{\Dfootnote{\latintext \textit{facere} Hil. \textit{perpetrare}
Cod.\,Ver.}}
\edtext{suneqwr'hjhsan}{\Dfootnote{\latintext \textit{illis} \dots\
\textit{permissum est} Hil. \textit{permissi sunt} Cod.\,Ver.}}. >'esti g`ar
\edtext{<o}{\lemma{\abb{<o\ts{1}}}\Dfootnote{\latintext > Thdt.(A)}}
\edtext{proist'amenoc t~wn >ekklhsi~wn}{\Dfootnote{\latintext \textit{gubernator
ecclesiarum} Hil. \textit{qui prestat patrocinium ecclesiis} Cod.\,Ver.}}
\edtext{k'urioc}{\Dfootnote{<o k'urioc \latintext Thdt.(A)}}, <o
\edtext{<up`er
\edtext{\abb{to'utwn ka`i}}{\Dfootnote{\latintext Ath.(O\corr) > Ath.(O*)}}}{\lemma{<up`er to'utwn ka`i}\Dfootnote{\latintext \textit{pro ecclesiis et} Hil.
\textit{pro his et} Cod.\,Ver.}}
\edtext{<up`er}{\lemma{\abb{<up`er\ts{2}}}\Dfootnote{\latintext > Ath. Cod.\,Ver.}} p'antwn <hm~wn j'anaton
<upome'inac ka`i
\edtext{\abb{di'' <eauto~u}}{\Dfootnote{\latintext Ath. \greektext d'' a>uto~u
\latintext Thdt.(A*S) \greektext di'' a>ut`ac \latintext Thdt.(BnGzTA\corr)
\textit{propter eas ecclesias} Hil. \textit{propterea} Cod.\,Ver.(\textit{propter eas} coni. Ballerini) \greektext
di'' a>ut~wn \latintext susp. J�licher}} t`hn e>ic
\edtext{o>urano`uc}{\Dfootnote{o>uran`on \latintext Ath. Thdt.(A)}} >'anodon p~asin <hm~in
dedwk'wc.
\pend
\pstart
\kap{4}p'alai \edtext{m`en o~>un}{\Dfootnote{m`en
\dt{Thdt.(F) \textit{quidem} Hil. \textit{itaque} Cod.\,Ver.}}}
\edtext{gray'antwn}{\Dfootnote{>'egrayan \dt{Thdt. \textit{scribentibus} Hil. Cod.\,Ver.} +
t~wn per`i \dt{Ath.} + o<i per`i \dt{Thdt.}}}
\edtext{\abb{E>useb'iou}}{\Dfootnote{\dt{coni. Erl.} E>us'ebion \dt{Ath. Thdt. \textit{Eusevio} Hil. \textit{Eusebio} Cod.\,Ver.}}}\edindex[namen]{Eusebius!Bischof von Nikomedien} \edtext{ka`i
\edtext{\abb{M'ari}}{\Dfootnote{\dt{coni. Erl.} M'arin
\dt{Thdt.(-B)} mak'arin \dt{Thdt.(B) \textit{Mario} Hil. \textit{Mari}
Cod.\,Ver.}}}
\edtext{ka`i}{\lemma{\abb{ka`i\ts{2}}}\Dfootnote{\latintext > Hil. Cod.\,Ver.}}
\edtext{\abb{Jeod'wrou}}{\Dfootnote{\dt{coni. Erl.}
Je'odwron \dt{Thdt. \textit{Theodoro} (pr. \textit{o} ex \textit{u}, tert. \textit{o} in ras.) Hil.
\textit{Theodoro} Cod.\,Ver.}}}
\edtext{ka`i}{\lemma{\abb{ka`i\ts{3}}}\Dfootnote{\latintext > Hil. Cod.\,Ver.}}
\edtext{\abb{Jeogn'iou}}{\Dfootnote{\dt{coni. Erl.} Je'ognion \dt{(ras. inter} g \dt{et} n\dt{)} \dt{Thdt(A)} Jeog'onion \dt{Thdt.(-A)}
\dt{\textit{Diognito} Hil. \textit{Theo} Cod.\,Ver.}}}
\edtext{ka`i}{\lemma{\abb{ka`i\ts{4}}}\Dfootnote{\latintext > Hil. Cod.\,Ver.}}
\edtext{\abb{O>ursak'iou}}{\Dfootnote{\dt{coni. Erl.} O>urs'akion \dt{Thdt.(-n)} >Ars'akion
\dt{Thdt.(n) \textit{Ursatio} Hil. \textit{Ursacio} Cod.\,Ver.}}}
\edtext{ka`i}{\lemma{\abb{ka`i\ts{1}}}\Dfootnote{\latintext > Cod.\,Ver.}}
\edtext{\abb{O>u'alentoc}}{\Dfootnote{\dt{coni. Erl.} O>u'alenta \dt{Thdt. \textit{Valente} Hil.
Cod.\,Ver.}}}
\edtext{\abb{\edtext{ka`i}{\lemma{\abb{ka`i\ts{2}}}\Dfootnote{\latintext > Hil. Cod.\,Ver.}}
\edtext{Mhnof'antou}{\Dfootnote{Mhn'ofanton \dt{Thdt.(-B)}
N'efanton \dt{Thdt.(B) \textit{Minophanto} Cod.\,Ver.}}} ka`i
\edtext{Stef'anou}{\Dfootnote{St'efanon \dt{Thdt. \textit{Stephano}
Cod.\,Ver.}}}}}{\Dfootnote{\dt{> Hil.}}}}{\lemma{\abb{ka`i\ts{1} \dots\
Stef'anou}}\Dfootnote{\dt{> Ath.}}}\edindex[namen]{Theodorus!Bischof von Heraclea}\edindex[namen]{Theognis!Bischof
von Nicaea}\edindex[namen]{Maris!Bischof von Chalcedon}\edindex[namen]{Valens!Bischof von Mursa}\edindex[namen]{Ursacius!Bischof von Singidunum}\edindex[namen]{Menophantus!Bischof von Ephesus}\edindex[namen]{Stephanus!Bischof von Antiochien}
>Ioul'iw|\edindex[namen]{Julius!Bischof von Rom} t~w|
\edtext{sulleitourg~w|}{\Dfootnote{\latintext \textit{consacerdoti} Cod.\,Ver. Hil.}}
\edtext{\abb{<hm~wn}}{\Dfootnote{\latintext > Cod.\,Ver.}}
\edtext{\abb{t~w|}}{\Dfootnote{\latintext > Thdt.}}
\edtext{t~hc
\edtext{<Rwma'iwn}{\Dfootnote{\latintext \textit{Romanae} Cod.\,Ver. Hil.}}
\edtext{\abb{>ekklhs'iac}}{\Dfootnote{\latintext > Ath.(E) in textu, sed in mg.
(E\corr)}}
\edtext{>episk'opw|}{\Dfootnote{>episk'opou \latintext Thdt.(T)}} kat`a t~wn
proeirhm'enwn
\edtext{sulleitourg~wn}{\Dfootnote{\latintext \textit{conministros} Cod.\,Ver. \textit{coepiscopos} Hil.}} <hm~wn}{\lemma{\abb{t~hc \dots\ <hm~wn}}
\Dfootnote{\latintext > Thdt.(L)}} --
\edtext{l'egomen}{\Dfootnote{l'egw \latintext Thdt.(nG) \textit{dicimus} Hil.
\textit{hoc est} Cod.\,Ver.}}
\edtext{\abb{d`h}}{\Dfootnote{\latintext Thdt.(rzT) Ath. \greektext d`e
\latintext Thdt.(BA) \textit{autem} Hil. > Cod.\,Ver.}}
\edtext{\abb{>Ajanas'iou}}{\Dfootnote{+ to~u (to~u \latintext > Thdt.(rz)) \greektext
>episk'opou
>Alexandre'iac \latintext Thdt. Cod.\,Ver.}}\edindex[namen]{Athanasius!Bischof von Alexandrien} ka`i
\edtext{\abb{Mark'ellou}}{\Dfootnote{+ ka`i >Asklhp~a \latintext Ath. \greektext + to~u
(to~u \latintext > Thdt.(Bn)) \greektext >episk'opou
>Agkurogalat'iac (>Agk'urac Galat'iac \latintext Thdt.(A)) \greektext ka`i >Asklhp~a to~u
G'azhc \latintext Thdt.
+ \textit{episcopum Ancyrae Galatiae nec non etiam Asclepam episcopum Gazae}
Cod.\,Ver.}}\edindex[namen]{Markell!Bischof von Ancyra} --
\edtext{\edtext{\abb{>'egrayan}}{\Dfootnote{+ d`e \latintext Thdt.}}
ka`i}{\Dfootnote{\latintext ]\textit{ut scripserunt} Hil.}} o<i >ap`o t~wn
>'allwn
\edtext{mer~wn}{\Dfootnote{\latintext \textit{locis} Hil. \textit{partibus}
Cod.\,Ver.}}
\edtext{\abb{>ep'iskopoi}}{\Dfootnote{\latintext > Hil.}},
\edtext{\edtext{marturo~untec m`en}{\Dfootnote{\latintext \textit{adhibentes
quidem testimonium} Cod.\,Ver.}}
\edtext{\abb{>ep`i}}{\Dfootnote{\latintext > Ath. Cod.\,Ver.}} t~h| kajar'othti
\edtext{to~u sulleitourgo~u}{\Dfootnote{\latintext \textit{conministri} Cod.\,Ver.
\textit{episcopi} Hil.}} <hm~wn >Ajanas'iou}{\Dfootnote{\latintext ] \textit{ut
episcopi nostri} Hil.}}\edindex[namen]{Athanasius!Bischof von Alexandrien}, t`a
d`e
\edtext{par`a t~wn per`i E>us'ebion}{\Dfootnote{\latintext \textit{ab Eusebio}
Hil. \textit{Eusebii et eius sociorum} Cod.\,Ver.}}\edindex[namen]{Eusebianer} gen'omena
\edtext{\abb{o>ud`en}}{\Dfootnote{\latintext Thdt. \greektext mhd`en \latintext
Ath.}} <'eteron >`h
\edtext{yeud~h}{\Dfootnote{ye'udh \latintext Thdt.}} ka`i
\edtext{sukofant'iac}{\Dfootnote{\latintext \textit{falsitatis emendatiis}
Hil. \textit{mendacium et calumniis} Cod.\,Ver.}}
\edtext{\abb{e>~inai}}{\Dfootnote{\latintext > Cod.\,Ver. \textit{fuisse} Hil.}}
\edtext{mest'a}{\Dfootnote{\latintext \textit{plen\oline{a}} Hil.}}.
\edtext{\abb{ka`i e>i ka`i t`a m'alista >ek to~u}}{\Dfootnote{e>i ka`i t`a (t'a \latintext Thdt.(H\slin)) \greektext m'alista >ek to~u \latintext Thdt. \textit{et licet ex eo quod} Cod.\,Ver. \textit{nam si ex eo quo} Hil.}}
\edtext{\edtext{klhj'entac}{\Dfootnote{klhj'enta \latintext Thdt.(T)}}
\edtext{\abb{a>uto`uc}}{\Dfootnote{+ to'utouc \latintext Thdt.(A)}}
par`a to~u >agaphto~u
\edtext{\abb{<hm~wn}}{\Dfootnote{\latintext > Thdt.(BArz) \greektext \responsio\ <hm~wn \latintext post \greektext sulleitourgo~u \latintext Thdt.(T)}} ka`i sulleitourgo~u >Ioul'iou
\edtext{m`h >apant~hsai ka`i >ek t~wn graf'entwn par`a to~u
\edtext{\abb{a>uto~u}}{\Dfootnote{+ >episk'opou \latintext Thdt.}}
>Ioul'iou}{\lemma{\abb{m`h \dots\ >Ioul'iou}} \Dfootnote{\latintext > Ath.(B)}}
faner`a to'utwn <h sukofant'ia
\edtext{p'efhnen}{\Dfootnote{g'egonen \latintext Thdt.(B) \greektext p'efhnen
\latintext Thdt.(B\mg)}}}{\lemma{\abb{klhj'entac \dots\ p'efhnen}}
\Dfootnote{\latintext ] \textit{vocati sunt a Iulio episcopo, carissimo fratre nostro, claruit noluerunt noluisse eos venire ex ipsis littereris, quibus
eorum mendacia detecta sunt} Hil. \textit{vocati sunt a dilecto
nostro et consacerdote Iulio, noluerunt occurrere et ex his, quae scripta sunt
ab eodem Iulio, manifestat sit ipsorum calumnia} Cod.\,Ver.}}\edindex[namen]{Julius!Bischof von Rom}
-- >~hljon g`ar
\edtext{\abb{>`an}}{\Dfootnote{\latintext > Thdt.(n)}} e>'iper
\edtext{>ej'arroun}{\Dfootnote{\latintext \textit{habuissent fiduciam}  Hil.
\textit{confidissent} Cod.\,Ver.}}
o<~ic >'epraxan
\edtext{\abb{\edtext{\abb{ka`i}}{\Dfootnote{+ o~<ic \latintext Thdt.(H)}}
pepoi'hkasi}}{\Dfootnote{\latintext > Thdt.(B) Hil.}}
\edtext{kat`a t~wn
\edtext{sulleitourg~wn <hm~wn}{\Dfootnote{\latintext \textit{coepiscopos
nostros} Hil. \textit{fratres nostros et conministros} Cod.\,Ver.}} -- <'omwc
\edtext{\abb{ka`i}}{\Dfootnote{\latintext > Thdt.(B) Cod.\,Ver.}} >ex <~wn
pepoi'hkasin}{\lemma{\abb{kat`a \dots\ pepoi'hkasin}} \Dfootnote{\latintext >
Thdt.(n)}} >en
\edtext{ta'uth|}{\Dfootnote{a>ut~h \latintext Thdt.(nGF)}} t~h| <ag'ia|
\edtext{\abb{ka`i}}{\Dfootnote{\latintext > Thdt.(s)}} meg'alh| sun'odw|,
fanerwt'eran
\edtext{t`hn <eaut~wn suskeu`hn}{\Dfootnote{\latintext \textit{falsitatis suae
commenta} Hil. \textit{suam factionem} Cod.\,Ver.}} >ap'edeixan.
\pend
\pstart
\kap{5}>apant'hsantec
\edtext{g`ar}{\Dfootnote{\latintext \textit{etenim} Cod.\,Ver. \textit{enim et} Hil.}}
\edtext{e>ic}{\Dfootnote{\latintext \textit{ad} Hil. \textit{aput} Cod.\,Ver.}}
t`hn
\edtext{\abb{Serd~wn}}{\Dfootnote{\latintext Ath.(BORE) \greektext Sard~wn
\latintext Ath.(K) \greektext Sard'ewn \latintext Thdt.(ArzT) \greektext
S'ardewn \latintext Thdt.(H) \greektext Sardikhs'iwn \latintext Thdt.(B)
\textit{Sardicensium}  Hil. \textit{Sardicam} Cod.\,Ver.}}\edindex[synoden]{Serdica!a. 343}
\edtext{\abb{p'olin}}{\Dfootnote{\latintext > Cod.\,Ver.}},
\edtext{\abb{>id'ontec}}{\Dfootnote{\latintext Thdt.(NszT) Ath. \greektext
e>id'ontec \latintext Thdt.(B) \greektext e>id'otec \latintext Thdt.(A)
\greektext >id'ontec te \latintext Thdt.(H) \textit{et videntes} Hil. \textit{ac
videntes} Cod.\,Ver.}}
\edtext{\abb{to`uc >adelfo`uc
\edtext{\abb{<hm~wn}}{\Dfootnote{+ Pa~ulon ka`i \latintext
Thdt.(T*)}}}}{\Dfootnote{\latintext > Hil.}} >Ajan'asion\edindex[namen]{Athanasius!Bischof von Alexandrien}
\edtext{ka`i}{\lemma{\abb{ka`i\ts{1}}}\Dfootnote{\latintext > Cod.\,Ver.}} M'arkellon\edindex[namen]{Markell!Bischof von Ancyra}
\edtext{ka`i}{\lemma{\abb{ka`i\ts{2}}}\Dfootnote{\latintext > Hil.}}
\edtext{>Asklhp~an}{\Dfootnote{\latintext \textit{Asclepium} Cod.\,Ver.
Hil.}}\edindex[namen]{Asclepas!Bischof von Gaza}
\edtext{\abb{ka`i to`uc >'allouc}}{\Dfootnote{\latintext > Cod.\,Ver.}}
>efob'hjhsan e>ic kr'isin >elje~in. ka`i
\edtext{o>uq}{\Dfootnote{\latintext \textit{non} Hil. \textit{non solum}
Cod.\,Ver.}} <'apax
\edtext{o>ud`e de'uteron}{\Dfootnote{\latintext \textit{neque iterum} Hil.
\textit{et bis} Cod.\,Ver.}} >all`a
\edtext{ka`i}{\lemma{\abb{ka`i\ts{2}}}\Dfootnote{\latintext > Thdt.(T) Hil. Cod.\,Ver.}}
\edtext{poll'akic}{\Dfootnote{\latintext \textit{sepius} Cod.\,Ver.
\textit{frequenter} Hil.}} klhj'entec
\edtext{o>uk >ep'hkousan}{\Dfootnote{o>uq <up'hkousan \latintext Thdt.(BArLT)
\textit{contempserunt} Hil. \textit{non responderunt} Cod.\,Ver.}}
\edtext{ta~ic kl'hsesi}{\Dfootnote{\latintext \textit{invitationem} Hil.
\textit{vocationibus} Cod.\,Ver.}}, ka'itoi p'antwn
\edtext{\edtext{<hm~wn}{\Dfootnote{t~wn \latintext Thdt.(Br) \responsio\
\greektext t~wn \latintext post \greektext sunelj'ontwn \latintext Thdt.(T) >
Thdt.(Az) \greektext <hm~wn t~wn \latintext coni. Schwartz}} sunelj'ontwn
>episk'opwn}{\Dfootnote{\latintext \textit{synodi omnium nostrorum, qui
convenimus episcopi} Hil. \textit{videlicet episcopis conventibus} Cod.\,Ver.}},
\edtext{\abb{ka`i}}{\Dfootnote{\latintext > Thdt.}} m'alista
\edtext{to~u e>ughrot'atou <Os'iou to~u}{\Dfootnote{to~u sulleitourgo~u
>Ajanas'iou to~u \latintext Thdt.(B) \textit{venerabilis senectae Ossium, qui}
+ in mg. \textit{Ossius e\oline{ps} prefuit synhodo Serdicensi} Hil. \textit{bonae
senectutis Osio, qui} Cod.\,Ver.}}\edindex[namen]{Ossius!Bischof von Cordoba}
\edtext{\abb{ka`i}}{\Dfootnote{\dt{Thdt(A\corr) > Cod.\,Ver.}}}
\edtext{di`a t`on qr'onon ka`i
\edtext{t`hn}{\Dfootnote{di`a t`hn \latintext Thdt.}}
<omolog'ian ka`i }{\lemma{\abb{di`a t`on \dots\ <omolog'ian}}
\Dfootnote{\latintext > Thdt.(s)}}
\edtext{di`a t`o toso~uton k'amaton <upomemenhk'enai}{\Dfootnote{\latintext
\textit{tanti temporis probatam fidem, qui tantum laborem id aetatis pro
ecclesiae utilitati sustinuit} Hil. \textit{tantum laborem} Cod.\,Ver.}}
\edtext{\abb{p'ashc a>ido~uc}}{\Dfootnote{\latintext Ath. \greektext p'ashc
tim~hc te ka`i a>ido~uc \latintext Thdt.(A) \greektext p'ashc tim~hc >a"id'iou
\latintext Thdt.(BzT) \greektext p'ashc tim~hc \latintext Thdt.(r)  \textit{omni
reverentia} Hil. \textit{ex omnem reverentia} Cod.\,Ver.}}
\edtext{\abb{>ax'iou tugq'anontoc}}{\lemma{\abb{}} \Dfootnote{\responsio\ tugq'anontoc >ax'iou
\latintext  Thdt. \textit{ut} \dots\ \textit{dignissimus habeatur} Hil.
\textit{dignus videtur} Cod.\,Ver.}}
\edtext{>anamen'ontwn}{\Dfootnote{\latintext Hil. interpunxit et continuit:
\textit{expectantibus omnibus}; + \textit{et} ante \greektext >anamen'ontwn
\latintext Cod.\,Ver.}} ka`i protrepom'enwn a>uto`uc
\edtext{e>iselje~in
\edtext{\abb{e>ic  t`hn kr'isin}}{\Dfootnote{\latintext > Cod.\,Ver.}}}{\Dfootnote{]
e>ic kr'isin >elje~in \latintext Thdt.(n)}},
\edtext{<'in''}{\Dfootnote{<'ina \latintext Thdt. \textit{ut} Hil. \textit{qo} (\textit{o} s.l.)
Cod.\,Ver.}}
\edtext{\abb{<'aper}}{\Dfootnote{\latintext Thdt.(A\mg) \textit{omnia, quae}
Hil. \textit{ea, quae} Cod.\,Ver.}}
\edtext{\edtext{>ap'ontwn}{\Dfootnote{>ap'ontwn kat`a \latintext Thdt.(F)
\greektext kata p'antwn (kat \latintext et \greektext t \latintext s.l.) Thdt.(T) \responsio\ \greektext kat`a \latintext post
\greektext >ap'ontwn t~wn \latintext Thdt.(A\slin)}} t~wn sulleitourg~wn
\edtext{\abb{<hm~wn}}{\Dfootnote{\latintext > Thdt. Cod.\,Ver.}} >ejr'ulhsan ka`i
>'egrayan kat'' a>ut~wn}{\lemma{>ap'ontwn \dots\ kat'' a>ut~wn}
\Dfootnote{\latintext \textit{de consacerdotibus nostris vel dixerunt vel
scripserunt} Hil. \textit{de absentibus conministris rumigerunt et scripserunt}
Cod.\,Ver.}},
\edtext{ta~uta}{\Dfootnote{p'anta \latintext Thdt.(F) > Hil. \greektext +
o<~utoi \latintext coni. Scheidweiler}}
\edtext{par'ontec}{\Dfootnote{par'ontwn \latintext Thdt.(rzT)}}
\edtext{>el'egxai dunhj~wsin}{\Dfootnote{\latintext \textit{possent convincere}
Hil. \textit{convincant} Cod.\,Ver.}}.
\edtext{\edtext{\abb{>all''}}{\Dfootnote{\latintext > Hil.}} o>uk ~>hljon
klhj'entec}{\lemma{\abb{}}\Dfootnote{\responsio\ >all`a klhj'entec o>uk >~hljon
\latintext  Thdt.(z)}},
\edtext{kaj`wc}{\Dfootnote{kaj'aper \latintext Ath.}}
\edtext{proe'ipomen}{\Dfootnote{proe'ipamen \latintext Thdt.(F) \textit{diximus}
Cod.\,Ver. \textit{praediximus} Hil.}},
\edtext{\abb{deikn'untec}}{\lemma{\abb{}} \Dfootnote{\latintext + \textit{sed} ante \greektext deikn'untec
\latintext  Cod.\,Ver.}}
\edtext{ka`i >ek to'utwn}{\Dfootnote{k>an to'utw \latintext Thdt.(T)
\textit{etiam ex his} Hil. \textit{ex hoc} Cod.\,Ver.}} t`hn sukofant'ian
\edtext{a>ut~wn}{\Dfootnote{<eaut~wn \latintext Ath.}} ka`i
\edtext {m'onon o>uq`i}{\Dfootnote{\latintext \textit{non solum} Hil. Cod.\,Ver.}}
\edtext{t`hn >epiboul`hn
\edtext{\abb{ka`i}}{\Dfootnote{\latintext > Thdt.(s), \greektext ka`i t`hn \latintext Thdt.(BAszT)}}
suskeu`hn}{\Dfootnote{] \latintext \textit{commentatitiam fraudem aut exquisitam versutiam} Hil.
\textit{insidias et factiones} Cod.\,Ver. \greektext \responsio\ t`hn suskeu`hn ka`i >epiboul`hn \latintext
Thdt.(n)}}
\edtext{\abb{<`hn}}{\Dfootnote{\latintext > Cod.\,Ver.}} pepoi'hkasi
\edtext{bo~wntec}{\Dfootnote{\latintext \textit{prudentes} Hil.
\textit{clamantes} Cod.\,Ver.}} di`a t~hc parait'hsewc. o<i g`ar jarro~untec
\edtext{o~<ic l'egousi}{\Dfootnote{\latintext \textit{suis dictis} Cod.\,Ver.
\textit{probare ea, quae absentes dicunt} Hil.}},
\edtext{to'utoic ka`i e>ic pr'oswpon sust~hnai d'unantai}{\lemma{to'utoic \dots\
d'unantai} \Dfootnote{\latintext \textit{hii et ad in praesentia esse potuerunt}
Cod.Ver \textit{hoc praesentes parati sunt convincere} Hil.}}.
\edtext{>epeid`h}{\Dfootnote{>epe'i \latintext Thdt.(B)}}
\edtext{d`e}{\Dfootnote{\latintext \textit{sed} Hil. \textit{igitur} Cod.\,Ver.}}
\edtext{o>uk >ap'hnthsan}{\Dfootnote{\latintext \textit{non occurrerunt}
Cod.\,Ver. \textit{venire non ausi sunt} Hil.}}, nom'izomen
\edtext{loip`on}{\Dfootnote{\latintext \textit{de cetero} Hil. \textit{iam}
Cod.\,Ver.}}
\edtext{mhd'ena}{\Dfootnote{\latintext \textit{nominem} + im mg. \textit{multas
persecutiones pertulisse e\oline{p}os catholicos vel \oline{sc}m athanasi\oline{u} ab Arrianus refert}
Hil. \textit{nullum} Cod.\,Ver.}} >agnoe~in, k>`an
\edtext{>eke~inoi}{\Dfootnote{\latintext \textit{illi} Hil. \textit{illi} Cod.\,Ver.* \textit{ibi} Cod.\,Ver.\corr  \greektext >eke~i \latintext Thdt.(BGF) \greektext e>ik~h \latintext
Thdt.(SL)}}
\edtext{\abb{p'alin}}{\Dfootnote{\latintext > Thdt.(z)}}
\edtext{kakourge~in
\edtext{>ejel'hswsin}{\Dfootnote{>ej'elousin \latintext Thdt.(A)  Ath.(E*)
\greektext >ej'elwsin \latintext Ath.(E\corr)}}}{\Dfootnote{\latintext \textit{malitiam
suam exercere voluerint} Hil. \textit{voluerint astute agere} Cod.\,Ver.}},
\edtext{\abb{<'oti}}{\Dfootnote{\latintext > Hil.}} mhd`en
\edtext{>'eqontec}{\Dfootnote{\latintext \textit{sufficientes} Hil.
\textit{habentes} Cod.\,Ver.}} kat`a t~wn
\edtext{sulleitourg~wn}{\Dfootnote{\latintext \textit{conministros} Cod.\,Ver. \textit{consacerdotes} Hil.}}
\edtext{<hm~wn}{\Dfootnote{\latintext \textit{nostros} Cod.\,Ver. \textit{nostro}
Hil.}}
\edtext{>el'egxai}{\Dfootnote{\latintext \textit{probare} Hil. \textit{arguere}
Cod.\,Ver.}}
\edtext{to'utouc}{\Dfootnote{\latintext \textit{quos} Hil. \textit{hos}
Cod.\,Ver.}}
\edtext{\abb{m`en}}{\Dfootnote{\latintext > Cod.\,Ver.}}
\edtext{diab'allousin}{\Dfootnote{\latintext \textit{incusant} Hil.
\textit{accusant} Cod.\,Ver.}} >ap'ontac, par'ontac d`e
\edtext{\abb{fe'ugousin}}{\Dfootnote{\latintext Ath. \textit{fugiunt} Hil.
\greektext diafe'ugousin \latintext Thdt. \textit{subterfugiunt} Cod.\,Ver.}}.
\pend
\pstart
\skipnumbering
\pend
\pstart
\kap{6}\edtext{>'efugon}{\Dfootnote{\latintext \textit{fugierunt} Hil. Cod.\,Ver.}} g'ar,
\edtext{>agaphto`i}{\Dfootnote{\latintext \textit{dilectissimi} Hil. Cod.\,Ver.}}
>adelfo'i, o>u m'onon
\edtext{di`a
\edtext{\abb{t`hn}}{\Dfootnote{+ kat`a \latintext Thdt.(BAzT)}} to'utwn
sukofant'ian}{\Dfootnote{\latintext ] \textit{propter hos, quos falso
adpetiverunt} Hil. \textit{propter calumniam, quam illis ingerebant}
Cod.\,Ver.}},
\edtext{\abb{>all'' <'oti ka`i}}{\lemma{\abb{}} \Dfootnote{\responsio\ >all' ka`i <'oti \latintext
Thdt.(n) \textit{verum etiam} Hil. \textit{sed etiam propter} Cod.\,Ver.}}
\edtext{\edtext{to`uc}{\Dfootnote{to~ic \latintext Thdt.(BA)}}
\edtext{>ep`i}{\Dfootnote{>en \latintext Thdt.(F)}} diaf'oroic
\edtext{\abb{>egkl'hmasin}}{\Dfootnote{\latintext > Ath.}} >egkalo~untac
\edtext{a>uto~ic}{\lemma{\abb{}} \Dfootnote{a>uto~ic
\latintext Ath.(B\corr) ex \greektext a>uto`uc \latintext Ath.(B*)}} >eje'wroun >apant'hsantac}{\lemma{to`uc >ep`i \dots\
>apant'hsantac} \Dfootnote{\latintext \textit{hos, qui etiam ex diversis locis
convenerunt, ut eosdem multis criminibus arguerent} Hil. \textit{eos, qui
diversis criminibus ipsos appetebant, quos videbant occurrisse} Cod.\,Ver.}}.
\edtext{desm`a g`ar >~hn ka`i
s'idhra}{\Dfootnote{\latintext \textit{vincula enim et catenae} Cod.\,Ver. \textit{ferrum enim et vincula}
Hil.}}
\edtext{profer'omena}{\Dfootnote{prosfer'omena \latintext Thdt.(AzT)
\textit{proferebantur} Cod.\,Ver. \textit{proferebant} Hil.}},
\edtext{\edtext{>ap''}{\lemma{\abb{}} \Dfootnote{ka`i >ap'' \latintext Thdt.;
Hil. et Cod.\,Ver. non interpunxerunt ante
\greektext >ap''}}
\edtext{>exor'iac}{\Dfootnote{>exorist'iac \latintext Ath.}} >epanelj'ontec
>'anjrwpoi}{\Dfootnote{\latintext \textit{de exilio reverentes viri} Hil.
\textit{ab hominibus de exilio reversis} Cod.\,Ver.}}
\edtext{ka`i}{\Dfootnote{\latintext ante \greektext ka`i \latintext interpunxit
et continuit \textit{et rursum} Hil. \textit{et} Cod.\,Ver., qui non hic interpunxit
sed post \greektext >elj'ontec ~>hsan}}
\edtext{par`a t~wn}{\Dfootnote{par'' a>ut~wn \latintext Thdt.(L)}}
\edtext{\abb{>'eti}}{\Dfootnote{\latintext Thdt.(H) Ath. \textit{adhuc} Hil.
Cod.\,Ver. \greektext >eke~i \latintext Thdt.(BA\corr NszT)}} kateqom'enwn
\edtext{e>ic >exorist'ian}{\Dfootnote{>en >exor'iaic \latintext Thdt. \textit{in
exilio} Hil. Cod.\,Ver.}}
\edtext{\edtext{>elj'ontec ~>hsan}{\Dfootnote{>epanelj'ontec \latintext
Thdt.(B)}}
\edtext{\abb{\edtext{sulleitourgo`i}{\Dfootnote{leitourgo`i \latintext Thdt.(F) \greektext sulleitourgo~ic (o~ic \latintext ex \greektext oi \latintext Thdt.(A\corr)) Thdt.(A)}},
\edtext{\abb{suggene~ic}}{\Dfootnote{+ te \latintext s.l. Thdt.(A\corr)}}}}{\Dfootnote{\dt{Thdt., Ath. post} suggene~ic \dt{interpunxit}}} ka`i
\edtext{\abb{f'iloi}}{\Dfootnote{+ d`e \latintext Ath. ras. post \greektext f'iloi \latintext Thdt.(A)}} t~wn di''
\edtext{a>uto`uc}{\Dfootnote{a>ut~wn \latintext Ath.}}
\edtext{>apojan'ontwn}{\Dfootnote{paj'ontwn \latintext Thdt.(F)}}
pareg'enonto}{\lemma{>elj'ontec ~>hsan \dots\ pareg'enonto}
\Dfootnote{\latintext ante \greektext >elj'ontec ~>hsan \latintext interpunxit
et continuit \textit{venerunt enim ministri, parentes et amici mortuorum propter
ipsos} Cod.\,Ver. \textit{missi adfines et proximi, amici et fratres querellas vel
eorum, qui asupersunt detullerunt vel eorum, qui in ipsis exiliis decesserunt,
indignam iniuriam prosecuti sunt} Hil.}};
\edtext{ka`i t`o m'egiston}{\Dfootnote{\latintext \textit{et, quod est maximum}
Hil. \textit{et, quod maius est} Cod.\,Ver.}}, >ep'iskopoi par~hsan, \edtext{~<wn
<o m`en}{\Dfootnote{\latintext \textit{et quidam} Cod.\,Ver. \textit{quorum alter}
Hil.}} t`a s'idhra ka`i t`ac
\edtext{kat'hnac}{\Dfootnote{<al'useic \latintext Ath. carhenas Hil.}}
\edtext{pro'eferen}{\Dfootnote{peri'eferen \latintext Ath.(B)
\textit{praeferebat} Hil. \textit{offerebant} Cod.\,Ver.}}
\edtext{\edtext{<`ac}{\Dfootnote{<`a \latintext Thdt.(BAr)}} di''
\edtext{a>uto`uc}{\Dfootnote{a>ut~wn \latintext Thdt.(T)}}
\edtext{>ef'oresen}{\Dfootnote{>ef'orhsen \latintext
Thdt.(A)}}}{\Dfootnote{\latintext \textit{quas per eos cervitibus suis
portaverat} Hil. \textit{quas propter ipsos portaverant} Cod.\,Ver.}},
\edtext{o<i d`e t`on
\edtext{>ek}{\Dfootnote{>ep`i \latintext Thdt.(A)}} t~hc diabol~hc a>ut~wn
j'anaton
\edtext{\abb{>emart'uranto}}{\Dfootnote{\latintext Ath.(E\corr) ex \greektext >emart'urato \latintext Ath.(E*)}}}{\lemma{o<i d`e  \dots\ >emart'uranto}
\Dfootnote{\latintext \textit{alteri ex eorum insimulatione mortem sibi intemtam
testificabantur} Hil. \textit{alii ex ipsorum insimulationibus mortem contestati
sunt} Cod.\,Ver.}}.
\edtext{e>ic
\edtext{toso~uton}{\Dfootnote{toso~uto \latintext Thdt.(B)}} g`ar >'efjasan
>apono'iac}{\Dfootnote{\latintext \textit{in tantum enim desperationis
eruperunt} Hil. \textit{in tantam enim pervenerant inmanitatem} Cod.\,Ver.}} <wc
ka`i
\edtext{>episk'opouc}{\Dfootnote{>ep'iskopon \latintext Thdt. \textit{episcopos}
Hil. Cod.\,Ver.}}
\edtext{>epiqeire~in >anele~in}{\Dfootnote{>epeqe'irhsan \latintext Thdt.(T)
\textit{auderent episcopos velle
interficere} Cod.\,Ver. > Hil.}};
\edtext{ka`i >ane~ilon >`an}{\Dfootnote{\latintext
\textit{et interfecissent} Cod.\,Ver. \textit{interfecissent quidem} Hil.}} e>i m`h
\edtext{\edtext{>ex'efugon}{\Dfootnote{>ex'efuge \latintext Thdt.}}
\edtext{t`ac qe~irac a>ut~wn}{\Dfootnote{\latintext \textit{ipsorum cruentas
manus} Hil. \textit{manus eorum} Cod.\,Ver.}}}{\lemma{\abb{>ex'efugon \dots\ a>ut~wn}}\Afootnote{\latintext vgl. 2Cor 11,33}}\edindex[bibel]{Korinther II!11,33|textit}.
\edtext{>ap'ejanen}{\Dfootnote{\dt{\textit{decessit} Hil.} >an'esth \dt{Thdt.
\textit{surrexit} Cod.\,Ver.}}}
\edtext{go~un <o}{\Dfootnote{o~>un <o \latintext Ath.(KORE) \greektext
\responsio\ <o o~>un \latintext  Ath.(B) \textit{enim} Hil. \textit{itaque} Cod.
Ver.}}
\edtext{sulleitourg`oc}{\Dfootnote{\latintext \textit{conminister} Cod.\,Ver. \textit{coepiscopos} Hil.}} <hm~wn <o
\edtext{makar'ithc}{\Dfootnote{\latintext \textit{beatissimus} Hil.
\textit{beatus} Cod.\,Ver.}}
\edtext{Je'odouloc}{\Dfootnote{\latintext \textit{Theodosus}
Hil.}}\edindex[namen]{Theodul!Bischof von Traianupolis}, fe'ugwn
\edtext{a>ut~wn t`hn diabol'hn}{\Dfootnote{\latintext \textit{ipsorum
infestationem} Hil. \textit{eorum insimulationes} Cod.\,Ver.}};
\edtext{kek'eleusto}{\Dfootnote{\latintext \textit{iussus est} Hil. \textit{iussus \dots\ fuerat} Cod.\,Ver.}} g`ar
>ek
\edtext{diabol~hc}{\Dfootnote{\latintext \textit{insimulationibus} Cod.\,Ver. \textit{insimulatione} Hil.}}
\edtext{a>ut~wn}{\Dfootnote{a>ut`on \latintext Ath.(BO) \textit{ipsorum} Hil.
\textit{eorum} Cod.\,Ver.}} >apojane~in. >'alloi
\edtext{d`e}{\Dfootnote{\latintext \textit{autem} Hil. \textit{et} Cod.\,Ver.}}
\edtext{xif~wn plhg`ac}{\lemma{\abb{}} \Dfootnote{\responsio\ plhg`ac xif~wn
\latintext Ath. \textit{gladiorum signa, plagas et cicatrices} Hil.
\textit{gladiorum vulnera} Cod.\,Ver.}}
\edtext{>epede'iknunto}{\Dfootnote{\latintext \textit{ostendebant} Hil.
\textit{demonstrantes} Cod.\,Ver.}},
\edtext{>'alloi}{\Dfootnote{>'alloi d`e \latintext Thdt.(A) Ath. \textit{alii}
Hil. \textit{nonnulli} Cod.\,Ver.}}
\edtext{lim`on <upomemenhk'enai par'' a>ut~wn}{\Dfootnote{\latintext \textit{se
fame ab ipsis excruciatos} Hil. \textit{famem sustinuisse ab ipsis} Cod.\,Ver.}}
\edtext{>apwd'uronto}{\Dfootnote{>apwd'uranto \latintext Thdt.(rzT)}}. ka`i
\edtext{\abb{ta~uta}}{\Dfootnote{+ m`en \latintext Ath.(KO)}}
\edtext{o>uq o<i tuq'ontec}{\Dfootnote{\latintext \textit{nobiles} Cod.\,Ver.
\textit{non innobiles} Hil.}} >emart'uroun
\edtext{>'anjrwpoi}{\Dfootnote{\latintext \textit{homines} Cod.\,Ver.
\textit{viri} Hil.}},
\edtext{>all''}{\Dfootnote{>all`a ka`i \latintext Thdt.(z)}}
\edtext{>ekklhs'iai <'olai >~hsan}{\Dfootnote{\latintext \textit{de ecclesiis
omnibus electi} Hil. \textit{ecclesiae omnes erant} Cod.\,Ver.}},
\edtext{<up`er ~<wn}{\Dfootnote{\latintext \textit{propter quas} Hil.
\textit{pro quibus} Cod.\,Ver.}}
\edtext{o<i >apant'hsantec
\edtext{ka`i}{\Dfootnote{ka`i o<i \latintext Thdt.(A)}} presbe'uontec
\edtext{>ed'idaskon}{\Dfootnote{>ed'idaxan \latintext Thdt.(B)}}}{\lemma{o<i
>apant'hsantec \dots\ >ed'idaskon} \Dfootnote{\latintext \textit{hi convenerunt,
res gestas edocebant} Hil. \textit{occurrentes et aligantes ducebant}
Cod.\,Ver.}}, strati'wtac
\edtext{xif'hreic}{\Dfootnote{\latintext \textit{armatos} Hil \textit{cum
gladiis} Cod.\,Ver.}}, >'oqlouc met`a <rop'alwn, dikast~wn >apeil'ac, plast~wn
gramm'atwn
\edtext{<upobol'ac}{\Dfootnote{\latintext \textit{submissiones} Cod.\,Ver.
\textit{subpositiones} Hil.}} -- >anegn'wsjh g`ar gr'ammata
\edtext{t~wn per`i Je'ognion plattom'enwn}{\Dfootnote{t~wn per`i Je'ognion
\latintext Thdt.(LT) \greektext t~wn per`i Jeog'onion \latintext Thdt.(BArF)
\textit{factae a Theognito falsae} Hil. \textit{Theogenis}
Cod.\,Ver.}}\edindex[namen]{Theognis!Bischof von Nicaea} kat`a
\edtext{t~wn sulleitourg~wn}{\Dfootnote{t~wn \latintext + ras. quinque circa litterarum, \greektext sulleit. \latintext in mg.  Ath.(B)}} <hm~wn, >Ajanas'iou\edindex[namen]{Athanasius!Bischof von Alexandrien}
\edtext{ka`i}{\lemma{\abb{ka`i}}\Dfootnote{\latintext > Cod.\,Ver.}} Mark'ellou\edindex[namen]{Markell!Bischof von Ancyra}
\edtext{\abb{ka`i >Asklhp~a}}{\Dfootnote{\latintext > Hil.}}\edindex[namen]{Asclepas!Bischof von Gaza}, <'ina ka`i
\edtext{basil'eac}{\Dfootnote{basil'ewc \latintext Thdt.(B) \greektext basil'ea
\latintext Thdt.(n) \textit{imperator} Hil. \textit{imperatores} Cod.\,Ver.}}
kat'' a>ut~wn
\edtext{kin'hswsi}{\Dfootnote{\latintext \textit{commoveretur} Hil.
\textit{moverent} Cod.\,Ver.}};
ka`i ta~uta >'hlegxan o<i gen'omenoi t'ote di'akonoi
\edtext{Jeogn'iou}{\Dfootnote{Jeogon'iou \latintext Thdt.(BrF) \greektext
Jeogn'iou (\latintext ras. inter \greektext g \latintext et \greektext n) \latintext Thdt.(A)  \greektext Diogn'iou \latintext Ath.(R)
\textit{ipsius Theogniti} Hil. \textit{Theogenio} Cod.\,Ver.}}\edindex[namen]{Theognis!Bischof von Nicaea} --
\edtext{pr`oc to'utoic}{\Dfootnote{pr`oc to'utwn \latintext Ath. \textit{ad haec} Hil. \textit{nam
et} Cod.\,Ver.}} parj'enwn gumn'wseic, >emprhsmo`uc >ekklhsi~wn,
\edtext{fulak`ac}{\Dfootnote{\latintext \textit{carceres} Hil. \textit{carcerem}
Cod.\,Ver.}} kat`a t~wn
\edtext{sulleitourg~wn}{\Dfootnote{leitourg~wn \latintext Ath. \textit{ministros
dei} Hil. \textit{conministros} Cod.\,Ver.}};
\edtext{\abb{ka`i}}{\Dfootnote{\latintext > Hil.}} ta~uta p'anta
\edtext{\abb{di'' o>ud`en <'eteron >`h}}{\Dfootnote{\latintext > Hil.}} di`a
t`hn
\edtext{dus'wnumon}{\Dfootnote{\latintext \textit{iniquam atque execrabilem}
Hil. \textit{inmemorabilem} Cod.\,Ver.}} a<'iresin t~wn
\edtext{>Areiomanit~wn}{\Dfootnote{\latintext \textit{Arrianorum} Cod.\,Ver.
\textit{Arriomanitarum} Hil.}}.
\edtext{\edtext{o<i g`ar paraito'umenoi t`hn
\edtext{pr`oc to'utouc}{\Dfootnote{pr`oc to'utoic \latintext Thdt.(zT)}}
\edtext{koinwn'ian}{\Dfootnote{qeiroton'ian ka`i koinwn'ian \latintext Thdt.(zT)
\greektext qeiroton'ian \latintext Thdt.(Ar)}}}{\lemma{o<i g`ar \dots\
koinwn'ian} \Dfootnote{\latintext \textit{qui enim communioni eorum
renuntiabant} Hil. \textit{excusantes enim horum ordinationem et communionem}
Cod.\,Ver.}}
\edtext{>an'agkhn e>~iqon
\edtext{peiraj~hnai}{\Dfootnote{peirasj~hnai \latintext Ath.(B)  ex \greektext
peiraj~hnai \latintext Ath.(B*) \greektext peir~asjai \latintext Ath.(KORE)}}
to'utwn}{\Dfootnote{\latintext \textit{necessitatem patiebantur ista tolerare}
Hil. \textit{necessario haec sustinebant} Cod.\,Ver.}}.}{\lemma{\abb{o<i g`ar
\dots\ to'utwn.}} \Dfootnote{\latintext > Thdt.(B)}}
\pend
\pstart
\skipnumbering
\pend
\pstart
\skipnumbering
\pend
\pstart
\kap{7}ta~uta to'inun
\edtext{sunor~wntec}{\Dfootnote{proor~wntec \latintext Thdt.(A)}}
\edtext{\edtext{\abb{e>ic}}{\Dfootnote{\latintext > Thdt.(A)}} sten`on e>~iqon
t`a t~hc proair'esewc}{\lemma{e>ic sten`on \dots\ proair'esewc}
\Dfootnote{\latintext \textit{in angusto se teneri pervidebant} Hil.
\textit{perducti sunt ad suae volumptatis} (\textit{m} del.) \textit{interitum}
Cod.\,Ver.}}.
\edtext{>h|sq'unonto}{\Dfootnote{a>isqun'omenoi \latintext Thdt. \textit{erubescentes} Hil. \textit{erubescebant} Cod.\,Ver.}}
\edtext{g`ar}{\Dfootnote{m`en g`ar \latintext Ath. \textit{quidem} Hil. \textit{enim} Cod.\,Ver.}}
\edtext{\abb{<omologe~in}}{\Dfootnote{\latintext > Thdt.}}
\edtext{<`a dedr'akasi}{\Dfootnote{\latintext \textit{sua \dots\ delicta} Cod.\,Ver. \textit{ea, quae conmiserunt} Hil.}}.
\edtext{di`a t`o}{\Dfootnote{di`a d`e t`o \latintext Ath. \textit{eo quod} Hil. \textit{et, quoniam} Cod.\,Ver.}}
\edtext{m`h d'unasjai
\edtext{>'eti}{\Dfootnote{loip`on \latintext Ath.(BKO) \greektext loip`on >'eti
\latintext Ath.(RE)}}
\edtext{\abb{ta~uta}}{\Dfootnote{\latintext > Thdt.(A) \greektext t`a a>ut`a
\latintext Thdt.(B)}} kr'uptesjai}{\Dfootnote{\latintext] \textit{de cetero non
possent ea diutius occultare} Hil. \textit{omnibus patefacta celari non
poterant} Cod.\,Ver.}}
\edtext{\abb{>ap'hnthsan}}{\Dfootnote{\latintext + \textit{tandem} Hil.}}
\edtext{\abb{e>ic}}{\Dfootnote{\latintext > Cod.\,Ver.}} t`hn
\edtext{\abb{Serd~wn}}{\Dfootnote{\latintext Ath.(BORE) \greektext Sard~wn \latintext
Ath.(K) \greektext Sard'ewn \latintext Thdt.(Arz) \greektext S'ardewn \latintext
Thdt.(HF\corr) \greektext Sarda'iwn \latintext Thdt.(T) \greektext Sardikhs'iwn
\latintext Thdt.(B) \textit{Sardicensi} Hil. \textit{Sardicam}
Cod.\,Ver.}}\edindex[synoden]{Serdica!a. 343}
\edtext{p'olin}{\Dfootnote{\latintext \textit{civitate} Hil. > Cod.\,Ver.}},
\edtext{<'ina di`a t~hc >af'ixewc <up'onoian <wc m`h plhmmel'hsantec d'oxwsin
>apof'eresjai}{\lemma{<'ina \dots\ >apof'eresjai} \Dfootnote{\latintext
\textit{ut propter praesentiam suspicionem quasi mali auctus viderentur
excludere} Hil. \textit{quo per suum adventum sinceritatis opinionem adquirant}
Cod.\,Ver.}}.
\pend
\pstart
\kap{8}\edtext{\edtext{>id'ontec}{\Cfootnote{\dt{des. Thdt.(H)}}}
o~>un}{\Dfootnote{>id'ontec d`e  \latintext Ath. \greektext
e>id'ontec o~>un \latintext Thdt.(G) \greektext e>id'otec o~>un \latintext
Thdt.(BnS) \textit{visitaque} Cod.\,Ver. \textit{videntes enim} Hil.}}
\edtext{to`uc par'' a>ut~wn sukofanthj'entac\edlabel{X}}{\Dfootnote{\latintext
\textit{eos, qui a se fuerant falsis obpressi testimoniis} Hil. \textit{his,
quibus calumnias ingerebant} Cod.\,Ver.}}
\edtext{\edtext{ka`i to`uc par'' a>ut~wn}{\linenum{|\xlineref{X}}
\lemma{\abb{sukofanthj'entac ka`i to`uc par'' a>ut~wn}} \Dfootnote{\latintext >
Thdt.(s)}} paj'ontac}{\Dfootnote{\latintext \textit{eos etiam, quos post
vehementissime obpugnarunt} Hil. \textit{et adsemflicti (d} s.l.\textit{)}
Cod.\,Ver.}},
\edtext{to`uc kathg'orouc, to`uc >el'egqouc}{\Dfootnote{to`uc te kathg'orouc
ka`i to`uc >el'egqouc \latintext Thdt.(T) \textit{ex alia parte accusatores,
conscientiae etiam suorum criminum proditores} Hil. \textit{accusatoribus,
testibus} Cod.\,Ver.}}
\edtext{pr`o >ofjalm~wn}{\Dfootnote{\latintext \textit{ante oculos} Cod.\,Ver.
\textit{prae oculis} \dots\ \textit{suis} Hil.}}
\edtext{>'eqontec}{\Dfootnote{blep'ontec \latintext Thdt.(BszT)
\textit{habentes} Hil. \textit{inspectis} (\textit{s} s.l.) Cod.\,Ver.}},
\edtext{>elje~in}{\Dfootnote{e>iselje~in \latintext Thdt. \textit{venire} Hil. Cod.Ver}}
\edtext{o>uk >ed'unanto}{\Dfootnote{o>uk >hd'unanto \latintext Thdt.(BSFT)
\greektext o>uk >ebo'ulonto \latintext Ath. \textit{non potuerunt} Hil.
\textit{non poterant} Cod.\,Ver.}} klhj'entec,
\edtext{ka'itoi}{\Dfootnote{\latintext \textit{videlicet} Cod.\,Ver.
\textit{maxime} Hil.}} t~wn
\edtext{sulleitourg~wn}{\Dfootnote{\latintext \textit{consacerdotum} Cod.\,Ver. \textit{conministris} Hil.}}
<hm~wn
\edtext{>Ajanas'iou}{\lemma{\abb{}} \Dfootnote{+ paulo~u \latintext ante \greektext
>Ajanas'iou \latintext Thdt.(T*)}}\edindex[namen]{Athanasius!Bischof von Alexandrien}
\edtext{\abb{ka`i}}{\Dfootnote{\latintext > Thdt.(T) Hil. Cod.\,Ver.}} Mark'ellou\edindex[namen]{Markell!Bischof von Ancyra}
ka`i
\edtext{>Asklhp~a}{\Dfootnote{\latintext \textit{Asclepa}
Cod.\,Ver. \textit{Asclepi} Hil., qui post \textit{Asclepi} + \textit{conscientia
perturbati}}}\edindex[namen]{Asclepas!Bischof von Gaza}
\edtext{poll~h|
\edtext{\abb{t~h|}}{\Dfootnote{\latintext > Thdt.(T)}} parrhs'ia|
\edtext{\abb{qrwm'enwn}}{\Dfootnote{+ ka`i \latintext
Thdt.}}}{\lemma{poll~h| t~h| parrhs'ia| qrwm'enwn}\Dfootnote{\latintext \textit{multa usis confitentia} Cod.\,Ver. > Hil.}} \edtext{>apodurom'enwn ka`i >epikeim'enwn ka`i
\edtext{prokaloum'enwn}{\Dfootnote{proskaloum'enwn \latintext Thdt.(zA)
\greektext proskalloum'enwn \latintext Thdt.((\greektext s \latintext s.l.)T)}} a>uto`uc ka`i
>epaggellom'enwn}{\lemma{>apodurom'enwn \dots\ >epaggellom'enwn}
\Dfootnote{\latintext \textit{qui et inplorabant iudicium provocabant
accusatores} Hil. \textit{inplorantibus et urgentibus adque provocantibus et
pollicentibus quoque} Cod.\,Ver.}}
\edtext{\edtext{\abb{m`h m'onon}}{\Dfootnote{\latintext > Thdt.(B)}} >el'egqein
t`hn sukofant'ian}{\Dfootnote{] \latintext \textit{ut eos non solum convincerent
de his, quae adversum sese falsa confixerant} Hil. \textit{non solum eorum
arguere calumniam} Cod.\,Ver.}}
\edtext{>all`a ka`i}{\Dfootnote{\latintext \textit{verum etiam} Hil. \textit{sed
insuper} Cod.\,Ver.}}
\edtext{deikn'unai}{\Dfootnote{\latintext \textit{demonstrare} Cod.\,Ver.
\textit{eodocerent} Hil.}}
\edtext{<'osa kat`a t~wn >ekklhsi~wn
\edtext{\abb{a>ut~wn}}{\Dfootnote{\latintext Thdt.(A\slin)}}
\edtext{>eplhmm'elhsan}{\Dfootnote{>ekoin'wnhsan \latintext Ath.}}}{\lemma{<'osa \dots\ >eplhmm'elhsan}\Dfootnote{\latintext \textit{quae
adversus ecclesias impia ac nefaria conmiserunt} Hil. \textit{quae reliquerint
in ecclesiis ipsorum} Cod.\,Ver.}}. o<i d`e toso'utw| f'obw|
\edtext{\edtext{\abb{to~u}}{\Dfootnote{\latintext > Thdt.(F)}}
suneid'otoc}{\Dfootnote{\latintext > Hil.}}
\edtext{katesq'ejhsan}{\Dfootnote{kathsq'ejhsan \latintext Thdt.(A)
\textit{conprehensi sunt} Hil. \textit{percussi sunt} Cod.\,Ver.}}
\edtext{\abb{<wc}}{\Dfootnote{\latintext + \textit{etiam }Cod.\,Ver.}}
\edtext{\edtext{fuge~in}{\Dfootnote{fe'ugein \latintext Ath.(KORE)}}
a>uto'uc}{\Dfootnote{\latintext \textit{fugam facerent} Hil. \textit{verterentur
in fugam} Cod.\,Ver.}} ka`i
\edtext{di`a t~hc fug~hc}{\Dfootnote{\latintext \textit{per eandem fugam} Hil.
\textit{per eam} Cod.\,Ver.}}
\edtext{t`hn sukofant'ian}{\Dfootnote{\latintext \textit{falsitatem} Hil.
\textit{calumniam} Cod.\,Ver.}}
\edtext{a>ut~wn}{\Dfootnote{<eaut~wn \latintext Ath.(B\corr KOR) > Thdt.(L)
\textit{suam} Hil. Cod.Ver}}
\edtext{>el'egxai ka`i <'aper >eplhmm'elhsan di`a t~wn drasm~wn <omolog~hsai}{\lemma{>el'egxai \dots\ <omolog~hsai} \Dfootnote{\latintext
\textit{detegerent ac nudarent} Hil. \textit{demonstrarent et delicta
faterentur} Cod.\,Ver.}}.
\pend
\pstart
\kap{9}\edtext{e>i ka`i}{\Dfootnote{\latintext \textit{tamen si} + in mg.
\textit{de falsis criminibus s\oline{c}i Athanasi} Hil. \textit{et licet ex
hoc, quod noluerint, et} Cod.\,Ver.}}
\edtext{\abb{t`a m'alista}}{\Dfootnote{+  to'inun \latintext Ath. > Hil. Cod.\,Ver.}} o>u m'onon >ek t~wn prot'erwn,
>all`a ka`i
\edtext{>ek to'utwn}{\Dfootnote{\latintext \textit{ex istis} Hil. \textit{ex
praesentibus} Cod.\,Ver.}}
\edtext{<h kakotrop'ia ka`i <h sukofant'ia a>ut~wn de'iknutai}{\lemma{\abb{}}\Dfootnote{\responsio\ <h sukofant'ia ka`i <h kakotrop'ia
a>ut~wn de'iknutai \latintext Thdt.(r) \greektext \responsio\ a>ut~wn <h kakotrop'ia
de'iknutai ka`i sukofant'ia \latintext Ath. \textit{malitia eorum et falsitas
ostenditur} Hil. \textit{ipsorum versutia conprobata est} Cod.\,Ver.}},
\edtext{\abb{<'omwc}}{\Dfootnote{\latintext > Hil.}}
\edtext{<'ina mhd`e}{\Dfootnote{<'ina m`h \latintext Thdt.(T) \textit{ut nec}
Hil. \textit{ne} Cod.\,Ver.}}
\edtext{>ek t~hc fug~hc}{\Dfootnote{\latintext \textit{existimata fuga} Hil.
\textit{ex fuga} Cod.\,Ver.}}
\edtext{pr'ofas'in tina}{\Dfootnote{\latintext \textit{occasionem aliquam} Hil.
\textit{quandam occasionem} Cod.\,Ver.}} <et'erac
\edtext{kakourg'iac}{\Dfootnote{\latintext \textit{malae artis} Hil.
\textit{malitiae} Cod.\,Ver.}} por'isasjai
\edtext{dunhj~wsin}{\Dfootnote{do~ien \latintext Thdt.(r) \textit{possint} Hil.
\textit{valeant} Cod.\,Ver.}},
\edtext{>eskey'ameja}{\Dfootnote{\latintext \textit{probavimus} Hil.
\textit{cogitavimus} Cod.\,Ver.}}
\edtext{\edtext{kat`a t`on}{\lemma{\abb{}}\Dfootnote{\responsio\ t`on kat`a
\latintext Thdt.(B)}} t~hc >alhje'iac l'ogon}{\Dfootnote{\latintext
\textit{secundum veritatem et rationem} Hil. \textit{secundum veritatis
rationem} Cod.\,Ver.}} t`a
\edtext{par'' >eke'inwn}{\Dfootnote{\latintext \textit{ab illis} Hil. \textit{ab
his} Cod.\,Ver.}} dramatourghj'enta
\edtext{>exet'asai}{\Dfootnote{\latintext \textit{inquiri oportuisse} Hil.
\textit{ventilare} Cod.\,Ver.}}.
\edtext{ka`i to~uto
\edtext{proj'emenoi}{\Dfootnote{prosj'emenoi \latintext
Ath.(E)}}}{\Dfootnote{\latintext \textit{et} Hil. \textit{quo facto}
Cod.\,Ver.}},
\edtext{e<ur'hkamen}{\Dfootnote{e<'uromen \latintext Ath. \textit{invenimus}
Hil. \textit{reperimus} Cod.\,Ver.}} a>uto`uc
\edtext{\abb{ka`i}}{\Dfootnote{\latintext > Thdt. \textit{etiam} Hil.
\textit{et} Cod.\,Ver.}}
\edtext{>ek t~wn
\edtext{praqj'entwn}{\Dfootnote{leqj'entwn \latintext Thdt.(F) \greektext gr.
leqj'entwn \latintext Thdt.(A\mg)}}}{\Dfootnote{\latintext \textit{de his} Hil.
\textit{ex actitatis} Cod.\,Ver.}}
\edtext{sukof'antac}{\Dfootnote{\latintext \textit{fuisse mendaces} Hil.
\textit{calumniatores} Cod.\,Ver.}} ka`i mhd`en <'eteron >`h
\edtext{>epiboul`hn}{\Dfootnote{\latintext \textit{inpugnationem} Hil.
\textit{insidias} Cod.\,Ver.}}
\edtext{kat`a
\edtext{\abb{t~wn}}{\Dfootnote{\latintext > Thdt.(F)}} sulleitourg~wn
<hm~wn}{\Dfootnote{\latintext ] \textit{adversus consacerdotes nostros} Hil.
\textit{conministris nostris} Cod.\,Ver.}}
\edtext{pepoihk'otac}{\Dfootnote{\latintext \textit{exercuisse} Hil.
\textit{preparasse} Cod.\,Ver.}}.
\pend
\pstart
\kap{10}<`on g`ar
\edtext{>'elegon}{\Dfootnote{>'elege \latintext Thdt.(A)}} par`a >Ajanas'iou\edindex[namen]{Athanasius!Bischof von Alexandrien}
pefone~usjai
\edtext{>Ars'enion}{\Dfootnote{\latintext \textit{nomine Asseni\oline{u}} Hil.
\textit{Arsenium} Cod.Ver}}\edindex[namen]{Arsenius!Bischof von Hypsele},
\edtext{o~<utoc z~h|
ka`i >en to~ic z~wsin >exet'azetai}{\lemma{o~<utoc z~h| \dots\
>exet'azetai}\Dfootnote{\latintext \textit{habetur in numero vivorum} Hil.
\textit{vivit et moratur inter vivos} Cod.\,Ver.}}.
\edtext{>ap`o d`e to'utou}{\Dfootnote{\latintext \textit{unde ex hoc} Hil.
\textit{ex quo} Cod.\,Ver.}} ka`i
\edtext{\edtext{t`a per`i t~wn >'allwn}{\lemma{\abb{}}\Dfootnote{\responsio\
per`i t~wn >'allwn t`a \latintext Thdt.(B)}} jrulhj'enta par''
a>ut~wn}{\lemma{t`a per`i \dots\ par'' a>ut~wn} \Dfootnote{\latintext
\textit{cetera, quae ab ipsis iactabantur} Hil. \textit{illa, quae de aliis
verbosabantur} Cod.\,Ver.}}
\edtext{fa'inetai}{\Dfootnote{\latintext \textit{patuit} Hil. \textit{apparet}
Cod.\,Ver.}}
\edtext{pl'asmata}{\Dfootnote{\latintext \textit{plena mendacii et falsitatis}
Hil. \textit{facta fuisse} Cod.\,Ver.}}.
\edtext{>epeid`h d`e ka`i}{\Dfootnote{\latintext \textit{qui autem} Hil.
\textit{quoniam vero} Cod.\,Ver.}} per`i pothr'iou
\edtext{>ejr'uloun}{\Dfootnote{\latintext \textit{quedam loquebantur} Hil.
\textit{mentionem fecerunt} Cod.\,Ver.}}
\edtext{\abb{<wc}}{\Dfootnote{\latintext > Hil. Cod.\,Ver.}}
\edtext{klasj'entoc}{\Dfootnote{jlasj'entoc \latintext Ath.(BKO)}} par`a
\edtext{Makar'iou}{\Dfootnote{\latintext \textit{Machario} Hil.}}\edindex[namen]{Macarius!Presbyter in Alexandrien} to~u presbut'erou >Ajanas'iou\edindex[namen]{Athanasius!Bischof von Alexandrien},
\edtext{>emart'urhsan}{\Dfootnote{\latintext \textit{testificati sunt Hil.
adhybuerunt testimonium} Cod.\,Ver.}}
\edtext{\abb{m`en}}{\Dfootnote{\latintext > Hil. Cod.\,Ver. \greektext + a>uto~ic
\latintext Thdt.}}
\edtext{o<i paragen'omenoi}{\Dfootnote{o<i paragen'amenoi \latintext Thdt.(BG)
\textit{qui praesentes fuerunt} Hil. \textit{hi, qui} \dots\
\textit{venerunt} Cod.\,Ver.}}
\edtext{>ap`o t~hc >Alexandre'iac ka`i Mare'wtou
[\edtext{\abb{ka`i}}{\Dfootnote{\dt{del. Gelzer}}}]
\edtext{\abb{t~wn t'opwn}}{\Dfootnote{\latintext Thdt.(rz) Ath. \greektext t~wn
loip~wn t'opwn \latintext Thdt.(A) \greektext t~wn
loip~wn \latintext Thdt.(B) \greektext t~wn >'allwn t'opwn \latintext
Thdt.(T)}}}{\lemma{>ap`o t~hc \dots\ t~wn t'opwn} \Dfootnote{\latintext
\textit{ex Alexandria de eodem loco} Hil. \textit{de Alexandriae et Mareotae
locis ipsis} Cod.\,Ver.}},
\edtext{<'oti mhd`en to'utwn
\edtext{p'epraktai}{\Dfootnote{p'eprakto \latintext
Ath.(K)}}}{\Dfootnote{\latintext \textit{eo quod nihil tale esset factum} Hil.
\textit{nihil tale esse commissum} Cod.\,Ver.}}; ka`i o<i >ep'iskopoi
\edtext{\abb{d`e}}{\Dfootnote{\latintext > Cod.\,Ver.}}
\edtext{gr'afontec}{\Dfootnote{gr'ayantec \latintext Thdt. \greektext graf'entec
\latintext Ath.(K) \textit{scribentes} Hil. Cod.\,Ver.}}
\edtext{\abb{o<i}}{\Dfootnote{\latintext > Thdt.(B)}} >ap`o t~hc A>ig'uptou
pr`oc >Io'ulion\edindex[namen]{Julius!Bischof von Rom} t`on
\edtext{sulleitourg`on}{\Dfootnote{\latintext \textit{consacerdotem} Hil. Cod.\,Ver.}} <hm~wn
\edtext{<ikan~wc}{\Dfootnote{\latintext \textit{abunde} Hil.
\textit{sufficienter} Cod.\,Ver.}} diebebaio~unto mhd`e <up'onoian
\edtext{\abb{<'olwc}}{\Dfootnote{\latintext > Hil. Cod.\,Ver.}}
\edtext{\abb{toia'uthn}}{\Dfootnote{\latintext > Thdt.(T) \textit{huiusmodi}
Hil. \textit{talem} Cod.\,Ver.}}
\edtext{\abb{>eke~i}}{\Dfootnote{\latintext > Thdt. Cod.\,Ver. \textit{idem}
Hil.}} gegen~hsjai.
\edtext{>'allwc te}{\Dfootnote{\latintext \textit{tant\oline{u}} \dots\ \textit{etiam}
Hil. \textit{deinde} Cod.\,Ver.}}
\edtext{\abb{<`a}}{\lemma{\abb{}} \Dfootnote{\responsio\ <`a \latintext post
\greektext a>uto~u \latintext Thdt. \textit{quae habent autem} post \greektext a>uto~u \latintext Hil.}}
\edtext{l'egousin}{\Dfootnote{e>inai \latintext Thdt.(s)}} <upomn'hmata >'eqein
kat''
\edtext{\abb{a>uto~u}}{\Dfootnote{\latintext > Cod.\,Ver. \textit{ipsum} Hil.}},
\edtext{\edtext{kat`a monom'ereian}{\Dfootnote{\latintext \textit{absente parte altera} Hil. \textit{ex una parte} Cod.\,Ver.}}}{\lemma{\abb{}}
\Afootnote{\latintext vgl. Ath., apol.\,sec. 27~f.37.41}}
\edtext{sun'esthke}{\Dfootnote{sun'esth \latintext Ath.(RE) Thdt.(B) \greektext
sun'ebh \latintext Thdt.(rzT et (\greektext b \latintext ex A\corr) A)
\textit{constat} Hil. \textit{constitit} Cod.\,Ver.}} gegen~hsjai.
\edtext{\edtext{\abb{ka`i}}{\Dfootnote{\latintext Thdt.(A\slin)}} <'omwc
ka`i}{\lemma{ka`i <'omwc ka`i}\Dfootnote{\latintext \textit{simul tamen} Hil. \textit{et} Cod.\,Ver.}}
\edtext{>en to~ic <upomn'hmasi to'utoic}{\Dfootnote{\latintext \textit{ipsi
satis actis} Hil. \textit{in ipsis gestis} Cod.\,Ver.}} >ejniko`i ka`i
\edtext{\abb{kathqo'umenoi >hrwt~wnto; >ex ~<wn e~<ic}}{\Dfootnote{\latintext >
Hil.}} kathqo'umenoc >erwt'wmenoc >'efasken
\edtext{>'endon e~>inai}{\Dfootnote{\latintext \textit{intus se fuisse} Hil.
\textit{se intus fuisse} Cod.\,Ver.}}
\edtext{\abb{<'ote}}{\Dfootnote{+ <o \latintext Ath.(BKO)}}
\edtext{Mak'arioc >ep'esth t~w| t'opw|}{\Dfootnote{\latintext \textit{superventum fecit
Macharius} Hil. \textit{Macarius ad locum supervenit} Cod.\,Ver.}}\edindex[namen]{Macarius!Presbyter in Alexandrien}, ka`i <'eteroc
>erwt'wmenoc >'elege
\edtext{t`on jrulo'umenon par'' a>ut~wn}{\Dfootnote{\latintext
\textit{famosissimum illum} Hil. \responsio\ \textit{quem dicunt} post \greektext katake~isjai >en kell'iw| \latintext Cod.\,Ver.}}
\edtext{>Isq'uran}{\Dfootnote{\latintext \textit{Scyrum} Hil. \textit{Ischiram}
Cod.\,Ver.}}\edindex[namen]{Ischyras!Presbyter in der Mareotis}
\edtext{\abb{noso~unta}}{\Dfootnote{+ t'ote \latintext Ath. \textit{infirmantem}
Hil. \textit{aegrotum} Cod.\,Ver.}}
\edtext{katake~isjai}{\Dfootnote{ke~isjai \latintext Thdt.(T)}}
\edtext{>en kell'iw|}{\Dfootnote{\latintext \textit{in ecclesia} Hil. \textit{in
cella} Cod.\,Ver.}},
<wc
\edtext{>ap`o to'utou}{\Dfootnote{\latintext \textit{ex hoc} Hil. Cod.\,Ver.}}
fa'inesjai
\edtext{%
\edtext{mhd'' <'olwc}{\Dfootnote{mhd'olwc \latintext Ath.}} gegen~hsja'i ti
\edtext{\abb{t~wn}}{\Dfootnote{+ <'olwn \latintext Thdt.}}
musthr'iwn}{\Dfootnote{\latintext ] \textit{ex toto mysterium non fuisse
celebratum} Hil. \textit{nihil de mysteriis fuisse celebratum} Cod.\,Ver.}},
\edtext{di`a t`o}{\Dfootnote{\latintext \textit{eo quod} Hil. \textit{et ex eo quod} Cod.\,Ver.}} to`uc kathqoum'enouc
\edtext{>'endon e~>inai}{\Dfootnote{e>id'enai \latintext Thdt.(B) \textit{intus consistebant} Cod.\,Ver. \textit{intus fuerint} Hil.}}
\edtext{\abb{ka`i t`on >Isq'uran m`h pare~inai}}{\Dfootnote{\latintext >
Hil.}}\edindex[namen]{Ischyras!Presbyter in der Mareotis},
\edtext{>all`a noso~unta
\edtext{katake~isjai}{\Dfootnote{ke~isjai \latintext Thdt.(LT) + \greektext >en
kell'iw| \latintext Thdt.(sT)}}}{\Dfootnote{\latintext \textit{sed male habens
iacuerint} Hil. \textit{sed aegrotum fuisse} Cod.\,Ver.}}.
\edtext{\abb{ka`i g`ar}}{\Dfootnote{+ ka`i \latintext Ath. Thdt.(B)}} a>ut`oc
\edtext{<o pamp'onhroc >Isq'urac}{\Dfootnote{\latintext \textit{totius malitiae
Scyrus} Hil. \textit{pessimus Ischyras} Cod.\,Ver.}}\edindex[namen]{Ischyras!Presbyter in der Mareotis},
\edtext{yeus'amenoc}{\Dfootnote{\latintext \textit{commentator} Hil. \textit{qui
mentitus est} Cod.\,Ver.}}
\edtext{>ep`i t~w| e>irhk'enai}{\Dfootnote{\latintext \textit{dicebat} Hil.
\textit{dicendo} Cod.\,Ver.}}
\edtext{kekauk'enai}{\Dfootnote{\latintext \textit{exusisse} Hil.
\textit{iussisse} Cod.\,Ver.}} t`on >Ajan'asi'on\edindex[namen]{Athanasius!Bischof von Alexandrien}
\edtext{tina t~wn
\edtext{je'iwn}{\Dfootnote{<ier~wn \latintext Thdt.(L)}}
\edtext{bibl'iwn}{\Dfootnote{b'iblwn \latintext
Thdt.(Nz)}}}{\Dfootnote{\latintext \textit{quaedam de divinis scripturis librum}
Hil. \textit{quosdam divinos codices} Cod.\,Ver.}} ka`i
\edtext{\abb{dielegqje`ic}}{\Dfootnote{\latintext coni. Valois e Hil., sine \greektext ka`i \latintext ante \greektext dielegqje`ic \latintext susp. Opitz \greektext
dielegqje`is yeus'amenoc \latintext susp. Parmentier \greektext diaye'usasjai
\latintext Thdt.(ArzT) \greektext ye'usasjai \latintext Thdt.(B) \greektext
dihl'egqjai \latintext Ath.(BKOE) \greektext dihl'eqjai \latintext Ath.(R) \textit{ad hoc convinctus} Hil. \textit{et in
hoc suum mendatium demonstravit} Cod.\,Ver.}},
\edtext{<wmol'oghse}{\Dfootnote{\latintext \textit{confiteri coepit} Hil.
\textit{nam confessus est} Cod.\,Ver.}}
\edtext{\abb{kat''
\edtext{>eke~ino}{\Dfootnote{>eke'inou \latintext Thdt.(A)}}
kairo~u}}{\Dfootnote{\latintext > Hil. \textit{illo tempore} Cod.\,Ver.}}
nose~in,
\edtext{<'ote}{\Dfootnote{\latintext \textit{quando} Hil. \textit{quo}
Cod.\,Ver.}}
Mak'arioc\edindex[namen]{Macarius!Presbyter in Alexandrien}
par~hn,
\edtext{\abb{ka`i katake~isjai}}{\Dfootnote{\latintext > Hil. \textit{et
iacuisse} Cod.\,Ver.}}, <wc ka`i >ek to'utou
\edtext{sukof'anthn a>ut`on}{\Dfootnote{\latintext \textit{falsum testem illum
esse} Hil. \textit{calumniator} Cod.\,Ver.}}
\edtext{de'iknusjai}{\Dfootnote{e~>inai \latintext Ath.}}.
\edtext{>am'elei t~hc sukofant'iac ta'uthc misj`on}{\lemma{>am'elei \dots\
misj`on} \Dfootnote{\latintext \textit{denique falsitatis praemium} Hil.
\textit{unde ob hanc calumniam \dots\ mercedem} Cod.\,Ver.}}
\edtext{\edtext{\abb{a>ut~w|}}{\Dfootnote{\latintext > Thdt.(BArz) \responsio\
\greektext a>ut`on \latintext ante \greektext ta'uthc \latintext Thdt.(T)}} t~w|
>Isq'ura|}{\Dfootnote{\latintext] \textit{eidem Scyro} Hil. \textit{ipsi
Ischyrae} Cod.\,Ver.}}\edindex[namen]{Ischyras!Presbyter in der Mareotis} ded'wkasin
\edtext{\edtext{>episkop~hc}{\Dfootnote{>episk'opou \latintext Thdt.}}
>'onoma}{\Dfootnote{\latintext \textit{honorem episcopatus} Hil.
\textit{ nomen} Cod.\,Ver.}}
\edtext{t~w| mhd`e presbut'erw| tugq'anonti}{\Dfootnote{\latintext
\textit{homini, qui neque presbiter quidem fuit} Hil. \textit{qui nec \oline{prb}
est} Cod.\,Ver.}}. >apant'hsantec g`ar d'uo
\edtext{presb'uteroi}{\Dfootnote{presbut'eroic \latintext Thdt.(B) \greektext
presb'uteroi + o<i \latintext Thdt.(LA\corr) + \textit{nunc ad concilium}
Hil.}}, s`un
\edtext{Melit'iw|}{\Dfootnote{Melet'iw| \latintext Thdt.(rzT) \textit{Melio}
Hil. \textit{Melitio} Cod.\,Ver.}}\edindex[namen]{Melitius!Bischof von Lycopolis}
\edtext{\edtext{pot`e}{\Dfootnote{\latintext \textit{eo temporis} Hil.
\textit{quondam} Cod.\,Ver.}} gen'omenoi}{\Dfootnote{pot`e gen'amenoi \latintext
Thdt.(A) \greektext paragen'omenoi \latintext Thdt.(F)}} <'usteron
\edtext{\abb{d`e}}{\Dfootnote{\latintext > Hil.}}
\edtext{<up`o}{\Dfootnote{par`a \latintext Thdt.(A)}} to~u
\edtext{makar'iou}{\Dfootnote{makar'itou \latintext Ath. \textit{beato memoriae
viro} Hil. \textit{beato} Cod.\,Ver.}} >Alex'androu\edindex[namen]{Alexander!Bischof von Alexandrien} to~u
\edtext{genom'enou}{\Dfootnote{\latintext \textit{tunc} Cod.\,Ver. > Hil.}}
>episk'opou >Alexandre'iac
\edtext{deqj'entec}{\Dfootnote{prosdeqj'entec \latintext Thdt.(A)}},
\edtext{ka`i n~un}{\Dfootnote{o<`i ka`i \latintext Thdt. \textit{nunc} Hil.
\textit{et nunc} Cod.\,Ver.}} s`un >Ajanas'iw|\edindex[namen]{Athanasius!Bischof von Alexandrien}
\edtext{>'ontec}{\Dfootnote{\latintext \textit{agunt} Hil. \textit{constituti}
Cod.\,Ver.}}
\edtext{>emart'urhsan}{\Dfootnote{\latintext \textit{testati sunt} Hil.
\textit{testimonium prebuerunt} Cod.\,Ver.}}
\edtext{mhd`e p'wpote}{\Dfootnote{\latintext \textit{nusquam omnino} Hil.
\textit{numquam} Cod.\,Ver.}}
\edtext{to~uton}{\Dfootnote{\latintext \textit{istum} Hil. \textit{hunc}
Cod.\,Ver.}} presb'uteron
\edtext{Melit'iou}{\Dfootnote{Melet'iou \latintext Thdt.(rzT) > Thdt.(A)
\textit{Melliti} Hil. \textit{Melitii} Cod.\,Ver.}}\edindex[namen]{Melitius!Bischof von Lycopolis} gegen~hsjai,
\edtext{\abb{
\edtext{mhd'' <'olwc}{\Dfootnote{mhd`e <'olwc \latintext Ath. \textit{neque omnino} Hil.
\textit{et omnino non} Cod.\,Ver.}} >esqhk'enai
\edtext{Mel'ition}{\Dfootnote{Mel'etion \latintext Thdt.(rzT) \textit{eundem
Mellitium} Hil. \textit{Melitium} Cod.\,Ver.}}}}{\Dfootnote{\latintext >
Thdt.(B)}}\edindex[namen]{Melitius!Bischof von Lycopolis}
\edtext{e>ic
\edtext{t`on}{\Dfootnote{t`hn \latintext Thdt.(FB\corr)}}
Mare'wthn}{\Dfootnote{\latintext \textit{in eodem loco apud Mareotem} Hil.
\textit{aput Mareotam} Cod.\,Ver.}}\edindex[namen]{Mareotis} >ekklhs'ian
\edtext{>`h}{\Dfootnote{\latintext \textit{nec} Hil. \textit{vel} Cod.\,Ver.}}
leitourg'on.
\edtext{\abb{ka`i}}{\Dfootnote{\latintext > Hil.}}
\edtext{<'omwc}{\Dfootnote{<'olwc \latintext Thdt.(BrFT) \textit{tamen} Hil.
Cod.\,Ver.}}
\edtext{t`on
\edtext{mhd`e}{\Dfootnote{mht`e \latintext Thdt.(T)}} presb'uteron
tugq'anonta}{\Dfootnote{\latintext \textit{illum} \dots, \textit{qui ante nunc
presbiter fuit} Hil. \textit{eum, qui nec esset \oline{prb}} Cod.\,Ver.}} n~un
\edtext{\edtext{\abb{<wc}}{\Dfootnote{\latintext > Thdt.(ALT)}}
>ep'iskopon}{\Dfootnote{] \latintext \textit{velut episcopum} Hil. \textit{pro
episcopo} Cod.\,Ver.}}
\edtext{pro'hgagon}{\Dfootnote{pros'hgagon \latintext Thdt.(B)
\textit{produxerunt} Cod.\,Ver. \greektext >'hgagon \latintext Ath. \textit{secum
adduxerunt} Hil.}}, <'ina
\edtext{t~w| >on'omati
\edtext{\abb{to'utw|}}{\Dfootnote{+ >ag'agwsi \latintext Ath.(B*), exp.
Ath.(B\corr)}}}{\Dfootnote{] \latintext \textit{nomine episcopatus} Hil.
\textit{hoc nomine} Cod.\,Ver.}}
\edtext{d'oxwsin}{\Dfootnote{d'oxwsin \latintext (ras. inter \greektext x \latintext et \greektext w\latintext ) Thdt.(A) \greektext
de'ixwsin \latintext Thdt.(NG) \textit{videantur} Hil. > Cod.\,Ver.}}
\edtext{\abb{t~hc sukofant'iac}}{\Dfootnote{\latintext coni. Wintjes \greektext >ep`i
t~h| sukofant'ia| \latintext Thdt.(BrFT) \greektext t~h| sukofant'ia| \latintext
Ath. \greektext >ep`i t~hc sukofant'iac \latintext Thdt.(AL)  \textit{falsitatem
suam} Hil. \textit{eorum calumnias} Cod.\,Ver.}}
\edtext{katapl'httein}{\Dfootnote{>epipl'httein \latintext Thdt.(rzT) \greektext
pl'httein \latintext Thdt.(B) \textit{tegere} Hil. \textit{obstupefaciant}
Cod.\,Ver.}}
\edtext{to`uc >ako'uontac}{\Dfootnote{\latintext \textit{apud audientes} Hil
\textit{audientes} Cod.\,Ver.}}.
\pend
\pstart
\kap{11}>anegn'wsjh d`e
\edtext{\abb{ka`i}}{\Dfootnote{\latintext > Cod.\,Ver.}} t`o s'uggramma
\edtext{to~u sulleitourgo~u
\edtext{\abb{<hm~wn}}{\Dfootnote{\latintext > Ath.}} Mark'ellou}{\Dfootnote{]
\latintext \textit{quem conscripsit frater et coepiscopus noster Marcellus} Hil.
\textit{conministri nostri Marcelli} Cod.\,Ver.}}\edindex[namen]{Markell!Bischof von Ancyra} ka`i
\edtext{e<ur'ejh}{\Dfootnote{h<ur'ejh \latintext Ath.}}
\edtext{t~wn per`i E>us'ebion}{\Dfootnote{\latintext \textit{Eusebi et, qui cum
ipfo} (\textit{o} ex \textit{u}) \textit{fuerant} + in mg. \textit{de falsis criminibus que Marcello e\oline{p}o
int\oline{e}tabant heretisi} Hil. \textit{Eusebii} Cod.\,Ver.}}\edindex[namen]{Eusebianer}
\edtext{<h kakoteqn'ia}{\Dfootnote{\latintext \textit{exquisita malitia} Hil.
\textit{commendatio} Cod.\,Ver.}};
\edtext{<`a g`ar <wc zht~wn}{\Dfootnote{\latintext \textit{quae enim ut
proponens} Hil. \textit{quae enim proposita questione} Cod.\,Ver.}} <o
M'arkelloc\edindex[namen]{Markell!Bischof von Ancyra}
\edtext{e>'irhken}{\Dfootnote{\latintext \textit{posuit} Hil. \textit{dicebat}
Cod.\,Ver.}},
\edtext{ta~uta <wc
\edtext{<wmologhm'ena}{\Dfootnote{<omologo'umena \latintext
Ath.}}}{\Dfootnote{\latintext \textit{haec eadem quasi iam conprobans proferret}
Hil. \textit{haec tamquam ab ipso confessa} Cod.\,Ver.}}
\edtext{diabebl'hkasin}{\Dfootnote{\latintext \textit{adsimilarunt} Hil.
\textit{fingebant} Cod.\,Ver.}}. >anegn'wsjh
\edtext{go~un}{\Dfootnote{o~>un \latintext Thdt.(Az) \greektext + ta~uta ka`i
\latintext Thdt. \textit{et haec et} Cod.\,Ver.}} t`a <ex~hc
\edtext{ka`i}{\Dfootnote{\latintext \textit{lecta etiam} Hil.}} t`a pr`o a>ut~wn
\edtext{t~wn zhthm'atwn}{\Dfootnote{\latintext \textit{quaestioni} Hil. \textit{questionum}
Cod.\,Ver.}}, ka`i >orj`h <h p'istic
\edtext{\edtext{\abb{to~u}}{\Dfootnote{+ te
\latintext Thdt.(B)}} >andr`oc}{\Dfootnote{] \latintext \textit{eius} Hil.
\textit{viri} Cod.\,Ver.}}
\edtext{e<ur'ejh}{\Dfootnote{<hur'ejh \latintext Thdt.(BGFT)}}. o>'ute g`ar
>ap`o t~hc <ag'iac
\edtext{Mar'iac}{\Dfootnote{\latintext \textit{virgine Maria} Hil.
\textit{Maria} Cod.\,Ver.}}, <wc a>uto`i
\edtext{diebebai'wsanto}{\Dfootnote{diabebaio~untai \latintext Thdt.(T)
\textit{configebant} Hil. \textit{firmarunt} Cod.\,Ver.}},
\edtext{\abb{>arq`hn}}{\Dfootnote{\latintext dupl. Thdt.(B)}}
\edtext{>ed'idou}{\Dfootnote{\latintext \textit{dabatur} Hil. \textit{dedit}
Cod.\,Ver.}}
\edtext{t~w| to~u jeo~u l'ogw|}{\Dfootnote{to~u \latintext > Ath.(B) \greektext t~w| je~w| l'ogw| \latintext Thdt.(A)
\textit{deo verbo} Hil. \textit{verbo dei} Cod.\,Ver.}}, o>'ute \edtext{t'eloc
>'eqein t`hn basile'ian a>uto~u}{\lemma{t'eloc \dots\ a>uto~u}
\Dfootnote{\latintext \textit{finem habere regnum eius} Hil. \textit{finem ei in
regno} Cod.\,Ver.}}
\edtext{\abb{>all`a ka`i}}{\Dfootnote{\latintext > Hil. \textit{sed} Cod.\,Ver.}}
\edtext{\abb{t`hn basile'ian}}{\Dfootnote{\latintext >
Hil. \textit{imperium} Cod.\,Ver.}}
\edtext{>'anarqon ka`i
\edtext{>atele'uthton}{\Dfootnote{>akat'apauston \latintext Thdt.}}
e>~inai}{\Dfootnote{\latintext \textit{sine principio ac sine fine esse} Hil.
\textit{sine initio et fine} \dots\ \textit{esse} Cod.\,Ver.}}
\edtext{\abb{t`hn to'utou}}{\Dfootnote{\latintext > Hil. \textit{salvatoris}
Cod.\,Ver.}}
\edtext{>'egrayen}{\Dfootnote{>'egrayan \latintext Thdt.(r)}}.
\kap{12}\edtext{\abb{ka`i}}{\Dfootnote{\latintext > Hil. \textit{etiam} Cod.\,Ver.}}
\edtext{>Asklhp~ac}{\Dfootnote{\latintext \textit{Asclepius} Hil.
\textit{Asclepas} Cod.\,Ver.}}\edindex[namen]{Asclepas!Bischof von Gaza}
\edtext{d`e}{\Dfootnote{\latintext \textit{sed} Hil. > Cod.\,Ver.}}
\edtext{<o sulleitourg`oc}{\Dfootnote{\latintext \textit{quoepiscopus noster}
Hil. \textit{conminister} Cod.\,Ver.}}
\edtext{pro'hnegken}{\Dfootnote{pros'hnegken \latintext Thdt.(BAszT)
\textit{protulit} Hil. Cod.\,Ver.}}
\edtext{<upomn'hmata}{\Dfootnote{\latintext \textit{acta} Hil. \textit{gesta}
Cod.\,Ver.}}
\edtext{gegenhm'ena}{\Dfootnote{gen'omena \latintext Ath. \textit{confecta} Cod.\,Ver. \textit{quae confecta sunt} Hil.}}
\edtext{>en >Antioqe'ia|}{\Dfootnote{>enant'ia \latintext Thdt.(s) \textit{apud
Anthiociam} Hil. \textit{aput Anthiochiam} Cod.\,Ver.}}\edindex[synoden]{Antiochien!a. 327},
\edtext{\edtext{par'ontwn}{\Dfootnote{par'ontwn ka`i \latintext Thdt.(A)
\greektext par`a \latintext Thdt.(BN)}}
\edtext{\abb{t~wn}}{\Dfootnote{\latintext > Thdt.(sFT)}}
kathg'orwn}{\Dfootnote{\latintext ] \textit{praesentibus adversariis} Hil.
\textit{praesentibus accusatoribus} Cod.\,Ver.}} ka`i E>useb'iou\edindex[namen]{Eusebius!Bischof von Caesarea} to~u >ap`o
Kaisare'iac; ka`i >ek t~wn >apof'asewn t~wn
\edtext{dikas'antwn}{\Dfootnote{dikast~wn \latintext Thdt.(B)
\textit{iudicandum} Cod.\,Ver. \textit{iudicatum} Hil.}}
\edtext{>episk'opwn}{\Dfootnote{\latintext \textit{episcopum} Hil.}}
\edtext{\edtext{>'edeixen}{\Dfootnote{\latintext ostendisse Hil.}}
<eaut`on}{\Dfootnote{>'edeixan a>ut`on \latintext Thdt.(s)}}
\edtext{>aj~won}{\Dfootnote{\latintext \textit{inreprehensibilem} Hil.
\textit{innocentem} Cod.\,Ver.}} e>~inai.
\pend
\pstart
\kap{13}\edtext{e>ik'otwc
\edtext{o~>un}{\Dfootnote{go~un \latintext Thdt.(T)}},
\edtext{\abb{>agaphto`i >adelfo'i}}{\lemma{\abb{}} \Dfootnote{\responsio\ >adelfo`i >agaphto'i
\latintext Thdt.(BAN) \textit{\oline{ff} \oline{d}\oline{d}} Cod.\,Ver. > Hil.}},
\edtext{kalo'umenoi}{\Dfootnote{\latintext \textit{acciti} Hil. \textit{vocati}
Cod.\,Ver.}} poll'akic
\edtext{o>uq <up'hkousan}{\Dfootnote{o>uq <upako'uousin \latintext Thdt.
\textit{non ausi sunt venire} Hil. \textit{non responderunt} Cod.\,Ver.}},
e>ik'otwc
\edtext{>'efugon}{\Dfootnote{\latintext Hil. non interpunxit post \greektext
>'efugon \latintext sed post \greektext >elaun'omenoi}}.
\edtext{<up`o g`ar to~u suneid'otoc}{\Dfootnote{\latintext \textit{conscientiae
suae inpulsu} Hil. \textit{conscientia enim} Cod.\,Ver.}} >elaun'omenoi
\edtext{\abb{fug~h|}}{\Dfootnote{\latintext + \textit{enim} Hil.}}
\edtext{t`ac
\edtext{sukofant'iac}{\Dfootnote{\latintext \textit{falsitatis} Hil.
\textit{calumnias} Cod.\,Ver.}}
\edtext{<eaut~wn}{\Dfootnote{a>ut~wn \latintext Thdt.(Br)}}}{\lemma{\abb{}} \Dfootnote{\responsio\ t`ac <eaut~wn sukofant'iac \dt{Thdt.(A*)}}} >ebeba'iwsan, ka`i
\edtext{pisteuj~hnai kat'' a>ut~wn}{\Dfootnote{\dt{\textit{credi ea} Hil. \textit{credidisse} Cod.\,Ver.}}} pepoi'hkasin,
\edtext{<`a}{\Dfootnote{<'aper \latintext Thdt.}} par'ontec
\edtext{\abb{o<i kathgoro~untec}}{\Dfootnote{\latintext + \textit{eorum} Hil.}}
>'elegon ka`i
\edtext{>epede'iknunto}{\Dfootnote{>epede'iknuon \latintext Ath.(BOR)
\greektext >apede'iknuon \latintext Ath.(K)}}}{\lemma{\abb{e>ik'otwc \dots\
>epede'iknuon}} \Dfootnote{\latintext > Ath.(E)}}.
\edtext{>'eti to'inun}{\Dfootnote{>epe`i to'inun \latintext Ath. \textit{quid
igitur} Hil. \textit{qui itaque} Cod.\,Ver.}} pr`oc to'utoic p~asi ka`i to`uc
\edtext{p'alai}{\Dfootnote{\latintext \textit{olim} Hil. > Cod.\,Ver.}}
\edtext{kajairej'entac}{\Dfootnote{kathgorhj'entac \latintext Thdt.}} ka`i
\edtext{>ekblhj'entac}{\Dfootnote{\latintext \textit{eiectos ecclesia} Hil.
\textit{eiectos ab ecclesia} Cod.\,Ver.}}
\edtext{di`a t`hn >Are'iou a<'iresin}{\Dfootnote{\latintext \textit{propter
heresim Arrii} Hil. \textit{Arrianos} Cod.\,Ver.}}\edindex[namen]{Arianer} o>u m'onon >ed'exanto, >all`a
ka`i
\edtext{\abb{e>ic}}{\Dfootnote{\latintext Thdt.(B\slin)}} me'izona bajm`on
\edtext{pro'hgagon}{\Dfootnote{\latintext \textit{provexerunt} Hil.
\textit{perduxerunt} Cod.\,Ver.}}, diak'onouc m`en e>ic
\edtext{presbut'erion}{\Dfootnote{presbutere~ion \latintext Ath.(R*E), \greektext
-rion \latintext Ath.(R\corr)}},
\edtext{>ap`o d`e presbut'erwn}{\Dfootnote{\latintext \textit{de presbiteris}
Hil. \textit{ex presbiterato} Cod.\,Ver.}}
\edtext{e>ic >episkop'hn}{\Dfootnote{e>ic >episk'opouc \latintext Ath.
\textit{in episcopatum} Hil. \textit{episcopum} Cod.\,Ver.}}, di''
\edtext{o>ud`en <'eteron}{\Dfootnote{\latintext \textit{nihil aliud} Hil.
\textit{nullam aliam causam} Cod.\,Ver.}} >`h <'ina
\edtext{t`hn >as'ebeian}{\Dfootnote{\latintext \textit{impiam doctrinam} Hil.
\textit{impietatem} Cod.\,Ver.}}
\edtext{diaspe~irai}{\Dfootnote{<upospe~irai \latintext Thdt.(L)}} ka`i
\edtext{plat~unai}{\Dfootnote{\latintext \textit{latitent} Cod.\,Ver.
\textit{dilatare} Hil.}}
\edtext{\abb{dunhj~wsi}}{\Dfootnote{\latintext > Cod.\,Ver.}}
\edtext{ka`i}{\Dfootnote{\dt{\textit{vero} Cod.\,Ver.}}}
t`hn e>useb~h
\edtext{diafje'irwsi p'istin}{\lemma{\abb{}}\Dfootnote{\responsio\ p'istin
diafje'irwsi \latintext  Ath.(E)}}.
\pend
\pstart
\kap{14}e>is`i d`e
\edtext{\edtext{\abb{to'utwn}}{\Dfootnote{\latintext > Cod.\,Ver.}}
met`a\edlabel{meta}}{\lemma{\abb{}}\Dfootnote{\responsio\ met`a to'utwn \latintext  Ath.(R*)
\greektext to'utwn met`a \latintext Ath.(R\corr)}}
\edtext{to`uc per`i E>us'ebion}{\xxref{meta}{nun}\lemma{\abb{met`a \dots\ n~un}}\Dfootnote{\dt{> Thdt.(L)}}\lemma{\dBar\ to`uc per`i E>us'ebion}\killnumber\Dfootnote{\latintext \textit{Eusevio
duos} Hil. \textit{Eusebium et eius socios} Cod.\,Ver.}}
\edtext{n~un\edlabel{nun}}{\Dfootnote{\dt{\textit{etiam} Hil. \textit{nunc}
Cod.\,Ver.}}}\edindex[namen]{Eusebianer}
\edtext{>'exarqoi}{\Dfootnote{\latintext \textit{auctores} Hil.
\textit{primates} Cod.\,Ver.}}
\edtext{Je'odwroc}{\Dfootnote{\latintext \textit{Thedorus} Hil. \greektext + <o
\latintext Ath. Thdt.(NG)}}\edindex[namen]{Theodorus!Bischof von Heraclea} >ap`o
\edtext{<Hrakle'iac}{\Dfootnote{\latintext \textit{Heraclia} Hil. \textit{Eraclea} Cod.\,Ver.}},
\edtext{\abb{N'arkissoc}}{\Dfootnote{+ <o
\latintext Ath.}}\edindex[namen]{Narcissus!Bischof von Neronias} >ap`o
\edtext{Nerwni'adoc}{\Dfootnote{Neron'idoc \latintext Thdt.(B) \textit{Marona}
Cod.\,Ver. \textit{Nerodiade} Hil.}}
\edtext{t~hc Kilik'iac}{\Dfootnote{\latintext \textit{de Cilicia} Hil.
\textit{Ciliciae} Cod.\,Ver.}},
\edtext{\abb{St'efanoc}}{\Dfootnote{+ <o \latintext Ath. Thdt.(N)
\textit{Stephan\oline{u}}
Hil.}}\edindex[namen]{Stephanus!Bischof von Antiochien}
\edtext{>ap`o >Antioqe'iac}{\Dfootnote{\latintext \textit{ex Antiocia} Hil.
\textit{Antiochiae} Cod.\,Ver.}},
\edtext{%
\edtext{\abb{Ge'wrgioc}}{\Dfootnote{+ <o \latintext Ath.}} >ap`o
Laodike'iac}{\lemma{Ge'wrgioc >ap`o
Laodike'iac}\Dfootnote{\latintext \textit{Gorgius Laudocia} + -- \textit{licet timens
non adfuerit de Oriente} -- Hil. > Cod.\,Ver.}}\edindex[namen]{Georg!Bischof von Laodicea},
\edtext{\abb{>Ak'akioc}}{\Dfootnote{+ <o \latintext Ath.}}\edindex[namen]{Acacius!Bischof von Caesarea}
\edtext{>ap`o Kaisare'iac}{\Dfootnote{\latintext \textit{ex Caesarea} Hil.
\textit{Cessareae} Cod.\,Ver.}} t~hc
\edtext{Palaist'inhc}{\Dfootnote{\latintext \textit{Palestina} Hil. +
\textit{Narcissus Marmie} Cod.\,Ver.}},
\edtext{\abb{Mhn'ofantoc}}{\Dfootnote{+ <o \latintext Ath.(K)}}\edindex[namen]{Menophantus!Bischof von Ephesus}
\edtext{>ap`o >Ef'esou}{\Dfootnote{\latintext \textit{et Epheso} Hil.
\textit{Ephesi} Cod.\,Ver.}}
\edtext{t~hc}{\Dfootnote{t~h \latintext Thdt.(T)}} >As'iac,
\edtext{\abb{\edtext{\abb{O>urs'akioc}}{\Dfootnote{+ <o \latintext
Ath.}}
\edtext{>ap`o
\edtext{\abb{Siggido'unou}}{\Dfootnote{\latintext Thdt.(B) \greektext
Mugd'onou (mug \latintext in ras. Thdt.(A\corr) et ras. ante \greektext m \latintext et inter \greektext 'o \latintext et \greektext n) \latintext Thdt.(A) \greektext
Mugd'onou \latintext (ras. inter \greektext 'o \latintext et \greektext n) \latintext Thdt.(F) \greektext Mugd'onou \latintext Thdt.(rL)
\greektext Siggid'onou \latintext Ath.}}}{\Dfootnote{\latintext ] \textit{a
Singiduno} Hil. \textit{Singiduno} Cod.\,Ver.}} t~hc
\edtext{Mus'iac}{\Dfootnote{\latintext \textit{Moesiae} Cod.\,Ver. \textit{Maesiae}
Hil.}}}}{\Dfootnote{\latintext > Thdt.(T)}}\edindex[namen]{Ursacius!Bischof von Singidunum},
O>u'alhc\edindex[namen]{Valens!Bischof von Mursa}
\edtext{>ap`o Murso~u}{\Dfootnote{>ap`o Mours~wn \latintext Ath.(BREO\corr)
\greektext >ap`o Mars~wn \latintext Ath.(KO*) \textit{ex Myrsa} Hil.
\textit{Morsa} Cod.\,Ver.}} t~hc
Pannon'iac.
\edtext{\abb{ka`i g`ar}}{\Dfootnote{\latintext > Cod.\,Ver.}}
\edtext{o~<utoi}{\Dfootnote{\latintext \textit{praedicti} Hil. \textit{qui}
Cod.\,Ver.}}
\edtext{to`uc
\edtext{\abb{s`un}}{\Dfootnote{\latintext Thdt.(A\slin)}} a>uto~ic
>elj'ontac}{\Dfootnote{] to~ic s`un a>uto~ic >eljo~usin \latintext Thdt.
\textit{qui} \dots\ \textit{secum venerunt} Hil. \textit{illos, cum quibus}
\dots\ \textit{venerunt} Cod.\,Ver.}} >ap`o t~hc <E'w|ac
\edtext{o>uk
\edtext{>ep'etrepon}{\Dfootnote{>'eprepen \latintext Ath.(E*)}}
o>'ute
\edtext{e>ic t`hn <ag'ian s'unodon
\edtext{e>iselje~in}{\Dfootnote{sunelje~in \latintext
Thdt.(A)}}}{\Dfootnote{\latintext \textit{sanctum concilium intrare} Hil.
\textit{aput Asiam in concilio venire} Cod.\,Ver.}}
o>'ute}{\lemma{o>uk \dots\ o>'ute \dots\ o>'ute} \Dfootnote{\latintext \textit{non \dots\ neque} Hil. \textit{neque \dots\ nec} Cod.\,Ver. }}
\edtext{<'olwc}{\Dfootnote{\latintext \textit{omnino} Hil. \textit{penitus}
Cod.\,Ver.}} e>ic t`hn
\edtext{>ekklhs'ian}{\Dfootnote{\latintext \textit{ecclesia} \dots\
\textit{sanctam} Hil.}} to~u jeo~u
\edtext{parabale~in}{\Dfootnote{paraballe~in \latintext Ath. + \greektext
suneq'wrhsan \latintext Thdt.(ArzT) \greektext + suneqwr'hjhsan \latintext
Thdt.(B)}}.
\edtext{ka`i
>erq'omenoi d`e}{\lemma{ka`i \dots\ d`e} \Dfootnote{\latintext \textit{etenim} Hil. \textit{etiam} Cod.\,Ver.}} e>ic t`hn
\edtext{Serdik`hn}{\Dfootnote{Sardik`hn \latintext Ath.(BKO) Thdt.(B\corr)
\textit{Serdicam} Hil. \textit{Sardicam} Cod.\,Ver.}}\edindex[synoden]{Serdica!a. 343}
\edtext{kat`a t'opouc}{\Dfootnote{\latintext \textit{per singula loca} Hil.
\textit{per singula quaeque loca} Cod.\,Ver.}}
\edtext{sun'odouc}{\Dfootnote{\latintext \textit{synodus} Hil.
\textit{conventus} Cod.\,Ver.}} >epoio~unto
\edtext{pr`oc}{\Dfootnote{kaj'' \latintext Ath.(BKO) \textit{inter} Hil.
Cod.\,Ver.}}
\edtext{<eauto`uc}{\Dfootnote{a>uto�c \latintext Thdt.(r)}} ka`i
\edtext{sunj'hkac met`a >apeil~wn}{\Dfootnote{\latintext \textit{cum
interminationibus posuerunt pactiones sibi} Cod.\,Ver. \textit{pactiones cum
interminationibus} Hil.}}, <'wste
\edtext{\edtext{>elj'ontac}{\Dfootnote{e>iselj'ontac \latintext Ath.(E)}}
a>uto`uc e>ic t`hn
\edtext{Serdik`hn}{\Dfootnote{Sardik'hn \latintext Ath.(BKO) (\greektext a \latintext ex
\greektext e)\latintext Thdt.(B\corr)
\textit{Serdicam} Hil. \textit{Sardicam} Cod.\,Ver.}}}{\Dfootnote{\latintext
\textit{cum Sardicam venissent} Cod.\,Ver. \textit{venientes Serdicam}
Hil.}}\edindex[synoden]{Serdica!a. 343}
\edtext{mhd`e}{\Dfootnote{\latintext \textit{non} Hil. Cod.\,Ver.}} <'olwc e>ic
t`hn kr'isin
\edtext{>elje~in}{\Dfootnote{e>iselje~in \latintext Thdt.(s)}},
\edtext{\edtext{m'hte}{\Dfootnote{\latintext Thdt.(BArz) \greektext m'hd'' \latintext
Ath. \textit{neque} Hil. \textit{nec}
Cod.\,Ver.}}
\edtext{>ep`i t`o a>ut`o}{\Dfootnote{\latintext \textit{in unum} Hil. >
Cod.\,Ver.}}}{\Dfootnote{>`h \latintext Thdt.(T)}} sunelje~in
\edtext{t~h|
\edtext{\abb{<ag'ia|}}{\Dfootnote{+ ka`i meg'alh| \latintext
Thdt.}}}{\Dfootnote{] \latintext \textit{c\oline{o} sc\oline{o}} Hil. \textit{cum sancta} Cod.\,Ver.}}
\edtext{sun'odw|}{\lemma{\abb{}}\Dfootnote{sun'odw \latintext in mg. Thdt.(A\corr)}},
\edtext{>all`a m'onon >elj'ontac ka`i
\edtext{>afosi'wsei}{\Dfootnote{>ep`i >afosi'wsei \latintext Thdt.(A)}} t`hn
\edtext{\edtext{>epidhm'ian}{\Dfootnote{>apodhm'ian \latintext Ath.(RE)}} <eaut~wn}{\lemma{\abb{}}\Dfootnote{\responsio\ <eaut~wn
>epidhm'ian \latintext  Thdt.}}
>epideixam'enouc taq'ewc fuge~in.}{\lemma{>all`a m'onon \dots\ fuge~in.}
\Dfootnote{\latintext \textit{sed tantum venientes imaginarie adventum suum
demonstrarent ad velociter fugerunt} Cod.\,Ver. \textit{et ad id solum venerunt
coetum praesentiam suam qui stat imfugerunt} Hil.}} ta~uta
\edtext{g`ar}{\Dfootnote{\latintext \textit{enim} Hil. \textit{autem} Cod.\,Ver.}}
gn~wnai dedun'hmeja par`a t~wn
\edtext{sulleitourg~wn}{\Dfootnote{\latintext \textit{conministris} Cod.\,Ver. \textit{consacerdotibus} Hil.}} <hm~wn
\edtext{\abb{((Mak))ar'iou}}{\lemma{\abb{Makar'iou}}\Dfootnote{\dt{Ath. Thdt.} >Are'iou \dt{coni. Opitz e Ath., h.Ar. 15,4 \textit{Ario} + \textit{se}
Hil. \textit{Macario} Cod.\,Ver.}}}\edindex[namen]{Arius!Bischof von Petra}
\edtext{>ap`o Palaist'inhc}{\Dfootnote{\latintext \textit{ex Palesna} Hil.
\textit{Palestinae} Cod.\,Ver.}}
\edtext{\abb{ka`i}}{\Dfootnote{\latintext > Hil}}
\edtext{>Aster'iou}{\Dfootnote{\latintext \textit{Stefano} Hil.}}\edindex[namen]{Asterius!Bischof in Arabien} >ap`o >Arab'iac t~wn
\edtext{>elj'ontwn}{\Dfootnote{sunelj'ontwn \latintext Thdt.(rz)}} s`un
\edtext{a>uto~ic}{\Dfootnote{\latintext \textit{ipsis} Hil. Cod.\,Ver.}}
\edtext{ka`i}{\Dfootnote{\latintext \textit{quidem, sed et} Hil. \textit{et}
Cod.\,Ver.}} >anaqwrhs'antwn
\edtext{>ap`o t~hc >apist'iac a>ut~wn}{\Dfootnote{\latintext \textit{ab eorum
perfidia} Hil. \textit{ab eadem incredulitate} Cod.\,Ver.}}.
\edtext{o~<utoi}{\Dfootnote{\latintext \textit{hii} Hil. \textit{isti}
Cod.\,Ver.}} g`ar
\edtext{\abb{>elj'ontec}}{\Dfootnote{\latintext > Thdt.(A)}} e>ic t`hn <ag'ian
s'unodon, t`hn
\edtext{\abb{m`en}}{\Dfootnote{\latintext > Cod.\,Ver.}} b'ian
\edtext{<`hn}{\Dfootnote{\latintext \textit{quam} Hil. > Cod.\,Ver.}} >'epajon
\edtext{\edtext{>apwd'uronto}{\Dfootnote{>apwd'uranto \latintext Ath.(KORE)
Thdt.(BzT) \textit{conquerebantur} Hil. \textit{cum lacrimis exposuerunt}
Cod.\,Ver.}}, o>ud`en d`e
\edtext{par'' a>uto~ic}{\Dfootnote{\latintext \textit{ab ipsis} Hil.
\textit{aput eos} Cod.\,Ver.}}
\edtext{>orj`on}{\Dfootnote{\latintext \textit{recte} Hil. \textit{rectum} Cod.\,Ver.}}}{\lemma{\abb{>apwd'uronto \dots\ >orj`on}}
\Dfootnote{\latintext > Thdt.(A)}} >'elegon
\edtext{pr'attesjai}{\Dfootnote{\latintext \textit{agi} Hil. \textit{fieri}
Cod.\,Ver.}},
\edtext{prostij'entec}{\Dfootnote{prosj'entes \latintext Thdt.(r) \textit{addebant} Cod.\,Ver. \textit{adserentes} Hil.}} ka`i to~uto <wc
\edtext{\abb{>'ara}}{\Dfootnote{\latintext > Hil. Cod.\,Ver.}}
\edtext{\abb{e~>ien >eke~i pollo`i}}{\Dfootnote{\latintext Ath. \greektext
e~>ien \latintext Thdt.(r) \greektext e~>inai \latintext Thdt.(BAzT)
\textit{essent ibi plurimi} Hil. \textit{esse quosdam} Cod.\,Ver.}} t~hc >orj~hc
\edtext{\abb{>antipoio'umenoi}}{\Dfootnote{\latintext > Hil. Cod.\,Ver.}} d'oxhc
\edtext{ka`i kwlu'omenoi
\edtext{par''}{\Dfootnote{>ex \latintext Thdt.}} a>ut~wn >elje~in
\edtext{\abb{>enta~uja}}{\Dfootnote{+ ka`i \latintext Thdt.}} di`a t`o
\edtext{>apeile~in}{\Dfootnote{>apelje~in \latintext Thdt.(S)}}
\edtext{\abb{ka`i >epagg'ellesjai}}{\Dfootnote{\latintext Ath. \greektext ka`i
>ent'ellesjai \latintext Thdt. \greektext ka`i >epanate'inesjai \latintext
coni. Scheidweiler}} kat`a t~wn boulom'enwn >anaqwre~in >ap'' a>ut~wn}{\lemma{ka`i
kwlu'omenoi \dots\ >ap'' a>ut~wn} \Dfootnote{\latintext \textit{qui
prohibebantur ab ipsis venire propositis ad nos interminationibus} Hil.
\textit{et interminationibus adque accusationibus prohibere ad nos} Cod.\,Ver.}}.
\edtext{to'utou go~un
\edtext{\abb{<'eneka ka`i}}{\Dfootnote{\latintext >
Thdt.(A)}}}{\Dfootnote{\latintext ] \textit{ac igitur causa} Hil. \textit{hac de
causa} Cod.\,Ver.}}
\edtext{\abb{>en}}{\Dfootnote{\latintext > Ath.(K)}} <en`i
\edtext{o>'ikw|}{\Dfootnote{\latintext \textit{loco} Hil. \textit{domo}
Cod.\,Ver.}}
\edtext{\edtext{\abb{p'antac}}{\Dfootnote{\latintext Ath. \greektext p'antec
\latintext Thdt.}} me~inai
\edtext{>espo'udasan}{\Dfootnote{>'espeusan \latintext
Thdt.(B)}}}{\lemma{\abb{}} \Dfootnote{\responsio\ >espo'udasan p'antec me~inai
\latintext  Thdt.(T) \textit{omnes manere sategerunt} Hil. \textit{conmanere
studuerunt} Cod.\,Ver.}}
\edtext{mhd`e t`o braq'utaton}{\Dfootnote{\latintext \textit{et nec parum
temporis} Hil. \textit{nullo momento} Cod.\,Ver.}}
\edtext{\edtext{\abb{>idi'azein a>uto`uc}}{\Dfootnote{\latintext Ath. \greektext
>idi'azein a>uto~ic \latintext Thdt.(ArzT) \greektext \responsio\ a>uto~ic
>idi'azein \latintext Thdt.(B)}} >epitr'eyantec}{\Dfootnote{] \latintext
\textit{neq. habere eos liberam facultatem permiserunt} Hil. \textit{eos
separari permittentes} Cod.\,Ver.}}.
\pend
\pstart
\kap{15} >epe`i o>~un o>uk >'edei parasiwp~hsai
\edtext{o>ud`e}{\Dfootnote{\latintext \textit{nec} Hil. \textit{et} Cod.\,Ver.}}
\edtext{>anekdihg'htouc}{\Dfootnote{>anekdik'htouc \latintext Thdt.(NS)
\textit{inultas} Hil. \textit{inocultas} Cod.\,Ver.}}
\edtext{>e~asai}{\Dfootnote{\latintext  \textit{retinere} Hil.
\textit{relinquere} Cod.\,Ver.}}
\edtext{t`ac sukofant'iac}{\Dfootnote{\latintext \textit{falsitatis} Hil.
\textit{eorum} \dots\ \textit{calumnias} Cod.\,Ver.}}, t`a desm'a, to`uc
\edtext{f'onouc}{\Dfootnote{fj'onouc \latintext Thdt.(rF)}},
\edtext{t`ac plhg'ac}{\Dfootnote{\latintext \textit{pugnas} Hil. \textit{plagas}
Cod.\,Ver.}},
\edtext{\abb{\edtext{t`ac}{\Dfootnote{t`a \latintext coni. Opitz}} per`i t~wn
plast~wn >epistol~wn
\edtext{\abb{suskeu'ac}}{\Dfootnote{\latintext > Ath.}}}}{\lemma{\abb{}} \Dfootnote{\responsio\ t`ac per`i \dots\ suskeu'ac
\latintext post \greektext parj'enwn \latintext  Thdt.(T) \textit{falsas
epistulas} Hil. Cod.\,Ver.}},
\edtext{t`ac a>ik'iac}{\Dfootnote{\latintext \textit{verberationes} Hil.
\textit{mulctationes} Cod.\,Ver. \greektext \responsio\ t`ac a>ikei'ac  \latintext post
\greektext parj'enwn \latintext Thdt(B) > Thdt.(T)}},
\edtext{t`ac}{\Dfootnote{t`a per`i t`ac \latintext Thdt.(T)}} gumn'wseic t~wn
parj'enwn, t`ac
\edtext{>exor'iac}{\Dfootnote{>exorist'iac \latintext Ath.}}, t`ac katal'useic
t~wn >ekklhsi~wn, to`uc >emprhsmo'uc,
\edtext{t`ac metaj'eseic >ap`o mikr~wn p'olewn e>ic me'izonac
paroik'iac}{\lemma{t`ac metaj'eseic \dots\ paroik'iac} \Dfootnote{\latintext
\textit{translationes de ecclesia ad maiores} Hil. \textit{translationes de
minimis ad maiores civitates} Cod.\,Ver.}}, ka`i pr'o
\edtext{\abb{ge}}{\Dfootnote{\latintext > Hil. Cod.\,Ver.}} p'antwn t`hn kat`a
t~hc >orj~hc
\edtext{\abb{p'istewc}}{\Dfootnote{+ n~un \latintext Thdt.}}
\edtext{>epanast~asan}{\Dfootnote{\latintext \textit{insurgentes} Hil.}}
\edtext{\abb{dus'wnumon}}{\Dfootnote{\latintext > Thdt.(B) Hil.}}
\edtext{>areian`hn a<'iresin}{\Dfootnote{\latintext \textit{Arrianae heresis
Arrianam doctrin\oline{a}} Hil. \textit{Arrianorum heresim} Cod.\,Ver.}}
\edtext{\abb{di'' a>ut~wn}}{\Dfootnote{\latintext > Hil. \textit{per ipsos} Cod.\,Ver.}},
\edtext{\edtext{\abb{to'utou}}{\Dfootnote{+ go~un \latintext Thdt.}}
<'eneken}{\Dfootnote{] \latintext \textit{ex hac ex causa} Hil. \textit{hac de
causa} Cod.\,Ver.}} to`uc
\edtext{m`en}{\Dfootnote{\latintext \textit{quidem} Hil. > Cod.\,Ver.}}
\edtext{>agaphto`uc}{\Dfootnote{\latintext \textit{carissimos} Hil.
\textit{dilectissimos} Cod.\,Ver.}}
\edtext{>adelfo`uc
\edtext{\abb{<hm~wn}}{\Dfootnote{\latintext > Hil.}}}{\lemma{\abb{}}
\Dfootnote{\responsio\ <hm~wn >adelfo`uc \latintext Thdt.(B) Ath.(K)}} ka`i
\edtext{sulleitourgo'uc}{\Dfootnote{\latintext \textit{coepiscopos} Hil. \textit{consacerdotes} Cod.\,Ver. \greektext + <hm~wn \latintext Ath.(E) Hil. +
\greektext Pa~ulon t`on Kwnstantinoup'olewc >ep'iskopon ka`i \latintext
Thdt.(T*)}}, >Ajan'asion\edindex[namen]{Athanasius!Bischof von Alexandrien}
\edtext{\abb{t`on t~hc >Alexandre'iac}}{\Dfootnote{\latintext > Ath. \greektext
+ >ep'iskopon \latintext Thdt.(BArz) Cod.\,Ver.}} ka`i M'arkellon\edindex[namen]{Markell!Bischof von Ancyra}
\edtext{\abb{t`on t~hc >Agkurogalat'iac}}{\Dfootnote{\latintext > Ath.
\textit{Acyrogalatiae} Hil. \textit{Ancyrae Galatiae} Cod.\,Ver.}} ka`i
\edtext{>Asklhp~an}{\Dfootnote{\latintext \textit{Asclepium} Hil.
\textit{Asclepam} Cod.\,Ver.}}\edindex[namen]{Asclepas!Bischof von Gaza}
\edtext{\abb{t`on G'azhc}}{\Dfootnote{\latintext > Ath. \textit{Gaiae} Hil.
\textit{Gazae} Cod.\,Ver.}}
\edtext{\abb{ka`i}}{\Dfootnote{\latintext > Thdt.(T)}} to`uc
\edtext{\abb{s`un a>uto~ic}}{\Dfootnote{\latintext > Thdt.(BrFT) \textit{cum
ipsis} Hil. Cod.\,Ver.}} sulleitourgo~untac t~w|
\edtext{kur'iw|}{\Dfootnote{q\oline{w} \latintext Thdt.(BrzT) \textit{deo} Hil.
\textit{domino} Cod.\,Ver.}}, >aj'wouc ka`i kajaro`uc
\edtext{\abb{e~>inai}}{\Dfootnote{\latintext > Hil.}}
\edtext{>apefhn'ameja}{\Dfootnote{\latintext \textit{pronunciamus} Hil.
\textit{pronuntiamus} Cod.\,Ver.}} gr'ayantec
\edtext{\abb{ka`i}}{\Dfootnote{\latintext > Hil. Cod.\,Ver.}}
\edtext{e>ic t`hn <ek'astou paroik'ian}{\Dfootnote{\latintext \textit{ad
unamquamquam eorum provinciam} Hil. \textit{ad singulorum provincias} Cod.\,Ver.}}
<'wste
\edtext{gin'wskein}{\Dfootnote{gign'wskein \latintext Thdt.(BArz) \greektext
>epigign'wskein \latintext Thdt.(T)}}
\edtext{<ek'asthc >ekklhs'iac}{\Dfootnote{\latintext \textit{singularum ecclesiarum} Hil. \textit{cuiusque \dots\ ecclesiae} Cod.\,Ver.}}
\edtext{to`uc lao`uc}{\Dfootnote{\latintext \textit{plebes} Hil. \textit{populi}
Cod.\,Ver.}}
\edtext{to~u >id'iou >episk'opou}{\Dfootnote{\latintext \textit{sacerdotes sui}
Hil. \textit{sui episcopi} Cod.\,Ver.}} t`hn kajar'othta ka`i
\edtext{to~uton}{\Dfootnote{\latintext \textit{se} Hil. \textit{hunc}
Cod.\,Ver.}}
\edtext{\abb{m`en}}{\Dfootnote{\latintext > Hil. Cod.\,Ver.}} >'eqein
\edtext{\abb{>ep'iskopon}}{\Dfootnote{\latintext Thdt.(A\mg) + \textit{suum}
Hil.}}
\edtext{\abb{
\edtext{ka`i}{\lemma{\abb{ka`i\ts{2}}}\Dfootnote{\latintext > Cod.\,Ver.}}
prosdok~an}}{\Dfootnote{\latintext > Hil. \textit{expectare} Cod.\,Ver.}},
\edtext{\edtext{to`uc d`e e>ic t`ac >ekklhs'iac a>ut~wn
\edtext{>epelj'ontac}{\Dfootnote{>epanelj'ontac \latintext Thdt.(BA)
Ath.(E)}}}{\lemma{to`uc d`e \dots\ >epelj'ontac} \Dfootnote{\latintext
\textit{illos, autem, qui se eorum ecclesiis inmerserunt} Hil. \textit{adgressos
vero ecclesiarum ipsorum} Cod.\,Ver.}}
\edtext{d'ikhn}{\Dfootnote{\latintext \textit{ore} Hil. \textit{iudicium}
Cod.\,Ver.}} l'ukwn}{\lemma{\abb{to`uc \dots\ l'ukwn}}\Afootnote{\latintext vgl. Act 20,29; Mt
7,15}}\edindex[bibel]{Apostelgeschichte!20,29|textit}\edindex[bibel]{Matthaeus!7,15|textit},
\edtext{\abb{Grhg'orion}}{\Dfootnote{\latintext Ath.(R\mg) \textit{id est
Gregorium} Hil. \textit{Georgium} Cod.\,Ver.}}\edindex[namen]{Gregor!Bischof von Alexandrien} t`on
\edtext{\edtext{\abb{>en}}{\Dfootnote{\latintext > Thdt.(T)}}
\edtext{>Alexandre'ia|}{\Dfootnote{>Alexandre'a \latintext
Thdt.(T)}}}{\Dfootnote{\latintext ]\textit{in Alexandria} Hil.
\textit{Alexandriae} Cod.\,Ver.}},
Bas'ileion\edindex[namen]{Basilius!Bischof von Ancyra}
\edtext{\abb{t`on >en >Agk'ura|}}{\Dfootnote{\latintext > Cod.\,Ver.}}, ka`i
\edtext{\abb{Kuintian`on}}{\Dfootnote{\latintext Ath. \greektext Kuntian`on
\latintext Thdt.(Brz) \greektext Kuhtian`on \latintext Thdt.(A) \greektext
Kwnst'antion \latintext Thdt.(T) \textit{Quincianum} Hil.
Cod.\,Ver.}}\edindex[namen]{Quintianus!Bischof von Gaza} t`on
\edtext{>en G'azh|}{\Dfootnote{\latintext \textit{in Gaza} Hil. \textit{Gazae}
Cod.\,Ver.}},
\edtext{\abb{to'utouc}}{\Dfootnote{\latintext > Hil. Cod.\,Ver.}} mhd`e
\edtext{\abb{>episk'opouc >onom'azein}}{\Dfootnote{\latintext ] \textit{nomen
habere episcopi} Hil. \textit{episcopos nominari} Cod.\,Ver. \greektext + mhd`e
Qristiano`uc \latintext Thdt. + \textit{nec Christianos penitus apellari}
Cod.\,Ver.}} mhd`e
\edtext{\abb{<'olwc}}{\Dfootnote{\latintext > Cod.\,Ver.}}
\edtext{koinwn'ian tin`a}{\Dfootnote{\latintext \textit{communionis} \dots\
\textit{participatum} Hil. \textit{aliquam} \dots\ \textit{communionem}
Cod.\,Ver.}}
\edtext{\edtext{pr`oc a>uto`uc}{\Dfootnote{\latintext \textit{eorum} Hil.
\textit{cum his} Cod.\,Ver.}} >'eqein
\edtext{mhd`e}{\Dfootnote{\latintext \textit{vel} Cod.\,Ver.}} d'eqesja'i
\edtext{tina}{\Dfootnote{\latintext \textit{ab aliquem} Hil. > Cod.\,Ver.}}}{\lemma{\abb{pr`oc
\dots\ tina}}\Dfootnote{\latintext > Thdt.(L)}}
\edtext{par'' a>ut~wn}{\Dfootnote{par`a to'utwn \latintext Ath. \textit{eorum} Hil. Cod.\,Ver.}}
gr'ammata
\edtext{m'hde}{\Dfootnote{m'hte \latintext Thdt. \textit{neque} Hil. \textit{vel} Cod.\,Ver.}}
gr'afein
\edtext{pr`oc a>uto'uc}{\Dfootnote{\latintext \textit{ad eos} Hil. \textit{ad
ipsos} Cod.\,Ver.}}.
\pend
\pstart
\kap{16}
\edtext{to`uc
\edtext{\abb{d`e}}{\Dfootnote{+ ge \latintext Ath.}} per`i}{\Dfootnote{]
\latintext \textit{illos autem, hoc est} Hil. \textit{cum suis} Cod.\,Ver.}}
\edtext{\abb{Je'odwron}}{\Dfootnote{+ t`on (\latintext > Thdt.(L)) \greektext
>ap`o <Hrakle'iac t~hc E>ur'wphc \latintext Thdt. + \textit{Heracleatum Europae}
Cod.\,Ver.}}\edindex[namen]{Theodorus!Bischof von Heraclea}
\edtext{\abb{ka`i}}{\Dfootnote{\latintext > Thdt.(BF) Hil. Cod.\,Ver.}}
\edtext{\abb{N'arkisson}}{\Dfootnote{\latintext > Thdt.(FT) Cod.\,Ver. \greektext
+ t`on >ap`o Nerwni'adoc t~hc Kilik'iac \latintext Thdt. + \textit{de Maroniae
Ciliciae} Cod.\,Ver.}}\edindex[namen]{Narcissus!Bischof von Neronias}
\edtext{\abb{ka`i}}{\Dfootnote{\latintext > Thdt.(BAF) Hil.}}
\edtext{\abb{\abb{>Ak'akion}}}{\Dfootnote{\latintext > Thdt.(sFT)
\textit{Achaicum} Hil. \textit{Accatium} Cod.\,Ver. \greektext + t`on >ap`o Kaisare'iac t~hc Palaist'inhc
\latintext Thdt. et post \greektext Palaist'inhc: + >Ak'akion \latintext Thdt.(s)
\greektext + N'arkisson (+ leg'omenon \latintext Thdt.(T)) \greektext ka`i
>Ak'akion \latintext Thdt.(FT), et ante \textit{Acatium}: + \textit{Cessareae Palestinae
Narcissimum et} Cod.\,Ver.}}\edindex[namen]{Acacius!Bischof von Caesarea}
\edtext{ka`i}{\lemma{\abb{ka`i\ts{2}}}\Dfootnote{\latintext > Hil. Cod.\,Ver.}}
\edtext{\abb{St'efanon}}{\Dfootnote{+ >ap`o >Antioqe'iac \latintext Thdt. +
\textit{Antiocensem} Cod.\,Ver.}}\edindex[namen]{Stephanus!Bischof von Antiochien}
\edtext{ka`i}{\lemma{\abb{ka`i\ts{3}}}\Dfootnote{\latintext > Hil. Cod.\,Ver.}}
\edtext{O>urs'akion}{\Dfootnote{O>urs'akion \latintext Ath.(B) + (+ \greektext
t`on \latintext Thdt.(B)) \greektext
>ap`o \latintext (\greektext Siggido'unou \latintext Thdt.(B) \greektext
Singido'unou \latintext Thdt.(T) \greektext Sigd'onou \latintext Thdt.(ANL) \greektext Mugd'onou \latintext
Thdt.(s) \greektext Mugdo'unoc (mu \latintext in ras.) Thdt.(F)) \greektext t~hc
Mus'iac \latintext Thdt. + \textit{Singidunensem Moesiae}
Cod.\,Ver.}}\edindex[namen]{Ursacius!Bischof von Singidunum}
\edtext{ka`i}{\lemma{\abb{ka`i\ts{1}}}\Dfootnote{\latintext > Thdt.(T)}}
\edtext{\abb{O>u'alenta}}{\Dfootnote{+ t`on >ap`o Murs~wn t~hc Pannon'iac
\latintext Thdt. + \textit{Mursensem Panoniae} Cod.\,Ver.}}\edindex[namen]{Valens!Bischof von Mursa}
\edtext{ka`i}{\lemma{\abb{ka`i\ts{2}}}\Dfootnote{\latintext > Hil. Cod.\,Ver.}}
\edtext{Mhn'ofanton}{\Dfootnote{\latintext \textit{Minophantum} Cod.\,Ver. \greektext + (+
t`on \latintext Thdt.(rzT))
\greektext >ap`o >Ef'esou \latintext Thdt. + \textit{Ephesium}
Cod.\,Ver.}}\edindex[namen]{Menophantus!Bischof von Ephesus}
\edtext{ka`i}{\lemma{\abb{ka`i\ts{3}}}\Dfootnote{\latintext > Cod.\,Ver.}}
\edtext{\abb{Ge'wrgion}}{\Dfootnote{+ t`on >ap`o Laodike'iac \latintext Thdt. +
\textit{Laudeciae} Cod.\,Ver.}}\edindex[namen]{Georg!Bischof von Laodicea},
\edtext{e>i ka`i}{\Dfootnote{\latintext \textit{licet} Cod.\,Ver. \textit{etsi} Hil.}}
\edtext{\abb{fobhje`ic m`h pareg'eneto}}{\Dfootnote{\latintext Ath. Hil.
Cod.\,Ver. \greektext fobhj'entec m`h pareg'ento \latintext (ras. inter \greektext n \latintext et \greektext t) \latintext ex \greektext
fobhje`ic m`h pareg'eneto \latintext Thdt.(A\corr) \greektext fobhj'entec m`h
pareg'enonto \latintext Thdt.(BrzT) + \textit{ut dictum est} Hil.}}
\edtext{\abb{>ap`o t~hc <E'w|ac}}{\Dfootnote{\latintext > Hil.}},
\edtext{<'omwc}{\Dfootnote{\latintext \textit{sed tamen} Hil. \textit{tamen}
Cod.\,Ver.}}
\edtext{di`a t`o}{\Dfootnote{\latintext \textit{quia} Hil. \textit{eo quod} Cod.\,Ver.}} >ap`o to~u
\edtext{makar'iou}{\Dfootnote{makar'itou \latintext Ath. \textit{aetat\c{e}
memoriae} Hil. \textit{beato} Cod.\,Ver.}} >Alex'androu\edindex[namen]{Alexander!Bischof von Alexandrien}
\edtext{\abb{to~u genom'enou >episk'opou >Alexandre'iac}}{\Dfootnote{\latintext
> Ath. \textit{episcopo quondam Alexandriae} Hil. \textit{episcopo Alexandriae}
Cod.\,Ver.}}
\edtext{kajh|r~hsjai}{\Dfootnote{kajairej~hnai \latintext Ath.(B)}}
\edtext{a>ut`on}{\Dfootnote{a>uto`uc \latintext Thdt.(A\corr BrzT)}}
\edtext{\abb{ka`i}}{\Dfootnote{\latintext > Cod.\,Ver.}}
\edtext{di`a t`o e~>inai ka`i a>ut`on ka`i to'utouc t~hc >Are'iou
man'iac}{\lemma{di`a t`o \dots\ man'iac} \Dfootnote{di`a t`o ka`i to'utouc
(to`uc \latintext Thdt.(BrL) \greektext a>uto`uc \latintext Thdt.(FT))
\greektext s`un a>ut~w| (a>uto~ic \latintext Thdt.(A\corr rLT)) \greektext t~hc
>Are'iou man'iac e~>inai \latintext Thdt. \textit{quod sit tam ille qui etiam
ceteri, qui comprehensi sunt, furoris Arriani} Hil. \textit{propter se ac suos
collegas Arrianos} Cod.\,Ver.}}\edindex[namen]{Arianer}
ka`i di`a t`a
\edtext{>epeneqj'enta}{\Dfootnote{\latintext \textit{obiecta} Hil. \textit{inlata} Cod.\,Ver.}}
\edtext{a>uto~ic}{\Dfootnote{a>ut~w| \latintext Thdt.(BF) et \greektext ~w|
\latintext in ras. Thdt(A\corr) \textit{eos} Cod.\,Ver. > Hil.}} >egkl'hmata,
\edtext{to'utouc}{\Dfootnote{\latintext \textit{hos omnes} Hil. \textit{omni}
Cod.\,Ver.}}
\edtext{pamyhfe`i}{\Dfootnote{pamyhf`i \latintext Ath. \textit{aequali sorte}
Hil. > Cod.\,Ver.}} kaje~ilen <h <ag'ia s'unodoc
\edtext{>ap`o t~hc >episkop~hc}{\Dfootnote{\latintext \textit{de episcopatus
gradu} Hil. \textit{episcopatu} Cod.\,Ver.}}; ka`i
\edtext{>ekr'inamen}{\Dfootnote{\latintext \textit{iudicavimus} Hil. \textit{statuimus} Cod.\,Ver.}} m`h m'onon
\edtext{\edtext{a>uto`uc}{\Dfootnote{\latintext \textit{illos} Hil. \textit{eos}
Cod.\,Ver.}} >episk'opouc}{\lemma{\abb{}}\Dfootnote{\responsio\ >episk'opouc
a>uto`uc \latintext  Thdt.(BrzT)}} m`h e>~inai, >all`a
\edtext{mhd`e}{\Dfootnote{\latintext \textit{neque} \dots \textit{quidem} Hil.
\textit{nec} Cod.\,Ver.}}
\edtext{koinwn'iac}{\Dfootnote{\latintext cummunionem Hil.}} met`a t~wn pist~wn
\edtext{a>uto`uc}{\Dfootnote{a>uto~u \latintext Thdt.(T) > Hil. Cod.\,Ver.}}
\edtext{kataxio~usjai}{\Dfootnote{\latintext \textit{habere} Hil.
\textit{mereri} Cod.\,Ver.}}. to`uc
\edtext{g`ar}{\Dfootnote{d`e \latintext Ath.(B)}}
\edtext{\abb{qwr'izontac}}{\Dfootnote{+ t~hc to~u patr`oc o>us'iac ka`i
je'othtoc \latintext Thdt. + \textit{ad substantiam patris hac de et qualitate}
Cod.\,Ver.}}
\edtext{\abb{t`on u<i`on}}{\Dfootnote{\latintext Ath. Thdt.(A) \responsio\ \greektext t`on u<i`on \latintext ante
\greektext t~hc to~u \latintext  Thdt.(L) > Thdt.(BrFT)}}
\edtext{\abb{ka`i}}{\Dfootnote{\latintext > Cod.\,Ver.}}
\edtext{>apallotrio~untac}{\Dfootnote{>apallotri'wtac \latintext Ath.(O*)
\greektext allotrio~untac \latintext Thdt.(z)}} t`on l'ogon >ap`o to~u patr`oc
qwr'izesjai
\edtext{\abb{>ap`o}}{\Dfootnote{\latintext > Ath. Cod.\,Ver.}} t~hc
\edtext{kajolik~hc >ekklhs'iac}{\Dfootnote{>ekklhsiastik~hc e>utax'iac
\latintext Thdt.(T)}} pros'hkei ka`i
\edtext{>allotr'iouc e>~inai}{\Dfootnote{\latintext \textit{alienos esse a} Hil. \textit{alienari} Cod.\,Ver.}} to~u
\edtext{\abb{Qristian~wn}}{\Dfootnote{\latintext Ath. Thdt.(A) \greektext
Qristiano~u \latintext Thdt.(Brz) \greektext Qristianiko~u \latintext Thdt.(T)
\textit{Christianorum} Cod.\,Ver. \textit{Christiano} Hil.}}
\edtext{>on'omatoc}{\Dfootnote{\latintext \textit{nomine} Hil. \textit{nomina} Cod.\,Ver.}}. >'estwsan
to'inun
\edtext{\abb{<hm~in}}{\Dfootnote{\latintext Thdt.(rF) \textit{nobis} Hil. \greektext ka`i <hm~in \latintext Thdt.(BT) \greektext <um~in \latintext Ath. Thdt.(LA\corr) \greektext + ka`i p~asin \latintext Thdt.
\textit{omnibus vobis} Cod.\,Ver.}} >an'ajema,
\edtext{di`a}{\lemma{di`a t`o} \Dfootnote{\latintext \textit{eo quod} Hil. \textit{propter quod} Cod.\,Ver.}}
\edtext{t`o
\edtext{kekaphleuk'enai}{\Dfootnote{\latintext \textit{ausi sunt
adulterare} Hil. \textit{negotiati sunt} Cod.\,Ver.}}
\edtext{\abb{a>uto`uc}}{\Dfootnote{\latintext > Ath.}} t`on
l'ogon}{\lemma{\abb{t`o \dots l'ogon}}\Afootnote{\latintext 2Cor 2,17}}\edindex[bibel]{Korinther II!2,17|textit} t~hc
>alhje'iac.
\pend
\pstart
\kap{17}\edtext{\abb{>apostolik'on}}{\Dfootnote{+ g`ar \latintext Thdt. Cod.\,Ver.
\textit{apostoli cuius} Hil. \textit{apostolica} Cod.\,Ver.}}
\edtext{\abb{>esti}}{\Dfootnote{+ t`o \latintext Ath.(B)}} par'aggelma;
\edtext{((e>'i
tic <um~ac
\edtext{e>uaggel'izetai}{\Dfootnote{\latintext \textit{evangelizaverit} Hil.
\textit{annuntiaverit} Cod.\,Ver.}} par'' <`o parel'abete, >an'ajema
>'estw))}{\lemma{\abb{}}\Afootnote{\latintext Gal
1,9}}\edindex[bibel]{Galater!1,9}.
to'utoic mhd'ena koinwne~in
\edtext{paragge'ilate}{\Dfootnote{\latintext \textit{denuntiamus} Hil.
\textit{praecipite} Cod.\,Ver.}}; ((o>udem'ia g`ar
\edtext{koinwn'ia}{\Dfootnote{\latintext \textit{communicatio} Hil. \textit{communio} Cod.\,Ver.}}
\edtext{fwt`i pr`oc sk'otoc}{\Dfootnote{\latintext \textit{luci ad tenebras}
Hil. \textit{tenebris cum luce} Cod.\,Ver.}}.))
\edtext{\abb{to'utouc p'antac}}{\Dfootnote{\latintext Thdt.(ANF)
\textit{homines} Hil. \textit{omnes} Cod.\,Ver. \greektext to'utwn p'antac
\latintext Ath.(RE) Thdt.(s) \greektext to'utwn p'antec \latintext Ath.(BKO) \greektext to'utouc p'antwn \latintext Thdt.(BT)
\greektext to'utouc \latintext Thdt.(L)}} makr`an poie~ite;
\edtext{((\edtext{o>udem'ia g`ar sumfwn'ia}{\Dfootnote{\latintext \textit{nulla enim
participatio} Hil. \textit{nec enim est aliqua conventio} Cod.\,Ver.}} Qrist~w|
pr`oc
\edtext{Bel'iar}{\Dfootnote{\latintext \textit{Belial} Hil. \textit{Beliab}
Cod.\,Ver.}}))}{\lemma{\abb{}}\Afootnote{\latintext 2Cor
6,14~f.}}\edindex[bibel]{Korinther II!6,14~f.}.
\edtext{\abb{ka`i}}{\Dfootnote{\latintext > Hil.}}
\edtext{ful'axasje}{\Dfootnote{\latintext \textit{observate} Hil. Cod.\,Ver.}},
\edtext{>agaphto'i}{\Dfootnote{>adelfo`i >agaphto'i \latintext Thdt.
\textit{carissimi} Hil. \textit{fratres dilecti} Cod.\,Ver.}},
\edtext{m'hte gr'afein
\edtext{pr`oc a>uto`uc}{\Dfootnote{a>uto~ic \latintext Thdt.(s)}} m'hte
gr'ammata par'' a>ut~wn
\edtext{d'eqesjai}{\Dfootnote{d'eqesje \latintext Ath.(O)}}}{\lemma{m'hte
gr'afein \dots\ d'eqesjai} \Dfootnote{\latintext \textit{ut neque scribatis ad
eos neque eorum accipiatis litteras} Hil. \textit{eorum litteras suscipere vel
hiis scribere} Cod.\,Ver.}}.
\edtext{spoud'asate}{\Dfootnote{\latintext \textit{curate} Hil. \textit{studete}
Cod.\,Ver.}}
\edtext{\abb{d`e}}{\Dfootnote{\latintext > Thdt.(sT) Cod.\,Ver. \greektext +
m~allon \latintext Ath.}}
\edtext{\abb{ka`i}}{\Dfootnote{\latintext > Thdt.(N)}} <ume~ic,
\edtext{\abb{>agaphto`i}}{\Dfootnote{\latintext > Ath. \textit{dilectissimi}
Hil. \textit{dilecti} Cod.\,Ver.}} >adelfo`i ka`i
\edtext{sulleitourgo'i}{\Dfootnote{\latintext \textit{consacerdotes} Hil. \textit{conministri} Cod.\,Ver.}},
\edtext{<wc}{\Dfootnote{\latintext \textit{quasi} Hil. > Cod.\,Ver.}}
\edtext{\edtext{t~w| pne'umati par'ontec}{\Dfootnote{t~w| pne'umati sun'ontec
\latintext
Ath.(B) \textit{praesente spiritu} \dots\ \textit{interfueritis} Hil.
\textit{spiritu praesentes} Cod.\,Ver.}}}{\lemma{\abb{t~w| \dots\ par'ontec}}\Afootnote{\latintext
vgl. 1Cor 5,3; Col 2,5}}\edindex[bibel]{Korinther I!5,3|textit}\edindex[bibel]{Kolosser!2,5|textit}
\edtext{t~h| sun'odw| <hm~wn}{\Dfootnote{\latintext \textit{huic} \dots\
\textit{synodo} Hil. \textit{in synodo nostra} Cod.\,Ver.}}
\edtext{\abb{sunepiyhf'isasjai}}{\Dfootnote{\latintext Ath.(BKOE\corr)
\textit{confirmare} Hil., qui + \textit{omnia, quae a
nobis instatuta sunt} ante \textit{confirmare} \greektext suneyhf'isasjai \latintext Ath.(RE*)
\greektext sunain'esai ka`i yhf'isasjai \latintext Thdt.(BArz)
\textit{consentire et pronuntiare} Cod.\,Ver. \greektext sunain'esai ka`i
sumyhf'hsasjai \latintext Thdt.(T)}}
\edtext{di'' <upograf~hc
\edtext{<umet'erac}{\Dfootnote{<hmet'erac \latintext Ath.(O*)}}}{\Dfootnote{\latintext \textit{per litteras vestras} Hil. \textit{prescriptione vestra} Cod.\,Ver.}},
\edtext{<up`er to~u}{\Dfootnote{\latintext \textit{ut} Hil. \textit{quo}
Cod.\,Ver.}} par`a p'antwn t~wn
\edtext{\abb{pantaqo~u}}{\Dfootnote{\latintext Ath. Thdt.(BrzT) \greektext
<apantaqo~u \latintext Thdt.(A) \textit{ubique} Cod.\,Ver. > Hil.}}
\edtext{\abb{sulleitourg~wn}}{\Dfootnote{\latintext \textit{episcopis} Hil. \textit{conministros} Cod.\,Ver. \greektext + <hm~wn \latintext Thdt. +
\textit{nostros} Cod.\,Ver.}}
\edtext{<omofwn'ian}{\Dfootnote{t`hn <omofwn'ian \latintext Ath. \textit{idem}
Hil. \textit{adsensus} Cod.\,Ver.}}
\edtext{dias'wzesjai}{\Dfootnote{\latintext \textit{servetur} Cod.\,Ver.
\textit{sentire} Hil. + \greektext <h je'ia pr'onoia kajwsiwm'enouc <um~ac ka`i
e>ujumo~untac diaful'attoi, >agaphto`i >adelfo`i. <'Osioc >ep'iskopoc
<up'egraya (>'egraye \latintext Ath.(E)) \greektext , ka`i o<'utw p'antec. \latintext Ath. + \textit{adque unam esse
omnium voluntatem ex litterarum consensione sit manifestum. optamus, fratres,
vos in domino bene valere} Hil.}}.
\pend
% \endnumbering
\end{Leftside}
\begin{Rightside}
\begin{translatio}
\beginnumbering
\pstart
\noindent\kapR{1}Die heilige Synode, die sich durch die Gnade Gottes
% [aus Rom, den spanischen und gallischen Provinzen, den Provinzen Italien\LitNil,
% Kampanien, Kalabrien, Afrika, Sardinien, Pannonien, Moesien, Dakien, Dardanien, der
% anderen\LitNil dakischen Provinz, den Provinzen Makedonien, Thessalien, Achaia, den
% epirischen Provinzen\LitNil, den Provinzen Thrakien, Rhodope, Asien, Karien, Bithynien,
% Hellespont, Phrygien, Pisidien, Kappadokien, Pontos, Kilikien, der anderen phrygischen
% Provinz\LitNil, Pamphylien, Lydien, den Kykladen, den Provinzen �gypten, Thebais,
%Libyen,
% Galatien Pal�stina und Arabien]\footnoteA{Dieses Synodalschreiben wurde sicher zun�chst
% ohne diese Auf"|listung der Provinzen verschickt, wie es auch Athanasius �berliefert
% (Hilarius l��t dieses Pr�skript aus); stattdessen folgten im Anschlu� entsprechende
% Unterschriften (wie in Dok. ; \ref{sec:B}; \ref{sec:BriefSerdikaMareotis};
% \ref{sec:BriefAthMareotis}). Erst sp�ter wurden diese Unterschriften durch eine
% zusammenfassende Provinzliste ersetzt (so die �berlieferung bei Theodoret und im Codex
% Veronensis LX; auch in Dok. \ref{sec:BriefSerdikaAlexandrien}). Zu beachten ist, da�
% Theodoret �ber den Codex Veronensis LX hinausgehend einen Block von kleinasiatischen
% Provinzen nennt (Asien ... Kykladen), von denen (bis auf Asien und den Kykladen) keine
% Unterschriften �berliefert sind. Entweder wurde hier versehentlich eine
% Unterschriftenliste der Teilnehmer der �stlichen Teilsynode eingebaut, oder aber, da
%hier
% nicht alle Provinzen der Teilnehmer der �stlichen Teilsynode abgedeckt werden, Theodoret
% konnte auf eine andere Unterschriftenliste zur�ckgreifen, auf der auch noch weitere
% kleinasiatische Bisch�fe der Erkl�rung der westlichen Teilsynode zugestimmt haben.}
in Serdica versammelt hat, gr��t im
Herrn die Bisch�fe allerorts und Mitdiener der katholischen Kirche, die geliebten Br�der.
\pend
\pstart
\kapR{2}Die von Arius Besessenen begehrten vielf�ltig und oft
gegen die Diener Gottes, die den rechten Glauben bewahren, auf. Sie versuchten
n�mlich, die Rechtgl�ubigen zu vertreiben, indem sie heimlich eine falsche Lehre
verbreiteten. So massiv war schlie�lich ihr Widerstand gegen den Glauben, da�
er nicht einmal der Gottesfurcht der gottesf�rchtigsten Kaiser verborgen blieb. Mit Hilfe der Gnade Gottes
jedoch f�hrten uns eigens die gottesf�rchtigsten Kaiser\footnoteA{Vgl. Socr., h.\,e. II 20,1--3; Soz., h.\,e. III 11,3; auch Dok. \ref{sec:B},3 (Tagesordnung durch die Kaiser) und Ath., h.\,Ar. 15,2; 44,2; apol.\,sec. 36,1~f. und ind.\,ep.\,fest. 15.} aus verschiedenen Provinzen
und St�dten zusammen und sorgten daf�r, da� diese heilige Synode in Serdica
stattfand, damit alle Zwietracht entfernt, aller falscher
Glaube ausgemerzt und von allen nur die Verehrung Christi bewahrt werde.
\pend
\pstart
\kapR{3}Es kamen also auch die Bisch�fe aus dem Osten, die ebenfalls
auf Befehl der gottesf�rchtigsten Kaiser einberufen worden waren, vor allem aufgrund der Ger�chte, die sie so h�ufig �ber
unsere geliebten Br�der und Mitdiener Athanasius, den Bischof von Alexandrien,
und Markell, den Bischof von Ancyra in Galatia, verbreitet hatten.\footnoteA{Bez�glich Athanasius wurde vor allem das kirchenrechtliche Problem diskutiert, ob seine Absetzung auf der Synode in Tyrus rechtens gewesen sei (vgl. bes. Dok. \ref{sec:BerichteAntiochien341}; \ref{sec:BriefSynode341}; \ref{sec:BriefJuliusII}), bez�glich Markell dagegen theologische Fragen (vgl. bes. Dok. \ref{ch:Konstantinopel336}; \ref{sec:MarkellJulius}; \ref{sec:BriefJuliusII} und dieses Dokument). Die zus�tzliche Erw�hnung des Asklepas von Gaza bei Theodoret und im Cod.\,Ver. d�rfte eine sp�tere Erg�nzung sein.}
Vielleicht haben ihre Verleumdungen ja auch euch erreicht,
vielleicht haben sie versucht, auch eure Ohren zu ersch�ttern, damit ihr
glaubt, was sie gegen die Rechtschaffenen sagen, sie aber jeden Verdacht gegen ihre
�ble H�resie �bert�nchen k�nnen.
Es wurde ihnen aber nicht gestattet, dies im gro�en Ma�stab umzusetzen. Der Herr ist
es n�mlich, der die Kirche leitet, der f�r diese und uns alle den Tod erlitten
und uns allen durch sich selbst den Weg zum Himmel bereitet hat.
\pend
\pstart
\kapR{4}Nachdem also vor einiger Zeit Eusebius, Maris, Theodorus, Theognis,
Ursacius, Valens, Menophantus und Stephanus unserem
Mitdiener Julius, dem Bischof der Kirche von Rom, gegen unsere zuvor genannten
Mitdiener geschrieben hatten,\footnoteA{Es
kann nicht einwandfrei bestimmt werden, auf welchen Brief hier Bezug genommen
wird; da aber auf l�nger zur�ckliegende Ereignisse zur�ckgegriffen wird, ist wohl
nicht Dok. \ref{sec:BriefSynode341}, auch nicht der in Dok. \ref{sec:BriefJuliusII},8 erw�hnte Brief gemeint, sondern
noch fr�here Briefe (vgl. Ath., h.\,Ar. 9,1; apol.\,sec. 19,3; 20,1; 22,3~f.; 83,4; Socr.,
h.\,e. II 17,4 und Einleitung zu Dok. \ref{sec:BriefJulius}). Die Namen sind bei Athanasius offenbar nachtr�glich
gek�rzt zu der bei ihm �blichen Kurzform ">die um Eusebius"<; bei Hil. fehlen die beiden letzten Namen
Menophantus und Stephanus. Die Herkunft gerade dieser Gruppennennung ist unklar; sie
entspricht nicht der Mareotis-Kommission (vgl. z.B. Ath., apol.\,sec. 73,1), die sich ja
auch nur mit Athanasius, nicht Markell, befa�t hatte, k�nnte aber eventuell
einem Briefwechsel nach der Konstantinopler Synode gegen Markell entnommen sein
(vgl. Dok. \ref{ch:Konstantinopel336}). In Vergleich zu den aktuellen Gegnern, in � 14 und
16 aufgez�hlt, fehlen hier
Narcissus, Acacius und Georg.}~-- wir meinen Athanasius und Markell~-- schrieben auch die Bisch�fe der anderen
Regionen\footnoteA{Vgl. Dok. \ref{sec:BriefJuliusII},14; 27; vgl. auch den langen Brief in Ath., apol.\,sec. 3--20.} und bezeugten die Unschuld
unseres Mitdieners Athanasius sowie da� die Aussagen von denen um
Eusebius nichts anderes seien als L�ge und reine Verleumdung. Auch wenn
sich schon aus der Tatsache, da� sie nicht (zur Synode in Rom) erschienen, obwohl sie von unserem geliebten Mitdiener Julius\footnoteA{Vgl. Dok. \ref{sec:BriefJulius}.} eingeladen worden waren,
und sich aus dem Schreiben des Julius selbst\footnoteA{Vgl. Dok. \ref{sec:BriefJuliusII}.} ihre Verleumdung sehr deutlich
zu erkennen gibt~-- sie w�ren n�mlich
gekommen, wenn sie �berzeugt gewesen w�ren von den Ma�nahmen, die sie gegen unsere
Mitdiener ergriffen und durchgef�hrt haben~--, offenbarten sie ihr
Komplott jedoch am deutlichsten dadurch, wie sie auf dieser heiligen und gro�en Synode
auftraten.
\pend
\pstart
\kapR{5}Denn als sie nach Serdica kamen und unsere Br�der Athanasius, Markell,
Asklepas und die anderen sahen, f�rchteten sie sich, zur Verhandlung zu
erscheinen.
Und sie leisteten den Einladungen, die nicht nur einmal, nicht zweimal, sondern mehrmals
ausgesprochen wurden, nicht Folge, obwohl doch alle Bisch�fe, die wir zusammengekommen
waren, und vor allem der hochbetagte Ossius\footnoteA{Ossius (vgl. allg.
Ath., h.\,Ar. 42--45), seit ca. 300 Bischof von Cordoba, Confessor (Soz.,
h.\,e. I 10), Berater von Konstantin und von ihm nach Alexandrien geschickt zur
Beilegung des ">arianischen"< Streits (Soz., h.\,e. I 16,5; Socr., h.\,e. I 7,1; vgl.
Dok \ref{ch:18} = Urk. 18,1), f�hrende Pers�nlichkeit der ">westlichen"< Synode (vgl. Dok. \ref{sec:BriefOssiusProtogenes}, die Unterschriftenlisten und seinen Brief an Constantius in Ath., h.\,Ar. 44);
unterschrieb 357 die Erkl�rung der Synode von Sirmium (h.\,Ar. 45; Soz., h.\,e. IV
6,13; 12,6~f.).}, der wegen seines Alters, seines Bekenntnisses und weil er so
gro�es Leid ertragen hatte, allen Respekt verdient gehabt h�tte, warteten und
sie dr�ngten, zur Verhandlung zu kommen, damit sie dies, was sie in
Abwesenheit unserer Mitdiener in Umlauf gebracht und gegen sie geschrieben hatten, in Anwesenheit  beweisen k�nnten. Aber obwohl sie eingeladen worden waren,
kamen sie
nicht, wie wir gerade zuvor sagten. Auch damit zeigten sie ihre Verleumdung und
durch ihre Ausrede schrien sie beinahe die Nachstellung und das Komplott,
das sie geschmiedet hatten, lauthals heraus. Denn die, die �berzeugt sind von dem,
was sie sagen, k�nnen dem auch von Angesicht zu Angesicht standhalten. Da sie
jedoch nicht erschienen sind, kann unserer Meinung nach keiner mehr daran zweifeln~-- auch wenn
jene nochmals intrigieren wollten~--, da� sie keinen Beweis gegen unsere
Mitdiener haben und sie diese, wenn sie nicht da sind, wohl verleumden, deren Gegenwart aber meiden.
\pend
\pstart

\pend
\pstart
\kapR{6}Sie flohen n�mlich, geliebte Br�der, nicht nur, weil sie diese verleumdet hatten, sondern weil sie bemerkten,
da� auch die gekommen waren, die sie mit verschiedenen Anklagen belasten w�rden.
Denn Fesseln und Eisen wurden vorgezeigt, Menschen waren da, die aus der Verbannung
zur�ckgekehrt waren, und Diener, die gekommen waren f�r die, die sich noch in der Verbannung befanden,
und es waren Verwandte und Freunde derer anwesend, die durch jene umgekommen
waren. Und was das Schlimmste ist: Es waren Bisch�fe anwesend, von denen einer die
Eisen und Ketten vorzeigte, die er ihretwegen tragen mu�te, die �brigen
ihre Verurteilung zum Tod aufgrund deren Beschuldigung bezeugten.
Sie waren n�mlich in solchem Ausma� dem Wahnsinn verfallen, da� sie versucht hatten, sogar Bisch�fe
umzubringen, und sie h�tten diese auch umgebracht, wenn sie nicht ihren H�nden
entflohen w�ren. Es starb jedenfalls unser Mitdiener, der
selige Theodul\footnoteA{Theodul von Traianopolis in Thrakien; vgl. Ath., fug. 3,4;
h.\,Ar. 19,2.}, als er vor ihrer Beschuldigung floh. Infolge ihrer Verleumdung war
n�mlich der Befehl ergangen, ihn zu t�ten. Manche zeigten Schwerthiebe, wieder andere beklagten, da� sie ihretwegen Hunger erdulden
mu�ten. Dies bezeugten nicht irgendwelche Leute, sondern es waren
ganze Kirchen, deren Gesandtschaften und Delegationen uns stellvertretend
von mit Schwertern bewaffneten Soldaten berichteten, sowie von Horden mit
Keulen, von Drohungen der Richter und vom Unterschieben gef�lschter
Briefe~-- Es wurden n�mlich Briefe vorgelesen, die im Umkreis des Theognis
gef�lscht worden\footnoteA{Theognis von Niz�a; vgl. Dok. \ref{ch:TheognisNiz}; \ref{ch:31} = Urk. 31; Mitglied auch der Mareotis-Kommission (vgl. Dok. \ref{sec:BriefJuliusII},32 und Ath., apol.\,sec. 13,2; 73,1; 77); er verstarb aber wohl vor der Synode von Serdica, da er nicht mehr zu den von der
westlichen Teilsynode Exkommunizierten geh�rt. Um welche Briefe es sich
handelt, kann nicht mehr bestimmt werden (vgl. allg. Socr., h.\,e. I 27,7; Soz.,
h.\,e. II 22,1).} und gegen unsere
Mitdiener Athanasius, Markell und Asclepas gerichtet waren, um auch die
Kaiser gegen sie aufzubringen. Dies enth�llten die, die damals Diakone des
Theognis waren~--, dar�ber hinaus von entbl��ten Jungfrauen,
von verbrannten Kirchen und von Haftstrafen gegen unsere Mitdiener,
und dies alles wegen nichts anderem als wegen der verfluchten
H�resie der Ariusbesessenen. Denn die, die sich die Gemeinschaft mit diesen
verbeten hatten, wurden gezwungen, derartiges zu ertragen.
\pend
\pstart

\pend
\pstart

\pend
\pstart
\kapR{7}Als sie dies einsahen, hatten sie keine Wahl mehr.
Denn sie sch�mten sich zwar zuzugeben, was sie getan hatten, da sie aber keine M�glichkeit mehr hatten,
diese Taten noch weiterhin zu verbergen, kamen sie nach Serdica, um durch ihr
Eintreffen den Anschein zu erwecken, als h�tten sie sich nichts zu Schulden kommen lassen, und so den Verdacht zu entkr�ften.
\pend
\pstart
\kapR{8}Als sie nun jene als Ankl�ger erblickten, die von ihnen verleumdet worden waren und
durch sie gelitten hatten, und die Beweise vor Augen hatten,
waren sie trotz Einladung nicht in der Lage zu kommen, zumal unsere Mitdiener
Athanasius, Markell und Asklepas mit sehr gro�er Selbstsicherheit auftraten und sich beschwerten, mit
Nachdruck darauf pochten, von ihnen verlangten und sie dazu aufriefen, nicht nur ihre Verleumdung zu beweisen, sondern
auch aufzudecken, wie sehr sie sich gegen ihre Kirchen vergangen hatten. Die
aber wurden von so gro�en Gewissensbissen niedergezwungen, da� sie flohen, 
durch ihre Flucht ihre Verleumdung best�tigten und durch ihr Davonlaufen zugaben, was
sie verbrochen hatten.
\pend
\pstart
\kapR{9}Obgleich ihre Niedertracht und Verleumdung sich nicht nur bei fr�heren,
sondern gerade auch bei diesen jetzigen Ereignissen zeigt,
wollten wir dennoch gemessen an der Wahrheit das, was von ihnen
zur Schau gestellt worden war, pr�fen, damit sie nicht ihre Flucht als einen
Vorwand zu weiteren �beltaten mi�brauchen k�nnen. Aber auch unter diesen Voraussetzungen haben wir
aus ihren Taten nur herausgefunden, da� sie Verleumder sind und
nichts anderes getan haben, als unseren Mitdienern mit Hinterlist zu begegnen.
\pend
\pstart
\kapR{10}Denn Arsenius\footnoteA{Vgl. Dok. \ref{sec:BriefJuliusII},28 mit Anm.}, von dem sie
behaupteten, da� er von Athanasius get�tet worden sei, lebt und kann
unter den Lebenden ausfindig gemacht werden. Von daher ist ersichtlich, da� auch die Berichte,
die von ihnen �ber andere in die Welt gesetzt wurden, Erfindungen sind.
Nachdem sie auch von einem Kelch schwatzten, der angeblich von Macarius\footnoteA{Gemeint ist die ">Ischyras-Aff�re"<; vgl. Dok. \ref{sec:BriefJuliusII},33 mit Anm.}, dem Presbyter
des Athanasius, zerbrochen wurde, bezeugten die, die aus Alexandrien und den Orten der Mareotis gekommen
waren, da� nichts von diesen Dingen geschehen sei. Auch die Bisch�fe, die aus
�gypten an unseren Mitdiener Iulius geschrieben hatten,\footnoteA{Brief in Ath., apol.\,sec.
3--20.} best�tigten in ausreichender Weise, da� absolut kein solcher
Verdacht dort aufgekommen sei.
Au�erdem sind die Aufzeichnungen, die sie behaupten, gegen ihn zu haben,
einseitig parteiisch zusammengestellt. Nach diesen Berichten
wurden sogar Heiden und Katechumenen befragt: Von denen sagte ein
Katechumene, als er gefragt wurde, er sei drinnen (in der Kirche) gewesen,
als Macarius vor Ort eintraf. Und ein anderer erkl�rte, als er gefragt wurde, da� der von
ihnen so oft erw�hnte Ischyras krank in seiner Zelle gelegen habe, so da� sich auch von
daher zeigte, da� �berhaupt keine Mysterien gefeiert wurden,
da die Katechumenen drinnen (in der Kirche) waren und Ischyras nicht anwesend war, sondern krank
darniederlag. Denn sogar dieser aller�belste Ischyras, der schon gelogen hatte,
als er behauptete, Athanasius habe einige der heiligen Schriften
verbrannt, aber demaskiert worden war, gestand, zu jenem Zeitpunkt, als Makarius zugegen war,
krank gewesen zu sein und darniedergelegen zu haben, so
da� er sich auch in dieser Sache als Verleumder zeigt. �brigens gaben sie dem
Ischyras selbst als Lohn f�r diese Verleumdung den Titel eines Bischofs, obwohl
er noch nicht einmal Presbyter war.\footnoteA{Vgl. Ath., apol.\,sec. 11,7;
12,1; 76.} Es waren n�mlich zwei Presbyter gekommen, die damals zu Melitius geh�rten,\footnoteA{Eine
Melitianer-Liste in Ath., apol.\,sec. 71,6; zu Melitius vgl. Dok. \ref{ch:23} = Urk. 23,5--12.} sp�ter aber
vom seligen Alexander, dem damaligen Bischof von Alexandrien,
aufgenommen worden waren und nun zu Athanasius hielten. Diese bezeugten, da� dieser niemals Presbyter des Melitius
gewesen sei, und da� Melitius �berhaupt keine Kirche oder einen Kirchendiener in der
Mareotis gehabt habe. Und dennoch brachten sie diesen Mann, der nicht einmal
Presbyter war, nun gleichsam als Bischof herbei, um, wie sie meinen, mit diesem Titel das
Publikum ihrer Verleumdungen zu beeindrucken.
\pend
\pstart
\kapR{11}Es wurden auch das Werk unseres Mitdieners Markell
vorgelesen\footnoteA{Vgl. Dok. \ref{ch:Konstantinopel336}; \ref{sec:MarkellJulius}.}
und die �blen Machenschaften der Anh�nger des Eusebius aufgedeckt. Was n�mlich Markell als Hypothese ausgesprochen hatte, dies verleumdeten sie als Bekenntnisaussage.
Es wurden jedenfalls die Stellen vor und nach diesen Hypothesen vorgelesen, und der Glaube
des Mannes wurde f�r recht befunden. Denn weder wies er dem Wort Gottes einen
Anfang aus der heiligen Maria zu,
wie sie behaupteten, noch schrieb er, da� seine Herrschaft ein Ende habe,
sondern da� auch dessen Herrschaft ohne Anfang und ohne Ende sei.
\kapR{12}Auch der Mitdiener Asklepas brachte Aufzeichnungen herbei, die in
Antiochia in Gegenwart seiner Ankl�ger und des Eusebius von Caesarea verfa�t
worden waren.\footnoteA{Vgl. Dok. \ref{sec:Asclepas}.} Und er konnte anhand
der Aussagen der urteilenden Bisch�fe aufzeigen, da� er unschuldig war.
\pend
\pstart
\kapR{13}Zu Recht also, geliebte Br�der, gehorchten sie nicht, obwohl sie
mehrmals eingeladen wurden, zu Recht flohen sie. Denn von ihrem Gewissen getrieben
best�tigten sie durch ihre Flucht ihre Verleumdungen und
haben damit erreicht, da� man dem Glauben schenkte, was die anwesenden
Ankl�ger gegen sie vortrugen und nachwiesen. Zus�tzlich zu all diesen Dingen nahmen sie
auch noch diejenigen, die vor langer Zeit wegen der H�resie des Arius abgesetzt und verbannt worden waren,
nicht nur auf, sondern versetzten sie sogar in eine h�here Position: Diakone
ins Presbyteramt, Presbyter ins Bischofsamt. Dies taten sie aus keinem anderen Grund, als um die
Gottlosigkeit auszustreuen und zu verbreiten und den rechten Glauben
zu vernichten.
\pend
\pstart
\kapR{14}Ihre Anf�hrer sind nach Eusebius und seinen Anh�ngern nun
aber Theodorus von Heraclea, Narcissus von Neronias in Cilicia, Stephanus von Antiochien,
Georg von Laodicea, Acacius von Caesarea in Palaestina, Menophantus von Ephesus
in Asia, Ursacius von Singidunum in Moesia und Valens von Mursa in
Pannonia.\footnoteA{Vgl. � 16 und die Liste der Verurteilten in Dok. \ref{sec:B},6.
Zu Theodorus von Heraclea vgl. Dok. \ref{sec:BriefJuliusII},32 Anm.;
zu Narcissus von Neronias vgl. Dok. \ref{ch:AntIV}, Einleitung; zu Georg von Laodicea und
Acacius von Caesarea vgl. Dok. \ref{sec:BerichteAntiochien341},2,5; zu Ursacius von Singidunum
und Valens von Mursa vgl. Dok. \ref{sec:BriefJuliusII},32;
\ref{sec:SerdicaWestBekenntnis},2; \ref{sec:B},4. Stephanus ist erst kurz vor der Synode von Serdica Bischof von Antiochien geworden, da auf der antiochenischen Synode 341 noch Flacillus als Bischof amtierte (vgl. Dok. \ref{sec:BerichteAntiochien341},1,6). Stephanus ist schon vorher einmal von Eustathius (vgl. Dok. \ref{sec:Eustathius}) aus Antiochien vertrieben worden (Ath., h.\,Ar. 4) und konnte sich nun auch als Bischof nicht lange halten (vgl. Dok. \ref{ch:Makrostichos}, Einleitung). Von Menophantus ist �berliefert, da� er sich auf der Synode von Niz�a 325 f�r Arius ausgesprochen hatte (Thdt., h.\,e. I 7,13), sp�ter aber wie auch Narcissus noch an einer antiochenischen Synode, die den Gegenbischof Georg f�r Alexandrien ernannte, beteiligt war (Soz., h.\,e. IV 8,4).}
Denn auch diese erlaubten denen, die mit ihnen aus dem Osten gekommen waren, weder zur heiligen Synode
zu kommen noch sich �berhaupt zu der Kirche Gottes dazuzugesellen.
Schon w�hrend ihrer Reise nach Serdica veranstalteten sie an verschiedenen Orten eigene Synoden und
trafen Vereinbarungen,\footnoteA{�ber Synoden auf dem Weg nach Serdica ist nichts bekannt, es ist aber davon auszugehen, da� die Zusammenstellung der �stlichen Delegation einige Vorbereitung und Absprachen bedurfte.} gespickt mit Drohungen, da� sie, wenn sie in Serdica ankommen, weder
irgendeine Gerichtsverhandlung besuchen noch mit der heiligen Synode am selben Ort zusammenkommen,
sondern nur kurz auftauchen, der Form halber die Ankunft kundtun und schnell wieder verschwinden wollten.
Dies konnten wir n�mlich von unseren Mitbr�dern
Makarius\footnoteA{Gemeint ist Arius von Petra. Hier liegt ein Wortspiel vor.} aus Palaestina und Asterius aus
Arabien\footnoteA{Zwei ">�berl�ufer"<, vgl. die Liste in der Einleitung.}
erfahren, die mit ihnen gekommen waren, sich dann aber von deren Unglauben lossagten.
Diese n�mlich kamen zur heiligen Synode und beklagten die Gewalt, die sie
erlitten hatten, und sagten, da� bei jenen Unrecht begangen worden sei.
Sie f�gten auch diese Information hinzu, da� auf jener Seite doch viele dabei seien,
die eigentlich nach dem rechten Glauben strebten, aber von ihnen gehindert w�rden, hierher zu kommen,
mittels Drohungen und Verw�nschungen gegen die, die sich von ihnen trennen wollten.
Aus diesem Grund beeilten sie sich, da� alle in einem Haus blieben, und erlaubten
ihnen nicht einmal f�r k�rzeste Zeit, f�r sich zu sein.
\pend
\pstart
\kapR{15}Da es also nicht m�glich war, die Verleumdungen,
die Verhaftungen, die Morde, die Schl�ge, die Intrigen durch die erfundenen Briefe,
die Mi�handlungen, die Entbl��ungen von Jungfrauen, die Vertreibungen, die Zerst�rungen von Kirchen, die
Brandstiftungen, die Versetzungen aus kleineren St�dten in gr��ere Kirchen,
und allem voran die verfluchte arianische H�resie, die durch sie gegen den
rechten Glauben erwuchs, zu verschweigen oder unkommentiert zu lassen, deswegen
haben wir unsere geliebten Br�der und Mitdiener Athanasius von
Alexandrien, Markell von Ancyra in Galatia, Asclepas von Gaza und die, die mit ihnen dem
Herrn dienen, f�r straffrei und unschuldig erkl�rt und auch an die Kirche
eines jeden geschrieben, da� das jeweilige Kirchenvolk die Unschuld seines
eigenen Bischofs anerkennen und diesen als Bischof ansehen und zur�ckerwarten
solle, die aber, die nach Art von W�lfen in ihre Kirchen gekommen seien,
Gregor in Alexandrien, Basilius in Ancyra und Quintianus in Gaza,\footnoteA{Gregor wurde
auf der antiochenischen Synode 338 an Stelle von Athanasius als Bischof von Alexandria 
eingesetzt (Socr., h.\,e. II 8; Index 339; vgl. Dok. \ref{sec:BriefJuliusII},41--45;
\ref{sec:BriefSerdikaAlexandrien},16; \ref{sec:BriefAthMareotis},4; \ref{sec:RundbriefSerdikaOst},1).
Basilius wurde anstelle von Markell Bischof von Ancyra (vgl. Dok. \ref{ch:Konstantinopel336}), bedeutsam
besonders in den 50er Jahren (Synode von Sirmium 351, Synode von
Rimini""/""Seleukia), Januar 360 aber abgesetzt (Socr., h.\,e. II 42; Soz., h.\,e. IV 24,3); zu den Unruhen bei
der R�ckkehr des Markell nach Ancyra um 344 vgl. Socr., h.\,e. II 26; Soz., h.\,e.
III 24,4. Quintianus war der Gegenbischof f�r Asclepas von Gaza
(vgl. Dok. \ref{sec:Asclepas}), ist sonst unbekannt.} weder als Bisch�fe zu bezeichnen
noch �berhaupt eine Form von Gemeinschaft mit ihnen zu halten oder irgendwelche Briefe von ihnen
entgegenzunehmen oder an sie zu schreiben.
\pend
\pstart
\kapR{16}Den Kreis um Theodor\footnoteA{S.\,o. � 14.},
Narcissus, Acacius, Stephanus, Ursacius, Valens, Menophantus und Georg, der zwar
aus Furcht nicht aus dem Osten hergereist, daf�r aber bereits vom
seligen Alexander, dem einstigen Bischof von Alexandrien, abgesetzt worden war,
setzte die heilige Synode einstimmig von ihrem Bischofsamt ab, zumal sowohl Georg als
auch diese dem arianischen Wahn verfallen waren, und wegen der gegen
sie vorliegenden Anklagen. Und wir entschieden nicht nur, da� sie keine Bisch�fe sind,
sondern da� sie auch keiner Gemeinschaft mit den Gl�ubigen wert sind. Denn es ist angemessen,
da� die, die den Sohn absondern und den Logos vom Vater entfremden, von der katholischen Kirche
abgesondert werden und dem christlichen Namen fremd sind. Sie
sollen daher f�r uns ausgeschlossen sein, weil sie das Wort der Wahrheit
verschachert haben.
\pend
\pstart
\kapR{17}Der apostolische Befehl lautet: ">Wenn euch einer ein anderes
Evangelium verk�ndet, als ihr empfangen habt, sei er ausgeschlossen."< Befehlt,
da� niemand mit diesen Leuten Gemeinschaft h�lt; es gibt n�mlich keine
Gemeinschaft des Lichtes mit der Finsternis. Diese alle haltet von euch fern;
es gibt n�mlich keine Gemeinschaft zwischen Christus und Beliar. Und gebt
acht, Geliebte, nicht an sie zu schreiben und keine Briefe von
ihnen anzunehmen. Sputet euch, geliebte Br�der und Mitdiener, da� auch ihr
">gleichsam im Geiste anwesend"< mit eurer Unterschrift unsere Synode best�tigt,
auf da� von allen Mitdienern �berall die Einheit bewahrt werde.
\pend
\endnumbering
\end{translatio}
\end{Rightside}
\Columns
\end{pairs}
\selectlanguage{german}
% \thispagestyle{empty}
% \renewcommand*{\goalfraction}{.8}

%%% Local Variables:
%%% mode: latex
%%% TeX-master: "dokumente_master"
%%% End:

%%%% Input-Datei OHNE TeX-Pr�ambel %%%%
\section{Theologische Erkl�rung der ">westlichen"< Synode}
% \label{sec:43.2}
\label{sec:SerdicaWestBekenntnis}
\begin{praefatio}
  \begin{description}
  \item[Herbst 343] Zum Datum vgl. oben Dok. \ref{ch:SerdicaEinl}.
    Der Aufbau der Glaubenserkl�rung ergibt sich aus dem Text
    selbst. Zun�chst (� 1) werden diejenigen verworfen, die Christus
    nicht die wahre Gottheit und Sohnschaft zuerkennen, da sie ihm
    eine gesch�pfliche Entstehung in der Zeit zuweisen, ferner (� 2)
    Ursacius\index[namen]{Ursacius!Bischof von Singidunum} und
    Valens\index[namen]{Valens!Bischof von Mursa}, die den Sohn
    aufgrund von Niedrigkeitsaussagen herabmindern, und schlie�lich
    diejenigen, die von drei Hypostasen reden. An diesen letzten Punkt
    ankn�pfend folgt (� 3) das Bekenntnis zur einen Hypostase, welches
    schlie�lich gegen mehrere Kritikpunkte (� 4: Was ist dann die
    Hypostase des Sohnes? � 5: Ist der Sohn nicht erst gezeugt? � 6:
    Ist dann der Sohn mit dem Vater zu identifizieren? � 7: Inwiefern
    ist dann der Vater gr��er als der Sohn (Io
    14,28)\index[bibel]{Johannes!14,28}? Liegt die Einheit denn nicht
    nur in der Gesinnung?) verteidigt und mit dem Bekenntnis zur
    Ewigkeit dieser einen Hypostase (� 8) abgesichert wird. Die
    Erkl�rung wird fortgesetzt (� 9) mit einem Bekenntnis zum Heiligen
    Geist, zur Inkarnation (hier ein kurzer R�ckgriff auf die
    Verwerfung des Ursacius\index[namen]{Ursacius!Bischof von
      Singidunum} und Valens\index[namen]{Valens!Bischof von Mursa}),
    Auferstehung und Richteramt Christi. Es folgt (� 10) noch ein
    exegetischer Nachtrag zur Frage der Einheit, da offenbar besonders
    die Schriftstelle Io 17,21\index[bibel]{Johannes!17,21} diskutiert
    worden ist. Unklar bleibt, wie die Themenangabe in dem Brief des
    Ossius\index[namen]{Ossius!Bischof von Cordoba} und
    Protogenes\index[namen]{Protogenes!Bischof von Serdica} an
    Julius\index[namen]{Julius!Bischof von Rom} (Dok.
    \ref{sec:BriefOssiusProtogenes}) damit zu vereinbaren ist.
    %% Dieser
    % Brief belegt, da� es drei Fragen gab, die verhandelt wurden. Ein
    % Thema war der Satz ">quod erat quando non erat"<, womit das
    % Problem der Gleichewigkeit des Sohnes mit dem Vater formuliert
    % ist, das auch aus dem Beginn der theologischen Erkl�rung selbst
    % als eine der Fragen herausgelesen werden kann. Die beiden
    % anderen
    % Frage sind leider in der �berlieferung des Briefes ausgefallen.

    Aufgrund zahlreicher �bereinstimmungen der theologischen Erkl�rung
    mit der Theologie des Markell von
    Ancyra\index[namen]{Markell!Bischof von Ancyra}, auf die an den
    entsprechenden Stellen im Text hingewiesen wird, gilt die
    Glaubenserkl�rung auf jeden Fall als von
    Markell\index[namen]{Markell!Bischof von Ancyra} beeinflu�t.
  \item[�berlieferung] Zum Verh�ltnis der Theodoret�berlieferung zum
    Codex Veronensis LX vgl. Dok.
    \ref{sec:SerdicaRundbrief}. Konjekturen bei
    \cite{Loofs:Glaubensbekenntnis}, \cite{Tetz:Ante},
    \cite{Hall:Creed}, \cite{Abramowski:Arianerrede} und
    \cite{Ulrich1994}. Da die theologische Erkl�rung durch Hilarius
    und Athanasius nicht �berliefert wird, sondern nur durch Theodoret
    und den Codex Veronensis LX, ist die Echtheit dieses Textes schon
    aufgrund dieser �berlieferungslage bestritten worden. Dazu kommt,
    da� Athanasius sich in tom. 5 von diesem Dokument distanziert
    (vgl. \cite[344]{Brennecke2006}). Dies kann jedoch dadurch erkl�rt
    werden, da� Athanasius zu dem Zeitpunkt der Abfassung von tom. 5
    kein Interesse mehr daran hatte, an den �berholten
    Begrifflichkeiten dieses Textes festzuhalten, da sie einer
    theologischen Einigung zu jener Zeit im Wege gestanden h�tten. Von
    daher ist es verst�ndlich, wenn Athanasius darauf verzichtet,
    diese theologische Erkl�rung wiederzugeben.  Hilarius �bergeht
    diesen Text, da f�r ihn inzwischen das Nicaenum im Vordergrund
    stand.
    
    \begin{figure}[h] %% Stemma
      \begin{scriptsize}\begin{center}
          \begin{tikzpicture}
            \node (Lat) at (0,0) {Lateinische Fassung}; \node (Gr) at
            (0,-1) {Griechische �bersetzung}; \node (Thdt) at (-2,-2)
            {Thdt.}; \node (Cod) at (2,-2) {Cod.\,Ver.}; \draw (Lat)
            -- (Gr); \draw (Gr) -- (Thdt); \draw (Gr) -- (Cod);
          \end{tikzpicture}
        \end{center}
        \legend{Stemma f�r Dok. \ref{sec:SerdicaWestBekenntnis}}
      \end{scriptsize}
    \end{figure}

    Es lassen sich keine sprachlichen und inhaltlichen Br�che im
    Rundbrief finden, die darauf hinweisen, da� die theologische
    Erkl�rung urspr�nglich in den Brief der Synode integriert
    war. Insgesamt erf�llt die Glaubenserkl�rung die Forderung der
    Kaiser, auch �ber den Glauben zu verhandeln.

    % F�r den Umgang mit dem Codex Veronensis LX gilt dasselbe wie in
    % Dok.\ref{sec:SerdicaRundbrief}. Wenn der Text des Codex
    % Veronensis LX hier jedoch angegeben werden mu�, werden alle
    % notwendigen und eindeutigen Korrekturen dennoch gekennzeichnet.
  \item[Fundstelle]Thdt., h.\,e. II 8,37--52 (\editioncite[112,16--118,4]{Hansen:Thdt}); Cod.\,Ver. f. 86a--88a
  \end{description}
\end{praefatio}
\begin{pairs}
\selectlanguage{polutonikogreek}
\begin{Leftside}
% \beginnumbering
\pstart
\hskip -1.4em\edtext{\abb{}}{\killnumber\Cfootnote{\hskip -1.1em\latintext Thdt. (B A
N+GS(s)=r L+F(v)=z T) Cod.\,Ver.}}\specialindex{quellen}{section}{Theodoret!h.\,e.!II 8,37--52}\specialindex{quellen}{section}{Codices!Veronensis LX!f. 86a--88a}
\kap{1}>Apokhr'uttomen
\edtext{\abb{d`e}}{\Dfootnote{\latintext > Cod.\,Ver.}} >eke'inouc ka`i >exor'izomen t~hc kajolik~hc
>ekklhs'iac
\edtext{to`uc}{\Dfootnote{to`uc m`h \latintext Br(\greektext m`h \latintext in
ras. G)z}} diabebaioum'enouc,
\edtext{<'oti je'oc >estin
\edtext{dhlon'oti <o Qrist'oc}{\Dfootnote{dhlon'oti
 <o \oline{qc} \dt{BT}  <o \oline{qc} d~hlon <'oti \dt{A}
 d~hlon <o \oline{qc} \dt{rz}}},
\edtext{>all`a m`hn}{\Dfootnote{>all`a l'egein \latintext r \greektext >all`a
m`hn l'egontac <'oti \latintext T}} >alhjin`oc je`oc o>uk
>estin,
\edtext{<'oti u<i'oc >estin,
\edtext{\abb{>all`a}}{\Dfootnote{ka`i \latintext coni. Loofs e Cod.\,Ver.}}
\edtext{>alhjin`oc u<i`oc o>uk >'estin}{\Dfootnote{gr. >alhjin`oc \oline{jc} o>uk
>'estin \latintext in mg. A\corr}}}{\lemma{\abb{<'oti u<i'oc \dots\ o>uk
>'estin}}\Dfootnote{\latintext > F}},
<'oti gennht'oc >estin <'ama ka`i
\edtext{\abb{genht'oc}}{\Dfootnote{\latintext A \greektext >ag'ennhtoc
\latintext A\corr\ \greektext >ag'ennhtoc \latintext BrzT}}}{\lemma{<'oti je'oc
>estin \dots\ <'ama ka`i genht'oc}\Dfootnote{\latintext \textit{quod deus
quidem est Christus, sed verus deus non est quia filius es(t), sed et verus
filius non est quod natu(s) est, simul et factus.} Cod.\,Ver.}}.
\edtext{o<'utwc
\edtext{\abb{g`ar}}{\Dfootnote{\latintext > A}}
\edtext{\abb{>'ejoc a>uto~ic}}{\Dfootnote{\dt{coni. Hall} e>i'wjasi \latintext coni. Loofs e Cod.\,Ver. \greektext
<eauto`uc \latintext BArz \greektext a>uto`uc \latintext T \greektext kaj'
<eauto`uc \latintext coni. Koetschau}} noe~in
\edtext{{t`on}}{\Dfootnote{t`o \latintext coni. Loofs}} gegennhm'enon
\edtext{\abb{<omologo~untec}}{\Dfootnote{\latintext coni. Loofs e Cod.\,Ver. \greektext
<omologo~usin \latintext Thdt., \greektext <omologo~usin \latintext \dots\
\greektext gegenhm'enon \latintext > B}},
\edtext{<'oti}{\Dfootnote{ka`i <'oti \latintext T,
\responsio\ \greektext <'oti \latintext post \greektext e~>ipon \latintext
coni. Tetz}},
\edtext{\abb{<'wsper e~>ipon}}{\Dfootnote{\latintext coni. Tetz \greektext o<'utwc e~>ipon
\latintext Thdt. \greektext <wc proe'ipomen \latintext coni. Loofs e \textit{sicut
supra dixerunt} Cod.\,Ver.}}, ((t`o
\edtext{gegennhm'enon}{\Dfootnote{gegenhm'enon \latintext A}}
\edtext{\abb{gegenhm'enon}}{\Dfootnote{\latintext in mg. A\corr\ \greektext
gegennhm'enon \latintext NGzT > AS}}
\edtext{>est'in}{\Dfootnote{e~>inai \latintext T}})) \edlabel{Loofs1}\edtext{ka`i}{{\xxref{Loofs1}{Loofs2}}\lemma{\abb{ka`i \dots\ >'eqei}}\Dfootnote{\dt{del. Loofs}}}
<'oti to~u Qristo~u
\edtext{\abb{pr`o a>i'wnwn}}{\Afootnote{\latintext 1Cor 2,7}} >'ontoc
did'oasin a>ut~w|
\edtext{>arq`hn ka`i t'eloc}{\Dfootnote{>arq`hn to~u e~>inai \latintext
coni. Loofs \greektext >arq`hn ka`i e>'ige >arq`hn ka`i t'eloc \latintext
coni. Scheidweiler}},
\edtext{<'oper}{\Dfootnote{<'osper \latintext
coni. Christophorson}} o>uk >en kair~w|, >all`a pr`o pant`oc
\edtext{\abb{qr'onou}}{\Dfootnote{\latintext A \greektext kairo~u \latintext
BrzT}}
\edlabel{Loofs2}\edtext{>'eqei}{\Dfootnote{t`hn g'enesin >'eqei
\latintext coni. Christophorson \greektext >'eqein a>uton t`hn >arq`hn l'egontec
\latintext coni. Scheidweiler}}.}{\lemma{o<'utwc g`ar e>i'wjasi \dots\ pr`o pant`oc
qr'onou >'eqei.}\Dfootnote{\latintext ... \textit{sic enim intellegere
consuerunt natum qui fatentur, sicut supra dixerunt quia quod natum est factum
est; et Christo, cum sit ante secula, dant initium et finem, quod non ex tempore
sed ant(e) omne tempus habet} \latintext Cod.\,Ver.}}\edindex[bibel]{Korinther I!2,7}
\pend
\pstart
\kap{2}ka`i <up'oguon
\edtext{\abb{d`e}}{\Dfootnote{\latintext > Cod.\,Ver.}} d'uo >'eqeic >ap`o t~hc >asp'idoc t~hc
\edtext{>Areian~hc}{\Dfootnote{>areianik~hc \latintext NGT}} >egenn'hjhsan,
O>u'alhc\edindex[namen]{Valens!Bischof von Mursa} ka`i
\edtext{O>urs'akioc}{\Dfootnote{>Ars'akioc \latintext B}}\edindex[namen]{Ursacius!Bischof von Singidunum};
\edtext{o<'i tinec}{\Dfootnote{\latintext \textit{qui} Cod.\,Ver.}} kauq~wntai
ka`i o>uk >amfib'allousi
\edtext{l'egontec}{\Dfootnote{\latintext \textit{dicere} Cod.\,Ver.}} <eauto`uc
\edtext{Qristiano`uc}{\Dfootnote{\latintext \textit{christos} Cod.\,Ver.}} e>~inai
ka`i
<'oti <o l'ogoc ka`i
\edtext{<'oti}{\lemma{\abb{<'oti\ts{2}}}\Dfootnote{\latintext > Cod.\,Ver.}} t`o pne~uma
\edtext{ka`i}{\lemma{\abb{ka`i\ts{1}}}\Dfootnote{\latintext > Cod.\,Ver.}}
\edtext{>estaur'wjh}{\Dfootnote{>e\oline{str}'wjh \latintext Thdt. \greektext >etr'wjh
\latintext coni. Loofs e \textit{vulneratus est} Cod.\,Ver.}} ka`i >esf'agh ka`i
>ap'ejanen ka`i >an'esth ka'i, <'oper t`o t~wn a<iretik~wn s'usthma
\edtext{filoneike~i}{\Dfootnote{\latintext \textit{adsolet \dots\ contendere} Cod.\,Ver.}},
diaf'orouc e>~inai t`ac
\edtext{<upost'aseic}{\Dfootnote{\latintext \textit{substantias} Cod.\,Ver.
\greektext o>us'iac \latintext T}} to~u patr`oc ka`i to~u u<io~u ka`i
\edtext{to~u <ag'iou pne'umatoc}{\lemma{\abb{}}\Dfootnote{\responsio\ p\oline{nc} to~u
<ag'iou \latintext rF \textit{sp\oline{u}sc\oline{i}} Cod.\,Ver.}} ka`i
\edtext{\abb{e~>inai}}{\Dfootnote{\latintext > T}} keqwrism'enac.
\pend
\pstart
\kap{3}<hme~ic d`e ta'uthn pareil'hfamen ka`i dedid'agmeja,
\edtext{ta'uthn}{\Dfootnote{ka`i ta'uthn \latintext A}} >'eqomen t`hn kajolik`hn
ka`i >apostolik`hn par'adosin ka`i p'istin
\edtext{\abb{ka`i}}{\Dfootnote{\latintext A\slin}} <omolog'ian; m'ian
\edtext{\abb{e~>inai}}{\Dfootnote{+ ka`i \latintext A*}}
\edtext{<up'ostasin}{\Dfootnote{o>us'ian \dt{T \textit{substantiam} Cod.
Ver.}}},
\edtext{<`hn a>uto`i
\edtext{\abb{o<i <'Ellhnec}}{\Dfootnote{\latintext coni. Stockhausen e Cod.\,Ver. \greektext o<i
<Ellhniko'i \latintext coni. Wintjes \greektext o<i a<iretiko`i \latintext Thdt. del.
Schwartz}} 
\edtext{o>us'ian}{\Dfootnote{ka`i o>us'ian \dt{coni. Loofs}}} prosagore'uousi}{\lemma{<`hn a>uto`i \dots\ prosagore'uousi}
\Dfootnote{\latintext \textit{quam ipsi graeci (u)sian
appellant} Cod.\,Ver. > T}}, to~u patr`oc ka`i to~u u<io~u ka`i to~u <ag'iou
pne'umatoc.
\pend
\pstart
\kap{4}\edtext{ka`i e>i zhto~ien, t'ic
\edtext{to~u u<io~u <h <up'ostas'ic}{\lemma{\abb{}}\Dfootnote{\responsio\ <h
<up'ostasic to~u u<io~u \latintext L}}
\edtext{>estin,}{\Dfootnote{; >estin \latintext T}}
\edtext{<omologo~umen, <wc}{\Dfootnote{<omologoum'enwc \latintext BA (= \textit{pro
certo} Cod.\,Ver.?)}}
\edtext{a<'uth}{\Dfootnote{a>ut`h \latintext N}}
\edtext{\abb{~>hn}}{\Dfootnote{\latintext del. Parmentier}} <h m'onh to~u patr`oc
<omologoum'enh,
\edtext{mhd'e}{\Dfootnote{ka`i mhd'e \latintext A*}} pote pat'era qwr`ic u<io~u
\edtext{mhd`e}{\Dfootnote{mhd'' a~>u \latintext T}} u<i`on qwr`ic patr`oc
gegen~hsjai
\edtext{\abb{mhd`e e~>inai d'unasjai}}{\Dfootnote{\latintext > B}}}{\lemma{ka`i
e>i zhto~ien \dots\ d'unasjai}\Dfootnote{\latintext \textit{et si qu(a)erit
quis filii substantiam est pro certo h(a)ec quam solius sine patre} Cod.\,Ver. \greektext ka`i e>i zhto~ien, t'ic to~u u<io~u <h
<up'ostasic, >'estin <omologoum'enwc a<'uth, <`h ~>hn m'onou to~u patr'oc.
<omologo~umen mhd'e pote pat'era qwr`ic u<io~u, mhd`e u<i`on qwr`ic patr`oc
gegen~hsjai mhd`e e~>inai d'unasjai \latintext coni. Loofs e Cod.\,Ver.
\greektext ka`i e>i zhto~ien tic to~u u<io~u t`hn <up'ostasin, >'estin
<omologoum'enwc a<'uth m`h m'onou to~u patr'oc; <omo~u l'egomen mhd'e pote
pat'era qwr`ic u<io~u mhd`e u<i`on qwr`ic pne'umatoc gegen~hsjai mhd`e e~>inai
d'unasjai \latintext coni. Tetz \greektext ka`i e>i zhto~ien, t'ic to~u u<io~u <h
<up'ostasic, >'estin <omologoum'enwc a<'uth, <`h ~>hn m'onou to~u patr'oc.
<omo~u l'egomen  mhd'e pote pat'era qwr`ic u<io~u mhd`e u<i`on qwr`ic patr`oc
gegen~hsjai mhd`e e~>inai d'unasjai \latintext coni. Ulrich}},
\edtext{\edtext{\abb{<'oti >'esti l'ogoc pne~uma}}{\Dfootnote{\latintext coni. Tetz e Cod.\,Ver.
\greektext <'o >esti l'ogoc p\oline{na} \latintext BF \textit{quod est verbum,
spiritus} Cod.\,Ver. \greektext ~<w >esti l'ogoc p\oline{na} \latintext L \greektext <'o
>esti l'ogoc p\oline{nc} \latintext N \greektext <'oc >esti l'ogoc p\oline{rc} \latintext s
\greektext ~<w| (~<w| \latintext in ras. A\corr) \greektext >esti l'ogoc p\oline{na} (n
\latintext e \greektext r \latintext A\corr) \greektext o>uk >'eqwn \latintext A
\greektext o<~u m'h >esti l'ogoc pne~uma \latintext coni. Scheidweiler del. Abramowski}}.
\edtext{>atop'wtaton g'ar >esti}{\Dfootnote{\latintext  Cod.\,Ver. interpunxit
ante \greektext pne~uma \latintext et continuit \textit{absurdus enim est} et
interpunxit iterum}}
l'egein
\edtext{pot`e}{\Dfootnote{\latintext \textit{nunc} Cod.\,Ver.}} pat'era
\edtext{\abb{m`h}}{\Dfootnote{\latintext > G}}
\edtext{gegen~hsjai}{\Dfootnote{\latintext post \greektext gegen~hsjai
\latintext interpunxit Parmentier (secundum A ante ras.)}}
\edtext{\abb{pat'era}}{\lemma{\abb{pat'era}}\Dfootnote{
\latintext > BN Cod.\,Ver. interpunxit et \greektext di`a to~uto, >'oti d~hlon >esti pat'era \latintext add.
Loofs}\Cfootnote{\dt{inc. Thdt.(V) \hskip 1ex Thdt.(B A N+GS(s)=r L+FV(v)=z T)
Cod.\,Ver.}}},
\edtext{\abb{<'oti, <wc noe~itai pat`hr}}{\Dfootnote{\latintext add. Wintjes e
\textit{quoniam quod intellegitur pater} Cod.\,Ver.}} qwr`ic u<io~u
\edtext{m'hte}{\Dfootnote{mhd`e \latintext B \greektext m'hpote \latintext L}}
>onom'azesjai \edtext{m'hte}{\Dfootnote{mhd`e \latintext B}} e>~inai
d'unasjai}{\lemma{\abb{<'oti >'esti \dots\ d'unasjai}}\Dfootnote{\latintext >
T}},
\edtext{>'estin}{\Dfootnote{+ d`e \dt{GTS\ras\  ante}  >'estin \dt{interpunxit Loofs et connexit}  >'esti \dots\ martur'ia \dt{cum sequentibus citationibus}}} a>uto~u
\edtext{to~u u<io~u martur'ia}{\Dfootnote{martur'ia to~u u<io~u l'egousa
\latintext A}};
\edtext{((\edtext{>eg`w}{\lemma{\abb{}}\Dfootnote{\latintext + \textit{et} ante \greektext >eg`w \latintext Cod.\,Ver.}} >en t~w|
\edtext{patr`i}{\Dfootnote{\latintext \textit{patre} ex \textit{pater} Cod.\,Ver.\corr}} ka`i <o pat`hr >en
>emo'i))}{\lemma{\abb{}}\Afootnote{\latintext Io 14,10}}\edindex[bibel]{Johannes!14,10}
\edtext{\abb{ka`i}}{\Dfootnote{\latintext > A}}
\edtext{((>eg`w ka`i <o pat`hr <'en
>esmen))}{\lemma{\abb{}}\Afootnote{\latintext Io 10,30}}\edindex[bibel]{Johannes!10,30}.
\pend
\pstart
\kap{5}\edtext{o>ude`ic <hm~wn >arne~itai
\edtext{gegennhm'enon}{\Dfootnote{t`on gegennhm'enon \latintext A \greektext gegenhm'enon \latintext v \textit{natum}
Cod.\,Ver.}}, >all'a
\edtext{\edtext{\abb{kt'isin}}{\Dfootnote{\latintext coni. Opitz \greektext tisin
\latintext BN \greektext tis`in \latintext G \greektext tis`i \latintext S
\greektext tisi \latintext V \greektext t'isi \latintext FT \textit{quibusdam} (sequitur interpunctio)
Cod.\,Ver. \greektext kt'isma \latintext coni. Tetz}}
\edtext{\abb{gegennhm'enon}}{\Dfootnote{\latintext BNT \textit{natum} Cod.\,Ver.
\greektext gegenhm'enon \latintext sv}}
\edtext{pant'apasin}{\Dfootnote{p~asin \latintext T \textit{omnino}
Cod.\,Ver.}}}{\lemma{kt'isin gegenhm'enon pant'apasin}\Dfootnote{gegenhm'enon pr`o
p'antwn \latintext LA\corr}},
\edtext{\edtext{<'aper}{\Dfootnote{<'oper \latintext V \textit{sicut}
Cod.\,Ver.}}
\edtext{\abb{>a'orata}}{\Dfootnote{\latintext ex \greektext <'orata \latintext A\corr (cf. \textit{visibilia et invisibilia} Cod.\,Ver.)}} ka`i <orat`a}{\Afootnote{\latintext vgl. Col
1,16}}
\edtext{prosagore'uetai}{\Dfootnote{prosagore'uousi \latintext L(\greektext ousi \latintext in ras.)A\corr}}
\edtext{\abb{gennhj'enta}}{\Dfootnote{\latintext vT \greektext genhj'enta
\latintext B \greektext gegen~hsjai \latintext r \greektext poiht`hn ka`i
\latintext L(in ras.)A\corr}},
\edtext{teqn'ithn}{\Dfootnote{\latintext \textit{artifice(m)} Cod.\,Ver.}}
\edtext{ka`i}{\lemma{\abb{ka`i\ts{1}}}\Dfootnote{\latintext > Cod.\,Ver.}}
>arqagg'elwn ka`i >agg'elwn
\edtext{ka`i}{\lemma{\abb{ka`i\ts{1}}}\Dfootnote{\latintext > Cod.\,Ver.}}
\edtext{k'osmou}{\Dfootnote{k'osmw \latintext rv}} ka`i
\edtext{\abb{to~u >anjrwp'inou g'enouc}}{\Dfootnote{\latintext LTA\corr
\greektext t~w >anjrwp'inw g'enei \latintext Brv}}, <'oti
fhs'in}{\lemma{o>ude`ic <hm~wn \dots\ <'oti fhs'in}\Dfootnote{o>ude`ic <hm~wn
>arne~itai t`o gegennhm'enon, >all'' o>uq <eaut~w| gegennhm'enon l'egomen t`on
l'ogon t`on p'antote <'onta. >all`a t'isin gegennhm'enon? pant'apasin p~asi,
<'aper <orat`a ka`i >a'orata prosagore'uetai, gennhj'enta teqn'ithn ka`i
>arqagg'elwn ka`i >agg'elwn ka`i k'osmou ka`i to~u >anjrwp'inou g'enouc, <'oti
fhs'in; \latintext coni. Loofs \greektext o>ude`ic <hm~wn >arne~itai t`o
gegennhm'enon, >all'a tisin gegennhm'enon, pant'apasin <'aper >a'orata ka`i
<orat`a prosagore'uetai, gennhj'enta teqn'ithn ka`i >arqagg'elwn ka`i >agg'elwn
ka`i k'osmou ka`i t~w| >anjrwp'inw| g'enei, <'oti fhs'in; \latintext coni. Parmentier
\greektext o>ude`ic <hm~wn >arne~itai t`o gegennhm'enon, >all`a kt'isma
gegennhm'enon pant'apasin <'wsper >a'orata ka`i <orat`a prosagore'uein
gennhj'enta teqn'ithn ka`i >arqagg'elwn ka`i >agg'elwn ka`i k'osmou ka`i to~u
>anjrwp'inou g'enouc, <'oti fhs'in; \latintext coni. Tetz}}\edindex[bibel]{Kolosser!1,16|textit};
\edtext{((<h p'antwn teqn~itic
\edtext{>ed'idax'en}{\Dfootnote{\latintext \textit{docu(i)t} Cod.\,Ver.}} me
\edtext{sof'ia}{\Dfootnote{\latintext \textit{sapientiam}
Cod.\,Ver.}}))}{\lemma{\abb{}}\Afootnote{\latintext Sap 7,21}}\edindex[bibel]{Weisheit!7,21} ka`i
\edtext{((p'anta di'' a>uto~u
\edtext{\abb{>eg'eneto}}{\Dfootnote{+ ka`i qwr`ic
a>uto~u >eg'eneto o>ud'en \latintext rT >
Cod.\,Ver.}}))}{\lemma{\abb{}}\Afootnote{\latintext Io 1,3}}\edindex[bibel]{Johannes!1,3}.
\edtext{\edtext{\abb{o>ud`e}}{\Dfootnote{\latintext BLvA\corr\ \greektext
\responsio\ o>uk \latintext post \greektext e~>inai \latintext r \greektext
o>u \latintext A \greektext p~wc \latintext T}}
\edtext{p'antote g`ar}{\lemma{\abb{}}\Dfootnote{\responsio\ g`ar p'antote
\latintext T}}
\edtext{\abb{e~>inai}}{\Dfootnote{\latintext > B}} >hd'unato e>i >arq`hn
>'elaben, <'oti <o p'antote >`wn >arq`hn o>uk >'eqei l'ogoc
\edtext{\abb{je`oc}}{\Dfootnote{+ d`e \latintext Thdt.}}, o>ud'epote <upom'enei
t'eloc}{\lemma{o>ud`e \dots\ t'eloc}\Dfootnote{\dt{\textit{numquam enim
">esse"< poterit accipere initium; quoniam qui semper est initium non habet,
verbum deus, nec suscipiaens finem} Cod.\,Ver.} o>ud'epote g`ar to~u
e~>inai >hd'unato >arq`hn labe~in, <'oti <o p'antote >`wn >arq`hn o>uk >'eqei
l'ogoc je'oc, o>ud'e pote <upom'enei t'eloc. \dt{coni. Loofs e Cod.\,Ver.} o>u
p'antwn g`ar >arq`h e~>inai >hd'unato e>i >arq`hn >'elaben; <'oti d`e <o p'antwn
>`wn >arq`h >arq`hn o>uk >'eqei, l'ogoc je'oc >esti o>ud'e pote <upom'enei
t'eloc. \dt{susp. Parmentier}}}.
\pend
\pstart
\kap{6}o>u l'egomen t`on pat'era u<i`on e>~inai o>ud`e
\edtext{p'alin}{\Dfootnote{\latintext \textit{igitur} Cod.\,Ver.}} t`on u<i`on
pat'era e>~inai; >all'' <o pat`hr pat'hr >esti ka`i
\edtext{<o u<i`oc patr`oc
\edtext{u<i'oc}{\Dfootnote{>estin u<i'oc \latintext T}}}{\Dfootnote{\latintext
\textit{filium patris filium} Cod.\,Ver.}}.
\edtext{<omologo~umen
\edtext{\abb{d'unamin e~>inai to~u patr`oc}}{\lemma{\abb{}}\Dfootnote{
\responsio\ to~u \oline{prc} d'unamin e~>inai \latintext z \greektext \responsio\ to~u
\oline{prc} e~>inai d'unamin \latintext T \textit{potentiam patris esse} Cod.\,Ver.}}
t`on u<i'on}{\lemma{\abb{<omologo~umen \dots\ u<i'on}}\Dfootnote{\latintext >
B}};
<omologo~umen
\edtext{t`on}{\Dfootnote{a>ut`on \latintext coni. Loofs}}
\edtext{\abb{l'ogon}}{\Dfootnote{\latintext in ras. A\corr}}
\edtext{jeo~u}{\Dfootnote{to~u jeo~u \latintext B}} patr`oc
\edtext{e~>inai}{\Dfootnote{e~>inai u<i`on \latintext T > Cod.\,Ver. del. Loofs}},
\edtext{par'' <`on}{\Dfootnote{par`on \latintext B}} <'eteroc o>uk >'estin, ka`i
t`on l'ogon
\edtext{>alhj~h}{\Dfootnote{\latintext \textit{vero} Cod.\,Ver.}} je`on ka`i
\edtext{sof'ian ka`i
\edtext{d'unamin}{\Dfootnote{>alhj~h d'unamin \latintext T}}}{\lemma{\abb{sof'ian ka`i d'unamin}}\Afootnote{\latintext vgl. 1Cor 1,24}}\edindex[bibel]{Korinther I!1,24|textit}.
\edtext{>alhj~h d`e u<i`on}{\Dfootnote{ka`i u<i`on >alhj~h \latintext T
\greektext u<i`on \latintext A}} \edtext{paradid'oamen}{\Dfootnote{parad'idomen
\latintext B}},
>all'' o>uq
\edtext{<'wsper}{\Dfootnote{\latintext \textit{secundum quod} Cod.\,Ver.}} o<i loipo`i u<io`i prosagore'uontai
\edtext{t`on u<i`on l'egomen}{\lemma{\abb{}}\Dfootnote{\responsio\ l'egomen
t`on u<i`on \latintext L}},
<'oti
\edtext{\abb{>eke~inoi}}{\Dfootnote{\latintext in ras. A\corr}}
\edtext{\abb{>`h di`a t`o u<iojete~isjai >`h to~u}}{\Dfootnote{\dt{coni. Erl. ex
\textit{aut propter adoptionem vel} Cod.\,Ver.} >`h di`a t`o
u<iojete~isjai to~u / u<i'ojetoi e~>inai \dt{coni. Wintjes} >`h di`a t`o j'esei u<io`i e~>inai
to~u \dt{susp. Parmentier} >`h \dt{(}>`h \dt{> sT)} di`a to~uto jeo`i e~>ien to~u \dt{BrT} >`h di`a
to~uto jeo`i >`h >en t~w \dt{z(in ras.)A\corr.} >`h di`a u<iojes'ian >`h
to~u \dt{coni. Loofs e Cod.\,Ver.}}}
\edtext{\edtext{>anagenn~asjai}{\Dfootnote{genn~asjai \latintext coni. Tetz e Cod.\,Ver.}}
\edtext{q'arin}{\Dfootnote{q'ariti \latintext NG, quos sequitur
Wintjes}}}{\Dfootnote{\latintext \textit{ob nativitate} Cod.\,Ver.}}
\edtext{>`h}{\Dfootnote{~<hc \latintext T}}
\edtext{di`a t`o kataxiwj~hnai u<io`i prosagore'uontai}{\Dfootnote{\latintext
\textit{quod merentur filii vocari} Cod.\,Ver.}},
\edtext{o>u}{\Dfootnote{o>ud`e \latintext B}} di`a t`hn m'ian
\edtext{<up'ostasin}{\Dfootnote{o>us'ian \dt{T \textit{substantiam} Cod.\,Ver.}}},
\edtext{<'htic}{\Dfootnote{e>'i tic \latintext B}} >est`i to~u patr`oc ka`i to~u
u<io~u.
\pend
\pstart
<omologo~umen
\edtext{\abb{ka`i}}{\Dfootnote{\latintext > A Cod.\,Ver.}}
\edtext{\abb{monogen~h}}{\Afootnote{\latintext Io 1,18}}\edindex[bibel]{Johannes!1,18} ka`i
\edtext{\abb{prwt'otokon}}{\Afootnote{\latintext Col 1,15}}\edindex[bibel]{Kolosser!1,15}, >all`a monogen~h t`on l'ogon,
\edtext{<`oc}{\Dfootnote{<'oti \latintext coni. Tetz \textit{quod} Cod.\,Ver.}} p'antote >~hn
\edtext{\abb{ka`i}}{\Dfootnote{\latintext > Cod.\,Ver.}}
\edtext{\abb{>'estin >en t~w| patr'i}}{\Afootnote{\latintext vgl. Io
14,10}}\edindex[bibel]{Johannes!14,10|textit}.
t`o prwt'otokoc
\edtext{d`e}{\Dfootnote{\latintext \textit{sane} Cod.\,Ver.}}
\edtext{t~w| >anjr'wpw|}{\Dfootnote{t~wn >anjr'wpwn \latintext N et ante ras. A
\greektext >an(jrwp)'iw \latintext T \textit{at hominem} Cod.\,Ver.}} \edtext{\Ladd{diaf'erei}}{\lemma{\abb{diaf'erei}}\Dfootnote{\latintext add. Abramowski}},
\edtext{diaf'erei}{\Dfootnote{diafore~i \latintext A}}
\edtext{\abb{d`e}}{\Dfootnote{\latintext BArz \greektext ka`i \latintext T,
quem sequitur Loofs, qui interpunxit ante \greektext t`o prwt'otokoc \latintext et del.
interpunctionem ante \greektext diaf'erei; \latintext interpunctionem post \greektext >anjr'wpw| \latintext coni. Parmentier, qui attinet \greektext t~w| >anjr'wpw| \latintext ad \greektext <omologo~umen, \latintext quod cum Acc. et cum Dat. construi potest}} t~h|
\edtext{\abb{kain~h|}}{\Dfootnote{\latintext T Loofs \greektext koin~h
\latintext BArz \textit{communem} Cod.\,Ver.}} kt'isei, <'oti ka`i
\edtext{((prwt'otokoc >ek
\edtext{\abb{t~wn}}{\Dfootnote{\latintext > BzT}}
nekr~wn))}{\lemma{\abb{}}\Afootnote{\latintext Col 1,18}}\edindex[bibel]{Kolosser!1,18}.
\pend
\pstart
\edtext{\abb{<omologo~umen <'ena e>~inai je'on}}{\Dfootnote{\latintext > Cod.\,Ver.}}, <omologo~umen
\edtext{m'ian}{\Dfootnote{m'ian e~>inai \latintext B}} patr`oc ka`i u<io~u
je'othta.
\pend
\pstart
\kap{7}o>ud'e tic >arne~ita'i
\edtext{\abb{pote}}{\Dfootnote{\latintext > AB (s.l.) A\corr}}
\edtext{\edtext{t`on pat'era to~u u<io~u
me'izona}{\lemma{\abb{t`on \dots\ me'izona}}\Afootnote{\latintext vgl. Io 14,28}}}{\lemma{\abb{}}
\Dfootnote{\responsio\ t`on me'izona to~u u<io~u p\oline{ra} \latintext T}}\edindex[bibel]{Johannes!14,28|textit}, o>u di''
\edtext{>'allhn}{\Dfootnote{\latintext \textit{alia} Cod.\,Ver.}}
\edtext{<up'ostasin}{\Dfootnote{o>us'ian \latintext T \textit{substantiam}
Cod.\,Ver.}},
\edtext{o>ud`e}{\Dfootnote{o>u \latintext BA \textit{nec} Cod.\,Ver.}}
\edtext{di`a}{\Dfootnote{di`a t`hn \latintext BAz \greektext t`hn \latintext r >
T Cod.\,Ver. \greektext tina \latintext coni. Loofs}} diafor'an, >all''
\edtext{\abb{<'oti}}{\Dfootnote{\latintext > Cod.\,Ver.}} a>ut`o t`o >'onoma to~u
patr`oc me~iz'on
\edtext{\edtext{>esti}{\Dfootnote{\latintext \textit{et} Cod.\,Ver.}} to~u
u<io~u}{\lemma{\abb{}}\Dfootnote{\responsio\ to~u u<io~u >estin \latintext T}}.
\edtext{\edtext{\abb{a<'uth d`e}}{\Dfootnote{+ >estin \latintext add. Tetz e Cod.\,Ver.}}
\edtext{\abb{a>ut~wn}}{\Dfootnote{\latintext > N}} <h bl'asfhmoc ka`i
diefjarm'enh <ermhne'ia;
\edtext{to'utou <'eneka}{\Dfootnote{<'oti \latintext coni. Tetz del. Loofs et interpunxit
post \greektext <ermhne'ia; + <'oti \latintext add. Scheidweiler}} e>irhk'enai a>ut`on
filoneiko~usin \edtext{((>eg`w ka`i
\edtext{<o}{\Dfootnote{<'oti \latintext T}} pat`hr <'en
>esmen))}{\lemma{\abb{}}\Afootnote{\latintext Io 10,30}} di`a t`hn sumfwn'ian
ka`i t`hn <om'onoian.}{\lemma{a<'uth d`e \dots\ <om'onoian}
\Dfootnote{\latintext \textit{h(a)ec autem est maledica eorum et corrupta
interpretatio contendentium quod \dt{<<ego et pater unum sumus>>} propter
consensum dixerit et concordiam} Cod.\,Ver.}}\edindex[bibel]{Johannes!10,30}
\edtext{kat'egnwmen}{\Dfootnote{\latintext \textit{inprovemus} Cod.\,Ver.}}
\edtext{\abb{d`e}}{\Dfootnote{\latintext > BA Cod.\,Ver.}} p'antec o<i
kajoliko`i t~hc mwr~ac ka`i o>iktr~ac a>ut~wn diano'iac.
\edtext{\edtext{\abb{ka`i <'wsper}}{\Dfootnote{\latintext rz \greektext <'wsper
\latintext B \greektext <'wsper g`ar \latintext T \greektext ka`i (ka`i
\latintext in mg. A\corr) \greektext <'wsper g`ar o<i (g`ar o<i \latintext del.) A}}
>'anjrwpoi jnhto`i
\edtext{>epeid`h}{\Dfootnote{>epe`i \latintext T}} 
\edtext{diaf'eresjai}{\Dfootnote{diafje'iresjai \latintext BrF}} >'hrxanto
proskekrouk'otec diqonoo~usi ka`i e>ic diallag`hn
\edtext{\abb{>epan'iasin}}{\Dfootnote{\latintext coni. Tetz \greektext >ep'aneisin
\latintext Thdt.}}}{\lemma{ka`i <'wsper \dots\ >epan'iasin}
\Dfootnote{\latintext \textit{sicut homines mortales differunt, ut adsolent,
inter se, et offendent(e)s concordant et in grati(a) revertuntur} Cod.\,Ver.}},
o<'utwc
\edtext{\abb{diast'aseic ka`i diq'onoia}}{\Dfootnote{\latintext coni. Wintjes e Cod.\,Ver.
\greektext di'astasic ka`i diq'onoia \latintext ArzT \greektext diast'aseic ka`i
deiqono'iac \latintext B \greektext diast'aseic ka`i diq'onoiai \latintext coni. Loofs
\textit{separationes et discordia} Cod.\,Ver.}} metax`u
\edtext{\abb{patr`oc}}{\Dfootnote{\latintext A\mg V\mg + \greektext ka`i
\latintext L}} jeo~u pantokr'atoroc
\edtext{\abb{ka`i to~u u<io~u}}{\Dfootnote{\latintext > B \textit{et eius
filium} Cod.\,Ver. \greektext to~u \latintext > T}}
\edtext{\abb{e~>inai d'unatai}}{\Dfootnote{\latintext Br(\greektext d'unatai
\latintext bis S)z \greektext d'unatai e~>inai \latintext T \greektext dunat`on
e~>inai \latintext A \textit{esse potuerunt} Cod.\,Ver. \greektext >hd'unanto
e~>inai \latintext coni. Loofs e Cod.\,Ver. \greektext e~>inai >hd'unanto \latintext
coni. Tetz e Cod.\,Ver.}},
\edtext{\abb{l'egousin}}{\Dfootnote{\latintext > Cod.\,Ver. del. Tetz}},
\edtext{<'oper}{\Dfootnote{\latintext \textit{hoc} Cod.Ver post
interpunctionem}}
\edtext{>atop'wtaton}{\Dfootnote{\latintext \textit{absurdum est} Cod.\,Ver.}}
ka`i
\edtext{no~hsai}{\Dfootnote{<upono~hsai \latintext rzA\corr\
\textit{intellegere} Cod.\,Ver.}} ka`i
\edtext{<upolabe~in}{\Dfootnote{<upobale~in \latintext A}}. <hme~ic d`e
\edtext{\abb{ka`i}}{\Dfootnote{\latintext > T Cod.\,Ver.}} piste'uomen ka`i
diabebaio'umeja ka`i o<'utw noo~umen, <'oti
\edtext{<h <ier`a fwn`h}{\Dfootnote{\latintext \textit{sacra voce} Cod.\,Ver. \greektext <ier~a| fwn~h| \latintext coni. Loofs e Cod.\,Ver.}}
\edtext{>el'alhsen}{\Dfootnote{\latintext \textit{locutus est dicens}
Cod.\,Ver.}}
\edtext{((>eg`w ka`i <o pat`hr <'en
>esmen))}{\lemma{\abb{}}\Afootnote{\latintext Io 10,30}}\edindex[bibel]{Johannes!10,30}
\edtext{ka`i}{\lemma{\abb{ka`i\ts{2}}}\Dfootnote{\latintext > T Cod.\,Ver. \greektext ka`i
to~uto \latintext coni. Scheidweiler}} di`a t`hn t~hc
\edtext{<upost'asewc}{\Dfootnote{o>us'iac \latintext T \textit{sustantie}
Cod.\,Ver.}} <en'othta, <'htic \edtext{\abb{>est`i}}{\Dfootnote{\latintext > T}}
m'ia to~u patr`oc ka`i
\edtext{\abb{m'ia}}{\Dfootnote{\latintext > T Cod.\,Ver.}} to~u u<io~u.
\pend
\pstart
\kap{8}\edtext{\abb{ka`i}}{\Dfootnote{\latintext > Cod.\,Ver. del. Loofs}}
\edtext{to~uto
\edtext{\abb{d`e}}{\Dfootnote{\latintext > Cod.\,Ver.}}}{\lemma{to~uto d`e}\Dfootnote{to~uton
\latintext coni. Tetz \greektext di`a to~uto \latintext coni. Koetschau}}
\edtext{piste'uomen}{\Dfootnote{\latintext Loofs interpunxit post \greektext
p'antote}}
\edtext{\abb{p'antote}}{\Dfootnote{\latintext > T}}
\edtext{>an'arqwc}{\Dfootnote{\latintext \textit{sine principium} Cod.\,Ver.}}
\edtext{\abb{ka`i}}{\Dfootnote{\latintext > Cod.\,Ver.}}
\edtext{>ateleut'htwc}{\Dfootnote{\latintext \textit{sine fine} Cod.\,Ver.}}
\edtext{to~uton}{\Dfootnote{\responsio\ to~uton \latintext post \greektext
basile'uein \latintext B > r}} met`a
\edtext{\abb{to~u}}{\Dfootnote{\latintext > s}} patr`oc basile'uein
\edtext{ka`i}{\Dfootnote{\latintext \textit{hac} Cod.\,Ver.}}
\edtext{m`h
>'eqein
\edtext{m'hte}{\Dfootnote{mhd'ena \latintext coni. Tetz}}
\edtext{\abb{qr'onon}}{\Dfootnote{+ dior'izonta \latintext rzT(in ras.)A\corr}} m'hte}{\lemma{\abb{m`h \dots\ m'hte \dots\ m'hte}}\Dfootnote{\latintext \textit{nullum \dots\ nec} Cod.\,Ver.}}
\edtext{\abb{>'ekleiyin}}{\Dfootnote{\latintext coni. Loofs e \textit{defectum}
Cass. \greektext >ekle'ipein \latintext BArz \textit{minui} Cod.\,Ver.
\greektext >ekleipe~in \latintext T}} a>uto~u t`hn basile'ian, <'oti
\edtext{\abb{<`o}}{\Dfootnote{\latintext > Az}} p'antote >'estin
\edtext{o>ud'e pote}{\Dfootnote{o>'ut'epote \latintext AL \greektext o>'upote
\latintext v \textit{numquam} Cod.\,Ver.}}
\edtext{to~u e>~inai >'hrxato}{\Dfootnote{\latintext \textit{accipit initium} Cod.\,Ver.}} o>ud`e
\edtext{>ekle'ipein}{\Dfootnote{>eklipe~in \latintext N \greektext >ekleipe~in
\latintext T}}
\edtext{d'unatai}{\Dfootnote{\latintext \textit{peterit} Cod.\,Ver.*}}.
\pend
\pstart
\kap{9}\edtext{\abb{piste'uomen}}{\Dfootnote{+ d`e \latintext rzA\corr}} ka`i
paralamb'anomen t`on par'aklhton t`o <'agion pne~uma, <'oper <hm~in a>ut`oc <o
k'urioc
\edtext{ka`i}{\lemma{\abb{ka`i\ts{1}}}\Dfootnote{\latintext > T Cod.\,Ver.}} >ephgge'ilato
ka`i >'epemyen.
\edtext{\edtext{\abb{ka`i to~uto piste'uomen pemfj`en}}{\Dfootnote{\dt{> A\corr z} + ka`i <o l'ogoc o>u p'eponjen \dt{add. Abramowski}}}.
\edtext{\edtext{ka`i to~uto}{\Dfootnote{ka`i <'oti <o je`oc \latintext T}} o>u
\edtext{\abb{p'eponjen}}{\Dfootnote{\latintext BrT \greektext peponj'oc
\latintext z \greektext peponj'wc ('wc \latintext in ras. A\corr) A \greektext p'eponjen +
>'apage \latintext N}}, >all''
\edtext{\abb{<o}}{\Dfootnote{\latintext > A}} >'anjrwpoc}{\lemma{ka`i \dots\  >'anjrwpoc}\Dfootnote{ka`i
to~uton piste'uomen paj'onta, >all'' >'anjrwpon \latintext coni. Tetz}}}{\lemma{ka`i to~uto piste'uomen \dots\ >'anjrwpoc}\Dfootnote{\latintext \textit{et hunc credimus passum, sed homo} Cod.\,Ver.}} <`on
\edtext{>ened'usato}{\Dfootnote{\latintext \textit{se induit} Cod.\,Ver.}}, <`on
\edtext{\abb{>an'elaben}}{\Dfootnote{+ m`en \latintext T}}
\edtext{>ek Mar'iac t~hc parj'enou}{\Dfootnote{>ek t~hc <uperend'oxou ka`i
<upereuloghm'enou despo'inhc <hm~wn jeot'okou ka`i >ae`i parj'enou mar'iac
\latintext N}}, t`on >'anjrwpon
\edtext{\abb{t`on paje~in dun'amenon}}{\Dfootnote{\latintext in mg. A\corr}},
<'oti >'anjrwpoc jnht'oc, je`oc d`e >aj'anatoc. piste'uomen <'oti
\edtext{\abb{t~h|}}{\Dfootnote{\latintext > B}} tr'ith| <hm'era| >an'esth
\edtext{o>uq}{\Dfootnote{o>uq`i \latintext T}} <o je`oc >en t~w| >anjr'wpw|,
>all'' <o >'anjrwpoc >en t~w| je~w >an'esth, <'ontina ka`i pros'hnegke
\edtext{t~w| patr`i <eauto~u}{\lemma{\abb{}}\Dfootnote{\responsio\ <eauto~u
p\oline{ri} \latintext N}} d~wron, <`on
\edtext{\abb{>hleuj'erwsen}}{\Dfootnote{+ >ek t~hc <amart'iac ka`i t~hc fjor~ac
\latintext rz(in mg.)A\corr}}. piste'uomen
\edtext{\abb{d`e}}{\Dfootnote{\dt{> Cod.\,Ver.}}} <'oti e>uj'etw| kair~w|
\edtext{ka`i}{\lemma{ka`i\ts{1}}\Dfootnote{\latintext \textit{hac} Cod.\,Ver.}} <wrism'enw|
p'antac
\edtext{ka`i}{\lemma{\abb{ka`i\ts{2}}}\Dfootnote{\latintext > sz}} per`i p'antwn a>ut`oc
\edtext{krine~i}{\Dfootnote{\latintext \textit{iudicavit} Cod.\,Ver.}}.
\pend
\pstart
\kap{10}tosa'uth
\edtext{\abb{d'e}}{\Dfootnote{\latintext > N}}
\edtext{>estin a>ut~wn}{\lemma{\abb{}}\Dfootnote{\responsio\ a>ut~wn >estin
\latintext A}}
\edtext{\abb{<h}}{\Dfootnote{\latintext A\slin}} >'anoia
\edtext{\abb{ka`i
\edtext{o<'utw}{\Dfootnote{to'utw| \latintext coni. Loofs \greektext toso'utw| \latintext susp. Tetz}} paqe~i
\edtext{sk'otw|}{\Dfootnote{sk'otei \latintext B}} <h
\edtext{\abb{di'anoia}}{\Dfootnote{\latintext e \greektext >anoia \latintext
A\corr}} a>ut~wn >ektet'uflwtai}}{\Dfootnote{\latintext > Cod.\,Ver.}},
\edtext{<'ina m`h dunhj~wsin >ide~in
\edtext{\abb{t`o f~wc}}{\Afootnote{\latintext vgl. Io 1,9; 1Io 2,8}} t~hc
>alhje'iac. o>u
\edtext{suni~asin}{\Dfootnote{sun'hsasin \latintext B}} <~w| l'ogw|
e>'irhtai}{\lemma{<'ina m`h dunhj~wsin \dots\ e>'irhtai}\Dfootnote{\latintext
\textit{ut non videant lucem veritatis nec intellegant} [\textit{a}]\textit{ut vero dictum est}
\latintext Cod.\,Ver.}}\edindex[bibel]{Johannes!1,9|textit}\edindex[bibel]{Johannes I!2,8|textit}
\edtext{((<'ina
\edtext{\abb{ka`i}}{\Dfootnote{\latintext > Cod.\,Ver. del. Tetz}} a>uto`i >en <hm~in
<`en >~wsi))}{\lemma{\abb{}}\Afootnote{\latintext Io 17,21}}\edindex[bibel]{Johannes!17,21}. saf'ec
\edtext{>esti}{\Dfootnote{\latintext \textit{es(t)} Cod.\,Ver.}} di`a
\edtext{\abb{t'i}}{\Dfootnote{+ >en <hm~in \latintext add. Scheidweiler}} ((<'en)); <'oti o<i
>ap'ostoloi pne~uma <'agion to~u jeo~u >'elabon; >all'' <'omwc a>uto`i o>uk
\edtext{>ekl'hjhsan}{\Dfootnote{\latintext coni. Tetz e \textit{sunt vocati} Cod.\,Ver.
\greektext ~>hsan \latintext Thdt.}} pne~uma, o>ud'e tic a>ut~wn
\edtext{>`h}{\lemma{\abb{>`h\ts{1}}}\Dfootnote{\latintext > B Cod.\,Ver.}}
\edtext{l'ogoc}{\Dfootnote{\latintext \textit{sol} Cod.\,Ver.}} >`h sof'ia >`h d'unamic
\edtext{~>hn o>ud`e}{\Dfootnote{o>ud`e \latintext T}} monogen`hc >~hn.
\edtext{((<'wsper))
\edtext{fhs`in ((>eg`w}{\lemma{\abb{}}\Dfootnote{\responsio\ >eg'w fhsin
\latintext A}} ka`i s`u <'en >esmen, o<'utwc ka`i a>uto`i >en <hm~in <`en
>~wsin))}{\lemma{\abb{}}\Afootnote{\latintext Io 17,21}}\edindex[bibel]{Johannes!17,21}.
\edtext{\edtext{\edtext{>all''}{\Dfootnote{>all`a ka`i \latintext AL}} >akrib~wc
di'esteile <h je'ia fwn'h;
\edtext{((>en <hm~in
\edtext{\abb{<`en}}{\Dfootnote{\latintext > Cod.\,Ver.}}
>~wsi))}{\lemma{\abb{}}\Afootnote{\latintext Io 17,21}}}{\lemma{\abb{>all''
\dots\ ~>wsi}}\Dfootnote{\latintext > V}} fhs'in}{\lemma{\abb{>all'' \dots\
fhs'in}}\Dfootnote{\latintext > T}}\edindex[bibel]{Johannes!17,21};
\edtext{\abb{o>uk e~>ipen}}{\Dfootnote{\latintext > L}}; ((<'wsper <hme~ic
\edtext{<'en}{\Dfootnote{\latintext \textit{unum} ex \textit{unus} Cod.\,Ver.\corr}}
>esmen,
\edtext{>eg`w}{\Dfootnote{ka`i >eg`w \latintext rv}} ka`i <o pat'hr));
\edtext{\abb{>all'' <wc o<i}}{\Dfootnote{\latintext coni. Wintjes \greektext >all''
<'ina o<i \latintext rzT \greektext >all'' o<i \latintext BA \greektext >all'' \latintext coni. Tetz \textit{sed}
Cod.\,Ver.}} majhta`i >en
\edtext{\abb{<eauto~ic}}{\Dfootnote{\latintext Bv \greektext
a>uto~ic \latintext ArLT \textit{semetipsos} Cod.\,Ver.}} s'uzugoi ka`i
<hnwm'enoi <'en
\edtext{e>isi}{\Dfootnote{~>wsi \latintext T \textit{sint} Cod.\,Ver.}}
\edtext{\abb{t~h| p'istewc <omolog'ia|}}{\Dfootnote{\latintext coni. Loofs
\greektext t~h p'istei t~h <omolog'ia \latintext BArT \greektext t~h p'istei
ka`i <omolog'ia \latintext z \greektext t~h| p'istei
ka`i t~h| <omolog'ia| \latintext coni. Tetz \textit{fide confessione} Cod.\,Ver.}},
\edtext{ka`i}{\lemma{\abb{}}\Dfootnote{<'ina ka`i
\latintext coni. Loofs \textit{et ut} Cod.\,Ver.}}
\edtext{\edtext{\abb{>en}}{\Dfootnote{\latintext > B}} t~h|}{\lemma{>en t~h|}\Dfootnote{p'antec \latintext coni. Tetz}} q'ariti ka`i
\edtext{\edtext{\abb{t~h|}}{\Dfootnote{\latintext del. Loofs}}
e>usebe'ia|}{\lemma{t~h| e>usebe'ia|}\Dfootnote{>en t~h e>usebe'ia \latintext z \greektext e>usebe'ia
\latintext AT \textit{pietate} Cod.\,Ver.}} t~h| to~u
\edtext{\abb{jeo~u patr`oc}}{\lemma{\abb{}}\Dfootnote{jeo~u p\oline{rs} \latintext BArT \greektext jeo~u ka`i p\oline{rs} \latintext z \textit{patris} Cod.\,Ver.}} ka`i t~h| to~u kur'iou
\edtext{\abb{ka`i}}{\Dfootnote{\latintext > Cod.\,Ver.}} swt~hroc <hm~wn sugqwr'hsei
\edtext{ka`i}{\lemma{ka`i\ts{2}}\Dfootnote{\latintext \textit{hac} Cod.\,Ver.}} >ag'aph| <`en e>~inai dunhj~wsin.
\pend
% \endnumbering
\end{Leftside}
\begin{Rightside}
\begin{translatio}
\beginnumbering
\pstart
\noindent\kapR{1}Vor aller �ffentlichkeit weisen wir jene von uns und verbannen sie aus der katholischen
Kirche, die davon �berzeugt sind, da� Christus wohl Gott sei, aber
eben nicht wahrer Gott,\footnoteA{Vgl. schon das Nicaenum (Dok. \ref{ch:24} = Urk. 24).} da� er Sohn sei, aber nicht wahrer Sohn\footnoteA{Vgl. die Diskussion bei Markell; Dok. \ref{sec:MarkellJulius},5.}, und da� er
zugleich gezeugt und geworden\footnoteA{Eine erstmalige Differenzierung dieser Begriffe ">werden"< und ">zeugen"<; zuvor wurde nur zwischen ">zeugen/werden"< und ">schaffen"< bzw. ">machen"< unterschieden (Nicaenum, Dok. \ref{ch:24} = Urk. 24). Vielleicht liegt nur eine zuf�llige Differenzierung durch die �bersetzung aus dem Lateinischen vor (natus/factus) vor, da sich  keine Rezeption innerhalb dieser theologischen Erkl�rung selbst findet (� 5), wo �ber die Zeugung des Sohnes reflektiert wird. Vgl. dazu die Position der Gegenseite in Ant. IV (Dok. \ref{ch:AntIV}), wo sehr wohl das ">Gezeugtsein"< des Sohnes bekannt wird.} sei.
Sie sind n�mlich gewohnt, so zu denken, wenn sie den
\frq Gezeugten\flq{} bekennen, da, wie sie behaupteten, ">das Gezeugte geworden"< ist
und da sie Christus einen Anfang und
ein Ende geben, was er, da Christus au�erhalb der �onen ist, zwar nicht innerhalb der Zeit, so doch jenseits aller Zeit
habe.\footnoteA{Damit ist auch die erste der drei behandelten Fragen angesprochen, die von Ossius und Protogenes (Dok. \ref{sec:BriefOssiusProtogenes}) mit ">quod erat quando
non erat"> umschrieben wird. Das Problem ist, da� kein
">Arianer"> Christus ein Ende gegeben hat. Nur Markell selbst gibt dem
Menschsein Christi nach dem Abschlu� der Heils�konomie ein Ende (fr.
105; 109; 111 Seibt/Vinzent; vgl. aber auch Dok. \ref{sec:MarkellJulius}). Dieser Kritikpunkt an Markell wird nun umgekehrt den �stlichen Theologen vorgeworfen, da� sie eigentlich diejenigen seien, die dem Sohn ein Ende zuweisen, da sie ihn auch erst entstehen lassen (wenn auch vor der Zeit).}
\pend
\pstart
\kapR{2}Und k�rzlich sind zwei Vipern aus der arianischen Schlange entsprungen,
Valens und Ursacius.\footnoteA{Valens von Mursa und Ursacius von Singidunum waren zu dieser Zeit die
Hauptvertreter der Dreihypostasentheologie im Westen und waren schon als Mitglieder der Mareotis-Kommission fr�he Gegner des Athanasius (vgl. Dok. \ref{sec:BriefJuliusII}, 32; \ref{sec:SerdicaRundbrief},14). Sie haben sich zwar 347, nach der R�ckkehr des Athanasius aus dem Exil, kurz f�r eine Kirchengemeinschaft mit Athanasius ausgesprochen (Ath., apol.\,sec. 58), sp�ter aber vor allem bei der Synode von Sirmium 357 eine f�hrende Rolle inne (Ath., syn. 28; Hil., syn. 11). Seit Beginn der 50er Jahre waren sie die wichtigsten theologischen Ratgeber des Kaisers.} Die r�hmen sich und scheuen
sich nicht zu sagen, da� sie Christen seien und da� der Logos und der Geist gekreuzigt und geopfert worden, gestorben und auferstanden
seien\footnoteA{Valens und Ursacius haben offensichtlich die Unterordnung des Sohnes/Logos unter den Vater mit den entsprechenden Niedrigkeitsaussagen �ber den Inkarnierten begr�ndet (vgl. Ath., Ar. I 45; III 27), worauf sowohl Markell als auch Athanasius mit der hermeneutischen Differenzierung reagierten, da� bei Schriftaussagen jeweils zu pr�fen sei, ob sie sich auf die Gottheit oder den angenommenen Menschen beziehen. Da� hier neben dem Logos auch das Pneuma in die Diskussion einbezogen wird, liegt an der Theologie Markells (s. � 4 mit Anm.), f�r den der Logos in seiner Gottheit Geist ist.} und~-- worauf die Gruppe der
H�retiker mit Nachdruck besteht~-- da� die Hypostasen des Vaters und
des Sohnes und des Heiligen Geistes verschieden und getrennt seien.
\pend
\pstart
\kapR{3}Wir haben folgendes �bernommen und gelehrt bekommen, wir haben folgende
katholische und apostolische �berlieferung, folgenden Glauben und folgendes
Bekenntnis:
Eine Hypostase -- die die Griechen selbst
Usie nennen -- haben der Vater, der Sohn und der Heilige Geist.
\pend
\pstart
\kapR{4}Und wenn sie danach fragen sollten, welche die Hypostase des Sohnes ist, bekennen wir, 
da� diese die ist, die als die eine des Vaters bezeugt wird\footnoteA{Diese im Griechischen
umst�ndliche Ausdrucksweise ist durch die �bersetzung aus dem lateinischen
Original (vielleicht \textit{ut haec erat sola patris confessa}) zu erkl�ren und hat zu
zahlreichen Konjekturen gef�hrt.}, auch 
da� der Vater niemals ohne den Sohn und der Sohn niemals ohne den Vater
gewesen ist oder sein kann, da der Logos Geist ist\footnoteA{Diese auf den ersten Blick befremdliche Begr�ndung erkl�rt sich aus der Theologie Markells, f�r den Geist eher das Wesen der Gottheit als eine dritte trinitarische Person des Heiligen Geists meint. Wenn der Logos Geist ist, so ist damit seine Gottheit ausgesagt und die Einheit in Gott beschrieben (fr. 47; 48; 61; 64; 67; 73 Seibt/Vinzent; vgl. Eus., e.\,th. II 1; III 5).}. Es
ist n�mlich absolut widersinnig zu sagen, da� der Vater einmal nicht Vater
gewesen sei, da ja klar ist, da� kein Vater ohne einen Sohn so angesprochen werden
oder als Vater existieren kann, denn der Sohn selbst bezeugt: ">Ich bin
im Vater und der Vater ist in mir"< und ">Ich und der Vater sind eins"<.
\pend
\pstart
\kapR{5}Niemand von uns leugnet, da� er gezeugt ist, aber da� er ganz und gar als Gesch�pf
gezeugt ist, wie die unsichtbaren und die sichtbaren Dinge als
gezeugt bezeichnet werden, er, der Sch�pfer der Erzengel und Engel, der Welt und
des Menschengeschlechtes, da es hei�t: ">Die Weisheit, die Sch�pferin aller Dinge,
lehrte mich"< und ">Alles ist durch ihn geworden"<. Auch k�nnte er n�mlich nicht
ewig\footnoteA{Die Gleichewigkeit des Sohnes wird von Markell im Brief an
Julius (Dok. \ref{sec:MarkellJulius}) betont. Der Satz bringt zum Ausdruck, da� eine Leugnung der
wesenhaften Ursprungslosigkeit sich nach westlichem Verst�ndnis stets in einer Differenz auch der
Arten von Ewigkeit zwischen Vater und Sohn �u�ern mu�, weshalb der Satz
\griech{>~hn pote <'ote o>uk >~hn} im Westen stets zeitlich interpretiert wurde
und die Verwerfung des Satzes \griech{>~hn pote qr'onoc <'ote o>uk >~hn} durch
die �stlichen Bisch�fe nicht als Beleg f�r die Gleichewigkeit des Sohnes
akzeptiert wurde.} sein, wenn er einen Anfang h�tte, denn der ewig Seiende hat
als Gott Logos keinen Anfang und erf�hrt kein Ende.
\pend
\pstart
\kapR{6}Wir behaupten nicht, da� der Vater Sohn ist, auch nicht umgekehrt, da� der
Sohn Vater ist. Vielmehr ist der Vater Vater und der Sohn Sohn des Vaters.\footnoteA{Dieser Vorwurf wird in der Ekthesis macrostichos referiert (Dok. \ref{ch:Makrostichos},11). Vgl. zum Sabellianismus-Vorwurf allg. Dok. \ref{sec:Eustathius},1 Anm.} Wir
bekennen, da� der Sohn die Kraft des Vaters ist.
Wir bekennen, da� er das Wort Gottes des Vaters ist, neben dem kein anderer ist, und
da� das Wort wahrer Gott, Weisheit und Kraft ist\footnoteA{Vgl. Dok. \ref{sec:MarkellJulius},5.}.
Wir lehren ihn aber als wahren Sohn, aber wir reden vom Sohn nicht in dem Sinn,
wie die �brigen als S�hne\footnoteA{Die Pr�dikation des Logos als Sohn und der Gedanke
der Zeugung in der Pr�existenz widerspricht nicht der These vom Einflu� Markells
auf diesen Text. Wenn jede Analogie zum menschlichen
Sohnesbegriff ausgeschlossen war, konnte Markell ihn aber offenbar f�r den
pr�existenten Logos verwenden, wie u.a. sein Brief an Julius best�tigt (Dok. \ref{sec:MarkellJulius},5).} bezeichnet werden, da jene aufgrund von Adoption,
Wiedergeburt oder Wertsch�tzung S�hne genannt werden, nicht aufgrund der
einen Hypostase, die dem Vater und dem Sohn eigen ist.
\pend
\pstart
Wir bekennen ihn als Eingeborenen und Erstgeborenen,\footnoteA{Gerade die nun
folgende Differenzierung der beiden Begriffe nach Gottheit und Menschheit
best�tigt den Einflu� Markells. Seine Gegner haben beide Begriffe dem
pr�existenten Logos zugesprochen, was er entschieden ablehnt (Dok. \ref{sec:AntII},1,2; fr. 10; 12--15
Seibt/Vinzent); vgl. ferner Ath., Ar. II 62--64.} aber als ">eingeboren"< bekennen wir den Logos, der allzeit im Vater war
und ist. Das ">erstgeboren"< schreiben wir dem Menschen zu. Es bezieht sich aber
auf die neue Sch�pfung, da er auch der ">Erstgeborene von den Toten"< ist.
\pend
\pstart
Wir bekennen, da� ein Gott ist; wir bekennen eine Gottheit des Vaters und
des Sohnes.
\pend
\pstart
\kapR{7}Und niemand leugnet je, da� der Vater gr��er ist als der Sohn, nicht aufgrund
einer anderen Hypostase, nicht wegen irgendeines
Unterschiedes, sondern weil der Name des Vaters als solcher gr��er ist als der
des Sohnes.
Folgendes aber ist ihre blasphemische und verderbte Auslegung: Sie behaupten veranla�t durch diese Schriftstelle,
er habe das ">Ich und der Vater sind eins"< wegen
der �bereinstimmung und Meinungsgleichheit\footnoteA{Markell wendet sich gegen
die Interpretation von Io 10,30 als Gesinnungs�bereinstimmung bereits in seiner
Auseinandersetzung mit Asterius (fr. 74, 75, 125 Seibt/Vinzent). Auch in Dok. \ref{sec:AntII},6 wird die Einheit
von Vater und Sohn durch \greektext sumfwn'ia \latintext erkl�rt, vgl. ebenfalls Dok. \ref{ch:Makrostichos},15 und Ath., Ar. III 10.} gesagt. Wir
Katholiken aber haben alle ihre t�richte und kl�gliche Ansicht verdammt. Wie
bei sterblichen Menschen, die erst anfangen zu streiten, im Zorn aufeinanderprallen und
dann wieder Frieden schlie�en, so k�nnten auch, sagen sie,
zwischen dem Vater, dem allm�chtigen Gott, und dem Sohn Entzweiungen und
Uneinigkeit sein, was zu denken und zu erw�gen v�llig absurd ist. Wir
aber glauben, sind fest davon �berzeugt und denken folgenderma�en, da� die heilige Stimme gesagt hat
">Ich und der Vater sind eins"< auch aufgrund der Einheit der Hypostase, welche die eine des Vater
und die eine des Sohnes ist.
\pend
\pstart
\kapR{8}Auch dies glauben wir aber, da� dieser allzeit ohne Anfang und Ende mit dem
Vater herrscht und seine Herrschaft weder Zeit noch Untergang
hat,\footnoteA{Vgl. zu diesem umstrittenen Punkt Dok. \ref{sec:MarkellJulius}. Die Fragmente,
in denen Markell im Anschlu� an 1 Cor 15,24--28 von einem Ende der Herrschaft
Christi spricht, sind nicht auf ein Ende der Herrschaft des Logos zu beziehen,
sondern auf eine Trennung des Logos von seiner Menschennatur (fr. 105; 109; 111 Seibt/Vinzent).}
da das, was ewig ist, weder einmal anfangen hat noch aufh�ren kann zu existieren.
\pend
\pstart
\kapR{9}Wir glauben und nehmen an den Parakleten, den Heiligen Geist, den uns der Herr
selbst verhei�en und gesandt hat. Und dieser (der Geist), glauben wir, ist gesandt worden.
Und dieser hat nicht gelitten, sondern der Mensch, den er anzog,\footnoteA{Zur Identifizierung des Logos mit Geist oben � 2 und 4 mit Anmerkungen.} den er aus Maria,
der Jungfrau, angenommen hat, den Menschen, der die F�higkeit besitzt zu leiden,
da der Mensch sterblich ist, Gott aber unsterblich.
Wir glauben, da� am dritten Tag nicht der Gott im Menschen,
sondern der Mensch in Gott auferstanden ist, den er auch seinem Vater als Gabe
dargebracht hat, und den er befreit hat.
Wir glauben, da� er selbst zur rechten und festgesetzten Zeit alle und �ber alle
richten wird.
\pend
\pstart
\kapR{10}So gro� aber ist ihre Unwissenheit und so tiefe Finsternis hat sich 
auf ihr Denken gelegt,
da� sie das Licht der Wahrheit nicht sehen k�nnen. Sie verstehen
nicht, in welchem Sinne gesagt ist ">damit auch sie in uns eins seien"<. Klar
ist, weswegen gesagt ist ">eins"<: Weil die Apostel den Heiligen Geist Gottes
empfangen haben. Aber dennoch wurden sie selbst nicht Geist genannt, und keiner von
ihnen war Logos,
Weisheit oder Kraft und auch nicht eingeboren. Es hei�t: ">Wie ich und du eins
sind, so sollen auch sie in uns eins sein"<. Aber die g�ttliche Stimme
unterschied genau: ">In uns sollen sie eins sein"<, sagt sie; sie meint nicht:
">Wie wir eins sind, ich und der Vater"<, sondern, wie die J�nger untereinander
verbunden und vereinigt im Bekenntnis des Glaubens eins sind, sollen sie auch in
der Gnade und der Verehrung Gottes des Vaters und in der Vergebung und Liebe
des Herrn und Erl�sers eins sein k�nnen.\footnoteA{Zu dieser Diskussion um Io 17,20--23 vgl. Ath., Ar. III 19--25; 27; Eusebius, e.\,th. III 18,4--19,4.}
\pend
\endnumbering
\end{translatio}
\end{Rightside}
\Columns
\end{pairs}
\selectlanguage{german}
% \thispagestyle{empty}
%%%% Input-Datei OHNE TeX-Pr�ambel %%%%
% \cleartooddpage
\section[Liste der Unterschriften unter der theologischen Erkl�rung der ">westlichen"< Synode][Unterschriften unter der theologischen Erkl�rung der ">westlichen"< Synode]{Liste der Unterschriften unter der theologischen Erkl�rung der ">westlichen"< Synode}
% \label{sec:43.2}
\label{sec:SerdikaUnterschriften}
\begin{praefatio}
  \begin{description}
  \item[Herbst 343/um 346]Zum Datum der Synode von Serdica vgl. die
    Einleitung zu Dok. \ref{ch:SerdicaEinl}. Zur Datierung der
    Erg�nzungen der Unterschriftenliste vgl. den Abschnitt zur
    �berlieferung.
  \item[�berlieferung]Nur bei den ersten 78 Namen handelt es sich um
    Unterschriften der an der Synode von Serdica teilnehmenden
    Bisch�fe. Die �brigen Namen, die jeweils unter dem Namen einer
    Provinz zusammengefa�t sind, enstammen von Athanasius der
    eigentlichen serdicensischen Liste redaktionell hinzugef�gten
    Unterschriftenlisten sp�terer, die Beschl�sse der Synode von
    Serdica best�tigende Synoden in den Provinzen Gallien, Africa,
    �gypten, Italia annonaria, Zypern und Palaestina.  Die Namen
    Nr. 79--112 sind die Unterschriften der Teilnehmer einer gallischen Synode, die
    eventuell am 12. Mai 346 stattgefunden hat (CChr.SL 148, 27,1
    Munier; das Datum l��t sich aus den gef�lschten Akten der K�lner
    Synode herleiten, vgl. dazu \cite[41]{Brennecke:Euphrates};
    \cite[181]{Crabbe:Cologne}; so schon Duchesne, Fastes �piscopaux
    de l'ancienne Gaule.).  Die 94 Namen Nr. 149--242 sind die
    Unterschriften einer �gyptischen Synode, die nach der R�ckkehr des
    Athanasius nach Alexandrien am 21. Oktober 346 (ind.\,ep.\,fest
    18; hist.\,Ath. 1,1) entweder noch im selben oder erst im
    folgenden Jahr stattgefunden hat (Soz., h.\,e. IV 1,3;
    vgl. ep.\,fest. 19, in dem einige der Unterzeichner als neu
    eingesetzte Bisch�fe genannt werden). Sicherlich noch im Jahr 346,
    genauer im Sommer oder Fr�hherbst w�hrend der R�ckreise des
    Athanasius von Antiochien nach Alexandrien, hat eine Synode in
    Jerusalem (Namen Nr. 270--284) stattgefunden, deren Schreiben an
    die �gyptischen und libyschen Bisch�fe Athanasius in
    apol.\,sec. 57 (136,24--137,20 Opitz) mit Verweis auf die
    wiederholte Unterschriftenliste �berliefert.  �ber entsprechende
    Synoden in Africa, Italia annonaria und auf Zypern ist sonst
    nichts bekannt.  Bei den hinzugef�gten Listen partikularer Synoden
    ist mit redaktioneller Bearbeitung von Seiten des Athanasius zu
    rechnen, der sicherlich Doppelungen der Namen von Bisch�fen, die
    als Teilnehmer der Synode von Serdica bereits in Serdica selbst
    unterzeichnet hatten, vermeiden wollte.
  \item[Fundstelle]Ath., apol.\,sec. 48--50,3 (\editioncite[123,26--132,3]{Opitz1935})\specialindex{quellen}{section}{Athanasius!apol.\,sec.!48--50,3}
  \end{description}
\end{praefatio}
\begin{pairs}
\selectlanguage{polutonikogreek}
\begin{Leftside}
% \beginnumbering
\pstart
\hskip -.85em\edtext{\abb{}}{\killnumber\Cfootnote{\hskip -1em\dt{Ath.(BKO RE)}}}
\begin{footnotesize}
\hskip -1em Ta~uta gr'ayasa <h >en Sardik~h|\index[synoden]{Serdica!a. 343} s'unodoc >ap'esteile ka`i pr`oc to`uc m`h
dunhj'entac >apant~hsai, ka`i geg'onasi ka`i a>uto`i s'umyhfoi to~ic krije~isi.
t~wn d`e >en t~h| sun'odw| gray'antwn ka`i t~wn >'allwn >episk'opwn t`a
>on'omat'a >esti
\edtext{t'ade}{\Dfootnote{ta~uta \latintext KORE}};
\end{footnotesize}
\pend
\pstart
\noindent \dt{1.} <'Osioc \edindex[namen]{Ossius!Bischof von Cordoba}>ap`o Span'iac
\pend
\pstart
\noindent \dt{2.} >Io'ulioc <R'wmhc\edindex[namen]{Julius!Bischof von Rom} di'' >Arqid'amou\edindex[namen]{Archidamus!Presbyter in Rom} ka`i Filox'enou\edindex[namen]{Philoxenus!Presbyter in Rom} presbut'erwn
\pend
\pstart
\noindent \dt{3.} Prwtog'enhc\edindex[namen]{Protogenes!Bischof von Serdica}
\edtext{Serdik~hc}{\Dfootnote{Sardik~hc \latintext KO}}
\pend
\pstart
\noindent \dt{4.} Gaud'entioc\edindex[namen]{Gaudentius!Bischof von Na"issus}
\pend
\pstart
\noindent \dt{5.} Maked'onioc\edindex[namen]{Macedonius!Bischof von Ulpiana}
\pend
\pstart
\noindent \dt{6.} Seb~hroc\edindex[namen]{Severus!Bischof von Chalcis|dub}\edindex[namen]{Severus!Bischof von Ravenna|dub}
\pend
\pstart
\noindent \dt{7.} \edtext{Prait'extatoc}{\Dfootnote{Prht'extatoc \dt{K*}}}\edindex[namen]{Praetextatus!Bischof von Barcilona}
\pend
\pstart
\noindent \dt{8.} O>urs'ikioc \edindex[namen]{Ursacius!Bischof von Brixia}
\pend
\pstart
\noindent \dt{9.} Lo'ukilloc\edindex[namen]{Lucius!Bischof von Hadrianopolis|dub}\edindex[namen]{Lucius!Bischof von Verona|dub}
\pend
\pstart
\noindent \dt{10.} E>ug'enioc\edindex[namen]{Eugenius!Bischof von Heraclea Lyncestis}
\pend
\pstart
\noindent \dt{11.} Bit'alioc\edindex[namen]{Vitalis!Bischof von Aquae|dub}\edindex[namen]{Vitalis!Bischof ">Vertaresis"<|dub}
\pend
\pstart
\noindent \dt{12.} Kalep'odioc\edindex[namen]{Calepodius!Bischof von Neapolis}
\pend
\pstart
\noindent \dt{13.} Flwr'entioc \edindex[namen]{Florentius!Bischof von Emerita Augusta}
\pend
\pstart
\noindent \dt{14.} B'assoc \edindex[namen]{Bassus!Bischof von Diocletianopolis}
\pend
\pstart
\noindent \dt{15.} Bik'entioc \edindex[namen]{Vincentius!Bischof von Capua}
\pend
\pstart
\noindent \dt{16.} Sterk'orioc \edindex[namen]{Stercorius!Bischof von Canusium}
\pend
\pstart
\noindent \dt{17.} Pall'adioc \edindex[namen]{Palladius!Bischof von Dium}
\pend
\pstart
\noindent \dt{18.} \edtext{Domitian'oc}{\Dfootnote{Dometian'oc \latintext
KO}}\edindex[namen]{Domitianus!Bischof von Asturica}
\pend
\pstart
\noindent \dt{19.} \edtext{Qalb'hc}{\Dfootnote{Qalb'ic \latintext BRE}}\edindex[namen]{Calvus!Bischof von Castra Martis}
\pend
\pstart
\noindent \dt{20.} Ger'ontioc \edindex[namen]{Gerontius!Bischof von Beroea}
\pend
\pstart
\noindent \dt{21.} Prot'asioc \edindex[namen]{Protasius!Bischof von Mailand}
\pend
\pstart
\noindent \dt{22.} \edtext{E>ul'ogioc}{\Dfootnote{E>'ulogoc \latintext
BRE}}\edindex[namen]{Eulogius!Bischof}
\pend
\pstart
\noindent \dt{23.} Porf'urioc\edindex[namen]{Porphyrius!Bischof von Philippi}
\pend
\pstart
\noindent \dt{24.} Di'oskoroc\edindex[namen]{Dioscurus!Bischof von Therasia}
\pend
\pstart
\noindent \dt{25.} Z'wsimoc\edindex[namen]{Zosimus!Bischof|dub}\edindex[namen]{Zosimus!Bischof von Horreum Margi|dub}\edindex[namen]{Zosimus!Bischof von Lychnidus|dub}
\pend
\pstart
\noindent \dt{26.} >Iannou'arioc\edindex[namen]{Januarius!Bischof von Beneventum}
\pend
\pstart
\noindent \dt{27.} Z'wsimoc\edindex[namen]{Zosimus!Bischof|dub}\edindex[namen]{Zosimus!Bischof von Horreum Margi|dub}\edindex[namen]{Zosimus!Bischof von Lychnidus|dub}
\pend
\pstart
\noindent \dt{28.} >Al'exandroc \edindex[namen]{Alexander!Bischof von Corone|dub}\edindex[namen]{Alexander!Bischof von Cyparissia|dub}\edindex[namen]{Alexander!Bischof von Larisa|dub}
\pend
\pstart
\noindent \dt{29.} E>ut'uqioc\edindex[namen]{Eutychius!Bischof von Methone}
\pend
\pstart
\noindent \dt{30.} Swkr'athc\edindex[namen]{Socras!Bischof von Phoebia am Asopus}
\pend
\pstart
\noindent \dt{31.} Di'odwroc\edindex[namen]{Diodorus!Bischof von Tenedus}
\pend
\pstart
\noindent \dt{32.} Mart'urioc\edindex[namen]{Martyrius!Bischof|dub}\edindex[namen]{Martyrius!Bischof von Naupactus|dub}
\pend
\pstart
\noindent \dt{33.} E>uj'hrioc\edindex[namen]{Eutherius!Bischof in Pannonien|dub}\edindex[namen]{Eutherius!Bischof von Ganus|dub}
\pend
\pstart
\noindent \dt{34.} E>'ukarpoc\edindex[namen]{Eucarpus!Bischof|dub}\edindex[namen]{Eucarpus!Bischof von Opus|dub}
\pend
\pstart
\noindent \dt{35.} >Ajhn'odwroc\edindex[namen]{Athenodorus!Bischof von Elatia}
\pend
\pstart
\noindent \dt{36.} E>irhna~ioc\edindex[namen]{Irenaeus!Bischof von Scyrus}
\pend
\pstart
\noindent \dt{37.} >Ioulian'oc\edindex[namen]{Julianus!Bischof von Thebae Heptapylus}
\pend
\pstart
\noindent \dt{38.} >Al'upioc\edindex[namen]{Alypius!Bischof von Megara}
\pend
\pstart
\noindent \dt{39.} >Iwn~ac\edindex[namen]{Jonas!Bischof von Parthicopolis}
\pend
\pstart
\noindent \dt{40.} >A'etioc\edindex[namen]{A"etius!Bischof von Thessalonike}
\pend
\pstart
\noindent \dt{41.} <Resto~utoc\edindex[namen]{Restutus!Bischof}
\pend
\pstart
\noindent \dt{42.} \edtext{Markell\ladd{~in}oc}{\lemma{\abb{Mark'ellos}}\Dfootnote{\dt{coni. Erl.} Markell~inoc \dt{codd.}}}\edindex[namen]{Markell!Bischof von Ancyra}
\pend
\pstart
\noindent \dt{43.} >Aprian'oc\edindex[namen]{Aprianus!Bischof von Poetovio}
\pend
\pstart
\noindent \dt{44.} Bit'alioc\edindex[namen]{Vitalis!Bischof von Aquae|dub}\edindex[namen]{Vitalis!Bischof ">Vertaresis"<|dub}
\pend
\pstart
\noindent \dt{45.} O>u'alhc \edindex[namen]{Valens!Bischof von Oescus}
\pend
\pstart
\noindent \dt{46.} <Ermog'enhc\edindex[namen]{Hermogenes!Bischof von Sicyon}
\pend
\pstart
\noindent \dt{47.} K'astoc\edindex[namen]{Castus!Bischof von Caesaraugusta}
\pend
\pstart
\noindent \dt{48.} Dometian'oc \edindex[namen]{Dometianus!Bischof von Acaria Constantias (?)}
\pend
\pstart
\noindent \dt{49.} \edtext{\abb{Fourtounat'i\Ladd{an}oc}}{\Dfootnote{\dt{Erl.} Fourtoun'atoc \dt{KE}
Fortoun'atioc \dt{R}}}\edindex[namen]{Fortunatianus!Bischof von Aquileia}
\pend
\pstart
\noindent \dt{50.} M'arkoc \edindex[namen]{Marcus!Bischof von Siscia}
\pend
\pstart
\noindent \dt{51.} >Annian'oc \edindex[namen]{Annianus!Bischof von Castellona}
\pend
\pstart
\noindent \dt{52.} <Hli'odwroc \edindex[namen]{Heliodorus!Bischof von Nicopolis}
\pend
\pstart
\noindent \dt{53.} Mousa~ioc\edindex[namen]{Musaeus!Bischof von Thebae Phthiotides}
\pend
\pstart
\noindent \dt{54.} >Ast'erioc\edindex[namen]{Asterius!Bischof in Arabien}
\pend
\pstart
\noindent \dt{55.} Parhg'orioc \edindex[namen]{Paregorius!Bischof von Scupi}
\pend
\pstart
\noindent \dt{56.} Plo'utarqoc \edindex[namen]{Plutarchus!Bischof von Patras}
\pend
\pstart
\noindent \dt{57.} <Um'enaioc \edindex[namen]{Hymenaeus!Bischof von Hypata}
\pend
\pstart
\noindent \dt{58.} >Ajan'asioc \edindex[namen]{Athanasius!Bischof von Alexandrien}
\pend
\pstart
\noindent \dt{59.} Lo'ukioc\edindex[namen]{Lucius!Bischof von Adrianopel|dub}\edindex[namen]{Lucius!Bischof von Verona|dub}
\pend
\pstart
\noindent \dt{60.} \edtext{>Am'antioc}{\Dfootnote{>Am'anteioc \latintext
RE\corr}}\edindex[namen]{Amantius!Bischof von Viminacium}
\pend
\pstart
\noindent \dt{61.} >'Areioc\edindex[namen]{Arius!Bischof von Petra}
\pend
\pstart
\noindent \dt{62.} >Askl'hpioc \edindex[namen]{Asclepas!Bischof von Gaza}
\pend
\pstart
\noindent \dt{63.} Dion'usioc \edindex[namen]{Dionysius!Bischof von Elis}
\pend
\pstart
\noindent \dt{64.} M'aximoc \edindex[namen]{Maximus!Bischof von Luca}
\pend
\pstart
\noindent \dt{65.} Tr'ufwn\edindex[namen]{Tryphon!Bischof von Macaria}
\pend
\pstart
\noindent \dt{66.} >Al'exandroc\edindex[namen]{Alexander!Bischof von Corone|dub}\edindex[namen]{Alexander!Bischof von Cyparissia|dub}\edindex[namen]{Alexander!Bischof von Larisa|dub}
\pend
\pstart
\noindent \dt{67.} >Ant'igonoc \edindex[namen]{Antigonus!Bischof von Pallene}
\pend
\pstart
\noindent \dt{68.} A>ilian'oc \edindex[namen]{Aelianus!Bischof von Gortyna}
\pend
\pstart
\noindent \dt{69.} P'etroc \edindex[namen]{Petrus!Bischof}
\pend
\pstart
\noindent \dt{70.} S'umforoc \edindex[namen]{Symphorus!Bischof von Hierapytna}
\pend
\pstart
\noindent \dt{71.} Mous'wnioc \edindex[namen]{Musonius!Bischof von Heraclium}
\pend
\pstart
\noindent \dt{72.} E>'utuqoc \edindex[namen]{Eutychus!Bischof}
\pend
\pstart
\noindent \dt{73.} \edtext{Filol'ogioc}{\Dfootnote{Filol'ogoc \latintext
E}}\edindex[namen]{Philologius!Bischof}
\pend
\pstart
\noindent \dt{74.} Spoud'asioc\edindex[namen]{Spudasius!Bischof}
\pend
\pstart
\noindent \dt{75.} Z'wsimoc\edindex[namen]{Zosimus!Bischof|dub}\edindex[namen]{Zosimus!Bischof von Horreum Margi|dub}\edindex[namen]{Zosimus!Bischof von Lychnidus|dub}
\pend
\pstart
\noindent \dt{76.} Patr'ikioc\edindex[namen]{Patricius!Bischof}
\pend
\pstart
\noindent \dt{77.} >Ad'olioc\edindex[namen]{Adolius!Bischof}
\pend
\pstart
\noindent \dt{78.} Sapr'ikioc\edindex[namen]{Sapricius!Bischof}
\pend
\pstart
\vspace{.5\baselineskip}\noindent\textit{Gall'iac}
\pend
\pstart
\noindent \dt{79.} \edtext{\abb{Maximi\ladd{a}n'oc}}{\Dfootnote{\dt{coni. Opitz} Maximian'oc
\dt{codd.}}}\edindex[namen]{Maximinus!Bischof von Trier}
\pend
\pstart
\noindent \dt{80.} \edtext{\abb{Bhr'issimoc}}{\Dfootnote{\latintext coni. Opitz \greektext
Bhr'isshmoc \latintext RE \greektext Bhr'isimoc \latintext BKO}}\edindex[namen]{Verissimus!Bischof von Lugdunum}
\pend
\pstart
\noindent \dt{81.} B'hktouroc\edindex[namen]{Victurus!Bischof in Gallien (Metz?)}
\pend
\pstart
\noindent \dt{82.} Balent~inoc\edindex[namen]{Valentinus!Bischof in Gallien (Arles?)}
\pend
\pstart
\noindent \dt{83.} Dhsid'erioc\edindex[namen]{Desiderius!Bischof in Gallien (der Lingonen?)}
\pend
\pstart
\noindent \dt{84.} E>ul'ogioc\edindex[namen]{Eulogius!Bischof in Gallien (Amiens?)}
\pend
\pstart
\noindent \dt{85.} Sarb'atioc\edindex[namen]{Servatius!Bischof von Tongern}
\pend
\pstart
\noindent \dt{86.} Dusk'olioc\edindex[namen]{Dyscolius!Bischof in Gallien (der Remer?)}
\pend
\pstart
\noindent \dt{87.} Souper'iwr\edindex[namen]{Superior!Bischof in Gallien (der Nervier?)}
\pend
\pstart
\noindent \dt{88.} Merko'urioc\edindex[namen]{Mercurius!Bischof in Gallien (Soissons?)}
\pend
\pstart
\noindent \dt{89.} \edtext{Diklopet'oc}{\Dfootnote{Dhklopet'oc \latintext
BRE}}\edindex[namen]{Diclopetus!Bischof in Gallien}
\pend
\pstart
\noindent \dt{90.} E>us'ebioc\edindex[namen]{Eusebius!Bischof in Gallien (Rouen?)}
\pend
\pstart
\noindent \dt{91.} Sebhr~inoc\edindex[namen]{Severinus!Bischof in Gallien (Sens?)}
\pend
\pstart
\noindent \dt{92.} S'aturoc\edindex[namen]{Satyrus!Bischof in Gallien}
\pend
\pstart
\noindent \dt{93.} Mart~inoc\edindex[namen]{Martinus!Bischof in Gallien (Mainz?)}
\pend
\pstart
\noindent \dt{94.} Pa~uloc\edindex[namen]{Paulus!Bischof in Gallien}
\pend
\pstart
\noindent \dt{95.} \edtext{>Optatian'oc}{\Dfootnote{>Optatiatian'oc \dt{BKO} >Optatatian'oc
\dt{E}}}\edindex[namen]{Optatianus!Bischof in Gallien (Troyes?)}
\pend
\pstart
\noindent \dt{96.} Nik'asioc\edindex[namen]{Nicasius!Bischof in Gallien}
\pend
\pstart
\noindent \dt{97.} B'iktwr\edindex[namen]{Victor!Bischof in Gallien (Worms?)}
\pend
\pstart
\noindent \dt{98.} Sempr'wnioc\edindex[namen]{Sempronius!Bischof in Gallien}
\pend
\pstart
\noindent \dt{99.} \edtext{\abb{Balerian'oc}}{\Dfootnote{\dt{K} Balerin'oc \dt{BO
RE}}}\edindex[namen]{Valerianus!Bischof in Gallien (Auxerre?)}
\pend
\pstart
\noindent \dt{100.} P'akatoc\edindex[namen]{Pacatus!Bischof in Gallien}
\pend
\pstart
\noindent \dt{101.} >Iess~hc\edindex[namen]{Jesses!Bischof in Gallien (Speyer?)}
\pend
\pstart
\noindent \dt{102.} >Ar'istwn\edindex[namen]{Ariston!Bischof in Gallien}
\pend
\pstart
\noindent \dt{103.} Simpl'ikioc\edindex[namen]{Simplicius!Bischof in Gallien (Autun?)}
\pend
\pstart
\noindent \dt{104.} Metian'oc\edindex[namen]{Metianus!Bischof in Gallien}
\pend
\pstart
\noindent \dt{105.} >'Amantoc\edindex[namen]{Amandus!Bischof in Gallien (Stra�burg?)}
\pend
\pstart
\noindent \dt{106.} \edtext{\abb{A\Ladd{>i}millian'oc}}{\Dfootnote{\dt{coni. Opitz}
>Amillian'oc \dt{codd.}}}\edindex[namen]{Aemillianus!Bischof in Gallien}
\pend
\pstart
\noindent \dt{107.} \edtext{>Ioustinian'oc}{\Dfootnote{>Ioustinianian'oc \latintext
E}}\edindex[namen]{Justinianus!Bischof in Gallien (Basel?)}
\pend
\pstart
\noindent \dt{108.} Biktwr~inoc\edindex[namen]{Victorinus!Bischof in Gallien (Paris?)}
\pend
\pstart
\noindent \dt{109.} Satorn~iloc\edindex[namen]{Satornilus!Bischof in Gallien}
\pend
\pstart
\noindent \dt{110.} >Abound'antioc\edindex[namen]{Abundantius!Bischof in Gallien}
\pend
\pstart
\noindent \dt{111.} Dwnatian'oc\edindex[namen]{Donatianus!Bischof in Gallien}
\pend
\pstart
\noindent \dt{112.} M'aximoc\edindex[namen]{Maximus!Bischof in Gallien}
\pend
\pstart
\vspace{.5\baselineskip}\noindent\textit{>Afrik~hc}
\pend
\pstart
\noindent \dt{113.} N'essoc\edindex[namen]{Nessus!Bischof in Africa}
\pend
\pstart
\noindent \dt{114.} Gr'atoc\edindex[namen]{Gratus!Bischof von Carthago}
\pend
\pstart
\noindent \dt{115.} Meg'asioc\edindex[namen]{Megasius!Bischof in Africa}
\pend
\pstart
\noindent \dt{116.} Kolda~ioc\edindex[namen]{Coldaeus!Bischof in Africa}
\pend
\pstart
\noindent \dt{117.} <Rogatian'oc\edindex[namen]{Rogatianus!Bischof in Africa}
\pend
\pstart
\noindent \dt{118.} Kons'ortioc\edindex[namen]{Consortius!Bischof in Africa}
\pend
\pstart
\noindent \dt{119.} <Rouf~inoc\edindex[namen]{Rufinus!Bischof in Africa}
\pend
\pstart
\noindent \dt{120.} Mann~inoc\edindex[namen]{Manninus!Bischof in Africa}
\pend
\pstart
\noindent \dt{121.} \edtext{Kessilian'oc}{\Dfootnote{Kesshlian'oc \latintext
RE}}\edindex[namen]{Cessilianus!Bischof in Africa}
\pend
\pstart
\noindent \dt{122.} >Erennian'oc\edindex[namen]{Erennianus!Bischof in Africa}
\pend
\pstart
\noindent \dt{123.} Marian'oc\edindex[namen]{Marianus!Bischof in Africa}
\pend
\pstart
\noindent \dt{124.} O>ual'erioc\edindex[namen]{Valerius!Bischof in Africa}
\pend
\pstart
\noindent \dt{125.} Dun'amioc \edindex[namen]{Dynamius!Bischof in Africa}
\pend
\pstart
\noindent \dt{126.} Muz'onioc\edindex[namen]{Musonius!Bischof in Africa}
\pend
\pstart
\noindent \dt{127.} >Io~ustoc\edindex[namen]{Justus!Bischof in Africa}
\pend
\pstart
\noindent \dt{128.} Kelest~inoc\edindex[namen]{Celestinus!Bischof in Africa}
\pend
\pstart
\noindent \dt{129.} Kuprian'oc\edindex[namen]{Cyprianus!Bischof in Africa}
\pend
\pstart
\noindent \dt{130.} B'iktwr\edindex[namen]{Victor!Bischof in Africa\dt{2}}
\pend
\pstart
\noindent \dt{131.} >Onor~atoc\edindex[namen]{Honoratus!Bischof in Africa}
\pend
\pstart
\noindent \dt{132.} Mar~inoc\edindex[namen]{Marinus!Bischof in Africa}
\pend
\pstart
\noindent \dt{133.} Pant'agajoc\edindex[namen]{Pantagathus!Bischof in Africa}
\pend
\pstart
\noindent \dt{134.} \edtext{F'hlix}{\Dfootnote{F'ilhx \latintext KO}}\edindex[namen]{Felix!Bischof in Africa (Baiana?)1}
\pend
\pstart
\noindent \dt{135.} Ba'udioc\edindex[namen]{Baudius!Bischof in Africa}
\pend
\pstart
\noindent \dt{136.} L'iber\edindex[namen]{Liber!Bischof in Africa}
\pend
\pstart
\noindent \dt{137.} Kap'itwn\edindex[namen]{Capito!Bischof in Africa}
\pend
\pstart
\noindent \dt{138.} \edtext{Minerb'alhc}{\Dfootnote{Miners'alhc \dt{B} Minerb'alic
\dt{RE}}}\edindex[namen]{Minervalis!Bischof in Africa}
\pend
\pstart
\noindent \dt{139.} K'osmoc\edindex[namen]{Cosmus!Bischof in Africa}
\pend
\pstart
\noindent \dt{140.} B'iktwr\edindex[namen]{Victor!Bischof in Africa}
\pend
\pstart
\noindent \dt{141.} <Esper'iwn\edindex[namen]{Hesperion!Bischof in Africa}
\pend
\pstart
\noindent \dt{142.} \edtext{F'hlix}{\Dfootnote{F'ilhx \latintext KO}}\edindex[namen]{Felix!Bischof in Africa (Baiana?)2}
\pend
\pstart
\noindent \dt{143.} Sebhrian'oc\edindex[namen]{Severianus!Bischof in Africa}
\pend
\pstart
\noindent \dt{144.} >Opt'antioc\edindex[namen]{Optantius!Bischof in Africa}
\pend
\pstart
\noindent \dt{145.} <'Esperoc\edindex[namen]{Hesperus!Bischof in Africa}
\pend
\pstart
\noindent \dt{146.} Fid'entioc\edindex[namen]{Fidentius!Bischof in Africa}
\pend
\pstart
\noindent \dt{147.} Salo'ustioc\edindex[namen]{Salustius!Bischof in Africa}
\pend
\pstart
\noindent \dt{148.} Pasq'asioc\edindex[namen]{Paschasius!Bischof in Africa}
\pend
\pstart
\vspace{.5\baselineskip}\noindent\textit{A>ig'uptou}
\pend
\pstart
\noindent \dt{149.} Libo'urnioc\edindex[namen]{Liburnius!Bischof in �gypten}
\pend
\pstart
\noindent \dt{150.} >Am'antioc\edindex[namen]{Amantius!Bischof von Nilopolis}
\pend
\pstart
\noindent \dt{151.} \edtext{F'hlix}{\Dfootnote{F'ilhx \latintext KO}}\edindex[namen]{Felix!Bischof in �gypten}
\pend
\pstart
\noindent \dt{152.} >Isqur'ammwn\edindex[namen]{Ischyrammon!Bischof in �gypten}
\pend
\pstart
\noindent \dt{153.} <Rwm~uloc\edindex[namen]{Romulus!Bischof in �gypten}
\pend
\pstart
\noindent \dt{154.} Tiber~inoc\edindex[namen]{Tiberinus!Bischof in �gypten}
\pend
\pstart
\noindent \dt{155.} Kons'ortioc\edindex[namen]{Consortius!Bischof in �gypten}
\pend
\pstart
\noindent \dt{156.} <Hrakle'idhc\edindex[namen]{Heracleides!Bischof in �gypten1}
\pend
\pstart
\noindent \dt{157.} \edtext{Fortoun'atioc}{\Dfootnote{Fourtoun'atioc \latintext
E*}}\edindex[namen]{Fortunatius!Bischof in �gypten}
\pend
\pstart
\noindent \dt{158.} \edtext{Di'oskoroc}{\lemma{\abb{}} \Dfootnote{\responsio\ Di'oskoroc
\latintext post \greektext Fortounatian'oc \latintext B}}\edindex[namen]{Dioscurus!Bischof in �gypten1}
\pend
\pstart
\noindent \dt{159.} \edtext{Fortounatian'oc}{\Dfootnote{Fourtounatian'oc
\dt{B\corr}}}\edindex[namen]{Fortunatianus!Bischof in �gypten}
\pend
\pstart
\noindent \dt{160.} \edtext{\abb{B\Ladd{l}ast'am\Ladd{m}wn}}{\Dfootnote{\dt{coni. Opitz} Bast'amwn
\dt{codd.}}}\edindex[namen]{Blastammon!Bischof in �gypten}
\pend
\pstart
\noindent \dt{161.} D'atulloc\edindex[namen]{Datyllus!Bischof in �gypten}
\pend
\pstart
\noindent \dt{162.} >Andr'eac\edindex[namen]{Andreas!Bischof der Arsenoitis}
\pend
\pstart
\noindent \dt{163.} \edtext{Ser~hnoc}{\Dfootnote{Ser~inoc \latintext
BKO}}\edindex[namen]{Serenus!Bischof von Aphrodition}
\pend
\pstart
\noindent \dt{164.} >'Areioc\edindex[namen]{Arius!Bischof von Panus}
\pend
\pstart
\noindent \dt{165.} Je'odwroc\edindex[namen]{Theodorus!Bischof von Diospolis magna}
\pend
\pstart
\noindent \dt{166.} \edtext{E>uag'orac}{\Dfootnote{agorac \latintext
B*}}\edindex[namen]{Euagoras!Bischof in �gypten}
\pend
\pstart
\noindent \dt{167.} >Hl'iac\edindex[namen]{Elias!Bischof in �gypten1}
\pend
\pstart
\noindent \dt{168.} Tim'ojeoc\edindex[namen]{Timotheus!Bischof in �gypten (Diospolis parva?)}
\pend
\pstart
\noindent \dt{169.} >Wr'iwn\edindex[namen]{Orion!Bischof von Sethro"itis}
\pend
\pstart
\noindent \dt{170.} >Andr'onikoc\edindex[namen]{Andronicus!Bischof von Tentyra}
\pend
\pstart
\noindent \dt{171.} Pafno'utioc\edindex[namen]{Paphnutius!Bischof in �gypten (Sa"is?)}
\pend
\pstart
\noindent \dt{172.} \edtext{<Erm'iac}{\Dfootnote{<Erme'iac \latintext B\corr
K}}\edindex[namen]{Hermias!Bischof in �gypten}
\pend
\pstart
\noindent \dt{173.} >Arab'iwn\edindex[namen]{Arabion!Bischof von Stathma}
\pend
\pstart
\noindent \dt{174.} Yen'osiric\edindex[namen]{Psenosiris!Bischof in �gypten}
\pend
\pstart
\noindent \dt{175.} >Apoll'wnioc\edindex[namen]{Apollonius!Bischof in �gypten1}
\pend
\pstart
\noindent \dt{176.} Mo~uic\edindex[namen]{Mu"is!Bischof von Laton}
\pend
\pstart
\noindent \dt{177.} \edtext{Sarap'ampwn}{\Dfootnote{Samar'apwn \latintext
K}}\edindex[namen]{Sarapampon!Bischof in �gypten}
\pend
\pstart
\noindent \dt{178.} F'ilwn\edindex[namen]{Philon!Bischof in der Theba"is}
\pend
\pstart
\noindent \dt{179.} F'ilippoc\edindex[namen]{Philippos!Bischof in �gypten (Panyphis?)}
\pend
\pstart
\noindent \dt{180.} >Apoll'wnioc\edindex[namen]{Apollonius!Bischof in �gypten2}
\pend
\pstart
\noindent \dt{181.} Pafno'utioc\edindex[namen]{Paphnutius!Bischof in �gypten (Sa"is?)2}
\pend
\pstart
\noindent \dt{182.} Pa~uloc\edindex[namen]{Paulus!Bischof in �gypten}
\pend
\pstart
\noindent \dt{183.} Di'oskoroc\edindex[namen]{Dioscurus!Bischof in �gypten2}
\pend
\pstart
\noindent \dt{184.} Neil'ammwn\edindex[namen]{Nilammon!Bischof von Syene}
\pend
\pstart
\noindent \dt{185.} \edtext{Ser~hnoc}{\Dfootnote{Ser~inoc \latintext
KO}}\edindex[namen]{Serenus!Bischof in �gypten}
\pend
\pstart
\noindent \dt{186.} >Ak'ulac\edindex[namen]{Aquila!Bischof in �gypten}
\pend
\pstart
\noindent \dt{187.} >Awt~ac\edindex[namen]{Aotas!Bischof in �gypten}
\pend
\pstart
\noindent \dt{188.} \edtext{<Arpokrat'iwn}{\Dfootnote{>Apokrat'iwn \latintext
B}}\edindex[namen]{Harpocration!Bischof in �gypten (Alphokranon?)}
\pend
\pstart
\noindent \dt{189.} \edtext{>Isa'ak}{\Dfootnote{>Is'ak \latintext BOR}}\edindex[namen]{Isaak!Bischof in �gypten}
\pend
\pstart
\noindent \dt{190.} Je'odwroc\edindex[namen]{Theodorus!Bischof von Coptos}
\pend
\pstart
\noindent \dt{191.} >Apoll'wc\edindex[namen]{Apollus!Bischof von Xo"is}
\pend
\pstart
\noindent \dt{192.} \edtext{>Ammwnian'oc}{\Dfootnote{>Ammwnan'oc \latintext
B}}\edindex[namen]{Ammonianus!Bischof in �gypten}
\pend
\pstart
\noindent \dt{193.} Ne~iloc\edindex[namen]{Nilus!Bischof in �gypten}
\pend
\pstart
\noindent \dt{194.} <Hr'akleioc\edindex[namen]{Heraclius!Bischof in �gypten}
\pend
\pstart
\noindent \dt{195.} >Are'iwn\edindex[namen]{Arion!Bischof von Antinous}
\pend
\pstart
\noindent \dt{196.} >Aj'ac\edindex[namen]{Athas!Bischof von Schedia}
\pend
\pstart
\noindent \dt{197.} >Ars'enioc\edindex[namen]{Arsenius!Bischof von Hypsele}
\pend
\pstart
\noindent \dt{198.} >Agaj'ammwn\edindex[namen]{Agathammon!Bischof in �gypten}
\pend
\pstart
\noindent \dt{199.} J'ewn\edindex[namen]{Theon!Bischof in �gypten}
\pend
\pstart
\noindent \dt{200.} >Apoll'wnioc\edindex[namen]{Apollonius!Bischof in �gypten3}
\pend
\pstart
\noindent \dt{201.} >Hl'iac\edindex[namen]{Elias!Bischof in �gypten2}
\pend
\pstart
\noindent \dt{202.} Pannino'ujioc\edindex[namen]{Panninuthius!Bischof in �gypten}
\pend
\pstart
\noindent \dt{203.} >Andrag'ajioc\edindex[namen]{Andragathus!Bischof in der Garyathis}
\pend
\pstart
\noindent \dt{204.} Nemes'iwn\edindex[namen]{Nemesius!Bischof von Paralus}
\pend
\pstart
\noindent \dt{205.} Sarap'iwn\edindex[namen]{Sarapion!Bischof von Apollon inf.}
\pend
\pstart
\noindent \dt{206.} >Amm'wnioc\edindex[namen]{Ammonius!1Bischof in �gypten (Pachnemunis?)}
\pend
\pstart
\noindent \dt{207.} >Amm'wnioc\edindex[namen]{Ammonius!2Bischof in �gypten (Pachnemunis?)}
\pend
\pstart
\noindent \dt{208.} X'enwn\edindex[namen]{Xenon!Bischof in �gypten}
\pend
\pstart
\noindent \dt{209.} Ger'ontioc\edindex[namen]{Gerontius!Bischof in �gypten}
\pend
\pstart
\noindent \dt{210.} K'ointoc\edindex[namen]{Quintus!Bischof in �gypten}
\pend
\pstart
\noindent \dt{211.} Lewn'idhc\edindex[namen]{Leonides!Bischof in �gypten}
\pend
\pstart
\noindent \dt{212.} Semprwnian'oc\edindex[namen]{Sempronianus!Bischof in �gypten}
\pend
\pstart
\noindent \dt{213.} F'ilwn\edindex[namen]{Philon!Bischof in �gypten}
\pend
\pstart
\noindent \dt{214.} <Hrakle'idhc\edindex[namen]{Heracleides!Bischof in �gypten2}
\pend
\pstart
\noindent \dt{215.} <I'erakuc\edindex[namen]{Hieracys!Bischof in �gypten}
\pend
\pstart
\noindent \dt{216.} <Ro~ufoc\edindex[namen]{Rufus!Bischof in �gypten}
\pend
\pstart
\noindent \dt{217.} \edtext{Pas'ofioc}{\Dfootnote{Pans'ofioc \latintext
K}}\edindex[namen]{Pasophius!Bischof in �gypten}
\pend
\pstart
\noindent \dt{218.} Maked'onioc\edindex[namen]{Macedonius!Bischof in �gypten}
\pend
\pstart
\noindent \dt{219.} >Apoll'odwroc\edindex[namen]{Apollodorus!Bischof in �gypten}
\pend
\pstart
\noindent \dt{220.} Flabian'oc\edindex[namen]{Flavianus!Bischof in �gypten}
\pend
\pstart
\noindent \dt{221.} Y'ahc\edindex[namen]{Psa"is!Bischof in �gypten}
\pend
\pstart
\noindent \dt{222.} S'urouc\edindex[namen]{Syrus!Bischof in �gypten}
\pend
\pstart
\noindent \dt{223.} >Apfo~uc\edindex[namen]{Apphus!Bischof in �gypten}
\pend
\pstart
\noindent \dt{224.} Sarap'iwn\edindex[namen]{Sarapion!Bischof von Antipyrgus|textit}
\pend
\pstart
\noindent \dt{225.} <Hsa'iac\edindex[namen]{Isaias!Bischof in �gypten}
\pend
\pstart
\noindent \dt{226.} Pafno'utioc\edindex[namen]{Paphnutius!Bischof in �gypten (Sa"is?)3}
\pend
\pstart
\noindent \dt{227.} Tim'ojeoc\edindex[namen]{Timotheus!Bischof in �gypten (Diospolis parva?)2}
\pend
\pstart
\noindent \dt{228.} >Elour'iwn\edindex[namen]{Elurion!Bischof in �gypten}
\pend
\pstart
\noindent \dt{229.} G'aioc\edindex[namen]{Gaius!Bischof in �gypten1}
\pend
\pstart
\noindent \dt{230.} Mousa~ioc\edindex[namen]{Musaeus!Bischof in �gypten}
\pend
\pstart
\noindent \dt{231.} Pist'oc\edindex[namen]{Pistus!Bischof in �gypten}
\pend
\pstart
\noindent \dt{232.} <Hrakl'ammwn\edindex[namen]{Heraclammon!Bischof in �gypten}
\pend
\pstart
\noindent \dt{233.} <'Hrwn\edindex[namen]{Heron!Bischof in �gypten}
\pend
\pstart
\noindent \dt{234.} >Hl'iac\edindex[namen]{Elias!Bischof in �gypten3}
\pend
\pstart
\noindent \dt{235.} >An'agamfoc\edindex[namen]{Anagamphus!Bischof in �gypten}
\pend
\pstart
\noindent \dt{236.} >Apoll'wnioc\edindex[namen]{Apollonius!Bischof in �gypten4}
\pend
\pstart
\noindent \dt{237.} G'aioc\edindex[namen]{Gaius!Bischof in �gypten2}
\pend
\pstart
\noindent \dt{238.} Filot~ac\edindex[namen]{Philotas!Bischof in �gypten}
\pend
\pstart
\noindent \dt{239.} Pa~uloc\edindex[namen]{Paulus!Bischof in �gypten2}
\pend
\pstart
\noindent \dt{240.} Tij'ohc\edindex[namen]{Titho"es!Bischof von Clysma}
\pend
\pstart
\noindent \dt{241.} E>uda'imwn\edindex[namen]{Eudaemon!Bischof von Lycon}
\pend
\pstart
\noindent \dt{242.} >Io'ulioc\edindex[namen]{Julius!Bischof in �gypten}
\pend
\pstart
\vspace{.5\baselineskip}\noindent\textit{O<i >en t~w| kanal'iw| t~hc >Ital'iac}
\pend
\pstart
\noindent \dt{243.} Prob'atioc\edindex[namen]{Probatius!Bischof in Italien}
\pend
\pstart
\noindent \dt{244.} Bi'atwr\edindex[namen]{Viator!Bischof in Italien}
\pend
\pstart
\noindent \dt{245.} Fakound~inoc\edindex[namen]{Facundinus!Bischof von Tadinum}
\pend
\pstart
\noindent \dt{246.} >Iws~hc\edindex[namen]{Joses!Bischof in Italien}
\pend
\pstart
\noindent \dt{247.} Noum'hdioc\edindex[namen]{Numidius!Bischof in Italien}
\pend
\pstart
\noindent \dt{248.} Sphr'antioc\edindex[namen]{Sperantius!Bischof in Italien}
\pend
\pstart
\noindent \dt{249.} Seb~hroc\edindex[namen]{Severus!Bischof von Ravenna|dub}\edindex[namen]{Severus!Bischof von Chalcis|dub}
\pend
\pstart
\noindent \dt{250.} <Hrakleian'oc\edindex[namen]{Heraclianus!Bischof in Italien (Pisaurum?)}
\pend
\pstart
\noindent \dt{251.} Faust~inoc\edindex[namen]{Faustinus!Bischof von Bononia}
\pend
\pstart
\noindent \dt{252.} >Antwn~inoc\edindex[namen]{Antoninus!Bischof von Mutina}
\pend
\pstart
\noindent \dt{253.} <Hr'akleioc\edindex[namen]{Heraclius!Bischof in Italien}
\pend
\pstart
\noindent \dt{254.} O>uit'alioc\edindex[namen]{Vitalius!Bischof in Italien}
\pend
\pstart
\noindent \dt{255.} \edtext{F'hlix}{\Dfootnote{F'ilhx \latintext KO}}\edindex[namen]{Felix!Bischof in Italien (Bellunum?)}
\pend
\pstart
\noindent \dt{256.} Krhsp~inoc\edindex[namen]{Crispinus!Bischof in Italien}
\pend
\pstart
\noindent \dt{257.} Paulian'oc \edindex[namen]{Paulianus!Bischof in Italien}
\pend
\pstart
\vspace{.5\baselineskip}\noindent\textit{K'uprou}
\pend
\pstart
\noindent \dt{258.} \edtext{A>ux'ibioc}{\Dfootnote{A>ux'hbioc \latintext
BKO}}\edindex[namen]{Auxibius!Bischof auf Zypern}
\pend
\pstart
\noindent \dt{259.} F'wtioc\edindex[namen]{Photius!Bischof auf Zypern}
\pend
\pstart
\noindent \dt{260.} \edtext{\abb{Gel'asios}}{\Dfootnote{\dt{susp. Opitz} Ghr'asioc \dt{codd.}}}\edindex[namen]{Gelasius!Bischof von Salamis}
\pend
\pstart
\noindent \dt{261.} \edtext{>Afrod'isioc}{\Dfootnote{>Afrod'hsioc \latintext
B*}}\edindex[namen]{Aphrodisius!Bischof auf Zypern}
\pend
\pstart
\noindent \dt{262.} E>irhnik'oc\edindex[namen]{Irenicus!Bischof auf Zypern}
\pend
\pstart
\noindent \dt{263.} Noun'eqioc\edindex[namen]{Nunechius!Bischof auf Zypern}
\pend
\pstart
\noindent \dt{264.} >Ajan'asioc\edindex[namen]{Athanasius!Bischof auf Zypern}
\pend
\pstart
\noindent \dt{265.} Maked'onioc\edindex[namen]{Macedonius!Bischof auf Zypern}
\pend
\pstart
\noindent \dt{266.} Trif'ullioc\edindex[namen]{Triphyllius!Bischof von Ledra}
\pend
\pstart
\noindent \dt{267.} Spur'idwn\edindex[namen]{Spuridon!Bischof von Trimythuntos}
\pend
\pstart
\noindent \dt{268.} Norban'oc\edindex[namen]{Norbanus!Bischof auf Zypern}
\pend
\pstart
\noindent \dt{269.} \edtext{Swsikr'athc}{\lemma{\abb{}} \Dfootnote{\responsio\ Swsikr'athc
\latintext ante
\greektext Norban'oc \latintext B*}}\edindex[namen]{Sosicrates!Bischof von Carpassus}
\pend
\pstart
\vspace{.5\baselineskip}\noindent\textit{Palaist'inhc}
\pend
\pstart
\noindent \dt{270.} M'aximoc \edindex[namen]{Maximus!Bischof von Jerusalem}
\pend
\pstart
\noindent \dt{271.} >A'etioc \edindex[namen]{A"etius!Bischof in Palaestina}
\pend
\pstart
\noindent \dt{272.} >'Areioc \edindex[namen]{Arius!Bischof von Petra}
\pend
\pstart
\noindent \dt{273.} \edtext{Jeod'osioc}{\Dfootnote{\dt{an legendum} Je'odwros\dt{? (vgl. apol.sec. 57,7 [137,18 Opitz])}}} \edindex[namen]{Theodosius!Bischof in Palaestina}
\pend
\pstart
\noindent \dt{274.} German'oc \edindex[namen]{Germanus!Bischof in Palaestina (Neapolis?)1}
\pend
\pstart
\noindent \dt{275.} Silouan'oc \edindex[namen]{Silvanus!Bischof von Azotus}
\pend
\pstart
\noindent \dt{276.} Pa~uloc\edindex[namen]{Paulus!Bischof in Palaestina (Maximianopolis?)}
\pend
\pstart
\noindent \dt{277.} Kla'udioc \edindex[namen]{Claudius!Bischof in Palaestina}
\pend
\pstart
\noindent \dt{278.} Patr'ikioc \edindex[namen]{Patricius!Bischof in Palaestina}
\pend
\pstart
\noindent \dt{279.} >Elp'idioc \edindex[namen]{Elpidius!Bischof in Palaestina}
\pend
\pstart
\noindent \dt{280.} German'oc \edindex[namen]{Germanus!Bischof in Palaestina (Neapolis?)2}
\pend
\pstart
\noindent \dt{281.} E>us'ebioc \edindex[namen]{Eusebius!Bischof in Palaestina}
\pend
\pstart
\noindent \dt{282.} Zhn'obioc \edindex[namen]{Zenobius!Bischof in Palaestina}
\pend
\pstart
\noindent \dt{283.} Pa~uloc\edindex[namen]{Paulus!Bischof in Palaestina (Maximianopolis?)2}
\pend
\pstart
\noindent \dt{284.} P'etroc\edindex[namen]{Petrus!Bischof in Palaestina (Nicopolis?)}
\pend
% \pstart
% O<i m`en o>~un to~ic <up`o t~hc sun'odou grafe~isin <upogr'ayantec o<~utoi,
% <'eteroi d`e ple~isto'i e>isin o<i ka`i pr`o
% \edtext{\abb{ta'uthc}}{\Dfootnote{\latintext > K}} t~hc sun'odou gr'ayantec
% <up`er <hm~wn >ap'o te t~hc >As'iac\edindex[namen]{Asien} ka`i Frug'iac\edindex[namen]{Phrygien} ka`i >Isaur'iac\edindex[namen]{Isaurien}, ka`i t`a
% >on'omata a>ut~wn >en ta~ic >id'iaic >epistola~ic >emf'eretai, >egg`uc xg*,
% <omo~u tmd*.
% \pend
% \endnumbering
\end{Leftside}
\begin{Rightside}
\begin{translatio}
\beginnumbering
\pstart
\begin{footnotesize}
\noindent Dies schrieb die Synode von Serdica und sandte es an die, die nicht kommen konnten und die
dann auch selbst denen zustimmten, die die Entscheidung gef�llt hatten. Die Namen derer,
die auf der Synode unterschrieben haben, und der �brigen Bisch�fe sind folgende:
\end{footnotesize}
\pend
\pstart
\noindent 1. Ossius aus Spanien
\pend
\pstart
\noindent 2. Julius von Rom vertreten durch seine Presbyter Archidamus und Philoxenus
\pend
\pstart
\noindent 3. Protogenes von Serdica
\pend
\pstart
\noindent 4. Gaudentius (von Na"issus)
\pend
\pstart
\noindent 5. Macedonius (von Ulpiana)
\pend
\pstart
\noindent 6. Severus
\pend
\pstart
\noindent 7. Praetextatus (von Barcilona)
\pend
\pstart
\noindent 8. Ursacius (von Brixia)
\pend
\pstart
\noindent 9. Lucius
\pend
\pstart
\noindent 10. Eugenius (von Heraclea Lyncestis)
\pend
\pstart
\noindent 11. Vitalis (">Vertaresis"<)
\pend
\pstart
\noindent 12. Calepodius (von Neapolis)
\pend
\pstart
\noindent 13. Florentius (von Emerita Augusta)
\pend
\pstart
\noindent 14. Bassus (von Diocletianopolis)
\pend
\pstart
\noindent 15. Vincentius (von Capua)
\pend
\pstart
\noindent 16. Stercorius (von Canusium)
\pend
\pstart
\noindent 17. Palladius (von Dium)
\pend
\pstart
\noindent 18. Domitianus (von Asturica)
\pend
\pstart
\noindent 19. Calvus (von Castra Martis)
\pend
\pstart
\noindent 20. Gerontius (von Beroea)
\pend
\pstart
\noindent 21. Protasius (von Mailand)
\pend
\pstart
\noindent 22. Eulogius
\pend
\pstart
\noindent 23. Porphyrius (von Philippi)
\pend
\pstart
\noindent 24. Dioscurus (von Therasia)
\pend
\pstart
\noindent 25. Zosimus
\pend
\pstart
\noindent 26. Januarius (von Beneventum)
\pend
\pstart
\noindent 27. Zosimus
\pend
\pstart
\noindent 28. Alexander
\pend
\pstart
\noindent 29. Eutychius (von Methone)
\pend
\pstart
\noindent 30. Socras (von Phoebia am Asopus)
\pend
\pstart
\noindent 31. Diodorus (von Tenedus)
\pend
\pstart
\noindent 32. Martyrius
\pend
\pstart
\noindent 33. Eutherius
\pend
\pstart
\noindent 34. Eucarpus (von Opus)
\pend
\pstart
\noindent 35. Athenodorus (von Elatia)
\pend
\pstart
\noindent 36. Irenaeus (von Scyrus)
\pend
\pstart
\noindent 37. Julianus
\pend
\pstart
\noindent 38. Alypius (von Megara)
\pend
\pstart
\noindent 39. Jonas (von Parthicopolis)
\pend
\pstart
\noindent 40. A"etius (von Thessalonike)
\pend
\pstart
\noindent 41. Restutus
\pend
\pstart
\noindent 42. Marcellus (von Ancyra)
\pend
\pstart
\noindent 43. Aprianus (von Poetovio)
\pend
\pstart
\noindent 44. Vitalis
\pend
\pstart
\noindent 45. Valens (von Oescus)
\pend
\pstart
\noindent 46. Hermogenes (von Sicyon)
\pend
\pstart
\noindent 47. Castus (von Caesaraugusta)
\pend
\pstart
\noindent 48. Dometianus (von Acaria Constantias?)
\pend
\pstart
\noindent 49. Fortunatianus (von Aquileia)
\pend
\pstart
\noindent 50. Marcus (von Siscia)
\pend
\pstart
\noindent 51. Annianus (von Castellona)
\pend
\pstart
\noindent 52. Heliodorus (von Nicopolis)
\pend
\pstart
\noindent 53. Musaeus (von Thebae Phthiotides)
\pend
\pstart
\noindent 54. Asterius (aus Arabien)
\pend
\pstart
\noindent 55. Paregorius (von Scupi)
\pend
\pstart
\noindent 56. Plutarchus (von Patras)
\pend
\pstart
\noindent 57. Hymenaeus (von Hypata)
\pend
\pstart
\noindent 58. Athanasius (von Alexandrien)
\pend
\pstart
\noindent 59. Lucius
\pend
\pstart
\noindent 60. Amantius (von Viminacium)
\pend
\pstart
\noindent 61. Arius (von Petra)
\pend
\pstart
\noindent 62. Asclepius (von Gaza)
\pend
\pstart
\noindent 63. Dionysius (von Elis)
\pend
\pstart
\noindent 64. Maximus (von Luca)
\pend
\pstart
\noindent 65. Tryphon (von Macaria)
\pend
\pstart
\noindent 66. Alexander
\pend
\pstart
\noindent 67. Antigonus (von Pallene)
\pend
\pstart
\noindent 68. Elianus (von Gortyna)
\pend
\pstart
\noindent 69. Petrus
\pend
\pstart
\noindent 70. Symphorus (von Hierapytna)
\pend
\pstart
\noindent 71. Musonius (von Heraclium)
\pend
\pstart
\noindent 72. Eutychus
\pend
\pstart
\noindent 73. Philologius
\pend
\pstart
\noindent 74. Spudasius
\pend
\pstart
\noindent 75. Zosimus
\pend
\pstart
\noindent 76. Patricius
\pend
\pstart
\noindent 77. Adolius
\pend
\pstart
\noindent 78. Sapricius
\pend
\pstart
\vspace{.5\baselineskip}\noindent\textit{Aus Gallien}
\pend
\pstart
\noindent 79. Maximinus (von Trier)
\pend
\pstart
\noindent 80. Verissimus (von Lugdunum)
\pend
\pstart
\noindent 81. Victurus (von Metz?)
\pend
\pstart
\noindent 82. Valentinus (von Arles?)
\pend
\pstart
\noindent 83. Desiderius (B. der Lingonen?)
\pend
\pstart
\noindent 84. Eulogius (von Amiens)
\pend
\pstart
\noindent 85. Servatius (von Tongern)
\pend
\pstart
\noindent 86. Dyscolius (B. der Remer?)
\pend
\pstart
\noindent 87. Superior (B. der Nervier?)
\pend
\pstart
\noindent 88. Mercurius (von Soissons?)
\pend
\pstart
\noindent 89. Diclopetus
\pend
\pstart
\noindent 90. Eusebius (von Rouen?)
\pend
\pstart
\noindent 91. Severinus (von Sens?)
\pend
\pstart
\noindent 92. Satyrus
\pend
\pstart
\noindent 93. Martinus (von Mainz?)
\pend
\pstart
\noindent 94. Paulus
\pend
\pstart
\noindent 95. Optatianus (von Troyes?)
\pend
\pstart
\noindent 96. Nicasius 
\pend
\pstart
\noindent 97. Victor (von Worms?)
\pend
\pstart
\noindent 98. Sempronius
\pend
\pstart
\noindent 99. Valerianus (von Auxerre?)
\pend
\pstart
\noindent 100. Pacatus
\pend
\pstart
\noindent 101. Jesses
\pend
\pstart
\noindent 102. Ariston
\pend
\pstart
\noindent 103. Simplicius
\pend
\pstart
\noindent 104. Metianus
\pend
\pstart
\noindent 105. Amandus
\pend
\pstart
\noindent 106. Aemillianus
\pend
\pstart
\noindent 107. Justinianus
\pend
\pstart
\noindent 108. Victorinus
\pend
\pstart
\noindent 109. Satornilus
\pend
\pstart
\noindent 110. Abundantius
\pend
\pstart
\noindent 111. Donatianus
\pend
\pstart
\noindent 112. Maximus
\pend
\pstart
\vspace{.5\baselineskip}\noindent\textit{Aus Afrika}
\pend
\pstart
\noindent 113. Nessus
\pend
\pstart
\noindent 114. Gratus
\pend
\pstart
\noindent 115. Megasius
\pend
\pstart
\noindent 116. Coldaeus
\pend
\pstart
\noindent 117. Rogatianus
\pend
\pstart
\noindent 118. Consortius
\pend
\pstart
\noindent 119. Rufinus
\pend
\pstart
\noindent 120. Manninus
\pend
\pstart
\noindent 121. Cessilianus
\pend
\pstart
\noindent 122. Erennianus
\pend
\pstart
\noindent 123. Marianus
\pend
\pstart
\noindent 124. Valerius
\pend
\pstart
\noindent 125. Dynamius
\pend
\pstart
\noindent 126. Musonius
\pend
\pstart
\noindent 127. Justus
\pend
\pstart
\noindent 128. Celestinus
\pend
\pstart
\noindent 129. Zyprianus
\pend
\pstart
\noindent 130. Victor
\pend
\pstart
\noindent 131. Honoratus
\pend
\pstart
\noindent 132. Marinus
\pend
\pstart
\noindent 133. Pantagathus
\pend
\pstart
\noindent 134. Felix
\pend
\pstart
\noindent 135. Baudius
\pend
\pstart
\noindent 136. Liber
\pend
\pstart
\noindent 137. Capito
\pend
\pstart
\noindent 138. Minervalis
\pend
\pstart
\noindent 139. Cosmus
\pend
\pstart
\noindent 140. Victor
\pend
\pstart
\noindent 141. Hesperion
\pend
\pstart
\noindent 142. Felix
\pend
\pstart
\noindent 143. Severianus
\pend
\pstart
\noindent 144. Optantius
\pend
\pstart
\noindent 145. Hesperus
\pend
\pstart
\noindent 146. Fidentius
\pend
\pstart
\noindent 147. Salustius
\pend
\pstart
\noindent 148. Paschasius
\pend
\pstart
\vspace{.5\baselineskip}\noindent\textit{Aus �gypten}
\pend
\pstart
\noindent 149. Liburnius
\pend
\pstart
\noindent 150. Amantius
\pend
\pstart
\noindent 151. Felix
\pend
\pstart
\noindent 152. Ischyrammon
\pend
\pstart
\noindent 153. Romulus
\pend
\pstart
\noindent 154. Tiberinus
\pend
\pstart
\noindent 155. Consortius
\pend
\pstart
\noindent 156. Heracleides
\pend
\pstart
\noindent 157. Fortunatius
\pend
\pstart
\noindent 158. Dioscurus
\pend
\pstart
\noindent 159. Fortunatianus
\pend
\pstart
\noindent 160. Blastammon
\pend
\pstart
\noindent 161. Datyllus
\pend
\pstart
\noindent 162. Andreas
\pend
\pstart
\noindent 163. Serenus
\pend
\pstart
\noindent 164. Arius
\pend
\pstart
\noindent 165. Theodorus
\pend
\pstart
\noindent 166. Euagoras
\pend
\pstart
\noindent 167. Elias
\pend
\pstart
\noindent 168. Timotheus
\pend
\pstart
\noindent 169. Orion
\pend
\pstart
\noindent 170. Andronicus
\pend
\pstart
\noindent 171. Paphnutius
\pend
\pstart
\noindent 172. Hermias
\pend
\pstart
\noindent 173. Arabion
\pend
\pstart
\noindent 174. Psenosiris
\pend
\pstart
\noindent 175. Apollonius
\pend
\pstart
\noindent 176. Mu"is
\pend
\pstart
\noindent 177. Sarapampon
\pend
\pstart
\noindent 178. Philon
\pend
\pstart
\noindent 179. Philippos
\pend
\pstart
\noindent 180. Apollonius
\pend
\pstart
\noindent 181. Paphnutius
\pend
\pstart
\noindent 182. Paulus
\pend
\pstart
\noindent 183. Dioscurus
\pend
\pstart
\noindent 184. Nilammon
\pend
\pstart
\noindent 185. Serenus
\pend
\pstart
\noindent 186. Aquila
\pend
\pstart
\noindent 187. Aotas
\pend
\pstart
\noindent 188. Harpocration
\pend
\pstart
\noindent 189. Isaak
\pend
\pstart
\noindent 190. Theodorus
\pend
\pstart
\noindent 191. Apollus
\pend
\pstart
\noindent 192. Ammonianus
\pend
\pstart
\noindent 193. Nilus
\pend
\pstart
\noindent 194. Heraclius
\pend
\pstart
\noindent 195. Arion
\pend
\pstart
\noindent 196. Athas
\pend
\pstart
\noindent 197. Arsenius
\pend
\pstart
\noindent 198. Agathammon
\pend
\pstart
\noindent 199. Theon
\pend
\pstart
\noindent 200. Apollonius
\pend
\pstart
\noindent 201. Elias
\pend
\pstart
\noindent 202. Panninuthius
\pend
\pstart
\noindent 203. Andragathus
\pend
\pstart
\noindent 204. Nemesius
\pend
\pstart
\noindent 205. Sarapion
\pend
\pstart
\noindent 206. Ammonius
\pend
\pstart
\noindent 207. Ammonius
\pend
\pstart
\noindent 208. Xenon
\pend
\pstart
\noindent 209. Gerontius
\pend
\pstart
\noindent 210. Quintus
\pend
\pstart
\noindent 211. Leonides
\pend
\pstart
\noindent 212. Sempronianus
\pend
\pstart
\noindent 213. Philon
\pend
\pstart
\noindent 214. Heracleides
\pend
\pstart
\noindent 215. Hieracys
\pend
\pstart
\noindent 216. Rufus
\pend
\pstart
\noindent 217. Pasophius
\pend
\pstart
\noindent 218. Macedonius
\pend
\pstart
\noindent 219. Apollodorus
\pend
\pstart
\noindent 220. Flavianus
\pend
\pstart
\noindent 221. Psaes
\pend
\pstart
\noindent 222. Syrus
\pend
\pstart
\noindent 223. Apphus
\pend
\pstart
\noindent 224. Sarapion
\pend
\pstart
\noindent 225. Isaias
\pend
\pstart
\noindent 226. Paphnutius
\pend
\pstart
\noindent 227. Timotheus
\pend
\pstart
\noindent 228. Elurion
\pend
\pstart
\noindent 229. Gaius
\pend
\pstart
\noindent 230. Musaeus
\pend
\pstart
\noindent 231. Pistus
\pend
\pstart
\noindent 232. Heraclammon
\pend
\pstart
\noindent 233. Heron
\pend
\pstart
\noindent 234. Elias
\pend
\pstart
\noindent 235. Anagamphus
\pend
\pstart
\noindent 236. Apollonius
\pend
\pstart
\noindent 237. Gaius
\pend
\pstart
\noindent 238. Philotas
\pend
\pstart
\noindent 239. Paulus
\pend
\pstart
\noindent 240. Titho"es
\pend
\pstart
\noindent 241. Eudaemon
\pend
\pstart
\noindent 242. Julius
\pend
\pstart
\vspace{.5\baselineskip}\noindent\textit{Die an der Stra�e Italiens}
\pend
\pstart
\noindent 243. Probatius
\pend
\pstart
\noindent 244. Viator
\pend
\pstart
\noindent 245. Facundinus
\pend
\pstart
\noindent 246. Joses
\pend
\pstart
\noindent 247. Numedius
\pend
\pstart
\noindent 248. Sperantius
\pend
\pstart
\noindent 249. Severus
\pend
\pstart
\noindent 250. Heraclianus
\pend
\pstart
\noindent 251. Faustinus
\pend
\pstart
\noindent 252. Antoninus
\pend
\pstart
\noindent 253. Heracleius
\pend
\pstart
\noindent 254. Vitalius
\pend
\pstart
\noindent 255. Felix
\pend
\pstart
\noindent 256. Crispinus
\pend
\pstart
\noindent 257. Paulianus
\pend
\pstart
\vspace{.5\baselineskip}\noindent\textit{Aus Zypern}
\pend
\pstart
\noindent 258. Auxibius
\pend
\pstart
\noindent 259. Photius
\pend
\pstart
\noindent 260. Gelasius
\pend
\pstart
\noindent 261. Aphrodisius
\pend
\pstart
\noindent 262. Irenicus
\pend
\pstart
\noindent 263. Nunechius
\pend
\pstart
\noindent 264. Athanasius
\pend
\pstart
\noindent 265. Macedonius
\pend
\pstart
\noindent 266. Triphyllius
\pend
\pstart
\noindent 267. Spuridon
\pend
\pstart
\noindent 268. Norbanus
\pend
\pstart
\noindent 269. Sosicrates
\pend
\pstart
\vspace{.5\baselineskip}\noindent\textit{Aus Palaestina}\footnoteA{Die Liste apol.sec. 57,7 nennt noch Macrinus (von Jamnia).}
\pend
\pstart
\noindent 270. Maximus
\pend
\pstart
\noindent 271. A"etius
\pend
\pstart
\noindent 272. Arius
\pend
\pstart
\noindent 273. Theodosius
\pend
\pstart
\noindent 274. Germanus
\pend
\pstart
\noindent 275. Silvanus
\pend
\pstart
\noindent 276. Paulus
\pend
\pstart
\noindent 277. Claudius
\pend
\pstart
\noindent 278. Patricius
\pend
\pstart
\noindent 279. Elpidius
\pend
\pstart
\noindent 280. Germanus
\pend
\pstart
\noindent 281. Eusebius
\pend
\pstart
\noindent 282. Zenobius
\pend
\pstart
\noindent 283. Paulus
\pend
\pstart
\noindent 284. Petrus
\pend
% \pstart
% Dies sind diejenigen, die die Akten der Synode unterschrieben haben, die meisten
% anderen
% sind aber die, die auch vor dieser Synode f�r uns Stellung genommen haben und aus
% Asien
% Phrygien und Isaurien kommen. Die Namen derer sind in ihren eigenen Briefen
% enthalten,
% beinahe 63. Insgesamt betr�gt die Zahl 344.
% \pend
\endnumbering
\end{translatio}
\end{Rightside}
\Columns
\end{pairs}
% \thispagestyle{empty}
%%% Local Variables:
%%% mode: latex
%%% TeX-master: "dokumente_master"
%%% End:

%%%% Input-Datei OHNE TeX-Pr�ambel %%
\section{Brief der ">westlichen"< Synode an Kaiser Constantius}
% \label{sec:43.2}
\label{sec:BriefSerdikaConstantius}
\begin{praefatio}
  \begin{description}
  \item[Herbst 343]Zum Datum vgl. Einleitung zu
    Dok. \ref{ch:SerdicaEinl}. Dieser Brief behandelt den dritten, von
    den Kaisern der Synode vorgegebenen Tagesordnungspunkt
    (vgl. Dok. \ref{sec:B},3), wie in der Kirche mit Absetzungen und
    Verurteilungen umgegangen werden soll. Drei Aspekte werden an
    Constantius\index[namen]{Constantius, Kaiser} herangetragen: Er
    m�ge k�nftig daf�r sorgen, da� (weltliche) Richter nicht �ber
    Kleriker entscheiden (� 2), ferner da� die lokalen Statthalter
    H�retikern keine Auftrittsm�glichkeiten gew�hren (� 5), und
    schlie�lich m�ge er es erlauben und einrichten, da� verbannte
    Kleriker zur�ckkehren d�rfen (� 8). Der letzte Punkt betrifft
    nat�rlich besonders Athanasius\index[namen]{Athanasius!Bischof von
      Alexandrien} und Markell\index[namen]{Markell!Bischof von
      Ancyra}. Durch die generelle Disqualifizierung der f�hrenden
    Pers�nlichkeiten der anderen (">�stlichen"<) Synode als
    verurteilte Arianer (� 9~f.), auf welche ewige H�llenstrafen
    warten, wird das Problem, da� sich widersprechende
    Synodalentscheide zwischen Ost und West besonders in
    Personalfragen diesen dritten Tagesordnungspunkt erst notwendig
    gemacht haben, vom Tisch gewischt.
  \item[�berlieferung]Der als Liber I ad Constantium des Hilarius
    zusammen mit einem sich anschlie�enden, erz�hlenden Text
    �berlieferte Brief der Synode von
    Serdica\index[synoden]{Serdica!a. 343} an
    Constantius\index[namen]{Constantius, Kaiser} ist wohl
    urspr�nglich nach dem erz�hlenden Text coll.\,antiar. B II 11
    einzureihen, vgl. \cite{Wilmart:Constantium}. Dies legt
    coll.\,antiar. B II 11,6 (\editioncite[154,22--25]{Hil:coll}) sowie die inhaltlichen
    Parallelen mit den �brigen Briefen, die wir von dieser Synode
    besitzen, nahe.  Auch der Verweis auf eine bereits erfolgte
    Erw�hnung der Synode von Nicaea\index[synoden]{Nicaea!a. 325} in
    dem erz�hlenden Text des Liber I ad Constantium best�tigt diese
    Verortung, da coll.\,antiar. B II 11 sich als Bericht an die
    theologische Erkl�rung der Synode von Nicaea anschlie�t. Bereits
    vor Beginn des 6. Jh.'s sind der Brief an
    Constantius\index[namen]{Constantius, Kaiser} und der nachfolgende
    Bericht von den Collectanea antiariana Parisina getrennt
    �berliefert worden, da der �lteste Codex, in dem der Liber I ad
    Constantium namentlich als solcher tradiert ist, in das 6. Jh. zu
    datieren ist.  Ein Hinweis darauf, da� der Text urspr�nglich
    griechisch abgefa�t war, findet sich in � 7: Hinter der
    Konstruktion \textit{non prius \dots\ quam \dots\ raptos et
      inretitos} steht wohl urspr�nglich eine Konstruktion
    \griech{prin} + AcI.
  \item[Fundstelle]Hil., ad Const. (\editioncite[181,7--184,13]{Hil:coll})
  \end{description}
\end{praefatio}
\begin{pairs}
\selectlanguage{latin}
\begin{Leftside}
% \beginnumbering
% % % \pstart
% % % \edtext{\abb{Oratio synodi sardicensis ad Constantium
% % % imperatorem}}{\Dfootnote{\textit{coni. Fed e coll.antiar. B II 11 p.154,24} incipit
% % % eiusdem ad Constantium \textit{B} incipit liber eiusdem ad Constantium \textit{C}
% % epistola
% % % sancti Hylarii episcopi transmissa ad Constantium(Constantium Augustum \textit{M}
% % % Constantinum \textit{O}) \textit{JEMOW} incipit tractatus eiusdem ad eundem
% Constantium
% % % \textit{G} > \textit{L}}}
% % % \pend
\pstart
\hskip -1.2em\edtext{\abb{}}{\killnumber\Cfootnote{\hskip -.7em BC(\griech{g}) + 
JELMOW(\griech{p}) + GT = \griech{l}}}\specialindex{quellen}{section}{Hilarius!ad Const.}
\kap{1}\edtext{Benignifica}{\Dfootnote{benefica \textit{G et plures alii codd. gallici}
benigna \textit{susp. Coustant} benifica \textit{susp. Wilmart}}} natura tua, domine
beatissime Auguste, cum
benigna voluntate concordat et, quoniam de fonte paternae pietatis tuae 
\edtext{misericordia}{\Dfootnote{misericordiam \textit{C*}}} largiter
profluit, quod rogamus, facile nos impetrare posse confidimus. non solum verbis,
sed etiam lacrimis deprecamur, ne diutius catholicae ecclesiae 
gravissimis
iniuriis afficiantur et intolerabiles sustineant 
\edtext{persecutiones}{\Dfootnote{passiones \textit{G}}} 
\edtext{\abb{et}}{\lemma{\abb{}} \Dfootnote{\textit{ras. post} et \textit{C}}} contumelias et, quod est
nefarium, a
fratribus nostris. 
\pend
\pstart
\kap{2}provideat et decernat clementia tua, ut 
\edtext{\abb{omnes}}{\Dfootnote{+ se \textit{\greektext g\latintext
GE*L*W\ts{2}\slin}}} ubique iudices, quibus provinciarum
amministrationes creditae sunt, ad quos sola cura et sollicitudo publicorum
negotiorum pertinere debet, a religiosa se observantia abstineant neque posthac 
\edtext{praesumant}{\Dfootnote{praesument \textit{L}}} atque usurpent et putent
se causas cognoscere clericorum et 
\edtext{innocentes}{\Dfootnote{innocenter \textit{BC\ts{1}}}} homines variis 
\edtext{afflictationibus}{\Dfootnote{afflictionibus \textit{GM}}}, minis,
violentia, terroribus frangere atque vexare.
\pend
\pstart
\kap{3}\edtext{intellegit}{\Dfootnote{intellit \textit{C*}}} singularis et
ammirabilis sapientia tua non decere, non oportere cogi et conpelli invitos et
repugnantes, ut se his subiciant et 
\edtext{addicant}{\Dfootnote{abdicant \textit{G}}} vi oppressi, qui non cessant 
\edtext{adulterinae}{\Dfootnote{adulterae \textit{B*}}}
doctrinae 
\edtext{corrupta}{\Dfootnote{corrupt\c{e} \textit{M*}}} semina 
\edtext{aspargere}{\Dfootnote{aspergere \textit{C\ts{2}E\ts{1}GLMO} asspergere
\textit{W} spargere \textit{JE\ts{2}}}}.
\edtext{iccirco}{\Dfootnote{idcirco \textit{CJEMOW}}} laboratis et salutaribus
consiliis rem publicam 
\edtext{\abb{regitis}}{\Dfootnote{+ et \textit{E*LW}}}, 
\edtext{excubatis}{\Dfootnote{escubatis \textit{BC*} excubabis
\textit{O}}} etiam 
\edtext{\abb{et}}{\Dfootnote{> \textit{BC\ts{2}\slin}}} 
\edtext{vigilatis}{\Dfootnote{iugulatis (\textit{alt. u mut. in a}) \textit{M}}}, ut
omnes, quibus imperatis, dulcissima libertate potiantur. 
non alia ratione, quae
turbata sunt, componi, quae 
\edtext{divulsa}{\Dfootnote{diversa \textit{ELW}}} sunt, 
\edtext{coherceri}{\Dfootnote{coerceri \textit{B\ts{?}L} cohaerere \textit{B\ts{1}}
coh\c{e}cere (\textit{ras. inter} e \textit{et} c) \textit{C} coh\c{e}ceri (\textit{ras. inter} e \textit{et} c) \textit{C\ts{2}}}} 
\edtext{possint}{\Dfootnote{possunt \textit{C\ts{2}G\greektext p}}}, nisi
unusquisque nulla servitutis necessitate 
\edtext{\abb{astrictus}}{\Dfootnote{(ctus in ras.) \textit{O}}} integrum habeat
vivendi arbitrium.
\pend
\pstart
\kap{4}certe vox exclamantium a tua mansuetudine exaudiri 
\edtext{debet}{\Dfootnote{decet \textit{coni. Badius}}}: <<catholicus sum, nolo
esse hereticus; Christianus sum, 
\edtext{non}{\Dfootnote{nolo esse \textit{JE\ts{2}}}} Arrianus\edindex[namen]{Arius!Presbyter in Alexandrien}; 
\edtext{\abb{et}}{\Dfootnote{\textit{exp. B}}} melius 
\edtext{\abb{est}}{\Dfootnote{\textit{del.} Coustant}} mihi in hoc saeculo mori, quam alicuius 
\edtext{privati}{\Dfootnote{privari \textit{E*L}}} potentia dominante
castam veritatis virginitatem corrumpere.>> aequumque debet videri sanctitati
tuae, gloriosissime Auguste, ut, qui timent dominum deum et 
\edtext{divinum}{\Dfootnote{di et unum \textit{B*}}} iudicium, non
polluantur aut contaminentur exsecrandis blasphemiis, sed habeant potestatem, ut
eos sequantur episcopos et praepositos, qui et inviolata 
\edtext{conservant}{\Dfootnote{conserbant \textit{B}}} foedera 
\edtext{caritatis}{\Dfootnote{civitatis \textit{O}}} et cupiunt perpetuam et 
\edtext{sinceram}{\Dfootnote{sincer \textit{C*}}} habere pacem. nec fieri
potest nec ratio patitur, ut repugnantia congruant, dissimilia conglutinentur,
vera et falsa 
\edtext{misceantur}{\Dfootnote{commisceantur \textit{E*} misceantur \textit{E\corr}}},
lux et tenebrae confundantur, dies quoque et nox 
\edtext{habeant}{\Dfootnote{habet \textit{B*} habent (n \textit{s.l.}) \textit{B\ts{2}}}} aliquam
coniunctionem. 
\pend
\pstart
\kap{5}si igitur, quod sine dubitatione et speramus et credimus, haec 
\edtext{permovent}{\Dfootnote{permonent \textit{M}}} 
\edtext{\abb{non}}{\Dfootnote{> \textit{J}}} insitam, sed ingenitam tuam
bonitatem, 
\edtext{praecipe}{\Dfootnote{praecipi \textit{M}}}, ut non studium, non gratiam,
non favorem locorum rectores gravissimis hereticis praestent. permittat lenitas
tua 
\edtext{\abb{populis}}{\lemma{\abb{}} \Dfootnote{ut \textit{post} populis \textit{add. Lypsius}}}, quos 
\edtext{\abb{voluerint}}{\Dfootnote{\textit{coni. Erasmus} volunt
\textit{\greektext l}}}, quos putaverint, quos delegerint,
audiant 
\edtext{docentes}{\Dfootnote{dicentes (o \textit{s.} i) \textit{L}}} et divina 
\edtext{mysteriorum}{\Dfootnote{mysterium \textit{C} misterium C\ts{2} ministeriorum \textit{MO}}}
sollemnia concelebrent, 
\edtext{pro}{\Dfootnote{et pro \textit{coni. Lypsius}}} incolumitate et
beatitudine tua offerant preces.
\pend
\pstart
\kap{6}non quisquam perversus aut invidus maligna 
\edtext{loquatur}{\Dfootnote{loquantur \textit{M} loquuntur \textit{W\ts{1}}}}.
nulla quidem suspicio 
\edtext{erit}{\Dfootnote{exerit \textit{E*L} ex
erit \textit{MO}(\l\ \textit{s.l.})\textit{W}} est \textit{coni. Coustant} ex \textit{B}} non modo seditionis, 
\edtext{non}{\Dfootnote{sed nec \textit{coni. Badius}}} asperae obmurmurationis.
quieta sunt omnia et verecunda. et nunc, qui Arriana et pestifera contagione 
\edtext{inquinati}{\Dfootnote{inquinate C*}} sunt, non cessant ore impio
et sacrilego animo 
\edtext{evangeliorum}{\Dfootnote{evangelicorum \textit{M}}} sinceritatem
corrumpere et rectam apostolorum regulam depravare. 
\pend
\pstart
\kap{7}divinos 
\edtext{prophetas}{\Dfootnote{prof\c{e}tas \textit{C*}}} non intellegunt. 
callidi et
astuti artificio quodam utuntur, 
\edtext{\abb{ut}}{\Dfootnote{\textit{BE*W\mg > CJGMO}}}
inclusam perniciosam corruptelam 
\edtext{\abb{exquisitorum}}{\Dfootnote{\textit{\greektext g} \latintext
exquisitio \textit{\greektext p} \latintext inquisitorum \textit{G}}} verborum velamine 
\edtext{contegant}{\Dfootnote{contegunt \textit{C\ts{2}JE\ts{2}GMO}}}, 
\edtext{non}{\Dfootnote{et non \textit{coni. Coustant}}} prius venenatum
virus 
\edtext{effundant}{\Dfootnote{effundunt \textit{E\ts{2}MOJ\corr} meffundunt \textit{J*} infundunt \textit{coni. Gillotius}}},
quam simplices et innocentes sub praetextu nominis Christiani 
\edtext{\abb{rapti}}{\Dfootnote{\textit{coni. Bauer} raptos \textit{\griech{l}}}} atque 
\edtext{\abb{inretiti}}{\Dfootnote{\textit{coni. Bauer} inretitos \textit{\griech{g}JEGLMWT} inrectos \textit{O} inretitos involvant
\textit{coni. Badius}}}, ne soli pereant, 
\edtext{\abb{sed}}{\Dfootnote{> T}} participes 
\edtext{horrendi}{\Dfootnote{orrendi \textit{BC*}}} criminis sui reos
faciant.
\pend
\pstart
\kap{8}
et hoc obsecramus pietatem tuam, ut eos, qui adhuc -- 
\edtext{egregii}{\Dfootnote{egaegii (r \textit{s}. a) \textit{B}}} videlicet
sacerdotes, qui tanti nominis praepollent dignitate -- aut in exilio aut in
desertis locis tenentur, iubeas ad sedes suas remeare, ut ubique grata libertas
sit et 
\edtext{iucunda}{\Dfootnote{iocunda \textit{G\greektext p}}}
laetitia.
\pend
\pstart
\kap{9}quis non videt, quis non intellegit? post 
\edtext{quadringentos}{\Dfootnote{quadrigentos \textit{C*}}} fere annos,
postquam dei unigenitus filius humano generi pereunti subvenire dignatus est,
quasi ante non apostoli, non post eorum martyria et excessus fuerint Christiani,
novella nunc et teterrima lues non corrupti 
\edtext{aeris}{\Dfootnote{eris \textit{O} > \textit{M}}}, sed exsecrandorum 
\edtext{\abb{blasphemiorum}}{\Dfootnote{(i \textit{eras.} \textit{E}, i
\textit{exp. L})\textit{\greektext l} \latintext blasphemorum \textit{\greektext
J} \latintext at execr\dots\ blasph\dots\ \textit{C\mg}}} Arriana effusa est. ita
illi, qui ante crediderunt, inanem spem immortalitatis habuerunt? 
\pend
\pstart
\kap{10}nuper didicimus commenta haec fuisse inventa 
\edtext{\abb{et}}{\Dfootnote{> \textit{E\corr JOM}}} a duobus 
\edtext{Eusebiis}{\Dfootnote{Eusebium
Caesariensem et Eusebium Nicomediensem
Arrianos affirmat \textit{in mg. B\ts{3}}}}\edindex[namen]{Eusebius!Bischof von
Caesarea}\edindex[namen]{Eusebius!Bischof von Nikomedien} et a Narcisso\edindex[namen]{Narcissus!Bischof von
Neronias} et a \edtext{Theodoro}{\Dfootnote{Theodoto \greektext l}}\edindex[namen]{Theodorus!Bischof von
Heraclea} et ab Stefano\edindex[namen]{Stephanus!Bischof von Antiochien} et 
\edtext{\abb{Acacio}}{\Dfootnote{\textit{coni. Coustant} Achaico \textit{\greektext
l}}}\edindex[namen]{Acacius!Bischof von Caesarea}
et Menofanto\edindex[namen]{Menophantus!Bischof von Ephesus} et 
imperitis atque improbis duobus adulescentibus Ursacio\edindex[namen]{Ursacius!Bischof von Singidunum} et
Valente\edindex[namen]{Valens!Bischof von Mursa}. quorum epistulae proferuntur et idoneis
testibus etiam 
\edtext{convincuntur}{\Dfootnote{convinguntur \textit{G}}},
qui eos magis oblatrantes quam disputantes
\edtext{audierint}{\Dfootnote{audierunt \textit{LMO} audier\oline{t}
\textit{J}}}.
quibus qui communionem suam inprudenter et incaute commiscent, quia 
\edtext{fient}{\Dfootnote{sunt \textit{coni. Coustant}}} socii scelerum, 
\edtext{participes}{\Dfootnote{participem \textit{C*}}} criminum necesse 
\edtext{\abb{est}}{\Dfootnote{> (\textit{ras. s.l.}) \textit{L}}} 
\edtext{eos, qui iam}{\Dfootnote{eos, quia (ia \textit{s.} q \textit{E}) iam
\textit{JE} etiam eos, qui \textit{coni. Gillotius} etiam eos qui iam \textit{coni. Badius}
cum eis, qui iam \textit{susp. Coustant}}} 
\edtext{\abb{in}}{\Dfootnote{\textit{E\slin} > \textit{L}}} hoc saeculo 
\edtext{\abb{abiecti sunt}}{\Dfootnote{\textit{dupl. C\ts{1}}}} et abdicati, cum 
\edtext{advenerit}{\Dfootnote{venerit \textit{B*}}} dies
iudicii, pati supplicia sempiterna.
\pend
% \endnumbering
\end{Leftside}
\begin{Rightside}
\begin{translatio}
\beginnumbering
% \pstart
% Schreiben der Synode von Serdica an Kaiser Constantius
% \pend
\pstart
\noindent\kapR{1,1}Dein g�tiges Wesen, seligster Herr, Augustus, geht einher mit
deinem g�tigen Willen, und da ja Barmherzigkeit reichlich aus der Quelle deiner
v�terlichen Liebe flie�t, haben wir festes Vertrauen, da� wir das, was wir
erbitten, leicht erlangen k�nnen. Nicht nur mit Worten, sondern sogar unter
Tr�nen flehen wir darum, da� den katholischen Kirchen nicht l�nger schwerstes
Unrecht widerf�hrt und sie keine unertr�glichen Verfolgungen und Schm�hungen
mehr aushalten m�ssen, und zwar, was ein Frevel ist, von unseren eigenen Br�dern. 
\pend
\pstart
\kapR{1,2}M�ge deine Milde Sorge tragen und entscheiden, da� �berall alle
Richter, denen die Verwaltung der Provinzen anvertraut ist und denen nur
die Leitung und Organisation der �ffentlichen Aufgaben obliegt,
sich von der �berwachung religi�ser Angelegenheiten fernhalten und hier k�nftig nicht
vorgreifen, Befugnisse beanspruchen und glauben, sie k�nnten die F�lle
der Kleriker untersuchen und unschuldige Menschen durch verschiedene Qualen,
Drohungen, Gewalt und Schrecken zerbrechen und qu�len.
\pend
\pstart
\kapR{2,1}Deine einzigartige und bewundernswerte Weisheit versteht, da� es sich
nicht ziemt und unangemessen ist, die Unwilligen und Widerstrebenden mit Zwang
zusammenzutreiben, damit sie sich von Gewalt bedr�ngt jenen unterwerfen
und zustimmen, die nicht davon ablassen, die verdorbenen
Samen ihrer falschen Lehre auszustreuen. Daher lenkt ihr den Staat mit
ausgefeilten, heilvollen Erlassen, pa�t auf und haltet Wache, damit
alle, denen ihr gebietet, die s��este Freiheit genie�en k�nnen. Auf keine andere
Weise kann das, was durcheinander ist, geordnet, was auseinandergerissen,
zusammengehalten werden, als wenn nicht jeder einzelne, durch keinerlei sklavischen
Zwang gebunden, eine unversehrte Entscheidungsfreiheit �ber sein Leben hat.
\pend
\pstart 
\kapR{2,2}Mit Sicherheit mu� von deiner Milde die Stimme derer geh�rt werden, die
rufen: ">Ich bin katholisch, ich will kein H�retiker sein; ich bin Christ, kein
Arianer; und es ist f�r mich besser, in dieser Zeit zu sterben als unter
der Zwangsherrschaft irgendeines Privatmannes die jungfr�uliche Reinheit der
Wahrheit zu beflecken."< Ebenfalls mu� es deiner Heiligkeit, ruhmvollster Augustus,
gerecht erscheinen, da� diejenigen, die Gott, den Herrn, und das g�ttliche Gericht
f�rchten, nicht durch verwerf"|liche Irrlehren befleckt oder verseucht werden,
sondern die M�glichkeit haben, den Bisch�fen und Vorstehern zu folgen, die die
B�ndnisse der Liebe unverletzt bewahren und anhaltenden und reinen Frieden
haben wollen. Und weder kann es geschehen noch ertr�gt es die Vernunft, da� sich
Gegens�tze zusammenf�gen, Ungleiches sich fest verbindet, Wahres und
Falsches sich mischt, Licht und Finsternis verschmelzen und Tag und Nacht
irgendeine Verbindung haben. 
\pend
\pstart
\kapR{2,3}Wenn also, was wir ohne Zweifel hoffen und glauben, diese Zust�nde
deine nicht aufgesetzte, sondern angeborene G�te anr�hren, so
ordne an, da� die Statthalter vor Ort den schlimmsten H�retikern keine
Unterst�tzung, Gunst oder Bevorzugung gew�hren. Deine Sanftmut m�ge den V�lkern
erlauben, da� sie von denen unterrichtet werden, die sie w�hlen, denen sie zustimmen und die sie
beauftragen, und gestatte, da� sie mit ihnen zusammen die g�ttlichen Feiern der Geheimnisse begehen und f�r
deine Gesundheit und dein Seelenheil beten.
\pend
\pstart
\kapR{3,1}Kein schlechter oder mi�g�nstiger Mensch soll �bles sprechen. Es wird 
�berhaupt keine verd�chtigen Umst�nde geben, weder wegen eines Aufruhrs noch wegen
eines schroffen Aufmurrens. Alles ist ruhig und gesittet. Doch
jetzt h�ren die, die von der arianischen und verderbenbringenden Seuche besudelt
sind, nicht auf, mit gottloser Stimme und frevlerischem Sinn die Reinheit der
Evangelien zu verderben und die rechte Vorschrift der Apostel zu verbiegen. 
\pend
\pstart
\kapR{3,2}Die gotterf�llten Propheten verstehen sie nicht. Schlau und verschlagen
gebrauchen sie eine gewisse Kunstfertigkeit, um unter dem Deckmantel gew�hlter
Worte die eingeschlossene sch�dliche Verderbnis zu verbergen. Sie wollen den
vergifteten Saft nicht eher verspritzen, als bis die Einf�ltigen und
Unschuldigen unter dem Vorwand des christlichen Namens geraubt und ins Netz
gegangen sind, damit sie nicht allein zugrunde gehen, sondern jene zu schuldigen 
Mitt�tern an ihrem grausigen Verbrechen machen.
\pend
\pstart
\kapR{4}Auch dies erbitten wir von deiner Fr�mmigkeit: Befehle bitte, da� diejenigen, die
bisher -- offensichtlich hervorragende Geistliche, die Personen von Rang und Namen sind
-- entweder in der Verbannung oder an abgeschiedenen
Orten festgehalten werden, zu ihren Bischofssitzen zur�ckkehren, damit
�berall dankbare Freiheit und heitere Freude herrsche.
\pend
\pstart
\kapR{5,1}Wer sieht es nicht, wer versteht es nicht? Fast vierhundert
Jahre nachdem der einziggeborene Sohn Gottes beschlossen hatte, dem
untergehenden Menschengeschlecht zu Hilfe zu kommen, hat sich nun die neue
und ganz furchtbare arianische Seuche ausgebreitet, die nicht durch verdorbene Luft, sondern durch
verdammenswerte Gottesl�sterungen entstanden ist, als ob es
vorher keine Apostel und nach deren Martyrien und Tod keine Christen gegeben h�tte.
Hatten also die, die zuvor geglaubt haben, eine unbegr�ndete Hoffnung auf Unsterblichkeit? 
\pend
\pstart
\kapR{5,2}Unl�ngst haben wir erfahren, diese Hirngespinste seien von den beiden Eusebii,
von Narcissus, Theodorus, Stephanus, Acacius, Menophantus und den beiden unkundigen und
nichtsw�rdigen jungen M�nnern Ursacius und Valens erfunden worden. Deren Briefe werden
bekannt gemacht, und diejenigen, die geh�rt haben, wie diese mehr herumbellen als argumentieren,
lassen sich von diesen geeigneten Beweisen auch �berzeugen. 
Solche aber, die diese t�richter- und unvorsichtiger Weise in
ihre Gemeinschaft aufnehmen, erleiden notwendigerweise~-- weil sie Genossen ihrer Frevel
und Teilhaber ihrer Verbrechen werden~--, ewige Strafen, wenn der Tag des Gerichtes
gekommen sein wird, da sie schon in dieser Weltzeit verworfen und ausgeschlossen worden
sind.
\pend
\endnumbering
\end{translatio}
\end{Rightside}
\Columns
\end{pairs}
% \thispagestyle{empty}
%%% Local Variables: 
%%% mode: latex
%%% TeX-master: "dokumente_master"
%%% End: 

%%%% Input-Datei OHNE TeX-Pr�ambel %%%%
\section{Brief der ">westlichen"< Synode an Julius von Rom}
% \label{sec:43.2}
\label{sec:B}
\begin{praefatio}
  \begin{description}
  \item[Herbst 343]Zum Datum vgl. Dok. \ref{ch:SerdicaEinl}. Dieser
    Brief wurde zusammen mit Akten der Synode an
    Julius\index[namen]{Julius!Bischof von Rom} von Rom
    geschickt, der selbst nicht nach
    Serdica\index[synoden]{Serdica!a. 343} gereist war, sondern sich
    durch zwei Presbyter und einen Diakon vertreten lie�
    (s. Namensliste), mit der Bitte, die �brigen Bisch�fe in Sizilien,
    Sardinien und Italien zu informieren. Der Brief spricht die drei
    Tagesordnungspunkte an, die die Kaiser der Synode aufgetragen
    hatten: erstens �ber den Glauben zu verhandeln, zweitens �ber
    Personalfragen und drittens �ber die Art und Weise, wie in der
    Kirche mit Absetzungen und Verurteilungen umgegangen werden
    sollte.
  \item[�berlieferung]Der Brief (einschlie�lich der Nomina
    haereticorum) mit den Unterschriften ist bei Hilarius und abh�ngig von ihm in Zusammenhang mit der kanonistischen
    �berlieferung bezeugt, und zwar in der Collectio Sanblasiana (B),
    der Collectio codicis Diessen (Di), der Collectio Hadriana Aucta
    (E), der Collectio Dionysiana Aucta (E\tsub{4}R) und der Collectio
    codicis S. Mauri (M\tsub{2}; ohne die Unterschriftenliste). Die
    Unterschriftenliste �berliefern zus�tzlich die Collectio Vaticani
    (\griech{F}), die Collectio codicis Iustelli (J), in dem wir die
    Sammlung der Prisca fassen k�nnen, die Collectio codicis Parisini
    (P) und die Collectio Dionysio-Hadriana (\griech{D}).  Thompson
    edierte 1990 den Brief ohne die Unterschriftenliste neu
    (\cite[169--182]{Thompson:Papal}). Er ver�nderte dabei die Siglen,
    bezog M\tsub{1}, M\tsub{3} und B\tsub{6} in die Textkonstituierung
    mit ein und kollationierte auch B\tsub{1--3}, M\tsub{2} sowie
    E\tsub{2} neu. Die hier verwendeten Siglen f�r die
    Kanones�berlieferung folgen der Edition von Thompson, w�hrend die
    Siglen der Collectanea antiariana Parisina wie bei Feder
    beibehalten wurden.  Brief und Unterschriftenliste in der
    Kanones�berlieferung lassen sich auf einen einzigen Archetyp der
    Collectanea antiariana Parisina des Hilarius zur�ckf�hren, von dem
    sich auch A ableitet. Die Unterschriftenliste ist innerhalb der
    Kanones�berlieferung am urspr�nglichsten in der Sammlung B
    erhalten.
  \item[Fundstelle]Hil., coll.\,antiar. B II 2--4 (\editioncite[126,7--139,7]{Hil:coll})
  \end{description}
\end{praefatio}
\begin{pairs}
\selectlanguage{latin}
\begin{Leftside}
% \beginnumbering
\pstart
\hskip -1.1em\edtext{\abb{}}{\killnumber\Cfootnote{\hskip -1em ACTS 
B(B\tsub{1}B\tsub{2}B\tsub{3}B\tsub{4}B\tsub{6})DiE(E\tsub{2}E\tsub{4})M(M\tsub{1}M\tsub{2}M\tsub{3})}}\specialindex{quellen}{section}{Hilarius!coll.\,antiar.!B II 2--4}
% \edtext{\abb{Incipit}}{\Dfootnote{> M\tsub{1}}}
% \edtext{exemplum}{\Dfootnote{exemplar \textit{BMDiE}}}
% \edtext{\abb{epistulae}}{\Dfootnote{+ synodi (sinodi \textit{B\tsub{3}} synodis \textit{B\tsub{2} B\tsub{4}})
% Serdicensis (Sardicensis \textit{B\tsub{1}} Erdicensis \textit{B\tsub{4}} Serdicens \textit{M\tsub{3}}) \textit{BMDiE}}}\edindex[synoden]{Serdica!a. 343} factae ad Iulium urbis Romae\edindex[namen]{Julius!Bischof von Rom}
% \edtext{episcopum}{\Dfootnote{episcopo \textit{M\tsub{2}E\tsub{4}*}}},
% \edtext{Iulio episcopo a synodo
% \edtext{directae}{\Dfootnote{directa \textit{A} \latintext + exemplum de fide
% catholica \textit{B\tsub{2}}}}}{\lemma{\abb{Iulio \dots\ directae}} \Dfootnote{>
% \textit{BMDiE}}}\edindex[namen]{Julius!Bischof von Rom}.
%\pend
%\pstart
\kap{1}quod semper
\edtext{credidimus}{\Dfootnote{credimus
\textit{B\tsub{1}B\tsub{2}M\tsub{2}Di}}} etiam
\edtext{\abb{nunc}}{\Dfootnote{> \textit{E\tsub{4}}}}
\edtext{\abb{sentimus}}{\lemma{\abb{}} \Dfootnote{sentimus ut \textit{coni. Baronius}}};
\edtext{experientia}{\Dfootnote{experientiam \textit{M\tsub{1}} expermentiam \textit{M\tsub{3}}}}
\edtext{\abb{enim}}{\Dfootnote{> \textit{E\tsub{4}}}} probat et confirmat,
\edtext{quae}{\Dfootnote{que \textit{M\tsub{1}M\tsub{3}}}}
\edtext{quique}{\Dfootnote{quisque \textit{coni. C}}} auditione audivit. verum est
enim quod beatissimus
\edtext{magister gentium Paulus
\edtext{apostolus}{\Dfootnote{apostulus \textit{B\tsub{2}} apastolus \textit{B\tsub{4}}}}}{\lemma{\abb{magister \dots\ apostolus}}\Afootnote{vgl. 2Tim
1,11}}\edindex[bibel]{Timotheus II!1,11|textit} de se locutus est:
\Ladd{\edtext{�nam etsi corpore absens sum, sed spiritu vobiscum sum�}{\lemma{nam \dots\ sum}\Dfootnote{\textit{coni. Wickham} quia experimentum quaeritis eius, qui in me loquitur
Christus (2Cor 13,3) \textit{susp. Baronius}}\lemma{\abb{}}\Afootnote{Col 2,5}}}\edindex[bibel]{Kolosser!2,5}
quamquam utique, quia in
\edtext{ipso}{\Dfootnote{ipsos \textit{E\tsub{2}}}} dominus Christus
habitavit,
\edtext{dubitari}{\Dfootnote{dubitare \textit{BM\tsub{2}DiE}}} non possit
\edtext{\abb{spiritum}}{\lemma{\abb{}} \Dfootnote{spiritum sanctum \textit{coni. Faber}}} per
\edtext{animam}{\Dfootnote{animum \textit{B\tsub{2}B\tsub{3}B\tsub{4}B\tsub{6}DiM\tsub{2}E\tsub{2}E\tsub{4}}}}
\edtext{eius}{\Dfootnote{etiam \textit{coni. Baronius}}} locutum et organum corporis
\edtext{personasse}{\Dfootnote{personasset \textit{E\tsub{2}E\tsub{4}}}}.
\edtext{et tu}{\Dfootnote{si \textit{coni. Baronius}}} itaque, dilectissime frater, corpore separatus mente
\edtext{\abb{concordi ac}}{\Dfootnote{\textit{coni. Faber}
concordia \textit{ABMDiE}}}
\edtext{voluntate}{\Dfootnote{voluntati \textit{E\tsub{2}}}}
\edtext{adfuisti}{\Dfootnote{affuisti \textit{E\tsub{2} E\tsub{4}}}\lemma{\abb{adfuisti}}\Afootnote{vgl. 1Cor 5,3; Col. 2,5}}
\edindex[bibel]{Korinther I!5,3|textit}\edindex[bibel]{Kolosser!2,5|textit}
\edtext{et}{\Dfootnote{sed \textit{coni. Baronius}}} honesta fuit et necessaria
\edtext{excusatio}{\Dfootnote{excusatione \textit{E\tsub{4}} excusatio de (absentia) \textit{coni. Baronius}}} absentiae, ne
\edtext{aut}{\Dfootnote{ut \textit{E del. Baronius}}} lupi
scismatici
\edtext{furtum facerent et raperent per insidias}{\lemma{\abb{furtum \dots\
insidias}} \Dfootnote{> \textit{BMDiE}}}
\edtext{\abb{aut}}{\Dfootnote{>
\textit{B\tsub{3}B\tsub{4}} ut
\textit{E\tsub{2}}}} canes heretici
\edtext{rabido}{\Dfootnote{rapido \textit{B\tsub{2}DiE\tsub{2}} ravido \textit{A}}} furore
\edtext{exciti}{\Dfootnote{excitati \textit{BMDiE}}} insani
oblatrarent aut certe serpens
\edtext{diabolus}{\Dfootnote{diabulus
\textit{B\tsub{1}B\tsub{2}Di} diabolum
\textit{B\tsub{3}}}}
\edtext{blasphemiorum}{\Dfootnote{blasfemiorum \textit{B\tsub{1}B\tsub{3}B\tsub{4}M\tsub{2}E\tsub{4}} blasphemiarum \textit{coni. Baronius}}}
\edtext{venenum}{\Dfootnote{veninum \textit{B\tsub{3}}}}
\edtext{effunderet}{\Dfootnote{effunderit \textit{B\tsub{1}}}}. hoc
enim
\edtext{optimum}{\Dfootnote{obtimum \textit{B\tsub{4}}}} et valde congruentissimum esse videbitur, si ad
\edtext{caput}{\Dfootnote{capud \textit{AB\tsub{3}E\tsub{4}}}}, id est
\edtext{\abb{ad}}{\Dfootnote{> \textit{B\tsub{4}}}} Petri apostolici
sedem, de singulis
\edtext{quibusque}{\Dfootnote{quibuscumque \textit{coni. Baronius}}} provinciis
\edtext{domini}{\Dfootnote{domino \textit{E}}}
\edtext{referant}{\Dfootnote{refferant \textit{B\tsub{1}}}}
sacerdotes.
\pend
\pstart
\kap{2}\edtext{quoniam}{\Dfootnote{q\oline{m} \textit{AM\tsub{2}E\tsub{2}E\tsub{4}}}} ergo universa,
\edtext{quae}{\Dfootnote{qu\oline{o} \textit{B\tsub{3}Di}}}
\edtext{gesta}{\Dfootnote{iesta \textit{A}}}
\edtext{\abb{sunt}}{\Dfootnote{+ et \textit{B\tsub{1}}}},
\edtext{quae}{\lemma{quae\ts{1}}\Dfootnote{qui \textit{M\tsub{2}}}} acta,
\edtext{quae}{\lemma{quae\ts{2}}\Dfootnote{que \textit{B\tsub{3}M\tsub{3}}}}
constituta, et
\edtext{chartae}{\Dfootnote{carthe \textit{A}}} continent et vivae
\edtext{voces}{\Dfootnote{vocis
\textit{B\tsub{2}B\tsub{3}Di}}}
\edtext{carissimorum}{\Dfootnote{karissimorum \textit{B\tsub{4}DiE\tsub{2}}}}
\edtext{fratrum}{\Dfootnote{fratruum \textit{B\tsub{1}}}} et conpresbyterorum
\edtext{\abb{nostrorum}}{\Dfootnote{> \textit{B\tsub{4}}}}
\edtext{Arcydami}{\Dfootnote{Arcidami \textit{M\tsub{2}} Archidami
\textit{B\tsub{1}E\tsub{2}E\tsub{4}}}}\edindex[namen]{Archidamus!Presbyter in Rom} et
\edtext{Filoxeni}{\Dfootnote{Filioseni \textit{A}}}\edindex[namen]{Philoxenus!Presbyter in Rom} et
\edtext{carissimi}{\Dfootnote{karissimi \textit{B\tsub{4}DiE\tsub{2}} + legati Roman\oline{u} Arcydamus p\oline{b}t Filiosen\oline{u} p\oline{b}t
leo diaconus synodo Serdicensi affuerunt \textit{A\mg}}}
\edtext{filii}{\Dfootnote{fili \textit{AB\tsub{2}B\tsub{3}B\tsub{6}M\tsub{1}M\tsub{3}DiE\tsub{4}}}}
\edtext{\abb{nostri}}{\Dfootnote{> \textit{E\tsub{2}}}} Leonis\edindex[namen]{Leo!Diakon in Rom} diaconi
\edtext{verissime}{\Dfootnote{virissime \textit{B\tsub{2}Di}}} et fideliter
exponere
\edtext{poterunt}{\Dfootnote{potuerunt \textit{AM\tsub{2}}}},
\edtext{paene}{\Dfootnote{pene \textit{B\tsub{2}} poene \textit{E\tsub{2}}}} supervacuum
\edtext{videtur}{\Dfootnote{videatur \textit{BM\tsub{1}M\tsub{2}DiE} videtitur \textit{M\tsub{3}}}} eadem
\edtext{his}{\Dfootnote{hiis \textit{B\tsub{1}}}}
\edtext{inferre}{\Dfootnote{inferris \textit{E\tsub{4}} inferri \textit{coni. Baronius} inserere
\textit{coni. Faber}}}
\edtext{\abb{litteris}}{\Dfootnote{+ quae \textit{B\tsub{1}B\tsub{2}DiE\tsub{4}} + que
\textit{B\tsub{3}B\tsub{4}E\tsub{2}} + qua \textit{B\tsub{6}M}}}.
\edtext{patuit}{\Dfootnote{patuerunt \textit{B\tsub{1}}}}
\edtext{apud}{\Dfootnote{aput \textit{B\tsub{1}B\tsub{2}B\tsub{4}}}}
\edtext{omnes}{\Dfootnote{omne \textit{E\tsub{2}}}}, qui convenerunt ex partibus
Orientis, qui se appellant
\edtext{episcopos}{\Dfootnote{episcopus \textit{A*} episcopis \textit{B\tsub{1}}}} -- quamquam
\edtext{\abb{sint}}{\Dfootnote{\textit{B\tsub{2}\slin}}} ex his certi auctores,
quorum sacrilegas
\edtext{mentes}{\Dfootnote{mentis \textit{B\tsub{3}}}} Arriana 
\edtext{heresis}{\Dfootnote{hereses \textit{M\tsub{2}}}}
\edtext{pestifero}{\Dfootnote{pestiverfero \textit{B\tsub{1}}}}
\edtext{tinxit}{\Dfootnote{tincxit \textit{ADiE\tsub{2}E\tsub{4}}}}
\edtext{veneno}{\Dfootnote{venino \textit{B\tsub{3}}}} -- diu tergiversatos
propter
\edtext{diffidentiam}{\Dfootnote{differentiam \textit{BMDiE}}} ad iudicium venire noluisse
\edtext{tuamque}{\Dfootnote{tuam quae \textit{B\tsub{3}}}} et nostram
\edtext{reprehendere}{\Dfootnote{repraehendire \textit{B\tsub{3}} reprehendisse
\textit{coni. Baronius}}}
\edtext{communionem}{\Dfootnote{cummunionem \textit{A} munitionem \textit{Di}}}, quae nullam habebat
culpam, quia non solum
\edtext{octoginta}{\Dfootnote{LXXX \textit{BMDiE}}}
episcopis testantibus de innocentia Athanasi\edindex[namen]{Athanasius!Bischof von Alexandrien} pariter
\edtext{credimus}{\Dfootnote{credidimus \textit{coni. Faber}}},
sed et
\edtext{conventi}{\Dfootnote{converti \textit{B\tsub{2}B\tsub{3}B\tsub{4}B\tsub{6}MDi} convesti
\textit{B\tsub{3}}}}
\edtext{\abb{per}}{\Dfootnote{> \textit{E} pro \textit{B\tsub{1}\corr}}}
presbyteros tuos et per epistulam
\edtext{ad synodum}{\Dfootnote{et synodo \textit{A} et synodum \textit{coni. C\corr} a
synodo (sinodo\textit{B\tsub{3}}) \textit{B\tsub{2}B\tsub{3}B\tsub{4}B\tsub{6}DiM\tsub{1}M\tsub{2}M\tsub{3}}}},
quae futura erat in urbe Roma\edindex[synoden]{Rom!a. 341}, venire noluerunt
\edtext{\abb{et}}{\Dfootnote{> \textit{BMDiE}}}
satis
\edtext{erat iniquum}{\lemma{\abb{}} \Dfootnote{\responsio\ iniquam erat \textit{B\tsub{1}}}} illis
\edtext{contemnentibus}{\Dfootnote{confitentibus \textit{coni. Baronius}}},
\edtext{\abb{tot}}{\Dfootnote{\textit{del. Baronius}}} sacerdotibus testimonium
perhibentibus Marcello\edindex[namen]{Markell!Bischof von Ancyra} et Athanasio\edindex[namen]{Athanasius!Bischof von Alexandrien} denegare societatem.
\pend
\pstart
\kap{3}%
\edtext{tria}{\Dfootnote{trea \textit{BM\tsub{1}M\tsub{3}Di}}} fuerunt
quae tractanda erant. nam et ipsi
\edtext{religiosissimi}{\Dfootnote{relegiosissimi \textit{B\tsub{4}B\tsub{2}\ts{2}}
relegiosissemi \textit{B\tsub{3}} relegiossissimi \textit{B\tsub{1}} relegiosissi \textit{B\tsub{2}}}} imperatores
\edtext{permiserunt}{\Dfootnote{permisserunt \textit{B\tsub{1}}}}, ut
\edtext{de integro}{\Dfootnote{de integra \textit{M\tsub{2}} dent (i \textit{s}. en) egro
\textit{E\tsub{2}}}}
\edtext{universa}{\Dfootnote{universo \textit{B\tsub{3}}}} discussa disputarentur
et ante omnia
\edtext{de sancta
\edtext{fide}{\Dfootnote{fidei \textit{B\tsub{1}}}}}{\Dfootnote{de sanctae fidei
\textit{E}}}
\edtext{\abb{et de}}{\Dfootnote{> \textit{BMDiE}}}
integritate veritatis, quam violaverunt. secunda
\edtext{\abb{de}}{\Dfootnote{\textit{B\tsub{2}\ts{2} > cett.}}} personis,
\edtext{quos}{\Dfootnote{quas \textit{coni. Baronius}}}
\edtext{dicebant}{\Dfootnote{dicibant \textit{B\tsub{3}}}} esse
\edtext{deiectos}{\Dfootnote{diiectos \textit{B\tsub{3}B\tsub{6}M\tsub{1}M\tsub{3}} deiectas \textit{coni. Coustant} delectas \textit{coni. Baronius}}} de
iniquo iudicio,
\edtext{\abb{ut}}{\Dfootnote{\textit{coni. Faber} vel \textit{ABMDiE}}} si potuissent
\edtext{probare}{\Dfootnote{probo \textit{B\tsub{6}}}}, iusta
\edtext{fieret}{\Dfootnote{fierit \textit{B\tsub{2}Di}}} confirmatio.
tertia vero quaestio,
\edtext{\abb{quae
\edtext{vere quaestio}{\lemma{\abb{}} \Dfootnote{\responsio\ quaestio vere \textit{B\tsub{1}}}}}}{\Dfootnote{\textit{del. Baronius}}} appellanda est, quod graves et
\edtext{acerbas}{\Dfootnote{acervas \textit{B\tsub{1}B\tsub{2}B\tsub{3}B\tsub{6}M\tsub{1}M\tsub{3}}}}
\edtext{\abb{iniurias}}{\Dfootnote{\textit{coni. C} iniuriis \textit{ABM\tsub{2}M\tsub{3}DiE} inuriis \textit{M\tsub{1}}}},
\edtext{intolerabiles}{\Dfootnote{intollerabiles \textit{B\tsub{1}} intolerabilis \textit{B\tsub{3}}}} etiam et
nefarias
\edtext{contumelias}{\Dfootnote{contumilias \textit{B\tsub{1}}}} ecclesiis fecissent, cum raperent
\edtext{episcopos}{\Dfootnote{episcopis \textit{B\tsub{1}}}},
\edtext{presbiteros}{\Dfootnote{praesbiteros \textit{A} praesbyteros (presbyteros \textit{B\tsub(4)}) \textit{B}}},
\edtext{diacones}{\Dfootnote{diaconos \textit{B\tsub{2}} diac\oline{o}s \textit{E\tsub{2}} dia\oline{c} \textit{B\tsub{1}Di}}} et omnes
\edtext{\abb{clericos}}{\Dfootnote{> \textit{B\tsub{1}}}} in
\edtext{exilium}{\Dfootnote{exilio \textit{M}}} mitterent,
\edtext{ad}{\Dfootnote{a \textit{B\tsub{6}Di}}}
\edtext{deserta}{\Dfootnote{decerta \textit{B\tsub{4}}}} loca
\edtext{transducerent}{\Dfootnote{trasducerent \textit{B\tsub{2}B\tsub{4}} traducedrent \textit{B\tsub{6}M} transducerent et \textit{coni. Baronius}}},
\edtext{\abb{fame}}{\Dfootnote{+ et \textit{B\tsub{2}DiE\tsub{4}\corr}}}, siti, nuditate et
\edtext{omni egestate}{\Dfootnote{omnia aegestate \textit{B\tsub{1}}}}
\edtext{necarent}{\Dfootnote{negarent \textit{AB\tsub{2}MDi}}},
\edtext{alios clausos carcere et squalore et putore conficerent}{\lemma{\abb{alios
\dots\ conficerent}} \Dfootnote{> \textit{BMDiE}}},
\edtext{nonnullos}{\Dfootnote{nonnullo \textit{B\tsub{2}B\tsub{4}B\tsub{6}MDi}}}
\edtext{ferreis
\edtext{vinculis}{\Dfootnote{vinculi \textit{B\tsub{3}} vinculis stringerent \textit{coni. Coustant}}}}{\lemma{\abb{}}
\Dfootnote{\responsio\ vinculis ferreis \textit{B\tsub{1}}}}, ita
\edtext{ut cervices}{\Dfootnote{vicies \textit{coni. Baronius} acer vices \textit{E\tsub{4}}}} eis
\edtext{\abb{artissimis}}{\Dfootnote{\textit{coni. Faber} \latintext
arctissimis \textit{coni. C} partissimis \textit{AB\tsub{2}B\tsub{6}MDiE} parvissimis
\textit{B\tsub{1}B\tsub{3}} fartiissimis \textit{B\tsub{4}}}} circulis
\edtext{stranguilarentur}{\Dfootnote{strangularentur \textit{E} \latintext
stranquilarentur \textit{B\tsub{3} del. Baronius}}}. denique ex ipsis quidam
\edtext{vincti}{\Dfootnote{cuncti \textit{B\tsub{2}B\tsub{4}B\tsub{6}MDi}}} in eadem
\edtext{iniusta}{\Dfootnote{iniuste \textit{E}}}
\edtext{defecerunt}{\Dfootnote{deferunt \textit{B\tsub{2}M\tsub{3}Di}}}
\edtext{poena}{\Dfootnote{pena \textit{B\tsub{1}}}}, quorum
\edtext{ambigi}{\Dfootnote{ambici \textit{B\tsub{4}}}}
non potest
\edtext{martyrio}{\Dfootnote{martyrii \textit{coni. Baronius}}} gloriosam
\edtext{mortem}{\Dfootnote{morte \textit{B\tsub{6}M}}}
\edtext{extitisse}{\Dfootnote{exetitisse \textit{B\tsub{1}} extetisse \textit{B\tsub{3}} extitississe
\textit{E\tsub{4}}}}.
adhuc quoque audent quosdem
\edtext{retinere}{\Dfootnote{retinire \textit{B\tsub{3}}}} nec
\edtext{ulla causa}{\lemma{\abb{}}\Dfootnote{\responsio\ causa ulla \textit{E\tsub{4}}}}
fuit
\edtext{\abb{criminis}}{\Dfootnote{> \textit{E\tsub{2}} crimi nisi
\textit{B\tsub{1}}}}, nisi quod repugnarent et clamarent,
\edtext{quod}{\Dfootnote{quodque \textit{coni. Baronius}}}
\edtext{execrarentur}{\Dfootnote{exsecrarentur \textit{E\tsub{4}} exacrarentur \textit{M\tsub{1}M\tsub{3}} exsacrarentur (s \textit{s.l.}) \textit{M\tsub{2}} exegrarentur \textit{B\tsub{3}}}} Arrianam et
\edtext{Eusebianam}{\Dfootnote{Heuseviam \textit{A} Eusebiam \textit{B\tsub{2}B\tsub{3}B\tsub{4}B\tsub{6}DiM\tsub{2}} Eusebium \textit{B\tsub{1}}}}
\edtext{heresim}{\Dfootnote{heresem \textit{B\tsub{2}B\tsub{3}B\tsub{6}ME}}} et
\edtext{nollent}{\Dfootnote{nolent \textit{AM\tsub{1}} nolunt \textit{M\tsub{2}}}} habere cum
talibus communionem,
\edtext{\abb{qui autem}}{\Dfootnote{\textit{coni. Faber} qui autem (au \textit{E\tsub{4}}) in
\textit{B\tsub{2}B\tsub{4}B\tsub{6}MDiE} qui aut in \textit{AS\ts{2}} quia ut in
\textit{B\tsub{1}B\tsub{3}} iis autem, qui \textit{coni. Baronius}}}
\edtext{saeculo servire}{\Dfootnote{secum sentire \textit{coni. Baronius}}}
\edtext{maluerunt}{\Dfootnote{malluerunt \textit{B\tsub{1}Di} maluerint \textit{susp. Engelbrecht} noluerunt
\textit{coni.Faber}}},
\edtext{\abb{prodesse}}{\Dfootnote{\textit{M\tsub{2}} prodeesse \textit{B\tsub{2}B\tsub{3}B\tsub{6}M\tsub{3}} prod. \oline{ee} \textit{Di}
prodisse \textit{B\tsub{1}}}}.
\edtext{et qui ante
\edtext{fuerant}{\Dfootnote{fuerunt \textit{B\tsub{1}}}}
\edtext{deiecti}{\Dfootnote{deiecta \textit{E\tsub{4}*}}}}{\lemma{\abb{et \dots\
deiecti}} \Dfootnote{\textit{del. Baronius}}}, non solum recepti sunt, sed etiam ad
\edtext{clericalem}{\Dfootnote{cleric\oline{a} \textit{A}}} dignitatem promoti et
\edtext{acceperunt}{\Dfootnote{acciperunt \textit{B\tsub{1}B\tsub{2}B\tsub{3}}}} praemium
falsitatis.
\pend
\pstart
\kap{4}%
\edtext{quid}{\Dfootnote{qui \textit{M}}} autem
de impiis et de imperitis
\edtext{adulescentibus}{\Dfootnote{aduliscentibus \textit{B\tsub{3}B\tsub{4}M\tsub{2}} audulescentibus \textit{E\tsub{4} del. Baronius}}} Ursacio\edindex[namen]{Ursacius!Bischof von Singidunum}
et
\edtext{Valente statutum}{\Dfootnote{volente statutum \textit{B\tsub{1}} valentes tum \textit{M\tsub{2}} valente statum \textit{AB\tsub{2}B\tsub{6}}}}\edindex[namen]{Valens!Bischof von Mursa}
\edtext{sit}{\Dfootnote{fuit \textit{coni. Baronius}}}, accipe, beatissime
\edtext{\edtext{frater}{\Dfootnote{frafater \textit{B\tsub{1}\corr}}},
quia}{\Dfootnote{frater. quia \textit{coni. Faber}}} manifestum
erat \edtext{eos}{\Dfootnote{eum \textit{BMDiE} hos
\textit{coni. Faber}}}
\edtext{non cessare}{\Dfootnote{concessare \textit{BMDiE}}}
\edtext{adulterinae}{\Dfootnote{aduterinae \textit{M\tsub{3}}}} doctrinae
\edtext{\abb{letalia}}{\Dfootnote{(\textit{ras. inter} a \textit{et} l) \textit{A} laetalia \textit{B\tsub{1}} lethalia \textit{coni. Coustant} in
Italia \textit{coni. Baronius}}} semina spargere et quod
Valens\edindex[namen]{Valens!Bischof von Mursa}
\edtext{relicta}{\Dfootnote{relecta \textit{B\tsub{2}B\tsub{3}}}} ecclesia
\edtext{\abb{ecclesiam}}{\Dfootnote{> \textit{BMDiE}}} aliam invadere
\edtext{voluisset. eo}{\lemma{\abb{}} \Dfootnote{voluisset. eo (voluisset eo \textit{B\tsub{2}}) \textit{codd.} voluisset, et eo
\textit{coni. Baronius}}} tempore,
\edtext{quo}{\Dfootnote{quod \textit{B\tsub{3}M} \textit{que B\tsub{6}*}}}
\edtext{seditionem}{\Dfootnote{seditione \textit{E\tsub{2}}}}
\edtext{commovit}{\Dfootnote{commovet \textit{B\tsub{2}Di}}}, unus ex fratribus
nostris, qui
\edtext{fugere}{\Dfootnote{fugire \textit{B\tsub{2}M\tsub{2}Di}}} non potuit,
\edtext{viator}{\Dfootnote{Victor \textit{susp. Baronius}}} obrutus et
\edtext{\abb{conculcatus}}{\Dfootnote{\textit{E\tsub{4}*}}} in eadem
\edtext{Aquiliensium}{\Dfootnote{Aquileiensium \textit{B\tsub{2}B\tsub{3}B\tsub{6}M\tsub{1}M\tsub{3}Di}
Aquilegensium \textit{B\tsub{4}M\tsub{2}}}} \edindex[namen]{Aquileia}
\edtext{civitate}{\Dfootnote{civitatem \textit{ABMDiE\tsub{4}}}}
\edtext{tertia die}{\lemma{\abb{}} \Dfootnote{\responsio\ die tertia \textit{coni. Faber}}}
\edtext{defecit}{\Dfootnote{fecit \textit{B\tsub{6}M}}}.
\edtext{causa}{\Dfootnote{causam \textit{coni. Baronius}}} utique mortis
\edtext{\abb{\edtext{\abb{fuit}}{\Dfootnote{> \textit{BMDiE}
praebuit \textit{coni. Baronius}}} Valens, qui}}{\Dfootnote{\textit{del. Baronius}}}\edindex[namen]{Valens!Bischof von Mursa}
\edtext{perturbavit}{\Dfootnote{perturbabit \textit{B\tsub{2}B\tsub{3}} pertubabit
\textit{E\tsub{4}}}}, qui sollicitavit.
sed ea, quae
\edtext{beatissimis Augustis}{\Dfootnote{beamis agustis \textit{B\tsub{6}*M}}}
\edtext{significavimus}{\Dfootnote{significabimus \textit{B\tsub{2}DiM} signifecabimus \textit{B\tsub{3}}}}, cum
legeritis, facile pervidebitis nihil nos
\edtext{praetermisisse}{\Dfootnote{praeter misisse \textit{E\tsub{4}} praetermississe
\textit{AB\tsub{3}} praetermisse \textit{B\tsub{1}*}}}, quantum ratio
\edtext{patiebatur}{\Dfootnote{paciebatur \textit{M\tsub{2}}}}. et
\edtext{ne molesta}{\Dfootnote{nemo lesta \textit{E\tsub{4}}}} esset longa narratio,
\edtext{quae}{\Dfootnote{qui \textit{coni. Faber}}}
\edtext{fecissent}{\Dfootnote{fecisset \textit{M\tsub{2}M\tsub{3}}}}
\edtext{\abb{et quae
\edtext{commisissent}{\Dfootnote{commississent \textit{A} comississent \textit{B\tsub{1}}}}}}{\Dfootnote{> \textit{B\tsub{6}ME}}},
\edtext{insinuavimus}{\Dfootnote{insinuabimus \textit{E\tsub{4}} insinuamus
\textit{M\tsub{1}M\tsub{2}}}}.
\pend
\pstart
\kap{5}%
\edtext{tua}{\Dfootnote{tu \textit{AB\tsub{3}B\tsub{6}*E\tsub{4}}}} autem excellens prudentia
\edtext{disponere}{\Dfootnote{dispone \textit{B\tsub{2}\corr B\tsub{4}} disponi
\textit{Di}}}
\edtext{debet}{\Dfootnote{debes \textit{E}}}, ut per tua
\edtext{scripta}{\Dfootnote{scribta \textit{B\tsub{2}}}}
\edtext{in Sicilia, in Sardinia}{\Dfootnote{qui in Sicilia, (et
\textit{B\tsub{1}}) qui (que \textit{M\tsub{2}}) in Sardinia \textit{BMDiE}}}\edindex[namen]{Sizilien}\edindex[namen]{Sardinien},
\edtext{in Italia}{\Dfootnote{qui in Italia \textit{coni.
Coustant}}}\edindex[namen]{Italien}
[\edtext{\abb{sunt}}{\Dfootnote{\textit{del. Engelbrecht}}}] fratres nostri, quae acta sunt
\edtext{et}{\lemma{\abb{et\ts{1}}}\Dfootnote{> \textit{E\tsub{2}}}} quae
\edtext{definita}{\Dfootnote{difinita \textit{B\tsub{1}Di}}}, cognoscant, et ne ignorantes eorum accipiant litteras
communicatorias,
\edtext{\abb{id est
\edtext{\abb{epistolia}}{\Dfootnote{\textit{coni. Faber} epistola
\textit{A} epistulas \textit{coni TC} episcopalia \textit{S\ts{1}(episcopalis \textit{B\tsub{1}} e\oline{p}alia \textit{B\tsub{4}})BMDiE}}}}}{\Dfootnote{\textit{del. Baronius}}},
\edtext{\edtext{quos}{\Dfootnote{quas \textit{vel} quae \textit{coni. Thompson}}}
\edtext{iusta}{\Dfootnote{iuxta \textit{B\tsub{3}}}} sententia
\edtext{degradavit}{\Dfootnote{degratavit \textit{Di} degravavit
\textit{A}}}}{\Dfootnote{quos extra episcopatum synodi sententia declaravit
\textit{coni. Baronius}}}.
\edtext{perseverent}{\Dfootnote{perseverant \textit{B\tsub{1}} perseverens
\textit{M\tsub{2}}}} autem Marcellus\edindex[namen]{Markell!Bischof von Ancyra}
\edtext{\abb{et}}{\Dfootnote{> \textit{BMDiE}}}
Athanasius\edindex[namen]{Athanasius!Bischof von Alexandrien} et Asclepius\edindex[namen]{Asclepas!Bischof von Gaza} in
\edtext{nostra}{\Dfootnote{nostrae \textit{A}}}
\edtext{communione}{\Dfootnote{communionem \textit{B\tsub{2}}}}, quia
\edtext{obesse}{\Dfootnote{obsesse \textit{E\tsub{2}E\tsub{4}\corr}}} eis non
\edtext{poterat}{\Dfootnote{poterit \textit{coni. Coustant\corr}}} iniquum
iudicium,
\edtext{\edtext{fuga}{\Dfootnote{et fuga \textit{BDiM\tsub{2}M\tsub{3}E}}}
\edtext{\abb{et}}{\Dfootnote{> \textit{M\tsub{1}}}} tergiversatio}{\lemma{fuga et tergiversatio} \Dfootnote{de fuga et tergiversatione \textit{susp. Baronius} ex fuga et tergiversatione \textit{susp. Coustant}}} eorum, qui ad
\edtext{\abb{iudicium}}{\Dfootnote{+ consci \textit{B\tsub{1}}}} omnium
\edtext{episcoporum}{\Dfootnote{peccatorum \textit{B\tsub{1}B\tsub{3}}}},
\edtext{qui convenimus}{\Dfootnote{quo venimus \textit{coni. Baronius}}}, venire
noluerunt.
\edtext{cetera, sicuti superius commemoravimus}{\lemma{\abb{cetera \dots\
commemoravimus}} \Dfootnote{> \textit{BMDiE}}},
\edtext{plena relatio}{\Dfootnote{plenare relatio \textit{A}}} fratrum, quos
sincera caritas tua
\edtext{misit}{\Dfootnote{missit \textit{B\tsub{1}}}},
\edtext{unanimitatem}{\Dfootnote{unianitatem \textit{B\tsub{1}} unamitatem \textit{M\tsub{2}} immunitatem \textit{coni. Baronius} humanitatem
\textit{susp. Baronius}}} tuam
\edtext{perdocebunt}{\Dfootnote{perdocebit \textit{B\tsub{1}}}}.
eorum
\edtext{\abb{autem}}{\Dfootnote{> \textit{M}}} 
\edtext{nomina}{\Dfootnote{omnia \textit{B\tsub{6}*}}}, qui pro facinoribus
\edtext{\abb{suis}}{\Dfootnote{> \textit{B\tsub{3}M\tsub{2}}}} deiecti sunt,
\edtext{subicere}{\Dfootnote{subiicere \textit{coni. Baronius}}}
curavimus, ut sciret eximia gravitas tua, qui essent
\edtext{communione}{\Dfootnote{communioni \textit{B\tsub{2}Di} commonione \textit{B\tsub{1}}}}
\edtext{privati}{\Dfootnote{pribati \textit{E\tsub{2}}}}.
\edtext{ut ante}{\Dfootnote{uti ante \textit{B\tsub{2}B\tsub{6}M\tsub{1}E\tsub{2}Di} uni i ante \textit{B\tsub{1}} an uti te \textit{M\tsub{2}}}}
\edtext{praelocuti}{\Dfootnote{prolocuti \textit{coni. Baronius}}} sumus, omnes fratres et
coepiscopos nostros litteris tuis admonere digneris, ne
\edtext{epistolia}{\Dfootnote{epistolias \textit{B\tsub{1}} epistolas
\textit{T} epistola \textit{E\tsub{4}}}},
\edtext{id est litteras}{\Dfootnote{inde allatas \textit{coni. Baronius}}}
\edtext{communicatorias}{\Dfootnote{communicatoris \textit{M}}}
\edtext{eorum}{\Dfootnote{\textit{ras ante} eorum \textit{Di}}}, accipiant.
\pend
\pstart
\kap{6}%
Item
\edtext{nomina}{\lemma{\abb{}} \Dfootnote{Nomina tradita sunt scriptua continua;
ordo autem nominum in \textit{BMDiE} est hic:
5.1.6.2.7.3.4. \textit{B\tsub{1}} tradit nomina plane commixta et corrupta.}}
\edtext{hereticorum}{\Dfootnote{haeriticorum \textit{B\tsub{4}} herecicorum
\textit{B\tsub{1}}}}.
\pend
\pstart
\noindent 1. \edtext{Ursacius}{\Dfootnote{Ursasius \textit{B\tsub{1}}}}\edindex[namen]{Ursacius!Bischof von Singidunum} a
\edtext{Singiduno}{\Dfootnote{Sigiduno \textit{A} Singiduo \textit{B\tsub{2}B\tsub{3}B\tsub{4}MDiE\tsub{2}E\tsub{4}} Singuiduone \textit{B\tsub{1}} Syngidone \textit{coni. Baronius}}}.
\pend
\pstart
\noindent 2. Valens\edindex[namen]{Valens!Bischof von Mursa} a
\edtext{Mirsa}{\Dfootnote{Myrsa \textit{B\tsub{2}B\tsub{3}DiE} Mursa
\textit{coni. C}}}.
\pend
\pstart
\noindent 3. Narcissus\edindex[namen]{Narcissus!Bischof von Neronias} ab
\edtext{\abb{Irenopoli}}{\Dfootnote{\textit{coni. Feder} Ieropoli \textit{A} Hieropoli \textit{BMDiE} Neronopoli \textit{coni. Baronius}}}.
\pend
\pstart
\noindent 4. Stefanus\edindex[namen]{Stephanus!Bischof von Antiochien} ab Anthiocia.
\pend
\pstart
\noindent 5. Acacius\edindex[namen]{Acacius!Bischof von Caesarea} a
\edtext{Caesarea}{\Dfootnote{Cessaria \textit{B\tsub{1}} Caesariae
\textit{B\tsub{3}}}}.
\pend
\pstart
\noindent 6. \edtext{Menofantus}{\Dfootnote{Menofantis \textit{B\tsub{3}} Nemofantus
\textit{B\tsub{2}Di} Mofantus \textit{B\tsub{1}}}}\edindex[namen]{Menophantus!Bischof von Ephesus} ab Efeso.
\pend
\pstart
\noindent 7. Georgius\edindex[namen]{Georg!Bischof von Laodicea} a
\edtext{Laudocia}{\Dfootnote{Laodocia \textit{B\tsub{2}Di} Laodicia
\textit{B\tsub{1}B\tsub{3}B\tsub{4}E\tsub{2}} Laoditia \textit{M} Laocidia \textit{E\tsub{4}}}\lemma{\abb{Laudocia}}\Cfootnote{des. N}}.
\pend
\pstart
\kap{7}
\edtext{\edtext{Item}{\lemma{\abb{Item}}\Cfootnote{inc.
B\tsub{5}\griech{F}(V\tsub{1}V\tsub{2})JPR\griech{D}(D\tsub{1}D\tsub{2}D\tsub{3}D\tsub{4}D\tsub{5}D\tsub{6}D\tsub{7}D\tsub{8})H(E\tsub{2}H\tsub{2}H\tsub{3}H\tsub{4})}} nomina episcoporum infra}{\Dfootnote{et
subscripserunt, qui
convenerant \textit{a\greektext D\latintext E} subscripserunt autem omnes episcopi sic: ego ille episcopus illius civitatis et
provinciae illius ita credo, sicut scriptum est supra \textit{R} et
subscripserunt episcopi Romane ecclesie legati \textit{P} + incipit subscriptio (suprascriptio \textit{D\tsub{8}}) episcoporum \textit{\greektext D\latintext E}}},
qui in synodo fuerunt et suscripserunt
\edtext{idem}{\Dfootnote{iidem \textit{coni. C}}}
\edtext{<in>}{\lemma{\abb{in}} \Dfootnote{\textit{add. C}}} iudicio.
\pend
\pstart
\noindent 1. \edtext{Ossius}{\Dfootnote{Hosius \textit{Di}}}\edindex[namen]{Ossius!Bischof von Cordoba}
\edtext{ab}{\Dfootnote{a \textit{E} ad \textit{D\tsub{1}}}}
\edtext{Spania}{\Dfootnote{Hispania \textit{D\tsub{8}} (hi \textit{s.l.})
\textit{B\tsub{5}} Ispania \textit{P}}}
\edtext{Cordobensi}{\Dfootnote{Cordubensis (Cordobensis \textit{B\tsub{3}B\tsub{5}})
\textit{\greektext a} \latintext (Cordobensis \textit{D\tsub{8}} Cordubensi
\textit{E\tsub{2}H\tsub{2}} Cordrube\oline{n} \textit{H\tsub{4}}) \textit{\greektext
b} \textit{\latintext P} + legatis sanctae ecclesiae Romane
\textit{D\tsub{8}}}}.
\pend
\pstart
\noindent 2. \edtext{Annianus}{\Dfootnote{Annanianus \textit{AJ} Annanius \textit{D\tsub{8}}
Amanus \textit{B\tsub{5}}}}\edindex[namen]{Annianus!Bischof von Castellona}
\edtext{\abb{
\edtext{ab}{\Dfootnote{de \textit{V\tsub{1}}}}
\edtext{Spaniis}{\Dfootnote{Spania (hi \textit{s.l. B\tsub{5}}) \textit{\greektext a}
\latintext (Spani\c{e} \textit{D\tsub{1}} Asia \textit{D\tsub{2}})
\textit{\greektext b}}}}}{\Dfootnote{> \textit{P}; \textit{P om. nomen provinciae etiam
postea excepto n.7}}} de
\edtext{Castolona}{\Dfootnote{Castalona \textit{\greektext a\latintext P}
Cassalona (Moconis \textit{D\tsub{2}}) \textit{\greektext b}}}.
\pend
\pstart
\noindent 3. \edtext{Florentius}{\Dfootnote{Florentibus \textit{A} Florentinus \textit{coni. C}
Florennus \textit{S\ts{2}}}}\edindex[namen]{Florentius!Bischof von Emerita Augusta}
\edtext{ab}{\Dfootnote{de \textit{V\tsub{1}}}}
\edtext{Spaniis}{\Dfootnote{Hispaniis \textit{B\tsub{5}} Spadoniis \textit{B\tsub{2} \greektext D} \latintext Spadonis
\textit{B\tsub{1}}}}
\edtext{de}{\Dfootnote{ab \textit{P}}}
\edtext{Emerita}{\Dfootnote{Merita \textit{\greektext b} \latintext Emeritam
\textit{\greektext F}}}.
\pend
\pstart
\noindent 4. \edtext{Domitianus}{\Dfootnote{Dominian\oline{u} \textit{A*} Dominianus \textit{A\corr
S\ts{2}} Domacianus \textit{B\tsub{5}}}}\edindex[namen]{Domitianus!Bischof von Asturica}
ab
\edtext{Spaniis}{\Dfootnote{Hispanius \textit{B\tsub{5}} \oline{sp} \textit{J}
Spadoniis \textit{B\tsub{3}\greektext D} \latintext Spadonis
\textit{B\tsub{3}}}}
de
\edtext{Asturica}{\Dfootnote{Astorica \textit{B\tsub{1}} Austurica
\textit{D\tsub{4}D\tsub{6}D\tsub{8}} Auxtorica \textit{D\tsub{1}} Absturica
\textit{B\tsub{2}} Abstorica (- \oline{a} \textit{V\tsub{1}}) \textit{\greektext
F}}}.
\pend
\pstart
\noindent 5. \edtext{Castus}{\Dfootnote{Custus (Custos \textit{H\tsub{2}H\tsub{3}H\tsub{4}} Iustus
\textit{D\tsub{8}}) \textit{\greektext b}}}\edindex[namen]{Castus!Bischof von Caesaraugusta} ab
\edtext{Spaniis}{\Dfootnote{spa\oline{n} \textit{J} Hispaniis
\textit{B\tsub{5}}}}
\edtext{\abb{de}}{\Dfootnote{> \textit{P}}}
\edtext{\edtext{Caesarea}{\Dfootnote{Caesara \textit{A} Caesaria \textit{J} Cesaria
\textit{D\tsub{1}D\tsub{8}V\tsub{1}} Cessaria \textit{B\tsub{5}}}}
\edtext{Augusta}{\Dfootnote{Agusta \textit{B\tsub{1}B\tsub{5}} Ag\oline{us}
\textit{\greektext F} > \textit{D\tsub{1}}}}}{\Dfootnote{Cesaraugusta
\textit{P}}}.
\pend
\pstart
\noindent 6. \edtext{Praetextatus}{\Dfootnote{Prexextatus \textit{D\tsub{8}}}}\edindex[namen]{Praetextatus!Bischof von Barcilona} ab
\edtext{Spaniis}{\Dfootnote{Span \textit{J}
Hispaniis \textit{B\tsub{5}}}}
\edtext{\abb{de
\edtext{Barcilona}{\Dfootnote{Barcinola \textit{D\tsub{1}} Barchinola
\textit{D\tsub{2}} Barcellonia \textit{D} Baccillona (r \textit{s.} c
\textit{pr.}) \textit{E\tsub{2}} Bacillona \textit{H\tsub{2}H\tsub{3}H\tsub{4}}}}}}{\Dfootnote{>
\textit{B\tsub{5}}}}.
\pend
\pstart
\noindent 7. \edtext{\abb{\edtext{Maximus}{\Dfootnote{Maxus \textit{B\tsub{2}Di\greektext
F\latintext P}}} \edtext{a
\edtext{Tuscia}{\Dfootnote{Tuscia Augusta \textit{D\tsub{1}}
Uscia \textit{D\tsub{8}}}}}{\Dfootnote{Attuscia \textit{D\tsub{2}}}}
\edtext{\abb{de
Luca}}{\Dfootnote{> \textit{B\tsub{1}B\tsub{2}B\tsub{3}B\tsub{4}DiJ\greektext
Fb\latintext P} de Barcilona \textit{B\tsub{5}}}}}}{\Dfootnote{> \textit{J}}}.\edindex[namen]{Maximus!Bischof von Luca}
\pend
\pstart
\noindent 8. \edtext{Bassus
\edtext{a}{\Dfootnote{de \textit{A}}}
\edtext{Machedonia}{\Dfootnote{Mecodonia \textit{E\tsub{2}}}}
\edtext{\abb{de}}{\Dfootnote{> \textit{E\tsub{2}}}}
\edtext{Dioclecianopoli}{\Dfootnote{Diocletianopolim
\textit{B\tsub{3}} Diocletianopolitanus (- no \textit{V\tsub{2}}) \textit{\greektext
F} \latintext Deocletiano \textit{J} Dediocletuanopolim \textit{B\tsub{5}}
Diacletianopo. clianopolim \textit{B\tsub{1}} Philippis \textit{B\tsub{2}Di}
Diocletianeapolim (Dioclitianeapolim
\textit{D\tsub{2}D\tsub{5}} Dicletianepo\l \textit{D\tsub{8}}
Diocletianeapolitanus
\textit{E\tsub{2}}) \textit{\greektext b}}}}{\Dfootnote{] \textit{E\tsub{2} ponit n.10
ante n.8}, \textit{H\tsub{2}H\tsub{3}H\tsub{4} iungunt}: Marcellus a Macedonia
Diocletianopo\l (Diocletianeapolitanus \textit{H\tsub{2}})}}\edindex[namen]{Bassus!Bischof von Diocletianopolis}.
\pend
\pstart
\noindent 9. \edtext{\abb{\edtext{Porfirius}{\Dfootnote{Porfyrius \textit{B\tsub{3}D\tsub{8}H\tsub{2}} Porfurius
\textit{J}}} a
\edtext{Machedonia}{\Dfootnote{Macedia \textit{H\tsub{3}H\tsub{4}}}} de
Filippis}}{\Dfootnote{> \textit{B\tsub{2}Di}}}\edindex[namen]{Porphyrius!Bischof von Philippi}.
\pend
\pstart
\noindent 10. \edtext{Marcellus}{\Dfootnote{Marcellinus \textit{E\tsub{2}} Macercellus
\textit{B\tsub{1}} + a Macedonia \textit{A\greektext a\latintext E\tsub{2}} + a Machedonia \textit{\greektext D}}}\edindex[namen]{Markell!Bischof von Ancyra} de
\edtext{\edtext{Ancyra}{\Dfootnote{Anchira (Anchyra \textit{V\tsub{1}} Anchra \textit{J})
\textit{\greektext a} \latintext Anquira (Angyra \textit{D\tsub{8}} Antiquira
\textit{D\tsub{2}}) \textit{\greektext D}}}
\edtext{\abb{Galatiae}}{\Dfootnote{> \textit{\greektext
ab}}}}{\Dfootnote{] Ancyragalatia \textit{A}}}.
\pend
\pstart
\noindent 11. Euterius\edindex[namen]{Eutherius!Bischof von Ganus} a
\edtext{\abb{Tracia}}{\Dfootnote{\textit{coni. Feder} Procia \textit{A\greektext ab} \latintext civi. Prochitae \textit{coni. Crabbe}}} de
\edtext{\abb{Gannos}}{\Dfootnote{\textit{coni. Feder} Caindos \textit{A} Caindo \textit{coni. C}
Candos \textit{\greektext ab}}}.
\pend
\pstart
\noindent 12. \edtext{Asclepius}{\Dfootnote{Asclepas \textit{coni. Faber} Ascepius
\textit{H\tsub{4}}}}\edindex[namen]{Asclepas!Bischof von Gaza}
a Palestina de Gaza.
\pend
\pstart
\noindent 13. \edtext{Museus}{\Dfootnote{Musaeus \textit{coni. Ballerini} Mostus \textit{A}
Moysius \textit{S\ts{1}} Moystus (Moytus \textit{B\tsub{1}} Moestus
\textit{\greektext F}) \textit{a} \latintext (Moistus \textit{D\tsub{2}H\tsub{4}}
Zoistus \textit{D\tsub{8}}) \textit{\greektext b} \latintext Moyscus
\textit{coni. Crabbe} Moscus \textit{coni. Faber} Moschus \textit{coni.
Labbe}}}\edindex[namen]{Musaeus!Bischof von Thebae Phthiotides}
\edtext{a Tessalia}{\Dfootnote{Attessalia \textit{V\tsub{2}} Atthessalia \textit{V\tsub{1}} Thessalia \textit{S\ts{1}\greektext P\latintext JE} Thesalia \textit{DiD\tsub{2}D\tsub{8}}
Tessalonica \textit{A}}} de
\edtext{Thebis}{\Dfootnote{Thesuis \textit{B\tsub{1}} Thedis \textit{Di}}}.
\pend
\pstart
\noindent 14. \edtext{Vincentius}{\Dfootnote{Vicentius \textit{B\tsub{1}R} Vincentium
\textit{B\tsub{5}}}}\edindex[namen]{Vincentius!Bischof von Capua}
\edtext{\abb{a Campania}}{\Dfootnote{> \textit{\greektext ab}}} de
\edtext{\abb{Capua}}{\Dfootnote{+ legatus sanctae ecclesiae (Mari\c{e}
\textit{D\tsub{1}}) Romanae (Rom\c{e} \textit{Di}) \textit{\greektext ab}}}.
\pend
\pstart
\noindent 15. \edtext{\abb{Ianuarius
\edtext{\abb{a Campania}}{\Dfootnote{> \textit{\greektext a\latintext
D\tsub{8}}}} de
\edtext{Benevento}{\Dfootnote{Nevento \textit{B\tsub{1}} + legatus sanctae
(\textit{s. > J}) ecclesiae Romanae (Romae \textit{Di} ss \textit{J})
\textit{\greektext ab}}}}}{\Dfootnote{> \textit{B\tsub{5}D\tsub{1-7}E}}}\edindex[namen]{Januarius!Bischof von Beneventum}.
\pend
\pstart
\noindent 16. \edtext{Protogenes}{\Dfootnote{Protogenis \textit{B\tsub{1}B\tsub{5}}(Protagenis
\textit{V\tsub{2}}) \textit{\greektext F} \latintext (Protegenis \textit{D\tsub{1}})
\textit{\greektext b} \latintext Proteginis \textit{Di}}}\edindex[namen]{Protogenes!Bischof von Serdica}
\edtext{\abb{a Dacia}}{\Dfootnote{\textit{S\ts{1}} ab Acacia \textit{A} ab Acaia (Ataia \textit{V\tsub{2}}) \textit{\greektext a} \latintext ab Achaia
(Achaigia \textit{D\tsub{1}}) \textit{\greektext b}}} de
\edtext{Serdica}{\Dfootnote{Sardica \textit{E\tsub{2}H\tsub{2}H\tsub{4}} Serdia
\textit{D\tsub{2}} Serica \textit{B\tsub{1}}}}.
\pend
\pstart
\noindent 17. \edtext{Dioscorus}{\Dfootnote{Dyoscorus \textit{V\tsub{1}D\tsub{1}} Dyoserus
\textit{D\tsub{8}}}}\edindex[namen]{Dioscurus!Bischof von Therasia} de
\edtext{\abb{Terasia}}{\Dfootnote{\textit{coni. Feder} Therasiae \textit{coni. Crabbe} Tracia \textit{A\greektext P\latintext J\greektext F} Trachia \textit{Di}(Trotia \textit{D\tsub{2}})
\textit{\greektext b}}}.
\pend
\pstart
\noindent 18. \edtext{Himeneus}{\Dfootnote{Hymeneus (Himeneus \textit{B\tsub{1}V\tsub{2}} Himenius
\textit{B\tsub{5}}) \textit{\greektext a} \latintext Ymeneus (Ymineus
\textit{D\tsub{3}D\tsub{6}} Omenius \textit{D\tsub{8}}) \textit{\greektext b}
\latintext Himonius \textit{A} Hymenius \textit{S\ts{1}}}}\edindex[namen]{Hymenaeus!Bischof von Hypata}
\edtext{a}{\Dfootnote{ad \textit{B\tsub{5}}}}
\edtext{Tessalia}{\Dfootnote{Thessalonia \textit{D\tsub{3}}}} de
\edtext{\abb{Hypata}}{\Dfootnote{\textit{coni. Labbe} oparata
\textit{A} pearata \textit{S\ts{1}} pharata \textit{\greektext ab}}}.
\pend
\pstart
\noindent 19. \edtext{\abb{Lucius
\edtext{a}{\Dfootnote{de \textit{D\tsub{1}}}}
\edtext{Tracia}{\Dfootnote{Trachia \textit{\greektext b}
\latintext Tracia (o \textit{s.} ac) \textit{B\tsub{2}} Traochia
\textit{Di}}} de
\edtext{Cainopoli}{\Dfootnote{Caynopoli \textit{coni. C} Cainopolim (Cainop\oline{o}\l
\textit{J} Caimopolim \textit{B\tsub{1}}) \textit{\greektext a} \latintext
(Cainopoli \textit{H\tsub{2}H\tsub{3}H\tsub{4}} Camolim
\textit{D\tsub{1}} Diainopo\l \textit{D\tsub{8}}) \textit{\greektext b}
Hadrianopoli \textit{coni. Ed. regia} Nicopoli
\textit{coni. Hardouin}}}}}{\Dfootnote{> \textit{B\tsub{5}}; \textit{B\tsub{1} om.
nomina episcoporum a n.19 usque ad n.40}}}\edindex[namen]{Lucius!Bischof von Adrianopel}.
\pend
\pstart
\noindent 20. \edtext{Lucius}{\Dfootnote{item Lucius (Lucidus \textit{D\tsub{3}})
\textit{\greektext b}}}\edindex[namen]{Lucius!Bischof von Verona} ab
\edtext{Italia}{\Dfootnote{Hitalia \textit{V\tsub{2}}}}
\edtext{de}{\Dfootnote{a \textit{P}}}
\edtext{Verona}{\Dfootnote{Berina \textit{D\tsub{8}} Verano \textit{Di}}}.
\pend
\pstart
\noindent 21. \edtext{
\edtext{\abb{Eugenius}}{\Dfootnote{\textit{coni. Erl.} Euagrius \textit{codd.} Euagius \textit{B\tsub{5}\corr}}} a
Machedonia
\edtext{\abb{de}}{\Dfootnote{> \textit{E}}}
\edtext{\abb{Eraclia Linci}}{\Dfootnote{\textit{coni. Feder} Heraclea Laoci
\textit{coni. Ballerini} Eraclialineo \textit{A} Heraclianopolim (Heraclia\oline{np} \textit{J}
Herachianopolim \textit{B\tsub{5}} Agrianopolim \textit{Di}) \textit{\greektext a}
\latintext Eraclianopolim (> \textit{D\tsub{1}}  Heraclianopolim
\textit{D\tsub{4}D\tsub{6}} Heraclianopolitanus \textit{E}) \textit{\greektext
b}}}.}{\Dfootnote{> \textit{D\tsub{8}}}}\edindex[namen]{Eugenius!Bischof von Heraclea Lyncestis}
\pend
\pstart
\noindent 22. \edtext{Iulius}{\Dfootnote{Iulios \textit{A*}}}\edindex[namen]{Julianus!Bischof von Thebae Heptapylus} ab
\edtext{Acaia}{\Dfootnote{Acatia \textit{A} Acacia (Hacacia \textit{V\tsub{1}} Acaia \textit{B\tsub{1}}) \textit{\greektext a} \latintext (Achaia \textit{D\tsub{3}} Acias \textit{E}) \textit{\greektext b}}} de
\edtext{\abb{Tebe eptapilos}}{\Dfootnote{\textit{coni. Feder} Thebe eptapyleos
\textit{coni. Coustant} Thebeeptapoleos \textit{coni. Ed. regia}
Tebepsetafileos \textit{A} Theb\oline{e}pse (Theb epi\oline{sc} \textit{B\tsub{3}}
Chebe\oline{ps} \textit{B\tsub{1}} Thee\oline{ps} \textit{\greektext F} \latintext
Ligmedon \textit{B\tsub{5}}) \textit{\greektext a} Thebessem
(Thebessen \textit{D\tsub{2}} Tebessem \textit{D\tsub{4}} Tevessem \textit{E})
\textit{\greektext b}}}.
\pend
\pstart
\noindent 23. \edtext{Zosimus}{\Dfootnote{Diosimus \textit{A} Zomus \textit{E\tsub{2}} Dionisius
\textit{coni. Faber}}}\edindex[namen]{Zosimus!Bischof von Lychnidus} a
\edtext{\abb{Macedonia}}{\Dfootnote{\textit{coni. Labbe} Achaia \textit{A} Acaia (Acacia \textit{B\tsub{2}}) \textit{\greektext a}
Acacia (Acia \textit{D\tsub{4}} Dacia \textit{E\tsub{2}H\tsub{2}H\tsub{3}} Datia \textit{H\tsub{4}} >
\textit{D\tsub{1}}) \textit{\greektext b}}} de
\edtext{\abb{Lignido}}{\Dfootnote{\textit{coni. Hardouin} Lychnido
\textit{coni. Ed. regia} Lignedon \textit{A} Ligmedon (\textit{> B\tsub{5}}) \textit{\greektext a}
 \latintext (Ligmedone \textit{E} Ligiodemis
\textit{D\tsub{2}} Limedon \textit{D\tsub{8}} Eracheanopolim
\textit{D\tsub{1}}) \textit{\greektext b}}}.
\pend
\pstart
\noindent 24. \edtext{Athenodorus}{\Dfootnote{Anthenodorus \textit{D\tsub{4}} Anthenodorius
\textit{D\tsub{1}} Thenodorus \textit{D\tsub{8}} Atemodorus \textit{B\tsub{5}}}}\edindex[namen]{Athenodorus!Bischof von Elatia} ab
\edtext{Achaia}{\Dfootnote{Acacia \textit{B\tsub{2}} Achacia \textit{Di} \latintext Acacia
(Achia \textit{D\tsub{3}})
\textit{\greektext D} \latintext Dacia \textit{E}}} de
\edtext{\abb{Elatea}}{\Dfootnote{\textit{susp. LeQuien II 205} Plataea \textit{coni. Crabbe} Blatea
\textit{A} Blatena \textit{\greektext a} \latintext (Bletana \textit{E\tsub{2}}) \textit{\greektext b}}}.
\pend
\pstart
\noindent 25. \edtext{Diodorus}{\Dfootnote{Dyodorus \textit{E\tsub{2}H\tsub{2}H\tsub{3}}}}\edindex[namen]{Diodorus!Bischof von Tenedus}
\edtext{ab Asia de
\edtext{Tenedos}{\Dfootnote{Fenedos \textit{\greektext F} \latintext Thenodos
\textit{D\tsub{4}} Thenodus \textit{D\tsub{8}} Nethis \textit{E}}}}{\Dfootnote{ab
Acacia de Ligmedon (\textit{u. n.23}) ab Acia \textit{D\tsub{1}} ab Asilia
\textit{D\tsub{8}}}}.
\pend
\pstart
\noindent 26. Alexander\edindex[namen]{Alexander!Bischof von Larisa}
\edtext{a}{\Dfootnote{de \textit{D\tsub{8}} > \textit{D\tsub{1}}}}
\edtext{Thessalia}{\Dfootnote{Thesulia \textit{D\tsub{2}} Thesilia
\textit{D\tsub{6}} Thasalia \textit{D\tsub{8}} > \textit{D\tsub{1}}}} de
\edtext{Larissa}{\Dfootnote{Lurissa \textit{B\tsub{3}B\tsub{5}} Clarissa (Clorissa
\textit{D\tsub{8}}) \textit{\greektext b}}}.
\pend
\pstart
\noindent 27. \edtext{A"ethius}{\Dfootnote{Letius \textit{D\tsub{8}}}}\edindex[namen]{A"etius!Bischof von Thessalonike} a
\edtext{Machedonia}{\Dfootnote{Macedonica \textit{E\tsub{2}H\tsub{2}}}} de
\edtext{Tessalonica}{\Dfootnote{Thesa\l o \textit{D\tsub{8}} The\oline{ss}
\textit{E\tsub{2}} Thessa\l \textit{H\tsub{2}H\tsub{3}}}}.
\pend
\pstart
\noindent 28. Vitalis\edindex[namen]{Vitalis!Bischof von Aquae}
\edtext{
\edtext{a Dacia Ripensi}{\Dfootnote{a Dacia \textit{\griech{a} E} ad (ab \textit{D\tsub{3}}) Achia (Achaia \textit{D\tsub{1}D\tsub{6}D\tsub{8}}) \textit{\griech{D}}}} de
\edtext{Aquis}{\Dfootnote{Aquis Ripensi (Ripensis \textit{B\tsub{1}}) \textit{\griech{P}
DiJV\tsub{1}} Aquis Ripensis (Repensis \textit{D\tsub{3}} Rapiensis \textit{E\tsub{2}}
Rapien\oline{s} \textit{H\tsub{2}} Rapiens \textit{H\tsub{3}H\tsub{4}} Macarcae
\textit{D\tsub{8}}) \textit{\griech{b}}}}}{\lemma{a D. de Aq.}\Dfootnote{ad aqua de aq: Ripensi \textit{V\tsub{2}}}}.
\pend
\pstart
\noindent 29. \edtext{Paregorius}{\Dfootnote{Paragorius \textit{A} (Paracoriu
\textit{Di\corr}) \textit{\greektext a} Parigorius (Porigorius
\textit{D\tsub{2}} Pargorius \textit{D\tsub{5}} Prigorius \textit{D\tsub{8}})
\textit{\greektext b}}}\edindex[namen]{Paregorius!Bischof von Scupi}
\edtext{a Dardania}{\Dfootnote{ab Ardiania (\textit{pr.} i \textit{del.})
\textit{D\tsub{2}}}}
\edtext{de Scupis}{\Dfootnote{discupis \textit{D\tsub{2}} a Descopis
\textit{\greektext F} Scipis \textit{J}}}.
\pend
\pstart
\noindent 30. \edtext{\edtext{Trifon}{\Dfootnote{Trifom \textit{D\tsub{2}} Arifon
\textit{D\tsub{8}}}} ab
\edtext{Acaia}{\Dfootnote{Achia (Achaia
\textit{D\tsub{8}}) \textit{\greektext D} Acacia
\textit{B\tsub{1}E}}}}{\Dfootnote{Trifona. Bacaia \textit{J}}}\edindex[namen]{Tryphon!Bischof von Macaria?} de
\edtext{\abb{Macaria}}{\Dfootnote{\textit{coni. Feder} Macarce \textit{AB\tsub{2}Di\greektext F} \latintext
Macarcae \textit{J} Macarche (Masarc\c{e}
\textit{D\tsub{2}} Marcharce \textit{D\tsub{4}} Naiso
\textit{D\tsub{1}}) \textit{\greektext b} Marcarce \textit{B\tsub{1}B\tsub{5}} Marcarcae
\textit{B\tsub{3}} Maccare \textit{S\ts{1}} Marciarce \textit{coni. C} Maccroce \textit{coni. Faber} Macroce
\textit{coni.Hardouin} Marathone \textit{coni. Crabbe}}}.
\pend
\pstart
\noindent 31. \edtext{\abb{Athanasius}}{\Dfootnote{+ a tha \textit{D\tsub{1}}}}\edindex[namen]{Athanasius!Bischof von Alexandrien}
\edtext{ab Alexandria}{\Dfootnote{Alexandriae \textit{\griech{ab}}}}.
\pend
\pstart
\noindent 32. \edtext{Gaudentius}{\Dfootnote{Gaudetius \textit{H\tsub{3}}}}\edindex[namen]{Gaudentius!Bischof von Na"issus}
\edtext{a Dacia}{\Dfootnote{ab Acaia \textit{A} ad (ab
\textit{D\tsub{2}D\tsub{7}D\tsub{8}} ac \textit{D\tsub{3}}) Achaia (Achia
\textit{D\tsub{5}D\tsub{7}} Asia \textit{D\tsub{2}}) \textit{\greektext D} \latintext
ab Acia \textit{B\tsub{1}} ad acta \textit{Di} Agatia \textit{H\tsub{4}}}} de \edtext{Naiso}{\Dfootnote{Naison \textit{D\tsub{3}} Nasio \textit{D\tsub{8}} Natso
\textit{Di} Macari \textit{H\tsub{2}H\tsub{4}} Mari
\textit{H\tsub{3}} Bacharce \textit{D\tsub{1}}}}.
\pend
\pstart
\noindent 33. \edtext{Ionas}{\Dfootnote{Ionan \textit{A} Jonam \textit{coni. C} Iunas
\textit{D\tsub{1}} Zonas \textit{D\tsub{8}} Thonas \textit{Di} Iohanas
\textit{H\tsub{3}H\tsub{4}} Joannes \textit{coni. LeQuien II 75}}}\edindex[namen]{Jonas!Bischof von Parthicopolis} a
\edtext{Machedonia}{\Dfootnote{Madonia \textit{B\tsub{1}}}} de
\edtext{Particopoli}{\Dfootnote{Particopolim (Partico\oline{p} \textit{E\tsub{2}}
Particop\l \textit{H\tsub{2}H\tsub{4}} Partiop\l \textit{H\tsub{3}} parte
co\oline{p} \textit{D\tsub{8}}) \textit{\greektext b} \latintext Parthenopoli
\textit{coni. Binius}}}.
\pend
\pstart
\noindent 34. \edtext{\abb{
\edtext{Alypius}{\Dfootnote{Alipius \textit{\greektext F} \latintext Aolipius (Alipius \textit{E\tsub{2}}
Aolypius \textit{D\tsub{6}} Oylipius \textit{D\tsub{8}}) \textit{\greektext b}}} ab
\edtext{Acaia}{\Dfootnote{Achia (Achaia \textit{D\tsub{3}D\tsub{6}D})
\textit{\greektext D} \latintext Tacaia \textit{B\tsub{5}} Acacia \textit{E\tsub{2}}
acta \textit{Di}}} de
\edtext{\abb{Megara}}{\Dfootnote{\textit{coni. Ballerini} Megaris \textit{coni. Cochlaeus} Magari \textit{A} (Magri
\textit{B\tsub{3}B\tsub{5}}) \textit{\greektext ab}}}}}{\Dfootnote{> \textit{H\tsub{2}H\tsub{3}H\tsub{4}}, \textit{sed u. n.32}}}\edindex[namen]{Alypius!Bischof von Megara}.
\pend
\pstart
\noindent 35. Machedonius\edindex[namen]{Macedonius!Bischof von Ulpiana}
\edtext{a Dardania}{\Dfootnote{ab Ardania \textit{A} a Dardaniae \textit{B\tsub{1}}}} de
Ulpianis.
\pend
\pstart
\noindent 36. \edtext{Calvus}{\Dfootnote{Calvius \textit{\greektext a} \latintext (Salvius
\textit{D\tsub{8}}) \textit{\greektext b} \latintext Caius \textit{coni. T}}}\edindex[namen]{Calvus!Bischof von Castra Martis}
\edtext{a Dacia}{\Dfootnote{ad (ab \textit{D\tsub{3}}) Achia (Achaia
\textit{D\tsub{3}D\tsub{6}D\tsub{8}}) \textit{\greektext D} \latintext a Dacacia
\textit{E} a Dacta \textit{JDi}}}
\edtext{\abb{Ripensi}}{\Dfootnote{> \textit{H\tsub{4}} Repensi \textit{B\tsub{5}} Aripensis \textit{D\tsub{8}}}}
de
\edtext{Castramartis}{\Dfootnote{Castra Martis \textit{\greektext ab} \latintext Martis de castra
\textit{D\tsub{1}} Castra Artis \textit{Di}}}.
\pend
\pstart
\noindent 37. \edtext{Fortunatianus}{\Dfootnote{Fortunatius \textit{V\tsub{1}D\tsub{1}} Fortunianus
\textit{D\tsub{8}} Fortunatus
\textit{D\tsub{3}D\tsub{5}D\tsub{6}D\tsub{7}H\tsub{2}H\tsub{3}H\tsub{4}P} Furtunatus
\textit{D\tsub{2}}}}\edindex[namen]{Fortunatianus!Bischof von Aquileia}
\edtext{\abb{ab Italia
\edtext{de}{\Dfootnote{ab \textit{P}}}}}{\Dfootnote{> \textit{\greektext ab}}}
\edtext{Aquileia}{\Dfootnote{Aquileiensis (Aquilegensis \textit{B\tsub{2}}
Aquiligensis \textit{Di}) \textit{\greektext a} \latintext Aquilegensis
(Aquiligensi \textit{D\tsub{2}}) \textit{\greektext b}}}.
\pend
\pstart
\noindent 38. \edtext{Plutarcus}{\Dfootnote{Plotarcus \textit{A\greektext a} \latintext
Protarcus \textit{\greektext D}}}\edindex[namen]{Plutarchus!Bischof von Patras}
\edtext{ab}{\Dfootnote{ad \textit{D\tsub{1}}}}
\edtext{Acaia}{\Dfootnote{Acacia \textit{A} Achaia (Achia
\textit{D\tsub{1}D\tsub{2}}) \textit{\greektext b}}}
\edtext{de}{\Dfootnote{a de \textit{D\tsub{2}}}}
\edtext{Patras}{\Dfootnote{patres \textit{Di}}}.
\pend
\pstart
\noindent 39. \edtext{Eliodorus}{\Dfootnote{Heliodorus \textit{\greektext P\latintext
JV\tsub{1}\greektext b} \latintext Helioporus \textit{D\tsub{4}} Telrodorus
\textit{D\tsub{8}}}}\edindex[namen]{Heliodorus!Bischof von Nicopolis}
\edtext{a}{\Dfootnote{de \textit{\greektext b}}}
\edtext{Nicopoli}{\Dfootnote{Nicopolim (Nicopoli \textit{J}) \textit{\greektext
a} (Negopolim \textit{D\tsub{2}} Nicopoli \textit{H\tsub{2}H\tsub{4}})
\textit{\greektext b}}}.
\pend
\pstart
\noindent 40. \edtext{Euterius}{\Dfootnote{Eutarius \textit{A} Etherius \textit{B\tsub{5}}}}\edindex[namen]{Eutherius!Bischof in Pannonien} a
\edtext{Pannoniis}{\Dfootnote{Pannonis \textit{B\tsub{1}} PPanoniis (\textit{pr.}
p \textit{del.}) \textit{D\tsub{5}}}}.
\pend
\pstart
\noindent 41. \edtext{Arius}{\Dfootnote{Prius \textit{D\tsub{8}}}}\edindex[namen]{Arius!Bischof von Petra} a Palestina.
\pend
\pstart
\noindent 42. \edtext{Asterius}{\Dfootnote{Asturius \textit{A} (Austurius \textit{\greektext F})
\textit{\greektext a} (Austurius \textit{H\tsub{4}} Austusius \textit{D\tsub{2}}
Astusius \textit{D\tsub{4}} Sturius \textit{D\tsub{8}}) \textit{\greektext b}}}\edindex[namen]{Asterius!Bischof in Arabien}
\edtext{ab}{\Dfootnote{de (di \greektext \textit{F}) \textit{a} \textit{b}}}
\edtext{Arabia}{\Dfootnote{Arbia \textit{B\tsub{2}Di\greektext F}}}.
\pend
\pstart
\noindent 43. \edtext{\abb{Socras}}{\Dfootnote{\textit{coni. Engelbrecht} Socrates \textit{coni. Ballerini} Cocras
\textit{A} Chocras (Thocras \textit{B\tsub{3}})
\textit{\greektext a} \latintext (Choras \textit{D\tsub{2}} Thoras
\textit{D\tsub{8}}) \textit{\greektext b}}}\edindex[namen]{Socras!Bischof von Phoebia am Asopus} ab
\edtext{Acaia}{\Dfootnote{Achia
\textit{D\tsub{4}} Achicia \textit{D\tsub{8}} Achaiacha \textit{E\tsub{2}H\tsub{4}}
Achiacha \textit{H\tsub{3}H\tsub{4}}}}
\edtext{de
\edtext{Asopofoebiis}{\Dfootnote{\textit{coni. Feder} Asapofoebus \textit{A}
Asaspofoebiis \textit{S\ts{1}} Asapofebiis \textit{coni. C} Asopoli
\textit{coni. Ballerini}}}}{\Dfootnote{deas a pofoebus (deas a pofoedus
\textit{B\tsub{5}} de asabofebus \textit{Di} deas aspofoebus \textit{J})
\textit{\greektext a} \latintext deas (leas \textit{D\tsub{8}}) a pofibus (a
pofiliis \textit{H\tsub{3}H\tsub{4}} polfilius \textit{D\tsub{8}}) \textit{\greektext
b}}}.
\pend
\pstart
\noindent 44. \edtext{\abb{
\edtext{Stercorius}{\Dfootnote{Atercorius \textit{D\tsub{8}}}}
\edtext{\abb{ab Apulia}}{\Dfootnote{> \greektext \textit{ab}}} de
\edtext{Canusio}{\Dfootnote{Canusia \textit{\greektext b}}}.}}{\Dfootnote{>
\textit{J}}}\edindex[namen]{Stercorius!Bischof von Canusium}
\pend
\pstart
\noindent 45. \edtext{\edtext{Calepodius}{\Dfootnote{Calipodius \textit{\greektext a}
(Calypodius \textit{D\tsub{8}E} Caliopidius \textit{D\tsub{1}}) \textit{\greektext
b}}}
\edtext{\abb{a Campania}}{\Dfootnote{> \textit{\greektext ab}}}
\edtext{<de Neapoli>}{\lemma{+ de Neapoli} \Dfootnote{\textit{P} > \textit{A} Neapolitanus
legatus (legatus > \textit{V\tsub{1}}) sanctae (sanctae > \textit{J})
ecclesiae Romanae (Rom\c{e} \textit{Di} \oline{ss} \textit{J}) \textit{\greektext
ab}}}}{\lemma{\abb{}} \Dfootnote{\textit{D\tsub{8} enumerat bis episcopum},
\textit{sc. sub n.4 et 33}, \textit{sub n.4 sine verbis} legatus \textit{etc.}}}\edindex[namen]{Calepodius!Bischof von Neapolis}.
\pend
\pstart
\noindent 46. \edtext{Ireneus}{\Dfootnote{Ierenius \textit{A} (Iereneus \textit{B\tsub{5}}
Gerenius \textit{\greektext F}) \textit{\greektext a} Hierenius (Herenius \textit{D\tsub{2}}
Hirenius \textit{E} Perunius \textit{D\tsub{8}}) \textit{\greektext b}}}\edindex[namen]{Irenaeus!Bischof von Scyrus}
\edtext{ab Acaia}{\Dfootnote{Achagia \textit{D\tsub{1}D\tsub{4}
Achia D\tsub{2}}}} de
\edtext{Sciro}{\Dfootnote{Scyro \textit{coni. LeQuien II 232} Scoro
\textit{V\tsub{2}} Secoro \textit{A\greektext P\latintext DiJV\tsub{1}} (Socoro
\textit{D\tsub{1}E\tsub{2}} Sechroro \textit{D\tsub{2}} Helida \textit{D\tsub{8}}) \textit{\greektext b}}}.
\pend
\pstart
\noindent 47. \edtext{Martyrius}{\Dfootnote{Martirius \textit{B\tsub{5}V\tsub{2}D\tsub{1}H\tsub{4}}}}\edindex[namen]{Martyrius!Bischof von Naupactus}
ab
\edtext{Acaia}{\Dfootnote{Achia \textit{D\tsub{2}D\tsub{4}}}} de
\edtext{\abb{Naupacto}}{\Dfootnote{\textit{coni. Ballerini} Neapoli \textit{ADi}
Neapolim \textit{\greektext aD} \latintext Neapo\l \textit{E} Megapoli
\textit{coni. LeQuien II 187}}}.
\pend
\pstart
\noindent 48. \edtext{\abb{%
\edtext{Dionysius}{\Dfootnote{Dyonisius (Dionisius
\textit{B\tsub{1}B\tsub{2}\greektext F} Dyonius \textit{J*}) \textit{\greektext
a}}} ab
\edtext{Acaia}{\Dfootnote{Achia \textit{D\tsub{2}} Achigia
\textit{D\tsub{1}} Achagia \textit{D\tsub{4}} Acaida \textit{A}}} de
\edtext{Elida}{\Dfootnote{Elyda (Eylidu \textit{B\tsub{5}} Lida \textit{\greektext
F}) \textit{\greektext a} \latintext Helida (Elidos \textit{D\tsub{1}} Heda
\textit{E\tsub{2}H\tsub{2}} Geda \textit{H\tsub{3}H\tsub{4}}) \textit{\greektext
b}}}}}{\Dfootnote{> \textit{D\tsub{8}}}}\edindex[namen]{Dionysius!Bischof von Elis}.
\pend
\pstart
\noindent 49. \edtext{Severus}{\Dfootnote{Severius \textit{D\tsub{1}}}}\edindex[namen]{Severus!Bischof von Ravenna}
\edtext{\abb{ab Italia}}{\Dfootnote{> \textit{\greektext ab}}}
\edtext{de}{\Dfootnote{a \textit{P} > \textit{\greektext ab}}}
\edtext{Ravennensi}{\Dfootnote{Ravennensis (Ravanensis \textit{B\tsub{5}}) \textit{\greektext a} \latintext (Revenensis
\textit{D\tsub{1}D\tsub{2}}) \textit{\greektext b} \latintext Ravenna \textit{P}}}.
\pend
\pstart
\noindent 50. \edtext{Ursacius}{\Dfootnote{Versatius \textit{D\tsub{2}} Orsacius
\textit{\greektext F} Ursicinus \textit{D\tsub{3}} Madersatius \textit{D\tsub{8}}}}\edindex[namen]{Ursacius!Bischof von Brixia}
ab
\edtext{Italia}{\Dfootnote{Hitalia \textit{V\tsub{2}} Hytalia \textit{V\tsub{1}}}}
\edtext{de}{\Dfootnote{a \textit{P}}}
\edtext{Brixa}{\Dfootnote{Brixia \textit{P} Bryxa \textit{V\tsub{1}}}}.
\pend
\pstart
\noindent 51. \edtext{Protasius}{\Dfootnote{Portasius \textit{AB\tsub{2}\corr}}}\edindex[namen]{Protasius!Bischof von Mailand}
\edtext{\abb{ab Italia}}{\Dfootnote{> \textit{\greektext ab}}}
\edtext{\edtext{de}{\Dfootnote{a \textit{P}}}
Mediolano}{\Dfootnote{Mediolanensis (Medilanesis \textit{V\tsub{2}\corr}) \textit{\greektext
ab}}}.
\pend
\pstart
\noindent 52. \edtext{Marcus}{\Dfootnote{Marcius \textit{Di}}}\edindex[namen]{Marcus!Bischof von Siscia}
\edtext{\abb{a Savia}}{\Dfootnote{\textit{coni. Ballerini} ab Asia
\textit{A\greektext ab} \latintext a Dacia \textit{S\ts{1}}}} de 
\edtext{Siscia}{\Dfootnote{Fissia \textit{A} Scissia \textit{S\ts{1}} Sistia
\textit{D\tsub{2}}}}.
\pend
\pstart
\noindent 53. \edtext{Verissimus}{\Dfootnote{Virissimus \textit{D\tsub{3}} Verissemus
\textit{B\tsub{5}} Gerissimus \textit{D\tsub{8}}}}\edindex[namen]{Verissimus!Bischof von Lugdunum}
\edtext{\abb{a}}{\Dfootnote{> \textit{V\tsub{1}} ad (d \textit{del.}) \textit{V\tsub{2}}}}
Gallia
\edtext{de}{\Dfootnote{a \textit{P}}}
\edtext{Lugduno}{\Dfootnote{Lugdono \textit{B\tsub{1}} \textit{B\tsub{5}*}
\textit{D\tsub{4}D\tsub{5}D\tsub{6}} Luduno \textit{A} Lodudo \textit{V\tsub{2}*}
Lodino \textit{V\tsub{2}\corr} Ludono \textit{V\tsub{1}} + et ceteri numero
quadraginta quinque \textit{P}}}.
\pend
\pstart
\noindent 54. Valens\edindex[namen]{Valens!Bischof von Oescus}
\edtext{a Dacia}{\Dfootnote{Addacia \textit{H\tsub{4}} ab (de \textit{B\tsub{2}})
Acia \textit{AB\tsub{2}} ab (ad \textit{D\tsub{1}D\tsub{2}D\tsub{4}D\tsub{5}}) Achia
(Achaia \textit{D\tsub{3}D\tsub{8}}) \textit{\greektext D} \latintext de acta
\textit{Di}}}
\edtext{Ripensi}{\Dfootnote{Ripense \textit{A} (Ripensi \textit{B\tsub{2}}
Ripensis \textit{B\tsub{3}B\tsub{5}}) \textit{\greektext a} \latintext (Rpense
\textit{D\tsub{5}D\tsub{6}D\tsub{7}} Rep\oline{e}se \textit{H\tsub{4}} Pense
\textit{D\tsub{3}}) \textit{\greektext b}}}
\edtext{\abb{de Isco}}{\Dfootnote{\textit{coni. Farlati, Illyricom sacrum VII 612} de
Scio \textit{A} de Cio \textit{coni. C} de Sciopolitanus (Sciopolitano
\textit{V\tsub{2}} Sciopo\l \l \textit{J}) \textit{\greektext a} \latintext de
Scitopolitanus (Scitoponitanus \textit{D\tsub{1}} Scipolitanus \textit{D\tsub{2}}
Scitopo\l \textit{D\tsub{8}} Sciop\l \textit{E}) \textit{\greektext b}
\latintext Scythopolitanus \textit{coni. Cochlaeus}}}.
\pend
\pstart
\noindent 55. \edtext{Palladius}{\Dfootnote{Pallidius \textit{D\tsub{1}} Palladdiensis
(\textit{pr.} d \textit{del.}) \textit{B\tsub{5}}}}\edindex[namen]{Palladius!Bischof von Dium}
\edtext{a}{\Dfootnote{ad (d \textit{del.}) \textit{V\tsub{2}}}}
Machedonia de
\edtext{\abb{Dio}}{\Dfootnote{\textit{coni. Coustant} Diu \textit{A\greektext
ab}}}.
\pend
\pstart
\noindent 56. \edtext{Geroncius}{\Dfootnote{Erontius \textit{V\tsub{2}} Hyerontius
\textit{V\tsub{1}}}}\edindex[namen]{Gerontius!Bischof von Beroea}
\edtext{a}{\Dfootnote{de \textit{B\tsub{2}Di}}}
Machedonia de
\edtext{Bereu}{\Dfootnote{\textit{coni. Feder} Beroae \textit{coni. Crabbe} Beroea
\textit{coni. LeQuien II 71} brevi \textit{A} (> \textit{B\tsub{2}Di}) \textit{\greektext a} \latintext brebi (brebis \textit{D\tsub{3}} brevia
\textit{D\tsub{8}} bebri \textit{E}) \textit{\greektext b}}}.
\pend
\pstart
\noindent 57. \edtext{Alexander}{\Dfootnote{Alexandrus \textit{D\tsub{4}} Alexander de brevi \textit{B\tsub{2}Di}}}\edindex[namen]{Alexander!Bischof von Cyparissia} ab
\edtext{Acaia}{\Dfootnote{Achia \textit{D\tsub{3}} Acha \textit{E} \latintext Acia \textit{B\tsub{2}Di\greektext F}}} de
\edtext{\abb{Ciparissia}}{\Dfootnote{\textit{coni. Feder} Cyparissa \textit{coni. Ballerini}
Ceparisma \textit{A} Ceparisina \textit{S\ts{1}} Ceporisma (Ceporissima
\textit{B\tsub{5}Di}) \textit{\greektext a} \latintext (> \textit{D\tsub{2}} Cepolisma \textit{D\tsub{8}}) \textit{\greektext b}
\latintext Morenis \textit{coni. C}}}.
\pend
\pstart
\noindent 58.(59.) \edtext{Euticius}{\Dfootnote{Eutycius \textit{V\tsub{1}D\tsub{8}} Eutychius
\textit{D\tsub{5}D\tsub{7}} Eutichius (Euchyus
\textit{H\tsub{2}}) \textit{E}}}\edindex[namen]{Eutychius!Bischof von Methone}
\edtext{ab
\edtext{\abb{Acaia}}{\Dfootnote{+ de Moremis \textit{E}}}
\edtext{\abb{[\edtext{Ticius}{\Dfootnote{Tycius
\textit{V\tsub{1}} Tychius (Thitius \textit{D\tsub{2}}
Thichius \textit{D\tsub{3}} Tichius \textit{D\tsub{4}} Thycius \textit{D\tsub{8}})
\textit{\greektext b}}}
\edtext{\abb{ab Asia}}{\Dfootnote{> \textit{D\tsub{2}}}}]}}{\lemma{Ticius ab Asia} \Dfootnote{\textit{del. Engelbrecht}}}}{\lemma{\abb{}} \Dfootnote{ab Acaia Ticius >
\textit{B\tsub{2}Di}}} de
\edtext{\abb{Motonis}}{\Dfootnote{(Motenis \textit{B\tsub{5}} Montonis \textit{J})
\textit{\greektext a} \latintext (Motanis \textit{E\tsub{2}} Montanis
\textit{H\tsub{2}} M\oline{o}tanis \textit{H\tsub{4}} Moritanis \textit{H\tsub{3}}
Ciporisma \textit{D\tsub{2}}) \textit{\greektext b} Montonis \textit{A}
Methona \textit{coni. LeQuien II 230} Methonis \textit{coni. Hardouin}}}.
\pend
\pstart
\noindent 60. \edtext{\abb{Alexander ab
\edtext{Acaia}{\Dfootnote{Achaia \textit{\greektext b}}} de
\edtext{\abb{Coronis}}{\Dfootnote{\textit{coni. Feder} Corone \textit{coni. LeQuien I 1196}
Moremis \textit{A} (Moromis \textit{V\tsub{2}} Meromis \textit{V\tsub{1}})
\textit{\greektext a} \latintext (> \textit{E} Morenis \textit{D\tsub{4}})
\textit{\greektext b} \latintext Moreniis \textit{coni. Cochlaeus} Morenis
\textit{coni. Hardouin} Maroneae \textit{coni. Crabbe} Messene \textit{coni. Le
Quien II 196}}}.}}{\Dfootnote{\textit{del. C}}}\edindex[namen]{Alexander!Bischof von Corone}
\pend
\pstart
\noindent episcopi omnes numero
\edtext{sexaginta et unus}{\Dfootnote{sexaginta et unum (\textit{forte ex ILX})
\textit{A} unum et (unus de \textit{coni. Coustant}) sexaginta \textit{coni. C} LXI e\oline{p}i
\textit{A\mg} + et ceteri (ceteri > \textit{R}) subscripserunt (+ omnes
Episcopi diversarum provinciarum vel cicitatum \textit{\greektext b})
\textit{\greektext ab} \latintext + NUM CXXI \textit{D\tsub{6}}}}.
\pend
% \endnumbering
\end{Leftside}
\begin{Rightside}
\begin{translatio}
\beginnumbering
%\pstart
%Hier beginnt die Abschrift des an Iulius, den Bischof von Rom geschriebenen
%Briefes, dem Bischof Iulius von der Synode zugesandt.
%\pend
\pstart
\noindent\kapR{1}Was wir immer geglaubt haben, der Meinung sind wir auch jetzt; die Erfahrung
beweist und bekr�ftigt n�mlich, was ein jeder durch H�rensagen vernommen hat. 
Wahr ist n�mlich, was der hochselige Lehrer der V�lker,
der Apostel Paulus, �ber sich gesagt hat: ">Denn auch wenn ich k�rperlich abwesend bin, so
bin ich doch durch den Geist bei euch"<, obwohl freilich, weil ja gerade in ihm der Herr
Christus wohnte, nicht daran gezweifelt werden kann, da� der Geist durch seine Seele
gesprochen hat und durch das Instrument seines K�rpers ert�nte. Auch du, vielgeliebter
Bruder, warst daher, wenn auch k�rperlich getrennt, einm�tig in Geist und Gesinnung  
bei uns; und die Rechtfertigung f�r deine Abwesenheit war sowohl ehrenhaft als auch
notwendig, damit n�mlich die schismatischen W�lfe keinen Diebstahl begingen und durch ihre
Intrigen keinen Raubzug durchf�hrten, ferner die h�retischen Hunde, zu tollw�tiger Raserei
aufgestachelt, nicht wie wahnsinnig aufbellten oder die teufliche Schlange nicht so treffsicher das
Gift ihrer L�sterungen verspritzte.\footnoteA{�ber spezielle Unruhen in Rom, weshalb Julius
dort bleiben wollte, ist nichts bekannt.} So scheint es also am besten
und �beraus angemessen zu sein, wenn die Bisch�fe des Herrn aus den jeweiligen einzelnen Provinzen
dem Haupt, das hei�t, dem Stuhl des Apostels Petrus, Bericht erstatten.
\pend
\pstart
\kapR{2}Da also die Protokolle alles, was geschehen ist, was verhandelt
und beschlossen wurde, enthalten, als auch die m�ndlichen Berichte unserer teuersten Br�der
und Mitpresbyter Archidamus und Philoxenus und unseres teuersten Sohnes, des Diakons Leo,
dies v�llig wahrheitsgem�� und zuverl�ssig werden darlegen k�nnen, erscheint es beinahe
�berfl�ssig, dasselbe diesem Brief anzuvertrauen. Es war f�r alle klar, da� die, die aus
den Gebieten des Ostens zusammenkamen und sich Bisch�fe nennen~-- obwohl unter diesen
gewisse Anstifter sind, deren frevlerische Sinne die arianische H�resie mit
unheilbringendem Gift getr�nkt hat~--, nicht zur Verhandlung kommen wollten, nachdem sie lange Zeit aus Mi�trauen Ausfl�chte
gemacht hatten. Es war au�erdem klar, da� sie deine Gemeinschaft mit uns tadelten, die
keinerlei Schuld in sich trug, nicht nur weil wir die gleiche Ansicht hatten wie achtzig
weitere Bisch�fe, die die Unschuld des Athanasius bezeugten,\footnoteA{Gemeint ist der
Brief der alexandrinischen Synode aus dem Jahr 338, �berliefert in Ath., apol.\,sec. 3--19.}
sondern auch, weil sie entgegen der Vereinbarung, die sie mit deinen Presbytern und brieflich getroffen hatten, 
zu der Synode, die in Rom stattfinden sollte, nicht
kommen wollten, und weil es reichlich ungerecht gewesen w�re, nur weil jene sich ablehnend
�u�erten, w�hrend so viele Priester Zeugnis f�r sie ablegten, dem Marcellus und dem
Athanasius die Gemeinschaft zu verweigern.
\pend
\pstart
\kapR{3}Drei Punkte gab es, die zu behandeln waren. Denn sogar die h�chst frommen
Herrscher selbst haben es gestattet, da� alle Themen noch einmal neu verhandelt werden, und
zwar vor allem der heilige Glaube und die Unversehrtheit der Wahrheit, die sie
besch�digt haben. Zweitens die Personen, von denen es immer wieder hei�t,
sie seien aufgrund eines ungerechten Urteils abgesetzt worden, so da�, wenn sie dies 
beweisen k�nnten, ihr Fall rechtm��ig best�tigt werde. Drittens aber gab es
einen Streitpunkt, der wahrhaft ein Streitpunkt genannt werden mu�, weil jene den Kirchen
schweres und bitteres Unrecht, sogar unertr�gliche und gottlose Schmach angetan hatten,
als sie Bisch�fe verschleppten, Presbyter, Diakone und alle Kleriker in die Verbannung
schickten, an verlassene Orte �berf�hrten und durch Hunger, Durst, Beraubung der Kleider
und jedweden Mangel sterben lie�en, andere im Kerker einsperrten und durch Schmutz und F�ulnis
umbrachten und einige mit solchen eisernen Fesseln versahen, da� ihnen die H�lse mit ganz
engen Ringen zugeschn�rt wurden. Schlie�lich starben einige von diesen Gefesselten
unter genau dieser ungerechten Strafe, von deren Martyrium nicht bestritten werden kann,
da� es einen ruhmreichen Tod �berragt hat. Sogar bis jetzt wagen sie es, einige
festzuhalten, und es gab keinen Grund f�r eine Anklage, au�er da� sie sich widersetzten und
schrien, da� sie die arianische und eusebianische H�resie verfluchten und nicht wollten,
da� es von Vorteil sei, mit solchen Leuten Gemeinschaft zu haben, die lieber der Welt
dienen wollten. Und die zuvor verbannt worden waren, wurden von jenen nicht nur wieder aufgenommen, sondern sogar zu klerikaler W�rde erhoben und f�r ihre Falschheit belohnt.
\pend
\pstart
\kapR{4}Was aber bez�glich der frevlerischen und unerfahrenen jungen M�nner Ursacius und
Valens beschlossen worden ist, so vernimm, seligster Bruder, da� es offensichtlich war,
da� sie nicht davon ablassen wollten, die todbringenden Samen der falschen Lehre
auszustreuen, und da� Valens, nachdem er seine Kirche verlassen hatte, in eine andere
eindringen wollte.\footnoteA{Zu Valens von Mursa und Ursacius von Singidunum vgl. Dok.
\ref{sec:SerdicaWestBekenntnis},2. Welche Unruhen in Aquileia geschahen und warum sich
Valens dort aufhielt, ist unbekannt. Aus Aquileia unterschrieb
Fortunatianus f�r den Westen (s. Liste). Zu weiteren Unruhen vgl. auch Dok.
\ref{sec:SerdicaRundbrief},6.} Zu der Zeit, als er einen Aufruhr anzettelte, wurde einer
von unseren Br�dern, der nicht fliehen konnte, als Reisender �berw�ltigt und mit F��en 
getreten. Er starb in eben jener Stadt der
Aquileier am dritten Tag. Die Ursache f�r seinen Tod war jedenfalls Valens, der Unruhe
stiftete und aufwiegelte. Aber wenn ihr das lest, was wir den h�chstseligen Augusti angezeigt haben, werdet ihr leicht sehen, da� wir nichts �bergangen haben, soweit es die
Vernunft zulie�. Und damit nicht ein langer m�ndlicher Bericht beschwerlich werde, haben
wir ausf�hrlich Einblick gegeben in das, was sie getan und begangen hatten.
\pend
\pstart
\kapR{5}Deine herausragende Klugheit aber mu� es einrichten, da� durch deine Schriften
unsere Br�der in Sizilien, Sardinien und Italien erfahren, was verhandelt und was
festgelegt wurde, und da� sie nicht unwissend deren Empfehlungsschreiben, also unbedeutende
Briefe, von denen akzeptieren, die gerechtes Urteil abgesetzt hat. Marcellus, Athanasius
und Asclepas aber sollen in unserer Gemeinschaft verbleiben, weil ihnen die ungerechte
Verurteilung, die Flucht und die Ausfl�chte derer, die nicht zur Verhandlung aller
versammelten Bisch�fe kommen wollten, nicht schaden konnten. Die �brigen Sachverhalte wird,
wie wir weiter oben angemerkt haben, der umfassende Bericht der Br�der, die deine reine
Liebe gesandt hat, deiner Geneigtheit ausf�hrlich klarstellen. Die Namen derer aber, die f�r ihre
Untaten abgesetzt worden sind, haben wir unten anf�gen lassen, damit deine
au�erordentliche Erhabenheit wei�, welche Leute von der Gemeinschaft ausgeschlossen
sind. Wie wir vorher schon gesagt haben, m�gest du geruhen, alle unsere Br�der und
Mitbisch�fe durch deine Briefe zu ermahnen, keine unbedeutenden Briefe von ihrer Seite, das hei�t
Empfehlungsschreiben, zu akzeptieren.
\pend
\pstart
\kapR{6}Ebenso die Namen der H�retiker:\footnoteA{Vgl. auch die Erw�hnung in Dok.
\ref{sec:SerdicaRundbrief},14; 16 und die Anmerkung dort.}
\pend
\pstart
\noindent 1. Ursacius aus Singidunum
\pend
\pstart
\noindent 2. Valens aus Mursa
\pend
\pstart
\noindent 3. Narcissus aus Irenopolis (= Neronias)
\pend
\pstart
\noindent 4. Stephanus aus Antiochia
\pend
\pstart
\noindent 5. Acacius aus Caesarea
\pend
\pstart
\noindent 6. Menophantus aus Ephesus
\pend
\pstart
\noindent 7. Georgius aus Laodicea
\pend
\pstart
\kapR{7} Ebenso die Namen der Bisch�fe darunter, die an der Synode teilgenommen haben und
dasselbe in der Verhandlung unterschrieben haben.
\pend
\pstart
\noindent 1. Ossius aus Cordoba in Spania
\pend
\pstart
\noindent 2. Annianus aus Castellona in Spania
\pend
\pstart
\noindent 3. Florentius aus Emerita Augusta in Spania
\pend
\pstart
\noindent 4. Domitianus aus Asturica in Spania
\pend
\pstart
\noindent 5. Castus aus Caesaraugusta in Spania
\pend
\pstart
\noindent 6. Praetextatus aus Barcilona in Spania
\pend
\pstart
\noindent 7. Maximus aus Luca in Tuscia
\pend
\pstart
\noindent 8. Bassus aus Diocletianopolis in Macedonia
\pend
\pstart
\noindent 9. Porphyrius aus Philippi in Macedonia
\pend
\pstart
\noindent 10. Marcellus aus Ancyra in Galatia
\pend
\pstart
\noindent 11. Eutherius aus Ganus in Thracia
\pend
\pstart
\noindent 12. Asclepius aus Gaza in Palaestina
\pend
\pstart
\noindent 13. Musaeus aus Thebae Phthiotides in Thessalia
\pend
\pstart
\noindent 14. Vincentius aus Capua in Campania
\pend
\pstart
\noindent 15. Januarius aus Beneventum in Campania
\pend
\pstart
\noindent 16. Protogenes aus Serdica in Dacia
\pend
\pstart
\noindent 17. Dioscurus aus Therasia
\pend
\pstart
\noindent 18. Hymenaeus aus Hypata in Thessalia
\pend
\pstart
\noindent 19. Lucius aus Cainopolis/Hadrianopolis in Thracia
\pend
\pstart
\noindent 20. Lucius aus Verona in Italia
\pend
\pstart
\noindent 21. Eugenius aus Heraclea Lyncestis in Macedonia
\pend
\pstart
\noindent 22. Julianus aus Thebae Heptapylos in Achaia
\pend
\pstart
\noindent 23. Zosimus aus Lychnidus in Macedonia
\pend
\pstart
\noindent 24. Athenodorus aus Elatia in Achaia
\pend
\pstart
\noindent 25. Diodorus aus Tenedus in Asia
\pend
\pstart
\noindent 26. Alexander aus Larisa in Thessalia
\pend
\pstart
\noindent 27. A"etius aus Thessalonike in Macedonia
\pend
\pstart
\noindent 28. Vitalis aus Aquae in Dacia ripensis
\pend
\pstart
\noindent 29. Paregorius aus Scupi in Dardania
\pend
\pstart
\noindent 30. Tryphon aus Macaria in Achaia
\pend
\pstart
\noindent 31. Athanasius von Alexandrien
\pend
\pstart
\noindent 32. Gaudentius von Na"issus in Dacia
\pend
\pstart
\noindent 33. Jonas aus Parthicopolis in Macedonia
\pend
\pstart
\noindent 34. Alypius aus Megara in Achaia
\pend
\pstart
\noindent 35. Macedonius aus Ulpiana in Dardania
\pend
\pstart
\noindent 36. Calvus aus Castra Martis in Dacia ripensis
\pend
\pstart
\noindent 37. Fortunatianus aus Aquileia in Italia
\pend
\pstart
\noindent 38. Plutarchus aus Patras in Achaia
\pend
\pstart
\noindent 39. Heliodorus aus Nicopolis
\pend
\pstart
\noindent 40. Eutherius aus Pannonien
\pend
\pstart
\noindent 41. Arius aus Palaestina
\pend
\pstart
\noindent 42. Asterius aus Arabia
\pend
\pstart
\noindent 43. Socras aus Phoebia am Asopus in Achaia
\pend
\pstart
\noindent 44. Stercorius aus Canusium in Apulia
\pend
\pstart
\noindent 45. Calepodius aus Neapolis in Campania
\pend
\pstart
\noindent 46. Irenaeus aus Scyrus in Achaia
\pend
\pstart
\noindent 47. Martyrius aus Naupactus in Achaia
\pend
\pstart
\noindent 48. Dionysius aus Elis in Achaia
\pend
\pstart
\noindent 49. Severus aus Ravenna in Italia
\pend
\pstart
\noindent 50. Ursacius aus Brixia in Italia
\pend
\pstart
\noindent 51. Protasius aus Mailand in Italia
\pend
\pstart
\noindent 52. Marcus aus Siscia in Savia
\pend
\pstart
\noindent 53. Verissimus aus Lugdunum in Gallia
\pend
\pstart
\noindent 54. Valens aus Oescus in Dacia ripensis
\pend
\pstart
\noindent 55. Palladius aus Dium in Macedonia
\pend
\pstart
\noindent 56. Gerontius aus Beroea in Macedonia
\pend
\pstart
\noindent 57. Alexander aus Cyparissia in Achaia
\pend
\pstart
\noindent 58.(59.) Eutychius aus Methone in Achaia
\pend
\pstart
\noindent 60. Alexander aus Corone in Achaia
\pend
\pstart
\noindent Alle Bisch�fe waren 61 an der Zahl.
\pend
\endnumbering
\end{translatio}
\end{Rightside}
\Columns
\end{pairs}
% \thispagestyle{empty}
\clearpage
%%% Local Variables: 
%%% mode: latex
%%% TeX-master: "dokumente_master"
%%% End: 

% \cleartooddpage
\section[Fragment des Briefes des Ossius von Cordoba und des Protogenes von
Serdica an Julius von Rom][Ossius von Cordoba und Protogenes von
Serdica an Julius von Rom]{Fragment des Briefes des Ossius von Cordoba und des Protogenes von
Serdica\\an Julius von Rom}
% \label{sec:43.2}
\label{sec:BriefOssiusProtogenes}
% \diskussionsbedarf
\begin{praefatio}
  \begin{description}
  \item[Herbst 343]Zum Datum vgl. Einleitung zu
    Dok. \ref{ch:SerdicaEinl}. Der f�hrende Bischof der ">westlichen"<
    Synode, Ossius\index[namen]{Ossius!Bischof von Cordoba}
    (vgl. Anm. zu Dok. \ref{sec:SerdicaRundbrief},5), und der
    Ortsbischof von Serdica schreiben an
    Julius\index[namen]{Julius!Bischof von Rom} von Rom, um die
    Abfassung einer l�ngeren Glaubenserkl�rung zu
    rechtfertigen. Wahrscheinlich handelt es sich um den Begleitbrief,
    dem die Glaubenserkl�rung als Anlage beigef�gt wurde.
  \item[�berlieferung]Der Text ist nur sehr schlecht im Codex
    Veronensis �berliefert (vgl. den textkritischen
    Apparat). S. \edpageref{lacuna1},\lineref{lacuna1} ist eine L�cke
    anzunehmen; ebenso ist der Schlu� des Briefes nicht �berliefert.
    Bei der uns vorliegenden lateinischen Fassung handelt es sich um
    eine �bersetzung aus dem Griechischen; darauf deutet die
    Konstruktion des Satzes S. \refpassage{ne-1}{ne-2} (\textit{ne
      quis} = \griech{<'ina mhde`ic} wg. \textit{et excludatur};
    \textit{ne fiat} wird dann wieder mit \textit{ne} konstruiert, da
    dies im Griechischen notwendig ist; f�r das Lateinische
    au�ergew�hnlich ist zudem die Sperrung zwischen \textit{coegit}
    und \textit{exponere}). Zur �berlieferung des Codex Veronensis
    vgl. Dok.\ref{sec:SerdicaRundbrief}. 
  \item[Fundstelle]\refpassage{cod01}{cod02} Codex Veronensis LX,
    f. 80b--81a; \refpassage{soz01}{soz02} Soz., h.\,e. III 12,6
    (\editioncite[116,19--25]{Hansen:Soz})
  \end{description}
\end{praefatio}
\begin{pairs}
\selectlanguage{latin}
\begin{Leftside}
% \beginnumbering
\pstart
\hskip -1.2em\kap{1,1}\edtext{\abb{}}{\xxref{cod01}{cod02}\Cfootnote{Cod.\,Ver.}}
\specialindex{quellen}{section}{Codices!Veronensis LX!f.
80b--81a}Dilectissimo\edlabel{cod01} fratri Iulio\edindex[namen]{Julius!Bischof von Rom}
Osius\edindex[namen]{Ossius!Bischof von Cordoba} \edtext{\abb{et Protogenes}}{\Dfootnote{\textit{cod.\corr} et pro
et Protogenes \textit{cod*}}}\edindex[namen]{Protogenes!Bischof von
Serdica}.
\pend
\pstart
Meminimus et tenemus et habemus illam scripturam 
\edtext{\abb{quae}}{\Dfootnote{\textit{coni. Ballerini} que \textit{cod.}}} continet catholicam fidem factam
aput Niceam\edindex[synoden]{Nicaea!a. 325} et consenserunt omnes qui aderant episcopi.
tres enim questiones 
\edtext{\abb{motae}}{\Dfootnote{\textit{coni. Ballerini} mote \textit{cod.}}} sunt:
\edtext{%
\edtext{\abb{quod}}{\Dfootnote{\textit{coni. Opitz} quad \textit{cod.}}}
erat quando non erat}{\lemma{quod erat quando non erat}\Dfootnote{quod erat aliquando quando non erat, et quia ex nullis existentibus factus est, aut ex alia substantia vel essentia dicunt esse convertibilem, aut mutabilem Filium Dei \textit{susp. Ballerini} \Ladd{quod de non exstantibus est filius deo
vel ex alia substantia et non ex deo et} quod erat \Ladd{aliquando} quando non erat
\textit{coni. Tetz}}} \edtext{\abb{\Ladd{\dots}}}{\Dfootnote{\textit{lacunam susp. Opitz}}}.\edlabel{lacuna1} 
\edtext{sed quoniam post hoc discipuli Arrii
\edtext{\abb{blasphemias}}{\Dfootnote{\textit{vel} blasphemi \textit{susp. Ballerini} blasphemiae \textit{cod.}}}
conmoverunt}{\lemma{\abb{\responsio\ sed quoniam \dots\ conmoverunt}}
\Dfootnote{\textit{ante} tres enim \dots\ \textit{coni. Opitz}}}, ratio 
\edtext{\abb{quaedam}}{\Dfootnote{\textit{coni. Ballerini} quedam \textit{cod.}}} coegit,
ne\edlabel{ne-1} quis ex illis tribus argumentis
\edtext{\abb{circumventus}}{\Dfootnote{\textit{cod.\corr} circummentus \textit{cod.*}}}
\edtext{\abb{renovet}}{\Dfootnote{\textit{coni. Ballerini} renovent \textit{cod.} renuerit
\textit{coni. Opitz} removeat \textit{coni. Tetz}}} fidem 
\edtext{et excludatur eorum spolium et
\edtext{\abb{ne fiat}}{\Dfootnote{\textit{cod.\corr} ne fia \textit{cod.*} nefas \textit{coni. Tetz}}}
\edtext{latior et longior}{\Dfootnote{latiorem et longiorem \textit{coni. Opitz} latius
et longius \textit{coni. Tetz}}},
\edtext{exponere}{\lemma{\abb{}} \Dfootnote{\textit{Tetz interpunxit post}
exponere}} priori
\edtext{consentientes}{\Dfootnote{consentientem \textit{coni. Opitz}}}}{\Dfootnote{ut excludatur eorum scholium adversus Nicaenam fidem, et fiat latior et longior expositio priori consentientes \textit{susp. Ballerini e Sozomeno}}}\edlabel{ne-2}.
\pend
\pstart
\kap{2}ut igitur nulla
\edtext{\abb{reprehensio}}{\Dfootnote{\textit{coni. Ballerini} reprehentio \textit{cod.}}} fiat, 
\edtext{\abb{haec}}{\Dfootnote{\textit{coni. Ballerini} hec \textit{cod.}}}
significamus 
\edtext{\abb{tuae}}{\Dfootnote{\textit{coni. Ballerini} tue \textit{cod.}}} bonitati, frater dilectissime.
\edtext{\abb{priora}}{\Dfootnote{\textit{coni. Opitz} plura \textit{cod.}}} placuerunt
firma esse et fixa et haec plenius cum quadam sufficientia veritatis
dictari, ut omnes docentes et caticizantes clarificentur et
\edtext{\abb{repugnantes}}{\Dfootnote{\textit{susp. Ballerini} prepugnantes \textit{cod}
propugnantes \textit{coni. Ballerini}}} obruantur et teneant catholicam et apostolicam
fidem.
\pend
\pstart
\noindent\edtext{\abb{\Ladd{\dots}}}{\Dfootnote{\textit{lacunam susp. Erl.}}}\edlabel{cod02}
\pend
% \endnumbering
\end{Leftside}
\begin{Rightside}
\begin{translatio}
\beginnumbering
\pstart
\noindent Ossius\looseness=-1\ und Protogenes an den vielgeliebten Bruder Iulius.
\pend
\pstart
Wir gedenken jener Schrift, die den in Nicaea beschlossenen Glauben enth�lt, halten an ihr
fest und bewahren sie, und alle Bisch�fe, die anwesend waren, haben zugestimmt. Drei
Fragestellungen sind n�mlich aufgeworfen worden: Da� (eine Zeit) war, als er nicht war,
\Ladd{\dots}.\footnoteA{Die L�cke ist entsprechend der
niz�nischen Anathematismen zu erg�nzen.} Aber weil danach die Sch�ler des Arius
Blasphemien aufgebracht haben, zwang ein gewisses Ma� an Vernunft dazu, da� wir diesen in �bereinstimmung mit
dem fr�heren Glauben erl�utern, damit nicht irgendeine Zusammenkunft infolge
jener drei Streitfragen den Glauben erneuert und damit ausgeschlossen wird, da� der Glaube zur
Beute jener Leute wird, und damit es nicht geschieht, da� er erweitert oder erg�nzt wird.
\pend
\pstart
Damit es also zu keinem Tadel kommt, weisen wir deine G�te, vielgeliebter Bruder, auf
folgendes hin: Man hat beschlossen, da� das Fr�here bekr�ftigt und festgelegt ist, ebenso da�
dies ausf�hrlicher gesagt und damit sozusagen der Wahrheit gen�ge getan wird. Dadurch sollen alle, die
lehren und unterweisen, erleuchtet und die, die Gegenwehr leisten, zum Schweigen gebracht
werden und alle am katholischen und apostolischen Glauben festhalten.
\pend
\pstart
\noindent\Ladd{\dots}
\pend
\endnumbering
\end{translatio}
\end{Rightside}
\Columns
\end{pairs}

\autor{Regest bei Sozomenus}
\begin{pairs}
\selectlanguage{polutonikogreek}
\begin{Leftside}
\pstart
\edtext{\abb{}}{\xxref{soz01}{soz02}\Cfootnote{\dt{Soz. (BC=b T)}}}\specialindex{quellen}{section}{Sozomenus!h.\,e.!III 12,6}
% \kap{2,1}>ex'ejento d`e ka`i a>uto`i
% \edtext{\abb{thnika~uta}}{\Dfootnote{\dt{> Soz.(T)}}}
% p'istewc graf`hn <et'eran, platut'eran m`en t~hc >en Nika'ia|\edindex[synoden]{Nicaea!a. 325}, ful'attousan d`e
% t`hn a>ut`hn di'anoian ka`i o>u par`a pol`u diall'attousan t~wn >eke'inhc
% \edtext{<rhm'atwn}{\Dfootnote{<rht~wn \dt{Soz.(T)}}\lemma{\abb{<rhm'atwn}}\Cfootnote{\dt{des. Soz.(T)}}}.
% \pend
% \pstart
% \selectlanguage{polutonikogreek}
\hskip -1.35em\kap{2,1}>am'elei\edlabel{soz01} <'Osioc\index[namen]{Ossius!Bischof von Cordoba} ka`i Prwtog'enhc\edindex[namen]{Protogenes!Bischof von Serdica}, o<`i t'ote \edtext{<up\-~hrqon
>'arqontec}{\Dfootnote{>~hrqon \dt{Soz.(C)}}}
t~wn >ap`o t~hc d'usewc >en Sardik~h|\index[synoden]{Serdica!a. 343} sunelhluj'otwn, de'isantec >'iswc, m`h
nomisje~i'en tisi kainotome~in t`a d'oxanta to~ic >en Nika'ia|\edindex[synoden]{Nicaea!a. 325}, >'egrayan
>Ioul'iw|\edindex[namen]{Julius!Bischof von Rom} ka`i >emart'uranto k'uria t'ade <hge~isjai, kat`a qre'ian d`e
safhne'iac t`hn a>ut`hn di'anoian plat~unai, <'wste m`h >eggen'esjai to~ic t`a
>Are'iou\edindex[namen]{Arianer} frono~usin >apokeqrhm'enoic t~h| suntom'ia| t~hc graf~hc e>ic >'atopon
<'elkein to`uc >ape'irouc dial'exewc.\edlabel{soz02}
\pend
% \endnumbering
\end{Leftside}
\begin{Rightside}
\begin{translatio}
\beginnumbering
\pstart
\noindent Im �brigen schrieben Ossius und Protogenes, die damals die Vorsitzenden der aus
dem Westen in Serdica Zusammengekommenen waren, vielleicht aus Furcht, da�
einige glauben k�nnten, sie wollten die Beschl�sse im Vergleich zu denen in
Nicaea erneuern, an Julius und bezeugten, da� sie diese f�r g�ltig hielten, da�
sie aber denselben Sachverhalt der Deutlichkeit wegen breiter dargestellt h�tten,
damit nicht die, die mit den Lehren des Arius sympathisieren, die M�glichkeit
h�tten, aufgrund der Knappheit der Urkunde die in Diskussionen Unerfahrenen in
Verlegenheit zu bringen.
\pend
\endnumbering
\end{translatio}
\end{Rightside}
\Columns
\end{pairs}
% \thispagestyle{empty}
%%% Local Variables: 
%%% mode: latex
%%% TeX-master: "dokumente_master"
%%% End: 

%%%% Input-Datei OHNE TeX-Pr�ambel %%%%
\section[Briefe der ">westlichen"< Synode an die Kirche Alexandriens
und an die Bisch�fe �gyptens und Libyens][Briefe der ">westlichen"<
Synode]{Briefe der ">westlichen"< Synode an die Kirche Alexandriens
  und an die Bisch�fe �gyptens und Libyens}
% \label{sec:43.2}
\label{sec:BriefSerdikaAlexandrien}
\begin{praefatio}
  \begin{description}
  \item[Herbst 343]Zum Datum vgl. die Einleitung zu
    Dok. \ref{ch:SerdicaEinl}. Den Brief an die Kirche
    Alexandriens\index[namen]{Alexandrien} hat
    Athanasius\index[namen]{Athanasius!Bischof von Alexandrien} von
    der Synode erbeten, um neben seinen eigenen Aussagen
    (vgl. Dok. \ref{sec:BriefAthAlexParem}) eine synodale Best�tigung
    seiner Unschuld und Wiedereinsetzung als Bischof von Alexandrien
    in der Hand zu haben (� 1,19 [\refpassage{unschuld1}{unschuld2}]).
    Die Tatsache jedoch, da�
    Athanasius\index[namen]{Athanasius!Bischof von Alexandrien} erst
    nach dem Tod des alexandrinischen Gegenbischofs
    Gregor\index[namen]{Gregor!Bischof von Alexandrien} im Jahr 345 (�
    1,16) und nach der Zustimmung auch des Kaisers
    Constantius\index[namen]{Constantius, Kaiser} zur�ckkehren konnte,
    belegt, da� die Verh�ltnisse komplizierter waren, als der Brief
    vermuten l��t.\\
    Ein (fast) identischer Brief (� 2,1--2
    [\refpassage{brief2-1}{brief2-2}]; s.  �berlieferung), der aus
    diesem Grund auch nicht �berliefert ist, ging an die Bisch�fe
    �gyptens\index[namen]{Aegyptus} und Libyens\index[namen]{Libya}.\\
    Beiden war jeweils der Synodalbrief
    Dok. \ref{sec:SerdicaRundbrief} als Anhang beigef�gt (vgl. � 1,20
    [\refpassage{hypo1}{hypo2}] \griech{>ek t~wn <upotetagm'enwn}).
  \item[�berlieferung]Der Brief an die Kirche Alexandriens wird in
    Dok. \ref{sec:BriefAthAlexParem},8 (\refpassage{v437a}{v437b})
    erw�hnt.\\
    Die beiden Briefe an die Kirche Alexandriens und an die Bisch�fe
    �gyptens und Libyens waren nach Auskunft der Handschriften BKO
    v�llig miteinander identisch (vgl. � 2,2
    [\refpassage{ident}{brief2-2}]).  Demgegen�ber weist die
    Handschrift R ab � 1,14 einige Randbemerkungen auf, die auf
    Unterschiede zwischen den beiden Briefen verweisen
    (\edpageref{semeiou},\lineref{semeiou}; \refpassage{R3}{R3};
    \refpassage{R4-1}{R4-2}; \refpassage{R5-1}{R5-2};
    \refpassage{plhn-1}{plhn-2}. Vgl. dazu auch
    \cite[104--106]{Opitz:Untersuchungen}); dieser Befund wird in
    gewisser Weise auch durch zwei Textvarianten
    (\edpageref{toinun},\lineref{toinun};
    \edpageref{ginwskhte},\lineref{ginwskhte}) der von der Vorlage von
    R abh�ngigen Handschrift E best�tigt, die den Brief an die
    Bisch�fe �gyptens und Libyens sonst �berhaupt nicht erw�hnt.\\
    Demzufolge scheint der Abschnitt 1,18--19
    (\refpassage{semeiou-2}{R3}) nur in dem Brief an die Kirchen
    Alexandriens gestanden zu haben, der Halbsatz in � 1,20
    (\refpassage{R5-1}{R5-2}) \griech{ka`i <h <umet'era -- g'enhtai}
    aber nur im Brief an die Bisch�fe �gyptens und Libyens. In welchem
    der beiden Briefe die beiden in R ebenfalls am Rand vermerkten
    St�cke \refpassage{kai}{dioper} \griech{ka`i -- Di'oper} und
    \refpassage{R4-1}{R4-2} \griech{Jeod'wrou -- Gewrg'iou}
    urspr�nglich standen, ist nicht klar.\\
    Ob es sich bei den Briefen ebenfalls um eine �bersetzung aus dem
    Lateinischen handelt, ist nicht festzustellen. Die von
    \textcite[115, Anm. zu Z. 12]{Opitz1935} angef�hrten Argumente
    k�nnen nicht �berzeugen.
  \item[Fundstelle]Ath., apol.\,sec. 37--41
    (\editioncite[115,12--119,3]{Opitz1935})
  \end{description}
\end{praefatio}
\autor{Brief der ">westlichen"< Synode an die Kirche Alexandriens}
\begin{pairs}
\selectlanguage{polutonikogreek}
\begin{Leftside}
% \beginnumbering
% \pstart
% \Ladd{Epistol`h t~hc >en Sardik~h| sunaqje'ishc sun'odou}
% \pend
\pstart
\hskip -1.5em\edtext{\abb{}}{\killnumber\Cfootnote{\hskip -1em\latintext Ath.(BKO
RE)}}\specialindex{quellen}{section}{Athanasius!apol.\,sec.!37--41}
\kap{1,1}<H <ag'ia s'unodoc <h kat`a jeo~u q'arin >en
Sar\-di\-k~h|\edindex[synoden]{Serdica!a. 343} sunaqje~isa
>ap`o <R'wmhc\edindex[namen]{Rom} ka`i Span'iwn\edindex[namen]{Spanien}
Gall'iwn\edindex[namen]{Gallien} >Ital'iac\edindex[namen]{Italien}
\edtext{\abb{Kampan'iac Kalabr'iac
>Apoul'iac}}{\Dfootnote{\latintext >
E}}\edindex[namen]{Campania}\edindex[namen]{Calabria}\edindex[namen]{Apulia}
>Afrik~hc\edindex[namen]{Africa} Sardan'iac\edindex[namen]{Sardinia}
Pannon'iwn\edindex[namen]{Pannonia} Mus'iwn\edindex[namen]{Moesia}
\edtext{Dak'iac}{\Dfootnote{Dak'iwn \latintext B}}\edindex[namen]{Dacia}
Nwr'ikou\edindex[namen]{Noricum}
\edtext{Sisk'iac}{\Dfootnote{Tousk'iac \latintext
E\corr}}\edindex[namen]{Siscia}
Dardan'iac\edindex[namen]{Dardania} >'allhc
Dak'iac\edindex[namen]{Dacia} Makedon'iac\edindex[namen]{Macedonia}
Jessal'iac\edindex[namen]{Thessalia} >Aqa'iac\edindex[namen]{Achaia}
>Hpe'irwn\edindex[namen]{Epirus} Jr'a|khc\edindex[namen]{Thracia} ka`i
<Rod'ophc\edindex[namen]{Rhodope} ka`i
Palaist'inhc\edindex[namen]{Palaestina} ka`i >Arab'iac\edindex[namen]{Arabia}
ka`i Kr'hthc\edindex[namen]{Creta} ka`i A>ig'uptou\edindex[namen]{Aegyptus}
presbut'eroic ka`i
diak'onoic ka`i p'ash| t~h| <ag'ia| >ekklhs'ia| to~u jeo~u t~h| >en
>Alexandre'ia|\edindex[namen]{Alexandrien} paroiko'ush|, >agaphto~ic
\edtext{\abb{>adelfo~ic}}{\Dfootnote{\latintext > R}}, >en kur'iw| qa'irein.
\pend
\pstart
\kap{2}ka`i pr`in m`en labe~in <hm~ac t`a gr'ammata t~hc e>ulabe'iac <um~wn
o>uk >hgnoo~umen, >all`a ka`i faner`on <hm~in >~hn, <wc <'oti o<i t~hc
duswn'umou t~wn >Areian~wn a<ir'esewc prost'atai poll`a ka`i dein'a, m~allon
d`e
\edtext{\abb{ka`i}}{\Dfootnote{\latintext > KORE}} >ep'' >ol'ejrw| t~hc <eaut~wn
yuq~hc >`h
kat`a t~hc >ekklhs'iac >emhqan~wnto.
\pend
\pstart
\kap{3}a<'uth g`ar >~hn a>ut~wn t'eqnh ka`i panourg'ia, ta'uthc >ae`i t~hc
janathf'orou geg'onasi proj'esewc, <'opwc\edlabel{tk:1} p'antac to`uc
<opoid'hpote
tugq'anontac t~hc >orj~hc d'oxhc ka`i t`hn t~hc kajolik~hc >ekklhs'iac
didaskal'ian kat'eqontac t`hn par`a t~wn pat'erwn a>uto~ic paradoje~isan
spoud'azein >ela'unein ka`i di'wkein. to`uc m`en g`ar plasto~ic >egkl'hmasin
<up'eballon, >'allouc e>ic >exorism`on >ap'estellon, >'allouc >en a>uta~ic ta~ic
timwr'iaic katep'onoun.
\pend
\pstart
\kap{4}>am'elei ka`i to~u >adelfo~u ka`i sunepisk'opou <hm~wn >Ajanas'iou t`hn
kajar'othta b'ia| ka`i turann'idi sullab'esjai >espo'udasan, ka`i di`a to~uto
o>'ute >epimel`hc o>'ute met`a p'istewc o>'uj'' <'olwc dika'ia g'egonen <h par''
>eke'inwn kr'isic. di'oper o>ud`e jarro~untec o<~ic >edramato'urghsan o>ud`e
o<~ic >ejr'ulhsan kat'' a>uto~u, >all`a ka`i jewro~untec, <wc o>u d'unantai
per`i to'utwn
\edtext{>apode'ixeic >'eqein}{\lemma{\abb{}} \Dfootnote{\responsio\ >'eqein
>apode'ixeic \latintext K}} >alhje~ic, par'ontec e>ic t`hn
\edtext{Serd~wn}{\Dfootnote{Sard~wn \latintext KO}}\edindex[synoden]{Serdica!a.
343}
p'olin o>uk >hj'elhsan e>ic
t`hn s'unodon p'antwn t~wn <ag'iwn >episk'opwn >apant~hsai.
\pend
\pstart
\kap{5}>ek d`h to'utou faner`a ka`i dika'ia kaj'esthken <h kr'isic to~u
>adelfo~u ka`i sunepisk'opou <hm~wn >Ioul'iou\edindex[namen]{Julius!Bischof von
Rom}. o>u g`ar >askept`i bebo'uleutai,
>all`a ka`i met'' >epimele'iac <'wrisen, <'wste mhd`e <'olwc
\edtext{dist'asai}{\Dfootnote{dist'axai \latintext RE}} per`i t~hc koinwn'iac
to~u >adelfo~u <hm~wn >Ajanas'iou\edindex[namen]{Athanasius!Bischof von
Alexandrien}. e>~iqe g`ar >episk'opwn\edlabel{tk:2} >ogdo'hkonta
>alhje~ic m'arturac, e>~iqe ka`i to~uto d'ikaion, <'oti di`a t~wn >agapht~wn
>adelf~wn <hm~wn t~wn
\edtext{presbut'erwn}{\Dfootnote{presbut'erwn \dt{(s.l.)} \dt{B\corr} >episk'opwn
\dt{B*}}} <eauto~u ka`i di`a gramm'atwn mej'wdeuse to`uc per`i
E>us'ebion\edindex[namen]{Eusebianer} to`uc \edlabel{tk:3a}o>uk >ep`i kr'isei,
>all'' >ep`i b'ia| m~allon
>epereidom'enouc\edlabel{tk:3b}. <'ojen o<i pantaqo~u p'antec >ep'iskopoi t`hn
koinwn'ian
>Ajanas'iou\edindex[namen]{Athanasius!Bischof von Alexandrien} >ebeba'iwsan di`a
t`hn kajar'othta a>uto~u.
\pend
\pstart
\kap{6}k>ake~ino d`e
\edtext{<um~wn <h}{\lemma{\abb{}} \Dfootnote{\responsio\ <h <um~wn BKO}} >ag'aph
\edtext{sunor'atw}{\Dfootnote{sunor~atai \latintext E}}. >epeid`h e>ic t`hn
<ag'ian s'unodon t`hn >en Sardik~h|\edindex[synoden]{Serdica!a. 343}
sunaqje~isan pareg'eneto, t'ote d'h \ladd{\edtext{\abb{kaj`a
proe'ipomen}}{\Dfootnote{\dt{del. Erl. (vide apol.sec. 36)}}}} ka`i di`a
gramm'atwn ka`i di`a >agr'afwn
\edtext{>entol~wn}{\Dfootnote{>entol`hn \latintext RE}} <upemn'hsjhsan o<i t~hc
<E'w|ac ka`i >ekl'hjhsan
\edtext{\abb{par'' <hm~wn <'wste}}{\Dfootnote{\latintext > BKO}} pare~inai.
>all'' >eke~inoi <up`o t~hc suneid'hsewc kataginwsk'omenoi >aprep'esi qr'wmenoi
prof'asesi fugodike~in >'hrxanto. >hx'ioun g`ar t`on >aj~won <wc <upe'ujunon
>ap`o t~hc <hmet'erac koinwn'iac >ekb'allesjai o>u sunor~wntec <wc >aprep'ec,
m~allon d`e >ad'unaton >~hn to~uto.
\pend
\pstart
\kap{7}ka`i t`a <upomn'hmata d`e t`a >en t~w| Mare'wth|\edindex[namen]{Mareotis}
gen'omena <up`o
pampon'hrwn ka`i >exwlest'atwn tin~wn newt'erwn, o<~ic o>uk >'an tic
>ep'isteusen o>ud`e t`on tuq'onta bajm`on to~u kl'hrou, sun'esthke kat`a
monom'ereian pepr'aqjai. o>'ute g`ar <o >adelf`oc <hm~wn
>Ajan'asioc\edindex[namen]{Athanasius!Bischof von Alexandrien} <o
>ep'iskopoc o>'ute Mak'arioc\edindex[namen]{Macarius!Presbyter in Alexandrien}
<o presb'uteroc <o kathgoro'umenoc <up'' a>ut~wn
par~hn. ka`i
\edtext{<'omwc}{\Dfootnote{o<'utwc \latintext K}} <h par'' a>ut~wn >er'wthsic,
m~allon d`e <upobol`h <h genom'enh p'ashc a>isq'unhc >~hn mest'h. p~h m`en g`ar
>ejniko'i, p~h d`e kathqo'umenoi >hrwt~wnto, o>uq <'ina <'aper
\edtext{\abb{>'isasin}}{\Dfootnote{\latintext > B*}} e>'ipwsin, >all''
<'ina <'aper par'' a>ut~wn memaj'hkasi ye'uswntai.
\pend
\pstart

\pend
\pstart
\kap{8}ka`i g`ar ka`i
\edtext{<um~wn}{\Dfootnote{<hm~wn \latintext B}} t~wn presbut'erwn frontiz'ontwn
>ep`i t~h| >apous'ia| to~u <umet'erou >episk'opou ka`i boulom'enwn pare~inai
>ep`i t~h| >exet'asei ka`i t`hn >al'hjeian de~ixai ka`i t`a yeud~h diel'egxai
o>ude`ic l'ogoc g'egonen; o>uk >ep'etreyan g`ar <um~ac pare~inai, >all`a
\edtext{\abb{ka`i}}{\Dfootnote{\latintext > O*}} mej''
\edtext{<'ubrewc}{\Dfootnote{<'ubrewn \latintext K}} >ex'ebalon.
\pend
\pstart
\kap{9}ka`i e>i ka`i t`a m'alista faner`a p~asi kaj'esthke ka`i >ek to'utwn <h
sukofant'ia, <'omwc >anaginwskom'enwn
\edtext{\abb{t~wn}}{\Dfootnote{\latintext dupl. R*}} <upomnhm'atwn
e<'uromen a>ut`on t`on pamp'onhron >Isq'uran\edindex[namen]{Ischyras!Presbyter
in der Mareotis} t`on >ep`i t~h| sukofant'ia|
misj`on par'' a>ut~wn lab'onta t`o doko~un >'onoma t~hc >episkop~hc
diel'egqonta
\edtext{<eauto~u}{\Dfootnote{a>uto~u \latintext B* \greektext <eaut`on
\latintext R*}} t`hn sukofant'ian. a>ut`oc g`ar <o
>Isq'urac\edindex[namen]{Ischyras!Presbyter in der Mareotis} >en a>uto~ic
to~ic <upomn'hmasi diel'alhse kat'' >eke'inhn t`hn <'wran, >en <~h|
Mak'arion\edindex[namen]{Macarius!Presbyter in Alexandrien}
>elhluj'enai e>ic t`o kell'ion <eauto~u diebebaio~uto, noso~unta t'ote
katake~isjai <eaut'on, ka'itoi
\edtext{t~wn}{\Dfootnote{t`on \latintext R}} per`i
E>us'ebion\edindex[namen]{Eusebianer} gr'ayai
tolmhs'antwn <est'anai t'ote t`on >Isq'uran\edindex[namen]{Ischyras!Presbyter in
der Mareotis} ka`i prosf'erein, <'ote
Mak'arioc\edindex[namen]{Macarius!Presbyter in Alexandrien}
>ep'esth.
\pend
\pstart
\kap{10}K>ake'inh d`e p~asi faner`a kaj'esthken <h sukofant'ia ka`i diabol'h,
<`hn >h|ti'asanto met`a ta~uta.
\edtext{>'efasan}{\Dfootnote{>'ejfasan \latintext E}} g`ar ka`i kateb'ohsan
f'onon dedrak'enai t`on >Ajan'asion\edindex[namen]{Athanasius!Bischof von
Alexandrien} ka`i
[\edtext{\abb{<wc}}{\Dfootnote{\latintext del. Opitz}}]
>Ars'eni'on\edindex[namen]{Arsenius!Bischof von Hypsele} tina
\edtext{Melitian`on}{\Dfootnote{Meliti\d{a}\d{n}`on \dt{B\corr} Melitinian`on
\dt{B*?}}} >ep'iskopon
>anh|rhk'enai, >ef'' <~w|
\edtext{prospoiht~w|}{\Dfootnote{prospoie~i t~w \latintext R* \greektext
prospoie'itw \latintext R\corr}} stenagm~w| ka`i peplasm'enoic d'akrusin
<upekr'inonto ka`i >hx'ioun to~u z~wntoc <wc tejnhk'otoc t`o s~wma
\edtext{\abb{>apodoj~hnai}}{\Dfootnote{\dt{E\mg} >apodo~unai \dt{E}}}. >all''
o>uk >'agnwsta g'egone t`a sof'ismata
to'utwn. >'egnwsan g`ar
\edtext{<'apantec}{\Dfootnote{p'antec \latintext RE}} z~hn t`on >'anjrwpon ka`i
>en to~ic z~wsin >exet'azesjai.
\pend
\pstart
\kap{11}ka`i\looseness=1\ >epeid`h <e'wrwn <eauto`uc o<i pr`oc p'anta e>uqere~ic
dielegqom'enouc >ep`i to~ic ye'usmasi --
\edtext{to'utoic}{\Dfootnote{to'uto'utoic \latintext B}} a>ut`oc g`ar z~wn <o
>Ars'enioc\edindex[namen]{Arsenius!Bischof von Hypsele} >ede'iknuen <eaut`on m`h
>anh|r~hsjai
\edtext{mhd`e}{\Dfootnote{m`h \latintext B}} tejnhk'enai --, <'omwc o>uq
<hs'uqasan, >all'' <et'erac sukofant'iac pr`oc ta~ic prot'eraic sukofant'iaic
>epez'htoun, <'ina p'alin mhqanhs'amenoi diab'alwsi t`on >'anjrwpon.
\pend
\pstart
\kap{12}t'i o>~un, >agaphto'i? o>uk >etar'aqjh <o >adelf`oc <hm~wn
>Ajan'asioc\edindex[namen]{Athanasius!Bischof von Alexandrien},
>all`a p'alin poll~h| parrhs'ia| qr'wmenoc proekale~ito ka`i >ep`i to'utoic
a>uto'uc. ka`i <hme~ic d`e h>uq'omeja ka`i proetrep'omeja >elje~in a>uto`uc e>ic
t`hn kr'isin ka'i, e>'iper d'unantai, diel'egxai. >`w t~hc poll~hc pleonex'iac.
>`w t~hc dein~hc <uperhfan'iac. m~allon d'e, e>i de~i
\edtext{t>alhj`ec}{\Dfootnote{t`o >alhj`ec \latintext ORE}} e>ipe~in, >`w kak~hc
ka`i <upeuj'unou suneid'hsewc. to~uto g`ar p~asi pefan'erwtai.
\pend
\pstart
\kap{13}<'ojen, >agaphto`i >adelfo'i, <upomimn'hskomen ka`i protrep'omeja <um~ac
pr`o p'antwn t`hn >orj`hn p'istin t~hc kajolik~hc >ekklhs'iac kat'eqein. poll`a
m`en g`ar ka`i dein`a ka`i qalep`a pep'onjate, poll`ac d`e <'ubreic ka`i
>adik'iac <up'emeinen <h kajolik`h >ekklhs'ia, >all''
\edtext{((<o
<upome'inac e>ic t'eloc o<~utoc swj'hsetai))}{\lemma{\abb{}}
\Afootnote{{\latintext Mt 10,22}}}\edindex[bibel]{Matthaeus!10,22}. di'oper k>`an >'eti
poie~in tolm'hswsi kaj''
\edtext{<um~wn}{\Dfootnote{<hm~wn \latintext B*}},
\edtext{\abb{<h}}{\Dfootnote{\latintext > K}} jl~iyic >ant`i qar~ac <um~in >'estw; t`a
g`ar toia~uta
paj'hmata m'eroc >est`i martur'iou ka`i a<i toia~utai <um~wn <omolog'iai ka`i
a<i b'asanoi o>uk >'amisjoi tugq'anousin, >all'' >apol'hyesje par`a to~u jeo~u
t`a >'epajla.
\pend
\pstart
\kap{14}di'oti m'alista >agwn'izesje <up`er t~hc <ugiaino'ushc p'istewc ka`i
\edtext{t~hc
\edtext{kajar'othtoc}{\Dfootnote{kajar'othc \latintext B*}} to~u >episk'opou
\edtext{<um~wn}{\Dfootnote{<hm~wn \latintext R*}} >Ajanas'iou to~u
sulleitourgo~u}{\lemma{t~hc kajar'othtoc
\dots\ to~u sulleitourgo~u}\Dfootnote{to~ic sulleitourgo~ic \latintext
E}}\edindex[namen]{Athanasius!Bischof von Alexandrien}
\edtext{<hm~wn}{\Dfootnote{<um~wn \latintext B*}}. ka`i g`ar o>ud`e <hme~ic
paresiwp'hsamen o>ud`e >hmel'hsamen t~hc
<um~wn >amerimn'iac q'arin, >all'' >efront'isamen ka`i pepoi'hkamen, <'aper <o
t~hc >ag'aphc l'ogoc >apaite~i. sump'asqomen g`ar to~ic p'asqousin >adelfo~ic
<hm~wn, ka`i t`a >eke'inwn paj'hmata >'idia <hgo'umeja;
ka`i\edlabel{kai} to~ic d'akrusin
\edtext{<um~wn}{\Dfootnote{<hm~wn \latintext E*}} t`a
\edtext{<hm'etera}{\Dfootnote{<um'etera \latintext B}} d'akrua sunem'ixamen,
o>uq <ume~ic d`e m'onoi pep'onjate, >adelfo'i, >all`a ka`i pollo`i >'alloi
sulleitourgo`i <hm~wn ta~uta >elj'ontec
\edtext{>apwd'uranto}{\Dfootnote{>apwd'uronto \latintext B}}.
\pend
\pstart

\pend
\pstart

\pend
\pstart
\kap{15}\edlabel{dioper}\edtext{Di'oper}{{\xxref{kai}{dioper}}\lemma{\abb{ka`i
\dots\  Di'oper}} \Dfootnote{\dt{R\mg\ del. Opitz}}}
\edtext{\abb{>anhn'egkamen}}{\Dfootnote{+ to'inun \dt{RE}}}\edlabel{toinun} ka`i >hxi'wsamen to`uc
e>usebest'atouc ka`i
\edtext{\abb{jeofilest'atouc}}{\Dfootnote{B}} basil'eac, <'opwc <h filanjrwp'ia
a>ut~wn ka`i to`uc >'eti k'amnontac ka`i piezom'enouc >anej~hnai  kele'ush|,
ka`i prost'axwsi mhd'ena t~wn dikast~wn, o<~ic per`i m'onwn t~wn dhmos'iwn
\edtext{m'elein}{\Dfootnote{m'ellein \latintext E}} pros'hkei, m'hte kr'inein
klhriko`uc m'hte <'olwc to~u loipo~u prof'asei t~wn >ekklhsi~wn >epiqeire~in ti
kat`a t~wn >adelf~wn, >all'' <'ina <'ekastoc qwr'ic tinoc diwgmo~u, qwr'ic tinoc
b'iac ka`i pleonex'iac, <wc e>'uqetai ka`i bo'uletai, z~h| ka`i mej'' <hsuq'iac
ka`i e>ir'hnhc t`hn kajolik`hn ka`i >apostolik`hn p'istin met'erqhtai.
\pend
\pstart
\kap{16}Grhg'orioc\edindex[namen]{Gregor!Bischof von Alexandrien} m'entoi <o
paran'omwc
\edtext{par`a}{\Dfootnote{<up`o \latintext KORE}} t~wn a<iretik~wn leg'omenoc
katastaj~hnai ka`i e>ic t`hn <umet'eran p'olin par'' a>ut~wn >apostale'ic, ka`i
to~uto g`ar ginwsk'etw <um~wn <h <omoyuq'ia, <'oti kr'isei t~hc <ier~ac p'ashc
sun'odou kajh|r'ejh,
\edtext{\abb{e>i}}{\Dfootnote{+ d`e \latintext O*}} ka`i t`a m'alista
o>udep'wpote o>ud`e <wc >ep'iskopoc <'olwc gen'omenoc >enom'isjh.
\pend
\pstart
\kap{17}qa'irete to'inun >apolamb'anontec
\edtext{<eaut~wn t`on >ep'iskopon}{\lemma{\abb{}} \Dfootnote{\responsio\ t`on
>ep'iskopon <eaut~wn \latintext K}}
>Ajan'asion\edindex[namen]{Athanasius!Bischof von Alexandrien}; di`a to~uto g`ar
ka`i met'' e>ir'hnhc a>ut`on >apel'usamen. <'ojen ka`i
paraino~umen p~asi to~ic >`h di`a f'obon >`h di`a peridrom'hn tinwn koinwn'hsasi
Grhgor'iw|\edindex[namen]{Gregor!Bischof von Alexandrien}, <'ina n~un
<upomnhsj'entec ka`i protrap'entec ka`i >anapeisj'entec
par'' <hm~wn
\edtext{pa'uswntai}{\Dfootnote{pa'usontai \latintext R}} t~hc pr`oc >eke~inon
musar~ac koinwn'iac ka`i loip`on
\edtext{\abb{<eauto`uc}}{\Dfootnote{\latintext > KO}} sun'aywsi t~h| kajolik~h|
\edtext{>ekklhs'ia|}{\Dfootnote{>ap`o to'utou to~u shme'iou m'eqri to'utou \dt{R\mg}}}\edlabel{semeiou}.
\pend
\pstart
\kap{18}\edlabel{semeiou-2}\edtext{>Epeid`h}{\Dfootnote{>epe`i \latintext
B}} d`e >'egnwmen <'oti ka`i
>Afj'onioc\edindex[namen]{Aphthonius!Presbyter in Alexandrien} ka`i
>Ajan'asioc\edindex[namen]{Athanasius!Presbyter in Alexandrien} <o Kap'itwnoc
ka`i
Pa~uloc\edindex[namen]{Paulus!Presbyter in Alexandrien} ka`i
Plout'iwn\edindex[namen]{Plution!Presbyter in Alexandrien} o<i sumpresb'uteroi
<hm~wn suskeu`hn ka`i a>uto`i
pep'onjasin <up`o t~wn per`i E>us'ebion\edindex[namen]{Eusebianer}, <'wste to`uc
m`en >exorismo~u
peiraj~hnai, to`uc d`e ka`i jan'atwn >apeil`ac diapefeug'enai, to'utou <'eneken
ka`i per`i to'utou dhl~wsai <um~in >anagka~ion <hghs'ameja, <'ina gin'wskhte
<'oti ka`i to'utouc >apedex'ameja ka`i >aj'wouc >apel'usamen e>id'otec <'oti
p'anta t`a
\edtext{par`a}{\Dfootnote{per`i \latintext R* \greektext per`a \latintext
R\corr}} t~wn per`i E>us'ebion\edindex[namen]{Eusebianer} kat`a t~wn >orjod'oxwn
gen'omena >ep`i
\edtext{d'oxh| ka`i sust'asei}{\lemma{\abb{}} \Dfootnote{\responsio\ sust'asei
ka`i d'oxh| \latintext K}} t~wn suskeuasj'entwn <up'' a>ut~wn g'egonen.
\pend
\pstart
\kap{19}>'eprepe
\edtext{\abb{m`en}}{\Dfootnote{\latintext > B}} o>~un t`on <um'eteron
>ep'iskopon, t`on
sulleitourg`on <hm~wn >Ajan'asion\edindex[namen]{Athanasius!Bischof von
Alexandrien}, per`i a>ut~wn <wc per`i
\edtext{\abb{>id'iwn}}{\Dfootnote{+ >id'ioic \latintext RE}}
\edtext{\abb{<um~in}}{\Dfootnote{\latintext dupl. BO}} dhl~wsai,
\edlabel{unschuld1}>epeid`h d`e
<up`er ple'ionoc martur'iac ka`i t`hn <ag'ian s'unodon >hj'elhsen <um~in
gr'ayai\edlabel{unschuld2}, di`a to~uto o>uk
\edtext{>aneball'omeja}{\Dfootnote{>anebal'omeja \latintext RE}}, >all`a ka`i
shm~anai <um~in >espoud'asamen, <'in'' <'wsper <hme~ic o<'utwc ka`i <ume~ic
a>uto`uc >apod'exhsje. >'axioi g`ar ka`i a>uto`i >epa'inou, <'oti di`a t`hn e>ic
Qrist`on e>us'ebeian ka`i a>uto`i >hxi'wjhsan par`a t~wn a<iretik~wn <'ubrin
\edtext{<upome~inai}{\Dfootnote{>'eqei kat`a
prosj'hkhn a<'uth <h >epistol`h t~hc met'' a>ut~hc >hgo~un t~hc pr`oc p'antac
>episk'opouc \dt{R\mg}}}\edlabel{R3}.
\pend
\pstart
\kap{20}\edtext{t'ina d'e >esti t`a par`a t~hc <ag'iac sun'odou dogmatisj'enta kat`a
\edlabel{R4-1}\edtext{Jeod'wrou ka`i
\edtext{Nark'issou}{\Dfootnote{Narkiso~u \latintext E*}} ka`i
Stef'anou ka`i
>Akak'iou ka`i
\edtext{Mhnof'antou}{\Dfootnote{Monof'antou \latintext B }} ka`i
O>ursak'iou
ka`i O>u'alentoc ka`i
Gewrg'iou}{\lemma{\abb{Jeod'wrou \dots\ Gewrg'iou}}
\Dfootnote{\latintext R\mg}}\edlabel{R4-2}, t~wn proistam'enwn t~hc
>areian~hc
a<ir'esewc ka`i plhmmelhs'antwn kaj'' <um~wn ka`i kat`a t~wn >'allwn
>ekklhsi~wn, gn'wsesje \edlabel{hypo1}>ek t~wn <upotetagm'enwn\edlabel{hypo2}.
>apeste'ilamen g`ar <um~in,
<'ina
\edlabel{R5-1}\edtext{ka`i <h <umet'era jeos'ebeia s'umyhfoc to~ic par'' \edtext{<hm~wn}{\Dfootnote{<hm~in \latintext E}} <orisje~isi
g'enhtai}{\lemma{\abb{ka`i <h <umet'era \dots\ g'enhtai}}\Dfootnote{\dt{> R} e>ic d`e t`hn
>episk'opwn  >epistol`hn pr'oskeitai ka`i to~uto; <'ina ka`i <h <umet'era
jeos'ebeia s'umyhfoc to~ic par'' <hm~wn <orisje'ish \dt{(sic!)} g'enhtai ka`i gin'wskhte \dt{R\mg}}}\edlabel{R5-2} ka`i
\edtext{>ek to'utwn gn~wte}{\Dfootnote{gin'wskhte \latintext E}}\edlabel{ginwskhte} <'oti <h
kajolik`h >ekklhs'ia o>u paror~a| to`uc e>ic a>ut`hn
\edtext{\abb{plhmmelo~untac}}{\Dfootnote{+ <h a>ut`h o>~un >epistol`h
>aparall'aktwc
kat`a p'anta >egr'afh pr`oc to`uc kat'' A>'igupton ka`i Lib'uhn >episk'opouc
\latintext BK}}.}{\lemma{\abb{t'ina \dots\ plhmmelo~untac}}\Dfootnote{\dt{del. Opitz}}}\edindex[namen]{Menophantus!Bischof von Ephesus}\edindex[namen]{Stephanus!Bischof von
Antiochien}\edindex[namen]{Narcissus!Bischof von
Neronias}\edindex[namen]{Acacius!Bischof von
Caesarea}\edindex[namen]{Theodorus!Bischof von
Heraclea}\edindex[namen]{Georg!Bischof von
Laodicea}\edindex[namen]{Ursacius!Bischof von
Singidunum}\edindex[namen]{Valens!Bischof von Mursa}
\pend
% \endnumbering
\end{Leftside}
\begin{Rightside}
\begin{translatio}
\beginnumbering
% \pstart
% \Ladd{Brief der in Serdica versammelten Synode}
% \pend
\pstart
\noindent\kapR{37,1}Die heilige Synode, die sich durch die Gnade Gottes aus Rom, den spanischen und
gallischen Provinzen, aus Italia, Campania, Calabria, Apulia, Africa, Sardinia, den
pannonischen und m�sischen Provinzen, Dacia, Noricum, Siscia, Dardania, der anderen
dakischen Provinz, Macedonia, Thessalia, Achaia, Epirus, Thracia, Rhodope, Palaestina,
Arabia, Creta, Aegyptus\footnoteA{Zu der Problematik der genannten Provinzen vgl.
\ref{sec:SerdicaRundbrief},1; sie werden nicht genannt in dem
Schreiben unten � 2,1.} in Serdica versammelt hat, gr��t im Herrn die Presbyter und
Diakone und die ganze heilige Kirche Gottes, die in Alexandrien ans�ssig ist, die
geliebten Br�der.
\pend
\pstart
\kapR{2}Schon bevor wir die Briefe eurer Fr�mmigkeit\footnoteA{Es wurden
offensichtlich aus �gypten Briefe (wie der in Ath., apol.\,sec. 3--19
erhaltene) zur Synode geschickt.} erhielten, waren wir nicht ahnungslos,
sondern es war uns sogar v�llig klar, da� die Anf�hrer der verha�ten arianischen H�resie
arglistig viele schreckliche Dinge anstellten, allerdings mehr zum Verderben ihrer eigenen
Seele als zum Schaden der Kirche.
\pend
\pstart
\kapR{3}Darin n�mlich bestand ihre Kunst und List, stets war es ihre todbringende
Absicht, da� sie daf�r sorgen, alle, die irgendwo rechten Glaubens waren und
sich an die Lehre der katholischen Kirche hielten, die ihnen von den V�tern
�berliefert worden war, zu vertreiben und zu verfolgen. Denn die einen
unterdr�ckten sie durch erfundene Vorw�rfe, andere schickten sie in die
Verbannung, wieder andere zerm�rbten sie eben mit den Bestrafungen.
\pend
\pstart
\kapR{4}Ohne Zweifel strebten sie auch danach, die Unschuld unseres Bruders und
Mitbischofs Athanasius mit Gewalt und Tyrannei zu attackieren, und deswegen war
ihr Urteil weder sorgf�ltig noch vertrauensw�rdig noch �berhaupt gerecht. 
Weil sie daher weder darauf vertrauten, was sie inszenierten, 
noch auf die Ger�chte, die sie gegen ihn in die Welt setzten,
sondern vielmehr sahen, da� sie hierf�r keine wirklichen
Beweise vorlegen konnten, da wollten sie, obwohl sie nach Serdica gekommen waren, nicht an der
Synode aller heiligen Bisch�fe teilnehmen.
\pend
\pstart
\kapR{5}Vor diesem Hintergrund erweist sich das Urteil unseres Bruders und Mitbischofs
Julius als klar und gerecht.\footnoteA{Die ">westliche"< Synode �bernimmt die Entscheidung des r�mischen
Bischofs Julius auf der r�mischen Synode von 341 (vgl. Dok. \ref{sec:BriefJulius}), die im Osten
exkommunizierten Bisch�fe in die Kirchengemeinschaft aufzunehmen.} 
Denn nicht ungepr�ft hat er einen Beschlu� gefa�t, sondern mit Umsicht
festgesetzt, keinesfalls an der Gemeinschaft mit unserem Bruder Athanasius zu zweifeln.
Er hatte n�mlich achtzig Bisch�fe als zuverl�ssige Zeugen\footnoteA{Vgl. Dok.
\ref{sec:B},2.}; Er hatte auch dies als Rechtfertigung, da� er mit Hilfe unserer geliebten
Mitbr�der, seiner Presbyter, und mit Briefen die um
Eusebius als Leute �berf�hrte, die sich nicht auf eine gerichtliche Untersuchung, sondern
vielmehr auf Gewalt st�tzten. Daher best�tigten alle Bisch�fe �berall die Gemeinschaft
mit Athanasius aufgrund seiner Unschuld. 
\pend
\pstart
\kapR{6}Auch folgendem sollte euere Liebe Beachtung schenken: Nachdem er zur heiligen in Serdica versammelten
Synode gekommen war, da wurden die aus dem Osten durch Briefe und m�ndliche Befehle
von uns aufgefordert und eingeladen zu kommen. Aber jene, von ihrem schlechten Gewissen
verurteilt, begannen, mit unangemessenen Vorw�nden ihre Flucht zu rechtfertigen. Sie
forderten n�mlich, da� ein Unschuldiger wie ein Schuldiger von unserer Gemeinschaft
ausgeschlossen werde, ohne einzusehen, wie ungeziemend, ja vielmehr unm�glich dies w�re.
\pend
\pstart
\kapR{7}Und es hat sich gezeigt, da� die Aufzeichnungen parteiisch angelegt waren, die in
der Mareotis von einigen ganz �blen und v�llig verdorbenen Gr�nschn�beln\footnoteA{Gemeint
sind Ursacius von Singidunum und Valens von Mursa.}, denen niemand auch nur irgendeinen Rang in
dem Klerus anvertraut h�tte, gemacht worden waren.\footnoteA{Zu der Mareotiskommission
und den folgenden Ereignissen vgl. Dok. \ref{sec:BriefJuliusII},32.} Denn weder unser Bruder,
der Bischof Athanasius, noch Macarius, der von ihnen angeklagte Presbyter, waren anwesend.
Und dennoch war ihr Verh�r, vielmehr die Unterstellungen, die man ihnen machte, voll von jeder Schande. Einerseits wurden n�mlich Heiden, andererseits  auch Katechumenen
befragt, nicht damit sie sagten, was sie wu�ten, sondern damit sie daherlogen, was ihnen von
jenen beigebracht worden war.
\pend
\pstart

\pend
\pstart
\kapR{8}Unter anderem wurdet ihr Presbyter, die ihr euch w�hrend der
Abwesenheit eueres Bischofs Sorgen machtet, mit keiner Silbe erw�hnt, obwohl ihr bei
der Untersuchung anwesend sein, die Wahrheit aufzeigen und die L�gen widerlegen
wolltet. Denn sie erlaubten euch nicht dabei zu sein, sondern vertrieben euch mit Beschimpfungen.
\pend
\pstart
\kapR{9}Auch wenn am allermeisten schon aus diesem Verhalten die Verleumdung allen klar vor Augen
stand, fanden wir dennoch, als die Berichte vorgelesen wurden, da� der absolut verdorbene
Ischyras, der von ihnen als Lohn f�r die Verleumdung den Schein-Titel eines Bischofs erhalten hatte, sich selbst
der Verleumdung �berf�hrte. Ischyras selbst n�mlich gab in den besagten
Aufzeichnungen preis, da� er in jener Zeit damals, als Macarius nach eigener Aussage zu seiner
Zelle kam, krank darniederlag, wohingegen die um Eusebius zu schreiben wagten, da�
Ischyras damals gestanden und die Eucharistie gefeiert habe, als Macarius
dazukam.\footnoteA{Vgl. Dok. \ref{sec:BriefJuliusII},33--35.} 
\pend 
\pstart 
\kapR{10}Auch jene
Intrige und Verleumdung wurde allen offenbar, die sie ihm danach zum Vorwurf
machten. Sie warfen n�mlich ein und erhoben ihre Stimme, da� Athanasius einen Mord begangen habe und
einen gewissen Arsenius\footnoteA{Vgl. Dok. \ref{sec:BriefJuliusII},29.}, einen
melitianischen Bischof, beseitigt habe, worauf sie mit gestellten Wehklagen und fingierten
Tr�nen theatralisch reagierten und forderten, man solle ihnen den K�rper des Lebenden wie
den eines Toten �bergeben. Aber ihre listenreichen Pl�ne blieben nicht unerkannt. Es
wu�ten n�mlich alle, da� der Mensch wohlauf war und zu den Lebenden geh�rte.
\pend
\pstart
\kapR{11}Und als sie, die zu allem bereit waren, sahen, da� sie bei diesen L�gen
ertappt worden waren~-- der lebende Arsenios selbst zeigte ihnen n�mlich, da� er weder beseitigt
noch get�tet worden war~--, gaben sie dennoch keine Ruhe, sondern spannen weitere Intrigen
zu den fr�heren, um den Mann mit neuen Machenschaften wiederum zu
verleumden.\footnoteA{Der Vorwurf der nicht-kanonischen Wahl zum Bischof (apol.\,sec.
6,4), der Gewaltanwendung bei seinem Einzug 338 (apol.\,sec. 6,4) und der Veruntreuung von
Getreide (apol.\,sec. 18,2).}
\pend
\pstart
\kapR{12}Was nun, meine Lieben? Unser Bruder Athanasius lie� sich nicht
ersch�ttern, sondern forderte sie mit viel Freimut wiederum auch bei diesen
Verleumdungen heraus. Auch wir baten und ermahnten sie, zur Untersuchung zu
kommen und Beweise vorzubringen, falls sie es k�nnten. Was f�r eine unb�ndige
Gier! Was f�r eine erschreckende Anma�ung! Vielmehr noch, wenn man die Wahrheit 
sagen mu�, was f�r ein schlechtes und schuldiges Gewissen! Dies
ist allen n�mlich deutlich geworden.
\pend
\pstart
\kapR{13}Daher, geliebte Br�der, erinnern und ermahnen wir euch, vor allen
Dingen am rechten Glauben der katholischen Kirche festzuhalten. Denn viel
Schreckliches und Schwieriges habt ihr erlitten, die katholische Kirche hat
viele frevelhafte Gewalttaten und Ungerechtigkeiten ertragen, aber ">wer aush�lt
bis ans Ende, wird gerettet werden."< Daher, auch wenn sie es immer noch wagen,
etwas gegen uns zu unternehmen, nehmt die Bedr�ngnis als Freude an. Solche Leiden sind
n�mlich Teil des Martyriums, und diese eure Bekenntnisse und Pr�fungen
bleiben nicht unbelohnt, sondern ihr werdet von Gott die Siegespreise
erhalten.
\pend
\pstart
\kapR{14}K�mpft deswegen vor allem f�r den gesunden Glauben und die Unschuld eures
Bischofs Athanasius, unseres Mitdieners. Denn auch wir schwiegen nicht, noch waren
wir ohne Sorge um eure Unbedr�cktheit, sondern wir �berlegten und taten, was immer das
Gesetz der Liebe fordert. Wir leiden n�mlich mit unseren leidenden Br�dern mit, und die
Leiden jener sehen wir als unsere eigenen an: Und wir mischten unsere Tr�nen mit eueren;
ihr habt nicht allein gelitten, Br�der, sondern viele andere unserer Mitdiener
beklagten ebenfalls diese Dinge, als sie kamen.
\pend
\pstart

\pend
\pstart

\pend
\pstart
\kapR{15}Deswegen haben wir den gottesf�rchtigsten und gottgeliebtesten Kaisern
Bericht erstattet\footnoteA{Vgl. Dok. \ref{sec:BriefSerdikaConstantius}.} und sie gebeten, da�
ihre G�te befehle, auch die, die immer noch gequ�lt und bedr�ngt sind, zu erl�sen. Und
sie m�gen anordnen, da� kein Richter, denen die Sorge um die rein weltlichen Angelegenheiten
obliegt, �ber Kleriker ein Urteil f�lle oder �berhaupt k�nftig unter dem 
Deckmantel der Kirche etwas gegen die Br�der unternehme, sondern da� jeder frei von irgendeiner
Verfolgung, frei von jeglicher Gewalt und von Betrug so lebe, wie er es erbittet und m�chte, und in
Ruhe und Frieden dem katholischen und apostolischen Glauben folge.
\pend
\pstart
\kapR{16}Gregor freilich, der bekanntlich von den H�retikern gesetzwidrig eingesetzt 
und von ihnen in eure Stadt geschickt worden war\footnoteA{Gregor, auf einer
antiochenischen Synode 338 als Bischof f�r den dort abgesetzten Athanasius vorgesehen
(vgl. Dok. \ref{sec:BriefJuliusII},41--45; Dok. \ref{sec:SerdicaRundbrief},15), wie zuvor
schon Pistus (vgl. Dok. \ref{sec:BriefJulius} Einleitung), wurde mit milit�rischen Mitteln
in Alexandrien eingef�hrt. Gregor blieb bis zu seinem Tod am 26.6.345 Bischof in
Alexandrien (vgl. Ath., h.\,Ar. 21,2), erst dann konnte Athanasius dorthin zur�ckkehren.}~--
auch dies soll n�mlich euere Eintracht erkennen, da� er durch das Urteil der ganzen heiligen
Synode abgesetzt worden ist, wenn man auch mehrheitlich der Auf"|fassung war, da� er
�berhaupt niemals auch nur ann�hernd Bischof gewesen ist.
\pend
\pstart
\kapR{17}Freut euch also, euren eigenen Bischof Athanasius wieder aufzunehmen; deswegen
haben wir ihn n�mlich in Frieden entlassen. Daher ermahnen wir alle,
die aus Furcht oder aufgrund der Umgarnung von gewissen Leuten Gemeinschaft mit
Gregor gehalten haben, da� sie nun von uns erinnert, ermahnt und �berzeugt die
abscheuliche Gemeinschaft mit ihm aufk�ndigen und sich k�nftig mit der katholischen
Kirche verbinden.
\pend
\pstart
\kapR{18}Da wir aber erfuhren, da� auch unsere Mitpresbyter Aphthonius, Athanasius, der
Sohn des Capito, Paulus und Plution\footnoteA{Vgl. apol.\,sec. 17,6. Aphtonius, Athanasius
und Plution sind Mitunterzeichner des Briefes der alexandrinischen Presbyter und Diakone
an die Synode von Tyrus (apol.\,sec. 73). Von Paulus ist sonst nichts bekannt. Athanasius,
der Sohn des Capito, war au�erdem an dem Versuch beteiligt, Macarius zu befreien, vgl.
P.Lond. 1914 (\cite[59~f.]{Bell:Jews})} selbst Nachstellung von denen um Eusebius
erduldet hatten, so da� die einen in Verbannung litten, die anderen sogar vor
Todesdrohungen flohen, deswegen hielten wir es f�r notwendig, euch auch dar�ber zu
informieren, damit ihr erkennt, da� wir auch diese aufnahmen und als unschuldig
freisprachen. Denn wir wissen, da� alles, was von Seiten der Anh�nger des Eusebius gegen die
Rechtgl�ubigen geschehen ist, zum Ruhm und zur Empfehlung derer gereicht, gegen die sie intrigiert
hatten.
\pend
\pstart
\kapR{19}Es w�re zwar Aufgabe eures Bischofs, unseres Mitdieners Athanasius, euch �ber
jene wie �ber seine eigenen Belange zu informieren, da er aber wegen der gr��eren
Glaubw�rdigkeit wollte, da� auch die heilige Synode euch schreibt, haben wir dies deshalb
nicht abgelehnt, sondern bem�hten uns vielmehr, euch dies kundzutun, damit auch ihr sie so wie wir aufnehmt. Auch sie verdienen n�mlich Lob, da sie ebenfalls aufgrund ihrer Fr�mmigkeit
gegen�ber Christus f�r wert erachtet wurden, Mi�handlungen von den H�retikern zu
erleiden.
\pend
\pstart
\kapR{20}Was aber von der heiligen Synode gegen Theodorus,
Narcissus, Stephanus, Acacius, Menophantus, Ursacius, Valens und Georgius, die Vorsteher
der arianischen H�resie, die sich an euch und den anderen Kirchen vergingen,
beschlossen wurde,\footnoteA{Vgl. Dok. \ref{sec:SerdicaRundbrief},14; 16.} das werdet ihr aus den unten
angef�gten Kopien\footnoteA{D.\,h. dem Synodalschreiben Dok. \ref{sec:SerdicaRundbrief}.}
erkennen. Wir haben dies n�mlich an euch weitergeleitet, damit auch euere Gottesfurcht die von uns
beschlossenen Dinge mittr�gt und ihr daraus erkennt, da� die katholische Kirche nicht
�ber die, die ihr Schaden zuf�gen, hinwegsieht.
\pend
\endnumbering
\end{translatio}
\end{Rightside}
\Columns
\end{pairs}

\autor{Brief der ">westlichen"< Synode an die Bisch�fe �gyptens und Libyens}
\begin{pairs}
\selectlanguage{polutonikogreek}
\begin{Leftside}
\pstart
\edlabel{brief2-1}
\hskip -1em\kap{2,1}\edtext{<H <ag'ia s'unodos <h kat`a jeo~u q'arin >en Sardik~h| sunaqje~isa
to~is kat'' A>'igupton ka`i Lib'uhn >episk'opoic ka`i sulleitourgo~ic,
>agaphto~ic >adelfo~ic, >en kur'iw| qa'irein. Ka`i pr`in m`en labe~in
t`a gr'ammata t~hc e>ulabe'ias <um~wn \dots}{\lemma{\abb{<H \dots\
    <um~wn}}\Dfootnote{\dt{OR > BK E}}}
\pend
\pstart
\begin{footnotesize}
  \kap{2}\edlabel{ident}\edtext{ka`i di`a t`o e>~inai a>ut`hn >'ishn kat`a p'anta
    t~hc pr`o a>ut~hc >'htoi t~hc pr`oc t`hn >ekklhs'ian
    >Alexandre'iac \edlabel{plhn-1}pl`hn t~wn pr`oc t`o t'eloc seshmiwm'enwn\edlabel{plhn-2}
    katele'ifjh >'agrafoc di`a t`o m`h deuterwj~hnai ka`i t`hn
    spoud`hn ka`i t`a q'arta.}{\lemma{\abb{ka`i \dots\
        q'arta}}\Dfootnote{\dt{R} <h d`e toia'uth >epistol`h
      katele'ifjh >'agrafoc di`a t`o e>~inai >'ish kat`a p'anta t~h|
      pr`o a>ut~hc >'htoi t~h| pr`oc t`hn >ekklhs'ian >Alexandre'iac
      stale'ish| <'wste m`h deuterwj~hnai kat`a t`o periss'on \dt{O}
      <h a>ut`h o>~un >epistol`h >aparall'aktwc kat`a p'anta >egr'afh
      pr`oc to`uc kat'' A>'igupton ka`i Lib'uhn >episk'opouc \dt{BK >
        E}}}\edlabel{brief2-2}
\end{footnotesize}
\pend
\end{Leftside}
\begin{Rightside}
\begin{translatio}
\beginnumbering
\pstart
\noindent\kapR{2,1}Die heilige Synode, die sich durch die Gnade Gottes in
Serdica versammelt hat, gr��t im Herrn die Bisch�fe und Mitdiener in
�gypten und Libyen, die geliebten Br�der.
Schon bevor wir die Briefe eurer Fr�mmigkeit erhielten, \dots
\pend
\pstart
\begin{footnotesize}
\kapR{2}Und weil dieser Brief in allem dem vorangehenden, d.\,h. dem an
die Kirche Alexandriens, gleich ist~-- abgesehen von dem, was gegen Ende
markiert ist~-- wurde er ungeschrieben gelassen, um nicht doppelt so
viel M�he und Papier aufwenden zu m�ssen.
\end{footnotesize}
\pend
\endnumbering
\end{translatio}
\end{Rightside}
\Columns
\end{pairs}
\selectlanguage{german}
% \thispagestyle{empty}
%%% Local Variables: 
%%% mode: latex
%%% TeX-master: "dokumente_master"
%%% End: 

\section[Brief des Athanasius von Alexandrien an Presbyter und Diakone Alexandriens und der Parembole][Brief an Presbyter und Diakone Alexandriens und der Parembole]{Brief des Athanasius von Alexandrien an Presbyter und Diakone Alexandriens\\und der Parembole}
% \label{sec:43.2}
\label{sec:BriefAthAlexParem}
\begin{praefatio}
  \begin{description}
  \item[Herbst 343]Zum Datum vgl. Einleitung zu
    Dok. \ref{ch:SerdicaEinl}. Dieser Brief wurde von
    Athanasius\index[namen]{Athanasius!Bischof von Alexandrien}
    zusammen mit Dok. \ref{sec:BriefSerdikaAlexandrien} und Akten der Synode an
    die Alexandriner
    verschickt.
  \item[�berlieferung]Zur �berlieferung des Codex Veronensis
    vgl. Dok. \ref{ch:SerdicaEinl}. Dem Text liegt wahrscheinlich ein
    griechisches Original zugrunde, eine �bersetzung l��t sich jedoch
    nicht eindeutig nachweisen.
  \item[Fundstelle]Codex Veronensis LX f. 99b--102a
  \end{description}
\end{praefatio}
\begin{pairs}
\selectlanguage{latin}
\begin{Leftside}
% \beginnumbering
\pstart
\hskip -1.5em\edtext{\abb{}}{\killnumber\Cfootnote{\hskip -.6em Cod.\,Ver. (cod.)}}\specialindex{quellen}{section}{Codices!Veronensis LX!f. 99b--102a}
\kap{1}
Athanasius\edindex[namen]{Athanasius!Bischof von Alexandrien} presbyteris et diaconibus
omnibus ecclesiae sanctae 
\edtext{aput}{\Dfootnote{apud \textit{coni. Maffei}}} Alexandriam\edindex[namen]{Alexandrien}
et 
\edtext{\abb{Parembulam}}{\Dfootnote{\textit{coni. Opitz} Pareambulam \textit{cod.} Parembolam \textit{coni. Maffei}}} 
\edtext{\abb{catholicae}}{\Dfootnote{\textit{coni. Ballerini} catholice \textit{cod.}}} dilectissimis
fratribus salutem.
\pend
\pstart
\kap{2}Haec scribentes oportet 
\edtext{\abb{epistulae}}{\Dfootnote{\textit{coni. Maffei} epistule \textit{cod.}}} principium gratiarum Christi actionibus
facere, fratres dilectissimi; nunc autem maxime \edtext{decet}{\Dfootnote{docet \textit{coni. Maffei}}}
hoc fieri, quoniam et 
\edtext{\abb{facta}}{\Dfootnote{\textit{coni. Maffei} factam \textit{cod.}}} multa 
\edtext{aput}{\Dfootnote{apud \textit{coni. Maffei}}} dominum et magna
habent gratiam, et 
\edtext{\abb{oportet}}{\Dfootnote{\textit{coni. Maffei} oporte \textit{cod.}}} credentes in eum non esse
ingratos tot eius beneficiis. gratias igitur agimus domino, qui nos semper
omnibus palam facit in fide, qui etiam in 
\edtext{presenti}{\Dfootnote{praesenti \textit{coni. Maffei}}} magna et mirabilia fecit
\edtext{\abb{ecclesiae; quae}}{\Dfootnote{\textit{coni. Maffei} ecclesie que \textit{cod.}}} enim rursum 
\edtext{\abb{affirmaverunt}}{\Dfootnote{\textit{coni. Maffei} adfirmaverunt \textit{coni. Opitz} afirmaverunt \textit{cod.}}} divulgantes
\edtext{heretici}{\Dfootnote{haeretici \textit{coni. Maffei}}} Eusebiani\edindex[namen]{Eusebianer} et Arrii\edindex[namen]{Arianer} 
\edtext{heredes}{\Dfootnote{haeredes \textit{coni. Maffei}}}, 
\edtext{\abb{haec}}{\Dfootnote{\textit{coni. Maffei} hec \textit{cod.}}} omnes qui convenerunt episcopi
pronuntiaverunt falsa 
\edtext{\abb{esse}}{\Dfootnote{\textit{coni. Opitz} ea esse \textit{coni. Maffei} essa esse \textit{cod.}}} et \edtext{fincta}{\Dfootnote{ficta coni. \textit{Maffei}}}. et 
\edtext{\abb{ii}}{\Dfootnote{\textit{coni. Maffei} is \textit{cod.}}} ipsi qui 
\edtext{aput}{\Dfootnote{apud \textit{coni. Maffei}}} multos putantur esse 
\edtext{\abb{terribiles}}{\Dfootnote{\textit{coni. Maffei} ettibiles \textit{cod.}}}, 
\edtext{\abb{tamquam}}{\Dfootnote{\textit{coni. Opitz} tanquam \textit{coni. Maffei} tam \textit{cod.}}} gigantes nominati, pro nihilo
habiti sunt~-- et merito: quemadmodum enim adveniente luce 
\edtext{\abb{tenebrae}}{\Dfootnote{\textit{coni. Maffei} tenebre \textit{cod.}}} arguuntur,
sic per 
\edtext{\abb{adventum}}{\Dfootnote{\textit{coni. Maffei} aventum \textit{cod.}}} iustorum iniquitas
revelatur, et presentibus egregiis debiles convincuntur. 
\pend
\pstart
\kap{3}%
\edtext{\abb{quae}}{\Dfootnote{\textit{coni. Maffei} que \textit{cod.}}} enim fecerunt 
\edtext{\abb{maledicae}}{\Dfootnote{\textit{coni. Maffei} maledica \textit{cod.}}} 
\edtext{heresis}{\Dfootnote{haeresis \textit{coni. Maffei}}} 
\edtext{\abb{Eusebii successores}}{\Dfootnote{\textit{coni. Maffei} Eusebiissucessores
\textit{cod.}}}\edindex[namen]{Eusebius!Bischof von Nikomedien},
Theodorus\edindex[namen]{Theodorus!Bischof von Heraclea} 
\edtext{\abb{Narcissus}}{\Dfootnote{+ et \textit{cod.}, \textit{quod del. Maffei}}}\edindex[namen]{Narcissus!Bischof von Neronias} 
Valens\edindex[namen]{Valens!Bischof von Mursa} Ursacius\edindex[namen]{Ursacius!Bischof von Singidunum} et in omnibus pessimus Georgius\edindex[namen]{Georg!Bischof von Laodicea} Stephanus\edindex[namen]{Stephanus!Bischof von Antiochien}
Acacius\edindex[namen]{Acacius!Bischof von Caesarea}
Minophantus\edindex[namen]{Menophantus!Bischof von Ephesus} et
eorum 
\edtext{\abb{collegae}}{\Dfootnote{\textit{coni. Maffei} college \textit{cod.}}}, nec vos ignoratis, dilecti, nam eorum dementia
omnibus patefacta est; 
\edtext{\abb{quae vero}}{\Dfootnote{\textit{coni. Maffei} quero \textit{cod.}}} contra ecclesias
conmiserunt, vestram nec hoc latuit sollertiam; primum enim vobis nocuerunt;
primum vestram ecclesiam corrumpere temptaverunt. sed hii qui tot 
\edtext{\abb{ac}}{\Dfootnote{\textit{coni. Maffei} hac \textit{cod.}}} tanta fecerunt et apud omnes
terribiles 
\edtext{estimati}{\Dfootnote{aestimati \textit{coni. Maffei}}} sunt, sicut 
predixi, tantum timuerunt, ut omnem exsuperent 
\edtext{\abb{cogitationem}}{\Dfootnote{\textit{coni. Maffei} cogitatione \textit{cod.}}}. neque enim solum
Romanam\edindex[namen]{Rom} synodum timuerunt, nec solum 
\edtext{\abb{se ibi}}{\Dfootnote{\textit{coni. Opitz} se \textit{coni. Maffei} sibi \textit{cod.}}} vocati excusaverunt; sed et nunc
cum Sardicam\edindex[synoden]{Serdica!a. 343} advenissent, sic infirmati sunt conscientia 
\edtext{\abb{ut}}{\Dfootnote{\textit{coni. Maffei} sic ut \textit{cod.}}},
cum vidissent iudices, mirarentur. sic mente conciderunt, ut vere quis posset
adversum eos dicere: 
�\edtext{ubi est 
\edtext{\abb{stimulus}}{\Dfootnote{\textit{coni. Maffei} stipulus \textit{cod.}}} tuus, mors? ubi est
victoria tua, mors?}{\lemma{\abb{}}\Afootnote{1Cor 15,55}}�\edindex[bibel]{Korinther I!15,55} nec enim illis
proficiebat, ut velint iudicare; iam non 
\edtext{\abb{poterant}}{\Dfootnote{\textit{coni. Maffei} peterunt \textit{cod.}}} circumvenire quos volebant; sed videbant viros fideles
curantes iustitiam, immo magis ipsum dominum nostrum videbant in 
\edtext{\abb{eis}}{\Dfootnote{\textit{cod\corr} eos \textit{cod*}}},
\edtext{quemadmodum tunc 
\edtext{demones}{\Dfootnote{daemones \textit{coni. Maffei}}} de sepulchris}{\lemma{\abb{quemadmodum \dots\ sepulchris}}\Afootnote{vgl. Lc
8,27~f.}}\edindex[bibel]{Lukas!8,27~f.|textit}; filii enim cum essent mendacii, non
ferebant veritatem videre. 
\pend
\pstart
\kap{4}%
sic Theodorus\edindex[namen]{Theodorus!Bischof von Heraclea} Narcissus\edindex[namen]{Narcissus!Bischof von Neronias} et Ursacius\edindex[namen]{Ursacius!Bischof von Singidunum} cum suis verba 
\edtext{\abb{dicebant}}{\Dfootnote{\textit{coni. Ballerini} dicebat \textit{cod.}}}: 
\edtext{omitte, quid nobis}{\lemma{\abb{omitte \dots\ nobis}}\Afootnote{vgl. Lc 4,34; Mc 1,24}}\edindex[bibel]{Lukas!4,34|textit}\edindex[bibel]{Markus!1,24|textit}
et vobis hominibus Christi? novimus quod veri estis, et timemus convinci; 
\edtext{\abb{veremur}}{\Dfootnote{\textit{cod.\corr} veremus \textit{cod*}}} in personam recognoscere
calumnias. Nihil est nobis et vobis: christiani enim vos estis, nos vero Christo
repugnantes; et apud vos quidem veritas pollet, nos vero circumvenire 
didicimus. putavimus abscondi nostra; non iam  
\edtext{\abb{credebamus}}{\Dfootnote{\textit{coni. Maffei} credevamus \textit{cod.}}} in iudicium venire:
quid ante tempus nostra convincitis, 
et \edtext{ante diem nos convincentes vexatis}{\lemma{\abb{ante \dots\ vexatis}}\Afootnote{vgl. Mt 8,29}}?\edindex[bibel]{Matthaeus!8,29|textit}
\pend
\pstart
\kap{5}et
licet sint moribus pessimi et 
\edtext{in tenebris ambulent}{\lemma{\abb{in \dots\ ambulent}}\Afootnote{1Io 1,6}},\edindex[bibel]{Johannes I!1,6} tamen
cognoverunt vix
tandem, quoniam 
\edtext{\abb{nulla}}{\Dfootnote{\textit{coni. Maffei} ulla \textit{cod.}}}  est 
\edtext{communio lucis et
tenebrarum}{\lemma{\abb{communio \dots\ tenebrarum}}\Afootnote{vgl. 2Cor 6,14}},\edindex[bibel]{Korinther II!6,14|textit} nec est aliqua 
\edtext{\edtext{\abb{consensio}}{\Dfootnote{\textit{coni. Maffei} consentio \textit{cod.}}} Christo cum
Belial}{\lemma{\abb{consensio \dots\ Belial}} \Afootnote{vgl. 2Cor 6,15}}\edindex[bibel]{Korinther II!6,15|textit}.
unde, fratres dilectissimi, 
\edtext{\abb{cum}}{\Dfootnote{\textit{coni. Maffei} eum \textit{cod.}}} scirent
\edtext{\abb{quae}}{\Dfootnote{\textit{coni. Maffei} quem \textit{cod.}}} fecerint 
\edtext{quaeque}{\Dfootnote{\textit{coni. Opitz} \textit{quecumque cod.} quaecunque \textit{coni. Maffei} quantumcunque \textit{susp. Ballerini} eorumque \textit{susp. Maffei}}} 
\edtext{conmiserint}{\Dfootnote{\textit{coni. Opitz} miserrimos \textit{cod.} acerrimos \textit{susp. Maffei}}}, videntes accusatores, testes 
\edtext{\abb{prae}}{\Dfootnote{\textit{coni. Maffei} pre \textit{cod.}}} oculis habentes,
imitati sunt Cain et illius more fugerunt, quoniam granditer erraverunt; 
\edtext{nec enim}{\Dfootnote{etenim \textit{coni. Maffei} etiam \textit{susp. Maffei}}} \Ladd{\edtext{\abb{tantum}}{\Dfootnote{\textit{add. Wintjes}}}} eius fugam sunt imitati, et 
\edtext{\abb{condemnationem}}{\Dfootnote{\textit{coni. Maffei} condemnatione \textit{cod.}}} 
\edtext{habuerunt}{\Dfootnote{non habuerunt \textit{coni. Opitz}}}: cognovit enim opera eorum sancta synodus; audivit
nostrum sanguinem proclamantem, audivit voces 
\edtext{\abb{laesorum}}{\Dfootnote{\textit{coni. Maffei} lesorum \textit{cod.}}} ab ipsis. cognoverunt
omnes episcopi 
\edtext{\abb{quae}}{\Dfootnote{\textit{coni. Maffei} que \textit{cod.}}} peccaverunt et quanta adversus ecclesias nostras et alias
operati sunt; et ideo hos quemadmodum Cain ecclesiis eiecerunt. quis enim non
lacrimatus est dum 
\edtext{\abb{vestrae litterae}}{\Dfootnote{\textit{coni. Maffei} verstre littere \textit{cod.}}} legerentur? quis non ingemuit aspiciens quos
exiliaverunt isti? quis non existimavit vestras suas esse tribulationes, fratres
dilectissimi? 
\edtext{\abb{quondam}}{\Dfootnote{\textit{coni. Maffei} condam \textit{cod.}}} vos 
\edtext{\abb{patiebamini}}{\Dfootnote{\textit{coni. Maffei} patievamini \textit{cod.}}}, cum hii delinquerent adversum vos, et
forte iam tempore 
\edtext{\abb{multo}}{\Dfootnote{\textit{coni. Maffei} multum \textit{cod.}}} bellum 
\edtext{quievit}{\Dfootnote{non quievit \textit{coni. Maffei}}}. nunc vero episcopi
convenientes omnes et audientes 
\edtext{\abb{quae}}{\Dfootnote{\textit{coni. Maffei} que \textit{cod.}}} passi estis, 
\edtext{\abb{sic}}{\Dfootnote{\textit{coni. Maffei} si \textit{cod.}}} dolebant, sic gemebant, quemadmodum
\edtext{\abb{tolerantes}}{\Dfootnote{\textit{coni. Maffei} tollerantes \textit{cod.}}} iniuriam tunc dolebatis, et illis 
\edtext{\abb{ac vestris erat}}{\Dfootnote{\textit{coni. Erl.} hac verberat \textit{cod.} \dots\ erat \textit{susp. Maffei} acerbus erat \textit{susp. Opitz}}} dolor
communis illo tempore quo processistis. 
\pend
\pstart
\kap{6}ob 
\edtext{\abb{haec}}{\Dfootnote{\textit{coni. Maffei} hec \textit{cod.}}} igitur et alia omnia 
\edtext{\abb{quae}}{\Dfootnote{\textit{coni. Ballerini} que \textit{cod.} \textit{del. Maffei}}} contra
ecclesias conmiserunt, cunctos 
\edtext{\abb{universa synodus}}{\Dfootnote{\textit{coni. Maffei} universassynodus \textit{cod.}}} sancta deposuit, et non solum
eos alienos iudicavit ab 
\edtext{\abb{ecclesia}}{\Dfootnote{\textit{cod.\corr} ecccllesia \textit{cod.*}}}, sed nec dignos vocari christianos 
\edtext{estimavit}{\Dfootnote{aestimavit \textit{coni. Maffei}}}.
qui enim 
\edtext{\abb{abnegantes}}{\Dfootnote{\textit{coni. Maffei} aabnegantes \textit{cod.}}} Christum, quemadmodum christiani vocentur? et qui contra
ecclesias 
\edtext{\abb{delinquunt}}{\Dfootnote{\textit{coni. Maffei} delinqunt \textit{cod.}}}, hii quemadmodum 
\edtext{\abb{poterunt}}{\Dfootnote{\textit{coni. Mafffei} potuerunt \textit{cod.}}} adesse ecclesiis? unde
mandavit sancta synodus ubique ecclesiis, ut 
\edtext{aput}{\Dfootnote{apud \textit{coni. Maffei}}} omnes notentur; ut hii qui ab 
\edtext{\abb{ipsis}}{\Dfootnote{\textit{coni. Maffei} ipsi \textit{cod.}}} decepti sunt iam ad plenitudinem
et veritatem revertantur. 
\pend
\pstart
\kap{7}nolite igitur deficere, fratres dilectissimi; tamquam
dei servi et fidem Christi confitentes 
\edtext{\abb{provehimini}}{\Dfootnote{\textit{coni. Opitz} proveemini \textit{cod.} probemini
\textit{coni. Maffei}}} in domino, et non deiciat vos tribulatio, neque ab 
\edtext{hereticis}{\Dfootnote{haereticis \textit{coni. Maffei}}} 
\edtext{\abb{adversum}}{\Dfootnote{\textit{cod\corr} adversus \textit{cod*}}} vos qui 
\edtext{\abb{exercentur}}{\Dfootnote{\textit{coni. Maffei} exercuntur \textit{cod.}}} dolores 
\edtext{\abb{contristent}}{\Dfootnote{\textit{cod\corr} contristrent \textit{cod.*}}}.
habetis enim mundum universum 
\edtext{condolentes}{\Dfootnote{condolentem \textit{coni. Maffei}}} vobis et~-- quod maius est~-- 
\edtext{habentes}{\Dfootnote{habentem \textit{coni. Maffei}}}
omnes vos in mentem. puto autem iam deceptos ab illis, videntes correptionem
factam a synodo, illos averti et 
\edtext{ex ore}{\Dfootnote{exhorrere \textit{susp. Maffei}}} ipsorum inpietatem. si vero post 
\edtext{\abb{haec}}{\Dfootnote{\textit{coni. Maffei} hec \textit{cod.}}}
adhuc manus est eorum excelsa, ne stupeatis vos neque formidetis si illi
saeviunt; sed orate et manus ad deum levate et 
\edtext{confidete}{\Dfootnote{confidite \textit{coni. Maffei}}} quoniam non 
\edtext{\abb{tardabit}}{\Dfootnote{\textit{coni. Maffei} tardavit \textit{cod.}}} dominus sed omnia vobis
faciet pro vestra  voluntate. 
\pend
\pstart
\kap{8}vellem quidem adhuc pluribus 
\edtext{epistulam}{\Dfootnote{epistolam \textit{coni. Maffei}}} 
\edtext{\abb{vobis scribere}}{\Dfootnote{\textit{coni. Maffei} vobisribere \textit{cod.}}}, et ut singula facta sunt significare; 
\edtext{\abb{sed}}{\Dfootnote{\textit{add. Maffei}}} quoniam presbyteri et diacones
idonei sunt nuntiare vobis presentes de omnibus 
\edtext{\abb{quae}}{\Dfootnote{\textit{coni. Maffei} q' \textit{cod.}}} viderunt, multa quidem scribere 
\edtext{\abb{cessavi}}{\Dfootnote{\textit{coni. Maffei} cessavit \textit{cod.}}}, illud tantum significo
necessarium putans ut 
\Ladd{\edtext{\abb{illud}}{\Dfootnote{\textit{add. Wintjes}}}}
prae oculis habentes timorem domini
\edtext{\edtext{preponatis}{\Dfootnote{praeponatis \textit{coni. Maffei}}}}{\lemma{\abb{}} \Dfootnote{eum \textit{ante} preponatis \textit{add. Maffei}}} et omnia cum vestra concordia celebretis intellegentes et
sapientes. orate pro nobis, habentes in mente viduarum necessitates, maxime
quoniam ad 
\edtext{\abb{eas}}{\Dfootnote{\textit{coni. Maffei} ea \textit{cod.}}} pertinentia inimici veritatis 
\edtext{\abb{abstulerunt}}{\Dfootnote{\textit{susp. Maffei} abstullerint \textit{cod.} obtulerunt
\textit{coni. Maffei}}}; 
\edtext{\abb{sed dilectio}}{\Dfootnote{\textit{coni. Maffei} sedilectio \textit{cod.}}}
vestra vincat 
\edtext{hereticorum}{\Dfootnote{haereticorum \textit{coni. Maffei}}} 
\edtext{\abb{malitiam}}{\Dfootnote{\textit{coni. Maffei} malitia \textit{cod.}}}: credimus enim quod
secundum orationes vestras dominus adnuens 
\edtext{\abb{dabit}}{\Dfootnote{\textit{coni. Maffei} davit \textit{cod.}}} mihi velocius vos videre.
interim tamen aput synodum 
\edtext{\abb{actitata}}{\Dfootnote{\textit{coni. Maffei} acitata \textit{cod.}}} 
\edtext{\abb{cognoscetis}}{\Dfootnote{\textit{coni. Maffei} cognoscitis \textit{cod.}}} \edlabel{v437a}ex scriptis ad vos
ab omnibus episcopis\edlabel{v437b} et de subiectis litteris depositionem 
\edtext{\abb{Theodori}}{\Dfootnote{\textit{coni. Maffei} Thodori \textit{cod.}}}\edindex[namen]{Theodorus!Bischof von Heraclea} Narcissi\edindex[namen]{Narcissus!Bischof von Neronias}
Stephani\edindex[namen]{Stephanus!Bischof von Antiochien} Acacii\edindex[namen]{Acacius!Bischof von Caesarea}
Georgii\edindex[namen]{Georg!Bischof von Laodicea} Minophanti\edindex[namen]{Menophantus!Bischof von Ephesus} 
\edtext{\abb{Ursacii}}{\Dfootnote{\textit{coni. Maffei} Sacii \textit{cod.}}}\edindex[namen]{Ursacius!Bischof von Singidunum} et Valentis\edindex[namen]{Valens!Bischof von Mursa}: nam 
\edtext{\abb{Gregori}}{\Dfootnote{\textit{coni. Maffei} Regori \textit{cod.}}}\edindex[namen]{Gregor!Bischof von Alexandrien} mentionem facere noluerunt,
qui enim penitus episcopi nomen non habuit, hunc nominare superfluum putaverunt.
sed tamen propter deceptos ab eo eius nominis mentionem fecerunt, non 
\edtext{\abb{quia dignum}}{\Dfootnote{\textit{coni. Maffei} quiaignum \textit{cod.}}}
erat eius nomen memorare, sed ut ab eo 
\edtext{\abb{decepti}}{\Dfootnote{\textit{coni. Maffei} deceptis \textit{cod.}}} cognoscant eius infamiam et
erubescent quod tali communicaverunt. 
\dag\edtext{tamen et hoc cum illis}{\Dfootnote{\textit{lacunam ante} tamen
\textit{significavit Maffei}; \textit{fortasse scribendum} tamen et hunc cum
illis depositum esse sciatis \textit{susp. Opitz}}}\dag. 
\pend
\pstart
\kap{9}incolumes vos in domino oro, 
\edtext{\abb{dilectissimi}}{\Dfootnote{\textit{coni. Ballerini} dilectissim \textit{cod.}}} et 
\edtext{desiderabiles}{\Dfootnote{\textit{coni. Ballerini} desidelabiles \textit{cod.}}} fratres.
\pend
% \endnumbering
\end{Leftside}
\begin{Rightside}
\begin{translatio}
\small
\beginnumbering
\pstart
\noindent\kapR{1}Athanasius gr��t alle Presbyter und Diakone, die vielgeliebten Br�der der
heiligen, katholischen Kirche in Alexandria und der Parembole.
\pend
\pstart
\kapR{2}Wenn wir dies schreiben, geh�rt es sich, vielgeliebte Br�der, den Brief mit den
Gnadenerweisen Christi zu beginnen. Besonders jetzt aber ziemt es sich, so vorzugehen,
weil viele und gro�e Taten dem Herrn zu verdanken sind und die, die an ihn
glauben, nicht undankbar sein d�rfen angesichts so vieler Wohltaten von ihm. Wir
danken also dem Herrn, der uns stets bei allen im Glauben bekannt macht und der
auch in der Gegenwart gro�e und wunderbare Dinge f�r die Kirche getan hat; was
n�mlich die h�retischen Eusebianer und Erben des Arius verbreiteten und erneut
bekr�ftigten, das haben alle Bisch�fe, die zusammengekommen sind, f�r falsch
und erfunden erkl�rt. Und gerade jene, die viele als furchteinfl��end einsch�tzen
und sie quasi Giganten nennen, wurden f�r nichts gehalten~-- und
zu Recht: Wie n�mlich, wenn das Licht heranr�ckt, das Dunkel deutlich wird, so
wird durch die Ankunft der Gerechten die Ungerechtigkeit entlarvt, und in der
Gegenwart der Herausragenden werden die Hinf�lligen �berf�hrt. 
\pend
\pstart
\kapR{3}Was n�mlich die Nachfolger der verruchten H�resie des Eusebius getan haben,
Theodorus, Narcissus, Valens, Ursacius und unter allen der Schlimmste,\footnoteA{Eventuell ist hier Georg von Laodicea mit Gregor von Alexandrien verwechselt und aus diesem Grund die Worte \textit{in omnibus pessimus} nachtr�glich eingef�gt worden, vgl. Dok. \ref{sec:BriefSerdikaMareotis},2, wo Gregor von Alexandrien als \textit{princeps pessimus} bezeichnet wird.} Georg,
sowie Stephanus, Acacius, Menophantus und deren Amtsbr�der,\footnoteA{Vgl. Dok.
\ref{sec:SerdicaRundbrief},14.} wi�t ihr sehr genau, verehrte Br�der, denn
deren Wahn ist f�r alle un�bersehbar. Welch Schaden sie jedoch den Kirchen
zugef�gt haben, auch das blieb eurer Klugheit nicht verborgen. Sie haben n�mlich
zuerst euch �bel zugerichtet; zuerst haben sie versucht, eure Kirche zu verderben. Diese aber,
die so viele ungeheure Dinge getan haben und die alle f�r furchtbar halten,
haben, wie ich vorher gesagt habe, nur Angst davor gehabt, nicht
jegliche Kritik ausschalten zu k�nnen. Sie f�rchteten n�mlich nicht nur die
r�mische Synode und brachten nicht nur damals Entschuldigungen vor, als sie dorthin gerufen
wurden, sondern auch jetzt, als sie nach Serdica gekommen waren, belastete sie 
ihr Gewissen so sehr, da� sie sich wunderten, als sie die
Richter sahen. Sie brachen innerlich zusammen, da� man
wahrhaft h�tte gegen sie sagen k�nnen: ">Tod, wo ist dein Stachel? Tod, wo ist
dein Sieg?"< Auch n�tzte es ihnen n�mlich nicht, da� sie Urteile f�llen wollten.
Nun konnten sie nicht nach Belieben Leute t�uschen, sondern sie sahen, da� sich
gl�ubige M�nner um die Gerechtigkeit sorgten, ja sie erkannten sogar unseren
Herrn selbst in jenen M�nnern wie damals die D�monen von den Grabh�hlen aus; weil sie
n�mlich S�hne der L�ge waren, ertrugen sie es nicht, die Wahrheit zu sehen.
\pend
\pstart
\kapR{4}So �u�erten sich Theodorus,
Narcissus und Ursacius zusammen mit ihren Anh�ngern:
\frq La� ab, was haben wir mit euch M�nnern Christi gemein?
Wir wissen, da� ihr die Wahrhaftigen seid, und f�rchten, �berf�hrt zu werden. Wir
scheuen uns, die pers�nlichen Verleumdungen erneut in Erinnerung zu
bringen. Nichts haben wir mit euch gemein: Ihr n�mlich seid Christen, wir aber
Gegner Christi. Bei euch freilich herrscht die Wahrheit, wir aber haben
gelernt, einen Bogen um sie zu machen. Wir haben geglaubt, das Unsere lie�e sich
verbergen. Wir glaubten, nicht mehr vor Gericht gestellt zu werden: Was
deckt ihr unsere Vergehen auf, ehe es Zeit ist, und qu�lt uns, indem ihr uns vor dem
(j�ngsten) Tag �berf�hrt?\flq
\pend
\pstart
\kapR{5}Und obwohl sie die sittlich Schlechtesten sind und in der
Finsternis wandeln, so sind sie sich dessen letztlich dennoch kaum bewu�t, da es ja keine
Gemeinschaft des Lichts mit der Finsternis gibt und keinerlei Eintracht zwischen Christus
und Belial besteht. Daher, vielgeliebte Br�der, weil sie wu�ten, was sie getan
und was sie begangen hatten, haben sie, als sie die Ankl�ger sahen und die Zeugen vor Augen hatten,
Kain nachgeahmt und sind ganz nach dessen Gewohnheit geflohen, weil sie sich ja gewaltig geirrt hatten.
Sie haben nicht nur seine Flucht nachgeahmt, sondern bekamen auch seine Verdammung: Die
heilige Synode erfuhr n�mlich von ihren Werken. Sie h�rte unser aufschreiendes Blut und
h�rte die Stimmen der von ihnen Gesch�digten. Alle Bisch�fe erfuhren, was sie
ges�ndigt hatten und wie viel Unheil sie in unseren und anderen Kirchen angerichtet hatten. Und
deshalb warfen sie diese wie Kain aus den Kirchen. Wer weinte denn nicht, w�hrend eure
Briefe verlesen wurden? Wer seufzte nicht auf, als er die erblickte, die jene in die
Verbannung schickten? Wer sah nicht eure Bedr�ngnisse als seine eigenen an, vielgeliebte
Br�der? Einst habt ihr gelitten, als diese sich gegen euch vergingen, und nur zuf�llig
ruhte der Krieg nun f�r lange Zeit. Jetzt aber empfanden die
Bisch�fe, als sie alle zusammenkamen und h�rten, was ihr erlitten habt, solchen Schmerz
und seufzten so, wie ihr damals Schmerz empfandet, als ihr Unrecht ertrugt,
und f�r jene wie f�r die Eueren war es ein gemeinsamer Schmerz zu jener
Zeit, da ihr auftratet. 
\pend
\pstart
\kapR{6}Deswegen also und wegen all der anderen Dinge, die sie gegen die Kirchen
begangen haben, hat die gesamte heilige Synode\footnoteA{Athanasius �bergeht
stillschweigend die Existenz der ">�stlichen"< Synode, da deren Teilnehmer f�r ihn
verurteilte ">Arianer"< sind.} sie alle
zusammen abgesetzt und sie nicht nur als Fremde in der Kirche beurteilt,
sondern sie auch f�r unw�rdig erachtet, Christen genannt zu werden. Wie n�mlich
k�nnten Leute, die Christus verleugnen, Christen genannt werden? Und die, die sich gegen
die Kirchen vergehen, wie k�nnen diese zu den Kirchen geh�ren? Daher gab die heilige
Synode den Kirchen �berall den Auftrag, da� diese bei allen gebrandmarkt werden, damit die,
die von eben diesen get�uscht wurden, jetzt zur F�lle und Wahrheit 
zur�ckkehren. 
\pend
\pstart
\kapR{7}Fallt also nicht ab, vielgeliebte Br�der! Gleichsam als Diener
Gottes und Bekenner des Glaubens an Christus schreitet 
im Herrn fort, und weder soll euch die Bedr�ngnis vom Weg abbringen noch sollen
euch die Schmerzen, die die H�retiker euch zuf�gen,
betr�ben. Ihr habt n�mlich alle Welt, die mit euch mitleidet und~-- was
wichtiger ist~-- die an euch alle denkt. Ich glaube aber, da� die, die von
jenen bereits get�uscht worden sind, sich abwenden und aus ihrem Mund die
Gottlosigkeit weicht, wenn sie sehen, da� von der Synode ein Tadel erfolgte.
Wenn aber nach diesen Ereignissen ihre Hand bis jetzt gegen euch erhoben ist,
so wundert und f�rchtet euch nicht, wenn sie w�ten, sondern betet, erhebt die
H�nde zu Gott und habt Vertrauen, weil der Herr nicht s�umen, sondern alles f�r
euch tun wird nach euerem Willen. 
\pend
\pstart
\kapR{8}Ich wollte freilich, ich k�nnte euch einen ausf�hrlicheren Brief schreiben und
zeigen, wie die einzelnen Dinge geschehen sind. Aber da die Presbyter und Diakone in der Lage
sind, euch pers�nlich �ber alles Bericht zu erstatten, was sie gesehen
haben, z�gere ich freilich, viel zu schreiben und deute nur dies an, was ich f�r
notwendig halte, damit ihr dies vor Augen habt und die Furcht vor dem Herrn an die
Spitze stellt und alles in eurer Einm�tigkeit verst�ndig und weise vollzieht. Betet f�r
uns und denkt an die Bed�rfnisse der Witwen, besonders weil die Feinde der Wahrheit das
ihnen Zustehende geraubt haben. Doch eure Liebe m�ge die Bosheit der H�retiker besiegen:
Wir glauben n�mlich, da� der Herr euren Gebeten zustimmen und mir gew�hren wird,
euch recht schnell zu sehen. In der Zwischenzeit werdet ihr das auf der Synode Verhandelte
dennoch aus den von allen Bisch�fen an euch gerichteten 
Schriftst�cken\footnoteA{Vgl. Dok. \ref{sec:BriefSerdikaAlexandrien}.} und aus den beigef�gten
Briefen erfahren, n�mlich die Absetzung des Theodorus, Narcissus, Stephanus, Acacius, Georg, Menophantes,
Ursacius und des Valens\footnoteA{In Dok. \ref{sec:B},6.}; Gregor wollten sie n�mlich
nicht erw�hnen; sie meinten, es sei �berfl�ssig, diesen zu nennen, weil er n�mlich den
Bischofstitel im strengen Sinn nicht besa�.\footnoteA{Vgl. Dok.
\ref{sec:BriefSerdikaAlexandrien},16.} Aber dennoch haben sie wegen der von ihm
Get�uschten seinen Namen erw�hnt, nicht weil sein Name einer Erinnerung wert w�re, sondern
damit die von ihm Get�uschten seine Ruchlosigkeit erkennen und sich sch�men, da� sie mit
einem solchen Gemeinschaft hatten. \dag\dots\dag
\pend
\pstart
Ich bete, da� ihr wohlbehalten seid im Herrn, vielgeliebte und ersehnte Br�der.
\pend
\endnumbering
\end{translatio}
\end{Rightside}
\Columns
\end{pairs}
% \thispagestyle{empty}
%%% Local Variables: 
%%% mode: latex
%%% TeX-master: "dokumente_master"
%%% End: 

\section[Brief der ">westlichen"< Synode an Presbyter und Diakone in der Mareotis][Brief an Presbyter und Diakone in der Mareotis]{Brief der ">westlichen"< Synode an Presbyter und Diakone in der Mareotis}
% \label{sec:43.2}
\label{sec:BriefSerdikaMareotis}
\begin{praefatio}
  \begin{description}
  \item[Herbst 343]Zum Datum vgl. die Einleitung zu
    Dok. \ref{ch:SerdicaEinl}. Wegen der umstrittenen Vorf�lle in der
    Mareotis\index[namen]{Mareotis}, die zur Absetzung des
    Athanasius\index[namen]{Athanasius!Bischof von Alexandrien} auf
    der Synode von Tyrus\index[synoden]{Tyrus!a. 335} 335 f�hrten,
    schickten die ">westliche"< Synode von
    Serdica\index[synoden]{Serdica!a. 343} und auch
    Athanasius\index[namen]{Athanasius!Bischof von Alexandrien}
    (s. Dok. \ref{sec:BriefAthMareotis}) ein extra Schreiben dorthin,
    um die Verh�ltnisse f�r Athanasius\index[namen]{Athanasius!Bischof
      von Alexandrien} zu kl�ren.
  \item[�berlieferung]Zur �berlieferung des Codex Veronensis
    vgl. Dok. \ref{ch:SerdicaEinl}. 

    Ein griechisches Original ist zu vermuten, aber nicht nachweisbar.
  \item[Fundstelle]Codex Veronensis LX f. 102a--103a
  \end{description}
\end{praefatio}
\begin{pairs}
\selectlanguage{latin}
\begin{Leftside}
% \beginnumbering
 \pstart
 \hskip -1.2em\edtext{\abb{}}{\killnumber\Cfootnote{\hskip
      -.6em Cod.\,Ver.}}\specialindex{quellen}{section}{Codices!Veronensis
    LX!f. 102a--103a} \kap{1}Sancta synodus secundum dei gratiam
  collecta Sardicae\edindex[namen]{Serdica} ecclesiis dei
  \edtext{aput}{\Dfootnote{apud \textit{coni. Maffei}}}
  \edtext{\abb{Mareotam}}{\Dfootnote{\textit{coni. Maffei} Maretam
      \textit{cod.}}}\edindex[namen]{Mareotis} cum presbyteris et
  diaconibus in domino salutem.  
\pend
\pstart
\kap{2}etiam ex his, fratres dilectissimi, 
\edtext{\abb{quae}}{\Dfootnote{\textit{add. Maffei}}} ad Alexandriam\edindex[namen]{Alexandrien}
per fratres directa sunt, scire potestis
quae 
\edtext{aput}{\Dfootnote{apud \textit{coni. Maffei}}} sanctam et 
\edtext{\abb{magnam}}{\Dfootnote{\textit{coni. Maffei} magna \textit{cod.}}} synodum secundum dei gratiam
Sardicae\edindex[synoden]{Serdica!a. 343} collectam sunt 
\edtext{\abb{actitata}}{\Dfootnote{\textit{coni. Maffei} actita \textit{cod.}}}. sed quia
et vos 
\edtext{\abb{scripsistis}}{\Dfootnote{\textit{cod.\corr} scriptis \textit{cod.*}}}
intolerabilia sustinuisse ab impiissimis 
\edtext{hereticis}{\Dfootnote{haerticis \textit{coni. Maffei}}}, quorum est princeps pessimus
Gregorius\edindex[namen]{Gregor!Bischof von Alexandrien}, hanc ob causam scribere et ad vestram reverentiam necessarium sancta
synodus 
\edtext{estimavit}{\Dfootnote{aestimavit \textit{coni. Maffei}}},
ut hiis consolati, magis 
\edtext{\abb{ac}}{\Dfootnote{\textit{coni. Maffei} agn \textit{cod.*} ag cod.\corr}} magis 
\edtext{\abb{habentes}}{\Dfootnote{\textit{coni. Maffei} habente \textit{cod.}}} in deo 
\edtext{\abb{promo\ladd{ti}tionem}}{\Dfootnote{\textit{coni. Opitz} promotitionem \textit{cod.} \responsio\ in
deo spem furturam promissionem repositam \textit{coni. Maffei}}}, spem futuram ac
repositam
diligentibus Christum consequamini. 
\pend
\pstart
\kap{3}Si igitur passi estis mala, nolite
contristari sed magis
gaudete, quoniam et vos meruistis pro nomine Domini iniurias tolerare. si vero
carceres et vincula
et 
\edtext{\abb{functiones}}{\Dfootnote{\textit{cod.} factiones \textit{coni. Maffei}
punitiones \textit{susp. Opitz}}} tolerastis, 
\edtext{haec}{\lemma{\abb{haec\ts{1}}}\Dfootnote{\textit{coni. Maffei} hec \textit{cod.}}} vos non contristabunt; 
\edtext{haec}{\lemma{\abb{haec\ts{2}}}\Dfootnote{\textit{coni. Maffei} hec \textit{cod.}}} enim et
ante vos patres sustinuerunt,
quorum unus est 
\edtext{\abb{beatus}}{\Dfootnote{\textit{cod.\corr} peatus \textit{cod.*}}} Paulus, propter quod et 
\edtext{\abb{vinctus}}{\Afootnote{vgl. Eph 3,1}}\edindex[bibel]{Epheser!3,1|textit}
vocatus. 
\pend
\pstart
\kap{4}audivimus quanta et Ingenius\edindex[namen]{Ingenius!Presbyter in Mareotis} 
presbyter passus est, et doluimus quidem propter
iniurias, libenter autem accepimus 
\edtext{\abb{sacram}}{\Dfootnote{\textit{coni. Maffei} sacra \textit{cod.}}} eius
voluntatem, quoniam propter Christum cuncta 
\edtext{\abb{sustinuit}}{\Dfootnote{\textit{coni. Maffei} suit \textit{cod.}}}. si
igitur adhuc vos premunt, 
\edtext{\abb{quae}}{\Dfootnote{\textit{coni. Maffei} que \textit{cod.}}} putatur
\edtext{tristitia esse in gaudium convertatur.}{\lemma{\abb{tristitia \dots\ convertatur}}\Afootnote{Io 16,20}}\edindex[bibel]{Johannes!16,20} 
\pend
\pstart
\kap{5}\edtext{scribsimus}{\Dfootnote{scripsimus \textit{coni. Maffei}}} enim piissimis imperatoribus ut ne de cetero talia
committantur adversum ecclesias; et credimus quod dominus faciet per 
\edtext{\abb{eorum}}{\Dfootnote{\textit{del. Maffei}}}\edlabel{eorum} religionem humanissimorum
imperatorum, ut et nos cum solatio et libertate deo gratias agentes et placentes
inveniamur in die iudicii. 
\pend
\pstart
\kap{6}\edtext{\abb{quae}}{\Dfootnote{\textit{coni. Maffei} que \textit{cod.}}} autem sunt actitata, sicut prediximus, 
\edtext{cognoscitis}{\Dfootnote{cognoscetis \textit{coni. Maffei}}} ex
dilectissimis fratribus nostris
qui vestras litteras portaverunt, hoc est presbyteris et diaconibus
Alexandrinis\edindex[namen]{Alexandrien}: 
\pend
\pstart

\pend
\pstart
\kap{7}episcopum enim vestrum, dilectissimum 
\edtext{fratrem et nostrum
conministrum}{\lemma{\abb{}}\Dfootnote{\responsio\ fratrem nostrum et conministrum
\textit{coni. Maffei}}}, Athanasium\edindex[namen]{Athanasius!Bischof von Alexandrien} innocentem et
sincerum ab omni
calumnia pronuntiavit sancta et magna synodus; 
\edtext{\abb{Theodorum}}{\Dfootnote{\textit{coni. Maffei} Thodorum \textit{cod.}}}\edindex[namen]{Theodorus!Bischof von Heraclea} vero Narcissum\edindex[namen]{Narcissus!Bischof von Neronias}
Stephanum\edindex[namen]{Stephanus!Bischof von Antiochien} Acacium\edindex[namen]{Acacius!Bischof von Caesarea} Georgium\edindex[namen]{Georg!Bischof von Laodicea}
Ursacium\edindex[namen]{Ursacius!Bischof von Singidunum} Valentem\edindex[namen]{Valens!Bischof von Mursa} et Minophantum\edindex[namen]{Menophantus!Bischof von Ephesus} episcopatu deposuit ob ea 
\edtext{\abb{quae}}{\Dfootnote{\textit{coni. Maffei} que \textit{cod.}}} deliquerunt et
ob impiissimam
\edtext{heresim}{\Dfootnote{haeresim \textit{coni. Maffei}}} cuius socii et 
\edtext{\abb{patroni}}{\Dfootnote{\textit{coni. Maffei} patrocini \textit{cod.}}}
videntur. 
\pend
\pstart
\kap{8}de Gregorio\edindex[namen]{Gregor!Bischof von Alexandrien} autem nec tantum credimus necessarium esse
scribere; olim enim depositus est, immo magis episcopus penitus non est
\edtext{estimatus}{\Dfootnote{aestimatus \textit{coni. Maffei}}} 
\edtext{quale enim eius opus}{\Dfootnote{eius enim opus \textit{coni. Maffei}}},
simile est eius ordinationi. si quis igitur 
\edtext{\abb{ab eo}}{\Dfootnote{\textit{coni. Maffei} habeo \textit{cod.}}} deceptus 
\edtext{\abb{est}}{\Dfootnote{\textit{coni. Maffei} es \textit{cod.}}}, erudiatur
et veritatem
cognoscat; 
\edtext{\abb{sin vero}}{\Dfootnote{\textit{coni. Opitz} si noluero \textit{cod.} si vero \textit{Maffei}}} 
\edtext{resistet}{\Dfootnote{resistit \textit{coni. Maffei}}} eius 
\edtext{\abb{inpietati}}{\Dfootnote{\textit{coni. Maffei} inpietate \textit{cod.}}}, gaudeat quod et ipse 
\edtext{adversatus est hunc}{\Dfootnote{aversatus est hunc \textit{susp. Opitz} adversatus est huic \textit{coni. Ballerini}}} quem sancta
synodus nec episcopum 
\edtext{estimavit}{\Dfootnote{aestimavit \textit{coni. Maffei}}}. nec enim nos latuit 
\edtext{\abb{quae}}{\Dfootnote{\textit{coni. Opitz} q' \textit{cod.} quid \textit{coni. Maffei}}} adversus vos
commiserit et quantum vos
presserit. Sed gaudete quoniam pro Christo patimini ab his qui Christum
blasphemant. 
\pend
\pstart
\kap{9}
\edtext{estimamus}{\Dfootnote{aestimamus \textit{coni. Maffei}}} autem quod iam omnis insolentia 
\edtext{\abb{cessabit}}{\Dfootnote{\textit{coni. Maffei} cessavit \textit{cod.}}}, increpatis ac depositis 
\edtext{\abb{noxiis}}{\Dfootnote{\textit{coni. Maffei} nox hiis \textit{cod.}}} qui 
\edtext{\abb{heresim}}{\Dfootnote{\textit{coni. Opitz} haeresim \textit{coni. Maffei} heresit \textit{cod.}}}
non nominandam
defendebant. 
\pend
\pstart
\kap{10}incolumes vos esse in domino opto.
\pend
\pstart
\noindent 1. Ego Osius\edindex[namen]{Ossius!Bischof von Cordoba} episcopus incolumes vos
in Domino 
\edtext{oro}{\Dfootnote{opto \textit{coni. Maffei}}}, dilectissimi fratres. 
\pend
\pstart
\noindent 2. Athanasius\edindex[namen]{Athanasius!Bischof von Alexandrien} episcopus vester
incolumes vos in Domino opto, dilectissimi fratres. 
\pend
\pstart
\noindent 3. Eliodorus\edindex[namen]{Heliodorus!Bischof von Nicopolis} similiter. 
\pend
\pstart
\noindent 4. Iohannes\edindex[namen]{Johannes!Bischof} similiter. 
\pend
\pstart
\noindent 5. Ionas\edindex[namen]{Jonas!Bischof von Parthicopolis} similiter. 
\pend
\pstart
\noindent 6. Dionisius\edindex[namen]{Dionysius!Bischof von Elis} similiter.
\pend
\pstart
\noindent 7. Paregorius\edindex[namen]{Paregorius!Bischof von Scupi} similiter. 
\pend
\pstart
\noindent 8. A"etius\edindex[namen]{A"etius!Bischof von Thessalonike} similiter. 
\pend
\pstart
\noindent 9. Valens\edindex[namen]{Valens!Bischof von Oescus}
\Ladd{\edtext{\abb{similiter}}{\Dfootnote{\textit{add. Maffei}}}}. 
\pend
\pstart
\noindent 10. Arrius\edindex[namen]{Arius!Bischof von Petra}
\Ladd{\edtext{\abb{similiter}}{\Dfootnote{\textit{add. Maffei}}}}. 
\pend
\pstart
\noindent 11. Porphyrius\edindex[namen]{Porphyrius!Bischof von Philippi} similiter. 
\pend
\pstart
\noindent 12. Athenodorus\edindex[namen]{Athenodorus!Bischof von Elatia} similiter. 
\pend
\pstart
\noindent 13. \edtext{\abb{Alypius}}{\Dfootnote{\textit{cod\corr} Alysius cod.*}}\edindex[namen]{Alypius!Bischof von Megara}
similiter.
\pend
\pstart
\noindent 14. Gerontius\edindex[namen]{Gerontius!Bischof von Beroea} similiter. 
\pend
\pstart
\noindent 15. Lucius\edindex[namen]{Lucius!Bischof von Adrianopel|dub}\edindex[namen]{Lucius!Bischof von Verona|dub} similiter.
\pend
\pstart
\noindent 16. Asterius\edindex[namen]{Asterius!Bischof in Arabien} similiter.
\pend
\pstart
\noindent 17. Basus\edindex[namen]{Bassus!Bischof von Diocletianopolis} similiter. 
\pend
\pstart
\noindent 18. Dioscurus\edindex[namen]{Dioscurus!Bischof von Therasia} similiter. 
\pend
\pstart
\noindent 19. Dometianus\edindex[namen]{Dometianus!Bischof von Acaria Constantias (?)}
similiter. 
\pend
\pstart
\noindent 20. \edtext{\abb{Calepodius}}{\Dfootnote{\textit{Erl.} Calapodius \textit{cod.}}}\edindex[namen]{Calepodius!Bischof von Neapolis} similiter. 
\pend
\pstart
\noindent 21. Alexander\edindex[namen]{Alexander!Bischof von Corone|dub}\edindex[namen]{Alexander!Bischof von Cyparissia|dub}\edindex[namen]{Alexander!Bischof von Larisa|dub} similiter. 
\pend
\pstart
\noindent 22. Plutarchus\edindex[namen]{Plutarchus!Bischof von Patras} similiter. 
\pend
\pstart
\noindent 23. Vincentius\edindex[namen]{Vincentius!Bischof von Capua} similiter. 
\pend
\pstart
\noindent 24. Vitalis\edindex[namen]{Vitalis!Bischof von Aquae|dub}\edindex[namen]{Vitalis!Bischof ">Vertaresis"<|dub} similiter. 
\pend
\pstart
\noindent 25. Severus\edindex[namen]{Severus!Bischof von Chalcis|dub}\edindex[namen]{Severus!Bischof von Ravenna|dub} similiter.
\pend
\pstart
\noindent 26. Restutus\edindex[namen]{Restutus!Bischof} similiter. 
\pend
\pstart
\noindent 27. Vincentius\edindex[namen]{Vincentius!Bischof} episcopus incolumes vos
in Domino opto, 
\edtext{\abb{dilectissimi}}{\Dfootnote{\textit{coni. Maffei} dilectissimis \textit{cod.}}} fratres: iussus
a fratribus meis et
coepiscopis scripsi et subscripsi pro ceteris. 
\pend
% \endnumbering
\end{Leftside}
\begin{Rightside}
\begin{translatio}
\beginnumbering
\pstart
\kapR{1}Die heilige Synode, durch Gottes Gnade in Serdica versammelt, gr��t die Kirchen in
der Mareotis mit ihren Presbytern und Diakonen im Herrn. 
\pend
\pstart
\kapR{2}Auch aus dem, vielgeliebte
Br�der, was durch die Br�der nach Alexandrien geschickt wurde,\footnoteA{Vgl. Dok. \ref{sec:BriefSerdikaAlexandrien}.} k�nnt ihr Kenntnis davon
haben, was auf der heiligen und gro�en Synode, die nach Gottes Gnade in Serdica stattfand,
verhandelt worden ist. Aber weil auch ihr geschrieben habt, ihr h�ttet
Unertr�gliches ausgehalten von den gottlosesten H�retikern, deren schlimmster F�hrer
Gregor ist, deshalb hielt es die heilige Synode f�r notwendig, auch an euer Ehrw�rden zu schreiben, 
damit ihr dadurch getr�stet mehr und mehr in Gott
fortschreitet und die k�nftige Hoffnung erlangt, die f�r jene, die Christus lieben,
aufgespart ist. 
\pend
\pstart
\kapR{3}Wenn ihr also �bel erlitten habt, seid nicht betr�bt, sondern vielmehr erfreut,
weil auch ihr es verdient habt, f�r den Namen des Herrn Ungerechtigkeiten
zu ertragen. Wenn ihr aber Kerker, Fesseln und Todesdrohungen ertragen habt,
soll euch dies nicht betr�ben; dies haben n�mlich auch
vor euch die V�ter erduldet, von denen einer der selige Paulus ist, der deswegen auch
Gefesselter genannt wird. 
\pend
\pstart
\kapR{4}Wir haben geh�rt, wie viel besonders der Presbyter Ingenius\footnoteA{Sonst nicht bekannt.} erlitten
hat, und haben freilich wegen der Ungerechtigkeiten Schmerzen empfunden, aber gerne seine
heilige Entschlossenheit vernommen, weil er alles um Christi Willen aushielt. Wenn sie euch also
bis jetzt bedr�ngen, so soll sich dies, was man f�r traurig h�lt, in Freude verwandeln. 
\pend
\pstart
\kapR{5}Wir haben n�mlich an die fr�mmsten Kaiser geschrieben, damit nicht
weiterhin solche Anschl�ge gegen die Kirchen unternommen werden. Und wir glauben, da� der Herr
es durch die Gottesfurcht dieser so menschenfreundlichen Herrscher einrichten wird, da�
auch wir am Tag des Gerichts getr�stet und in Freiheit als Gott Dankende
und ihm Wohlgef�llige erfunden werden.
\pend
\pstart
\kapR{6}Was aber verhandelt wurde, erfahrt ihr, wie wir vorher bereits gesagt
haben\footnoteA{S.\,o. � 2.}, von unseren vielgeliebten Br�dern, die die Briefe
an euch �berbracht haben, das
hei�t von den Presbytern und Diakonen aus Alexandrien. 
\pend
\pstart

\pend
\pstart
\kapR{7}Euren Bischof n�mlich, den vielgeliebten Bruder und unseren Mitdiener Athanasius,
hat die heilige und gro�e Synode f�r unschuldig und rein von allem betr�gerischem Vorwurf
erkl�rt. Theodorus aber, Narcissus, Stephanus, Acacius, Georg, Ursacius, Valens und
Menophantus\footnoteA{Vgl. Dok. \ref{sec:SerdicaRundbrief},14; \ref{sec:B},6.} hat sie ihres
Bischofsamtes enthoben aufgrund dessen, was sie verbrochen haben und wegen der h�chst
gottlosen H�resie, als deren Anh�nger und Schirmherren sie angesehen wurden. 
\pend
\pstart
\kapR{8}�ber Gregor\footnoteA{Vgl. Dok.
\ref{sec:BriefSerdikaAlexandrien},16.} jedoch brauchen wir unserer Meinung nach 
keine Worte verlieren;
er ist n�mlich bereits vor einiger Zeit abgesetzt worden, eigentlich wurde er nie
wirklich als Bischof angesehen. Er hat n�mlich genau so gehandelt, wie er damals in sein Amt
eingef�hrt wurde. Wenn
also jemand von ihm get�uscht worden ist, soll er aufgekl�rt werden und die Wahrheit erkennen.
Wenn jemand aber dessen Gottlosigkeit widerstanden haben sollte, m�ge er sich freuen, weil
er selbst schon sich gegen den gewendet hat, den die heilige Synode ebenfalls nicht als
Bischof ansieht. Es war uns n�mlich nicht verborgen, was er gegen euch
unternommen und wie sehr er euch unterdr�ckt hat. Doch seid froh dar�ber, weil ihr ja f�r
Christus unter denen leidet, die Christus l�stern. 
\pend
\pstart
\kapR{9}Wir sch�tzen aber, da� jetzt alle Unversch�mtheit aufh�ren wird, nachdem die
Schuldigen getadelt und abgesetzt worden sind, die die uns�gliche H�resie
stets verteidigt haben. 
\pend
\pstart
\kapR{10}Ich w�nsche, da� ihr wohlbehalten im Herrn seid. 
\pend
\pstart
\noindent 1. Ich, Bischof Ossius bete, da� ihr wohlbehalten im Herrn seid, geliebte
Br�der.
\pend
\pstart
\noindent 2. Ich, euer Bischof Athanasius w�nsche, da� ihr wohlbehalten im Herrn seid,
geliebte Br�der.
\pend
\pstart
\noindent 3. Heliodorus (von Nicopolis) ebenso.
\pend
\pstart
\noindent 4. Johannes ebenso.
\pend
\pstart
\noindent 5. Jonas (von Parthicopolis) ebenso.
\pend
\pstart
\noindent 6. Dionysius (von Elis) ebenso.
\pend
\pstart
\noindent 7. Paregorius (von Scupi) ebenso.
\pend
\pstart
\noindent 8. A"etius (von Thessalonike) ebenso.
\pend
\pstart
\noindent 9. Valens (von Oescus) ebenso.
\pend
\pstart
\noindent 10. Arius (von Petra) ebenso.
\pend
\pstart
\noindent 11. Porphyrius (von Philippi) ebenso.
\pend
\pstart
\noindent 12. Athenodorus (von Elatia) ebenso.
\pend
\pstart
\noindent 13. Alypius (von Megara) ebenso.
\pend
\pstart
\noindent 14. Gerontius (von Beroea) ebenso.
\pend
\pstart
\noindent 15. Lucius ebenso.
\pend
\pstart
\noindent 16. Asterius (aus Arabien) ebenso.
\pend
\pstart
\noindent 17. Bassus (von Diocletianopolis) ebenso.
\pend
\pstart
\noindent 18. Dioscurus (von Therasia) ebenso.
\pend
\pstart
\noindent 19. Dometianus (von Acaria Constantias?) ebenso.
\pend
\pstart
\noindent 20. Calepodius (von Neapolis) ebenso.
\pend
\pstart
\noindent 21. Alexander ebenso.
\pend
\pstart
\noindent 22. Plutarchus (von Patras) ebenso.
\pend
\pstart
\noindent 23. Vincentius (von Capua) ebenso.
\pend
\pstart
\noindent 24. Vitalis ebenso.
\pend
\pstart
\noindent 25. Severus ebenso. 
\pend
\pstart
\noindent 26. Restutus ebenso.
\pend
\pstart
\noindent 27. Ich, Bischof Vincentius w�nsche euch wohlbehalten im Herrn, vielgeliebte
Br�der:
Beauftragt von meinen Br�dern und Mitbisch�fen habe ich geschrieben und f�r die �brigen
unterschrieben.
\pend
\endnumbering
\end{translatio}
\end{Rightside}
\Columns
\end{pairs}
% \thispagestyle{empty}
\section[Brief des Athanasius von Alexandrien an Presbyter, Diakone und Kirchenvolk in der Mareotis][Brief des Athanasius von Alexandrien in die Mareotis]{Brief des Athanasius von Alexandrien an Presbyter, Diakone\\und Kirchenvolk in der Mareotis}
% \label{sec:43.2}
\label{sec:BriefAthMareotis}
% \diskussionsbedarf
\begin{praefatio}
  \begin{description}
  \item[Herbst 343] Zum Datum vgl. Einleitung zu
    Dok. \ref{ch:SerdicaEinl}.
  \item[�berlieferung]Die Namensliste S. \refpassage{Namen1}{Namen2}
    ist offensichtlich das Ergebnis einer Kompilation zweier
    urspr�nglich voneinander unabh�ngiger Namenslisten, da zum einen
    bei den Namen Nr. 3--18 (1--19) die Form der Unterschrift
    signifikant von der bei den anderen differiert (keine Nennung der
    Herkunftsorte und keine Formel \textit{incolumes vos opto} o.\,�.,
    sondern nur jeweils \textit{similiter}) und zum anderen die Namen
    Nr. 3, 5, (9), 13, 15, 16 und 18 sp�ter noch ein zweites Mal
    aufgef�hrt werden (Nr. 29, 47, [32/51], 31, 40, 54 und 37). Den
    �bergang von der ersten zur zweiten Liste scheint der Eintrag des
    Maximinus von Trier (Nr. 19) zu bilden, da vermerkt wird, da�
    diese Unterschrift brief"|lich (\textit{per epistulas})
    �bermittelt wurde, demnach also erst nachtr�glich angef�gt worden
    sein kann.

    Zur �berlieferung des Codex Veronensis
    vgl. Dok. \ref{ch:SerdicaEinl}. 

    Die Namen sind sehr schlecht �berliefert; f�r eine griechische
    Vorlage spricht die Verwechslung von D und L (\griech{D} und
    \griech{L}) in der Handschrift beim Namen Nr. 14.
  \item[Fundstelle]Codex Veronensis LX f. 103a--105a
  \end{description}
\end{praefatio}
\begin{pairs}
\selectlanguage{latin}
\begin{Leftside}
% \beginnumbering
\pstart
\hskip -1.3em\edtext{\abb{}}{\killnumber\Cfootnote{\hskip -.65em
Cod.Ver. (cod.)}}\specialindex{quellen}{section}{Codices!Veronensis LX!f. 103a--105a}
\kap{1}Athanasius\edindex[namen]{Athanasius!Bischof von Alexandrien} presbyteris et diaconibus et populo 
\edtext{\abb{catholicae}}{\Dfootnote{\textit{coni. Maffei} catholice \textit{cod.}}} ecclesiae aput
Mareotam\edindex[namen]{Mareotis} dilectissimis ac
desiderabilibus fratribus 
\edtext{in deo}{\Dfootnote{in domino \textit{coni. Maffei}}} salutem.
\pend
\pstart
\kap{2}\edtext{\abb{sancta}}{\Dfootnote{\textit{coni. Maffei} sanctam \textit{cod.}}} synodus laudavit in
Christo vestram religionem. omnes acceptos tulerunt in omnibus animum et
\edtext{\abb{fortitudinem}}{\Dfootnote{\textit{coni. Maffei} fortitudinam \textit{cod.}}}, quoniam minas non
timuistis, quod 
\edtext{\abb{tolerantes}}{\Dfootnote{\textit{coni. Maffei} tollerantes \textit{coni. Opitz} tellerantes \textit{cod.}}} iniurias et
persecutiones adversum pietatem prevaluistis. 
\edtext{\abb{litterae}}{\Dfootnote{\textit{coni. Maffei} littere \textit{cod.}}} itaque 
\edtext{\abb{vestrae}}{\Dfootnote{\textit{coni. Maffei} vestre \textit{cod.}}} dum legerentur
omnibus lacrimas commoverunt et omnes ad vestrum 
\edtext{protraxerunt}{\Dfootnote{pertaxerunt \textit{coni. Maffei}}} affectum;
dilexerunt vos et absentes, 
\edtext{\abb{ac}}{\Dfootnote{\textit{coni. Maffei} hac \textit{cod.}}} vestras persecutiones suas
\edtext{estimaverunt}{\Dfootnote{aestimaverunt \textit{coni. Maffei}}}.
\pend
\pstart
\kap{3}\edtext{indicio namque caritatis eorum sunt}{\lemma{indicio \dots\ sunt}
\Dfootnote{indicium \dots\ sunt \textit{coni. Maffei}}} 
\edtext{\abb{litterae}}{\Dfootnote{\textit{coni. Maffei} littere \textit{cod.}}} ad vos 
\edtext{\abb{datae}}{\Dfootnote{\textit{coni. Maffei} date \textit{cod.}}}; et
licet 
\edtext{\abb{sufficeret}}{\Dfootnote{\textit{coni. Maffei} sufficere \textit{cod.}}} vos
connumerare sanctae 
\edtext{\abb{per Alexandriam}}{\Dfootnote{\textit{coni. Maffei} per Alexandria
\textit{cod.}}}\edindex[namen]{Alexandrien} ecclesiae,
tamen 
\edtext{\abb{separatim}}{\Dfootnote{\textit{coni. Maffei} separatum \textit{cod.}}} vobis scripsit sancta
synodus, ut 
\edtext{\abb{adhortati}}{\Dfootnote{\textit{coni. Maffei} adortati \textit{cod.}}} non deficiatis ob 
\edtext{\abb{haec quae}}{\Dfootnote{\textit{coni. Maffei} hec que \textit{cod.}}} patiamini, sed gratias agatis domino
quod vestra patientia bonum fructum habebit. 
\pend
\pstart
\kap{4}olim itaque latebant 
\edtext{hereticorum}{\Dfootnote{haereticorum \textit{coni. Maffei}}} mores, nunc
tamen omnibus expansi sunt et patefacti: nam 
\edtext{\abb{sancta}}{\Dfootnote{\textit{coni. Maffei} sancte \textit{cod.}}} synodus advertit ab his
concinnatas adversus vos calumnias et eos habuit odio,
\edtext{adque}{\Dfootnote{atque \textit{coni. Maffei}}} omnium 
\edtext{\abb{consensu}}{\Dfootnote{\textit{coni. Maffei} consensum \textit{cod.}}} deposuit 
\edtext{\abb{Theodorum}}{\Dfootnote{\textit{coni. Maffei} Theoderum \textit{cod.}}}\edindex[namen]{Theodorus!Bischof von Heraclea} Valentem\edindex[namen]{Valens!Bischof von Mursa}
Ursacium\edindex[namen]{Ursacius!Bischof von Singidunum}: 
\Ladd{\edtext{\abb{quae enim}}{\Dfootnote{\textit{add. Opitz}; \textit{lac. suspiciunt Ballerini}}}} in Alexandria\edindex[namen]{Alexandrien} et
Mareota\edindex[namen]{Mareotis}, eadem etiam per alias ecclesias facta sunt. et quoniam 
\edtext{\abb{intolerabilis}}{\Dfootnote{\textit{coni. Maffei} intollerabilis \textit{cod.}}} est iam
crudelitas eorum et tyrannia adversus ecclesias celebrata, ideo episcopatu deiecti sunt 
\edtext{omnique}{\Dfootnote{omniumque \textit{susp. Opitz}}} communione alienati.
ceterum de Gregorio\edindex[namen]{Gregor!Bischof von Alexandrien} nec mentionem facere voluerunt: qui enim penitus episcopi
nomen nec habuit, hunc
nominare superfluum 
\edtext{estimaverunt}{\Dfootnote{aestimaverunt \textit{coni. Maffei}}}; sed propter 
\edtext{\abb{deceptos}}{\Dfootnote{\textit{coni. Maffei} decepto \textit{cod.}}} 
\edtext{\abb{ab eo}}{\Dfootnote{\textit{coni. Maffei} habeo \textit{cod.}}} 
\edtext{\abb{nominis}}{\Dfootnote{\textit{coni. Maffei} nomines \textit{cod.}}} eius mentionem fecerunt, non
quia dignus memoria videbatur, sed ut ex hoc qui ab illo decepti sunt eius
cognoscant infamiam et
erubescant 
\edtext{\abb{cuiusmodi}}{\Dfootnote{\textit{coni. Maffei} cuiusmodis \textit{cod.}}} factis homini
communicaverunt.
\pend
\pstart
\skipnumbering
\pend
\pstart
\skipnumbering
\pend
\pstart
\kap{5}\edtext{cognoscitis}{\Dfootnote{cognoscetis \textit{coni. Maffei}}} vero super eos
scripta ex 
\edtext{\abb{superadnexis}}{\Dfootnote{superadnexis \textit{coni. Maffei} superatnexis \textit{cod.}}}; et licet non
omnes scribere episcopi 
\edtext{\abb{occurrerunt}}{\Dfootnote{\textit{coni. Maffei} occurerunt \textit{cod.}}}, 
\edtext{\abb{attamen}}{\Dfootnote{\textit{coni. Maffei} adtamen \textit{cod.}}} 
\edtext{\abb{quae}}{\Dfootnote{\textit{add. Opitz}}} ab omnibus scripta sunt 
\edtext{pro}{\Dfootnote{et pro \textit{coni. Maffei}}}
omnibus scripserunt. 
\pend
\pstart
\kap{6}\edtext{invicem salutate in
osculo sancto salutant vos omnes fratres}{\lemma{\abb{invicem \dots\ fratres}}\Afootnote{vgl. Rom 16,16 u.\,�.}}\edindex[bibel]{Roemer!16,16|textit}.
\pend
\pstart
\noindent 1.\edlabel{Namen1} Protogenes\edindex[namen]{Protogenes!Bischof von Serdica} episcopus incolumes
vos in domino opto, dilectissimi et
desiderabiles. 
\pend
\pstart
\noindent 2. Athenodorus\edindex[namen]{Athenodorus!Bischof von Elatia} episcopus
incolumes vos in domino opto, fratres dilectissimi. 
\pend
\pstart
\noindent 3. Iulianus\edindex[namen]{Julianus!Bischof von Thebae Heptapylus|dub} episcopus
similiter.
\pend
\pstart
\noindent 4. Ammonius\edindex[namen]{Ammonius!Bischof} similiter. 
\pend
\pstart
\noindent 5. \edtext{\abb{Ap\ladd{a}rianus}}{\lemma{\abb{}}\Dfootnote{Aprianus \textit{coni. Erl.} Aparianus \textit{cod.}}}\edindex[namen]{Aprianus!Bischof von Poetovio|dub} similiter 
\pend
\pstart
\noindent 6. \edtext{Mar\Ladd{c}ellus}{\lemma{\abb{Marcellus}}\Dfootnote{\textit{coni. Maffei} Marellus
\textit{cod.}}}\edindex[namen]{Markell!Bischof von Ancyra} similiter. 
\pend
\pstart
\noindent 7. Gerontius similiter\edindex[namen]{Gerontius!Bischof von Beroea}. 
\pend
\pstart
\noindent 8. Porphyrius\edindex[namen]{Porphyrius!Bischof von Philippi} similiter. 
\pend
\pstart
\noindent 9. Zosimus\edindex[namen]{Zosimus!Bischof|dub}\edindex[namen]{Zosimus!Bischof von Lychnidus|dub}\edindex[namen]{Zosimus!Bischof von Horreum Margi|dub} similiter. 
\pend
\pstart
\noindent 10. Asclepius\edindex[namen]{Asclepas!Bischof von Gaza} similiter. 
\pend
\pstart
\noindent 11. Appianus\edindex[namen]{Appianus!Bischof} similiter.
\pend
\pstart
\noindent 12. Eulogius\edindex[namen]{Eulogius!Bischof} similiter.
\pend
\pstart
\noindent 13. Eugenius\edindex[namen]{Eugenius!Bischof von Heraclea Lyncestis|dub} similiter. 
\pend
\pstart
\noindent 14. \edtext{\abb{Diodorus}}{\Dfootnote{\textit{susp. Maffei} Liodorus \textit{cod.}}}\edindex[namen]{Diodorus!Bischof von Tenedus} similiter. 
\pend
\pstart
\noindent 15. Martyrius\edindex[namen]{Martyrius!Bischof von Naupactus|dub} similiter.
\pend
\pstart
\noindent 16. Eucarpus\edindex[namen]{Eucarpus!Bischof von Opus|dub} similiter. 
\pend
\pstart
\noindent 17. Lucius\edindex[namen]{Lucius!Bischof von Adrianopel} similiter. 
\pend
\pstart
\noindent 18. \edtext{\abb{Calvus}}{\Dfootnote{\textit{susp. Maffei} Caloes \textit{cod.} Galba \textit{susp. Ballerini}}}\edindex[namen]{Calvus!Bischof von Castra Martis|dub} similiter. 
\pend
\pstart
\noindent 19. \edtext{\abb{Maxim\Ladd{in}us}}{\Dfootnote{\textit{susp. Maffei} Maximus \textit{cod.}}}\edindex[namen]{Maximinus!Bischof von Trier} similiter per epistulas de
Galliis\edindex[namen]{Gallien} incolumes vos in domino opto,
dilectissimi. 
\pend
\pstart
\noindent 20. \edtext{Arc\Ladd{h}idamus}{\lemma{\abb{Archidamus}}\Dfootnote{\textit{coni. Erl.} Arcidamus \textit{cod.}}}\edindex[namen]{Archidamus!Presbyter in Rom} et
Philoxenus\edindex[namen]{Philoxenus!Presbyter in Rom} 
presbyteri et Leo\edindex[namen]{Leo!Diakon in Rom} diaconus de Roma incolumes vos
optamus.
\pend
\pstart
\noindent 21. Gaudentius\edindex[namen]{Gaudentius!Bischof von Na"issus} Na"isitanus
episcopus incolumes vos in domino opto. 
\pend
\pstart
\noindent 22. Florentius\edindex[namen]{Florentius!Bischof von Emerita Augusta} \edtext{\Ladd{E}meri\Ladd{ta}e}{\lemma{\abb{Emeritae}}\Dfootnote{\textit{susp. Maffei} Merie \textit{cod.}}}
\ladd{\edtext{\abb{Panoniae}}{\Dfootnote{\textit{del. Erl.} Hispaniae \textit{susp. Maffei} Spaniae \textit{susp. Ballerini}}}} similiter.
\pend
\pstart
\noindent 23. \edtext{\abb{Annianus}}{\Dfootnote{\textit{coni. Erl.} Ammianus \textit{cod.} Anianus \textit{susp. Maffei}}}\edindex[namen]{Annianus!Bischof von Castellona} de \edtext{Castello\Ladd{na}
\ladd{Pannonie}}{\lemma{\abb{Castellona}}\Dfootnote{\textit{coni. Erl.} Castellona Spaniae \textit{susp. Ballerini} Castello Pannonie \textit{cod.}}} similiter. 
\pend
\pstart
\noindent 24. Ianuarius\edindex[namen]{Januarius!Bischof von Beneventum} de 
\edtext{\abb{Benevento}}{\Dfootnote{\textit{susp. Maffei} Venevento \textit{cod.}}} similiter.
\pend
\pstart
\noindent 25. Praetextatus\edindex[namen]{Praetextatus!Bischof von Barcilona} 
\edtext{de \edtext{\abb{Barcilona}}{\Dfootnote{\textit{coni. Erl.} Narcidono \textit{cod.}}}
\ladd{\edtext{Pan\oline{n}}{\Dfootnote{\textit{del. Erl.}}}}}{\lemma{de Barcilona \ladd{Pan\oline{n}}} \Dfootnote{ab Hispania de Barcinona \textit{susp. Maffei} de Barcinone Spaniae \textit{susp. Ballerini}}} similiter. 
\pend
\pstart
\noindent 26. \edtext{\abb{Hymenaeus}}{\Dfootnote{\textit{susp. Ballerini} Hypeneus \textit{cod.} Hypeneris \textit{coni. Maffei}}}\edindex[namen]{Hymenaeus!Bischof von Hypata} de 
\edtext{\abb{Hypata}}{\Dfootnote{\textit{coni. Maffei} Hypato \textit{cod.}}}
Thessaliae similiter. 
\pend
\pstart
\noindent 27. Castus\edindex[namen]{Castus!Bischof von Caesaraugusta} de 
\edtext{\abb{Caesaraugusta}}{\Dfootnote{\textit{susp. Maffei} Caesarea Augusta \textit{susp. Ballerini} Augusta Caesareae \textit{coni. Maffei} Augusto Cesareae \textit{cod.}}}
similiter. 
\pend
\pstart
\noindent 28. \edtext{\abb{Severus}}{\Dfootnote{\textit{coni. Maffei} Severo
\textit{cod.}}}\edindex[namen]{Severus!Bischof von Chalcis} de 
\edtext{\abb{Chalcide}}{\Dfootnote{\textit{susp. Ballerini} Calciso \textit{cod.}}} Thessaliae
similiter. 
\pend
\pstart
\noindent 29. Iulianus\edindex[namen]{Julianus!Bischof von Thebae Heptapylus} de \edtext{\abb{Thebis Eptapylis}}{\Dfootnote{\textit{coni. Erl.} Thebis eptapolis \textit{cod.} Theriseptapoli \textit{coni. Maffei} Thebis Heptapyli \textit{susp. Ballerini}}} similiter. 
\pend
\pstart
\noindent 30. Lucius\edindex[namen]{Lucius!Bischof von Verona} de Verona similiter. 
\pend
\pstart
\noindent 31. \edtext{Eug\ladd{r}enius}{\lemma{\abb{Eugenius}}\Dfootnote{\textit{coni. Maffei} Eugrenius
\textit{cod.}}}\edindex[namen]{Eugenius!Bischof von Heraclea Lyncestis} de \edtext{He\Ladd{ra}clea}{\lemma{\abb{Heraclea}}\Dfootnote{\textit{susp. Ballerini} Hecleal \textit{coni. Maffei} Heclea \textit{cod.}}} \edtext{\abb{Lyncestis}}{\Dfootnote{\textit{coni. Erl.} Cycbinis \textit{coni. Maffei} Cychinis \textit{coni. Ballerini} Laoici \textit{susp. Ballerini} Lychynis \textit{cod.}}}
similiter. 
\pend
\pstart
\noindent 32. Zosimus \edtext{Lychni\Ladd{du}s}{\lemma{\abb{Lychnidus}}\Dfootnote{\textit{coni. Erl.} Lychnis \textit{cod.}}}\edindex[namen]{Zosimus!Bischof von
  Lychnidus} \Ladd{\edtext{\abb{similiter}}{\Dfootnote{\textit{coni. Erl.}}}}. 
\pend
\pstart
\noindent 33. \edtext{\abb{Stercorius de Canusio}}{\Dfootnote{\textit{susp. Ballerini} Sunosio \textit{cod.}}} Apuliae\edindex[namen]{Stercorius!Bischof von Canusium}
similiter.
\pend
\pstart
\noindent 34. Hermogenes\edindex[namen]{Hermogenes!Bischof von Sicyon} de \edtext{Syceono}{\Dfootnote{Sicyone \textit{susp. Ballerini}}}
similiter. 
\pend
\pstart
\noindent 35. Thrypho\edindex[namen]{Tryphon!Bischof von Macaria} de \edtext{\abb{Macaria}}{\Dfootnote{\textit{coni. Erl.} Magara \textit{cod.}}} similiter. 
\pend
\pstart
\noindent 36. Paregorius\edindex[namen]{Paregorius!Bischof von Scupi} \edtext{\abb{Scupinus}}{\Dfootnote{\textit{susp. Ballerini} Capinus \textit{cod.} Caspinus \textit{coni. Maffei} de Scupis \textit{susp. Maffei}}} similiter. 
\pend
\pstart
\noindent 37. \edtext{\abb{Calvus}}{\Dfootnote{\textit{susp. Maffei} Caloes \textit{cod.}}}\edindex[namen]{Calvus!Bischof von Castra Martis} \Ladd{\edtext{\abb{de}}{\Dfootnote{\textit{coni. Erl.}}}} \edtext{\abb{Castra Martis}}{\Dfootnote{\textit{coni. Erl.} Castromartis \textit{cod.}}}
similiter.
\pend
\pstart
\noindent 38. Ireneus\edindex[namen]{Irenaeus!Bischof von Scyrus} \edtext{\abb{Scyrus}}{\Dfootnote{\textit{coni. Erl.} Syconeus \textit{cod.} Secoreus \textit{vel} de Secoro \textit{susp. Maffei} de Scoro \textit{vel} de Scyro \textit{susp. Ballerini}}} similiter. 
\pend
\pstart
\noindent 39. Macedonius\edindex[namen]{Macedonius!Bischof von Ulpiana} \edtext{\abb{\Ladd{U}l\ladd{y}pianensis}}{\lemma{\abb{Ulpianensis}}\Dfootnote{\textit{susp. Maffei} Lypianensis \textit{cod.}}}
similiter. 
\pend
\pstart
\noindent 40. Martyrius\edindex[namen]{Martyrius!Bischof von Naupactus} Naupactis similiter.
\pend
\pstart
\noindent 41. Palladius\edindex[namen]{Palladius!Bischof von Dium} de Diu similiter.
\pend
\pstart
\noindent 42. \edtext{\abb{Verissimus}}{\Dfootnote{\textit{susp. Ballerini} Broseus \textit{cod.}}}\edindex[namen]{Verissimus!Bischof von Lugdunum} \edtext{\abb{Lugdunensis}}{\Dfootnote{\textit{susp. Maffei} Ludonensis \textit{cod.}}}
Galliae similiter. 
\pend
\pstart
\noindent 43. Ursacius\edindex[namen]{Ursacius!Bischof von Brixia} Brixensis similiter.
\pend
\pstart
\noindent 44. Amantius\edindex[namen]{Amantius!Bischof von Viminacium} 
\edtext{Viminacensus}{\Dfootnote{Viminacensis \textit{susp. Ballerini}}} per presbyterum
Maximum\edindex[namen]{Maximus!Presbyter in Viminacium} similiter. 
\pend
\pstart
\noindent 45. Alexander\edindex[namen]{Alexander!Bischof von Cyparissia} \edtext{\abb{Cyparissensis}}{\Dfootnote{\textit{susp. Ballerini} Gyparensis \textit{cod.}}}
Achaiae similiter.
\pend
\pstart
\noindent 46. Eutychius\edindex[namen]{Eutychius!Bischof von Methone} de \edtext{\abb{Mothona}}{\Dfootnote{\textit{cod.} Methona \textit{susp. Ballerini}}}
similiter. 
\pend
\pstart
\noindent 47. Aprianus\edindex[namen]{Aprianus!Bischof von Poetovio} de Petabione
Pannoniae
similiter.
\pend
\pstart
\noindent 48. Antigonus\edindex[namen]{Antigonus!Bischof von Pallene} \edtext{Pallensis}{\Dfootnote{Pellensis \textit{susp. Ballerini}}}
Macedoniae similiter. 
\pend
\pstart
\noindent 49. \edtext{\abb{Dometi\Ladd{an}us}}{\Dfootnote{\textit{coni. Erl.} Dometius \textit{cod.}}}\edindex[namen]{Dometianus!Bischof von Acaria Constantias (?)} de \edtext{Acaria}{\Dfootnote{\textit{an legendum} Castra \textit{?}}}
Constantias similiter. 
\pend
\pstart
\noindent 50. Olympius\edindex[namen]{Olympius!Bischof von Aenus} de \edtext{\abb{\Ladd{A}eno
Rodope\Ladd{s}}}{\lemma{\abb{Aeno Rhodopes}}\Dfootnote{\textit{susp. Opitz} Doliche \textit{susp. Maffei}}} similiter. 
\pend
\pstart
\noindent 51. Zosimus\edindex[namen]{Zosimus!Bischof von Horreum Margi} Orreomargensis
similiter. 
\pend
\pstart
\noindent 52. Protasius\edindex[namen]{Protasius!Bischof von Mailand} Mediolanensis
similiter. 
\pend
\pstart
\noindent 53. Marcus\edindex[namen]{Marcus!Bischof von Siscia} \edtext{Sis\ladd{i}censis}{\lemma{\abb{Siscensis}}\Dfootnote{\textit{coni. Erl.} Sisicensis \textit{cod.}}}
Saviae similiter. 
\pend
\pstart
\noindent 54. Eucarpus\edindex[namen]{Eucarpus!Bischof von Opus} 
\edtext{\abb{Opuntius}}{\Dfootnote{\textit{susp. Ballerini} Oponsius \textit{cod.}}}
Achaiae similiter.
\pend
\pstart
\noindent 55. Vitalis\edindex[namen]{Vitalis!Bischof ">Vertaresis"<} 
\dag\edtext{Vertaresis}{\Dfootnote{Vertarensis \textit{coni. Opitz} Bartanensis \textit{coni. Morcelli}}}\dag Africae similiter. 
\pend
\pstart
\noindent 56. Helianus\edindex[namen]{Aelianus!Bischof von Gortyna} de \edtext{\abb{Gortyna}}{\Dfootnote{\textit{coni. Feder} Tyrtanis \textit{cod.}}}
similiter. 
\pend
\pstart
\noindent 57. Synphorus\edindex[namen]{Symphorus!Bischof von Hierapytna} de 
\edtext{\abb{Hierapytna}}{\Dfootnote{\textit{coni. Erl.} Herapythis \textit{cod.} Hierapythnis \textit{susp. Opitz} Jerapytnis \textit{susp. Ballerini}}}
Cretae similiter. 
\pend
\pstart
\noindent 58. \edtext{Musonius}{\Dfootnote{\textit{coni. Erl.} Mosinius \textit{cod.}}}\edindex[namen]{Musonius!Bischof von Heraclium} Heracleae
similiter. 
\pend
\pstart
\noindent 59. Eucissus\edindex[namen]{Eucissus!Bischof von Cissamus} 
\edtext{\abb{Cissamensis}}{\Dfootnote{\textit{coni. Erl.} Chisamensus \textit{cod.} Chisamensis \textit{susp. Maffei}}} similiter.
\pend
\pstart
\noindent 60. Cydonius\edindex[namen]{Cydonius!Bischof von Cydonia} Cydonensis.\edlabel{Namen2}
\pend
% \endnumbering
\end{Leftside}
\begin{Rightside}
\begin{translatio}
\beginnumbering
\pstart
\noindent\kapR{1}Athanasius gr��t die Presbyter und Diakone und das Volk der katholischen Kirche
in der Mareotis, die vielgeliebten und ersehnten Br�der in Gott.
\pend
\pstart
\kapR{2}Die heilige Synode hat eure Gottesfurcht in Christus gelobt. Alle haben berichtet,
ihr h�ttet Mut und Tapferkeit in allen Dingen an den Tag gelegt, da ihr die
Drohungen nicht gef�rchtet habt und stark geblieben seid, obwohl ihr 
Ungerechtigkeiten und Nachstellungen gegen die Fr�mmigkeit erlitten habt.
Deshalb haben eure Briefe, als sie verlesen wurden, 
alle zu Tr�nen ger�hrt und bei allen Zuneigung zu euch geweckt.
Sie liebten euch auch als Abwesende und sahen eure Verfolgungen als die Ihren an. 
\pend
\pstart
\kapR{3}Zum Zeichen ihrer Liebe ist n�mlich ihr Brief an euch beigegeben.\footnoteA{Gemeint ist Dok. \ref{sec:BriefSerdikaMareotis}.} Und w�rde es
auch vielleicht gen�gen, euch der heiligen Kirche in Alexandrien zuzurechnen, so hat euch
die heilige Synode dennoch eigens geschrieben, damit ihr ermutigt werdet und nicht
wegen der Sachen, die ihr erleidet, abfallt, sondern dem Herrn dankt, 
auf da� eure Ausdauer gute Frucht bringen wird. 
\pend
\pstart
\kapR{4}Einst waren wohl die Unsitten der H�retiker verborgen, jetzt aber sind
sie dennoch vor allen ausgebreitet und offenbar gemacht, denn die heilige Synode
hat die von ihnen gegen euch ersonnenen falschen Anklagen abgewiesen, sie
mit Abscheu behandelt, au�erdem unter Zustimmung aller Theodorus, Valens und Ursacius
abgesetzt.\footnoteA{Vgl. Dok. \ref{sec:SerdicaRundbrief},14; \ref{sec:B},6.} Es ist
n�mlich in den anderen Kirchen dasselbe wie in Alexandria und der Mareotis geschehen.
Und weil ihre Grausamkeit nunmehr unertr�glich und ihre
Tyrannei gegen die Kirchen allgemein bekannt ist, deshalb wurden sie von ihrem
Bischofsamt abgesetzt und von jedweder Gemeinschaft ausgeschlossen. Den Gregor wollten
sie au�erdem nicht einmal erw�hnen; wer den Bischofstitel n�mlich nicht
wirklich besa�, den zu nennen hielten sie f�r �berfl�ssig. Indes wegen der von
ihm Get�uschten erw�hnten sie seinen Namen~-- nicht weil er der Erinnerung
w�rdig schien, sondern damit daraus diejenigen, die von jenem get�uscht worden sind,
seine Ruchlosigkeit erkennen und sich sch�men, mit einem Menschen, der solche
Untaten begangen hat, Gemeinschaft gehalten zu haben.\footnoteA{Vgl. Dok. \ref{sec:SerdicaRundbrief},15; \ref{sec:BriefSerdikaAlexandrien},16 und Dok. \ref{sec:BriefJuliusII}, 41-45} 
\pend
\pstart

\pend
\pstart

\pend
\pstart
\kapR{5}Entnehmt bitte das �ber sie Geschriebene aus den Anh�ngen.\footnoteA{Vgl. Dok. \ref{sec:SerdicaRundbrief}, \ref{sec:SerdicaWestBekenntnis}, \ref{sec:B}, Nr. 1--78 und Dok. \ref{sec:BriefSerdikaMareotis}, evtl. auch Dok. \ref{sec:BriefSerdikaAlexandrien}.}
Und m�glicherweise erscheinen nicht alle Bisch�fe als Schreiber, aber dennoch haben sie f�r alle
unterschrieben, was von allen verfa�t wurde. 
\pend
\pstart
\kapR{6}Gr��t euch gegenseitig mit heiligem Ku�. Es gr��en euch alle Br�der. 
\pend
\pstart
\noindent 1. Ich, Bischof Protogenes (von Serdica) w�nsche euch wohlbehalten im Herrn zu sein,
meine Vielgeliebten und Ersehnten.
\pend
\pstart
\noindent 2. Ich, Bischof Athenodorus (von Elatia) w�nsche euch wohlbehalten im Herrn zu sein,
meine vielgeliebten Br�der.
\pend
\pstart
\noindent 3. Bischof Julianus (von Thebae Heptapylus) ebenso.
\pend
\pstart
\noindent 4. Ammonius ebenso.
\pend
\pstart
\noindent 5. Aprianus (von Poetovio) ebenso.
\pend
\pstart
\noindent 6. Marcellus (von Ancyra) ebenso.
\pend
\pstart
\noindent 7. Gerontius (von Beroea) ebenso.
\pend
\pstart
\noindent 8. Porphyrius (von Philippi) ebenso.
\pend
\pstart
\noindent 9. Zosimus ebenso.
\pend
\pstart
\noindent 10. Asclepius (von Gaza) ebenso.
\pend
\pstart
\noindent 11. Appianus ebenso.
\pend
\pstart
\noindent 12. Eulogius ebenso.
\pend
\pstart
\noindent 13. Eugenius (von Heraclea Lyncestis) ebenso.
\pend
\pstart
\noindent 14. Diodorus (von Tenedus) ebenso.
\pend
\pstart
\noindent 15. Martyrius (von Naupactus) ebenso.
\pend
\pstart
\noindent 16. Eucarpus (von Opus) ebenso.
\pend
\pstart
\noindent 17. Lucius (von Adrianopel?) ebenso.
\pend
\pstart
\noindent 18. Calvus (von Castra Martis) ebenso.
\pend
\pstart
\noindent 19. Ich, Maximinus (von Trier) w�nsche euch ebenso durch Briefe aus beiden Gallien
wohlbehalten im Herrn zu sein, meine Vielgeliebten.
\pend
\pstart
\noindent 20. Wir, die Presbyter Archidamus und Philoxenus und der Diakon Leo aus Rom
w�nschen euch wohlbehalten zu sein.
\pend
\pstart
\noindent 21. Ich, Bischof Gaudentius von Na"issus w�nsche euch wohlbehalten im Herrn.
\pend
\pstart
\noindent 22. Florentius von Emerita Augusta ebenso.
\pend
\pstart
\noindent 23. Annianus von Castellona ebenso.
\pend
\pstart
\noindent 24. Januarius von Beneventum ebenso.
\pend
\pstart
\noindent 25. Praetextatus von Barcilona ebenso.
\pend
\pstart
\noindent 26. Hymenaeus von Hypata in Thessalia ebenso.
\pend
\pstart
\noindent 27. Castus von Caesaraugusta ebenso.
\pend
\pstart
\noindent 28. Severus von Chalcis in Thessalia ebenso.
\pend
\pstart
\noindent 29. Julianus von Thebae Heptapylus ebenso.
\pend
\pstart
\noindent 30. Lucius von Verona ebenso.
\pend
\pstart
\noindent 31. Eugenius von Heraclea Lyncestis ebenso.
\pend
\pstart
\noindent 32. Zosimus von Lychnidus ebenso.
\pend
\pstart
\noindent 33. Stercorius von Canusium in Apulia ebenso.
\pend
\pstart
\noindent 34. Hermogenes von Sicyon ebenso.
\pend
\pstart
\noindent 35. Tryphon von Macaria ebenso.
\pend
\pstart
\noindent 36. Paregorius von Scupi ebenso.
\pend
\pstart
\noindent 37. Calvus von Castra Martis ebenso.
\pend
\pstart
\noindent 38. Irenaeus von Scyrus ebenso.
\pend
\pstart
\noindent 39. Macedonius von Ulpiana ebenso.
\pend
\pstart
\noindent 40. Martyrius von Naupactus ebenso.
\pend
\pstart
\noindent 41. Palladius von Dion ebenso.
\pend
\pstart
\noindent 42. Verissimus aus Lugdunum in Gallia ebenso.
\pend
\pstart
\noindent 43. Ursacius von Brixia ebenso.
\pend
\pstart
\noindent 44. Amantius von Viminacium durch den Presbyter Maximus ebenso.
\pend
\pstart
\noindent 45. Alexander von Cyparissia in Achaia ebenso.
\pend
\pstart
\noindent 46. Eutychius von Methone ebenso.
\pend
\pstart
\noindent 47. Aprianus von Poetovio in Pannonia ebenso.
\pend
\pstart
\noindent 48. Antigonus von Pallene in Macedonia ebenso.
\pend
\pstart
\noindent 49. Dometianus von Acaria Constantias? ebenso.
\pend
\pstart
\noindent 50. Olympius von Aenus in Rhodope ebenso.
\pend
\pstart
\noindent 51. Zosimus von Horreum Margi ebenso.
\pend
\pstart
\noindent 52. Protasius von Mailand ebenso.
\pend
\pstart
\noindent 53. Marcus von Siscia in Savia ebenso.
\pend
\pstart
\noindent 54. Eucarpus von Opus in Achaia ebenso.
\pend
\pstart
\noindent 55. Vitalis ">Vertaresis"< in Africa ebenso.
\pend
\pstart
\noindent 56. Aelianus von Gortyna ebenso.
\pend
\pstart
\noindent 57. Symphorus von Hierapytna auf Creta ebenso.
\pend
\pstart
\noindent 58. Musonius von Heraclium ebenso.
\pend
\pstart
\noindent 59. Eucissus von Cissamus ebenso. 
\pend
\pstart
\noindent 60. Cydonius von Cydonia.
\pend
\endnumbering
\end{translatio}
\end{Rightside}
\Columns
\end{pairs}
% \thispagestyle{empty}
%%% Local Variables: 
%%% mode: latex
%%% TeX-master: "dokumente_master"
%%% End: 

\section{Rundbrief der ">�stlichen"< Synode}
% \label{sec:43.2}
\label{sec:RundbriefSerdikaOst}
\begin{praefatio}
  \begin{description}
  \item[Herbst 343]Zum Datum vgl. Einleitung zu
    Dok. \ref{ch:SerdicaEinl}. Dieser lange Rundbrief beginnt zun�chst
    mit einem Bekenntnis zur Traditionsverbundenheit (� 2), dem sich ein
    R�ckblick auf die Geschichte der Verurteilung des
    Markell\index[namen]{Markell!Bischof von Ancyra} (� 3--6) und des
    Athanasius\index[namen]{Athanasius!Bischof von Alexandrien}
    (� 7--14) anschlie�t, bevor anschlie�end auf die Vorg�nge der Synode
    in Serdica\index[synoden]{Serdica!a. 343} eingegangen
    wird. Offensichtlich gab es zun�chst eine gemeinsame Anh�rung
    (� 16), auf der die �stlichen Teilnehmer den Ausschlu� der schon
    Verurteilten forderten. Nach erfolglosen Gespr�chen schlugen sie
    offenbar vor, nochmals eine Delegation in die
    Mareotis\index[namen]{Mareotis} zu schicken, diesmal unparteiisch
    best�ckt (� 19), was aber abgelehnt wurde. Nachdem die ">westliche"<
    Synode wohl mit Kaiserbriefen zur Einheit aufgefordert hatte (� 23),
    reisten die ">�stlichen"< Synodalteilnehmer ab (� 24) und formulierten eine Verurteilung
    derer, die die schon Verurteilten bei sich aufnahmen (� 28),
    zus�tzlich eine Glaubenserkl�rung (� 29), warben mit diesem
    Rundbrief bei den Adressaten f�r ihre Position und warnten davor,
    die Neuerungen im Westen und deren Anspruch auf Entscheidungshoheit
    zu akzeptieren (� 26~f.). Zur Glaubenserkl�rung vgl.
    Dok. \ref{sec:BekenntnisSerdikaOst}.
  \item[�berlieferung]Zur �berlieferung der Collectanea antiariana des
    Hilarius vgl. Dok.  \ref{ch:SerdicaEinl}. Bei Hilarius bildet dieser Text
    zusammen mit den beiden folgenden Dokumenten
    \ref{sec:BekenntnisSerdikaOst} und \ref{sec:NominaepiscSerdikaOst}
    einen �berlieferungszusammenhang.
  \item[Fundstelle]Hil., coll.\,antiar. A IV 1 (\editioncite[48,9--67,20]{Hil:coll})
  \end{description}
\end{praefatio}
\begin{pairs}
\selectlanguage{latin}
\begin{Leftside}
% \beginnumbering
% \pstart
% Incipit decretum sinodi orientalium apud
% \edtext{\abb{Serdicam}}\index[namen]{Serdica}{\Dfootnote{\textit{coni. Coustant} Serdiciam \textit{codd.}\LitNil}}
% \edtext{\abb{episcoporum}}{\Dfootnote{\textit{coni. C vel edd.} episcopis \textit{A}}} a parte arrianorum\index[namen]{Arius, Presbyter von Alexandrien},
% \edtext{\abb{quod}}{\Dfootnote{\textit{coni. C vel edd.} que \textit{A}}} miserunt ad
% Africam\index[namen]{Afrika}.
% \pend
\pstart
\hskip -1.2em\edtext{\abb{}}{\killnumber\Cfootnote{\hskip -.6em
Hil.(ACTS)}}\specialindex{quellen}{section}{Hilarius!coll.\,antiar.!A IV 1}
\kap{1}\edtext{Gregorio}{\lemma{\abb{}} \Dfootnote{\textit{Faber scribit in
mg.}: Hic
reponenda sunt episcoporum nomina, quae infra habentur et legendum Amphioni Nic.
Donato Carth. etc.}}\edindex[namen]{Gregor!Bischof von Alexandrien} Alexandriae episcopo,
Nicomediae episcopo,
\edtext{\abb{Carthaginis}}{\Dfootnote{\textit{coni. C vel edd.} Carthagini
\textit{A}}} episcopo, Campaniae episcopo,
\edtext{\abb{Neapolis Campaniae}}{\Dfootnote{\textit{coni. C vel edd.} Neapoli
C\oline{a}pania \textit{A}}}
episcopo,
\edtext{\abb{Ariminensi clero}}{\Dfootnote{\textit{coni. Engelbrecht} Ariminiacleno
(\textit{forte ex} arimi\oline{n} clero) \textit{A} Ariminaclino \textit{coni. C}
Ariminiadeno \textit{coni. Faber} Arimini Adeno (Athenio) \textit{coni. Tillemont}}},
\edtext{\abb{Campaniae}}{\Dfootnote{\textit{coni. C vel edd.}
Ca\oline{m}p \textit{A}}} episcopo,
\edtext{\abb{Salonae}}{\Dfootnote{\textit{coni. Feder} Salonas
\textit{A} Salonarum
\textit{coni. Faber}}}
\edtext{\abb{Dalmatiae}}{\Dfootnote{\textit{coni. Faber} Dalmatus \textit{A}}}
episcopo,
\edtext{Amfioni}{\Dfootnote{\textit{sc. Nicomediae ep.}}}\edindex[namen]{Amphion!Bischof von Nikomedien},
\edtext{Donato}{\Dfootnote{\textit{sc. Carthaginis ep.}}}\edindex[namen]{Donatus!Bischof von Carthago},
\edtext{Desiderio}{\Dfootnote{\textit{sc. Campaniae ep.}}}\edindex[namen]{Desiderius!Bischof in Campanien},
\edtext{Fortunato}{\Dfootnote{\textit{sc. Neapolis ep.}}}\edindex[namen]{Fortunatus!Bischof (von Neapel?)},
\edtext{Euthicio}{\Dfootnote{\textit{sc. Campaniae ep.}}}\edindex[namen]{Eutychius!Bischof in Campanien},
\edtext{Maximo}{\Dfootnote{\textit{sc. Salonae ep.}}}\edindex[namen]{Maximus!Bischof von Salona},
\edtext{\abb{Sinferonti}}{\Dfootnote{\textit{coni. Feder} Sinferunti
\textit{A}}}\edindex[namen]{Sinferon!Bischof} et omnibus
per orbem terrarum
\edtext{\abb{consacerdotibus}}{\Dfootnote{\textit{coni. C vel edd.} cum sacerdotibus \textit{A}}} nostris,
presbiteris et diaconibus et omnibus,
\edtext{\abb{qui}}{\Dfootnote{\textit{coni. C vel edd.} que \textit{A}}} sub caelo sunt in ecclesia sancta
catholica, episcopi, qui
\edtext{\abb{a}}{\Dfootnote{\textit{coni. C} ad \textit{A}}} diversis
\edtext{Orientalium}{\lemma{\abb{}} \Dfootnote{Orientalium partium \textit{coni. Coustant}}} provinciis, id
est ex provincia Thebaida\edindex[namen]{Theba"is} et ex
provincia Palestina\edindex[namen]{Palaestina}, Arabia\edindex[namen]{Arabia},
\edtext{\abb{Foenice}}{\Dfootnote{\textit{coni. Feder} Finicae
\textit{A}}}\edindex[namen]{Phoenice}, Syria\edindex[namen]{Syria},
Mesopotamia\edindex[namen]{Mesopotamia}, Cilicia\edindex[namen]{Cilicia},
\edtext{\abb{Galatia}}{\Dfootnote{\textit{del. Faber}}}\edindex[namen]{Galatia},
\edtext{\abb{Isauria}}{\Dfootnote{\textit{coni. edd.} Pisauria
\textit{A}}}\edindex[namen]{Isauria}, Cappadocia\edindex[namen]{Cappadocia},
Galacia\edindex[namen]{Galatia}, Ponto\edindex[namen]{Pontus}, Bitinia\edindex[namen]{Bithynia}, Pamphilia\edindex[namen]{Pamphylia},
\edtext{\abb{Paflagonia}}{\Dfootnote{\textit{coni. C vel edd.} Peflagonia
\textit{A}}}\edindex[namen]{Paphlagonia}, Caria\edindex[namen]{Caria},
Frigia\edindex[namen]{Phrygia}, Pisidia\edindex[namen]{Pisidia} et ex
insulis
\edtext{Cicladon}{\Dfootnote{Cycladum \textit{coni. C}}}\edindex[namen]{Cyclades},
\edtext{Lidia, Asia, Europa, Hellesponto, Trachia, Emimonto}{\lemma{\abb{Lidia \dots\
Emimonto}}\Dfootnote{\textit{coni. C vel edd.} Lidiae Asiae Europe Hellespontu Trachiae
Emimontu
\textit{A}}}\edindex[namen]{Lydia}\edindex[namen]{Asia}\edindex[namen]{Europa}\edindex[namen]{Hellespontus}
\edindex[namen]{Thracia}\edindex[namen]{Haemimons} ad civitatem
Serdicam\edindex[namen]{Serdica} congregati
concilium
celebravimus: in domino aeterna salus.
\pend
\pstart
\kap{2}est quidem nobis omnibus indeficiens oratio, dilectissimi fratres,
primo, ut sancta domini
et catholica ecclesia dissensionibus omnibus et schismatibus carens unitatem
spiritus et vinculum
caritatis ubique conservet per fidem
\edtext{rectam}{\lemma{\abb{}} \Dfootnote{rectam sed \textit{coni. Faber}}}~-- et vitam
inmaculatam quoque tenere, amplecti,
custodire, servare omnibus invocantibus dominum est quidem iustum, praecipue
episcopis, qui
ecclesiis sanctissimis praesumus~--, secundum, ut ecclesiae regula sanctaque
parentum traditio atque
iudicia in perpetuum firma solidaque permaneant nec
\edtext{\abb{novis emergentibus}}{\Dfootnote{\textit{coni. Coustant} nobis semper
gentibus \textit{A}  nobis semper gentilibus \textit{coni. Faber}}} sectis traditionibusque
perversis, maxime in constituendis episcopis vel in
\edtext{exponendis}{\Dfootnote{deponendis \textit{coni. Faber}}}, aliquando turbetur,
\edtext{\abb{quominus}}{\Dfootnote{\textit{coni. C vel edd.} cominus \textit{A}}} teneat
\edtext{\abb{evangelica atque sancta}}{\Dfootnote{\textit{coni. Feder} evangelica. quae
s\oline{c}a \textit{A} evangeliaque (evangelia atque \textit{coni. C\corr}) sancta
\textit{coni. C} evangelica sanctaque \textit{coni. Faber}}} praecepta et quae
sanctis et beatissimis apostolis iussa sunt et maioribus
nostris
\edtext{atque}{\Dfootnote{et quae \textit{coni. Faber}}} a nobis ipsis in
hodiernum usque servata sunt et servantur.
\pend
\pstart
\kap{3}extitit namque temporibus nostris Marcellus\edindex[namen]{Markell!Bischof von Ancyra} quidam Galaciae,
haereticorum omnium
execrabilior pestis, quique sacrilega mente, ore profano perditoque argumento
velit Christi domini
regnum perpetuum, aeternum et sine tempore
\edtext{disterminare}{\Dfootnote{disterminasse \textit{coni. Faber} +
dogma Marcelli heretici \textit{A\mg}}} initium regnandi accepisse dominum dicens
ante quadringentos annos finemque ei venturam simul cum mundi
\edtext{\abb{occasu}}{\Dfootnote{\textit{coni. C vel edd.} occasum \textit{A}}}. etiam hoc asserere
\edtext{\abb{coepti}}{\Dfootnote{\textit{coni. Faber} coepit
\textit{A}}}
temeritate conatur, quod in corporis conceptione tunc factus sit
\edtext{imago
invisibilis dei}{\lemma{\abb{imago \dots\ dei}}\Afootnote{Col 1,15}}\edindex[bibel]{Kolosser!1,15}
tuncque et \edtext{\abb{panis}}{\Afootnote{Io 6,35}}\edindex[bibel]{Johannes!6,35} et
\edtext{\abb{ianua}}{\Afootnote{Io 10,7.9}}\edindex[bibel]{Johannes!10,7}\edindex[bibel]{Johannes!10,9} et \edtext{\abb{vita}}{\Afootnote{Io 11,25}}\edindex[bibel]{Johannes!11,25} effectus.
\edtext{\abb{et quidem hoc}}{\Dfootnote{\textit{coni. Feder} equidem hoc \textit{A} eadem haec
\textit{coni. Faber}}} non solum verbis
\edtext{nec}{\Dfootnote{ac \textit{coni. Coustant}}} loquacitatis
assertione confirmat, sed totum, quidquid fuerat sacrilega mente conceptum
blasphemoque ore prolatum
\edtext{\abb{concupiscentia nitenti}}{\Dfootnote{\textit{coni. Faber}
concupiscentiam intenti \textit{A} cum licentia impudenti \textit{vel} cum
audacia ingenti \textit{susp. Coustant}}}, libro
\edtext{\abb{blasphemiis}}{\Dfootnote{\textit{coni. C vel edd.} blasphemis \textit{A}}} et execrationibus
\edtext{\abb{pleno}}{\Dfootnote{\textit{coni. C vel edd.} plen\oline{u} \textit{A}}} indens etiam alia
\edtext{\abb{multo}}{\Dfootnote{\textit{coni. Coustant} multa \textit{A}}}
peiora
\edtext{in ipsum}{\Dfootnote{in ipso \textit{coni. Coustant} in ipsis
\textit{coni. Labbe}}} calumniam faciens
\edtext{scripturis divinis}{\lemma{\abb{}} \Dfootnote{\responsio\ divinis
scripturis \textit{coni. C}}} cum sua interpretatione et scelerata mente,
\edtext{quae contra illis}{\Dfootnote{qua quae contra illas \textit{vel} quae
contraria illis \textit{susp. Coustant}}} ex suo pestifero sensu applicabat, quibus
ex rebus aperte manifesteque constitit haereticus.
quique assertiones suas quibusdam squaloribus miscens, nunc falsitatibus
Sabelli\edindex[namen]{Sabellius}, nunc malitiae
Pauli\edindex[namen]{Paulus von Samosata}
\edtext{\abb{Samosatensis}}{\Dfootnote{\textit{coni. Faber} Samuensis \textit{A}}}, nunc blasphemiis
Montani\edindex[namen]{Montanus}, haereticorum omnium
\edtext{\abb{ducis}}{\Dfootnote{\textit{coni. C vel edd.} duci \textit{A}}},
\edtext{\abb{aperte}}{\Dfootnote{\textit{coni. Faber} a parte \textit{A}}} permiscens
unamque
confusionem de supradictis faciens ut imprudens Galata in
\edtext{\abb{aliud}}{\Dfootnote{\textit{coni. C vel edd.} alium \textit{A}}} evangelium declinavit, quod non est
aliud secundum quod beatus
\edtext{\abb{apostolus}}{\Dfootnote{\textit{del. Faber}}} Paulus tales condemnans
ait:
�\edtext{etsi angelus de caelo aliter
annuntiaverit vobis, quam quod accepistis, anathema sit.}{\lemma{\abb{}}
\Afootnote{Gal 1,8--9}}�\edindex[bibel]{Galater!1,8--9}
\pend
\pstart
\kap{4}magna\edlabel{43.11:42} autem fuit
\edtext{parentibus nostris atque maioribus}{\lemma{\abb{}}
\Dfootnote{\responsio\ parentibus atque maioribus nostris \textit{coni. Faber}}}
sollicitudo de supradicta praedicatione
sacrilega.
\edtext{\abb{condicitur}}{\Dfootnote{\textit{coni. edd.} conditur \textit{A}}} namque in
Constantinopolim\edindex[namen]{Konstantinopel} civitatem
sub praesentia beatissimae memoriae
Constantini\edindex[namen]{Konstantin, Kaiser} imperatoris concilium
\edtext{\abb{episcoporum}}{\lemma{\abb{}} \Dfootnote{episcoporum qui \textit{coni. Faber}}}.
ex multis Orientalium provinciis advenerunt, ut
hominem malis rebus inbutum salubri consilio reformarent et ille ammonitione,
sanctissima
\edtext{\abb{correptione}}{\Dfootnote{\textit{del. Faber}}}, correptus a
\edtext{sacrilega praedicatione}{\lemma{\abb{}}\Dfootnote{\responsio\
praedicatione sacrilega \textit{coni.C}}} discederet. quique increpantes illum et exprobantes
necnon
\edtext{\abb{etiam}}{\Dfootnote{\textit{del. C}}} caritatis affectu postulantes
multo tempore nec quicquam proficiebant.
namque post unam
et secundam multasque correptiones cum nihil proficere potuissent~-- perdurabat
enim et
contradicebat rectae fidei et contentione maligna ecclesiae catholicae
resistebat~--,
\edtext{exinde}{\Dfootnote{et inde \textit{coni. C*}}} illum
omnes
\edtext{\abb{horrere}}{\Dfootnote{\textit{coni. C vel edd.} horrore \textit{A}}} ac vitare
\edtext{\abb{coeperunt}}{\Dfootnote{\textit{coni. C vel edd.} ceperunt \textit{A}}}. et videntes,
quoniam
subversus est
\edtext{et}{\lemma{et\ts{1}}\Dfootnote{a \textit{coni. Faber}}}
\edtext{\abb{peccato}}{\Dfootnote{vel peccatis
\textit{coni. Faber} peccata \textit{A}}} et est a semet ipso
damnatus, actis eum ecclesiasticis damnaverunt, ne ulterius oves Christi
pestiferis contactibus
malis macularet.
tunc namque etiam eius aliquos pravissimos sensus contra fidem
rectam ecclesiamque
sanctissimam propter memoriam posterorum cautelamque suis sanctissimis
scripturis in archivo
ecclesiae condiderunt.
sed haec quidem secundum impietatem Marcelli\edindex[namen]{Markell!Bischof von Ancyra} haeretici
prima fuerunt; peiora
sunt deinde subsecuta. nam quis fidelium credat aut patiatur ea, quae ab ipso
male gesta atque
conscripta sunt quaeque digne anathematizata sunt
\edtext{\abb{iam}}{\Dfootnote{\textit{coni. Feder} nam \textit{A} \textit{del. vel lac.
susp. Coustant}}} cum ipso Marcello\edindex[namen]{Markell!Bischof von Ancyra} a parentibus nostris in
\edtext{\abb{Constantinopoli civitate}}{\Dfootnote{\textit{coni. C vel edd.} Constantinopolim
civitatem \textit{A}}}\edindex[namen]{Konstantinopel}? namque liber sententiarum extat in
ipsum ab episcopis
conscriptus, in
quo libro etiam isti qui nunc cum ipso sunt Marcello\edindex[namen]{Markell!Bischof von Ancyra} atque illi favent, id est
Protogenes\edindex[namen]{Protogenes!Bischof von Serdica}
\edtext{Sardicae}{\Dfootnote{Serdicae \textit{coni. Coustant}}}
civitatis episcopus et
\edtext{\abb{Cyriacus a Naiso}}{\Dfootnote{\textit{coni. Feder} Cyriacusanais \textit{A}
Siriacusanais \textit{coni. T} Siracusanus \textit{coni. Faber} \latintext
Ariminensis \textit{susp. Coustant}}}\edindex[namen]{Cyriacus!Bischof von Na"issus},
\edtext{\abb{in ipsum sententiam}}{\Dfootnote{\textit{coni. Coustant} in ipsa
sententia \textit{A}}} manu propria conscripserunt.
quorum manus
\edtext{valens}{\Dfootnote{Valens \textit{coni. C\corr}}} testatur fidem sanctissimam nullo
genere
\edtext{\abb{mutandam}}{\Dfootnote{\textit{coni. C vel edd.} mutanda \textit{A}}} nec ecclesiam
sanctam
praedicatione
falsa
\edtext{\abb{subvertendam}}{\Dfootnote{\textit{coni. C vel edd.} subvertenda \textit{A}}}, ne hoc
modo
\edtext{\abb{pestis}}{\Dfootnote{\textit{coni. C vel edd.} pestes \textit{A}}} ac lues animarum
\edtext{hominibus}{\Dfootnote{omnibus \textit{coni. Faber}}} gravissima
importetur Paulo dicente:
\edtext{�sive nos sive angelus de caelo aliter annuntiaverit vobis quam quod
accepistis, anathema sit.�}{\lemma{\abb{}}\Afootnote{Gal 1,8--9}}\edindex[bibel]{Galater!1,8--9}\edlabel{43.11:43}
\pend
\pstart
\kap{5}vehementer autem ammirati sumus, quatenus eum qui aliter, quam in
\edtext{\abb{vero}}{\Dfootnote{\textit{coni. edd.} viro \textit{A}}} est, audet evangelium
praedicare, quidam, qui
\edtext{\abb{se ecclesiasticos}}{\Dfootnote{\textit{coni. C vel edd.} secclesiasticos \textit{A}}} esse volunt,
facile ad communionem recipiunt nec
blasphemias eius, quae in ipsius libro signatae sunt, inquirentes nec illis,
qui
\edtext{\abb{sollicite}}{\Dfootnote{\textit{coni. Coustant} solliciti \textit{coni. C} sollici
\textit{A}}} cuncta
investigaverunt et invenientes iuste damnaverunt, consensum accomodare
voluerunt. etenim Marcellus\edindex[namen]{Markell!Bischof von Ancyra}
cum apud suos haereticus haberetur, peregrinationis auxilia requisivit,
scilicet, ut et illos
falleret, qui et ipsum et eius scripta pestifera ignorarent. Sed quidquid apud
illos
\edtext{\abb{agit}}{\Dfootnote{\textit{coni. C\corr} agi \textit{A}}}, impia sua
scripta sensusque profanos occultat falsa pro veris opponens, quaestum sibi de
simplicibus atque
innocentibus fecit. quique sub
\edtext{praetexto}{\Dfootnote{praetextu \textit{coni. C}}}
\edtext{ecclesiasticae}{\Dfootnote{ecclesiastici \textit{A*}}} regulae
multos ecclesiae pastores
\edtext{induxit}{\Dfootnote{seduxit \textit{coni. Coustant}}}
eosque in suam redigens potestatem, Sabelli\edindex[namen]{Sabellius} haeretici sectam inducens callida
fraude decepit
restaurans Pauli\edindex[namen]{Paulus von Samosata}
\edtext{\abb{Samosatensis}}{\Dfootnote{\textit{coni. Faber} Famitensis \textit{A}}} et artes et dolos.
mixta enim est omnium haereticorum sectis Marcelli\edindex[namen]{Markell!Bischof von Ancyra}
extranea traditio, sicut supra memoratum est. unde convenerat omnes ecclesiae
sanctae praepositos
memores dictorum domini Christi dicentis \edtext{�cavete vos a pseudoprophetis, qui
veniunt ad vos in
vestitu ovium, intrinsecus autem sunt
\edtext{\abb{lupi}}{\Dfootnote{\textit{coni. C vel edd.} lupis \textit{A}}} rapaces, a fructibus eorum
cognoscetis eos�}{\lemma{\abb{}}\Afootnote{Mt 7,15}}\edindex[bibel]{Matthaeus!7,15}
vitare ab huiusmodi et
\edtext{\abb{horrere}}{\Dfootnote{\textit{coni. C vel edd.} horrore \textit{A}}} nec illis facile communicare,
agnoscere illos ex actibus suis
\edtext{\abb{atque}}{\Dfootnote{\textit{coni. C vel edd.} adque \textit{A}}}
ex eorum scriptis sacrilegis praedamnare. nunc vero vehementer metuimus, ne
nostris temporibus
compleatur, quod scriptum est \edtext{�cum 
\edtext{\abb{dormirent}}{\Dfootnote{\textit{coni. Bauer} dormiunt \textit{A}}} homines, venit inimicus et
\edtext{sparsit}{\Dfootnote{spargit \textit{coni. C}}} zizania inter
frumentum,�}{\lemma{\abb{}}\Afootnote{Mt 13,25}}\edindex[bibel]{Matthaeus!13,25} cum enim
non vigilant
\edtext{\abb{hi}}{\Dfootnote{\textit{coni. C} hii \textit{A}}}, quos vigilare oportet
in custodia ecclesiae, falsa
\edtext{\abb{vera}}{\Dfootnote{\textit{coni. Engelbrecht} verba \textit{A}}}
\edtext{imitantia}{\Dfootnote{imitantia <veritatem> \textit{coni. Coustant} immutantia \textit{vel} nutantia \textit{coni. Fab}}} funditus, id quod rectum est,
\edtext{vertunt}{\Dfootnote{evertunt \textit{coni. Coustant}}}.
\pend
\pstart
\kap{6}quapropter nos, qui plene ex libro Marcelli\edindex[namen]{Markell!Bischof von
Ancyra} eius sectam sceleraque cognovimus, scripsimus vobis, dilectissimi fratres, ut
neque Marcellum\edindex[namen]{Markell!Bischof von Ancyra} neque eos, qui illi iunguntur,
ad communionem sanctae ecclesiae
\edtext{\abb{ammittatis}}{\Dfootnote{\textit{coni. C vel edd.} ammitatis \textit{A}}}
\edtext{memoresque}{\Dfootnote{memores quoque \textit{coni. C}}} sitis prophetae David
dicentis \edtext{�dixi eis, qui faciunt
facinus: ne feceritis facinus; et delinquentibus: ne exaltaveritis
\edtext{cornum}{\Dfootnote{cornu \textit{coni. C}}}; nolite in altum tollere
\edtext{cornum}{\Dfootnote{cornu \textit{coni. C}}} vestrum, ne loquamini adversus
deum iniqua.�}{\lemma{\abb{}}\Afootnote{Ps 74,5~f.}}\edindex[bibel]{Psalmen!74,5~f.} et credite
\edtext{\abb{Salomoni}}{\Dfootnote{\textit{coni. C vel edd.} Salomini \textit{A}}} dicenti
\edtext{�nolite transferre terminos aeternos, quos constituerunt patres vestri�}{\lemma{\abb{}}\Afootnote{Prov 22,28}}\edindex[bibel]{Sprueche!22,28}. quae cum ita
sint, nolite supradicti Marcelli\edindex[namen]{Markell!Bischof von Ancyra} pravissimi sequi errores et ea, quae ab ipso
iniusta praedicatione
contra dominum Christum inventa traditaque sunt,
\edtext{\abb{praedamnate}}{\Dfootnote{\textit{coni. C\corr} praedamnare \textit{A}}}, ne blasphemiis
sceleribusque etiam
ipsi participes sitis. sed propter compendium haec hactenus de Marcello\edindex[namen]{Markell!Bischof von Ancyra}.
\pend
\pstart
\kap{7}verum de
\edtext{\abb{Athanasio}}{\Dfootnote{+ De s\oline{c}o Athanasio falsa
omnia dicitis \textit{A\mg}}}\edindex[namen]{Athanasius!Bischof von Alexandrien} quondam
Alexandriae\edindex[namen]{Alexandrien} episcopo accipite, quae gesta sunt.
accusatus est
graviter
\edtext{\abb{iuxta}}{\Dfootnote{\textit{coni. C vel edd.} iusta \textit{A}}} deum sacrilegus et iuxta misteria
ecclesiae sanctae profanus. Quique propriis manibus
comminuit poculum deo et Christo dicatum ipsumque altare venerandum confregit
sedemque sacerdotalem
subvertit ipse atque ipsam basilicam, dei
\edtext{\abb{domum}}{\Dfootnote{\textit{coni. C} don\oline{u} \textit{A}}}, ecclesiam
Christi, demolitus usque ad solum est.
\edtext{\abb{presbiterum}}{\Dfootnote{\textit{coni. C\corr} praesbiterorum \textit{A}}} vero ipsum,
virum gravem et iustum, nomine
\edtext{\abb{Scyram}}{\Dfootnote{\textit{coni. Feder} Ischyram \textit{coni. Coustant}
Narcem
\textit{A} Narchen \textit{coni. C}}}\edindex[namen]{Ischyras!Presbyter in der Mareotis}, tradidit custodiae militari.
accusatus praeterea est de iniuriis,
\edtext{\abb{violentia}}{\Dfootnote{\textit{coni. C vel edd.} violentiae \textit{A}}}, caede atque ipsa
\edtext{\abb{episcoporum}}{\Dfootnote{\textit{S\ts{2}T} episcop\oline{u}
\textit{A} episcopi \textit{coni. C}}} internicione.
quique etiam diebus sacratissimis paschae tyrannico more saeviens ducibus atque
comitibus iunctus, quique
propter ipsum aliquos in
\edtext{custodiam}{\Dfootnote{custodia \textit{coni. C}}} recludebant, aliquos vero
verberibus flagellisque vexabant,
ceteros diversis tormentis ad communionem eius sacrilegam adigebant~-- nec
actus
\edtext{\abb{commissi}}{\Dfootnote{\textit{coni. C vel edd.}  commissa \textit{A}}} umquam ab
innocentibus fuerant~--, sperans hoc modo suos
\edtext{\abb{suamque}}{\Dfootnote{\textit{coni. Faber} su\oline{a} \textit{A}}}
\edtext{posse praevalere}{\lemma{\abb{}} \Dfootnote{\responsio\ praevalere posse \textit{coni. Faber}}}
factionem, ut per duces et
iudices perque carceres
\edtext{\abb{ipsos}}{\lemma{\abb{}} \Dfootnote{ipsos et \textit{coni. Coustant}}}, verbera diversaque
tormenta
\edtext{\abb{invitos}}{\Dfootnote{\textit{coni. C vel edd.} imvitos \textit{A}}} ad communionem suam cogeret,
nolentes adigeret, repugnantes atque resistentes sibi tyrannico more terreret.
\edtext{erant quidem
\edtext{illi}{\Dfootnote{illa \textit{coni. Faber}}}
gravia et acerba ab accusatoribus obiecta}{\lemma{\abb{erant \dots\ obiecta}}
\Dfootnote{\textit{del. C}}}.
\pend
\pstart
\kap{8}nam propter haec necessario
\edtext{\abb{concilium}}{\Dfootnote{\textit{coni. edd.} consilium \textit{A}}}
\edtext{\abb{apud Caesaream Palaestinam}}{\Dfootnote{+ Contra s.
Athanasium hereticorum concilia in Cessarea Palestina et in Tyro facta
Palaestinam \textit{A\mg}}}\edindex[synoden]{Caesarea!a. ?}\edindex[synoden]{Tyrus!a. 335} primo videtur
esse
condictum et, cum minime ab ipso vel
\edtext{ab
\edtext{\abb{eius}}{\Dfootnote{\textit{del. C}}}}{\Dfootnote{\textit{coni. Feder}
abe\oline{i} (i \textit{postea add.}) \textit{A}}} satellitibus ad supradictum
concilium occurreretur, post
alterum annum in Tyro\edindex[synoden]{Tyrus!a. 335} propter eius facinora necessario iterum celebratur.
advenerunt episcopi de
Machedonia\edindex[namen]{Macedonia} et de Pannonia\edindex[namen]{Pannonia}, Bitynia\edindex[namen]{Bithynia} et
\edtext{\abb{ex}}{\Dfootnote{\textit{del. C}}} omnibus partibus Orientis
imperatoris iussione constricti.
quique actis Athanasi\edindex[namen]{Athanasius!Bischof von Alexandrien} flagitia criminaque cognoscentes non passim nec temere
accusatoribus credunt,
sed suo de concilio inlustres et laudabiles episcopos eligunt atque in rem
praesentem, ubi res
gestae erant, de quibus arguebatur Athanasius\edindex[namen]{Athanasius!Bischof von Alexandrien}, mittunt. quique
\edtext{\abb{oculata}}{\Dfootnote{\textit{coni. Faber} occulta \textit{A}}} fide cuncta aspicientes
atque
\edtext{\abb{vera}}{\Dfootnote{(\textit{lac. inter} e \textit{et} r) \textit{A}}} cognoscentes ad concilia redeunt
eiusque
\edtext{\abb{crimina}}{\Dfootnote{+ Plane imic\oline{e} (impiam \textit{coni. C vel edd.})
sententiam dicunt \textit{A\mg}}}, quae ab accusatoribus obiciebantur, suo
testimonio vera confirmant. unde
\Ladd{\edtext{\abb{in}}{\Dfootnote{\textit{add. C\corr}}}} praesentem Athanasium\edindex[namen]{Athanasius!Bischof von Alexandrien} dignam pro
criminibus sententiam dicunt.
propter quod
\edtext{\abb{de}}{\Dfootnote{\textit{del. Faber}}} Tyro\edindex[namen]{Tyrus} fugiens imperatorem
appellat. audit etiam imperator. quique interrogatione
habita omnia eius flagitia recognoscens sua illum sententia in exilium
deportavit. his ergo ita
\edtext{\abb{cedentibus}}{\Dfootnote{\textit{coni. C\corr} credentibus \textit{A}}} rebus cum Athanasius\edindex[namen]{Athanasius!Bischof von Alexandrien} damnatus digne in exilio haberetur ob meritum
facinorum suorum
\edtext{sacrilegus}{\Dfootnote{sacrilegum \textit{A} sacrilegium \textit{coni. C\corr}}}
in deum, in mysteria sacra
\edtext{\abb{profanus}}{\Dfootnote{\textit{coni. C vel edd.} prophanus \textit{A}}}, in \edtext{basilicae
demolitione}{\lemma{\abb{}} \Dfootnote{\responsio\ demolitione basilicae
\textit{coni. C}}} violentus, in episcoporum
exitiis innocentiumque fratrum
\edtext{\abb{persecutione}}{\Dfootnote{\textit{coni. Coustant} persecutor
\textit{A}}} horrendus, ab omnibus episcopis in reiciendis malis
servatur auctoritas legis,
\edtext{\abb{canon}}{\Dfootnote{\textit{coni. C vel edd.} canone \textit{A}}} ecclesiae et apostolorum sancta
traditio.
\pend
\pstart
\kap{9}
\edtext{\abb{sed dum}}{\Dfootnote{\textit{coni. Faber} sed cum \textit{coni. C\corr}
secundum \textit{A}}} Athanasius\edindex[namen]{Athanasius!Bischof von Alexandrien} post damnationem suam reditum sibi de
exilio compararet, de Gallia\edindex[namen]{Gallien} ad
Alexandriam\edindex[namen]{Alexandrien} post plurimum tempus advenit. quique praeterita in nihilum ducens
acrius in nequitia
praevalebat. nam comparatione sequentium levia sunt, quae ab ipso prima commissa sunt. etenim per omnem viam reditus sui
\edtext{\abb{ecclesias}}{\Dfootnote{\textit{coni. Feder} ecclesi\oline{a} \textit{A}
ecclesiam \textit{coni. C}}} subvertebat, damnatos episcopos
aliquos restaurabat, aliquibus spem ad episcopatus reditum promittebat, aliquos
ex infidelibus
constituebat episcopos salvis et
\edtext{permanentibus integris}{\lemma{\abb{}} \Dfootnote{\responsio\
integris permanentibus \textit{coni. C}}} sacerdotibus per pugnas et caedes
gentilium,
nihil respiciens
\edtext{\abb{leges}}{\Dfootnote{\textit{coni. Faber} legis \textit{A}}},
\edtext{\abb{desperationi}}{\Dfootnote{\textit{coni. C\corr} desperatione \textit{A} + catholicos episco\oline{p} diversarum  civitati\oline{u} infamant \textit{A\mg}}}
tribuens totum. unde per vim, per caedem, per bellum
Alexandrinorum
\edtext{\abb{basilicas}}{\Dfootnote{\textit{coni. C} vasilica \textit{A}}}
depraedatur. constituto iam in eius
\edtext{loco}{\Dfootnote{locum \textit{coni. Faber}}} ex iudicio concilii sancto et
integro sacerdote
\ladd{\edtext{\abb{et}}{\Dfootnote{\textit{del. C}}}} ut barbarus
\edtext{\abb{hostis}}{\lemma{\abb{}} \Dfootnote{hostis et \textit{coni. C}}}, ut pestis sacrilega adductis
gentilium populis dei
templum incendit, altare comminuit et clam
\edtext{exul}{\Dfootnote{exiit \textit{coni. C} exit \textit{coni. Faber}}} de civitate
occulteque
\edtext{\abb{profugit}}{\Dfootnote{\textit{coni. C} profugiit \textit{A}}}.
\pend
\pstart
\skipnumbering
\pend
\pstart
\skipnumbering
\pend
\pstart
\kap{10}sed
\edtext{\abb{de}}{\Dfootnote{\textit{coni. Faber} et \textit{A}}} Paulo\edindex[namen]{Paulus!Bischof von Konstantinopel}
\edtext{\abb{Constantinopolitanae}}{\Dfootnote{\textit{coni. C vel edd.} Constantino Politano \textit{A}}}
civitatis
\edtext{\abb{quondam}}{\Dfootnote{\textit{coni. C vel edd.} cond\oline{a} (o ex u) \textit{A}}}
\edtext{\abb{episcopo}}{\Dfootnote{\textit{coni. C vel edd.} episcopus \textit{A}}} post reditum exilii sui, si
quis audierit, perhorrescet.
\Ladd{\edtext{\abb{\dots}}{\Dfootnote{\textit{lac. indicavit Feder; desideratur enim narratio de Paulo}}}} fuere namque et in Anquira\edindex[namen]{Ancyra} provinciae
Galatiae post reditum
Marcelli\edindex[namen]{Markell!Bischof von Ancyra} haeretici domorum incendia et genera diversa bellorum. nudi ab ipso ad
forum trahebantur
presbyteri et, quod cum lacrimis
\edtext{\abb{luctuque}}{\Dfootnote{\textit{coni. C vel edd.} luct\oline{u}quae \textit{A}}} dicendum
\edtext{\abb{est}}{\Dfootnote{\textit{del. C}}}, consecratum domini corpus ad
sacerdotum
colla suspensum palam publiceque
\edtext{\abb{profanabat}}{\Dfootnote{\textit{coni. C vel edd.} prophanabat \textit{A}}} virginesque sanctissimas deo
Christoque dicatas publice
in foro
\edtext{mediaque}{\Dfootnote{media in \textit{coni. C}}} in civitate concurrentibus
populis abtractis vestibus horrenda foeditate nudabat.
sed
\Ladd{\edtext{\abb{et}}{\Dfootnote{\textit{add. Coustant}}}} in civitate
\edtext{\abb{Gaza}}{\Dfootnote{\textit{coni. Faber} Gaia
\textit{A}}}\edindex[namen]{Gaza} provinciae Palaestinae post reditum suum Asclepas\edindex[namen]{Asclepas!Bischof von Gaza} altare comminuit
multasque seditiones effecit. praeterea Adrianopoli Lucius\edindex[namen]{Lucius!Bischof von Adrianopel} post reditum suum
sacrificium a sanctis
et integris sacerdotibus confectum, si fas est dicere, canibus proiciendum
iubebat. igitur cum haec
ita sint, numquidnam lupis tantis ac talibus oves Christi adhuc usque
\edtext{\abb{credemus}}{\Dfootnote{\textit{coni. Engelbrecht} credimus \textit{A}}}
et
\edtext{membra Christi
membra fornicatoriae faciemus}{\lemma{\abb{membra \dots\ faciemus}}\Afootnote{vgl. 1Cor 6,15}}?\edindex[bibel]{Korinther I!6,15|textit}
absit.
\pend
\pstart
\kap{11}Nam postea Athanasius\edindex[namen]{Athanasius!Bischof von Alexandrien} peragrans per diversas partes orbis terrarum,
\edtext{\abb{scilicet}}{\Dfootnote{\textit{coni. C vel edd.} silicet \textit{A}}} seducens
aliquos
\edtext{\abb{et}}{\Dfootnote{\textit{del. Faber}}} per suam fallaciam
\edtext{\abb{adulationemque}}{\Dfootnote{\textit{coni. C vel edd.} adulationeque \textit{A}}}
\edtext{\abb{pestiferam}}{\Dfootnote{\textit{coni. C vel edd.} pestiferum \textit{A}}} decipiens innocentes
episcopos, qui eius
facinora ignorabant, vel Egyptios\edindex[namen]{Aegyptus} aliquos
\edtext{actus eius}{\lemma{\abb{}} \Dfootnote{\responsio\ eius actus \textit{coni. Faber}}} ignorantes scripta a singulis emendicando
ecclesias pacificas perturbabat aut ipse sibi novas per suo voluntate
\edtext{\abb{fingebat}}{\Dfootnote{\textit{del. C}}}. nihil tamen hoc
valere potuit ad iudicium iam dudum a sanctissimis et inlustribus episcopis
consecratum. non enim
commendatio eorum, qui nec in concilio iudices fuerunt nec concilii iudicium
habuerunt aliquando nec
praesentes, cum supradictus Athanasius\edindex[namen]{Athanasius!Bischof von Alexandrien} audiretur, fuisse noscuntur, aut valere
illi potuit aut
\edtext{prode esse}{\Dfootnote{prodesse \textit{coni. C}}}. denique cum sibi haec in
\edtext{\abb{cassum}}{\Dfootnote{\textit{coni. Faber} casum \textit{A} + De s\oline{c}o Iulio qui \oline{n} consensit in damnatione Athanasii
\textit{A\mg}}} provenisse cognosceret, ad Iulium Romam\edindex[namen]{Julius!Bischof von Rom} perrexit, sed et
\edtext{\abb{ad Italiae}}{\Dfootnote{\textit{coni. C vel edd.} ad Italiam
\textit{A}}}\edindex[namen]{Italien} quosdam ipsius partis
episcopos.
\edtext{\abb{quos}}{\Dfootnote{\textit{coni. Faber} quod
\textit{A}}} seducens per epistularum falsitatem ab
\edtext{hisdem}{\Dfootnote{iisdem \textit{coni. C}}}
perfacile in communione receptus est.
\edtext{exinde}{\Dfootnote{deinde \textit{coni. C}}} coepere illi non tam pro ipso
quam pro suis actibus
laborare, quod illi pertemere credendo
\edtext{communicaverunt}{\Dfootnote{communicaverant \textit{coni. Faber}}}.
etenim si
\edtext{fuerunt}{\Dfootnote{fuerant \textit{coni. Faber}}} illae litterae aliquorum,
non tamen eorum, qui aut iudices fuerant
aut in concilio
\edtext{\abb{adsederant}}{\Dfootnote{\textit{coni. edd.rec.} assederant \textit{coni. Faber}
obsederant \textit{A}}}. quae quidem etiam si essent
aliquorum, temere illi fidem
pro se
\edtext{dicenti}{\Dfootnote{dicentes \textit{S\ts{1}S\ts{2}}}} habere numquam
deberent.
\pend
\pstart
\kap{12}
\edtext{\abb{sed et}}{\Dfootnote{= \textit{sicut et}}} iudices, qui illum
\edtext{\abb{digne}}{\Dfootnote{\textit{del. C}}}
\edtext{sententiaverunt}{\lemma{pro se \dots\ sentiaverunt} \Dfootnote{proferunt \textit{coni. C}}}, credere
\edtext{\abb{noluerunt ideo}}{\Dfootnote{\textit{coni. Engelbrecht} noluerunt Ideo \textit{A} noluerunt. Ideo
\textit{coni. Coustant} (\textit{a voce} Ideo \textit{novi capituli 12 initium sumens})}},
quia et alii
\edtext{\abb{quique}}{\Dfootnote{(= \textit{et ii, qui}) \textit{coni. Engelbrecht} qui quae
\textit{A} quique \textit{coni. C} quinque vel quoque \textit{coni. Faber} quique
qui \textit{coni. Coustant}}} in
\edtext{\abb{praeteritum}}{\Dfootnote{\textit{coni. Coustant} pretoritum \textit{A}
praetorium \textit{coni. C}}} pro suis facinoribus detecti sunt, nunc cum Marcello\edindex[namen]{Markell!Bischof von Ancyra} et
Athanasio\edindex[namen]{Athanasius!Bischof von Alexandrien} coniuncti
sunt~--
\edtext{\abb{dicimus}}{\Dfootnote{\textit{coni. Coustant} decimus \textit{A}}}\edlabel{Asclepas:Serdica}
autem
\edtext{\abb{Asclepan}}{\Dfootnote{\textit{coni. Feder} Asclepam \textit{coni. Coustant}
Asclepas \textit{A}}}\edindex[namen]{Asclepas!Bischof von Gaza},
\edtext{\abb{qui}}{\Dfootnote{\textit{del. C}}} ante decem et septem annos
episcopatus
honore discinctus est,\edlabel{Asclepas2:Serdica}
deinde Paulum\edindex[namen]{Paulus!Bischof von Konstantinopel} et Lucium\edindex[namen]{Lucius!Bischof von Adrianopel} et
\edtext{\abb{quotquot}}{\Dfootnote{\textit{coni. C vel edd.} quodquot \textit{A}}} talibus coniuncti sunt~--
circumeuntes simul
exteras regiones
persuadebant iudicibus non esse
\edtext{\abb{credendum}}{\Dfootnote{\textit{coni. C vel edd.} credendunt \textit{A}}} illis, qui in eos digne
sententiam
protulerunt, ut hoc
genere
\edtext{\abb{commercii}}{\Dfootnote{\textit{coni. C\corr} commercium \textit{A}}} sibi
\edtext{\abb{quondam}}{\Dfootnote{\textit{coni. C} cundam \textit{A} quendam
\textit{coni. Faber}}} ad episcopatus reditum procurarent. nec in ipsis
locis, in quibus
peccaverant, nec in proximis vel ubi accusatores habebant, se defendebant, sed
iuxta peregrinos et
longe a
\edtext{suis regionibus}{\lemma{\abb{}} \Dfootnote{\responsio\ regionibus suis
\textit{coni. C}}} positos, ignorantes gestorum fidem, iustam sententiam
conabantur perfringere
ad illos
\edtext{\abb{referentes}}{\Dfootnote{\textit{coni. C\corr} referenti \textit{A}}} actus suos, qui
illos per omnia nesciebant. astute hoc
faciunt. scientes enim de
iudicibus, accusatoribus, de testibus multos decessisse post tot tantasque
sententias putaverunt
aliud restaurare iudicium volentes apud nos causam dicere, qui neque illos
excusavimus neque
iudicavimus. qui enim iudicaverunt, iam ad dominum
\edtext{perrexerunt}{\Dfootnote{perrexere \textit{coni. C}}}.
\pend
\pstart
\kap{13}voluerunt autem etiam Orientalibus
\edtext{episcopis}{\lemma{\abb{}} \Dfootnote{id imponere \textit{ante} episcopis
\textit{add. Coustant}}}
\edtext{et veniunt}{\Dfootnote{ut veniant \textit{coni. Coustant}}} pro iudicibus
defensores, pro
defensoribus
\edtext{rei}{\Dfootnote{\textit{lac. inter} e \textit{et} i \textit{A}}}, eo tempore cum eorum defensio non
valebat, maxime cum tunc
defendi minime
potuerunt, quando illos accusatores sui
\edtext{\abb{faciem ad}}{\Dfootnote{\textit{coni. C vel edd.} faciaem ad \textit{A} facie ad \textit{coni. Faber}}} faciem arguebant. novam
legem
introducere
putaverunt, ut Orientales episcopi ab Occidentalibus iudicarentur. et volebant
ecclesiae iudicium
per eos posse constare, qui non tam illorum miserebantur, quam actibus suis. hoc
itaque nefas
quoniam numquam recepit ecclesiastica disciplina, quaesumus, dilectissimi
fratres, ut sceleratam
perniciem conatusque mortiferos perditorum nobiscum etiam ipsi damnetis.
\pend
\pstart
\kap{14}etenim
\edtext{adhuc cum}{\lemma{\abb{}} \Dfootnote{\responsio\ cum adhuc
\textit{coni. Faber}}} esset episcopus Athanasius\edindex[namen]{Athanasius!Bischof von Alexandrien}, Asclepam\edindex[namen]{Asclepas!Bischof von Gaza} depositum sua
sententia ipse
damnavit. sed et Marcellus\edindex[namen]{Markell!Bischof von Ancyra} similiter illi numquam communicavit. Paulus\edindex[namen]{Paulus!Bischof von Konstantinopel} vero
Athanasi\edindex[namen]{Athanasius!Bischof von Alexandrien}
\edtext{expositioni}{\Dfootnote{depositioni \textit{coni. Faber}}}
interfuit manuque propria sententiam scribens cum ceteris eum etiam ipse
damnavit. et quamdiu
quisque eorum episcopus fuit, sua iudicia confirmavit. sed cum diversis ex
causis diversisque
temporibus singuli eorum digne de ecclesia suis pro meritis pellerentur,
\edtext{\abb{maiorem}}{\Dfootnote{\textit{coni. C\corr} maiorum \textit{A}}}
concordiam una
conspiratione fecerunt donantes sibi quisque delicta, quae, cum episcopi essent,
pro auctoritate
divina damnarunt.
\pend
\pstart
\kap{15}namque
\edtext{\abb{quoniam}}{\Dfootnote{\textit{coni. edd.} q\oline{m} \textit{A} quando \textit{coni. C\corr}}}
Athanasius\edindex[namen]{Athanasius!Bischof von Alexandrien} in Italiam\edindex[namen]{Italien} et Galliam\edindex[namen]{Gallien} pergens sibi iudicium
conparavit post
\edtext{mortem}{\Dfootnote{morte \textit{A}}} aliquorum accusatorum, testium
\edtext{\abb{iudicumque}}{\Dfootnote{\textit{coni. C} iuditiumque \textit{A}}} et
credidit posse se denuo
tempore audiri,
\edtext{\abb{quo}}{\Dfootnote{\textit{coni. C vel edd.} co \textit{A} quod \textit{coni. Faber}}} eius
flagitia
\edtext{vetustate}{\Dfootnote{vetustas \textit{coni. Faber}}} temporis
\edtext{\abb{obscurarentur}}{\Dfootnote{\textit{coni. Feder} obscurarunt \textit{A} obscurarant \textit{vel} obscuraret \textit{coni. Faber}}}~-- cui consensum commodantes non
recte
\edtext{\abb{Iulius}}{\lemma{\abb{}} \Dfootnote{+ S\oline{cs} Iulius urbis Rome
e\oline{p}s in Serdicam synodum iussit fieri
\textit{A\mg}}}\edindex[namen]{Julius!Bischof von Rom} urbis Romae
episcopus,
\edtext{\abb{Maximinus}}{\Dfootnote{\textit{coni. Feder} Maximus
\textit{A}}}\edindex[namen]{Maximinus!Bischof} et Ossius\edindex[namen]{Ossius!Bischof von Cordoba} ceterique conplures ipsorum concilium apud
Serdicam fieri ex
imperatoris benignitate sumserunt~--, occurrimus ad Serdicam\edindex[synoden]{Serdica!a. 343} litteris
imperatoris conventi.
\edtext{\abb{quo}}{\Dfootnote{\textit{coni. C vel edd.} co \textit{A}}} cum
venissemus, didicimus in media ecclesia Athanasium\edindex[namen]{Athanasius!Bischof von Alexandrien}, Marcellum\edindex[namen]{Markell!Bischof von Ancyra}, omnes sceleratos
concilii sententia
pulsos et merito singulos pro suis facinoribus praedamnatos cum Ossio\edindex[namen]{Ossius!Bischof von Cordoba} et
\edtext{\abb{Protogene}}{\Dfootnote{\textit{coni. C vel edd.} Protogenes
\textit{A}}}\edindex[namen]{Protogenes!Bischof von Serdica} sedere simul et
disputare et~-- quod est deterius~-- divina misteria
\edtext{\abb{celebrare}}{\Dfootnote{\textit{coni. C vel edd.}  celibrare \textit{A}}}. nec
confundebatur Protogenes\edindex[namen]{Protogenes!Bischof von Serdica}
\edtext{Serdicae}{\Dfootnote{Sardices \textit{coni. C} Serdices \textit{coni. Faber}}} episcopus communicare Marcello\edindex[namen]{Markell!Bischof von Ancyra} haeretico, cuius et sectam et
\edtext{\abb{reprobum}}{\Dfootnote{\textit{coni. C vel edd.} reprobrum \textit{A}}}
sensum in concilio
quater sententiis episcoporum suscribens propria voce ipse damnaverat. unde
manifestum est, quia
sese ipse sua sententia condemnavit, cum se illi communicando participem fecit.
\pend
\pstart
\kap{16}verum nos tenentes ecclesiasticae regulae disciplinam et volentes
miseros in aliquantulum
\edtext{\abb{iuvare}}{\Dfootnote{\textit{coni. Engelbrecht} vivere \textit{A} remotos vivere \textit{vel} removere \textit{coni. Coustant}}}
mandavimus illis, qui cum Protogene\edindex[namen]{Protogenes!Bischof von Serdica} et Ossio\edindex[namen]{Ossius!Bischof von Cordoba} fuerunt, ut de suo coetu
damnatos excluderent
neque peccatoribus communicarent; deinde una nobiscum audirent ea, quae a
\edtext{nostris patribus}{\lemma{\abb{}} \Dfootnote{\responsio\ patribus nostris \textit{coni. C}}}
contra
ipsos in praeteritum
\edtext{\abb{fuerant}}{\Dfootnote{\textit{coni. Faber} fecerant \textit{A}}}
iudicata. nam liber Marcelli\edindex[namen]{Markell!Bischof von Ancyra}
\edtext{non}{\Dfootnote{cum \textit{vel} nam \textit{coni. Coustant}}} expectabat
accusationem~-- per se enim
aperte haereticus noscebatur~-- et ne falsis suggestionibus crederent non
eorum
\edtext{quis\-quam}{\Dfootnote{quisque \textit{susp. Coustant}}}
pravissimam mentem propter locum episcopatus
\edtext{\abb{obscurabat}}{\Dfootnote{\textit{coni. Engelbrecht} obscurant \textit{A}
occultaret \textit{vel} occultabat \textit{coni. Coustant}}}. at
\edtext{illi}{\Dfootnote{\textit{lac. post} illi \textit{A}}} contra haec
\edtext{\abb{resistebant}}{\Dfootnote{\textit{coni. C vel edd.} restitebant \textit{A}}}~-- qua
ratione, nescimus~-- nec se voluerunt ab ipsorum communione secernere
confirmantes Marcelli\edindex[namen]{Markell!Bischof von Ancyra}
haeretici sectam et facinora Athanasi\edindex[namen]{Athanasius!Bischof von Alexandrien} ceterorumque scelera fidei ecclesiasticae
pacique
praeponebant.
\pend
\pstart
\kap{17}his itaque cognitis nos octoginta episcopi, qui propter pacem
confirmandam ecclesiae ex
diversis
\edtext{longisque}{\Dfootnote{longinquisque \textit{coni. Faber}}} provinciis cum
ingenti exitu et labore ad Serdicam\edindex[namen]{Serdica} veneramus,
videntes haec non
sine lacrimis ferebamus. leve enim non erat, quod a se eos dimittere omnino
\edtext{\abb{negabant}}{\Dfootnote{\textit{coni. C vel edd.} negabat \textit{A}}}, quos patres
nostri merito suis pro criminibus ante damnaverunt. his itaque communicare nefas
duximus neque cum
\edtext{\abb{profanis}}{\Dfootnote{\textit{coni. C vel edd.} prophanis \textit{A}}} voluimus sancta domini sacramenta miscere servantes et
\edtext{\abb{tenentes}}{\lemma{\abb{}} \Dfootnote{tenentes regulam \textit{coni. C}}}
ecclesiasticae
\Ladd{\edtext{\abb{regulae}}{\Dfootnote{\textit{add. Feder}}}}
\edtext{\abb{disciplinam}}{\Dfootnote{\textit{coni. Feder} disciplinae \textit{A}}}.
\edtext{\abb{nam}}{\Dfootnote{\textit{del. Engelbrecht}}} ut non supradictos Ossium\edindex[namen]{Ossius!Bischof von Cordoba} a se
Protogenemque\edindex[namen]{Protogenes!Bischof von Serdica} dimitterent,
\edtext{conscientia}{\Dfootnote{consientia \textit{A} cum scientia \textit{coni. C\corr}}} forte
\edtext{\abb{mala}}{\lemma{\abb{}} \Dfootnote{mala impediebatur \textit{coni. Faber} mala
prohibebantur \textit{coni. Coustant}}},
\edtext{\abb{qua}}{\Dfootnote{\textit{coni. Coustant} qu\oline{a} \textit{A} (\textit{forte ex}
qu\oline{o} = quoniam \textit{susp. Engelbrecht})}}
\edtext{\abb{sibi}}{\Dfootnote{\textit{del. Coustant}}} eorum unusquisque
\edtext{\abb{metuebant}}{\Dfootnote{\textit{coni. Engelbrecht} debaebant (aebant \textit{in
ras.}) \textit{A} debebat \textit{coni. C} timebat  \textit{susp. Coustant}}}, nudari
\edtext{\abb{ab aliquo, quae}}{\Dfootnote{\textit{coni. Coustant} aliquando quae
\textit{coni. Engelbrecht} alico qu\oline{a} \textit{A} a quoquam \textit{coni. Faber}}}
\edtext{adpalam}{\Dfootnote{ac palam \textit{coni. Faber} palam \textit{coni. Coustant}}} fieri
magnopere pertimescebant.
nec
\edtext{\abb{sic}}{\Dfootnote{\textit{coni. Faber} sit \textit{A}}}
in
\edtext{\abb{eos}}{\Dfootnote{\textit{coni. C vel edd.} eis \textit{A}}}
\edtext{\abb{audebat}}{\Dfootnote{\textit{coni. C vel edd.} audebant \textit{A}}} eorum aliquis sententiam
dicere, ne et se ipse damnaret,
\edtext{\abb{quod}}{\Dfootnote{\textit{coni. Faber} quo \textit{A}}} illis contra
interdictum communicassent, quibus communicare nullo genere oporteret.
\pend
\pstart
\kap{18}verum nos iterum illos atque iterum rogabamus, ne firma solidaque
concuterent, ne
subverterent legem nec iura divina turbarent, ne cuncta confunderent atque
traditionem ecclesiae ne
quidem
\edtext{\abb{in}}{\Dfootnote{\textit{S\ts{2}} > \textit{A}}} modica parte
frustrarent, sed nec novam sectam inducerent aut
Orientalibus episcopis
conciliisque sanctissimis de Occidente venientes aliqua in
\edtext{\abb{parte}}{\Dfootnote{\textit{coni. Coustant} pace \textit{A}}}
\edtext{\abb{praeponerent}}{\Dfootnote{\textit{coni. Faber} proponerent
\textit{A}}}.
at illi contra
despicientes haec nobis minabantur
\edtext{\abb{et sese}}{\Dfootnote{\textit{coni. Fed} esse se \textit{A} et se
\textit{coni. Faber}}} Athanasium\edindex[namen]{Athanasius!Bischof von Alexandrien} et ceteros sceleratos
vindicaturos
pollicebantur, quasi vero aliud facere aut dicere
\edtext{\abb{possent}}{\Dfootnote{\textit{coni. Coustant} possunt \textit{A} possint
\textit{coni. Faber}}}, qui sceleratos omnes
ac perditos in suum
consortium
\edtext{receperunt}{\Dfootnote{reciperent \textit{coni. C}}}. praeterea
\edtext{\abb{ingenti}}{\Dfootnote{\textit{coni. Feder} ingente \textit{A}}} hoc
tumore proponunt audacia potius ac
temeritate quam
legitima ratione
\edtext{possessi}{\Dfootnote{professi \textit{coni. Oberth�r}}}. videntes enim, quod
\edtext{hi}{\Dfootnote{hii \textit{coni. C} ii \textit{coni. Faber}}}, qui damnatos
recipiunt, in
offensam crimenque
incurrunt ut violatores caelestium legum, tali auctoritate iudicium constituere
conabantur, ut sese
iudices
\edtext{\abb{iudicum}}{\Dfootnote{\textit{coni. C} iudicium \textit{A}}} dicere
vellent atque eorum, qui iam cum deo sunt, si fas est,
sententiam refricare.
nos vero rogabamus illos per plurimos dies, ut
\edtext{a sese}{\Dfootnote{se \textit{coni. C}}} damnatos abicerent et
<\edtext{\abb{se}}{\Dfootnote{\textit{add. Faber}}}>
ecclesiae sanctae
coniungerent atque patribus vera dicentibus consensum accomodarent. sed illi se
hoc facturos omnino
negabant.
\pend
\pstart
\kap{19}
\edtext{\abb{altercantibus}}{\Dfootnote{\textit{coni. C vel edd.} alter cantibus \textit{A}}} igitur nobis
\edtext{\abb{emerserunt}}{\Dfootnote{\textit{coni. C vel edd.} emerserus \textit{A}}} quinque episcopi ex
partibus
nostris, qui
\edtext{\abb{superstites}}{\Dfootnote{\textit{coni. C vel edd.} superstitites \textit{A}}}
de sexto numero eorum, qui ad Mareotam\edindex[namen]{Mareotis} directi fuerant,
\edtext{remanserunt}{\Dfootnote{remanserant \textit{coni. C}}}. et talem
illis optionem
proponunt mittendos esse ex utroque concilio episcopos aliquos ad loca, in
quibus Athanasius\edindex[namen]{Athanasius!Bischof von Alexandrien} scelera
flagitiaque commisit, ut sub
\edtext{\abb{testificatione}}{\Dfootnote{\textit{coni. edd.} testificationem \textit{A}}} dei cuncta
fideliter scriberent et,
si falsa inventa
fuerint, quae concilio nuntiavimus, ipsi damnemur nec imperatoribus nec concilio
nec cuiquam
episcopo conqueramur. quodsi verum constiterit, quod ante diximus, ad numerum
nostrum ex vobis quos
\edtext{elegistis}{\Dfootnote{delegistis \textit{coni. Faber}}}, id est, qui
Athanasio\edindex[namen]{Athanasius!Bischof von Alexandrien} post damnationem communicaverunt, sed et eos,
qui Athanasio\edindex[namen]{Athanasius!Bischof von Alexandrien} et
Marcello\edindex[namen]{Markell!Bischof von Ancyra} fautores defensoresque sunt,
\edtext{exponamus}{\Dfootnote{deponamus \textit{coni. C\corr}}} nec
imperatoribus nec concilio
nec cuiquam episcopo
\edtext{vestrum aliqui}{\lemma{\abb{}} \Dfootnote{\responsio\ aliqui vestrum
\textit{coni. C}}} conquerantur. hanc optionem a nostris propositam Ossius\edindex[namen]{Ossius!Bischof von Cordoba} et
Protogenes\edindex[namen]{Protogenes!Bischof von Serdica} omnesque eorum
socii suscipere timuerunt.
\pend
\pstart
\kap{20}inmensa autem confluxerat ad
\edtext{\abb{Serdicam}}{\Dfootnote{\textit{coni. Coustant} Sardiciam \textit{coni. C}
Sarditiam
\textit{A}}}\edindex[namen]{Serdica} multitudo sceleratorum omnium ac
perditorum
adventantium de
\edtext{\abb{Constantinupoli}}{\Dfootnote{\textit{coni. C vel edd.} Constantinupolim
\textit{A}}}\edindex[namen]{Konstantinopel}, de
\edtext{\abb{Alexandria.
\edtext{\abb{de hinc}}{\Dfootnote{\textit{coni. Feder} de hanc (\textit{duae hae voces tenuissima
linea deletae sunt}) \textit{A} \textit{del. C}}}}}{\Dfootnote{\textit{ita
interpunxit Engelbrecht}}}\edindex[namen]{Alexandrien}, qui rei homicidiorum,
rei sanguinis, rei
caedis, rei latrociniorum, rei praedarum, rei
\edtext{spoliarum}{\Dfootnote{spoliorum \textit{coni. Faber}}} nefandorumque
omnium
\edtext{sacrilegiorum}{\Dfootnote{sacrilegorum \textit{coni. Engelbrecht}}}
\Ladd{\edtext{\abb{et}}{\Dfootnote{\textit{add. C}}}}
criminum rei,
\edtext{\abb{qui}}{\Dfootnote{\textit{coni. C vel edd.} que \textit{A}}} altaria confregerunt, ecclesias
incenderunt domosque
privatorum conpilaverunt,
profanatores mysteriorum dei proditoresque sacramentorum Christi, qui impiam
sceleratamque
\edtext{\abb{haereticorum}}{\Dfootnote{\textit{coni. C vel edd.} haereti quorum \textit{A}}}
\edtext{\abb{doctrinam}}{\Dfootnote{\textit{coni. C vel edd.} doctrina \textit{A}}} contra ecclesiae fidem
adserentes sapientissimos
presbiteros dei, diacones,
sacerdotes
\edtext{\abb{atroci caede mactaverunt}}{\Dfootnote{\textit{coni. Feder} atrocicae
demactaverunt \textit{A} atrociter demactaverunt \textit{coni. C}}}. et eos omnes
secum collectos in suo
conventiculo habuerunt
\edtext{\abb{Ossius}}{\Dfootnote{\textit{coni. C vel edd.} Hossius
\textit{A}}}\edindex[namen]{Ossius!Bischof von Cordoba} et
Protogenes\edindex[namen]{Protogenes!Bischof von Serdica}.
\edtext{quosque}{\Dfootnote{eosque \textit{coni. Faber}}} honorantes nos omnes
diacones et sacerdotes dei
despiciebant, quia nec
ipsi volebamus talibus aliquando coniungi.
\edtext{quique}{\Dfootnote{iique \textit{coni. Faber}}} vulgo
\edtext{omnibusque}{\Dfootnote{honinibusque \textit{susp. Coustant}}}
\edtext{gentilibus}{\Dfootnote{gentibus \textit{coni. C}}}
id, quod inter nos
fuerat, referebant pro veris falsa fingentes nec discordiam inter nos ex causa
dei ortam, sed ex
humana praesumptione narrabant. divinis humana miscentes et ecclesiasticis rebus
privatas
adiungentes civitatis nobis concentum seditionemque conflarunt dicentes nos
gravem
\edtext{\abb{schismate}}{\Dfootnote{\textit{coni. C vel edd.} scihismati \textit{A}}}
civitati importasse
\edtext{\abb{iniuriam}}{\Dfootnote{\textit{coni. C vel. edd.} iniuria \textit{A}}}, nisi illis~-- quod nefas erat
-- communicaremus,
et haec frequenter
acclamabant. nos enim omnino illis communicare noluimus, nisi eos, quos
damnavimus, proiecissent et
dignum honorem concilio Orientis tribuerunt.
\pend
\pstart
\kap{21}quae vero ipsi egerint vel quale concilium habuerint, hinc addiscere
poteritis. nam
Protogenes\edindex[namen]{Protogenes!Bischof von Serdica}, sicut supra diximus, apud acta anathematizans Marcellum\edindex[namen]{Markell!Bischof von Ancyra} et Paulum\edindex[namen]{Paulus!Bischof von Konstantinopel} postea eos in
communionem suscepit. Dionisium\edindex[namen]{Dionysius!Bischof von Elis} vero ab
\edtext{\abb{Elida}}{\Dfootnote{\textit{coni. Feder} Elica \textit{A} Etica
\textit{S\ts{1}S\ts{2}}}} provinciae Acaiae, quem ipsi
exposuerunt, in concilio
habuerunt. simul secum et exponentes et expositi, iudices et rei
\ladd{\edtext{\abb{et}}{\Dfootnote{\textit{del. C}}}}
\edtext{\abb{communicantes}}{\Dfootnote{\textit{coni. C vel edd.} communicante \textit{A}}} mysteria divina
pertractant. Bassum\edindex[namen]{Bassus!Bischof von Diocletianopolis} autem
\edtext{de Diocletiana civitate}{\Dfootnote{de Dioclicianam civitatem
(\textit{utrumque} m \textit{del.}) \textit{A}}} in flagitiis sceleribusque
detectum et merito de
Siria\edindex[namen]{Syria} deportatum ordinaverunt episcopum. quique apud ipsos sceleratius vivens
detectus et ab ipsis
damnatus cum ipsis hodie videtur esse coniunctus.
\edtext{Aetio}{\Dfootnote{Aethio \textit{coni. Coustant} et. io \textit{A} et quia Io.
\textit{coni. Faber} et quia Ioannes \textit{coni. Labbe}}}\edindex[namen]{A"etius!Bischof von Thessalonike}
<\edtext{\abb{vero}}{\Dfootnote{\textit{add. Feder}}}>
\edtext{Tessalonicensi}{\Dfootnote{Tessalonicensis \textit{coni. C} Thessalonicensis
\textit{coni. Faber}}}
Protogenes\edindex[namen]{Protogenes!Bischof von Serdica} frequenter
\edtext{\abb{probra}}{\Dfootnote{\textit{coni. C vel edd.} proba \textit{A}}} multa criminaque obiecit,
\edtext{\abb{quod}}{\Dfootnote{\textit{coni. Faber} quo \textit{A}}} diceret illum
\edtext{concubas}{\Dfootnote{concubinas \textit{coni. Coustant}}} et habuisse et
habere, cui communicare
numquam voluit. nunc vero in amicitiam receptus, quasi peiorum consortio
expurgatus, apud ipsos
\edtext{\abb{habetur}}{\Dfootnote{\textit{coni. Faber} haberetur
\textit{A}}} ut iustus. Asclepas\edindex[namen]{Asclepas!Bischof von Gaza} autem cum ad Constantinopolim\edindex[namen]{Konstantinopel} civitatem propter
Paulum\edindex[namen]{Paulus!Bischof von Konstantinopel} venisset, post
immanitatem
\edtext{rerum}{\Dfootnote{eorum \textit{coni. Engelbrecht}}} atrocitatemque
\edtext{commisit}{\Dfootnote{quae commisit \textit{coni. Engelbrecht}}}, quae media in
ecclesia
Constantinupolitana\edindex[namen]{Konstantinopel}
\edtext{\abb{gesta}}{\Dfootnote{\textit{coni. C vel edd.} iesta \textit{A}}} sunt,
post mille homicidia, quae altaria ipsa humano sanguine
\edtext{\abb{coinquinaverunt}}{\Dfootnote{\textit{coni. C} conguinaverunt \textit{A}}}, post
interfectiones fratrum
extinctionesque gentilium, hodieque cum Paulo\edindex[namen]{Paulus!Bischof von Konstantinopel}, cuius causa haec gesta sunt,
communicare non
\edtext{\abb{cessat}}{\Dfootnote{\textit{coni. Faber} cessant \textit{A}}},
sed
\Ladd{\edtext{\abb{et}}{\Dfootnote{\textit{add. Faber}}}} illi, qui per
\edtext{Asclepan}{\Dfootnote{Asclepam \textit{coni. Faber}}}\edindex[namen]{Asclepas!Bischof von Gaza} Paulo\edindex[namen]{Paulus!Bischof von Konstantinopel} communicant
accipientes ab eodem scripta
atque ad illum
mittentes.
\pend
\pstart
\kap{22}ex
\edtext{\abb{hac}}{\Dfootnote{\textit{coni. Faber} ac \textit{A}}} igitur
coagulatione atque ex perditorum congregatione quale
potuerit
\edtext{\abb{celebrari}}{\Dfootnote{\textit{coni. C vel edd.} celebrare \textit{A}}}
concilium, in quo flagitia sua non tam puniebant, quam agnoscebant! non
enim
\edtext{\abb{secundum}}{\lemma{\abb{}} \Dfootnote{secungum eos \textit{coni. Coustant}}} nos, qui
ecclesiis sanctissimis praesedemus populisque rectores sumus,
\edtext{\abb{donantes}}{\lemma{\abb{}} \Dfootnote{sumus donantes \textit{coni. Coustant}}} et
dimittentes, quae ab ipsis
nec dimitti umquam possunt nec donari. quique etiam Marcello\edindex[namen]{Markell!Bischof von Ancyra} et Athanasio\edindex[namen]{Athanasius!Bischof von Alexandrien}
ceterisque sceleratis
flagitia,
\edtext{blasphemia}{\Dfootnote{blasphemias \textit{coni. Coustant}}}, quae nefas
\edtext{fuerat}{\Dfootnote{fuerit \textit{coni. Faber}}} dimittere, condonarunt, cum
scriptum
sit:
\edtext{�si peccaverit homo
in hominem, orabunt pro eo ad dominum; si autem in deum peccaverit homo, quis
\edtext{\abb{orabit}}{\Dfootnote{\textit{coni. C vel edd.} oravit \textit{A}}} pro eo? Nos
autem talem consuetudinem non habemus nec ecclesia dei.�}{\lemma{\abb{si \dots\ dei}}\Afootnote{vgl. 1Reg 2,25; 1Cor 11,16}}\edindex[bibel]{Koenige I!2,25|textit}\edindex[bibel]{Korinther I!11,16|textit}
sed nec haec docere
aliquos patimur nec novas traditiones inducere, ne proditores fidei
traditoresque scripturarum
divinarum,
\edtext{quod nefas est,
\edtext{\abb{dicamur}}{\Dfootnote{\textit{coni. Coustant} dicere \textit{A}}},
ne a domino}{\lemma{\abb{}} \Dfootnote{quod nefas est dicere, a domino \textit{susp. Feder}}} et ab hominibus condamnemur.
\pend
\pstart
\kap{23}verum illi supradicti etiam haec adversus nos machinabantur.
\edtext{\abb{et}}{\Dfootnote{\textit{coni. Faber} ut \textit{A}}}
quoniam
sciebant nos non
posse sibi scelestorum beneficio communicare, ex scriptis nos imperatorum
\edtext{\abb{terrere putabant}}{\Dfootnote{\textit{coni. edd.} terrae reputabant \textit{A}}}, ut
\edtext{\abb{invitos}}{\Dfootnote{\textit{coni. C vel edd.} imvitos \textit{A}}} ad suam communionem traherent,
et
\edtext{spectabant}{\Dfootnote{spectabunt \textit{coni. Faber}}} dividi totius mundi
\edtext{profundam}{\Dfootnote{profundum \textit{coni. C}}} ecclesiae
aeternamque pacem propter Athanasium\edindex[namen]{Athanasius!Bischof von Alexandrien} et Marcellum\edindex[namen]{Markell!Bischof von Ancyra},
\edtext{per}{\Dfootnote{propter \textit{coni. C}}} quos
\edtext{nomen domini
blasphematur in gentibus}{\lemma{\abb{nomen \dots\ gentibus}}\Afootnote{Esai 52,5}}.\edindex[bibel]{Jesaja!52,5}
quos oportebat,
\edtext{\abb{si}}{\Dfootnote{\textit{coni. C} sci \textit{A}}} in se aliquid timoris
dei habuissent, ne
\edtext{turbor}{\Dfootnote{turbo \textit{coni. C}}} iste, qui per
ipsos natus est,
permaneret, a sua pravissima praesumptione vel sero
\edtext{\abb{discedere}}{\Dfootnote{\textit{coni. Faber} discederent \textit{A}}}, ne ipsorum
causa
contra sese
scinderetur
\edtext{\abb{ecclesia}}{\Dfootnote{\textit{coni. C vel edd.} ecclesiam \textit{A}}}. aut si in
\edtext{his}{\Dfootnote{hiis \textit{coni. C} iis \textit{coni. Faber}}}, qui pro
ipsis
\edtext{certantur}{\Dfootnote{certant \textit{coni. C}}}, metus dei esset,
etiamsi
\edtext{nihil}{\lemma{\abb{}} \Dfootnote{nihil esse \textit{coni. Faber}}}
damnabile apud ipsos inveniretur, tamen
\edtext{vel}{\Dfootnote{et \textit{coni. Coustant} \textit{del. C}}} quod propter ipsos unitas
ecclesiae
scinditur et propter
\edtext{insanam}{\Dfootnote{insaniam \textit{A*}}} rabiem cupiditatemque
\edtext{honoris}{\Dfootnote{honorum \textit{coni. C}}} pax profunda subvertitur,
exsecrari eos ac
perhorrere
deberent.
\pend
\pstart
\kap{24}cum ita res currere videremus, ad suam patriam regredi nostrum
\edtext{unusquisque}{\Dfootnote{unusquique \textit{coni. Coustant}}} decrevit
placuitque nobis de Serdica\edindex[namen]{Serdica} scribere et ea, quae gesta sunt, nuntiare nostramque
sententiam
declarare. nos enim Athanasium\edindex[namen]{Athanasius!Bischof von Alexandrien} et Marcellum\edindex[namen]{Markell!Bischof von Ancyra}, qui in dominum impie blasphemantes
scelerati vixerunt,
\edtext{expositos}{\Dfootnote{depositos \textit{coni. Faber}}} olim atque
damnatos non possumus iterum in episcopatus \edtext{honore}{\Dfootnote{honorem
\textit{coni. C}}}
suscipere. quique
\edtext{crucifigentes
\edtext{iterum}{\Dfootnote{verum \textit{coni. C}}} filium dei atque illum denuo
publicantes}{\lemma{\abb{crucifigentes \dots\ publicantes}}\Afootnote{Hebr 6,6}}\edindex[bibel]{Hebraeer!6,6}
acerbis
\edtext{\abb{ictibus confixerunt}}{\Dfootnote{\textit{coni. Faber} actibus conflixerunt
\textit{A}}}. alter enim ipsorum blasphemando in filium dei atque in eius
\edtext{regnum}{\Dfootnote{\textit{coni. C vel edd.} regno \textit{A}}}
\edtext{aeterna morte
mortuus est semel}{\lemma{\abb{aeterna \dots\ semel}}\Afootnote{vgl. Rom 6,10}}\edindex[bibel]{Roemer!6,10|textit}; alter in
corpus domini
\Ladd{\edtext{\abb{et}}{\Dfootnote{\textit{add. Coustant}}}}
\edtext{misteria}{\Dfootnote{mysteriaque \textit{coni. C\corr}}} eius
\edtext{\abb{profano}}{\Dfootnote{\textit{coni. C vel edd.} prophano \textit{A}}} more
atrociter peccans
ceteraque
\edtext{\abb{flagitia}}{\Dfootnote{\textit{coni. C vel edd.} flagatia \textit{A}}} inmaniter gerens episcoporum
sententia eiectus est
\edtext{\abb{atque}}{\Dfootnote{\textit{coni. C vel edd.} adque \textit{A}}}
damnatus. quam ob rem,
quoniam a parentum traditione discedere non possumus, quia nec talem
auctoritatem sumsit ecclesia
nec talem potestatem a deo accepit, supradictos
\edtext{\abb{ad}}{\Dfootnote{\textit{coni. C vel edd.} at \textit{A}}} honorem dignitatemque
ecclesiae nec ipsi
\edtext{\abb{suscipimus}}{\Dfootnote{\textit{coni. Faber} suscepimus
\textit{A}}} et suscipientes digne damnamus. sed nec alios, qui aut olim aut
postea merito damnati
sunt, in ecclesia recipimus adhaerentes legibus dei traditionibusque paternis
atque ecclesiasticis
disciplinis, credentes prophetae dicenti:
�\edtext{\edtext{\abb{noli}}{\Dfootnote{\textit{coni. Faber} nolite \textit{A}}}
transgredi terminos aeternos,
quos posuerunt patres
tui}{\lemma{\abb{}}\Afootnote{Prov 22,28}}�\edindex[bibel]{Sprueche!22,28}. quare nos
numquam fixa solidaque
concutimus, sed magis ea,
quae sunt a parentibus
constituta, servamus.
\pend
\pstart
\kap{25}
\edtext{post multa igitur}{\lemma{\abb{}} \Dfootnote{+ Iniuste
cal\oline{u}nas catholicis ep\oline{i}s infertis id \oline{t} Ossio Protogines
Atanasio Marcello Ascleple (\textit{alt.} e \textit{ex} o) Paulo Iulio et
caeteris \textit{A\mg}}}, dilectissimi fratres, haec vobis ex aperto mandamus,
ne quis vestrum ab
aliquo circumventus aliquando communicet, id est Ossio\edindex[namen]{Ossius!Bischof von Cordoba},
\edtext{\abb{Protogeni}}{\Dfootnote{\textit{coni. C vel edd.} Protoiene
\textit{A}}}\edindex[namen]{Protogenes!Bischof von Serdica},
Athanasio\edindex[namen]{Athanasius!Bischof von Alexandrien},
Marcello\edindex[namen]{Markell!Bischof von Ancyra}, Asclepae\edindex[namen]{Asclepas!Bischof von Gaza},
Paulo\edindex[namen]{Paulus!Bischof von Konstantinopel}, Iulio\edindex[namen]{Julius!Bischof von Rom}, sed
\edtext{\abb{nec}}{\Dfootnote{\textit{coni. Coustant} ne \textit{A}}} cuiquam damnatorum de
ecclesia sancta
\edtext{reiectis}{\Dfootnote{reiecto \textit{susp. Feder}}} neque
sociis ipsorum, qui illis
sive per se sive per scripta communicant. quare
\edtext{vos nec}{\lemma{\abb{}}\Dfootnote{\responsio\ nec vos \textit{coni. Faber}}} ad illos
scribere umquam
debetis nec ab
ipsis scripta suscipere. iam quod superest, oramus vos, dilectissimi fratres, ut
unitati
\edtext{\abb{ecclesiae}}{\Dfootnote{\textit{coni. C vel edd.} haecclesiae \textit{A}}}
consulatis
\Ladd{\edtext{\abb{et}}{\Dfootnote{\textit{add. Faber}}}}
\edtext{paci}{\Dfootnote{pati \textit{A\ts{1}}}} perpetuae, sanctos episcopos
eligatis, in quibus et fides
integra et sancta sit
vita, abhorrentes eos, qui pro criminibus suis ab episcopatus honore discincti
sunt voluntque iterum
recipere locum, quem digne pro suis facinoribus amiserunt.
\edtext{\abb{magis}}{\Dfootnote{\textit{del. C}}} magisque
\edtext{hos}{\Dfootnote{eos \textit{coni. C}}}
\edtext{execramini}{\Dfootnote{execremini \textit{coni. C\corr}}}, quos post
facinus videtis peiora committere nec dominum audiunt dicentem:
\edtext{peccasti, quiesce.}{\lemma{\abb{peccasti}}\Afootnote{vgl. Sir 21,1}}\edindex[bibel]{Jesus Sirach!21,1}
\edtext{iuvenescunt}{\Dfootnote{\textit{S\ts{1}S\ts{2}} iubenescunt \textit{A}
iubere nesciunt \textit{coni. T} inhibere nesciunt \textit{coni. Faber} quiescere
\textit{vel} inhibere se \textit{vel} se continere nesciunt \textit{susp. Coustant}}} enim
et fortiores de sceleribus efficiuntur et, quanto plus in
gurgitem vitiorum
mergunt, tanto plus
\edtext{\abb{totum}}{\Dfootnote{\textit{del. C}}} orbem subvertunt. seditionibus
vacantes bella
persecutionesque acerrimas
sanctis ecclesiis ingerunt et tyrannico more populos
\edtext{dei}{\Dfootnote{domini \textit{coni. Faber}}} in suum dominium
captivare contendunt.
\pend
\pstart
\kap{26}namque ex his rebus pessimos eorum conatus agnoscite,
\edtext{quando}{\Dfootnote{quod \textit{coni. Oberth�r}}} talem
mundo tempestatis
\edtext{procellam}{\Dfootnote{procelle \textit{A\ts{1}}}} induxerunt,
\edtext{\abb{qua}}{\Dfootnote{\textit{coni. Feder} quae \textit{A} ut \textit{coni. C} quae \dots\ turbaret \textit{susp. Feder}}} Orientem prope totum
Occidentemque turbarent, ut
relinquentes singuli
ecclesiasticas curas populosque dei deserentes atque ipsam evangelii doctrinam
postponentes de
longinquo adveniremus,
\ladd{\edtext{\abb{se}}{\Dfootnote{\textit{del. Feder} scilicet \textit{coni. et postea del. Faber}}}} senes
aetate graves, corpore debiles, aegritudine
\edtext{infirmes}{\Dfootnote{infirmi \textit{coni. C}}}~--
trahebamurque per diversa nostrosque aegrotantes in itineribus deserebamus
propter perpaucos
scelestos olim digne damnatos, primatus ecclesiae contra fas appetentes~--
\edtext{geratque}{\Dfootnote{gereretque \textit{susp. Feder}}} curam
de nobis
imperium atque religiosi imperatores, tribuni et duces dirissima
\edtext{re publica}{\Dfootnote{republica \textit{coni. Faber}}} de
episcoporum vita
\edtext{\abb{statuque}}{\Dfootnote{\textit{coni. C vel edd.} statumque \textit{A}}} exercerentur. sed nec populi
silent. omnis etenim fraternitas omnibus
in provinciis
suspensa ac
\edtext{\abb{sollicita}}{\Dfootnote{\textit{coni. Faber} solida \textit{A}}} expectat, in quem finem
haec malorum procella
\edtext{succedit}{\Dfootnote{succedat \textit{coni. C\corr}}}.
cursusque ipse
publicus attritus ad nihilum deducitur. et quid pluribus? ortus occasusque mundi
propter unum vel
duos paucosque sceleratos impie sentientes et turpiter viventes funditus
vertitur et dura saevaque
tempestate turbatur, in quibus nulla religionis
\edtext{\abb{semina resederunt}}{\Dfootnote{\textit{coni. C vel edd.} seminare sederunt \textit{A}}}. quae si
habuissent, imitarentur
prophetam dicentem:
\edtext{�tollite et mittite me in mare et
\edtext{\abb{tranquillabit}}{\Dfootnote{\textit{coni. C vel edd.} tranquillavit \textit{A}}} mare a vobis,
\edtext{quoniam}{\Dfootnote{qu\oline{o} \textit{A} quomodo \textit{coni. C} quando
\textit{coni. Faber}}} haec
tempestas propter me facta est.�}{\lemma{\abb{}}\Afootnote{Ion 1,12}}\edindex[bibel]{Jona!1,12} sed ideo
haec non imitantur, quia
nec iustos sequuntur.
ita autem sceleratorum omnium duces quaerunt ecclesiae principatum quasi
\edtext{\abb{aliquod}}{\Dfootnote{\textit{coni. C} aliquo \textit{A}}}
tyrannidis regnum.
\pend
\pstart
\kap{27}nec hoc propter bonum quoque iustitiae inquirunt. non enim ecclesiis
consulunt, qui leges
iuraque divina
\Ladd{\edtext{\abb{ac}}{\Dfootnote{\textit{add. Engelbrecht} et \textit{add. Coustant}}}}
ceterorum decreta dissolvere perconantur. propterea hanc
novitatem moliebantur
inducere, quam horret vetus consuetudo ecclesiae, ut, in concilio Orientales
episcopi quidquid forte
statuissent, ab episcopis Occidentalibus
\edtext{\abb{refricaretur}}{\Dfootnote{\textit{coni. Faber}
refrigaretur \textit{A} refragaretur \textit{coni. C}}}, similiter
\edtext{\abb{et}}{\Dfootnote{\textit{del. C}}}, quidquid
Occidentalium partium
episcopi, ab Orientalibus solveretur. sed hoc ex illo suo pravissimo sensu
tractabant. verum omnium
conciliorum iuste legitimeque
\edtext{\abb{actorum}}{\Dfootnote{\textit{coni. Faber} auctorum
\textit{A}}} decreta firmanda maiorum nostrorum gesta
consignant. nam in
urbe Roma\edindex[namen]{Rom} sub Novato\edindex[namen]{Novatian! } et Sabellio\edindex[namen]{Sabellius} et Valentino\edindex[namen]{Valentinus!Gnostiker}
\edtext{\abb{haereticis}}{\Dfootnote{\textit{coni. C vel edd.} haeretis \textit{A}}} factum concilium ab
Orientalibus
confirmatum est. et iterum in Oriente, sub Paulo\edindex[namen]{Paulus von Samosata}
\edtext{\abb{a Samosatis}}{\Dfootnote{\textit{coni. Feder} Samosatis \textit{S\ts{1}} a Somos
ait \textit{A} Asonus ait \textit{coni. T} Samosatensi \textit{coni. Faber}}} quod statutum est,
ab omnibus est
signatum. ob quam rem hortamur vos, dilectissimi fratres, considerantes ordinem
ecclesiasticae
disciplinae et paci totius orbis consulentes, corripiatis eos, qui sceleratis
communicant, et malos
de ecclesiis radicitus amputetis, ut tempestate ruente ipsorum causa
\edtext{\abb{innatans}}{\Dfootnote{\textit{coni. Faber} innatam \textit{A}}}
Christus dominus
\edtext{omnibus imperet ventis procellisque maritimis}{\lemma{\abb{omnibus \dots\ maritimis}}
\Afootnote{vgl. Mt 8,26 u.\,�.}}\edindex[bibel]{Matthaeus!8,26|textit} discedere et tribuat ecclesiae
sanctae pacem perpetuam
et
\edtext{quietam}{\Dfootnote{quietem \textit{coni. Faber}}}.
\pend
\pstart
\kap{28}nos vero nulli iniuriam
\edtext{fecimus}{\Dfootnote{facimus \textit{coni. Faber}}}, sed legis praecepta
servamus. nam
iniuriati graviter et
male tractati sumus ab
\edtext{his}{\Dfootnote{iis \textit{coni. Faber}}}, qui volebant ecclesiae catholicae
regulam sua
pravitate turbare, sed
ante oculos habentes timorem dei,
\edtext{iudicium Christi verum et iustum}{\lemma{\abb{iudicium \dots\ iustum}}\Afootnote{vgl. Apc
16,7; 19,2}}\edindex[bibel]{Offenbarung!16,7|textit}\edindex[bibel]{Offenbarung!19,2|textit} considerantes
nullius personam
\edtext{\abb{accepimus}}{\Dfootnote{\textit{coni. C\corr} accipimus \textit{A}}} neque alicui
pepercimus, quominus ecclesiasticam disciplinam
servaremus. unde
\edtext{Iulium}{\lemma{\abb{}} \Dfootnote{+ vos magis
d\oline{a}nati estis qu\oline{a} Iulius vel ceteri catholici
\textit{A\mg}}}\edindex[namen]{Julius!Bischof von Rom} urbis
Romae, Ossium\edindex[namen]{Ossius!Bischof von Cordoba} et Protogenem\edindex[namen]{Protogenes!Bischof von Serdica} et Gaudentium\edindex[namen]{Gaudentius!Bischof von Na"issus} et Maximinum\edindex[namen]{Maximinus!Bischof von Trier} a Triveris damnavit omne
concilium secundum
antiquissimam legem ut auctores communionis Marcelli\edindex[namen]{Markell!Bischof von Ancyra} et Athanasi\edindex[namen]{Athanasius!Bischof von Alexandrien} ceterorumque
sceleratorum, quique
etiam homicidiis Pauli\edindex[namen]{Paulus!Bischof von Konstantinopel} Constantinopolitani
\Ladd{\edtext{\abb{et}}{\Dfootnote{\textit{add. Faber}}}} cruentis actibus eius
communicaverunt. Protogenem\edindex[namen]{Protogenes!Bischof von Serdica}
namque esse
\edtext{\abb{anathematizandum}}{\Dfootnote{\textit{coni. Faber} anathematizantem \textit{A} anathematizatum \textit{coni. Faber}}} una cum Marcello\edindex[namen]{Markell!Bischof von Ancyra} subscribendo frequenter sententiae
in ipsum vel in
eius librum inlatae, quique etiam in Paulum\edindex[namen]{Paulus!Bischof von Konstantinopel} Constantinupolitanum
\edtext{\abb{sententiam}}{\Dfootnote{\textit{coni. C} sentiae \textit{A}}}
dederunt et illi postea
communicaverunt. Gaudentium\edindex[namen]{Gaudentius!Bischof von Na"issus} autem ut inmemorem
\edtext{\abb{decessoris}}{\Dfootnote{\textit{coni. C vel edd.} decessori \textit{A}}} sui Cyriaci\edindex[namen]{Cyriacus!Bischof von Na"issus}
suscribentis sententiis in
sceleratos digne inlatis
\edtext{\abb{commixtumque}}{\Dfootnote{\textit{coni. C vel edd.} cum mixtumque \textit{A}}} criminibus Pauli\edindex[namen]{Paulus!Bischof von Konstantinopel},
quem etiam impudenter
defendebat. Iulium\edindex[namen]{Julius!Bischof von Rom}
vero urbis Romae ut principem et ducem malorum, qui primus ianuam communionis
sceleratis atque
damnatis aperuit ceterisque aditum fecit ad solvenda iura divina defendabatque
Athanasium\edindex[namen]{Athanasius!Bischof von Alexandrien}
praesumenter atque
\edtext{\abb{audaciter}}{\Dfootnote{\textit{coni. C vel edd.} audatiter \textit{A} audacter \textit{coni. Coustant}}},
hominem, cuius nec testes noverat nec accusatores.
sed Ossium\edindex[namen]{Ossius!Bischof von Cordoba} propter
supradictam causam et propter beatissimae memoriae Marcum\edindex[namen]{Marcus! }, cui graves semper
iniurias irrogavit, sed
et quod malos omnes pro criminibus suis digne damnatos totis viribus defendebat
et
\edtext{\abb{quod}}{\Dfootnote{\textit{coni. C vel edd.} quot \textit{A}}} convixerit
in Oriente cum sceleratis
\edtext{\abb{ac}}{\Dfootnote{\textit{coni. C} a \textit{A}}} perditis. turpiter namque
Paulino\edindex[namen]{Paulinus!Bischof in Dakien} quondam episcopo
Daciae individuus
amicus fuit, homini, qui primo
\edtext{maleficus}{\Dfootnote{maleficiis \textit{coni. C*} maleficus \textit{coni. C\corr} de
maleficiis \textit{coni. Coustant}}} fuerit accusatus et de ecclesia pulsus
usque in hodiernum
diem in apostasia permanens cum concubinis publice et meretricibus
\edtext{\abb{fornicetur}}{\Dfootnote{\textit{coni. C vel edd.} formicetur \textit{A}}},
cuius maleficiorum
libros Machedonius\edindex[namen]{Macedonius!Bischof von Mopsuestia} episcopus atque confessor a
\edtext{Mopsuestia}{\Dfootnote{\textit{coni. Erl.} Mobso \textit{A} Mopso \textit{coni. Faber}}}
combussit. sed
\ladd{\edtext{\abb{et}}{\Dfootnote{\textit{add. Faber}}}}
Eustasio\edindex[namen]{Eustathius!Bischof von Antiochien} et
\edtext{Quimatio}{\Dfootnote{Cimatio \textit{coni. Coustant}}}\edindex[namen]{Cymatius!Bischof von Paltus}
adhaerebat pessime et carus fuit,
\edtext{\abb{de}}{\Dfootnote{\textit{coni. Faber} ad \textit{A}}} quorum
\edtext{\abb{vita infami ac}}{\Dfootnote{\textit{coni. C\corr} vitam infamias
\textit{A} vitae infamia \textit{coni. Faber}}} turpi dicendum nihil
est; exitus enim
illorum eos omnibus declaravit. his itaque ac talibus iunctus ab initio Ossius\edindex[namen]{Ossius!Bischof von Cordoba},
\edtext{\abb{sceleratos}}{\Dfootnote{\textit{coni. C} sceratos \textit{A} sceleratis
\textit{coni. Faber}}} semper
\edtext{fovens}{\Dfootnote{favens \textit{coni. Faber}}}, contra ecclesiam veniebat et
inimicis dei semper ferebat
\edtext{\abb{auxilium}}{\Dfootnote{\textit{coni. C vel edd.} ausilium \textit{A}}}.
Maximinum\edindex[namen]{Maximinus!Bischof von Trier}
\edtext{\abb{vero}}{\Dfootnote{\textit{coni. Faber} verum \textit{A}}} a
Triveris, propter quod
\edtext{\abb{collegas}}{\Dfootnote{\textit{coni. Faber} colleges \textit{A} colligens
\textit{coni. C}}} nostros episcopos, quos ad Gallias\edindex[namen]{Gallien} miseramus,
noluerit suscipere et
quoniam Paulo\edindex[namen]{Paulus!Bischof von Konstantinopel} Constantinupolitano nefario homini ac perdito primus ipse
communicavit et quod ipse
tantae cladis causa fuit, ut Paulus\edindex[namen]{Paulus!Bischof von Konstantinopel} ad
\edtext{\abb{Constantinupolim}}{\Dfootnote{\textit{coni. C vel edd.} Constantinupulim
\textit{A}}}\edindex[namen]{Konstantinopel}
\edtext{\abb{civitatem}}{\Dfootnote{\textit{coni. C vel edd.} civitate \textit{A}}} revocaretur,
propter
\edtext{quem}{\Dfootnote{quam \textit{coni. C} quod \textit{coni. Faber}}}
homicidia multa facta sunt. causa
\edtext{ergo}{\Dfootnote{igitur \textit{coni. Faber}}} homicidiorum tantorum
ipse fuit, qui
Paulum\edindex[namen]{Paulus!Bischof von Konstantinopel} olim damnatum ad
\edtext{\abb{Constantinupolim}}{\Dfootnote{\textit{coni. C vel edd.} Constantinupoli
\textit{A}}}\edindex[namen]{Konstantinopel} revocavit.
\pend
\pstart
\kap{29}propter has igitur causas iustum duxit concilium, ut Iulium\edindex[namen]{Julius!Bischof von Rom} urbis Romae
et Ossium\edindex[namen]{Ossius!Bischof von Cordoba}
ceterosque supra memoratos discingeret
\edtext{\abb{atque}}{\Dfootnote{\textit{coni. C vel edd.} adque \textit{A}}} damnaret. quae cum ita sint,
custodire vos ab ipsis et
abstinere debetis, dilectissimi fratres, nec eos aliquando ad communionem
vestram
\edtext{ammittere}{\Dfootnote{admittere \textit{coni. C}}}, sed nec
ipsorum litteras accipere nec ad illos litteras dominicas dare. et quoniam
catholicam et apostolicam
fidem voluerunt infringere
\edtext{\abb{hi}}{\Dfootnote{\textit{coni. Engelbrecht} hii \textit{A} ii \textit{coni. Faber}}}, qui cum Ossio\edindex[namen]{Ossius!Bischof von Cordoba} erant,
\edtext{inducentes}{\Dfootnote{inducentesque \textit{coni. Faber}}} novam
sectam
\edtext{Iudeo}{\Dfootnote{in deo \textit{coni. C*} Iudaeicae \textit{susp. Engelbrecht}}}
\edtext{\abb{couniti}}{\Dfootnote{\textit{coni. Feder} croniti \textit{AS\ts{1}} uniti
\textit{coni. Coustant} concreti \textit{coni. Hardouin}}} Marcelli\edindex[namen]{Markell!Bischof von Ancyra},
\edtext{\abb{mixtam}}{\Dfootnote{\textit{coni. Feder} mixta \textit{A} iuxta
\textit{susp. Coustant}}}
\edtext{Sabellio et Paulo,
\edtext{\abb{iudaizantem}}{\Dfootnote{\textit{coni. Coustant} iuda artem \textit{AS}
inde artem \textit{coni. Faber} inde autem \textit{Hardouin}}}}{\Dfootnote{Sabellium et
Paulum iudaizantem \textit{susp. Coustant}}}\edindex[namen]{Sabellius}\edindex[namen]{Paulus von Samosata} necessario ordinavimus catholicae
ecclesiae fidem, quam
negaverunt supradicti, qui cum Ossio\edindex[namen]{Ossius!Bischof von Cordoba} sunt, et
\edtext{Marcelli}{\Dfootnote{Marcello \textit{Coustant}}}\edindex[namen]{Markell!Bischof von Ancyra}
\edtext{\abb{haeretici}}{\Dfootnote{\textit{A} haeresim \textit{coni. Coustant}}} induxerunt.
consequens est, ut
acceptis litteris nostris singuli consensum huic sententiae
\edtext{\abb{commodantes}}{\Dfootnote{\textit{coni. C vel edd.} commodantens \textit{A}}} decreta
nostra propria
subscriptione signetis.
\pend
% \endnumbering
\end{Leftside}
\begin{Rightside}
\begin{translatio}
\beginnumbering
% \pstart
% Anfang des Dekrets des Synode der �stlichen Bisch�fe bei Serdica von der Partei
% der Arianer, das sie nach Afrika schickten.
% \pend
\pstart
\noindent Gregor\footnoteA{Zu Gregor vgl. Dok \ref{sec:SerdicaRundbrief},15 Anm.; \ref{sec:BriefSerdikaAlexandrien},16 und Dok. \ref{sec:BriefJuliusII},41--45. Bei den �brigen
Adressaten sind wahrscheinlich die Namen mit den Bischofssitzen zusammenzuziehen; dabei
bleibt zu ber�cksichtigen, da� Orts- und Provinznamen wechseln, da die urspr�nglichen
Namen wohl verschiedene nachtr�gliche Erg�nzungen erfuhren. Dann erg�be sich: Amphion von
Nikomedien (Nachfolger des Eusebius, belegt auch in Ath., apol.\,sec. 7,2; Thdt., h.\,e. I 20), Donatus von Carthago, Desiderius aus Campanien (sonst unbekannt), Fortunatus von Neapel (114~f. Feder), (der Klerus von Rimini, ohne spezielle Bischofsnennung [Vakanz?]), Euthicius aus
Campanien (sonst unbekannt), Maximus von Salona in Dalmatien (118~f. Feder). Hier ergeben
sich aber folgende Probleme: Wer ist Sinferon? Au�erdem sind unter den Teilnehmern der
">westlichen"< Synode ein Calepodius aus Neapel (Nr. 20) und ein Gratus aus Carthago (Nr.
43) belegt.}, dem Bischof von Alexandrien, dem Bischof von Nikomedien, dem Bischof von
Carthago, dem Bischof von Campanien, dem Bischof von Neapel in Campanien, dem Klerus von Rimini, dem
Bischof von Campanien, dem Bischof von Salona in Dalmatien, Amphion, Donatus,
Desiderius, Fortunatus, Euthicius, Maximus, Sinferon und
allen unseren Mitpriestern auf dem Erdkreis, den Presbytern und Diakonen und allen, die
unter dem Himmel in der heiligen, katholischen Kirche sind, w�nschen die Bisch�fe, die wir
uns aus den verschiedenen Provinzen des Ostens,\footnoteA{Vgl. neben dieser Provinzliste
auch die Unterschriftenliste in Dok. \ref{sec:NominaepiscSerdikaOst}; zu den hier
genannten Provinzen lassen sich nur f�r Mesopotamien, Paphlagonien und Karien keine Namen
zuweisen, allerdings fehlt bei zehn Namen auch eine Ortsangabe. Weitere Listen der
beteiligten Provinzen an der �stlichen Teilsynode finden sich noch in Hil., syn. 33;
Cod. Veron. LX f. 79a; Cod. Par. syr. 62 f. 185a; Vigilius von Thapsus, c. Eutych.
5. In allen diesen Listen fehlt zwar Isaurien (wof�r aber Namen belegt sind), es sind aber
jene drei Provinzen auch aufgef�hrt, f�r die es keine Namen gibt. Zus�tzlich werden noch
Aegyptus (Hil.; Cod. Veron.; Cod. Par. syr.), M�sien, Pannoniis duabus (alle Zeugen); Dacia
(Cod. Veron.; Cod. Par. syr.) und Phrygiis duabus (Hil.; Vigilius) genannt.} d.h. aus der
Provinz Theba"is und aus den Provinzen Palaestina, Arabia, Phoenice, Syria, Mesopotamia,
Cilicia, Galatia, Isauria, Cappadocia, Galatia, Pontus, Bithynia, Pamphylia, Paphlagonia,
Caria, Phrygia, Pisidia und den Kykladen, sowie den Provinzen Lydia, Asia,
Europa, Hellespont, Thracia und Haemimons in Serdica versammelt und eine Synode veranstaltet  haben:
Ewiges Heil im Herrn!
\pend
\pstart
\kapR{1}Wir alle, vielgeliebte Br�der, beten ohne Zweifel unabl�ssig daf�r,
da� erstens die heilige und katholische Kirche des Herrn frei von allen Streitigkeiten und
Schismen bleibt und da� sie die Einheit des Geistes und das Band der Liebe �berall durch den rechten Glauben
bewahrt~-- und f�r alle, die den Herrn anrufen, ist es angemessen, ein unbeflecktes Leben einzuhalten, 
anzunehmen, zu h�ten und zu bewahren, vor allem f�r uns
Bisch�fe, die wir den so heiligen Kirchen vorstehen~--, da� zweitens die Regel der Kirche
und die heilige �berlieferung sowie die Entscheidungen der V�ter f�r immer in Kraft und
g�ltig bleiben und nicht irgendwann durch neu auftretende Sekten und pervertierte
�berlieferungen in Unordnung geraten, vor allem bei der Einsetzung und Absetzung von Bisch�fen, 
so da� man sich nicht an die evangelischen und heiligen Vorschriften h�lt und an
das, was von den heiligen und seligsten Aposteln befohlen und von unseren
Vorg�ngern und von uns selbst bis heute bewahrt worden ist und bewahrt wird.
\pend
\pstart
\kapR{2}Es ist n�mlich in unseren Zeiten ein gewisser Markell aus Galatia\footnoteA{Zu
Markell vgl. Dok. \ref{ch:Konstantinopel336}; \ref{sec:MarkellJulius}.} aufgetreten, eine verdammungsw�rdigere 
Pest als alle H�retiker, der mit frevelhaftem Sinn, unheiliger
Stimme und unheilvoller Beweisf�hrung das immerw�hrende, ewige und ohne Zeit seiende Reich
des Herrn Christus begrenzen will, indem er sagt, da� der Herr vor vierhundert Jahren
einen Anfang seiner Herrschaft erhalten habe und f�r ihn ein Ende kommen werde zugleich
mit dem Untergang der Welt. Auch dies versucht er bei seinem leichtfertigen Vorhaben zu
behaupten, da� Christus damals erst bei der Empf�ngnis seines K�rpers Bild des unsichtbaren
Gottes und auch Brot, T�r und Leben geworden sei. Und dies bekr�ftigt er nicht
nur mit Worten und in seiner geschw�tzigen Behauptung, sondern er gab das Ganze, was 
schon in frevelhaftem Sinne begonnen und mit blasphemischer, vor Begierde triefender Zunge
verbreitet worden war, in einem Buch voll von Blasphemien und Verw�nschungen wieder. 
In dieses Buch f�gte er auch anderes, noch um vieles Schlechtere hinzu und verleumdete die
g�ttlichen Schriften mit seiner Auslegung und seinem frevelhaften Gedankengeb�ude, 
welches Elemente vereint, die im Widerspruch zu jenen (heiligen Schriften) stehen und
seiner verderblichen Absicht erwachsen, wodurch er offenkundig und erwiesenerma�en
als H�retiker feststand.  
Und dieser Markell mischte seine Erkl�rungen mit einigem Schmutz, bald mit den
L�gen des Sabellius\footnoteA{Vgl. Dok. \ref{sec:Eustathius},1}, bald mit der
Schlechtigkeit eines Paulus von Samosata\footnoteA{Vgl. Dok. \ref{ch:Konstantinopel336},3,4.},
bald mit den Blasphemien des Montanus, des Anf�hrers aller H�retiker. Indem er ganz
offen diese Theorien mischte und einen Einheitsbrei aus den eben genannten H�resien
schuf, glitt er wie der unkluge Galater in ein anderes Evangelium ab, welches
nichts anderes ist als das, was der selige Apostel Paulus so sehr mit folgenden Worten verdammte:
">Auch wenn ein Engel vom Himmel euch anderes verk�ndet hat, als ihr empfangen habt, sei er
ausgeschlossen."<
\pend
\pstart
\kapR{3}Dennoch herrschte gro�e Sorge bei unseren V�tern und Vorg�ngern �ber die obengenannte,
gottlose Verk�ndigung. Man berief n�mlich nach Konstantinopel eine Bischofssynode
in Anwesenheit Kaiser Konstantins seligen Angedenkens ein. Sie kamen aus vielen �stlichen
Provinzen, damit sie diesen von vielen �beln getr�nkten Menschen durch einen heilsamen
Beschlu� besserten und jener durch Ermahnung und heiligsten Tadel gescholten werde und von der
gottlosen Verk�ndigung ablasse. Und sie nahmen ihn in die Mangel, pr�ften ihn und stellten
gewi� auch lange Zeit ihre Forderungen mit wohlwollender Liebe, aber erreichten absolut nichts.
Denn da sie nach einer, der zweiten und etlichen Ermahnungen nicht vorankommen konnten~-- er blieb n�mlich stur und widersprach dem rechten Glauben und widersetzte sich
in b�sartigem Streit der katholischen Kirche~--, begannen hierauf alle vor jenem
zur�ckzuschrecken und ihn zu meiden. Und als sie sahen, da� er von der S�nde zu
Fall gebracht und durch sich selbst verurteilt worden war, haben sie ihn durch kirchliche
Beschl�sse verurteilt, damit er nicht noch dar�ber hinaus die Schafe Christi durch
todbringende, �ble Ansteckungen beflecke. Damals haben sie n�mlich auch einige seiner
krummsten Ansichten gegen den rechten Glauben und die heiligste Kirche zur Erinnerung
f�r die Sp�teren und zum Schutz f�r ihre heiligsten Schriften im Archiv der Kirche
hinterlegt. Aber dies waren nur die ersten Erlebnisse mit der Gottlosigkeit
des H�retikers Markell. Schlimmere sind hierauf gefolgt. Denn welcher Gl�ubige 
glaubt oder ertr�gt schon das, was von jenem eigens �bles angestellt und zusammengeschrieben wurde
und was zusammen mit seiner Person zu Recht schon von unseren V�tern in Konstantinopel 
mit dem Anathema belegt worden ist? Es liegt n�mlich ein gegen ihn von den Bisch�fen verfa�ter
\textit{Liber sententiarum} vor, in dem auch die, die nun auf der Seite von Markell stehen und
ihn unterst�tzen, dies sind Protogenes, Bischof von Serdica,\footnoteA{Vgl. den Brief
Dok. \ref{sec:BriefOssiusProtogenes}.} und Cyriacus von Na"issus\footnoteA{In der
Unterschriftenliste steht schon der Name seines Nachfolgers Gaudentius (vgl. Dok.
\ref{ch:SerdicaEinl}: Nr. 41); vgl. Soz., h.\,e. III 11,8.}, gegen ihn mit eigener Hand Stellung
bezogen haben. Von deren Hand wird nachdr�cklich bezeugt, da� der heiligste Glaube in
keiner Weise ver�ndert werden und die heilige Kirche durch keine falsche Verk�ndigung
umgest�rzt werden darf, damit nicht auf diese Weise Verderben und f�r die Seelen schlimmstes Gift
in die Menschen hineingetragen werde, wie Paulus sagt: ">Seien es wir oder ein Engel
vom Himmel, der euch etwas anders verk�ndet hat, als ihr empfangen habt, er sei
ausgeschlossen."<
\pend
\pstart
\kapR{4}Vor allem aber haben wir uns dar�ber gewundert, wie lange und leichtfertig manche, 
die zur Kirche geh�ren wollen, ihn, der ein anderes Evangelium zu verk�nden wagt, als es wahr ist,
in die Gemeinschaft aufnehmen, ohne sich nach seinen Blasphemien zu
erkundigen, die in seinem eigenen Buch verzeichnet sind, wie sie aber jenen, die sorgsam alles
aufgedeckt, ermittelt und ihn zu Recht verurteilt hatten, die
Zustimmung verweigern wollten. Da n�mlich Markell bei seinen eigenen Leuten f�r einen H�retiker
gehalten wurde, griff er zu dem Hilfsmittel, in andere Gebiete zu reisen, nat�rlich um
auch jene zu t�uschen, die ihn selbst und seine sch�dlichen Schriften nicht kannten. Aber
was immer er bei jenen treibt, er verbirgt seine gottlosen Schriften und unheiligen
Ansichten, indem er Wahres durch Falsches ersetzt
und sich anbiedert bei den Einf�ltigen und Unschuldigen.
Unter dem Deckmantel kirchlicher Regel hat er viele
Hirten der Kirche verf�hrt, sie in seine Gewalt gebracht und durch hinterh�ltigen
Betrug get�uscht, w�hrend er die Sekte des H�retikers Sabellius einf�hrte und die Kunstgriffe und
Tricks des Paulus von Samosata wiederherstellte.\footnoteA{S.\,o. � 3.} Die fremde
�berlieferung des Markell ist n�mlich aus den Sekten aller H�retiker gemischt, woran schon oben
erinnert wurde. Daher hat man sich darauf geeinigt, da� alle, die der heiligen Kirche
vorstehen, sich die Worte des Herrn Christus vor Augen halten: ">H�tet euch vor den
falschen Propheten, die zu euch im Schafspelz kommen, inwendig aber rei�ende W�lfe sind;
an ihren Fr�chten werdet ihr sie erkennen"<, derartige Leute meiden, vor ihnen
zur�ckschrecken und nicht leichtfertig mit ihnen Gemeinschaft halten sollen. Sie sollen sie an 
ihren Taten zu erkennen und infolge ihrer frevelhaften Schriften verurteilen.
Nun aber f�rchten wir vor allem, da� sich in unseren Zeiten erf�llt, was geschrieben steht: ">Als
die Menschen schliefen, kam der Feind und streute Unkraut unter das Getreide."< Wenn
n�mlich die nicht wachsam sind, deren Aufgabe es ist, zum Schutz der Kirche Wache zu halten,
dann imitieren die falschen Lehren die wahren und stellen das, was recht ist, v�llig auf den Kopf.
\pend
\pstart
\kapR{5}Daher haben wir, die wir aus dem Buch Markells seine Sekte und seine Untaten ganz
genau kennen, euch geschrieben, vielgeliebte Br�der, damit ihr nicht Markell und die,
die sich ihm anschlie�en, zur Gemeinschaft der heiligen Kirche zula�t und an den Propheten
David denkt, der sagt: ">Ich habe denen gesagt, die Unrecht begehen: Begeht kein Unrecht,
und den S�ndern: Erhebt nicht das Horn. Hebt nicht euer Horn in die H�he, damit ihr nicht
Unrechtes gegen Gott sprecht."< Und glaubt Salomon, der sagt: ">Verr�ckt nicht die ewigen
Grenzen, die euere V�ter aufgestellt haben."< Da das so ist, folgt nicht den Irrlehren
des oben genannten v�llig verdorbenen Markell und verurteilt das, was von ihm mit
seiner unrechten Verk�ndigung gegen den Herrn Christus erfunden und weitergegeben worden ist, damit
nicht ihr selbst auch noch an seinen Blasphemien und Verbrechen Anteil habt. Aber um der K�rze
willen sei dies genug �ber Markell.
\pend
\pstart
\kapR{6}Aber h�rt, was verhandelt worden ist �ber Athanasius, einst Bischof von
Alexandrien.\footnoteA{Zu den Vorw�rfen gegen Athanasius vgl. Dok.
\ref{sec:BriefJuliusII},27--47 mit Anm.} Er ist schwerwiegend angeklagt als Frevler vor
Gott und als Gottloser im Umgang mit den Geheimnissen der heiligen Kirche. Er hat mit eigenen H�nden
einen Gott und Christus geweihten Kelch zerbrochen, den zu verehrenden Altar selbst
zertr�mmert, pers�nlich den Priestersitz umgest�rzt und die Basilika selbst, das Haus
Gottes, die Kirche Christi, bis auf die Grundmauern zerst�rt. Den Priester selbst aber,
einen w�rdigen und gerechten Mann namens Ischyras,\footnoteA{Vgl. Dok.
\ref{sec:BriefJuliusII},33} �bergab
er dem milit�rischen Gewahrsam. Er ist au�erdem angeklagt wegen Ungerechtigkeiten, Gewalt,
Mord und selbst der T�tung von Bisch�fen. Er w�tete auch in den heiligsten Tagen des
Osterfestes in tyrannischer Art im Verbund mit weltlichen F�hrern und Comites, die etliche
seinetwegen in Haft sperrten, andere aber mit Schl�gen und Gei�eln qu�lten und den Rest
mit verschiedenen Foltern zur frevelhaften Gemeinschaft mit ihm zwangen~-- und niemals
w�ren solche Taten von Unschuldigen begangen worden. Er hoffte, da� auf diese Weise die
Seinen und seine Partei erstarken k�nnten, um durch weltliche F�hrer, Richter und
selbst Gef�ngnisse, Schl�ge und verschiedene Foltern die Widerspenstigen zur Gemeinschaft
mit ihm zu zwingen, die Unwilligen dahinzutreiben und die Widerstreitenden und sich ihm
Widersetzenden durch seine tyrannische Art zu erschrecken. Es waren ihm also ernste und
schlimme Taten von den Ankl�gern vorgeworfen worden.
\pend
\pstart
\kapR{7}Aus diesem Grund sah man sich offensichtlich gezwungen, eine Synode einzuberufen, 
erstmals in Caesarea in Palaestina,\footnoteA{Diese Synode aus dem Jahr 334 (ind.\,ep.\,fest. 6) erw�hnt Athanasius selbst nicht; vgl. aber Soz., h.\,e. II 25,1; Thdt., h.\,e. I 28.}
und da sowohl er als auch seine Anh�nger der besagten Synode v�llig fernblieben, wurde
notwendigerweise nach einem weiteren Jahr erneut eine Synode in 
Tyrus\footnoteA{Zur Synode aus dem Jahr 335, auf der Athanasius nach den Ergebnissen der Mareotis-Kommission (vgl. dazu Dok. \ref{sec:BriefJuliusII},32) abgesetzt wurde, vgl. Eus., v.\,C. IV 42; Ath., apol.\,sec. 73--90; Socr., h.\,e. I 28--30; Soz., h.\,e. II 25; 28; Thdt., h.\,e. I 29~f.; Philost., h.\,e. II 33.} wegen
seiner Untaten abgehalten. Es kamen auf Befehl des Kaisers Bisch�fe aus
Makedonien und Pannonien, aus Bithynien und aus allen Teilen des Ostens angereist.
Als sie aus
den Akten die Schandtaten und Verbrechen des Athanasius erkannten, da glaubten sie den
Ankl�gern weder un�berlegt noch blindlings, sondern w�hlten aus ihrer Versammlung angesehene
und untadelige Bisch�fe und schickten sie in der aktuellen Angelegenheit dorthin, wo die
Dinge geschehen waren, deretwegen Athanasius angeklagt wurde. Sie betrachteten alles mit
sehendem Glauben und erkannten die Wahrheit. Daraufhin kehrten sie zu den
Synoden\footnoteA{Gemeint sind wahrscheinlich die Synoden von Tyrus (s.\,o. � 8) und von Jerusalem (vgl. Dok. \ref{sec:BriefJerusalem}).} zur�ck und best�tigten
seine Verbrechen, die ihm von dem Ankl�gern vorgeworfen wurden, durch ihr eigenes Zeugnis
als wahr. Daher verk�ndeten sie gegen�ber Athanasius in seiner Anwesenheit ein den Vergehen
angemessenes Urteil. Deswegen floh er aus Tyrus und legte beim Kaiser Berufung ein. Der
Kaiser gew�hrte ihm auch eine Anh�rung, und als er nach einer Befragung alle
seine Schandtaten erfahren hatte, schickte er ihn durch sein Urteil in die
Verbannung.\footnoteA{Nach diesem Treffen mit dem Kaiser in Konstantinopel schickte dieser Athanasius nach Trier in Gallien, vgl. ind.\,ep.\,fest. 8; Socr., h.\,e. I 34~f.; Soz., h.\,e. II 28.} Nachdem also diese Dinge so ausgegangen
waren, Athanasius zu recht verurteilt und zur Belohnung f�r seine Untaten im Exil festgehalten
worden war als Frevler gegen Gott, als Gottloser im Umgang mit den heiligen Geheimnissen, als
Gewaltt�ter durch seine Zerst�rung einer Basilika und befleckt durch die Ermordungen von
Bisch�fen und die Verfolgung unschuldiger Br�der, sind von allen Bisch�fen in der Abwehr
des B�sen die Autorit�t des Gesetzes, der Kanon der Kirche und die heilige Tradition der
Apostel bewahrt worden.
\pend
\pstart
\kapR{8}Unterdessen bereitete Athanasius nach seiner Verurteilung seine R�ckkehr aus dem
Exil vor und kam nach sehr langer Zeit von Gallien nach Alexandrien zur�ck. Er
hielt die vergangenen Taten f�r null und nichtig und ging noch weiter in seiner Bosheit. Denn
verglichen mit den folgenden Taten sind seine fr�heren kaum der Rede wert.
Auf dem gesamten Weg seiner R�ckkehr stellte er n�mlich Kirchen auf den Kopf: Er setzte etliche
verurteilte Bisch�fe wieder ein, manchen verhie� er Hoffnung zur R�ckkehr auf ihren
Bischofssitz und setzte einige aus den Reihen der Ungl�ubigen als Bisch�fe ein, obwohl 
Geistliche zur Verf�gung standen, die w�hrend der K�mpfe und Morde der Heiden wohlbehalten und integer geblieben waren, denn er nahm keine R�cksicht auf die Gesetze, sondern �berlies alles der Verzweiflung.
Daher pl�ndert er mit Gewalt, Mord und Krieg die Basiliken der Alexandriner. Nachdem 
bereits an seiner Stelle nach dem Urteil der Synode ein heiliger und unbescholtener
Geistlicher eingesetzt worden war, steckte er wie ein barbarischer Feind, wie eine gottlose
Seuche zusammen mit dem Volk der Heiden den Tempel Gottes in Brand, zerschlug den Altar
und entfloh klammheimlich als Verbannter aus der
Stadt.\footnoteA{Vgl. zu den Vorw�rfen Ath., apol.\,sec. 5 und Dok. \ref{sec:BriefJulius} Einleitung.}
\pend
\pstart

\pend
\pstart

\pend
\pstart
\kapR{9}Aber vor Paulus, dem einstigen Bischof von Konstantinopel, wird jeder nach dessen R�ckkehr
aus dem Exil zur�ckschrecken, der davon geh�rt hat. <\dots> denn es gab 
auch in Ancyra in der Provinz Galatia nach der R�ckkehr des H�retikers
Markell H�userbr�nde und verschiedene Arten von Unruhen. Nackt wurden von ihm pers�nlich
Presbyter zum Forum gezerrt und er lie�, was man unter Tr�nen und mit Trauer sagen mu�, 
den geweihten Leib des Herrn vor aller Augen �ffentlich entweihen und den 
Geistlichen um den Hals h�ngen, auch entbl��te er mit schauderhafter Abscheulichkeit
die heiligsten, Gott und Christus geweihten Jungfrauen mit zerrissenen Kleidern �ffentlich auf dem Forum mitten unter der B�rgerschaft vor dem zusammengelaufenen Volk.
Aber auch in der Stadt Gaza in der Provinz
Palaestina zerst�rte Asclepas\footnoteA{Zu Asclepas vgl. Dok. \ref{sec:Asclepas}.} nach
seiner R�ckkehr den Altar und entfachte viele Aufst�nde. Au�erdem befahl Lucius von
Adrianopel\footnoteA{Lucius von Adrianopel war Teilnehmer der westlichen Teilsynode (vgl.
Unterschriftenliste). Zu Lucius vgl. Dok. \ref{sec:BriefJuliusII},52; Socr., h.\,e. II
15,2; Soz., h.\,e. III 8,1; Ath., fug. 3,3 (Liste der Verbannten); Socr., h.\,e. II 23,39--43; Soz., h.\,e. III 24,3
(R�ckkehr nach der Synode von Serdica); Ath., h.\,Ar. 19,1 (zweite Verbannung nach Serdica,
w�hrend der er starb).} nach seiner R�ckkehr, das von heiligen und unbescholtenen Geistlichen
vollzogene Opfer, wenn man es erw�hnen darf, den Hunden vorzuwerfen. Sollen wir also unter solchen Umst�nden die Schafe Christi noch l�nger so gro�en und derartigen W�lfen anvertrauen und die Glieder Christi zu Gliedern einer Hure machen? Es sei fern!
\pend
\pstart
\kapR{10}Sp�ter n�mlich durchwanderte Athanasius die verschiedenen Teile des Erdkreises,
brachte nat�rlich etliche zum Abfall und t�uschte durch seine Gerissenheit und verderbenbringende
Schmeichelei unschuldige Bisch�fe, die seine Untaten nicht kannten, ja sogar einige �gyptische
Bisch�fe, die nichts von seinen Taten wu�ten. Und dabei versetzte er friedliche Kirchen in Unruhe
oder schuf sich eigenm�chtig und willk�rlich neue Kirchen, indem er Unterschriften von einzelnen
Leuten erschlich. Dennoch vermochte dies nichts auszurichten an dem schon seit
langem von den heiligsten und bedeutendsten Bisch�fen festgesetzten Urteil. Denn die
Empfehlung derer, die weder Richter auf der Synode waren noch je ein Stimmrecht auf ihr besa�en, sondern abwesend waren, als der obengenannte Athanasius verh�rt
wurde, konnte nichts f�r jenen ausrichten oder ihm n�tzlich sein. Als er schlie�lich
erkannte, da� dies f�r ihn ins Leere lief, brach er zu Julius nach Rom auf, aber auch zu
gewissen Bisch�fen seiner eigenen Partei in Italien. Indem er diese durch gef�lschte
Briefe in die Irre f�hrte, wurde er ohne Probleme von diesen in die Gemeinschaft
aufgenommen. So begann es, da� jene in Schwierigkeiten gerieten, nicht so sehr seinetwegen als wegen ihrer eigenen Taten, da sie ihm blind vertrauten und Gemeinschaft mit
ihm hielten. Denn auch wenn es entsprechende Briefe von irgendwelchen Leuten gab, so geh�rten diese dennoch nicht zu denen, die entweder Richter gewesen oder auf der Synode anwesend gewesen waren. Und auch wenn es diese Briefe von irgendwelchen Leuten g�be, d�rften jene niemals blindlings Vertrauen haben zu einem, der f�r sich spricht.
\pend
\pstart
\kapR{11}Die Richter aber, die jenen zu Recht verurteilt haben, wollten ihm auch deswegen
nicht glauben, weil einige andere Leute, die sich in der Vergangenheit durch ihre eigenen Schandtaten verraten haben, nun mit Markell und Athanasius verbunden sind~-- wir meinen aber Asclepas, der vor 17 Jahren der Ehre des Bischofsamtes enthoben worden ist, weiterhin Paulus und
Lucius und wer sonst noch mit jene Leuten in Verbindung steht. Gemeinsam zogen sie in Gebieten umher, die au�erhalb ihrer Befugnis liegen, und agitierten, da� man jenen
Richtern nicht glauben d�rfe, die zu Recht das Urteil gegen sie gef�llt hatten, um mit
diesen Agitationen daf�r zu sorgen, da� sie einst in ihr Bischofsamt zur�ckkehren
w�rden. Nicht an denselben Orten, wo sie ges�ndigt hatten, noch in den benachbarten
oder dort, wo sie Ankl�ger hatten, verteidigten sie sich, sondern versuchten bei fremden und
weit von ihren Gebieten entfernten Leuten, die den Beweis f�r die Geschehnisse nicht kannten, 
das rechte Urteil zu zerbrechen, indem sie ihnen von ihren eigenen Taten berichteten, die
sie ganz und gar nicht kannten. Hinterlistig tun sie dies. Denn in dem Wissen, da� viele
Richter, Ankl�ger und Zeugen gestorben waren, glaubten sie, nach so vielen und so
gewichtigen Urteilsspr�chen ein anderes Urteil herstellen zu k�nnen, und wollten auch vor
uns ihren Fall vorbringen, die wir jene weder freigesprochen noch verurteilt haben. Die
das Urteil n�mlich gef�llt haben, sind bereits zum Herrn gelangt.
\pend
\pstart
\kapR{12}Und sie wollten den �stlichen Bisch�fen sogar den Proze� machen, kamen aber als Verteidiger anstatt als Richter, (erwiesen sich) als Schuldige anstatt als Verteidiger, in jener Zeit, als ihre Verteidigung nichts vermochte, besonders weil
sie sich damals nicht im Geringsten verteidigen konnten, als ihre Ankl�ger sie von
Angesicht zu Angesicht beschuldigten. Sie glaubten, ein neues Gesetz einf�hren zu k�nnen,
so da� die �stlichen Bisch�fe von den westlichen beurteilt w�rden.\footnoteA{Vgl.
Dok. \ref{sec:BriefSynode341}. Vgl. Can.\,Serd. 3c; 4; 7.} Und sie wollten, da� das
Urteil der Kirche durch die gef�llt werde, die weniger die Taten jener bejammern als ihre eigenen. Da die kirchliche Ordnung dieses Unrecht niemals
angenommen hat, daher bitten wir, vielgeliebte Br�der, da� ihr die verbrecherische Seuche
und die todbringenden Versuche derer, die ins Verderben f�hren wollen, auch selbst mit uns
verurteilt.
\pend
\pstart
\kapR{13}Athanasius n�mlich hat damals, als er noch Bischof war, den abgesetzten Asclepas
selbst mit seiner eigenen Stimme verurteilt. Aber auch Markell hat gleicherma�en niemals
mit jenem Gemeinschaft gehalten. Paulus aber war bei der Absetzung des Athanasius anwesend.
Mit eigener Hand schrieb er seinen Urteilsspruch und verurteilte ihn selbst zusammen mit
den �brigen.\footnoteA{Dies ist der einzige Beleg f�r die Unterst�tzung der Verurteilung des Athanasius in Tyrus 335 durch Paulus von Konstantinopel. Paulus wird bald darauf selbst mehrmals abgesetzt, auch wenn er zeitweilig nach Konstantinopel zur�ckkehren kann (vgl. zu den Unruhen in Konstantinopel wegen Paulus Ath., h.\,Ar. 7; fug., 3,6; Socr., h.\,e. II 6~f.12~f.15~f.22.26; Soz., h.\,e. III 3~f.).} Und solange ein jeder von ihnen Bischof war,
hat er seine Urteile best�tigt. Als aber aus verschiedenen Gr�nden und zu verschiedenen
Zeiten die einzelnen von ihnen zu Recht f�r ihre Taten aus der Kirche vertrieben wurden,
schlossen sie in gemeinsamer Verschw�rung eine gr��ere Eintracht und vergaben sich
gegenseitig die Vergehen, die sie, als sie noch Bisch�fe waren, mit g�ttlicher Autorit�t
verurteilt hatten.
\pend
\pstart
\kapR{14}Denn da Athanasius auf seinen Reisen durch Italien und Gallien 
nach dem Tod einiger Ankl�ger, Zeugen und Richter die Verhandlungen f�r sich vorbereitete
und glaubte, er k�nne nochmals geh�rt werden zu einer Zeit, in der seine Schandtaten durch die lange Zwischenzeit verdunkelt worden waren, ~-- zu Unrecht gew�hrten ihm Julius, der Bischof Roms, Maximinus, Ossius und einige andere von ihnen ihm Unterst�tzung und haben es in die Wege geleitet, da� durch die G�te des Kaisers eine Synode bei Serdica stattfand~--, da reisten wir auf den Brief des Kaisers hin als Gruppe nach Serdica. Als wir dort ankamen, mu�ten wir erkennen, da� Athanasius, Markell, alle Verbrecher, die durch das Urteil
der Synode vertrieben worden waren, und alle die Personen, die f�r ihre Untaten zu Recht
verurteilt worden waren, mit Ossius und Protogenes zusammen in der Mitte der Kirche sitzen
und diskutieren und~-- was noch schlimmer ist~-- die heiligen Mysterien feiern. Auch
Protogenes, der Bischof von Serdica, sch�mte sich nicht, mit dem H�retiker Markell
Gemeinschaft zu halten, dessen Partei und verwerf"|liche Ansicht er selbst mit eigener Stimme
auf der Synode verurteilt hatte, indem er viermal die Entscheidungen der Bisch�fe
unterschrieb. Daher ist es erwiesen, da� er, Protogenes, sich selbst durch seine eigene Entscheidung verurteilt hat, da er sich zum Partner von jenem machte und mit ihm Gemeinschaft
hielt.
\pend
\pstart
\kapR{15}Wir aber, die wir uns an die Ordnung der kirchlichen Regel hielten und den Elenden
ein klein wenig helfen wollten, haben es jenen nahegelegt, die mit Protogenes und Ossius
zusammen waren, da� sie die Verurteilten aus ihrer Zusammenkunft ausschl�ssen und keine
Gemeinschaft mit S�ndern hielten. Hierauf sollten sie gemeinsam mit uns anh�ren, welche
Urteile von unseren V�tern gegen sie in der Vergangenheit gef�llt worden waren. Denn das
Buch des Markell brauchte keine Anklage~-- man erkannte n�mlich per se, da� es offenkundig h�retisch war~--, und keiner von ihnen verbarg wegen seines ">Ansehens"< als Bischof seine v�llig verdrehte Gesinnung, damit sie ja nicht falschen Vorstellungen glaubten. Doch jene widersetzten sich unserem Anliegen~-- aus welchem Grund, wissen wir nicht~-- und wollten sich nicht von ihrer Gemeinschaft trennen, st�rkten damit die Sekte des H�retikers Markell und zogen die Untaten des Athanasius und die Verbrechen der �brigen
dem kirchlichen Glauben und dem Frieden vor.
\pend
\pstart
\kapR{16}Nachdem wir dies erkannt hatten, konnten wir 80 Bisch�fe\footnoteA{Socr., h.\,e. II
20 und Soz., h.\,e. III 12 nennen 76 Bisch�fe. Die Unterschriftenliste ergibt 73 Namen; zu
erg�nzen sind aber z.\,B. noch Maris von Chalcedon und Ursacius von Singidunum, die nach � 19 anwesend gewesen sein m�ssen.}, die wir, um den Frieden der Kirche zu bewahren, aus
verschiedenen und weit entfernten Provinzen unter riesigem Aufwand und M�he nach
Serdica gekommen waren, diesen Anblick nicht ohne Tr�nen ertragen. Es war
n�mlich nicht leicht auszuhalten, da� sie sich g�nzlich weigerten, die von sich
wegzuschicken, die unsere V�ter zu Recht f�r ihre Verbrechen zuvor verurteilt hatten. Wir
hielten es daher f�r falsch, mit diesen Gemeinschaft zu haben, und wollten die heiligen
Sakramente des Herrn nicht mit Frevlern teilen, sondern der Ordnung der kirchlichen Regel
dienen und sie halten. Denn da� sie die Obengenannten, Ossius und Protogenes, nicht von
sich wegschickten, geschah vielleicht durch ein schlechtes Gewissen, durch das sich ein
jeder von ihnen f�rchtete, von irgendjemandem verraten zu werden, und sie f�rchteten sehr,
da� dies �ffentlich geschehen k�nnte. Und so wagte keiner von ihnen, gegen sie seine
Meinung zu sagen, damit er nicht auch sich selbst verurteilte, da sie trotz Verbot mit jenen
Gemeinschaft gehabt hatten, mit denen sie in keiner Weise Gemeinschaft h�tten haben
sollen.
\pend
\pstart
\kapR{17}Aber wir baten jene immer wieder, da� sie nicht Starkes und
Best�ndiges zerschl�gen, nicht das Gesetz auf den Kopf stellten und die g�ttlichen Rechte
in Unordung br�chten, nicht alles zerst�rten und die Tradition der Kirche auch nicht im
geringsten Ma� hintergingen, ferner keine neue Sekte einf�hrten und auch nicht die aus dem Westen Kommenden in irgendeiner Beziehung den �stlichen Bisch�fen und der heiligsten Synoden voranstellten. Sie jedoch verachteten unser Anliegen, drohten uns und k�ndigten an, da�
sie Athanasius und die �brigen Verbrecher in Schutz nehmen w�rden; es gab f�r sie, die alle Verbrecher und Verdorbenen in ihre Gemeinschaft aufgenommen hatten, wohl auch keine M�glichkeit,
etwas anders zu tun oder zu sagen. Sie brachten dies au�erdem mit ungeheuerer �berheblichkeit
vor, mehr von Frechheit und Planlosigkeit besessen als von rechter Vernunft. Als sie
n�mlich sahen, da� sie, die jene Verurteilten aufnehmen, selbst der Ungnade und Anklage
ausgesetzt sind als �bertreter der himmlischen Gesetze, versuchten sie einen
Gerichtshof mit solcher Befugnis zusammenzustellen, da� sie sich selbst Richter der Richter nannten und das Urteil derer, die bereits bei Gott sind, wenn man so sagen darf, neu aufrollen wollten. Wir aber baten jene sehr viele Tage hindurch, die Verurteilten von sich
zu sto�en, sich mit der heiligen Kirche zu verbinden und mit den V�tern, die die
Wahrheit sprachen, �bereinzustimmen. Aber jene weigerten sich g�nzlich, dies zu tun.
\pend
\pstart
\kapR{18}W�hrend wir also diskutierten, traten f�nf Bisch�fe von unserer Seite
auf, die von den sechs �briggeblieben sind, die in die
Mareotis gesandt worden waren.\footnoteA{Zu den Teilnehmern der Mareotis-Kommission vgl. Dok.
\ref{sec:BriefJuliusII},32. Theognis von Niz�a war inzwischen verstorben (vgl. zu
Theognis Dok. \ref{ch:TheognisNiz}).} Und sie unterbreiteten ihnen einen Vorschlag von der Art, da�
man einige Bisch�fe aus beiden Versammlungen zu den Orten schicken m�sse, an denen
Athanasius die Verbrechen und Untaten begangen hatte, damit sie unter der Zeugenschaft Gottes
alles getreulich aufschrieben; und falls das, was wir der Synode berichtet hatten, falsche
Erfindungen gewesen sein sollten, w�ren wir selbst verurteilt und w�rden uns nicht bei den
Kaisern, einer Synode oder irgendeinem Bischof beklagen. Wenn sich aber als wahr
herausstellen sollte, was wir zuvor gesagt hatten, la�t uns zus�tzlich zu unserer Zahl diejenigen
von euch ausschlie�en, die ihr erw�hlt habt, das hei�t die, die mit Athanasius nach seiner
Verurteilung Gemeinschaft hatten, aber auch die, die Athanasius und Markell gewogen sind
und sie verteidigen, ausschlie�en, und keiner von euch soll sich bei den Kaisern,
einer Synode oder irgendeinem Bischof beklagen. Diesen von uns vorgelegten
Vorschlag f�rchteten Ossius, Protogenes und alle ihre Anh�nger anzunehmen.
\pend
\pstart
\kapR{19}Ein ungeheuer gro�er Anteil von allen Verbrechern und Verdorbenen war aus Konstantinopel
und Alexandrien gekommen und nach Serdica gestr�mt, ferner solche, die wegen Mord,
Blutvergie�en, Totschlag, Raub- und Beutez�gen, Pl�nderungen, aller uns�glicher
Sakrilegien und Verbrechen angeklagt waren und die Alt�re zerschlugen, Kirchen anz�ndeten
und die H�user von Privatleuten pl�nderten und die Mysterien Gottes entweihten, auch
Verr�ter der Sakramente Christi, welche beharrlich bei der frevelhaften und verbrecherischen Lehre der H�retiker im Gegensatz zum Glauben der Kirche bleiben und in wildem Mord die weisesten
Presbyter Gottes, Diakone und Bisch�fe ermordeten. Und diese alle hatten Ossius und
Protogenes bei sich in ihrem kleinen Kreis versammelt. Und w�hrend sie diese ehrten, verachteten sie uns, alle Diakone und Priester Gottes, weil wir von unserer Seite aus irgendwann mit solchen Leuten keine Gemeinschaft mehr pflegen wollten. Sie verbreiteten bei der Masse und allen Heiden unsere internen Angelegenheiten. Dabei erfanden sie L�gen anstelle der Wahrheit und erz�hlten, da�
der Zwist zwischen uns nicht aus einem Streit um Gott heraus entstanden sei,
sondern aus menschlichem Vorurteil. Sie vermischten G�ttliches mit Menschlichem und verbanden die
kirchlichen mit den privaten Angelegenheiten, so stifteten sie die B�rgerschaft zum Beifall und Aufruhr f�r bzw. gegen uns an, indem sie sagten, da� wir ein Schisma verursachen und so schweres Unrecht in die B�rgerschaft hineintragen w�rden, wenn wir nicht mit jenen~-- was
aber Unrecht w�re~-- Gemeinschaft halten w�rden, und taten dies h�ufig lautstark kund.
Denn wir wollten absolut mit jenen keine Gemeinschaft haben, solange sie nicht jene, die wir
verurteilt hatten, von sich weisen und der Synode des Ostens die angemessene Ehre erweisen
w�rden.
\pend
\pstart
\kapR{20}Was sie aber selbst taten oder welche Art von Synode sie abhielten, werdet ihr
hier nun erfahren k�nnen. Denn Protogenes, der bei den Verhandlungen damals laut den Akten, wie
wir oben berichtet haben, Markell und Paulus verdammt hatte, nahm sie sp�ter in seine
Gemeinschaft auf. Dionysius von Elis in der Provinz Achaia,\footnoteA{Dionys von
Elis, Teilnehmer der ">westlichen"< Synode von Serdica (vgl. Unterschriftenliste), vgl. Ath.,
apol.\,Const. 3,6. Die Hintergr�nde seiner fr�heren Verurteilung sind unklar.} den sie
selbst ausgeschlossen hatten, war auf der Synode dabei. Ausschlie�ende und
Ausgeschlossene, Richter und Angeklagte hielten zugleich miteinander Gemeinschaft und
vollzogen die g�ttlichen Mysterien. Bassus aber aus der Stadt
Diocletianopolis\footnoteA{Bassus, Teilnehmer der ">westlichen"< Synode von Serdica (vgl.
Unterschriftenliste). Auch bei ihm sind die Hintergr�nde seiner fr�heren Verurteilung
unklar. Er unterschrieb sp�ter die Erkl�rung von Sirmium 351 (Hil., coll.\,antiar. B VII 9).}
weihten sie zum Bischof, obwohl er bei seinen Schandtaten und Verbrechen entlarvt
worden und verdienterma�en aus Syrien verbannt worden war. Dieser scheint heute mit
ihnen verbunden zu sein, obwohl er bei ihnen selbst �berf�hrt worden ist, einen allzu
verbrecherischen Lebenswandel zu f�hren, und von ihnen pers�nlich verurteilt worden ist. Dem
A"etius aus Thessalonike\footnoteA{In Thessalonike gab es offensichtlich ein Schisma mit
zwei Gegenbisch�fen (Eutychius und Musaeus), und dieser Einzelfall wurde auf der Synode
diskutiert und mit gesonderten Canones entschieden (Can.\,Serd. 19; 20), da� A"etius zwar
rechtm��iger Bischof sei, aber die von den Gegenbisch�fen geweihten Kleriker nach einer
�berpr�fung �bernommen werden k�nnen.} warf Protogenes aber h�ufig seine vielen
Schandtaten und Verbrechen vor, da er sagte, da� jener Konkubinen gehabt habe und noch
habe. Daher wollte er niemals mit ihm Gemeinschaft haben. Nun aber wieder in die
Freundschaft aufgenommen, gleichsam durch die Gemeinschaft mit noch Schlechteren
gereinigt, wird er bei ihnen wie ein Gerechter gesch�tzt. Asclepas aber, der wegen
Paulus nach Konstantinopel gekommen war, nachdem dieser ungeheuerliche und gr��liche Dinge
begangen hatte, die mitten in der Kirche von Konstantinopel geschehen waren, nach tausend
Morden, die sogar die Alt�re mit menschlichem Blut getr�nkt hatten, nach der Ermordung
der Br�der und der Ausl�schung der Heiden, weicht heute nicht davon ab, mit Paulus, dessentwegen 
diese Dinge geschehen waren, Gemeinschaft zu halten, aber auch jene nicht, die durch
Asclepas mit Paulus Gemeinschaft halten, indem sie von Paulus Briefe annehmen und an Asclepas
welche schicken.
\pend
\pstart
\kapR{21}Welch eine Synode h�tte also bei diesem Gemenge und bei dieser Ansammlung von verlorenen Seelen gehalten werden k�nnen, auf der sie n�mlich ihre eigenen Schandtaten nicht bestraften, sondern vielmehr anerkannten! 
Anders n�mlich als wir, die wir den heiligsten Kirchen vorsitzen und das Kirchenvolk
lenken, vergeben und �bergehen sie F�lle, bei denen eigentlich niemals
Vergebung und Nachla� gew�hrt werden kann. 
Die n�mlich vergeben auch dem Markell und dem Athanasius und den �brigen Verbrechern die Schandtaten und H�resien, die es Unrecht gewesen w�re nachzulassen,
da geschrieben steht: ">Wenn ein Mensch gegen einen
Menschen ges�ndigt hat, wird man f�r ihn zum Herrn beten. Wenn ein Mensch aber gegen Gott
ges�ndigt hat, wer wird dann f�r ihn beten? Weder wir noch die Kirche Gottes haben eine
solche Gewohnheit."< 
Vielmehr dulden wir weder, da� einige dies lehren, noch da� sie neue
�berlieferungen einf�hren, damit wir nicht Verr�ter des Glaubens und der Heiligen
Schriften genannt werden, was Unrecht ist, und somit nicht vom Herrn und von den Menschen
verurteilt werden.
\pend
\pstart
\kapR{22}Aber jene oben Genannten f�hrten auch folgendes gegen uns im Schilde: Da sie
wu�ten, da� wir ihre Verbrechen nicht mit Gemeinschaft belohnen k�nnen,
glaubten sie, uns durch die Briefe der Kaiser erschrecken zu k�nnen, um uns gegen unseren
Willen zur Gemeinschaft zu zerren. Sie beabsichtigten, da� der tiefe und ewige Friede der ganzen
Welt und der Kirche wegen Athanasius und Markell zerst�rt werde, durch die der
Name des Herrn bei den V�lkern entweiht wird. Wenn sie in sich auch nur einen Funken
Gottesfurcht gehabt h�tten, w�re es angemessen gewesen~-- damit diese Verwirrung, die durch sie entstanden ist, nicht andauere~--, 
da� sie von ihrem verkehrten Vorurteil, wenn auch sehr sp�t, Abstand nehmen,
damit sich nicht ihretwegen die Kirche gegen sich richtet und aufspaltet. Oder falls bei denen, die f�r
sich selbst k�mpfen, Gottesfurcht w�re, m��te man sie, auch wenn man nichts
Verurteilenswertes bei ihnen finden w�rde, dennoch ausschlie�en und vor ihnen zur�ckschrecken, da wegen
ihnen die Einheit der Kirche zerbrochen und aus wahnhafter und rasender Begierde
nach Ruhm der tiefe Friede zerst�rt wird.
\pend
\pstart
\kapR{23}Als wir sahen, wie die Dinge so liefen, entschied ein jeder von uns, in seine
Heimat zur�ckzukehren, und wir beschlossen, �ber Serdica zu berichten und das, was geschehen
ist, kundzutun und unsere Meinung darzulegen. Wir k�nnen n�mlich Athanasius und Markell,
die gottlos gegen den Herrn gefrevelt, einen verbrecherischen Lebenswandel gef�hrt haben und
einst ausgeschlossen und verurteilt worden waren, nicht wieder in ihr Bischofsamt
einsetzen. Die versahen den Sohn Gottes mit so derben Schl�gen, da� sie ihn
nochmals kreuzigten und ihn von neuem der �ffentlichkeit preisgaben. Der eine von ihnen
n�mlich ist ein f�r allemal gestorben im ewigen Tod, da er gegen den Sohn Gottes und gegen sein
Reich frevelte.\footnoteA{Gemeint ist Markell.} Der andere ist nach dem Urteil der
Bisch�fe ausgeschlossen und verurteilt worden, weil er gegen den Leib des Herrn und seine
Geheimnisse durch sein frevelhaftes Verhalten w�st s�ndigte und weitere ungeheuerliche
Schandtaten beging.\footnoteA{Gemeint ist Athanasius.} Daher, da wir von der
�berlieferung der V�ter nicht abweichen k�nnen, da die Kirche eine solche Autorit�t nicht
erhalten und eine solche Macht nicht von Gott empfangen hat, nehmen wir selbst die oben
Genannten nicht in die Ehre und W�rde der Kirche auf und verurteilen die, die sie aufnehmen,
zu Recht. Aber auch die anderen, die einst oder sp�ter zu Recht verurteilt
worden sind, nehmen wir nicht in die Kirche auf, da wir an den Gesetzen Gottes, an den �berlieferungen der V�ter und an den kirchlichen Lehren festhalten, da wir dem
Propheten glauben, der sagt: ">�berschreite nicht die ewigen Grenzen, die deine V�ter
gesetzt haben."< Daher zerschlagen wir niemals festgesetzte Traditionen, sondern bewahren
vielmehr das, was von den V�tern eingerichtet worden ist.
\pend
\pstart
\kapR{24}Nach so vielen Ereignissen also, vielgeliebte Br�der, empfehlen wir euch ganz offen,
da� keiner von euch irgendwann von irgendeinem umgarnt wird und mit ihnen Gemeinschaft h�lt,
das hei�t mit Ossius, Protogenes, Athanasius, Markell, Asclepas, Paulus und Julius,
sondern mit keinem der verurteilten und von der heiligen Kirche ausgesto�enen Personen
selbst und nicht mit ihren Anh�ngern, die mit ihnen pers�nlich oder schriftlich
Gemeinschaft pflegen. Daher d�rft ihr weder jemals an diese schreiben noch von ihnen
Schriftliches entgegennehmen. Was nun noch �brigbleibt, vielgeliebte Br�der, wir bitten
euch, da� ihr euch um die Einheit der Kirche und den immerw�hrenden Frieden sorgt, heilige
Bisch�fe ausw�hlt, bei denen reiner Glaube und heiliges Leben zu finden ist, und die
von euch weist, die f�r ihre Verbrechen vom Bischofsamt entbunden wurden und wieder den Platz
erhalten wollen, den sie zu Recht als Lohn f�r ihre Untaten verloren haben. St�ck f�r St�ck
sondert die aus, die ihr nach einer Schandtat noch Schlimmeres begehen seht und die den Herrn
nicht sagen h�ren: ">Schweig, du hast ges�ndigt!"< Sie erstarken n�mlich und werden
kr�ftiger durch ihre Verbrechen, und je mehr sie im Strudel der Laster versinken, desto
mehr stellen sie den ganzen Erdkreis auf den Kopf. Indem sie sich Zeit nehmen f�r
Aufst�nde, tragen sie Kriege und schlimmste Verfolgungen in die heiligen Kirchen hinein
und streben in tyrannischer Art danach, das Volk Gottes unter ihre Herrschaft zu bringen.
\pend
\pstart
\kapR{25}Aus diesen Tatsachen erkennt also ihre so verderblichen Anstrengungen,
als sie ein solch st�rmisches Unwetter �ber die Welt brachten, durch das sie beinahe den ganzen Osten und
Westen in Unruhe versetzten. Das hatte zur Folge, da� ein jeder von uns die kirchlichen Obliegenheiten
vernachl�ssigte, das Volk Gottes im Stich lie�, die Lehre des Evangeliums selbst erst einmal
hintanstellte und von weither hierherkam, vom Alter her ehrw�rdige Greise, k�rperlich
gebrechlich und von Krankheit geschw�cht. Wir zogen durch viele verschiedene Gebiete und lie�en
die Unseren krank auf den Reisen zur�ck wegen sehr weniger Verbrecher, die
einst zu Recht verurteilt wurden und gegen das Recht den Vorsitz in der Kirche fordern.
Und das Reich ist nun mit uns Christen besch�ftigt, und die frommen Kaiser, Tribune und Duces plagen sich mit den grausigsten Staatsangelegenheiten herum, die aus dem Leben und der Stellung der Bisch�fe herr�hren. Aber auch das Kirchenvolk schweigt nicht. Denn jede Gemeinschaft in allen Provinzen h�lt den Atem an und erwartet gespannt, auf welches Ende dieser Ungl�ckssturm
hinausl�uft. Und selbst die �ffentliche Post wird aufgerieben und zunichte gemacht.
Was soll man noch mehr sagen? Von Ost bis West wird die Welt wegen eines oder zweier und
weniger Verbrecher, die eine gottlose Gesinnung haben, sch�ndlich leben, und bei denen
keine Samen der Religion �briggeblieben sind, v�llig auf den Kopf gestellt und mit einem harten und w�tenden Sturm in Unruhe versetzt. Wenn sie diese Samen noch h�tten, w�rden sie den
Propheten nachahmen, der sagt: ">Nehmt mich und werft mich ins Meer, und ihr werdet das
Meer beruhigen, da dieser Sturm meinetwegen geschehen ist."< Aber sie ahmen dieses Vorbild deswegen
nicht nach, da sie nicht den Gerechten nachfolgen. So aber streben die Anf�hrer aller Verbrecher nach der Vorherrschaft �ber die Kirche wie bei einer tyrannischen Gewaltherrschaft.
\pend
\pstart
\kapR{26}Sie veranstalten dieses Unternehmen auch nicht um das Gut der Gerechtigkeit willen.
Die, die versuchen, die Gesetze, g�ttlichen Rechte und Beschl�sse der anderen aufzul�sen, sorgen
sich nicht um die Kirchen. Deswegen strebten sie danach, diese Neuheit einzuf�hren, 
vor der die alte Gewohnheit der Kirche erschaudert, damit von den westlichen Bisch�fen wieder aufgehoben werden k�nne, was immer die �stlichen Bisch�fe zuf�llig auf der Synode beschlossen haben, und entsprechend auch damit von den �stlichen aufgel�st werden k�nne, was immer die Bisch�fe der westlichen Gebiete beschlossen haben. Dies betrieben sie aber aus ihrem
eigenen, v�llig verdorbenen Verstand heraus. Die Berichte unserer Vorg�nger zeigen dagegen �bereinstimmend, da� die Beschl�sse aller mit Fug und Recht gef�hrten Synoden zu best�tigen sind. Denn in Rom fand zur
Zeit der H�retiker Novatian, Sabellius und Valentinus eine Synode statt, und die ist von den
�stlichen Bisch�fen akzeptiert worden.\footnoteA{Zu Novatian vgl. Dok. \ref{sec:BriefSynode341},1; zu Sabellius vgl. Dok. \ref{sec:Eustathius},1; zu Valentinus vgl. Iren., haer. III 4,3. Vgl. auch Dok. \ref{sec:BriefJuliusII}, 20.} Und im Gegenzug
ist von allen das unterzeichnet worden, was zur Zeit des Paulus von Samosata
im Osten beschlossen wurde.\footnoteA{Darauf wurde schon in Dok. \ref{sec:BriefSynode341}
verwiesen.} Daher ermahnen wir euch, vielgeliebte Br�der, bedenkt die Ordnung der
kirchlichen Lehre und sorgt euch um den Frieden auf dem gesamten Erdkreis,
tadelt die, die mit den Verbrechern Gemeinschaft halten, und rei�t die Schlechten mit 
Wurzel aus der Kirche aus, auf da� Christus, der Herr, �ber das Unwetter hinwegschreitet, das
ihretwegen niedergeht, allen Winden und Seest�rmen befiehlt zu weichen und der heiligen
Kirche dauerhaften Frieden und Ruhe gew�hrt.
\pend
\pstart
\kapR{27}Wir aber haben keinem Unrecht getan, sondern bewahren die Vorschriften des
Gesetzes. Denn wir haben schweres Unrecht erlitten und sind schlimm behandelt
worden von ihnen, die die Regel der katholischen Kirche durch ihre Verdorbenheit verwirren
wollten. Aber die Gottesfurcht vor Augen und das wahre und gerechte Urteil Christi
im Sinn haben wir keine Person aufgenommen und niemandem Schonung gew�hrt, um die
kirchliche Disziplin zu bewahren. Daher hat die gesamte Synode Julius von Rom, Ossius,
Protogenes, Gaudentius und Maximinus von Trier nach dem altehrw�rdigen Gesetz verurteilt, da sie
die Urheber f�r die Gemeinschaft mit Markell, Athanasius und den �brigen Verbrechern
waren, die auch in die Morde des Paulus von Konstantinopel und seine grausamen Taten
involviert waren. Protogenes n�mlich mu� man gemeinsam mit Markell aus der
Kirchengemeinschaft ausschlie�en, da er mehrmals das Urteil unterschrieb, das gegen Markell selbst
oder gegen dessen Buch gef�llt worden war, und sie gaben auch beide ihre Stimme gegen
Paulus von Konstantinopel ab und hatten sp�ter wieder Gemeinschaft mit ihm.
Gaudentius\footnoteA{Nachfolger von Cyriacus, Bischof von Na"issus. Er stand zun�chst der
�stlichen Delegation nahe, wechselte aber die Seiten (� 4, dort aber noch Cyriacus genannt);
vgl. Soz., h.\,e. III 11,8. In Serdica als
Anreger f�r Can. 4 und 11 bezeugt. Nach Serdica reist Athanasius zu ihm nach Na"issus
(Ath., apol.\,Const. 3,3); er unterschrieb sp�ter die Erkl�rung von Sirmium 351 (Hil., coll.\,antiar. B
VII 9).} aber ist auszuschlie�en, da er das Andenken an seinen Amtsvorg�nger
Cyriacus, der die zu Recht gegen die Verbrecher eingebrachten Antr�ge unterschrieben
hatte, nicht bewahrte und in die Verbrechen des Paulus einbezogen war, den er ebenfalls ohne Scham verteidigte. Julius von Rom m�ssen wir ohne Zweifel ausschlie�en als
Obersten und Anf�hrer der Schurken, der als erster die T�r zur Gemeinschaft mit den
Verbrechern und Verurteilten ge�ffnet hatte und den �brigen erst die M�glichkeit gab, die g�ttlichen Rechte aufzul�sen. Er verteidigte den Athanasius dreist und k�hn, einen Menschen,
dessen Zeugen und Ankl�ger er nicht kannte. Den Ossius hingegen schlie�en wir aus oben
genanntem Grund aus und wegen Marcus,\footnoteA{Offensichtlich ein schon verstorbener
Gegner des Ossius, sonst unbekannt.} seligen Gedenkens, dem er stets schweren
Schaden zuf�gte, aber auch, weil er alle Schurken, die f�r ihre Verbrechen
zu Recht verurteilt worden waren, mit allen Kr�ften verteidigte, und weil er im Osten mit
Verbrechern und Verdorbenen zusammengelebt hat. Denn sch�ndlicherweise war er ein
unzertrennlicher Freund des Paulinus, des einstigen Bischofs von
Dacia,\footnoteA{Paulinus aus Dacia ist sonst unbekannt.} eines Menschen, der zuerst als
Magier angeklagt und aus der Kirche versto�en wurde und bis zum heutigen Tag in
seiner Apostasie verharrt und �ffentlich mit Konkubinen und Prostituierten hurt. Seine
B�cher voll von Zauberspr�chen hat Macedonius, der Bischof und Bekenner von
Mopsuestia\footnoteA{Makedonius von Mopsuestia; vgl. Dok. \ref{sec:BriefJuliusII},32.},
verbrannt. Aber auch dem Eustathius und dem Cymatius\footnoteA{Eustathius von Antiochien,
vgl. Dok. \ref{sec:Eustathius}; Cymatius von Paltus, vgl. Ath., fug. 3,3; h.\,Ar. 5,2 und tom. 1.}
hing er (Paulinus) auf �belste Weise an und war ihnen teuer. �ber deren verruchtes und sch�ndliches
Leben sollte man kein Wort verlieren; ihr Ende hat sie bei allen
bekannt gemacht. Mit solchen und �hnlichen Leuten hat sich Ossius also von Beginn an verbunden, und indem er fortw�hrend die Verbrecher unterst�tzte, stellte er sich gegen die Kirche und gew�hrte stets den
Feinden Gottes Hilfe. Maximinus von Trier aber schlie�en wir deswegen aus, weil er
unsere Bischofskollegen, die wir nach Gallien geschickt hatten, nicht aufnehmen
wollte.\footnoteA{Vgl. die Umst�nde von Dok. \ref{ch:AntIV}.} Au�erdem stand er als erster pers�nlich mit Paulus von Konstantinopel, dem frevelhaften und verdorbenen Menschen, in Gemeinschaft und war selbst der Grund f�r so gro�es Unheil, da� Paulus nach Konstantinopel
zur�ckgerufen wurde, dessentwegen viele Morde geschehen sind. Er, der den einst verurteilten Paulus nach Konstantinopel zur�ckrief, war also selbst Ursache so schlimmer Morde.
\pend
\pstart
\kapR{28}Aus diesen Gr�nden hielt es also die Synode f�r gerechtfertigt, Julius von Rom, Ossius
und die �brigen oben Genannten ihres Amtes zu entheben und zu verurteilen. Unter diesen
Umst�nden m��t ihr euch vor ihnen h�ten und von ihnen fernhalten, vielgeliebte Br�der, und
d�rft sie niemals in euere Gemeinschaft aufnehmen, sondern weder ihre Briefe
entgegennehmen noch Herrenbriefe an sie schicken. Und da jene, die
mit Ossius zusammen waren, den katholischen und apostolischen Glauben zerbrechen wollten und
die neue Sekte des mit den Juden im Bunde stehenden Markell einf�hrten, die
judaisierende Mischung mit der Lehre des Sabellius und Paulus, haben wir
notwendigerweise den Glauben der katholischen Kirche ordentlich abgeschrieben, den die oben
Genannten leugneten, die mit Ossius zusammen sind und statt dessen den Glauben des H�retikers Markell einf�hrten. Daraus folgt, da� jeder einzelne von euch, sobald ihr unsere Briefe erhalten
habt, diesem Urteil zustimmen und unsere Beschl�sse mit eigener Unterschrift abzeichnen soll.
\pend
\endnumbering
\end{translatio}
\end{Rightside}
\Columns
\end{pairs}
% \thispagestyle{empty}

%%% Local Variables: 
%%% mode: latex
%%% TeX-master: "dokumente_master"
%%% End: 

%%%% Input-Datei OHNE TeX-Pr�ambel %%%%
% \renewcommand*{\goalfraction}{.7}
\section{Theologische Erkl�rung der ">�stlichen"< Synode}
% \label{sec:43.2}
\label{sec:BekenntnisSerdikaOst}
\begin{praefatio}
\begin{description}
\item[Herbst 343] Zum Datum vgl. oben die Einleitung in
  Dok. \ref{ch:SerdicaEinl}.
\item[�berlieferung] Die theologische Erkl�rung der ">�stlichen"<
  Synode wiederholt die sog. vierte antiochenische Erkl�rung
  (Dok. \ref{ch:AntIV}), f�gt aber noch einige weitere Anathematismen
  an (vgl. S. \refpassage{similiter1}{ecclesia1}). In dieser Form wird sie
  dann in der Ekthesis makrostichos (Dok. \ref{ch:Makrostichos})
  wieder aufgenommen. Es f�llt auf, da� die Autoren, die die Ekthesis
  makrostichos zitieren, die Ekthesis der ">�stlichen"< Synode nicht
  anf�hren, und umgekehrt diejenigen, die diese zitieren, jene nicht
  anf�hren.

  Es liegt nahe, da� auch diese Erkl�rung urspr�nglich griechisch
  abgefa�t war. Ein Hinweis daf�r k�nnte sein, da� an mehreren Stellen
  lateinische Synonyme verwendet werden, z.\,B. \textit{de},
  \textit{a} und \textit{ex} oder \textit{regnum} und
  \textit{imperium} oder \textit{vel} und \textit{aut} (vgl. auch die
  im folgenden genannten Unterschiede).
  Der Text ist zweimal durch Hilarius (in coll.\,antiar. und syn. in
  unterschiedlichen �bersetzungen), durch den Codex Veronensis und den
  Codex Parisinus syr. 62 �berliefert. Dabei unterscheidet sich die
  Version in coll.\,antiar. signifikant von den drei �brigen Versionen:
  \textit{coll.\,antiar.}: Das Pr�skript fehlt, weil der Text in
  einem �berlieferungszusammenhang mit Dok. \ref{sec:RundbriefSerdikaOst} tradiert ist.
  Die �bersetzung aus dem Griechischen ist sehr w�rtlich, eine lateinische
  bereits gepr�gte Terminologie scheint nicht verwendet worden zu sein
  (Stegreif-�bersetzung?), so wird \textit{institutorem} anstelle von
  \textit{factorem} f�r \griech{ktist'hc} verwendet, \textit{aevum} (2x)
  anstelle von \textit{saeculum} f�r \griech{a>'iwn},
  \textit{generatus est} anstelle von \textit{genitus est} f�r
  \griech{gennhj'enta}, \textit{sermo} anstelle von \textit{verbum}
  f�r \griech{l'ogoc}, \textit{induerit hominem} anstelle von
  \textit{homo factus est} f�r \griech{>enanjrwp'hsanta},
  \textit{adsumptus} anstelle von \textit{receptus} f�r
  \griech{>analhfj'enta}, \textit{incessabile} anstelle von
  \textit{sine cessatione} f�r \griech{>akat'apaustoc}; auch sonst finden sich andere
  Formulierungen als in den �brigen lateinischen Fassungen: \textit{adsumtionem suam in caelum} statt
  \textit{reditum in caelos} bzw. \textit{in caelum ascensionem} und
  \textit{et instruere de omnibus} statt \textit{ac memorari omnia}
  bzw. \textit{et monere omnia}, \textit{haereticos damnat} statt
  \textit{alienos novit}, \textit{hos omnes anathematizat et execratur
    sancta et catholica ecclesia} statt \textit{anathematizat sancta
    et catholica ecclesia}. Schlie�lich ist dem dritten Artikel hinzugef�gt:
  \textit{credimus et in sanctam ecclesiam, in remissam peccatorum, in
    carnis resurrectionem, in vitam aeternam}.
  Aber auch die anderen Versionen weisen spezifische Eigenheiten auf: 
  \textit{Cod.\,Ver. und \griech{S}} gehen beide auf eine gemeinsame Vorlage
  zur�ck. So unterscheiden sie sich beide in der Liste der genannten Provinzen; es wird \textit{propter
    quem et per quem} statt nur \textit{per quem} gesetzt, \textit{potestas}
  statt \textit{virtus}; nach \textit{lumen verum}
  wird \textit{sempiternum in omnibus} hinzugef�gt; Es hei�t \textit{sedens semper} statt
  \textit{sedet enim} bzw. \textit{est enim sedens} und \textit{spiritum
    sanctum dei} statt \textit{spiritum sanctum}; au�erdem wird zweimal der
  Geist und schlie�lich ein weiterer Anathematismus hinzugef�gt:
  \textit{fecit vel creavit vel declaravit vel intellectu est verbum
    dei quod omnia scit}. Daneben gibt es auch Sonderfehler bei
  beiden.
  In der Fassung in \textit{syn.} steht \textit{incarnatus est} statt \textit{homo factus
    est}; \textit{in fine mundi} wird ausgelassen. Es hei�t \textit{filium dei}
  statt \textit{filium}, \textit{eum ipsum}
  statt \textit{eundem}  bzw. \textit{ipsum} und \textit{innascibilem}
  statt \textit{non natum}.
  Insgesamt betrachtet scheint die �berlieferung durch syn. dem
  urspr�nglichen Text am ehesten zu gleichen; daher wurde dieser
  �berlieferung bei der Textkonstitution der Vorzug gegeben.

Das Verh�ltnis der vier Fassungen zueinander l��t sich in folgendem Stemma darstellen: 

  \begin{figure}[h] % Stemma
    \begin{scriptsize}
      \begin{center}
        \begin{tikzpicture}
          \node (Q) at (0,0) {\griech{W}}; 
          \node (VorlageA) at (1,-.5) {\griech{a}}; 
          \node (VorlageB) at (2,-1) {\griech{b}}; 
          \node (Hilcoll) at (-3,-1) {Hil., coll.antiar}; 
          \node (Hilsyn) at (0,-1) {Hil., syn.}; 
          \node (CodVer) at (3,-2) {Cod. Ver.};
          \node (Syr) at (1.5,-2) {\griech{S}}; 
          \draw (Q) -- (VorlageA); 
          \draw (VorlageA) -- (VorlageB); 
          \draw (VorlageA) -- (Hilsyn); 
          \draw (Q) -- (Hilcoll); 
          \draw (VorlageB) -- (Syr); 
          \draw (VorlageB) -- (CodVer);
        \end{tikzpicture}
      \end{center}
      %\legend{Stemma}
    \end{scriptsize}
  \end{figure}

  Um die �bersicht zu wahren, wurde darauf verzichtet die
  �berlieferung der Collectanea antiariana und von De synodis weiter
  aufzuschl�sseln. F�r die Collectanea antiariana, De synodis und den Codex Veronensis LX wurde die Textkonstitution der verwendeten Ausgaben herangezogen.
  Bei \griech{S} ist auf Grund der Unterschiede zwischen der
  lateinischen und syrischen Syntax nicht immer eindeutig zu
  entscheiden, welcher Text�berlieferung der syrische �bersetzer
  folgt.
\item[Fundstelle]Hil., syn. 34 (\cite[507A--C]{Hil:Syn}); Hil., coll.\,anitar. A IV 2 (\editioncite[68--73]{Hil:coll}); Cod. Veronensis LX f. 79a; Cod. Parisinus syr. 62 f. 185a--b
  (\editioncite[167,1--168,11]{Schulthess:Kanones})
\end{description}
\end{praefatio}
\begin{pairs}
\selectlanguage{latin}
\begin{Leftside}
% \beginnumbering
% \pstart
% \edtext{\abb{Fides secundum Orientis synodum}}{\lemma{+ Fides secundum Orientis synodum}
% \Dfootnote{\textit{syn.\LitNil}}}
% \pend
\pstart
\hskip -1.2em\edtext{\abb{}}{\killnumber\Cfootnote{\hskip -.95em Hil.(syn) Hil.(coll.\,anitar.)
Cod.\,Ver. \griech{S}}}\specialindex{quellen}{section}{Hilarius!syn.!34}\specialindex{quellen}{section}{Hilarius!coll.\,antiar.!A IV 2}\specialindex{quellen}{section}{Codices!Veronensis LX!f. 79a}\specialindex{quellen}{section}{Codices!Parisinus syr. 62!f. 185a--b}
\kap{1}\edtext{\abb{Sancta synodus \edtext{in Serdica congregata}{\linenum{|1|||||}\Dfootnote{congregata
est Sardice \textit{Cod.\,Ver.} \griech{S}}} ex diversis provinciis \edtext{Orientalium partium
Thebaida, Aegypto, Palestina, Arabia, Phoenice, Syria Coele, Mesopotamia, Cilicia,
Cappadocia, Ponto, Paphlagonia, Galatia, Bithynia, Hellesponto, Asia, Phrygiis duabus,
Pisidia, Cycladum insularum, Pamphylia, Caria, Lydia, Europe, Thracia, Emimonto,
Mysia, Pannoniis duabus}{\lemma{\abb{Orientalium partium \dots\ Pannoniis duabus}}\Dfootnote{\textit{syn.} de partibus Orientis, Thebaidis, Aegyti, Palestinae, Phoenices,
Syriae Coeles, Mesopotamiae, Cilicae, Cappadociae,
Ponti, Pephlagoniae, Galatiae, Phrygiae (> \griech{S}), Bithynie, Helisponto, Asiae,
Pisidiae (+ Phrygiae \griech{S}), Insularum, Panphyliae, Camae (Caraeae \griech{S}), Lydiae, Europae, Thraciae,
Hemimontis (Ponti \griech{S}), Pannoniae, Moesiae, Daciae \textit{Cod.\,Ver.} \griech{S}}}, hanc
\edtext{exposuimus}{\Dfootnote{exposuerunt \textit{Cod.\,Ver.}
    \griech{S}}}
fidem}}{{\lemma{\abb{}}\linenum{|1|||1||}\Dfootnote{Titel
  \textit{Synodicum Sardicae} \griech{S}}}\lemma{Sancta
synodus \dots\ fidem}\linenum{|1|||11||}\Dfootnote{Est autem fides nostra talis \textit{coll.\,antiar.}}}.
\edindex[namen]{Serdica}\edindex[namen]{Theba"is}\edindex[namen]{Aegyptus}\edindex[namen]{Arabia}\edindex[namen]{Palaestina}
\edindex[namen]{Phoenice}\edindex[namen]{Syria coele}\edindex[namen]{Mesopotamia}\edindex[namen]{Cilicia}\edindex[namen]{Cappadocia}\edindex[namen]{Galatia}\edindex[namen]{Pontus}\edindex[namen]{Paphlagonia}\edindex[namen]{Bithynia}\edindex[namen]{Hellespontus}\edindex[namen]{Asia}\edindex[namen]{Phrygia}\edindex[namen]{Pisidia}\edindex[namen]{Cyclades}
\edindex[namen]{Pamphylia}\edindex[namen]{Caria}\edindex[namen]{Lydia}\edindex[namen]{Europa}\edindex[namen]{Thracia}
\edindex[namen]{Moesia}\edindex[namen]{Pannonia}\edindex[namen]{Haemimons}
\pend
\pstart
\kap{2}\edtext{credimus}{\lemma{\abb{}}\Dfootnote{Titel \textit{symbolum fidei} \griech{S}}}
\edtext{\abb{in}}{\Dfootnote{> \textit{coll.\,antiar.}}}
\edtext{\abb{unum deum}}{\Afootnote{vgl. 1Cor 8,6; Eph 4,6}}\edindex[bibel]{Korinther I!8,6|textit}\edindex[bibel]{Epheser!4,6|textit} patrem
\edtext{\abb{omnipotentem}}{\Afootnote{vgl. Apc 1,8 u.�.}}\edindex[bibel]{Offenbarung!1,8|textit},
\edtext{\abb{\edtext{\abb{creatorem}}{\Afootnote{vgl. 1Petr 4,19}} et factorem
omnium}}{\Dfootnote{\griech{S} creatorem et
factorem universorum \textit{syn.} institutorem et creatorem omnium
\textit{coll.\,antiar.} creatorem omnium et factorem \textit{Cod.\,Ver.}}}\edindex[bibel]{Petrus I!4,19|textit},
\dt{">}\edtext{ex quo
\edtext{\abb{omnis paternitas}}{\Dfootnote{\textit{syn.} \griech{S} omnis creatura \textit{coll.\,antiar.} omnia \textit{Cod.\,Ver.}}} in
caelo et in terra
\edtext{nominatur}{\Dfootnote{nominantur \textit{Cod.\,Ver.}}}}{\lemma{\abb{}}\Afootnote{Eph 3,15}}\dt{"<},\edindex[bibel]{Epheser!3,15}
\pend
\pstart
\kap{3}\edtext{\abb{et}}{\lemma{\abb{}} \Dfootnote{+ credimus \textit{ante} et \textit{syn.}}} in
\edtext{\edtext{\abb{unigenitum}}{\Afootnote{vgl. Io 1,14.18; 3,16; 1Io
4,9}}}{\lemma{unigenitum}\Dfootnote{unum genitum \textit{coll.\,antiar.} unicum
\textit{Cod.\,Ver.}}}\edindex[bibel]{Johannes!1,14}\edindex[bibel]{Johannes!1,18|textit}\edindex[bibel]{Johannes!3,16|textit}\edindex[bibel]{Johannes I!4,9|textit}
\edtext{eius filium}{\lemma{\abb{}} \Dfootnote{\responsio\ filium eius
\textit{coll.\,antiar. Cod.\,Ver.}}} dominum nostrum Iesum Christum,
\edtext{\abb{qui}}{\Dfootnote{> \textit{Cod.\,Ver.}}}
\edtext{ante
\edtext{omnia saecula}{\Dfootnote{aevum \textit{coll.\,antiar.}}}}{\lemma{\abb{omnia saecula}}\Afootnote{vgl.
1Cor 2,7}}\edindex[bibel]{Korinther I!2,7|textit} ex patre
\edtext{genitus est}{\Dfootnote{generatus est \textit{coll.\,antiar.} natum
\textit{Cod.\,Ver.}}},
\edtext{deum ex deo}{\Dfootnote{et ex deo deus \textit{coll.\,antiar.} deum de deo
\textit{Cod.\,Ver.}}},
\edtext{\edtext{\abb{lumen}}{\Afootnote{vgl. Io 1,4~f.9; 1Io 2,8}} ex lumine}{\Dfootnote{ex luce lux \textit{coll.\,antiar.} lumen de
lumine \textit{Cod.\,Ver.}}},\edindex[bibel]{Johannes!1,4~f.|textit}
\edindex[bibel]{Johannes!1,9|textit}\edindex[bibel]{Johannes I!2,8|textit}
\edtext{\edtext{per quem facta sunt omnia}{\Dfootnote{propter quem et per quem omnia facta sunt \griech{S} per quem omnia et propter
quem omnia facta sunt \textit{Cod.\,Ver.}}}}{\lemma{\abb{per \dots\ omnia}}\Afootnote{vgl. 1Cor 8,6; Col 1,16; Hebr 1,2; Io 1,3}},\edindex[bibel]{Korinther I!8,6|textit}\edindex[bibel]{Kolosser!1,16|textit}\edindex[bibel]{Hebraeer!1,2|textit}
\edindex[bibel]{Johannes!1,3|textit}
\edtext{\edtext{quae in caelis et quae in terra}{\Dfootnote{in celo et in terra
\textit{Cod.\,Ver.} \griech{S}}}}{\lemma{\abb{quae \dots\ terra}}\Afootnote{Col 1,16; Eph
1,10}}\edindex[bibel]{Kolosser!1,16}\edindex[bibel]{Epheser!1,10},
\edtext{\abb{visibilia et invisibilia}}{\Afootnote{Col 1,16}}\edindex[bibel]{Kolosser!1,16}:
\edtext{qui est
\edtext{\abb{verbum}}{\Afootnote{Io 1,1~f.14; 1Io 1,1}}}{\lemma{qui est verbum}\Dfootnote{sermo qui sit
\textit{coll.\,antiar.} verbum
\textit{Cod.\,Ver.}}}\edindex[bibel]{Johannes!1,1~f.}\edindex[bibel]{Johannes!1,14}
\edindex[bibel]{Johannes I!1,1}
\edtext{et}{\lemma{\abb{et\ts{1}}}\Dfootnote{> \textit{Cod.\,Ver.}}}
\edtext{\abb{sapientia}}{\Afootnote{vgl. 1Cor 1,24.30}}
\edindex[bibel]{Korinther I!1,24|textit}\edindex[bibel]{Korinther I!1,30|textit}
\edtext{et}{\lemma{\abb{et\ts{2}}}\Dfootnote{> \textit{coll.\,antiar.}}}
\edtext{\edtext{\abb{virtus}}{\Afootnote{vgl. 1Cor 1,24}}}{\lemma{virtus}\Dfootnote{potestas
\textit{Cod.\,Ver.} \griech{S}}}\edindex[bibel]{Korinther I!1,24|textit}
\edtext{\abb{et
\edtext{\abb{vita}}{\Afootnote{vgl. Io 1,4; 11,25; 14,6; Col 3,4; 1Io 1,2; 5,20}}
et}}{\Dfootnote{> \textit{Cod.\,Ver.}}}
\edindex[bibel]{Johannes!1,4|textit}\edindex[bibel]{Johannes!11,25|textit}\edindex[bibel]{Johannes!14,6|textit}\edindex[bibel]{Kolosser!3,4|textit}\edindex[bibel]{Johannes I!1,2|textit}\edindex[bibel]{Johannes I!5,20|textit}
\edtext{\edtext{\abb{lumen verum}}{\Afootnote{Io 1,9; 1Io 2,8}}}{\lemma{lumen verum}\Dfootnote{lux vera
\textit{coll.\,antiar.} + sempiternum in
omnibus \textit{Cod.\,Ver.} \griech{S}}}\edindex[bibel]{Johannes!1,9}\edindex[bibel]{Johannes I!2,8},
qui 
\edtext{\edtext{\abb{in}}{\Dfootnote{> \textit{Cod.\,Ver.}}} novissimis
\edtext{diebus}{\Dfootnote{temporibus \textit{coll.\,antiar.
Cod.\,Ver.}}}}{\lemma{\abb{in \dots\ diebus}}\Afootnote{vgl. Hebr 1,2}}\edindex[bibel]{Hebraeer!1,2|textit} propter nos
\edtext{\edtext{\abb{homo factus est}}{\Afootnote{vgl. 1Cor 15,47}}}{\lemma{homo factus est}\Dfootnote{incarnatus est \textit{syn.}
induerit hominem \textit{coll.\,antiar.}}}\edindex[bibel]{Korinther I!15,47|textit} et
\edtext{\edtext{natus}{\Dfootnote{creatus \textit{Cod.\,Ver.} + sit
\textit{coll.\,antiar.}}}
\edtext{ex}{\Dfootnote{de \textit{coll.\,antiar.}}} sancta
\edtext{\abb{virgine}}{\Dfootnote{+ Maria \textit{coll.\,antiar.}}}}{\lemma{\abb{natus \dots\ virgine}}\Afootnote{vgl. Mt
1,23; Lc 1,27.34~f.}}\edindex[bibel]{Matthaeus!1,23|textit}\edindex[bibel]{Lukas!1,27|textit}\edindex[bibel]{Lukas!1,34~f.|textit},
\edtext{qui}{\Dfootnote{et \textit{coll.\,antiar.} > \textit{Cod.\,Ver.}}}
\edtext{\abb{crucifixus}}{\Afootnote{vgl. Mt 20,19; 27,22 u.\,�.; 1Cor 1,23; 2,2.8; 2Cor
13,4; Gal
3,1}}\edindex[bibel]{Matthaeus!20,19|textit}\edindex[bibel]{Matthaeus!27,22|textit}\edindex[bibel]{Korinther I!1,23|textit}\edindex[bibel]{Korinther I!2,2|textit}\edindex[bibel]{Korinther I!2,8|textit}\edindex[bibel]{Korinther II!13,4|textit}\edindex[bibel]{Galater!3,1|textit}
\edtext{\abb{est}}{\Dfootnote{> \textit{coll.\,antiar. Cod.\,Ver.}}} et
\edtext{\abb{mortuus}}{\Afootnote{1Cor 15,3}}\edindex[bibel]{Korinther I!15,3} et
\edtext{\edtext{\abb{sepultus}}{\Dfootnote{+ sit
\textit{coll.\,antiar.}}}}{\Afootnote{vgl. 1Cor 15,4}}\edindex[bibel]{Korinther I!15,4|textit}.
\edtext{et}{\lemma{\abb{et\ts{2}}}\Dfootnote{> \textit{coll.\,antiar.}}}
\edtext{\edtext{surrexit}{\Dfootnote{resurrexit \textit{Cod.\,Ver.}}}}{\lemma{\abb{surrexit}}\Afootnote{vgl. Mt
17,9par; 20,19par; 1Cor 15,4; 1Thess 4,14; auch Eph
1,20}}\edindex[bibel]{Matthaeus!17,9par|textit}\edindex[bibel]{Matthaeus!20,19par|textit}\edindex[bibel]{Korinther I!15,4|textit}\edindex[bibel]{Thessalonicher I!4,14|textit}\edindex[bibel]{Epheser!1,20|textit}
\edtext{ex}{\Dfootnote{a \textit{Cod.\,Ver. coll.\,antiar.}}} mortuis
\edtext{tertia die}{\Dfootnote{tertio die \textit{Cod.\,Ver.} \responsio\
\textit{ante} surrexit \textit{coll.\,antiar.}}}
\edtext{\edtext{et receptus}{\Dfootnote{et adsumptus \textit{coll.\,antiar.}
receptum
\textit{Cod.\,Ver.}}} in
\edtext{caelo}{\Dfootnote{caelum \textit{coll.\,antiar.} celis
\textit{Cod.\,Ver.}}}}{\lemma{\abb{et \dots\ caelo}}\Afootnote{Mc 16,19; Act
1,2}}\edindex[bibel]{Markus!16,19}\edindex[bibel]{Apostelgeschichte!1,2}
\edtext{\abb{est et}}{\Dfootnote{> \textit{coll.\,antiar. Cod.\,Ver.}}}
\edtext{\edtext{sedet}{\Dfootnote{sedentem \textit{Cod.\,Ver.}}}
\edtext{in}{\Dfootnote{ad \textit{Cod.\,Ver.}}}
\edtext{dextera}{\Dfootnote{dexteram \textit{Cod.\,Ver.}}} patris}{\lemma{\abb{sedet \dots\ patris}}\Afootnote{vgl.
Mc 16,19; Ps 110,1; Eph 1,20; Col 3,1; 1Petr 3,22;
Hebr 1,3}},\edindex[bibel]{Markus!16,19|textit}\edindex[bibel]{Psalmen!110,1|textit}\edindex[bibel]{Epheser!1,20|textit}\edindex[bibel]{Kolosser!3,1|textit}\edindex[bibel]{Petrus I!3,22|textit}\edindex[bibel]{Hebraeer!1,3|textit}
\edtext{venturus}{\Dfootnote{et venientem \textit{Cod.\,Ver.}}}
\edtext{in
\edtext{\abb{fine}}{\Afootnote{vgl. Mt 13,39; 24,3; 28,20}} mundi}{\Dfootnote{> \textit{syn.} in fine saeculorum \griech{S} in finem saeculi \textit{Cod.\,Ver.}}}\edindex[bibel]{Matthaeus!13,39|textit}\edindex[bibel]{Matthaeus!24,3|textit}
\edindex[bibel]{Matthaeus!28,20|textit}
\edtext{\abb{iudicare vivos et mortuos}}{\Afootnote{2Tim 4,1; 1Petr 4,5; vgl. Io 5,22;
Act 10,42}}\edindex[bibel]{Timotheus II!4,1}\edindex[bibel]{Petrus
I!4,5}\edindex[bibel]{Johannes!5,22|textit}\edindex[bibel]{Apostelgeschichte!10,42|textit} et
\edtext{reddere unicuique
secundum
\edtext{opera eius}{\Dfootnote{opera sua \textit{coll.\,antiar.} facta sua
\textit{Cod.\,Ver.}}}}{\lemma{\abb{reddere \dots\ eius}}\Afootnote{Mt 16,27; Rom 2,6; 2Tim 4,14; Apc 2,23;
22,12}},\edindex[bibel]{Matthaeus!16,27}\edindex[bibel]{Roemer!2,6}
\edindex[bibel]{Timotheus II!4,14}
\edindex[bibel]{Offenbarung!2,23}\edindex[bibel]{Offenbarung!22,12}
cuius
\edtext{regnum}{\Dfootnote{imperium \textit{Cod.\,Ver.}}}
\edtext{sine cessatione}{\Dfootnote{incessabile \textit{coll.\,antiar.}
sempiternum \textit{Cod.\,Ver.}}}
\edtext{permanet}{\Dfootnote{manebit \textit{Cod.\,Ver.}}}
\edtext{in inmensa saecula}{\Dfootnote{in aeterna saecula \textit{coll.\,antiar.} in eterna
saecula \textit{Cod.\,Ver.} \griech{S}}}.
\edtext{sedet enim}{\Dfootnote{est enim sedens \textit{coll.\,antiar.} sedens semper
\textit{Cod.\,Ver.} \griech{S}}}
\edtext{in dextera}{\Dfootnote{ad dexteram \textit{Cod.\,Ver.}}} patris non solum
in
\edtext{hoc}{\Dfootnote{isto \textit{coll.\,antiar.}}} saeculo,
\edtext{sed
\edtext{et}{\Dfootnote{etiam \textit{Cod.\,Ver.}}} in futuro}{\lemma{\abb{sed \dots\ futuro}}\Afootnote{vgl. Eph
1,21}}\edindex[bibel]{Epheser!1,21|textit}.
\pend
\pstart
\kap{4}credimus
\edtext{\abb{et}}{\Dfootnote{> \textit{coll.\,antiar.}}}
\edtext{in spiritum sanctum}{\Dfootnote{in (+ unum \textit{Cod.\,Ver.}) spiritum sanctum dei \textit{Cod.Ver} \griech{S}}}, hoc est
\edtext{\edtext{\abb{paraclitum}}{\Afootnote{Io 14,16.26; 15,26;
16,7.13}}}{\lemma{paraclitum}\Dfootnote{in paraclitum \textit{coll.\,antiar.} paracletum \textit{Cod.\,Ver.}}},\edindex[bibel]{Johannes!14,16}\edindex[bibel]{Johannes!14,26}\edindex[bibel]{Johannes!15,26}\edindex[bibel]{Johannes!16,7}\edindex[bibel]{Johannes!16,13}
quem
\edtext{\edtext{promittens}{\Dfootnote{pollicitus \textit{Cod.\,Ver.} pollicitus Christus \griech{S}}}
apostolis}{\lemma{\abb{promittens apostolis}}\Afootnote{vgl. Io 14,26; 15,26; Act 2,17--21.33; Tit 3,6}} post
\edtext{reditum in caelos}{\Dfootnote{adsumtionem suam in caelum
\textit{coll.\,antiar.} in celum ascensionem \textit{Cod.\,Ver.} \griech{S}}} misit docere
\edtext{eos}{\Dfootnote{illos \textit{coll.\,antiar.}}}
\edtext{ac memorari omnia}{\Dfootnote{et instruere de omnibus
\textit{coll.\,antiar.} et monere omnia \textit{Cod.\,Ver.}}}, per quem
\edtext{\abb{et}}{\Dfootnote{> \textit{coll.\,antiar.}}}
\edtext{\abb{sanctificantur}}{\Afootnote{vgl. Rom 15,16; 1Cor 6,11}}
\edtext{sincere in eum credentium animae}{\Dfootnote{animae in ipsum fideliter
credentes \textit{coll.\,antiar.} et vivunt religiosorum animae in eum credentes
\textit{Cod.\,Ver.} et sanctificant animas sincerorum sincere in eum credentium \griech{S} + credimus et in sanctam ecclesiam, in remissam peccatorum, in
carnis resurrectionem, in vitam aeternam
\textit{coll.\,antiar.}}}.\edindex[bibel]{Johannes!14,16}\edindex[bibel]{Johannes!14,26}\edindex[bibel]{Johannes!15,26}\edindex[bibel]{Johannes!16,7}\edindex[bibel]{Johannes!16,13}\edindex[bibel]{Johannes!14,26}\edindex[bibel]{Apostelgeschichte!2,17--21|textit}\edindex[bibel]{Apostelgeschichte!2,33|textit}
\edindex[bibel]{Johannes!15,26|textit}\edindex[bibel]{Titus!3,6|textit}\edindex[bibel]{Roemer!15,16|textit}\edindex[bibel]{Korinther I!6,11|textit}
\pend
\pstart
\kap{5}\edtext{eos autem, qui dicunt}{\Dfootnote{illos vero, qui dicunt
\textit{coll.\,antiar.} dicentes autem \textit{Cod.\,Ver.}}}
\edtext{de non exstantibus}{\Dfootnote{ex id, quod non fuit
\textit{coll.\,antiar.} ex nihilo \textit{Cod.\,Ver.}}}
\edtext{\abb{esse}}{\Dfootnote{> \textit{coll.\,antiar.}}}
\edtext{filium}{\Dfootnote{filium dei \textit{syn.} \griech{S}}}
\edtext{vel}{\Dfootnote{aut \textit{coll.\,antiar.}}} ex alia
\edtext{\abb{substantia}}{\Dfootnote{+ spiritum \textit{Cod.\,Ver.} \griech{S}}} et non ex
deo
\edtext{et}{\Dfootnote{aut qui dicunt \textit{coll.\,antiar.}}}
\edtext{quod erat}{\Dfootnote{erat \textit{Cod Ver.} fuisse \textit{coll.\,antiar.}}} aliquando tempus
\edtext{aut}{\Dfootnote{vel \textit{coll.\,antiar. Cod.\,Ver.}}} saeculum, quando
non
\edtext{erat}{\Dfootnote{fuit filius \textit{coll.\,antiar.}}},
\edtext{alienos novit}{\Dfootnote{haereticos damnat \textit{coll.\,antiar.}}}
sancta et catholica ecclesia. \edtext{\abb{similiter}}{\Dfootnote{> \griech{S}}}\edlabel{similiter1}
\edtext{et
\edtext{eos}{\Dfootnote{illos \textit{coll.\,antiar.}}}, qui
dicunt}{\Dfootnote{etiam dicentes \textit{Cod.\,Ver.}}} tres esse deos
\edtext{vel}{\Dfootnote{aut \textit{coll.\,antiar.}}} Christum non esse
\edtext{\abb{deum}}{\Dfootnote{+ sive spiritum non esse dei \textit{Cod.\,Ver.} \griech{S}}}
\edtext{et}{\Dfootnote{aut \textit{coll.\,antiar.} vel \textit{Cod.\,Ver.} \griech{S}}} ante
\edtext{saecula}{\Dfootnote{aevum \textit{coll.\,antiar.}}}
\edtext{neque Christum neque filium eum esse dei}{\Dfootnote{non fuisse Christum neque filium dei \textit{coll.\,antiar.} nec Christum nec filium nec spiritum dei esse \textit{Cod.\,Ver.} \griech{S}}}
\edtext{vel}{\Dfootnote{aut \textit{coll.\,antiar.}}}
\edtext{eundem}{\Dfootnote{ipsum \textit{coll.\,antiar.} eum ipsum
\textit{syn.}}}
\edtext{\abb{esse}}{\Dfootnote{> \textit{coll.\,antiar. Cod.\,Ver.}}} patrem et
filium et
\edtext{spiritum sanctum}{\lemma{\abb{}}\Dfootnote{\responsio\ sanctum spiritum
\textit{syn.}}}
\edtext{vel}{\Dfootnote{aut \textit{coll.\,antiar.}}}
\edtext{non natum}{\Dfootnote{innascibilem \textit{syn}}}
\edtext{\abb{filium}}{\Dfootnote{+ vel natum spiritum \textit{Cod.\,Ver.} \griech{S}}}
\edtext{vel quod neque}{\Dfootnote{aut non \textit{coll.\,antiar.} aut
\textit{Cod.\,Ver.} \griech{S}}}
\edtext{consilio}{\Dfootnote{sententia \textit{coll.\,antiar.} voluntate
\textit{Cod.\,Ver.}}}
\edtext{neque}{\Dfootnote{vel \textit{Cod.\,Ver.} \griech{S}}}
\edtext{voluntate}{\Dfootnote{arbitrio \textit{Cod.\,Ver.}}}
\edtext{pater genuerit filium}{\Dfootnote{pater genuit filium \textit{Cod.\,Ver.} deum patrem genuisse filium
\textit{coll.\,antiar.} + fecit sive creavit vel demonstravit, sed secundum
intellectum omnia scientem, verbum dei \textit{Cod.\,Ver.} + fecit vel creavit vel declaravit vel in intellectu est verbum dei quod omnia cum patre scit \griech{S}}},
\edtext{anathematizat sancta et catholica ecclesia}{\lemma{anathematizat \dots\
ecclesia} \Dfootnote{(+ etiam \griech{S}) hos omnes anathematizat (+ et execratur \textit{coll.\,antiar.}) sancta (+ et \textit{coll.\,antiar.}) catholica
ecclesia \textit{coll.\,antiar.} \textit{Cod.\,Ver.} \griech{S}}}.\edlabel{ecclesia1}
\pend
% \endnumbering
\end{Leftside}
\begin{Rightside}
\begin{translatio}
\beginnumbering
% \pstart
% Bekenntnis der �stlichen Synode
% \pend
\pstart
\noindent\kapR{1}Als heilige Synode, versammelt in Serdica aus den verschiedenen Provinzen der �stlichen
Gebiete, aus der Provinz Theba"is, Aegyptus, Palaestina, Arabia, Phoenice, Syria coelis,
Mesopotamia, Cilicia, Cappadocia, Pontus, Paphlagonia, Galatia, Bithynia,
Hellespontus, Asia, den beiden Provinzen Phrygia, der Provinz Pisidia, den Kykladischen Inseln,
der Provinz Pamphylia, Caria, Lydia, Europa, Thracia, Haemimons, Moesia und den beiden Provinzen
Pannonia haben wir folgenden Glauben dargelegt.
\pend
\pstart
\kapR{2}Wir glauben an den einen Gott, den Vater, den Allm�chtigen, den Sch�pfer
und Erschaffer von allem, nach dem alle Vaterschaft im Himmel und auf Erden
benannt wird, 
\pend
\pstart
\kapR{3}und an seinen eingeborenen Sohn, unsern Herrn Jesus Christus, der
vor allen Zeiten aus dem Vater gezeugt worden ist, Gott aus Gott, Licht aus
Licht, durch den alles im Himmel und auf Erden geschaffen wurde, das Sichtbare
und das Unsichtbare, der Wort, Weisheit, Kraft, Leben und wahres Licht ist, der
in den letzten Tagen f�r uns Mensch geworden und geboren worden ist aus der
heiligen Jungfrau, der gekreuzigt, gestorben und begraben worden ist und
am dritten Tag von den Toten auferstanden und in den Himmel aufgenommen worden ist
und zur Rechten des Vaters sitzt und der wiederkommen wird am Ende der Welt, um
die Lebenden und die Toten zu richten und jedem nach seinen Werken zu vergelten,
dessen Herrschaft unauf"|l�slich f�r alle Zeiten bestehen bleibt. Er sitzt n�mlich
zur Rechten des Vaters nicht nur in dieser Zeit, sondern auch in der k�nftigen.
\pend
\pstart
\kapR{4}Und wir glauben an den heiligen Geist, das hei�t den Beistand, den er gem�� seinem
Versprechen an die Apostel nach seiner R�ckkehr in den Himmel schickte, damit er
sie lehre und an alles erinnere, durch den auch die Seelen derer, die aufrichtig
an ihn glauben, geheiligt werden.
\pend
\pstart
\kapR{5}Die aber sagen, der Sohn sei aus nichts oder aus einer anderen Hypostase
geworden und nicht aus Gott, und es habe einmal eine Zeit oder eine Epoche gegeben, in der
er nicht war, die kennt die heilige und katholische Kirche als Fremde. Ebenso verdammt die
heilige und katholische Kirche die, die behaupten, da� es drei G�tter gebe oder da�
Christus nicht Gott sei und da� er vor den Zeiten weder Christus noch Sohn Gottes gewesen
sei oder da� Vater, Sohn und heiliger Geist derselbe seien oder da� der Sohn ungezeugt sei oder da�
der Vater den Sohn nicht nach seinem Ratschlu� und Willen gezeugt habe.
\pend
\endnumbering
\end{translatio}
\end{Rightside}
\Columns
\end{pairs}
% \thispagestyle{empty}
%%% Local Variables: 
%%% mode: latex
%%% TeX-master: "dokumente_master"
%%% End: 
%%%% Input-Datei OHNE TeX-Pr�ambel %%%%
% \renewcommand*{\goalfraction}{.9}
\section[Liste der Unterschriften unter der theologischen Erkl�rung der ">�stlichen"< Synode][Unterschriften unter der theologischen Erkl�rung der ">�stlichen"< Synode]{Liste der Unterschriften unter der theologischen Erkl�rung der ">�stlichen"< Synode}
% \label{sec:43.2}
\label{sec:NominaepiscSerdikaOst}
\begin{praefatio}
\begin{description}
\item[Herbst 343] Zum Datum vgl. Dok. \ref{ch:SerdicaEinl}.
\item[�berlieferung]Zur �berlieferung der Collectanea antiariana des
    Hilarius vgl. Dok.  \ref{ch:SerdicaEinl}.
\item[Fundstelle]Hil., coll.\,antiar. A IV 3 (\editioncite[74,1--78,10]{Hil:coll})
\end{description}
\end{praefatio}
\begin{pairs}
\selectlanguage{latin}
\begin{Leftside}
% \beginnumbering
\pstart
\hskip -1.2em\edtext{\abb{}}{\killnumber\Cfootnote{\hskip -.73em
Hil.(ACTS)}}\specialindex{quellen}{section}{Hilarius!coll.\,antiar.!A IV 3}
\edtext{[Opto vos in domino bene valere.]}{\lemma{opto \dots\ valere}\Dfootnote{\textit{del. Feder}}}
\pend
\pstart
\noindent 1. Stephanus\edindex[namen]{Stephanus!Bischof von Antiochien} episcopus
Antiochiae provinciae
\edtext{\abb{Siriae}}{\Dfootnote{\textit{coni. C vel edd.} Sirae
\textit{A}}}
\edtext{\abb{Coelae}}{\Dfootnote{\textit{coni. Engelbrecht} Coeles \textit{coni. Coustant}
Cylae \textit{A}}}, opto vos in domino bene
valere.
\pend
\pstart
\noindent 2. Olympius\edindex[namen]{Olympius!Bischof von Doliche} episcopus Doliceus, opto
vos in domino bene valere.
\pend
\pstart
\noindent 3. Gerontius\edindex[namen]{Gerontius!Bischof von Raphania} episcopus
\edtext{\abb{Raphaniae}}{\Dfootnote{\textit{coni. Hardouin}
Lamphaniae \textit{A}}},
\Ladd{\edtext{\abb{opto vos in domino bene valere}}{\Dfootnote{\textit{add. Feder}}}.}
\pend
\pstart
\noindent 4. Menofantus\edindex[namen]{Menophantus!Bischof von Ephesus} episcopus ab
Epheso, opto vos in domino bene valere.
\pend
\pstart
\noindent 5. Paulus\edindex[namen]{Paulus!Bischof} episcopus, opto vos in
domino bene valere.
\pend
\pstart
\noindent 6. Eulalius\edindex[namen]{Eulalius!Bischof von Amasia} episcopus
\edtext{\abb{ab Amasias}}{\Dfootnote{\textit{coni. Feder} ab Amatias
\textit{A} ab Amantia \textit{coni. C}}}, opto vos in domino
bene valere.
\pend
\pstart
\noindent 7. Machedonius\edindex[namen]{Macedonius!Bischof von Mopsuestia} episcopus a
\edtext{\abb{Mopsuestia}}{\Dfootnote{\textit{S\ts{2}} ab Marguetias
\textit{A}}}, opto vos in domino bene valere.
\pend
\pstart
\noindent 8. Thelafius\edindex[namen]{Thelaphius!Bischof von Chalcedon} episcopus a
Calchedonia, opto vos in domino bene valere.
\pend
\pstart
\noindent 9. Acacius\edindex[namen]{Acacius!Bischof von Caesarea} episcopus a Caesarea,
opto vos in domino bene valere.
\pend
\pstart
\noindent 10. Theodorus\edindex[namen]{Theodorus!Bischof von Heraclea} episcopus ab
\edtext{\abb{Heraclia}}{\Dfootnote{\textit{coni. C} Heradia \textit{A}}}, opto vos in
domino bene valere.
\pend
\pstart
\noindent 11. Quintianus\edindex[namen]{Quintianus!Bischof von Gaza} episcopus a
\edtext{\abb{Gaza}}{\Dfootnote{\textit{coni Coustant} Taxae
\textit{A} Taxe \textit{coni. C}}}, opto vos in domino bene valere.
\pend
\pstart
\noindent 12. Marcus\edindex[namen]{Marcus!Bischof von Arethusa} episcopus ab
\edtext{\abb{Aretusa}}{\Dfootnote{\textit{coni. Faber} Aretus \textit{A}}}, opto vos in domino bene
valere.
\pend
\pstart
\noindent 13. Cyrotus\edindex[namen]{Cyrotus!Bischof von Rhosus} episcopus a
\edtext{\abb{Roso}}{\Dfootnote{\textit{coni. Hardouin} Roeo \textit{A} Rhodo \textit{coni. Coleti}}}, opto vos in domino bene valere.
\pend
\pstart
\noindent 14. \edtext{Eugeus}{\Dfootnote{Eugenius \textit{coni. LeQuien I
1029}}}\edindex[namen]{Eugeus!Bischof von Lysinia} episcopus a
\edtext{\abb{Lisinia}}{\Dfootnote{\textit{coni. C vel edd.} Lysinia \textit{coni. LeQuien I 1029} Lisicia \textit{A}
Larissa \textit{coni. Hardouin}}}, opto vos in domino bene valere.
\pend
\pstart
\noindent 15. Antonius\edindex[namen]{Antonius!Bischof von Zeugma} episcopus a Zeumate,
opto vos in domino bene valere.
\pend
\pstart
\noindent 16. Antonius\edindex[namen]{Antonius!Bischof von Dokimion} episcopus a
\edtext{\abb{Docimio}}{\Dfootnote{\textit{coni. Feder} Docimo \textit{A}}}, opto vos in
domino bene valere.
\pend
\pstart
\noindent 17. \edtext{\abb{Dianius}}{\Dfootnote{\textit{coni. Hardouin} Dianaeus
\textit{coni. Coustant} Dion
\textit{A}}}\edindex[namen]{Dianius!Bischof von Caesarea} episcopus a Caesarea, opto vos in
domino bene valere.
\pend
\pstart
\noindent 18. Vitalis\edindex[namen]{Vitalis!Bischof von Tyrus} episcopus a Tyro, opto vos
in domino bene valere
\pend
\pstart
\noindent 19. \edtext{\abb{Eudoxius}}{\Dfootnote{\textit{coni. Coustant} Eudosius
\textit{A}}}\edindex[namen]{Eudoxius!Bischof von Germanicia}
episcopus a
\edtext{\abb{Germanicia}}{\Dfootnote{\textit{coni. Coustant} Germaniae \textit{A}
Germania \textit{coni. C}}}, opto vos in domino bene valere.
\pend
\pstart
\noindent 20. Dionisius\edindex[namen]{Dionysius!Bischof von Alexandria (ad Issum)} episcopus ab
Alexandria provinciae
\Ladd{\edtext{\abb{Ciliciae}}{\Dfootnote{\textit{add. Feder}}}},
opto vos in domino bene
valere.
\pend
\pstart
\noindent 21. Machedonius\edindex[namen]{Macedonius!Bischof von Berytos} episcopus a
\edtext{\abb{Birito}}{\Dfootnote{\textit{coni. Feder} Virito \textit{AS\ts{2}} Beryto
\textit{coni. Faber}}}, opto vos in domino bene valere.
\pend
\pstart
\noindent 22. Eusebius\edindex[namen]{Eusebius!Bischof von Dorylaeum} episcopus
\edtext{\abb{ab Dorilaio}}{\Dfootnote{\textit{coni. Feder} ab Dorlaiu \textit{A} ab
Dorylaeo \textit{coni. Hardouin} a Dorlani \textit{coni. Faber}}}, opto vos in domino
bene valere.
\pend
\pstart
\noindent 23. Basilius\edindex[namen]{Basilius!Bischof von Ancyra} episcopus ab Anquira,
opto vos in domino bene valere.
\pend
\pstart
\noindent 24. Prohaeresius\edindex[namen]{Prohaeresius!Bischof von Sinope} episcopus a
Sinopa, opto vos in domino bene valere.
\pend
\pstart
\noindent 25. Eustatius\edindex[namen]{Eustathius!Bischof von Epiphaneia} episcopus ab
Epiphania, opto vos in domino bene valere.
\pend
\pstart
\noindent 26. Pancratius\edindex[namen]{Pancratius!Bischof von Parnasus} episcopus a
\edtext{\abb{Parnaso}}{\Dfootnote{\textit{coni. Coustant} Pernaso \textit{A}}},
opto vos in domino bene valere.
\pend
\pstart

\pend
\pstart
\noindent 27. Eusebius\edindex[namen]{Eusebius!Bischof von Pergamum} episcopus a Pergamo,
opto vos in domino bene valere.
\pend
\pstart
\noindent 28. Sabinianus\edindex[namen]{Sabinianus!Bischof von Cadimena} episcopus a
\dag\edtext{\abb{Chadimena}}{\Dfootnote{\textit{coni. Feder} Chatimera \textit{A}
Cathimera \textit{coni. Coustant} Immeria \textit{coni. Hardouin}}}, opto
vos in domino bene valere.
\pend
\pstart
\noindent 29. Bitinicus\edindex[namen]{Bitinicus!Bischof von Zela} episcopus a
Zelon, opto vos in domino bene
valere.
\pend
\pstart
\noindent 30. Dominicus\edindex[namen]{Dominicus!Bischof von Polidiane} episcopus a
\dag\edtext{\abb{Polidiane}}{\Dfootnote{\textit{coni. Feder} Palladianu
\textit{A} Palladiano \textit{coni. C}}}, opto vos in domino bene valere.
\pend
\pstart
\noindent 31. \edtext{\abb{Pison}}{\Dfootnote{\textit{coni. Faber} Pyson
\textit{A}}}\edindex[namen]{Pison!Bischof von Trocnada}
episcopus a
\edtext{\abb{Trocnada}}{\Dfootnote{\textit{coni. Feder} Troada
\textit{A} Tenedo \textit{coni. Hardouin}}}, opto vos in domino bene
valere.
\pend
\pstart
\noindent 32. Cartherius\edindex[namen]{Carterius!Bischof von Aspona} episcopus ab Aspona,
opto vos in domino bene valere.
\pend
\pstart
\noindent 33. Filetus\edindex[namen]{Philetus!Bischof von Iuliopolis} episcopus a
\edtext{Iuliopoli}{\Dfootnote{ab Heliopoli \textit{coni. Hardouin}}},
opto vos in domino bene valere.
\pend
\pstart
\noindent 34. \edtext{\abb{Squirius}}{\Dfootnote{\textit{coni. Feder} Quirius \textit{A}}}\edindex[namen]{Ischyras!Presbyter in der Mareotis} episcopus a
\edtext{\abb{Mareota}}{\Dfootnote{\textit{coni. Faber} Marcota \textit{A}}},
opto vos in domino bene valere.
\pend
\pstart
\noindent 35. Filetus\edindex[namen]{Philetus!Bischof von Cratia} episcopus a Cratia, opto
vos in domino bene valere.
\pend
\pstart
\noindent 36. Timasarcus\edindex[namen]{Timasarcus!Bischof} episcopus, opto vos in
domino bene valere.
\pend
\pstart
\noindent 37. Eusebius\edindex[namen]{Eusebius!Bischof von Magnesia} episcopus a
\edtext{\abb{Magnesia}}{\Dfootnote{\textit{coni. Hardouin} Magnenia
\textit{AS\ts{2}} Magnienia \textit{coni. Coustant}}}, opto vos in domino bene valere.
\pend
\pstart
\noindent 38. \edtext{Quirius}{\Dfootnote{Cyriacus \textit{coni. LeQuien I
868}}}\edindex[namen]{Quirius!Bischof von Philadelphia} episcopus a
Filadelfia, opto vos in domino bene valere.
\pend
\pstart
\noindent 39. Pison\edindex[namen]{Pison!Bischof von Adana} episcopus
\edtext{\abb{ab Adanis}}{\Dfootnote{\textit{vel} a Varnis \textit{coni. Coustant} a Vanis
\textit{A} a Batnis \textit{coni. Hardouin}}}, opto vos in domino bene
valere.
\pend
\pstart
\noindent 40. Thimotheus\edindex[namen]{Timotheus!Bischof} episcopus, opto vos in
domino bene valere.
\pend
\pstart
\noindent 41. Eudemon\edindex[namen]{Eudaemon!Bischof von Tanis} episcopus a Thaneos, opto vos
in domino bene valere.
\pend
\pstart
\noindent 42. \edtext{\abb{Callinicus}}{\Dfootnote{\textit{coni. Hardouin} Callenicus
\textit{A}}}\edindex[namen]{Callinicus!Bischof von Pelusion}
episcopus a
\edtext{\abb{Pelusio}}{\Dfootnote{\textit{coni. Faber} Pelosio \textit{A}}},
opto vos in domino bene valere.
\pend
\pstart
\noindent 43. Eusebius\edindex[namen]{Eusebius!Bischof von Pergamum} episcopus
\edtext{\abb{a Pergamo}}{\Dfootnote{\textit{coni. C} a Pergamum \textit{A}}}, opto vos in domino bene valere.
\pend
\pstart
\noindent 44. \edtext{Leucadas}{\Dfootnote{Leucadius \textit{coni. LeQuien I
775}}}\edindex[namen]{Leucadas!Bischof von Ilion} episcopus ab
Ilio, opto vos in domino bene valere.
\pend
\pstart
\noindent 45. \edtext{\abb{Niconius}}{\Dfootnote{\textit{coni. LeQuien I 777} Noconius
\textit{A}}}\edindex[namen]{Niconius!Bischof der Troas} episcopus
\edtext{a Troados}{\Dfootnote{a Troadas \textit{A} a Troade \textit{coni. Coustant}}}, opto vos in domino bene
valere.
\pend
\pstart
\noindent 46. Adamantius\edindex[namen]{Adamantius!Bischof von Chios} episcopus a Cio, opto
vos in domino bene valere.
\pend
\pstart

\pend
\pstart
\noindent 47. Edesius\edindex[namen]{Edesius!Bischof von Kos} episcopus a Coo, opto vos in
domino bene valere.
\pend
\pstart
\noindent 48. \edtext{\abb{Theodulus}}{\Dfootnote{\textit{coni. Hardouin} Theodolus
\textit{A}}}\edindex[namen]{Theodul!Bischof von Neocaesarea}
episcopus a
\edtext{\abb{Neocaesarea}}{\Dfootnote{\textit{coni. Binius}
Neocaesarioso \textit{A}}}, opto vos in domino bene valere.
\pend
\pstart
\noindent 49. Sion\edindex[namen]{Sion!Bischof} episcopus, opto vos in domino bene
valere.
\pend
\pstart
\noindent 50. Theogenes\edindex[namen]{Theogenes!Bischof von Lycia} episcopus a
\edtext{\abb{Licia}}{\Dfootnote{\textit{coni. Feder} Lizia \textit{A}
Bizya \textit{coni. Hardouin} Byzia \textit{coni. Mansi}}}, opto vos in domino bene
valere.
\pend
\pstart
\noindent 51. Florentius\edindex[namen]{Florentius!Bischof von Ancyra} episcopus ab
\edtext{\abb{Ancyra}}{\Dfootnote{\textit{coni. Hardouin} Ancara
\textit{AS\ts{2}}}}, opto vos in domino bene valere.
\pend
\pstart
\noindent 52. Isaac\edindex[namen]{Isaak!Bischof von Letos} episcopus a
\edtext{\abb{Leto}}{\Dfootnote{\textit{coni. Coustant} Lueto {A}}}, opto vos in domino bene
valere.
\pend
\pstart
\noindent 53. Eudemon\edindex[namen]{Eudaemon!Bischof} episcopus --
\edtext{\abb{suscripsi}}{\Dfootnote{\textit{coni. C vel edd.} suscripso \textit{A}}} pro ipso -- opto vos in
domino bene valere.
\pend
\pstart
\noindent 54. Agapius\edindex[namen]{Agapius!Bischof von Tinos} episcopus a
\edtext{\abb{Theno}}{\Dfootnote{\textit{coni. Feder} Thenco
\textit{AS\ts{2}} Therinco \textit{S\ts{1}} Theriaco \textit{coni. C} Therico
\textit{coni. Faber} Therito \textit{coni. Hardouin}}}, opto vos in domino bene
valere.
\pend
\pstart
\noindent 55. Bassus\edindex[namen]{Bassus!Bischof von Carpathus} episcopus a
\edtext{\abb{Carpatho}}{\Dfootnote{\textit{coni. Hardouin} Car \textit{A}}}, opto vos in domino bene valere.
\pend
\pstart
\noindent 56. Narcissus\edindex[namen]{Narcissus!Bischof von Neronias} episcopus ab
\edtext{\abb{Irenopoli}}{\Dfootnote{\textit{vel} Neronopoli \textit{coni. Coustant} Anapoli
\textit{A} Amphipoli \textit{coni. LeQuien II 83}}}, opto vos in domino bene valere.
\pend
\pstart
\noindent 57. \edtext{\abb{Ambracius}}{\Dfootnote{\textit{coni. LeQuien I 919} Embracius
\textit{A}}}\edindex[namen]{Ambracius!Bischof von Milet} episcopus
\edtext{\abb{a Mileto}}{\Dfootnote{\textit{S\ts{2}} ami lecto (i \textit{del.})
\textit{A} ab Lecto \textit{T\corr} ab Nilecto \textit{T\ts{2}}}}, opto vos in domino
bene valere.
\pend
\pstart
\noindent 58. Lucius\edindex[namen]{Lucius!Bischof von Antinus} episcopus ab
\edtext{\abb{Antinoo}}{\Dfootnote{\textit{coni. Hardouin} ab Babtino
\textit{A}}}, opto vos in domino bene valere.
\pend
\pstart
\noindent 59. Nonnius\edindex[namen]{Nonnius!Bischof von Laodicea} episcopus a Laudocia,
opto vos in domino bene valere.
\pend
\pstart
\noindent 60. Pantagatus\edindex[namen]{Pantagathus!Bischof von Attalia} episcopus
\edtext{ab Attalias}{\Dfootnote{a Gallias \latintext \textit{A} ab Attalia \textit{coni. Hardouin} a
Gallia \textit{coni. Binius} a Gallatia \textit{coni. Coustant}}}, opto vos in domino
bene valere.
\pend
\pstart
\noindent 61. Flaccus\edindex[namen]{Flaccus!Bischof von Hierapolis} episcopus ab Ieropoli,
opto vos in domino bene valere.
\pend
\pstart
\noindent 62. \edtext{\abb{Sisinnius}}{\Dfootnote{\textit{coni. Coustant} Sysimius
\textit{A}}}\edindex[namen]{Sisinnius!Bischof von Perge} episcopus
a Perge, opto vos in domino bene valere.
\pend
\pstart
\noindent 63. Diogenes\edindex[namen]{Diogenes!Bischof} episcopus, opto vos in domino
bene valere.
\pend
\pstart
\noindent 64. Cresconius\edindex[namen]{Cresconius!Bischof} episcopus, opto vos in
domino bene valere.
\pend
\pstart
\noindent 65. Nestorius \edindex[namen]{Nestorius!Bischof}episcopus, opto vos in
domino bene valere.
\pend
\pstart
\noindent 66. \edtext{\abb{Ammonius}}{\Dfootnote{\textit{coni. Feder} Emmonius
\textit{A}}}\edindex[namen]{Ammonius!Bischof}
episcopus, opto vos in domino bene valere.
\pend
\pstart

\pend
\pstart
\noindent 67. Eugenius\edindex[namen]{Eugenius!Bischof} episcopus, opto vos in domino
bene valere.
\pend
\pstart
\noindent 68. Antonius\edindex[namen]{Antonius!Bischof von Bosra} episcopus a
\edtext{\abb{Bosra}}{\Dfootnote{\textit{coni. Feder} Gusra
\textit{A} Gusia \textit{coni. C}}}, opto vos in domino bene valere.
\pend
\pstart
\noindent 69. \edtext{\abb{Demofilus}}{\Dfootnote{\textit{coni. Faber} \latintext
Demofilius
\textit{A}}}\edindex[namen]{Demophilus!Bischof von Beroea} episcopus a Beroe, opto vos in
domino bene valere.
\pend
\pstart
\noindent 70. Euticius\edindex[namen]{Eutychius!Bischof von Philippopolis} episcopus a
Filippopoli, opto vos in domino bene valere.
\pend
\pstart
\noindent 71. Severus\edindex[namen]{Severus!Bischof von Gabula} episcopus a
\edtext{\abb{Gabula}}{\Dfootnote{\textit{coni. Feder} Cabula
\textit{A} Gabala \textit{coni. LeQuien II 797}}}, opto vos in domino bene valere.
\pend
\pstart
\noindent 72. Thimotheus\edindex[namen]{Timotheus!Bischof von Anchialus} episcopus ab
\edtext{\abb{Ancialo}}{\Dfootnote{\textit{coni. Feder} Anchialo
\textit{coni. Hardouin} Ancylu \textit{A} Ancylo \textit{S\ts{2}} Anquilo
\textit{coni. Faber} Anchilo \textit{coni. Coustant}}}, opto vos in domino bene valere.
\pend
\pstart
\noindent 73. Valens\edindex[namen]{Valens!Bischof von Mursa} episcopus a Mursa, opto vos
in domino bene valere.
\pend
%\pstart
%Explicit decretum synodi Orientalium apud Serdicam\edindex[namen]{Serdica} episcoporum a parte
%Arrianorum\edindex[namen]{Arius, Presbyter von Alexandrien},
%\edtext{\abb{quod}}{\Dfootnote{\textit{coni. C} quem \textit{A}}} miserum ad Africam\edindex[namen]{Afrika}.
%\pend
\endnumbering
\end{Leftside}
\begin{Rightside}
\begin{translatio}
\beginnumbering
\pstart

\pend
\pstart
\noindent 1. Ich, Stephanus, Bischof von Antiochia in der Provinz Koilesyrien, w�nsche euch
im Herrn alles Gute.
\pend
\pstart
\noindent 2. Ich, Olympius, Bischof von Doliche, w�nsche euch im Herrn alles Gute.
\pend
\pstart
\noindent 3. Ich, Gerontius, Bischof von Raphania, w�nsche euch im Herrn alles Gute.
\pend
\pstart
\noindent 4. Ich, Menophantus, Bischof von Ephesus, w�nsche euch im Herrn alles Gute.
\pend
\pstart
\noindent 5. Ich, Bischof Paulus, w�nsche euch im Herrn alles Gute.
\pend
\pstart
\noindent 6. Ich, Eulalius, Bischof von Amasia, w�nsche euch im Herrn alles Gute.
\pend
\pstart
\noindent 7. Ich, Macedonius, Bischof von Mopsuestia, w�nsche euch im Herrn alles Gute.
\pend
\pstart
\noindent 8. Ich, Thelaphius, Bischof von Chalcedon, w�nsche euch im Herrn alles
Gute.
\pend
\pstart
\noindent 9. Ich, Acacius, Bischof von Caesarea, w�nsche euch im Herrn alles Gute.
\pend
\pstart
\noindent 10. Ich, Theodor, Bischof von Heraclia, w�nsche euch im Herrn alles Gute.
\pend
\pstart
\noindent 11. Ich, Quintianus, Bischof von Gaza, w�nsche euch im Herrn alles Gute.
\pend
\pstart
\noindent 12. Ich, Marcus, Bischof von Arethusa, w�nsche euch im Herrn alles Gute.
\pend
\pstart
\noindent 13. Ich, Cyrotus, Bischof von Rhosus, w�nsche euch im Herrn alles Gute.
\pend
\pstart
\noindent 14. Ich, Eugeus, Bischof von Lysinia, w�nsche euch im Herrn alles Gute.
\pend
\pstart
\noindent 15. Ich, Antonius, Bischof von Zeugma, w�nsche euch im Herrn alles Gute.
\pend
\pstart
\noindent 16. Ich, Antonius, Bischof von Docimion, w�nsche euch im Herrn alles Gute.
\pend
\pstart
\noindent 17. Ich, Dianius, Bischof von Caesarea, w�nsche euch im Herrn alles Gute.
\pend
\pstart
\noindent 18. Ich, Vitalis, Bischof von Tyrus, w�nsche euch im Herrn alles Gute.
\pend
\pstart
\noindent 19. Ich, Eudoxius, Bischof von Germanicia, w�nsche euch im Herrn alles Gute.
\pend
\pstart
\noindent 20. Ich, Dionysius, Bischof von Alexandria in der Provinz Ciliciae, w�nsche
euch im Herrn alles Gute.
\pend
\pstart
\noindent 21. Ich, Macedonius, Bischof von Berytus, w�nsche euch im Herrn alles Gute.
\pend
\pstart
\noindent 22. Ich, Eusebius, Bischof von Dorylaeum, w�nsche euch im Herrn alles Gute.
\pend
\pstart
\noindent 23. Ich, Basilius, Bischof von Ancyra, w�nsche euch im Herrn alles Gute.
\pend
\pstart
\noindent 24. Ich, Prohaeresius, Bischof von Sinope, w�nsche euch im Herrn alles
Gute.
\pend
\pstart
\noindent 25. Ich, Eustathius, Bischof von Epiphaneia, w�nsche euch im Herrn alles Gute.
\pend
\pstart
\noindent 26. Ich, Pancratius, Bischof von Parnasus, w�nsche euch im Herrn alles Gute.
\pend
\pstart

\pend
\pstart
\noindent 27. Ich, Eusebius, Bischof von Pergamum, w�nsche euch im Herrn alles Gute.
\pend
\pstart
\noindent 28. Ich, Sabinianus, Bischof von \dag Cadimena, w�nsche euch im Herrn alles
Gute.
\pend
\pstart
\noindent 29. Ich, Bitinicus, Bischof von Zela, w�nsche euch im Herrn alles Gute.
\pend
\pstart
\noindent 30. Ich, Dominicus, Bischof von \dag Polidiane, w�nsche euch im Herrn alles
Gute.
\pend
\pstart
\noindent 31. Ich, Pison, Bischof von Trocnada, w�nsche euch im Herrn alles Gute.
\pend
\pstart
\noindent 32. Ich, Carterius, Bischof von Aspona, w�nsche euch im Herrn alles Gute.
\pend
\pstart
\noindent 33. Ich, Philetus, Bischof von Iuliopolis, w�nsche euch im Herrn alles Gute.
\pend
\pstart
\noindent 34. Ich, Ischyras, Bischof in der Mareotis, w�nsche euch
im Herrn alles Gute.
\pend
\pstart
\noindent 35. Ich, Philetus, Bischof von Cratia, w�nsche euch im Herrn alles Gute.
\pend
\pstart
\noindent 36. Ich, Bischof Timasarcus, w�nsche euch im Herrn alles Gute.
\pend
\pstart
\noindent 37. Ich, Eusebius, Bischof von Magnesia, w�nsche euch im Herrn alles Gute.
\pend
\pstart
\noindent 38. Ich, Quirius, Bischof von Philadelphia, w�nsche euch im Herrn alles Gute.
\pend
\pstart
\noindent 39. Ich, Pison, Bischof von Adana, w�nsche euch im Herrn alles Gute.
\pend
\pstart
\noindent 40. Ich, Bischof Timotheus, w�nsche euch im Herrn alles Gute.
\pend
\pstart
\noindent 41. Ich, Eudaemon, Bischof von Tanis, w�nsche euch im Herrn alles Gute.
\pend
\pstart
\noindent 42. Ich, Callinicus, Bischof von Pelusium, w�nsche euch im Herrn alles Gute.
\pend
\pstart
\noindent 43. Ich, Eusebius, Bischof von Pergamum, w�nsche euch im Herrn alles
Gute.
\pend
\pstart
\noindent 44. Ich, Leucadas, Bischof von Ilium, w�nsche euch im Herrn alles Gute.
\pend
\pstart
\noindent 45. Ich, Niconius, Bischof von der Troas, w�nsche euch im Herrn alles Gute.
\pend
\pstart
\noindent 46. Ich, Adamantius, Bischof von Chios, w�nsche euch im Herrn alles Gute.
\pend
\pstart

\pend
\pstart
\noindent 47. Ich, Edesius, Bischof von Cos, w�nsche euch im Herrn alles Gute.
\pend
\pstart
\noindent 48. Ich, Theodul, Bischof von Neocaesarea, w�nsche euch im Herrn alles Gute.
\pend
\pstart
\noindent 49. Ich, Bischof Sion, w�nsche euch im Herrn alles Gute.
\pend
\pstart
\noindent 50. Ich, Theogenes, Bischof von Lycia, w�nsche euch im Herrn alles Gute.
\pend
\pstart
\noindent 51. Ich, Florentius, Bischof von Ancyra, w�nsche euch im Herrn alles Gute.
\pend
\pstart
\noindent 52. Ich, Isaak, Bischof von Letos, w�nsche euch im Herrn alles Gute.
\pend
\pstart
\noindent 53. Ich, Bischof Eudaemon~-- ich habe f�r ihn unterschrieben~-- w�nsche
euch im Herrn alles Gute.
\pend
\pstart
\noindent 54. Ich, Agapius, Bischof von Tinos, w�nsche euch im Herrn alles Gute.
\pend
\pstart
\noindent 55. Ich, Bassus, Bischof von Carpathus, w�nsche euch im Herrn alles Gute.
\pend
\pstart
\noindent 56. Ich, Narcissus, Bischof von Irenopolis (= Neronias), w�nsche euch im Herrn alles Gute.
\pend
\pstart
\noindent 57. Ich, Ambracius, Bischof von Milet, w�nsche euch im Herrn alles Gute.
\pend
\pstart
\noindent 58. Ich, Lucius, Bischof von Antinus, w�nsche euch im Herrn alles Gute.
\pend
\pstart
\noindent 59. Ich, Nonnius, Bischof von Laodicea, w�nsche euch im Herrn alles Gute.
\pend
\pstart
\noindent 60. Ich, Pantagathus, Bischof von Attalia, w�nsche euch im Herrn alles Gute.
\pend
\pstart
\noindent 61. Ich, Flaccus, Bischof von Hierapolis, w�nsche euch im Herrn alles Gute.
\pend
\pstart
\noindent 62. Ich, Sisinnius, Bischof von Perge, w�nsche euch im Herrn alles Gute.
\pend
\pstart
\noindent 63. Ich, Bischof Diogenes, w�nsche euch im Herrn alles Gute.
\pend
\pstart
\noindent 64. Ich, Bischof Cresconius, w�nsche euch im Herrn alles Gute.
\pend
\pstart
\noindent 65. Ich, Bischof Nestorius, w�nsche euch im Herrn alles Gute.
\pend
\pstart
\noindent 66. Ich, Bischof Ammonius, w�nsche euch im Herrn alles Gute.
\pend
\pstart

\pend
\pstart
\noindent 67. Ich, Bischof Eugenius, w�nsche euch im Herrn alles Gute.
\pend
\pstart
\noindent 68. Ich, Antonius, Bischof von Bosra, w�nsche euch im Herrn alles Gute.
\pend
\pstart
\noindent 69. Ich, Demophilus, Bischof von Beroea, w�nsche euch im Herrn alles Gute.
\pend
\pstart
\noindent 70. Ich, Euticius, Bischof von Philippopolis, w�nsche euch im Herrn alles Gute.
\pend
\pstart
\noindent 71. Ich, Severus, Bischof von Gabula, w�nsche euch im Herrn alles Gute.
\pend
\pstart
\noindent 72. Ich, Timotheus, Bischof von Anchialus, w�nsche euch im Herrn alles Gute.
\pend
\pstart
\noindent 73. Ich, Valens, Bischof von Mursa, w�nsche euch im Herrn alles Gute.
\pend
%\pstart
%Ende des Dekrets der Synode der �stlichen, bei Serdica tagenden und auf Seiten der Arrianer stehenden Bisch�fe, das sie nach Afrika schickten.
%\pend
\endnumbering
\end{translatio}
\end{Rightside}
\Columns
\end{pairs}
% \thispagestyle{empty}
% DOKUMENT 44 %%%%
%% Erstellt von uh
% \renewcommand*{\goalfraction}{.7}
\chapter{Ekthesis makrostichos}
\thispagestyle{empty}
% \label{ch:45}
\label{ch:Makrostichos}
\begin{praefatio}
  \begin{description}
  \item[344]Nach dem Scheitern der Synode von
    Serdica\index[synoden]{Serdica!a. 343} reiste im Fr�hjahr 344 eine
    Delegation aus dem Westen
    (Vincentius\index[namen]{Vincentius!Bischof von Capua} von Capua,
    Euphrates\index[namen]{Euphrates!Bischof von K�ln} von K�ln mit
    dem magister militum Flavius Salia\index[namen]{Flavius
      Salia!Magister militum}) an die kaiserliche Residenz im Osten,
    Antiochien\index[namen]{Antiochien}, um Briefe der ">westlichen"<
    Synode von Serdica (Dok. \ref{sec:BriefSerdikaConstantius})
    und auch des Constans\index[namen]{Constans, Kaiser}
    (vgl. Ath., h.\,Ar. 20,2; apol.\,Const. 6,4; Socr., h.\,e. II
    22,3--5; Soz., h.\,e. III 20,1; Philost., h.\,e. III 12; Thdt.,
    h.\,e. II 8,54--56; Rufin, h.\,e. X 20) an
    Constantius\index[namen]{Constantius, Kaiser} zu
    �berbringen. Wahrscheinlich trat daraufhin in
    Antiochien\index[synoden]{Antiochien!a. 344?} wieder eine Synode
    zusammen (von einer Synode wissen wir aufgrund einer unr�hmlichen
    Angelegenheit des dortigen Bischofs
    Stephanus\index[namen]{Stephanus!Bischof von Antiochien}, der
    abgesetzt wurde, nachdem er
    Euphrates\index[namen]{Euphrates!Bischof von K�ln} eine
    Prostituierte auf das Zimmer geschickt hatte; vgl. Ath.,
    h.\,Ar. 20), auf der eventuell diese lange Erkl�rung formuliert
    wurde; sie wurde von einer Delegation
    (Eudoxius\index[namen]{Eudoxius!Bischof von Germanicia} von
    Germanicia, Martyrius\index[namen]{Martyrius!Bischof2},
    Macedonius\index[namen]{Macedonius!Bischof von Mopsuestia} von
    Mopsuestia nach Ath., syn.  26,1; Soz., h.\,e. III 11,2;
    Liberius nennt zus�tzlich noch Demophilus\index[namen]{Demophilus!Bischof von
      Beroea}\index[namen]{Liberius!Bischof von Rom}
    [\editioncite[91,18~f.]{Hil:coll}]) 344/345 in den Westen geschickt, da
    Constans\index[namen]{Constans, Kaiser} zu einer Synode nach
    Mailand\index[synoden]{Mailand!a. 345} geladen hatte (Ath.,
    syn. 26,1; Epiph., haer. 73,2). Die westlichen, in Mailand
    versammelten Bisch�fe waren aber nicht bereit, diese Erkl�rung zu
    akzeptieren, verlangten vielmehr eine klare Absage an den
    ">Arianismus"<, d.\,h. an jedwede Form der Hypostasentheologie
    (vielleicht wurde sogar eine Unterschrift unter das westliche
    Serdicense verlangt), so da� die �stliche Delegation entr�stet
    abreiste. Einzig und allein auf eine gemeinsame Verurteilung von
    Photinus\index[namen]{Photinus!Bischof von Sirmium} konnte man
    sich offenbar einigen (Hil., coll.\,antiar. B II 5 [\editioncite[142,
    17--19]{Hil:coll}]: \textit{igitur ad tollendum ex episcopatu Fotinum,
      qui ante biennium iam in Mediolanensi synodo erat hereticus
      damnatus}~\dots; [146,5~f.:] \textit{Fotinus hereticus
      deprehensus, olim reus pronuntiatus et a communione iam pridem
      unitatis absciscus},~\dots). Eine Reaktion des Ostens auf diese
    Ereignisse referiert Hil., coll.\,antiar. B II 9,2 (\editioncite[146,19--147,9]{Hil:coll}).
  \item[�berlieferung]Die lange theologische Erkl�rung wiederholt in
    � 1--4 die theologische Erkl�rung der ">�stlichen"<
    Synode von Serdica  (s. Dok. \ref{sec:BekenntnisSerdikaOst}).
    Socrates ist von der Version des Athanasius abh�ngig. Sozomenus
    �berliefert den Text nicht, sondern bietet nur ein Regest (Soz.,
    h.\,e. III 11,1).
  \item[Fundstelle]Ath., syn. 26 (\editioncite[251,22--254,12]{Opitz1935}); Socr.,
    h.\,e. II 19,3--28 (\editioncite[112,17--117,4]{Hansen:Socr}).
  \end{description}
\end{praefatio}
\begin{pairs}
\selectlanguage{polutonikogreek}
\begin{Leftside}
\beginnumbering
\pstart
\hskip -1.3em\edtext{\abb{}}{\killnumber\Cfootnote{\hskip -1em\latintext Ath.(BKPO R); Socr.(MF=b
AT)}}\specialindex{quellen}{chapter}{Athanasius!syn.!26}\specialindex{quellen}{chapter}{Socrates!h.\,e.!II 19,3--28}
\kap{1}Piste'uomen e>ic \edtext{\abb{<'ena je'on}}{\Afootnote{\latintext vgl. 1Cor 8,6;
Eph 4,6}},\edindex[bibel]{Korinther I!8,6|textit}\edindex[bibel]{Epheser!4,6|textit} 
pat'era, \edtext{\abb{pantokr'atora}}{\Afootnote{\latintext Apc 1,8
u.�.}},\edindex[bibel]{Offenbarung!1,8} \edtext{\abb{kt'isthn}}{\Afootnote{\latintext vgl.
1Petr 4,19}}\edindex[bibel]{Petrus I!4,19|textit} ka`i poiht`hn t~wn
\edtext{p'antwn}{\Dfootnote{<ap'antwn \latintext Socr.(AT)}}, \edtext{((>ex o<~u p~asa
patri`a >en \edtext{o>uran~w|}{\Dfootnote{o>urano~ic \latintext Socr.(bA)}} ka`i >ep`i
g~hc >onom'azetai))}{\lemma{\abb{}}\Afootnote{\latintext Eph
3,15}},\edindex[bibel]{Epheser!3,15} 
\pend
\pstart
\kap{2}ka`i e>ic t`on
\edtext{\abb{monogen~h}}{\Afootnote{\latintext Io 1,14.18; 3,16; 1Io
4,9}}\edindex[bibel]{Johannes!1,14}\edindex[bibel]{Johannes!1,18}\edindex[bibel]{Johannes!3,16}
\edindex[bibel]{Johannes I!4,9} a>uto~u u<i`on \edtext{t`on k'urion <hm~wn
>Ihso~un Qrist'on}{\lemma{\abb{}}\Dfootnote{\responsio\ >Ihso~un Qrist`on t`on k'urion
<hm~wn \latintext Socr.(b)}}, t`on \edtext{\abb{pr`o p'antwn
\edtext{\abb{t~wn}}{\Dfootnote{\latintext > Ath.(KPOR)}} a>i'wnwn}}{\Afootnote{\latintext
vgl. 1Cor 2,7}}\edindex[bibel]{Korinther I!2,7|textit} \edtext{>ek to~u patr`oc
gennhj'enta}{\lemma{\abb{}}\Dfootnote{\responsio\ gennhj'enta >ek to~u patr`oc \latintext
Socr.(bA)}}, je`on >ek jeo~u, f~wc >ek fwt'oc, \edtext{\abb{di'' o~<u \edtext{>eg'eneto
t`a p'anta}{\lemma{\abb{}}\Dfootnote{\responsio\ t`a p'anta >eg'eneto \latintext
Socr.(A)}}}}{\Afootnote{\latintext vgl. 1Cor 8,6; Col 1,16; Hebr 1,2; Io
1,3}}\edindex[bibel]{Korinther
I!8,6|textit}\edindex[bibel]{Kolosser!1,16|textit}\edindex[bibel]{Hebraeer!1,2|textit}
\edindex[bibel]{Johannes!1,3|textit} \edtext{\abb{\edtext{t`a}{\lemma{\abb{t`a\ts{2}}}\Dfootnote{\latintext
> Ath.(R)}} >en \edtext{\abb{to~ic}}{\Dfootnote{\latintext > Ath.}} o>urano~ic ka`i t`a
>ep`i t~hc g~hc}}{\Afootnote{\latintext Col 1,16; Eph
1,10}}\edindex[bibel]{Kolosser!1,16}\edindex[bibel]{Epheser!1,10}, \edtext{\abb{t`a
<orat`a ka`i t`a >a'orata}}{\Afootnote{\latintext Col
1,16}}\edindex[bibel]{Kolosser!1,16}, \edtext{\abb{l'ogon}}{\Afootnote{\latintext Io
1,1}}\edindex[bibel]{Johannes!1,1} >'onta ka`i \edtext{\abb{sof'ian ka`i
d'unamin}}{\Afootnote{\latintext vgl. 1Cor 1,24}}\edindex[bibel]{Korinther I!1,24|textit}
ka`i \edtext{\abb{zw`hn}}{\Afootnote{\latintext Io 1,4; 11,25;
14,6; 1Io 1,2; 5,20; Col 3,4}}\edindex[bibel]{Johannes!1,4}\edindex[bibel]{Johannes!11,25}\edindex[bibel]{Johannes!14,6}
\edindex[bibel]{Johannes I!1,2}\edindex[bibel]{Johannes I!5,20}\edindex[bibel]{Kolosser!3,4}
ka`i \edtext{\abb{f~wc >alhjin'on}}{\Afootnote{\latintext Io 1,9; 1Io
2,8}}\edindex[bibel]{Johannes!1,9}\edindex[bibel]{Johannes I!2,8}, t`on \edtext{\abb{>ep''
\edtext{>esq'atwn}{\Dfootnote{>esq'atou \latintext Socr.(bA)}} t~wn
<hmer~wn}}{\Afootnote{\latintext vgl. Hebr 1,2}}\edindex[bibel]{Hebraeer!1,2|textit} di''
<hm~ac \edtext{\abb{>enanjrwp'hsanta}}{\Afootnote{\latintext vgl. 1Cor
15,47}}\edindex[bibel]{Korinther I!15,47|textit} ka`i \edtext{\abb{gennhj'enta >ek t~hc
<ag'iac parj'enou}}{\Afootnote{\latintext vgl. Mt 1,23; Lc
1,27.34~f.}},\edindex[bibel]{Matthaeus!1,23|textit}\edindex[bibel]{Lukas!1,27|textit}
\edindex[bibel]{Lukas!1,34~f.|textit} \edtext{ka`i}{\lemma{ka`i\ts{1}}\Dfootnote{t`on \latintext Socr.}}
\edtext{\abb{staurwj'enta}}{\Afootnote{\latintext vgl. Mt 20,19; 27,22 u.�.; 1Cor 1,23;
2,2.8; 2Cor 13,4; Gal
3,1}}\edindex[bibel]{Matthaeus!20,19|textit}\edindex[bibel]{Matthaeus!27,22|textit}\edindex[bibel]{Korinther I!1,23|textit}\edindex[bibel]{Korinther I!2,2|textit}\edindex[bibel]{Korinther I!2,8|textit}\edindex[bibel]{Korinther II!13,4|textit}\edindex[bibel]{Galater!3,1|textit} ka`i
\edtext{\abb{>apojan'onta}}{\Afootnote{\latintext 1Cor 15,3}}\edindex[bibel]{Korinther I!15,3} \edtext{\abb{ka`i taf'enta}}{\Afootnote{\latintext vgl. 1Cor 15,4}\Dfootnote{\latintext > Ath.(B)}}\edindex[bibel]{Korinther I!15,4|textit} ka`i \edtext{\abb{>anast'anta}}{\Afootnote{\latintext vgl. Mt 17,9par; 20,19par; 1Cor 15,4; 1Thess 4,14; auch Eph 1,20}}\edindex[bibel]{Matthaeus!17,9par|textit}\edindex[bibel]{Matthaeus!20,19par|textit}\edindex[bibel]{Korinther I!15,4|textit}\edindex[bibel]{Thessalonicher I!4,14|textit}\edindex[bibel]{Epheser!1,20|textit} >ek
\edtext{\abb{t~wn}}{\Dfootnote{\latintext > Socr.}} nekr~wn t~h| tr'ith| <hm'era|
\edtext{\abb{ka`i}}{\Dfootnote{\latintext > Socr.(bA)}}
\edtext{\abb{\edtext{\abb{>analhfj'enta}}{\Dfootnote{+ te \latintext Socr.(A)}}
\edtext{\abb{e>ic}}{\Dfootnote{+ t`on \latintext Socr.(M\ts{1}T) Ath.(B*)}}
o>uran`on}}{\Afootnote{\latintext Mc 16,19; Act
1,2}}\edindex[bibel]{Markus!16,19|textit}\edindex[bibel]{Apostelgeschichte!1,2}
ka`i \edtext{\abb{kajesj'enta \edtext{>ek dexi~wn}{\Dfootnote{>en dexi~a \latintext
Socr.(T)}} to~u patr`oc}}{\Afootnote{\latintext vgl. Ps 110,1; Eph 1,20; Col 3,1; 1Petr
3,22; Hebr 1,3; Mc
16,19}}\edindex[bibel]{Psalmen!110,1|textit}\edindex[bibel]{Epheser!1,20|textit}\edindex[
bibel]{Kolosser!3,1|textit}\edindex[bibel]{Petrus
I!3,22|textit}\edindex[bibel]{Hebraeer!1,3|textit}\edindex[bibel]{Markus!16,19|textit}
\edtext{\abb{ka`i}}{\Dfootnote{\latintext > Socr.(bA)}}
\edtext{\abb{>erq'omenon}}{\Dfootnote{+ te \latintext Socr.(A)}} >ep`i
\edtext{\abb{suntele'ia|}}{\Afootnote{\latintext vgl. Mt 13,39; 24,3;
28,20}}\edindex[bibel]{Matthaeus!13,39|textit}\edindex[bibel]{Matthaeus!24,3|textit}
\edindex[bibel]{Matthaeus!28,20|textit} \edtext{to~u a>i~wnoc}{\Dfootnote{t~wn a>i'wnwn
\latintext Socr.}} \edtext{\abb{kr~inai z~wntac ka`i nekro`uc}}{\Afootnote{\latintext 2Tim
4,1; 1Petr 4,5; vgl. Io 5,22; Act 10,42}}\edindex[bibel]{Timotheus
II!4,1}\edindex[bibel]{Petrus
I!4,5}\edindex[bibel]{Johannes!5,22|textit}\edindex[bibel]{Apostelgeschichte!10,42|textit}
ka`i \edtext{\abb{>apodo~unai <ek'astw| kat`a \edtext{\abb{t`a}}{\Dfootnote{\latintext >
Ath.(B)}} >'erga a>uto~u}}{\Afootnote{\latintext vgl. Mt 16,27; Rom 2,6; 2Tim 4,14; Apc
22,12}}\edindex[bibel]{Matthaeus!16,27|textit}\edindex[bibel]{Roemer!2,6|textit}\edindex[bibel]{Timotheus
II!4,14|textit}\edindex[bibel]{Offenbarung!22,12|textit}, o<~u <h basile'ia >akat'apaustoc o>~usa
diam'enei e>ic \edtext{\abb{to`uc}}{\Dfootnote{\latintext > Socr.(bA)}} >ape'irouc
a>i~wnac; \edtext{kaj'ezetai}{\Dfootnote{k'ajhtai \latintext Socr.(T)}} g`ar >en dexi~a|
to~u patr`oc o>u m'onon >en t~w| a>i~wni to'utw|, >all`a ka`i >en t~w|
\edtext{\abb{m'ellonti}}{\Afootnote{\latintext Eph 1,21}}\edindex[bibel]{Epheser!1,21}.
\pend
\pstart
\kap{3}\edtext{\abb{piste'uomen}}{\Dfootnote{+ d`e \latintext Socr.(bA)}} ka`i e>ic t`o pne~uma
t`o <'agion, \edtext{\abb{tout'esti}}{\Dfootnote{+ e>ic \latintext Socr.(bA)}} t`on
\edtext{\abb{par'aklhton}}{\Afootnote{\latintext Io 14,16.26; 15,26;
16,7.13}}\edindex[bibel]{Johannes!14,16}\edindex[bibel]{Johannes!14,26}\edindex[bibel]{Johannes!15,26}\edindex[bibel]{Johannes!16,7}\edindex[bibel]{Johannes!16,13}, <'oper \edtext{\abb{>epaggeil'amenoc to~ic
>apost'oloic}}{\Afootnote{\latintext vgl. Io 14,26; 15,26; Act 2,17-21.33 Tit
3,6}}\edindex[bibel]{Johannes!14,26|textit}\edindex[bibel]{Apostelgeschichte!2,17--21|textit}
\edindex[bibel]{Apostelgeschichte!2,33|textit}\edindex[bibel]{Johannes!15,26|textit}
\edindex[bibel]{Titus!3,6|textit} met`a t`hn e>ic \edtext{o>uran`on}{\Dfootnote{o>urano`us
\latintext Socr.(A)}} >'anodon >ap'esteile did'axai \edtext{a>uto`uc ka`i
<upomn~hsai}{\lemma{\abb{}}\Dfootnote{\responsio\ ka`i <upomn~hsai a>uto`uc \latintext
Socr.(bA)}} p'anta, di'' o<~u ka`i \edtext{\abb{<agiasj'hsontai}}{\Afootnote{\latintext
vgl. Rom 15,16; 1Cor 6,11}\lemma{<agiasj'hsontai}\Dfootnote{<agi'azontai \latintext Socr.}}\edindex[bibel]{Roemer!15,16|textit}\edindex[bibel]{Korinther
I!6,11|textit} a<i t~wn
e>ilikrin~wc e>ic a>ut`on \edtext{pepisteuk'otwn}{\Dfootnote{pisteu'ontwn \latintext
Socr.(bA)}} yuqa'i.
\pend
\pstart
\kap{4}to`uc d`e l'egontac >ex o>uk >'ontwn t`on \edtext{u<i`on}{\Dfootnote{je`on
\latintext Ath.}} >`h >ex <et'erac <upost'asewc ka`i m`h >ek to~u jeo~u ka`i <'oti >~hn
\edtext{qr'onoc pot`e}{\lemma{\abb{}}\Dfootnote{\responsio\ pote qr'onoc \latintext
Socr.}} \edtext{\abb{>`h a>i'wn}}{\Dfootnote{\latintext > Socr.(M)}}, <'ote
\edtext{m`h}{\Dfootnote{o>uk \latintext Socr.}} >~hn, >allotr'iouc o>~iden <h
\edtext{\abb{kajolik`h ka`i <ag'ia}}{\Dfootnote{\latintext Ath.(BPOR) \greektext
\responsio\ <ag'ia ka`i kajolik`h \latintext Ath.(K) Socr.(T) \greektext <ag'ia kajolik`h
\latintext Socr.(bA)}} >ekklhs'ia. \edtext{\abb{<omo'iwc}}{\Dfootnote{+ d`e \latintext
Socr.(T) \greektext <omo'iouc \latintext Socr.(A)}}
\edtext{\abb{ka`i}}{\Dfootnote{\latintext > Socr.(M\textsuperscript{1})}} to`uc l'egontac
tre~ic e~>inai jeo`uc >`h t`on Qrist`on m`h e~>inai je`on
\edtext{\abb{>`h}}{\Dfootnote{\latintext > Socr.}} pr`o t~wn a>i'wnwn m'hte Qrist`on
\edtext{m'hte u<i`on a>ut`on e~>inai jeo~u}{\lemma{\abb{}}\Dfootnote{\responsio\ a>ut`on
m'hte u<i`on e~>inai jeo~u \latintext Ath.(BKPO) \greektext \responsio\ m'hte u<i`on
e~>inai jeo~u a>ut`on \latintext Socr.(A) \greektext \responsio\ m'hte u<i`on jeo~u
e~>inai a>ut`on \latintext Socr.(b)}} >`h t`on a>ut`on e~>inai pat'era ka`i u<i`on
\edtext{ka`i}{\lemma{ka`i\ts{2}}\Dfootnote{>`h \latintext Ath.}} \edtext{<'agion
pne~uma}{\lemma{\abb{}}\Dfootnote{\responsio\ pne~uma <'agion \latintext Socr.(T)}}
\edtext{>`h}{\Dfootnote{ka`i \latintext Socr.(FA)}} >ag'ennhton
\edtext{\abb{t`on}}{\Dfootnote{\latintext > Ath. Socr.(M\ts{1}A)}} u<i`on >`h <'oti o>u
boul'hsei o>ud`e \edtext{jel'hsei}{\Dfootnote{jel'hmati \latintext Socr.(A)}} >eg'ennhsen
\edtext{<o pat`hr t`on u<i`on}{\lemma{\abb{}}\Dfootnote{\responsio\ t`on u<i`on <o pat`hr
\latintext Ath.(B)}} >anajemat'izei <h <ag'ia \edtext{\abb{ka`i}}{\Dfootnote{\latintext >
Ath.}} kajolik`h >ekklhs'ia.
\pend
\pstart
\kap{5}o>'ute g`ar >ex o>uk >'ontwn \edtext{t`on u<i`on
l'egein}{\lemma{\abb{}}\Dfootnote{\responsio\ l'egein t`on u<i`on \latintext Socr.(bA)}}
>asfal'ec, >epe`i mhdamo~u to~uto t~wn jeopne'ustwn graf~wn
\edtext{f'eretai}{\Dfootnote{>emf'eretai \latintext Socr.(bA)}} per`i a>uto~u, o>'ute m`hn
>ex <et'erac \edtext{\abb{tin`oc}}{\Dfootnote{\latintext > Socr.(bA)}} <upost'asewc par`a
t`on pat'era proupokeim'enhc, >all'' >ek m'onou to~u jeo~u gnhs'iwc a>ut`on
\edtext{gegenn~hsjai}{\Dfootnote{gegen~hsjai \latintext Socr.(T)}}
\edtext{dioriz'omeja}{\Dfootnote{piste'uomen ka`i dioriz'omeja \latintext Socr.(T)
\greektext didask'omeja \latintext Socr.(bA)}}. \edtext{<`en g`ar t`o}{\Dfootnote{<'ena
g`ar t`on \latintext Socr.(M\textsuperscript{r})}} >ag'ennhton ka`i >'anarqon, t`on
Qristo~u pat'era, <o je~ioc did'askei l'ogoc. >all'' o>ud`e ((t`o >~hn pote
\edtext{\abb{<'ote}}{\Dfootnote{\latintext dupl. Ath.(B)}} o>uk >~hn)) \edtext{>ex >agr'afwn
>episfal~wc l'egontac}{\Dfootnote{to~ic >ex \dots\ l'egousin <wc \latintext
Socr.(M\textsuperscript{r})}} \edtext{qronik'on}{\Dfootnote{qronitik'on \latintext
Socr.(T)}} \edtext{\abb{ti}}{\Dfootnote{\latintext > Socr.(b)}} di'asthma proenjumht'eon
\edtext{a>uto~u}{\Dfootnote{a>uto`uc \latintext Socr.(M\textsuperscript{r})}}, >all'' >`h
m'onon t`on >aqr'onwc a>ut`on gegennhk'ota je'on; ka`i qr'onoi g`ar ka`i a>i~wnec
geg'onasi di'' a>uto~u.
\pend
\pstart
\kap{6}o>'ute m`hn sun'anarqon \edtext{ka`i}{\Dfootnote{o>'ute \latintext Socr.(bA)}}
sunag'ennhton \edtext{t~w| patr`i t`on u<i`on}{\lemma{\abb{}}\Dfootnote{\responsio\ t`on
u<i`on t~>w| patr`i \latintext Ath.(B) Socr.(bA)}} e>~inai nomist'eon; sunan'arqou g`ar
ka`i sunagenn'htou o>ude`ic kur'iwc pat`hr \edtext{>`h u<i`oc}{\Dfootnote{u<io~u \dt{coni.
Scheidweiler}}} leqj'hsetai. >all`a t`on m`en pat'era m'onon >'anarqon
\edtext{>'onta}{\Dfootnote{te \latintext Socr.(T)}} ka`i
\edtext{>ag'ennhton}{\Dfootnote{>an'efikton \latintext Socr.(bA)}} gegennhk'enai
>anef'iktwc ka`i p~asin >akatal'hptwc o>'idamen, t`on d`e u<i`on
\edtext{gegenn~hsjai}{\Dfootnote{gegen~hsjai \latintext Socr.(T)}}
\edtext{\abb{pr`o}}{\Dfootnote{+ t~wn \latintext Socr.}} a>i'wnwn ka`i mhk'eti <omo'iwc
t~w| patr`i >ag'ennhton e>~inai ka`i a>ut'on, \edtext{>all''}{\Dfootnote{>all`a \latintext
Socr.(A)}} >arq`hn \edtext{>'eqein}{\Dfootnote{>'eqei \latintext Ath.(R*)}} t`on
\edtext{\abb{genn'hsanta}}{\Dfootnote{\latintext > Ath.(K)}} pat'era; \edtext{((kefal`h
g`ar Qristo~u <o je'oc))}{\lemma{\abb{}}\Afootnote{\latintext 1Cor
11,3}}\edindex[bibel]{Korinther I!11,3}.
\pend
\pstart
\kap{7}o>'ute m`hn tr'ia <omologo~untec pr'agmata ka`i tr'ia pr'oswpa to~u patr`oc ka`i
to~u u<io~u ka`i to~u <ag'iou pne'umatoc kat`a t`ac graf`ac \edtext{tre~ic di`a
to~uto}{\lemma{\abb{}}\Dfootnote{\responsio\ di`a to~uto tre~ic \latintext Socr.(T)}}
\edtext{\abb{to`uc}}{\Dfootnote{\latintext > Socr.(T)}} jeo`uc poio~umen, >epeid`h t`on
a>utotel~h ka`i >ag'ennhton >'anarq'on te ka`i >a'oraton je`on <'ena m'onon o>'idamen,
t`on je`on ka`i pat'era to~u monogeno~uc, t`on m'onon m`en >ex <eauto~u t`o e>~inai
>'eqonta, m'onon \edtext{\abb{d`e}}{\Dfootnote{\latintext > Ath.(B) Socr.(T)}}
\edtext{\abb{to~ic}}{\Dfootnote{\latintext dupl. Socr.(T)}} >'alloic p~asin >afj'onwc
\edtext{to~uto}{\Dfootnote{t`o e~>inai \latintext Socr.(bA)}}
\edtext{qariz'omenon}{\Dfootnote{pareq'omenon \latintext Socr.(M\mg FA)}}.
\pend
\pstart
\kap{8}o>'ute m`hn <'ena je`on \edtext{m'onon l'egontec
e>~inai}{\lemma{\abb{}}\Dfootnote{\responsio\ m'onon e~>inai l'egontec \latintext Ath.(B)
\greektext \responsio\ l'egontec e~>inai m'onon \latintext Socr.(T)}} t`on to~u kur'iou
\edtext{\abb{<hm~wn}}{\Dfootnote{\latintext > Socr.(bA)}} >Ihso~u Qristo~u pat'era, t`on
m'onon >ag'ennhton, di`a to~uto >arno'umeja \edtext{\abb{ka`i}}{\Dfootnote{\latintext >
Socr.(b)}} t`on Qrist`on je`on e>~inai \edtext{pr`o a>i'wnwn}{\Dfootnote{proai'wnion
\latintext Socr.(bA)}}, <opo~io'i e>isin o<i >ap`o Pa'ulou\edindex[namen]{Paulus von
Samosata} to~u Samosat'ewc <'usteron a>ut`on met`a t`hn >enanjr'wphsin >ek prokop~hc
tejeopoi~hsjai l'egontec t~w| t`hn f'usin \edtext{\abb{yil`on}}{\Dfootnote{+ a>ut`on \latintext Socr.}} >'anjrwpon gegon'enai. o>'idamen g`ar ka`i a>ut'on, e>i ka`i
<upot'etaktai t~w| patr`i ka`i \edtext{t~w|}{\lemma{\abb{t~w|\ts{2}}}\Dfootnote{\latintext > Ath.(KR\corr)
Socr.(M\textsuperscript{*}T)}} je~w|, >all'' <'omwc \edtext{\abb{pr`o
a>i'wnwn}}{\Dfootnote{\latintext > Socr.(bA)}} gennhj'enta >ek to~u jeo~u
\edtext{\abb{je`on}}{\Dfootnote{+ ka`i \latintext Socr.(T)}} kat`a f'usin t'eleion e>~inai
ka`i >alhj~h ka`i m`h >ex \edtext{>anjr'wpwn}{\Dfootnote{>anjr'wpou \dt{coni.
Scheidweiler} + ka`i \latintext Socr.(A)}} met`a ta~uta je'on, >all'' >ek jeo~u
\edtext{>enanjrwp~hsai}{\Dfootnote{>enanjrwp'hsanta \latintext Socr.(T)}} di'' <hm~ac,
ka`i \edtext{mhd'epote}{\Dfootnote{mhd'epw t`e \latintext Socr.(A) \greektext mhd`e
p'wpote \latintext Socr.(b)}} \edtext{>apolwlek'ota}{\Dfootnote{>apolelwk'ota \latintext
Socr.(T)}} t`o e~>inai \edtext{\abb{je'on}}{\Dfootnote{\latintext > Ath.}}.
\pend
\pstart
\kap{9}bdeluss'omeja d`e pr`oc to'utoic ka`i >anajemat'izomen ka`i
\edtext{to`uc}{\Dfootnote{to~ic \latintext Ath.(R)}} l'ogon m`en m'onon a>ut`on
\edtext{yil`on to~u jeo~u}{\lemma{\abb{}}\Dfootnote{\responsio\ to~u jeo~u yil`on
\latintext Socr.}} ka`i >an'uparkton >epipl'astwc kalo~untac, >en <et'erw| t`o e>~inai
>'eqonta, n~un m`en <wc t`on proforik`on leg'omenon <up'o tinwn, n~un d`e <wc t`on
>endi'ajeton, Qrist`on d`e \edtext{\abb{a>ut`on}}{\Dfootnote{+ gegon'enai \latintext
Socr.(A)}} ka`i \edtext{u<i`on}{\Dfootnote{k'urion \latintext Socr.(b)}} \edtext{to~u
jeo~u}{\Dfootnote{ka`i je`on \latintext Socr.(M\textsuperscript{r}) \greektext jeo~u +
kalo~umen \latintext Socr.(T)}} ka`i mes'ithn ka`i e>ik'ona
\edtext{\abb{to~u}}{\Dfootnote{\latintext > Socr.(T)}} jeo~u \edtext{m`h}{\Dfootnote{to`uc
d`e m`h \latintext Socr.(T)}} e>~inai pr`o a>i'wnwn \edtext{\abb{j'elontac}}{\Dfootnote{+
a>ut`on \latintext Socr.(T)}}, >all'' \edtext{>ek t'ote}{\Dfootnote{>'ektote \latintext
Socr.}} Qrist`on a>ut`on gegon'enai ka`i u<i`on \edtext{\abb{to~u}}{\Dfootnote{\latintext
> Socr.(T)}} jeo~u, >ex o<~u t`hn \edtext{<hmet'eran}{\Dfootnote{<umet'eran \latintext
Ath.(B\corr)}} >ek t~hc parj'enou \edtext{s'arka
>ane'ilhfe}{\lemma{\abb{}}\Dfootnote{\responsio\ >ane'ilhfe s'arka \latintext Socr.(T)
\greektext s'arka \latintext > Ath.(K) \greektext >ane'ilhfe + <o \latintext Socr.(A)}}
pr`o tetrakos'iwn \edtext{\abb{o>uq}}{\Dfootnote{\latintext > Socr.(bA)}} <'olwn >et~wn.
\edtext{>ek t'ote}{\Dfootnote{>'ektote \latintext Socr.}} g`ar t`on Qrist`on
\edtext{>arq`hn}{\Dfootnote{>arq`h \latintext Socr.(A)}} basile'iac >esqhk'enai
\edtext{>ej'elousi}{\Dfootnote{j'elousin \latintext Socr.(bA) \greektext l'egousi
\latintext Socr.(T)}} ka`i t'eloc <'exein \edtext{a>ut`hn}{\Dfootnote{a>ut`on \latintext
Ath.(B)}} met`a t`hn sunt'eleian ka`i \edtext{\abb{t`hn}}{\Dfootnote{\latintext >
Socr.(b)}} \edtext{\abb{kr'isin}}{\Dfootnote{+ >agnoo~unta \latintext Socr.(T)}}.
\pend
\pstart
\kap{10}Toio~utoi d'e e>isin \edtext{\abb{o<i}}{\Dfootnote{\latintext > Socr.(A)}} >ap`o
Mark'ellou\edindex[namen]{Markell!Bischof von Ancyra} ka`i
\edtext{Fwteino~u}{\Dfootnote{Skoteino~u \latintext Ath.}}
\edindex[namen]{Photinus!Bischof von Sirmium} t~wn >Agkurogalat~wn, o<`i t`hn proai'wnion
\edtext{<'uparxin to~u Qristo~u ka`i t`hn je'othta}{\Dfootnote{<'uparx'in te ka`i je'othta
to~u Qristo~u \latintext Socr.}} ka`i t`hn >atele'uthton a>uto~u basile'ian <omo'iwc
\edtext{\abb{>Iouda'ioic}}{\Dfootnote{\latintext > Socr.(bA)}}
\edtext{>ajeto~usin}{\Dfootnote{>ajeto~untec \latintext Socr.(T)}} \edtext{>ep`i prof'asei
to~u sun'istasjai}{\Dfootnote{<'istasjai \latintext Socr.(b) \greektext >ep`i prof'asei
t~w <'istasjai \latintext Socr.(M\textsuperscript{r})}} doke~in \edtext{t`hn
monarq'ian}{\Dfootnote{t~h| monarq'ia| \latintext Ath. \greektext a>uto~ic o<'utw t`hn
monarq'ian \latintext Socr.(M\textsuperscript{1})}}. >'ismen g`ar \edtext{a>ut`on
<hme~ic}{\lemma{\abb{}}\Dfootnote{\responsio\ <hme~ic a>ut`on \latintext Ath.(P)}} o>uq
<apl~wc l'ogon proforik`on >`h >endi'ajeton to~u jeo~u, >all`a z~wnta je`on l'ogon kaj''
\edtext{\abb{<eaut`on}}{\Dfootnote{+ te \latintext Socr.(M\textsuperscript{r})}}
<up'arqonta ka`i u<i`on jeo~u ka`i Qrist`on, ka`i \edtext{o>u}{\Dfootnote{o<'utw \latintext
Socr.(M\textsuperscript{r})}} prognwstik~wc sun'onta ka`i sundiatr'ibonta pr`o
\edtext{a>i'wnwn}{\Dfootnote{a>i'wnw \latintext Socr.(M\textsuperscript{1})}} t~w|
<eauto~u patr`i ka`i pr`oc p~asan diakonhs'amenon a>ut~w| t`hn dhmiourg'ian e>'ite t~wn
\edtext{<orat~wn e>'ite t~wn >aor'atwn}{\lemma{\abb{}}\Dfootnote{\responsio\ >aor'atwn
e>'ite t~wn <orat~wn \latintext Socr.(bA)}}, \edtext{\abb{>all'' >enup'ostaton l'ogon
>'onta to~u patr`oc ka`i je`on >ek jeo~u}}{\lemma{\abb{>all'' \dots\
jeo~u}}\Dfootnote{\latintext > Ath.}}. o<~utoc g'ar >esti pr`oc <`on e>~ipen <o pat`hr
\edtext{\abb{<'oti}}{\Dfootnote{\latintext > Socr.(AT)}} ((\edtext{\abb{poi'hswmen
>'anjrwpon kat'' e>ik'ona <hmet'eran ka`i kaj''
<omo'iwsin}}{\lemma{\abb{}}\Afootnote{\latintext Gen
1,26}}))\edindex[bibel]{Genesis!1,26}, \edtext{<o}{\Dfootnote{<`oc \latintext Socr.}} ka`i
to~ic \edtext{patri'arqaic}{\Dfootnote{patr'asin \latintext Socr.(bA)}} a>utopros'wpwc
\edtext{>ofje'ic}{\Dfootnote{>'wfjh \latintext Socr.}}, dedwk`wc t`on n'omon ka`i lal'hsac
di`a t~wn profht~wn ka`i t`a teleuta~ia >enanjrwp'hsac ka`i t`on <eauto~u pat'era
\edtext{p~asin}{\Dfootnote{p~asi to~ic \latintext Socr.(A)}} >anjr'wpoic faner'wsac ka`i
basile'uwn e>ic to`uc >ateleut'htouc a>i~wnac. o>ud`en g`ar pr'osfaton <o
\edtext{Qrist`oc}{\Dfootnote{je`oc \latintext Ath.(O*)}} prose'ilhfen >ax'iwma,
\edtext{>all''}{\Dfootnote{>all`a \latintext Socr.(b)}} >'anwjen
\edtext{\abb{t'eleion}}{\Dfootnote{+ e~>inai \latintext Socr.(T)}} a>ut`on ka`i
\edtext{t~w| patr`i kat`a p'anta <'omoion}{\lemma{\abb{}}\Dfootnote{\responsio\ kat`a
p'anta <'omoion t~w| patr`i \latintext Socr.(T)}}
\edtext{\abb{e~>inai}}{\Dfootnote{\latintext > Socr.}} pepiste'ukamen.
\pend
\pstart
\kap{11}ka`i to`uc l'egontac d`e t`on a>ut`on e~>inai pat'era ka`i u<i`on ka`i <'agion
pne~uma, kaj'' <en`oc ka`i to~u a>uto~u pr'agmat'oc te ka`i pros'wpou t`a tr'ia >on'omata
>aseb~wc >eklamb'anontac e>ik'otwc \edtext{>apokhr'ussomen}{\Dfootnote{>apokhr'uttomen
\latintext Socr.(T)}} t~hc >ekklhs'iac, <'oti t`on >aq'wrhton ka`i >apaj~h pat'era
\edtext{\abb{qwrht`on <'ama ka`i pajht`on}}{\Dfootnote{\latintext >
Socr.(M\textsuperscript{1}) \greektext <'ama \latintext > Socr.(M\textsuperscript{r})}}
di`a t~hc >enanjrwp'hsewc <upot'ijentai. toio~utoi g'ar e>isin o<i
\edtext{Patropassiano`i}{\Dfootnote{Patropasiano`i \latintext Socr.(T) \greektext
Patropasano`i \latintext Socr.(b)}} \edtext{\abb{m`en}}{\Dfootnote{\latintext >
Socr.(bA)}} par`a <Rwma'ioic, Sabelliano`i\edindex[namen]{Sabellianer} d`e
\edtext{kalo'umenoi par'' <hm~in}{\Dfootnote{par'' <hm~in leg'omenoi \latintext
Socr.(bA)}}. o>'idamen g`ar <hme~ic t`on m`en >aposte'ilanta pat'era >en t~w| o>ike'iw|
t~hc >analloi'wtou je'othtoc >'hjei memenhk'enai, t`on d`e >apostal'enta Qrist`on t`hn
t~hc >enanjrwp'hsewc o>ikonom'ian peplhrwk'enai.
\pend
\pstart
\kap{12}\edtext{<omo'iwc}{\Dfootnote{<'omwc \latintext Ath.}}\looseness=1\ d`e ka`i to`uc
\edtext{o>u}{\Dfootnote{m`h \latintext Socr.(T)}} boul'hsei o>ud`e jel'hsei
\edtext{gegenn~hsjai}{\Dfootnote{gegen~hsjai \latintext Ath.(P)
Socr.(M\textsuperscript{1}T)}} t`on \edtext{u<i`on}{\Dfootnote{Qrist`on \latintext
Socr.(bA)}} e>irhk'otac \edtext{>aneulab~wc}{\Dfootnote{>aneublab~wc \latintext
Socr.(A)}}, \edtext{>an'agkhn d`e dhlon'oti}{\Dfootnote{>an'agkh ded'hlwka (dedhlwk'enai
\latintext Socr.(b)) \greektext <'oti \latintext Socr.(M\textsuperscript{1}FA)}}
\edtext{\abb{>abo'ulhton}}{\Dfootnote{+ o~>usan \latintext Socr.(M\textsuperscript{1}FA)
\greektext + o>us'ian \latintext Socr.(M\textsuperscript{r})}} ka`i >aproa'ireton
peritejeik'otac t~w| je~w|, <'ina >'akwn genn'hsh| t`on \edtext{u<i'on}{\Dfootnote{k'urion
\latintext Socr.(T)}}, dussebest'atouc ka`i t~hc
\edtext{>ekklhs'iac}{\Dfootnote{>alhje'iac \latintext Socr.}} x'enouc
\edtext{>epigin'wskomen}{\Dfootnote{gin'wskomen \latintext Ath.(B) \greektext
>epigin'wskwmen \latintext Socr.(T)}}, <'oti \edtext{\abb{te}}{\Dfootnote{\latintext >
Socr.(T)}} par`a t`ac \edtext{koin`ac}{\Dfootnote{kain`ac \dt{Ath.(KPOR)} kain`ac to~u
\dt{Ath.(B*)}}} \edtext{per`i jeo~u >enno'iac}{\lemma{\abb{}}\Dfootnote{\responsio\
>enno'iac per`i jeo~u \latintext Socr.(bA) \greektext per`i + to~u \latintext Socr.(T)}}
ka`i \edtext{\abb{d`h ka`i}}{\Dfootnote{\latintext > Socr.(T) \greektext d`h \latintext
Socr.(bA)}} par`a t`o bo'ulhma t~hc jeopne'ustou graf~hc toia~uta tetolm'hkasi per`i
a>uto~u dior'isasjai. a>utokr'atora g`ar <hme~ic t`on je`on ka`i k'urion a>ut`on <eauto~u
e>id'otec <ekous'iwc a>ut`on ka`i \edtext{>ejelont`hn}{\Dfootnote{>ejelont'i \latintext
Ath.(P) Socr.(T) \greektext >ej'elonta \latintext Socr.(A) \greektext j'elonta \latintext
Socr.(b)}} t`on u<i`on gegennhk'enai e>useb~wc <upeil'hfamen.
\pend
\pstart
\kap{13}piste'uontec \edtext{\abb{d`e}}{\Dfootnote{\latintext > Socr.(T)}} >emf'obwc ka`i
\edtext{t~w|}{\Dfootnote{t`o \latintext Socr.(bA)}} per`i
\edtext{<eauto~u}{\Dfootnote{a>uto~u \latintext Socr. \greektext a<uto~u \latintext coni.
Hansen}} \edtext{l'egonti}{\Dfootnote{leg'omenon \latintext Socr.(bA) \greektext + <'oti
\latintext Ath.(R) Socr.(T)}}; ((\edtext{\abb{k'urioc >'ektis'en me >arq`hn
\edtext{<od~wn}{\Dfootnote{<od`on \latintext Socr.(M\textsuperscript{1}F)}} a>uto~u e>ic
>'erga a>uto~u}}{\lemma{\abb{}}\Afootnote{\latintext Prov
8,22}}))\edindex[bibel]{Sprueche!8,22}, o>uq <omo'iwc a>ut`on to~ic di'' a>uto~u
genom'enoic kt'ismasin >`h poi'hmasi \edtext{gegen~hsjai}{\Dfootnote{gegenn~hsjai
\latintext Socr.(A\textsuperscript{1})}} noo~umen. >aseb`ec g`ar ka`i t~hc
>ekklhsiastik~hc p'istewc >all'otrion \edtext{t`o}{\Dfootnote{t~w| \latintext Socr.(T)}}
t`on kt'isthn to~ic di'' a>uto~u \edtext{kektism'enoic}{\Dfootnote{>ektism'enoic
\latintext Socr.}} dhmiourg'hmasi parab'allein ka`i t`on
\edtext{\abb{a>ut`on}}{\Dfootnote{+ nom'izein \latintext Socr.(T)}} t~hc
\edtext{gen'esewc}{\Dfootnote{genn'hsewc \latintext Socr.(T)}} to~ic
\edtext{>'alloic}{\Dfootnote{>allotr'ioic \latintext Socr.(bA)}} tr'opon >'eqein
\edtext{\abb{ka`i a>ut`on nom'izein}}{\Dfootnote{\latintext > Socr.(T)}}. m'onon g`ar ka`i
m'onwc t`on monogen~h u<i`on \edtext{\abb{gegenn~hsjai}}{\Dfootnote{\latintext > Socr.}}
gnhs'iwc te ka`i >alhj~wc did'askousin <hm~ac a<i je~iai
\edtext{\abb{grafa'i}}{\Dfootnote{+ gegenn~hsjai \latintext Socr.}}.
\pend
\pstart
\kap{14}>all'' o>ud`e t`on u<i`on kaj'' <eaut`on e>~inai z~hn te ka`i <up'arqein <omo'iwc
\edtext{t~w| patr`i}{\Dfootnote{to~u patr`oc \latintext Socr.(T)}} l'egontec di`a
\edtext{to~uto}{\Dfootnote{t`o \latintext Socr.(A)}} qwr'izomen a>ut`on to~u patr`oc
t'opouc ka`i diast'hmat'a tina metax`u t~hc sunafe'iac a>ut~wn swmatik~wc >epinoo~untec.
pepiste'ukamen g`ar >amesite'utwc a>uto`uc ka`i \edtext{\abb{>adiast'atwc}}{\lemma{\abb{}}
\Dfootnote{>adiast'aktwc \latintext Ath.(B*)}}
\edtext{\abb{>all'hloic}}{\Dfootnote{\latintext > Socr.(bA)}}
>episun~hfjai ka`i \edtext{>aqwr'istouc}{\Dfootnote{>aqwr'istwc \latintext Socr.(bA)}}
<up'arqein <eaut~wn, \edtext{<'olou}{\Dfootnote{<'olon \latintext Socr.}} m`en to~u
patr`oc \edtext{>ensternism'enou}{\Dfootnote{>enesternism'enou \latintext Socr.(b)}} t`on
u<i'on, <'olou d`e to~u u<io~u >exhrthm'enou ka`i prospefuk'otoc t~w| patr`i ka`i
\edtext{m'onou}{\Dfootnote{m'onon \latintext Socr.}} to~ic patr'w|oic k'olpoic
\edtext{>epanapauom'enou}{\Dfootnote{>enanapauom'enou \latintext Ath.(P), \greektext
>anapauom'enou \latintext Ath.(R) \greektext >anapau'omenon \latintext Socr.}} dihnek~wc.
\pend
\pstart
\kap{15}piste'uontec o>~un e>ic t`hn pant'eleion tri'ada t`hn <agiwt'athn,
\edtext{tout'estin e>ic t`on pat'era ka`i e>ic t`on u<i`on ka`i e>ic t`o pne~uma t`o
<'agion, ka`i}{\lemma{\abb{tout'estin \dots\ <'agion, ka`i}}\Dfootnote{\latintext >
Socr.}} \edtext{\abb{je`on}}{\Dfootnote{\latintext > Socr.(bA)}}
\edtext{\abb{m`en}}{\Dfootnote{\latintext > Socr.}} t`on pat'era l'egontec, je`on
\edtext{\abb{d`e}}{\Dfootnote{\latintext > Socr.(bA)}} ka`i t`on u<i'on, o>u d'uo to'utouc
jeo'uc, >all''
\edtext{<`en}{\Dfootnote{<'ena \latintext Socr.}} \edtext{<omologo~umen t~hc je'othtoc
>ax'iwma}{\Dfootnote{<omologo~umen kat`a t`o t~hc je'othtoc >ax'iwma \latintext Socr.(bA)
\greektext <omologo~umen kat`a t~hc je'othtoc t`o >ax'iwma \latintext Socr.(T)}} ka`i
m'ian >akrib~h t~hc \edtext{basile'iac}{\Dfootnote{>alhje'iac \latintext Socr.(A)}} t`hn
\edtext{sumfwn'ian}{\Dfootnote{sun'afeian \latintext Socr.(bA)}}, pantarqo~untoc m`en
\edtext{\abb{kaj'olou}}{\Dfootnote{+ patr`oc \latintext Socr.(T) \greektext + to~u patr`oc
\latintext Socr.(bA)}} p'antwn ka`i a>uto~u to~u u<io~u \edtext{\abb{m'onou to~u
patr'oc}}{\Dfootnote{\latintext > Socr.}}, to~u d`e u<io~u <upotetagm'enou t~w| patr'i,
>ekt`oc d`e a>uto~u \edtext{\abb{p'antwn}}{\Dfootnote{+ t`on \latintext Socr.(A)
\greektext + t~wn \latintext Socr.(bT)}} met'' a>ut`on basile'uontoc t~wn di'' a>uto~u
genom'enwn ka`i t`hn to~u <ag'iou pne'umatoc q'arin >afj'onwc to~ic <ag'ioic dwroum'enou
patrik~w| boul'hmati. o<'utw g`ar t`on per`i t~hc \edtext{e>ic Qrist`on}{\Dfootnote{>en
Qrist~w| \latintext Ath.(R) Socr.}} monarq'iac sun'istasjai l'ogon
\edtext{par'edosan}{\Dfootnote{par'edwkan \latintext Socr.(T) \greektext parad'edwkan
\latintext Socr.(M\textsuperscript{1}) \greektext paraded'wkasin \latintext Socr.(M\corr
FA)}} <hm~in o<i <iero`i l'ogoi.
\pend
\pstart
\kap{16}\edtext{ta~uta}{\Dfootnote{ta'uthn \latintext Socr.(T)}} >hnagk'asjhmen met`a t`hn
>en >epitom~h| \edtext{proekteje~isan}{\Dfootnote{proteje~isan \latintext Socr.(T)
\greektext >ekteje~isan \latintext Socr.(bA)}} p'istin \edtext{plat'uteron
\edtext{>epexerg'asasjai}{\Dfootnote{>eperg'asasjai \latintext Socr.(A)}} o>u kat`a
peritt`hn filotim'ian}{\lemma{\abb{plat'uteron \dots\ filotim'ian}}\Dfootnote{\latintext >
Socr.(T)}}, >all'' <'ina p~asan \edtext{\abb{t`hn}}{\Dfootnote{+ kat`a \latintext Socr.}}
t~hc <hmet'erac <upol'hyewc >allotr'ian \edtext{>anakaj'arwmen}{\Dfootnote{>apokaj'arwmen
\latintext Socr.}} <upoy'ian par`a to~ic t`a kaj'' <hm~ac >agnoo~usi ka`i gn~wsin o<i
kat`a t`hn d'usin p'antec <omo~u m`en t~hc sukofant'iac t~wn <eterod'oxwn t`hn
>ana'ideian, <omo~u d`e t~wn >anatolik~wn t`o >ekklhsiastik`on >en
\edtext{kur'iw|}{\Dfootnote{Qrist~w| \latintext Socr.}} fr'onhma, marturo'umenon
>abi'astwc <up`o t~wn jeopne'ustwn graf~wn \edtext{par`a to~ic
>adiastr'ofoic}{\Dfootnote{par'' a>uto~ic >adiastr'ofwc \latintext Socr.(bA)}}.
\pend
\endnumbering
\end{Leftside}
\begin{Rightside}
\begin{translatio}
\beginnumbering
\pstart 
\noindent\kapR{1}Wir glauben an einen Gott, Vater, Allm�chtigen,
Sch�pfer und Erschaffer von allem, nach dem alle Vaterschaft im Himmel
und auf Erden benannt wird.  
\pend 
\pstart 
Und an seinen eingeborenen
Sohn, unsern Herrn Jesus Christus, der vor allen Zeiten aus dem Vater
gezeugt worden ist, Gott aus Gott, Licht aus Licht, durch den alles im
Himmel und auf Erden wurde, das Sichtbare und das Unsichtbare, der
Wort, Weisheit, Kraft, Leben und wahres Licht ist, der in den letzten
Tagen f�r uns Mensch und geboren wurde aus der heiligen Jungfrau, der
gekreuzigt wurde, starb, begraben wurde und am dritten Tag von den
Toten auferstand und in den Himmel hinaufgenommen wurde und sich zur
Rechten des Vaters setzte und der wiederkommen wird am Ende der
Zeit, zu richten die Lebenden und die Toten und jedem nach seinen
Werken zu vergelten, dessen Herrschaft unauf"|l�slich ist und f�r alle
Zeiten bestehen bleibt; er sitzt n�mlich zur Rechten des Vaters nicht
nur in dieser Zeit, sondern auch in der k�nftigen.  
\pend 
\pstart 
Wir
glauben auch an den heiligen Geist, das hei�t an den Tr�ster, den er
den Aposteln versprochen hatte und nach seinem Aufstieg in den
Himmel sandte, damit er sie lehre und an alles erinnere, wodurch auch die
Seelen derer, die aufrichtig an ihn geglaubt haben, geheiligt werden.
\pend 
\pstart 
\kapR{2}Die aber sagen\footnoteA{Die urspr�nglich dem
  vierten Antiochenum angeh�ngten Anathemastismen, die formal in Anlehnung an die nicaenischen formuliert, aber theologisch anders gewichtet waren, erfuhren schon in Serdica eine Erg�nzung
  (vgl. Dok. \ref{sec:BekenntnisSerdikaOst}), die sich haupts�chlich
  gegen Markell und Photinus richtet. Diese Anathematismen werden auch hier
  wiederholt. Anschlie�end werden sie fast in derselben Reihenfolge
  (nur das Thema von � 6 wurde vorgezogen) ausf�hrlicher
  besprochen, einzig die Ausf�hrungen zum ersten Block der
  Anathematismen fallen relativ kurz aus (� 5). In �
  13--15 folgen zus�tzliche Bemerkungen �ber Argumente der
  Gegenseite und eine Zusammenfassung �ber das Verh�ltnis der Trias
  (Vater, Sohn und heiliger Geist) zur Monarchie, eine rechtfertigende
  Erkl�rung zur L�nge des Schriftst�cks (� 16) schlie�t dieses
  ab.}, der Sohn sei aus nichts oder aus einer anderen Hypostase und
nicht aus Gott, und da� es einmal eine Zeit oder Epoche gab, in der er
nicht war, die kennt die katholische und heilige Kirche als
Fremde. Ebenfalls verbannt die heilige und katholische Kirche die, die
sagen, es g�be drei G�tter oder Christus sei nicht Gott oder vor den
Zeiten w�re er nicht Christus noch Gottes Sohn, oder Vater, Sohn und
heiliger Geist seien derselbe oder der Sohn sei ungezeugt oder da�
der Vater den Sohn nicht aus seinem Wollen und Willen gezeugt habe.
\pend 
\pstart 
\kapR{3}Denn es ist weder abgesichert zu sagen, der
Sohn sei aus nichts\footnoteA{Da� der Sohn nicht aus nichts, auch
  nicht aus einer anderen Hypostase und nie einmal nicht war,
  wiederholen die Antiochener aus dem Nicaenum, um sich gegen den
  Vorwurf abzusichern, sie seien Arianer. Da die Antiochener dem Sohn
  aber dennoch einen zeitlosen Anfang zuweisen, da nur der Vater
  anfangslos und ungeworden ist (s. folgenden Abschnitt), hei�t es
  hier und in den Anathematismen eindeutig, da� es keine ">Zeit"<
  gegeben habe, in der der Sohn nicht war, da die Zeiten erst durch
  ihn wurden, und nicht: ">es einmal war, da� er nicht war."<}, da
dies nirgendwo in den von Gott eingegebenen Schriften �ber ihn
ausgesagt wird, noch freilich, er w�re aus irgendeiner anderen
Hypostase au�er dem Vater, sondern wir legen fest, da� er in Wahrheit aus
Gott allein gezeugt ist. Denn der g�ttliche Logos lehrt, da� einer das
Ungezeugte und Anfangslose ist, n�mlich der Vater des Christus.  Aber sie d�rfen
auch nicht irgendeinen zeitlichen Abstand f�r ihn annehmen,
wenn sie gef�hrlicherweise das nicht aus den Schriften stammende ">es
war einmal, da� er nicht war"< sagen, sondern nur Gott, der ihn
zeitlos gezeugt hat; denn auch die Zeiten und �onen sind durch ihn
geworden.  
\pend 
\pstart 
\kapR{4}Noch darf man freilich glauben, der
Sohn sei mit dem Vater zusammen anfangslos und ungeworden\footnoteA{Zu
  dieser Diskussion vgl. schon Dok. \ref{ch:1} = Urk. 1,3.5; Dok. \ref{ch:6} = Urk. 6,4; Dok. \ref{ch:14} = Urk. 14,44--46.51~f.; ferner Markell, fr. 123
  (Seibt/Vinzent); Eus., e.\,th.  I 2 (63,23 Klostermann/Hansen); I 11;
  II 6--7.}. Denn keiner kann mit Fug und Recht Vater oder Sohn genannt
werden von jemandem, mit dem er zusammen anfangslos und ungeworden
ist.  Sondern wir wissen, da� der Vater, der allein anfangslos und
ungezeugt ist, unerreichbar und f�r alle unbegreif"|lich gezeugt hat und
da� der Sohn vor den Zeiten gezeugt worden und niemals in irgendeiner
Weise wie der Vater auch selbst ungezeugt ist, sondern den zeugenden
Vater als Anfang hat: ">denn das Haupt Christi ist Gott."< 
\pend
\pstart 
\kapR{5}Auch machen wir nicht, indem wir drei reale Dinge und
drei Personen des Vaters und des Sohnes und des heiligen Geistes nach
den Schriften bekennen, deshalb drei G�tter\footnoteA{Den Antiochenern
  wurde von Markell (vgl. bes. fr. 48; 116~f.; 120 [Seibt/Vinzent] und
  Dok. 40) und auch dem Schreiben der ">westlichen"< Synode von
  Serdica vorgeworfen, mit ihrer Hypostasenlehre eine
  Drei-G�tter-Lehre zu verbreiten, womit sich schon Eusebius von Caesarea
  auseinandersetzt (Marcell. I 4; e.\,th. I 20; II 7; 23). Markell sprach
  dagegen immer von einer Person (vgl. fr. 92; 97 [Seibt/Vinzent])
  oder der Monas.}, da wir den unabh�ngigen und ungezeugten,
anfangslosen und unsichtbaren Gott als einen allein kennen, den Gott
und Vater des Eingeborenen, der als einziger aus sich heraus das Sein
hat und als einziger allen �brigen ohne Neid dies gew�hrt. 
\pend
\pstart 
\kapR{6}Auch leugnen wir freilich, wenn wir sagen, der Vater unseres
Herrn Jesus Christus sei ein Gott allein, der allein Ungezeugte,
deswegen nicht, da� auch Christus Gott vor Zeiten ist, wie die
Anh�nger des Paulus von Samosata, die sagen, sp�ter, nach seiner
Menschwerdung, sei er ausgehend davon, da� er von Natur aus nur Mensch gewesen ist,
aufgrund eines Fortschritts zu Gott gemacht
worden\footnoteA{Der Argumentationsstruktur nach wird hier wieder
  ein Vorwurf abgewehrt, diesmal dergestalt, da� die Antiochener
  Christus nicht seine Gottheit zuerkennen w�rden; dies warf schon
  Markell Eusebius von Caesarea vor (vgl. fr.  126--128 [Seibt/Vinzent]:
  Eusebius halte den Erl�ser f�r einen blo�en Menschen). Hinzu kommt die
  Vorstellung, Christus habe seine Gottheit erst aufgrund seiner
  Verdienste hinzuerworben (vgl. Ath., Ar. I 38,2.4; 43,6; II 28,5; auch Asterius, fr. 43; 45; Ath.,
  decr. 6,4); auch in Ath.,
  Ar. I 38,4 wird dies Paulus von Samosata zugeschrieben. F�r die
  Antiochener erneuere aber umgekehrt gerade Markell selbst die Lehren
  des Paulus von Samosata (vgl. Dok. \ref{ch:Konstantinopel336}; \ref{sec:AntIII}), da dieser nicht die
  volle Gottheit dem Sohne zuerkenne, sondern erst dem Inkarnierten
  (vgl. folgenden Abschnitt). �ber die tats�chlichen Lehren des Paulus
  von Samosata ist wenig bekannt; nach Eus., h.\,e. VII 27,2 hielt er
  Jesus Christus der Natur nach f�r einen gew�hnlichen Menschen, nicht
  vom Himmel herabgekommen (VII 30,16), sondern von unten (VII
  30,11). Nach Hil., ad Const. II (\editioncite[204]{Hil:coll})
  lehrte �hnliches auch Photinus (vgl. � 8) unter Berufung auf
  Rom 5,15: ">der Mensch Jesus
  Christus"<. Vgl. Dok. \ref{ch:Konstantinopel336},3,4 Anm.}.  Denn
wir wissen, da� er, auch wenn er dem Vater und Gott untergeordnet ist,
dennoch der Natur nach vollkommener und wahrer Gott ist, da er vor den Zeiten aus Gott gezeugt wurde, 
und da� er nicht von Menschen
abstammend danach Gott, sondern aus Gott kommend unsretwillen Mensch
geworden ist und niemals das Gott-Sein verloren hat.  
\pend 
\pstart
\kapR{7}Wir verabscheuen aber dar�ber hinaus und verurteilen auch die,
die ihn heuchlerisch nur blo�es Wort Gottes und existenzlos
nennen\footnoteA{Vgl. zu dieser Diskussion Markell, fr. 3; 5; 7; 66;
  71; 110 (Seibt/Vinzent); Eus., e.\,th. I 17; II 9--17.}, der in einem
anderen das Sein habe, einmal von einigen ausgesprochenes (Wort)
genannt, einmal innerliches, da sie wollen, da� er nicht Christus,
Sohn Gottes, Mittler und Bild Gottes vor den Zeiten sei, sondern von dem
Zeitpunkt an Christus und Sohn Gottes geworden sei, als er unser
Fleisch aus der Jungfrau vor knapp vierhundert
Jahren\footnoteA{Vgl. Markell, fr.  103; 104 (Seibt/Vinzent).}
angenommen hat. Denn sie wollen, da� Christus von da an mit seiner
Herrschaft begann und da� sie nach der Vollendung und dem Gericht
ein Ende haben wird\footnoteA{Vgl. Markell, fr. 101; 102; 103; 105
  (Seibt/Vinzent).}.  
\pend 
\pstart 
\kapR{8}Solche Menschen aber sind
die Anh�nger von Markell und Photinus\footnoteA{Photinus war ein
  Sch�ler von Markell (vgl. Hil., coll.\,antiar. B II 5,4; B II 9) und
  wird hier erstmals namentlich genannt und verurteilt. Die in der
  Athanasius-�berlieferung zu findende Lesart des Namens ">Skotinus"<
  (Photinus bringt nicht das Licht, sondern die Dunkelheit) ist wohl
  eine athanasianische Verunglimpfung des Namens.} aus
dem galatischen Ancyra, die die vorzeitliche Existenz Christi und
seine Gottheit und unendliche Herrschaft genauso wie die Juden
aufheben unter dem Vorwand, sie w�rden angeblich die Monarchie
bewahren. Denn wir kennen ihn nicht nur einfach als ausgesprochenes
oder innerliches Wort Gottes, sondern als lebendigen Gott Logos, der
f�r sich selbst existiert, und als Sohn Gottes und Christus, und nicht
als einen, der mit vorherigem Wissen vor den Zeiten mit seinem eigenen
Vater zusammen ist und bei ihm verweilt und f�r die gesamte Sch�pfung,
sei es der sichtbaren, sei es der unsichtbaren Dinge, bei ihm dient,
sondern der eigenexistierendes Wort des Vaters ist und Gott aus
Gott. Denn dieser ist es, zu dem der Vater gesagt hat: ">La�t uns
einen Menschen machen nach unserem Bild wie wir"<, der auch in eigener
Person von den Patriarchen gesehen worden ist, der das Gesetz
�bergeben und durch die Propheten geredet hat, der zum Schlu� Mensch
geworden ist und seinen Vater allen Menschen offenbart hat und f�r
unendliche Zeiten herrscht. Denn Christus hat nicht eine neue W�rde
erlangt, sondern wir glauben, da� er von Anfang an vollkommen und dem
Vater in jeder Hinsicht gleich ist\footnoteA{Eine hier erstmals von den
  Antiochenern in einem Bekenntnis verwendete Formulierung, die sp�ter
  zum Schlagwort der Hom�er werden sollte; vgl. aber z.\,B. auch Ath.,
  Ar. I 21,4; 40,4; II 18,2; ferner decr.  20,1.}.  
\pend 
\pstart
\kapR{9}Die aber sagen, Vater, Sohn und heiliger Geist seien ein und
derselbe\footnoteA{Auch dieses Thema wurde zwischen Eusebius und
  Markell diskutiert: Markell., fr. 47; 48; 73; 92; 97; 125
  (Seibt/Vinzent); Eus., e.\,th. I 1; 5; 14; vgl. auch die Dokumente zu
  Serdica (Dok. \ref{sec:SerdicaWestBekenntnis},6).}, und die die drei
Namen gottlos f�r ein und dieselbe reale Sache und Person halten,
versto�en wir zu Recht aus der Kirche, denn sie halten den
grenzenlosen und leidenslosen Vater f�r begrenzbar und zugleich f�r
leidensf�hig\footnoteA{Da� eine Identifikation von Vater, Sohn und
  heiligem Geist dazu f�hrt, dem Vater eine Leidensf�higkeit
  zuschreiben zu m�ssen, diskutiert noch nicht Eusebius in seinen
  antimarkellischen Schriften, findet man aber in antiphotinianischen
  Texten (Ps.-Ath., c.\,Sabell. 13); interessanterweise auch
  Dok. \ref{sec:SerdicaWestBekenntnis},9. Schon Eusebius aber kritisierte Markell als Sabellianer
  (vgl. Dok. \ref{sec:Eustathius},1 zu Sabellius).} aufgrund der
Menschwerdung. Denn um derartige handelt es sich, die bei den R�mern
Patropassianer, bei uns aber Sabellianer genannt werden. Denn wir
wissen, da� der sendende Vater an seinem vertrauten Ort der
unwandelbaren Gottheit bleibt, wogegen der gesandte Christus die
Heilsordnung der Menschwerdung erf�llte.  
\pend 
\pstart 
\kapR{10}Zugleich aber durchschauen wir auch die, die irrsinnigerweise sagen, der
Sohn sei nicht aus Wunsch und Wollen gezeugt\footnoteA{Hintergrund ist
  die Debatte, ob der Sohn aus dem Wesen oder aus dem Willen des
  Vaters gezeugt wird (Arius, Dok. \ref{ch:1} = Urk. 1,4; Asterius, fr. 16; 18--20;
  Ath., Ar. III 59--62).}, offensichtlich aber Gott einen willenslosen
und wahllosen Zwang zuschreiben, damit er unfreiwillig den Sohn zeuge,
als �u�erst gottlos und der Kirche fremd, da sie entgegen der allgemeinen
Ansichten �ber Gott und dazu entgegen der Aussageabsicht der von Gott
inspirierten Schrift solches �ber ihn festzustellen gewagt haben. Denn
wir, die wir wissen, da� Gott unabh�ngig und sein eigener Herr ist,
sind der gottesf�rchtigen �berzeugung, da� er freiwillig und
willentlich den Sohn gezeugt hat.  
\pend 
\pstart 
\kapR{11}Da wir aber
auch furchtsam dem glauben, der �ber sich sagt: ">Der Herr schuf mich
am Anfang seiner Wege zu seinen Werken"<, wissen wir, da� er nicht so
wie die durch ihn gewordenen Gesch�pfe oder Werke geworden
ist\footnoteA{Die Antiochener betonen, auch ohne Markells neue
  Interpretation von Prov 8,22 als Beschreibung des Inkarnierten (vgl.
  Markell., fr. 26--32 [Seibt/Vinzent]; Ath., Ar. II 44) k�nne dieser
  Vers korrekt verstanden werden, wenn man nur darauf achte, den Sohn
  nicht aufgrund des Verbes ">schaffen"< zu den Gesch�pfen zu z�hlen
  (vgl. Asterius, fr. 48 und Euseb, e.th. III 2).}. Denn es ist gottlos und dem kirchlichen
Glauben fremd, den Sch�pfer mit den durch ihn geschaffenen Werken zu
vergleichen und zu glauben, er habe auch selbst dieselbe Art der Entstehung wie
die anderen.  Denn die g�ttlichen Schriften lehren uns, da� einzig und
allein der eingeborene Sohn im eigentlichen Sinne und wahrhaftig
gezeugt worden ist.  
\pend 
\pstart 
\kapR{12}Aber auch wenn wir sagen,
der Sohn ist, lebt und existiert f�r sich wie der Vater, trennen wir
ihn deswegen nicht vom Vater\footnoteA{Dies ist der Hauptvorwurf von
  Markell und seinen Anh�ngern an die Hypostasentheologie der
  Antiochener (vgl. Dok.  \ref{sec:SerdicaWestBekenntnis},2), worauf
  hier entgegnet wird, da� eine Unterscheidung keine Trennung
  bedeute.} und denken nicht an irgendeinen Raum oder Abstand zwischen
deren Verbindung wie bei k�rperlichen Dingen. Wir glauben n�mlich, da�
sie ohne Mittler und ohne Abstand miteinander verbunden sind und
ungetrennt voneinander existieren, da� der Vater v�llig den Sohn an
seine Brust zieht und der Sohn v�llig am Vater h�ngt und klammert und
da� nur er ohne Unterla� im v�terlichen Scho� bleibt.  
\pend 
\pstart
\kapR{13}Wir glauben also an die vollkommene heiligste Dreiheit,
d.\,h. an den Vater und an den Sohn und an den heiligen Geist, und Gott
nennen wir den Vater, Gott aber auch den Sohn, und bekennen daher
nicht diese zwei G�tter, sondern eine W�rde der Gottheit und eine
genaue �bereinstimmung in der Herrschaft, wobei allein der Vater �berall �ber
alle Dinge herrscht und auch �ber seinen Sohn, der Sohn aber dem
Vater untergeordnet ist und au�er �ber ihn (sc. den Vater) �ber alle Dinge,
die durch ihn geworden sind, mit ihm herrscht und die Gnade des
heiligen Geistes ohne Neid den Heiligen nach dem v�terlichen Willen
verleiht.  Denn derart die Lehre �ber die Monarchie auf Christus zu
beziehen, haben uns die heiligen Worte �berliefert.  
\pend 
\pstart
\kapR{14}Wir sahen uns gezwungen, nach der Vorlage des Glaubens in
Kurzfassung dies ausf�hrlicher auszuarbeiten, nicht aufgrund von
�berfl�ssiger Ehrsucht, sondern damit wir alle fremden Verd�chtigungen
gegen unser Anliegen bereinigen bei denen, die uns nicht kennen, und
damit alle die im Westen sowohl die Unversch�mtheit der Verleumdung
der H�retiker kennen als auch die im Herrn begr�ndete kirchliche
Ansicht derer im Osten, die von den von Gott inspirierten Schriften ohne Gewalt
unter den Rechtschaffenen bezeugt wird.  
\pend
\endnumbering
\end{translatio}
\end{Rightside}
\Columns
\end{pairs}
% \renewcommand*{\goalfraction}{.9}
%%% Local Variables: 
%%% mode: latex
%%% TeX-master: "dokumente_master"
%%% End: 

%%%%%%%%%%%%%%%%%%%%%%%%%%%%%%%%%%%%%%%%%%%
%%%%%%%%%%%%%% INDICES ETC. %%%%%%%%%%%%%%%%%%%
%%%%%%%%%%%%%%%%%%%%%%%%%%%%%%%%%%%%%%%%%%%
\backmatter
\aliaspagestyle{chapter}{empty}
\renewcommand{\clearforchapter}{\clearpage}  % Chapter wieder neue Seite
\chapter*{Literatur}
\label{literatur}
\addcontentsline{toc}{chapter}{Literatur}
\markboth{Literatur}{Literatur}
\printbibliography[heading=edition,keyword=edition]
\printbibliography[heading=literatur,keyword=literatur]
% \bibliographystyle{jurabibtheoneu}
% \begin{btSect}{editionen}
% \section{Editionen}
% \btPrintAll
% \end{btSect}
% \begin{btSect}{dokumente}
% \section{Sekund�rliteratur}
% \btPrintAll
% \end{btSect}
%%%%%%%%% Index zu III/1-2
\chapter*{Vergleichender �berblick �ber die Z�hlung der in III/1 und 2\\edierten Texte}
\markboth{Vergleichender �berblick �ber die Z�hlung der in III/1 und 2 edierten Texte}{Vergleichender �berblick �ber die Z�hlung der in III/1 und 2 edierten Texte}
\addcontentsline{toc}{chapter}{Vergleichender �berblick �ber die Z�hlung der in III/1 und 2 edierten Texte}
\label{synopse}
\begin{longtable}[c]{p{1.5cm}p{2cm}p{1.5cm}}
 Urk. 1 & & Dok. 15\\
Urk. 2 & & Dok. 16\\
Urk. 3 & & Dok. 10\\
Urk. 4a & & Dok. 2.1\\
Urk. 4b & & Dok. 2.2\\
Urk. 5 & & Dok. 3\\
Urk. 6 & & Dok. 1\\
Urk. 7 & & Dok. 9\\
Urk. 8 & & Dok. 4\\
Urk. 9 & & Dok. 5\\
Urk. 10 & & Dok. 8\\
Urk. 11 & & Dok. 11\\
Urk. 12 & & Dok. 6\\
Urk. 13 & & Dok. 7\\
Urk. 14 & & Dok. 17\\
Urk. 15 & & Dok. 14\\
Urk. 16 & & Dok. 18\\
Urk. 17 & & Dok. 19\\
Urk. 18 & & Dok. 20\\
Urk. 19 & & Dok. 21\\
Urk. 20 & & Dok. 22\\
Urk. 21 & & Dok. 23\\
Urk. 22 & & Dok. 24\\
Urk. 23 & & Dok. 25\\
Urk. 24 & & Dok. 26\\
Urk. 25 & & Dok. 29\\
Urk. 26 & & Dok. 30\\
Urk. 27 & & Dok. 31\\
Urk. 28 & & Dok. 32\\
Urk. 29 & & Dok. 33\\
Urk. 30 & & Dok. 34\\
Urk. 31 & & Dok. 36\\
Urk. 32 & & Dok. 37\\
Urk. 33 & & Dok. 28\\
Urk. 34 & & Dok. 27\\
\end{longtable}
\chapter*[Register zu den in III/1 und 2 edierten Texten]{Register zu den in III/1 und 2
edierten Texten}
\addcontentsline{toc}{chapter}{Register zu den in III/1 und 2 edierten Texten}
\begin{quote}
\noindent\footnotesize Die Seitenangaben beziehen sich auf Seiten und Zeilen in den
beiden Faszikeln, nicht auf die Seitenangaben in diesem Band.
\end{quote}
\input{opitz_bibel.index}
\input{opitz_quellen.index}
\input{opitz_namen.index}
%%%%%%%%%%%%%%
\chapter[Register]{Register}
\renewcommand{\preindexhook}{\noindent\footnotesize Kursivierte Ziffern beziehen sich auf Stellen, an denen kein Zitat, sondern nur eine Anspielung vorliegt.}
\renewcommand{\indexname}{Verzeichnis der Bibelstellen}
\input{bibel.index}
\renewcommand{\preindexhook}{}
\renewcommand{\indexname}{Verzeichnis der Quellen}
\input{quellen.index}
\renewcommand{\preindexhook}{\noindent\footnotesize Kursivierte Ziffern beziehen sich auf Stellen, an denen nicht sicher auszumachen ist, ob es sich um die genannte Person handelt.}
\renewcommand{\indexname}{Verzeichnis der Personen- und Ortsnamen}
\input{namen.index}
\renewcommand{\preindexhook}{}
\renewcommand{\indexname}{Verzeichnis der Synoden}
\input{synoden.index}
\end{document}
%%% Local Variables:
%%% mode: latex
%%% TeX-master: t
%%% End:
