% \section{Brief des Euseb von Caesarea an Euphration von Balane�
\kapitel{Fragmente eines Briefes des Eusebius von Caesarea an Euphration von Balaneae (Urk. 3)}
\label{ch:3}

\kapnum{1}(Brief) des Eusebius, des Pamphilos' Sohn, an Euphration, dessen Anfang (lautet): Meinem Herrn
bekenne ich nach aller Gnade.

\begin{footnotesize}
Und nach weiteren Worten:
\end{footnotesize}

Denn wir sagen nicht, da� der Sohn zusammen mit dem Vater existiert, sondern da� der Vater vor dem
Sohn existiert.
Denn falls sie zusammen existierten, wie kann da der Vater Vater und der Sohn Sohn sein? Oder wie
ist der eine Erster, der andere aber Zweiter? Und der eine ungezeugt, der andere aber gezeugt? Denn
zwei, die einander genau gleich sind und zusammen existieren, d�rften der gleichen Ehre wert gehalten werden und sind
entweder beide, wie ich sagte, ungezeugt oder beide gezeugt. Doch keines von beiden ist wahr; denn
weder das Ungezeugte noch das Gezeugte d�rften beide zugleich existieren, sondern das eine wird f�r das Erste und
f�r st�rker sowohl der Ordnung als auch der Ehre nach als das Zweite gehalten, da es ja f�r das Zweite
auch zur Ursache sowohl f�r das Sein als auch das So-Sein wurde.

\kapnum{2}Abgesehen (davon) hat der Sohn Gottes selbst, der besser als alle genau sich auskennt und
wei�, da� er selbst ein anderer als der Vater und geringer und hervorgegangen ist, ganz und gar
fromm dies auch uns gelehrt und gesagt: ">Der Vater, der mich gesandt hat, ist gr��er als ich."<

\begin{footnotesize}
\kapnum{3}Und es wurde aus demselben Brief vorgelesen:
\end{footnotesize}

Er lehrt, da� derselbe auch der einzig Wahrhaftige ist, indem er sagt: ">damit sie dich als den
einzig wahrhaftigen Gott erkennen"<, nicht so, als ob Gott allein einer w�re, sondern als ob nur ein
einzig wahrhaftiger Gott existierte mit der absolut notwendigen Hinzuf�gung des \frq wahrhaftig\flq.
Denn der Sohn ist auch selbst Gott, aber nicht wahrhaftiger Gott. Denn ein einziger ist auch einzig
wahrhaftiger Gott, weil er niemanden vor sich hat. Denn wenn auch der Sohn selbst wahrhaftiger
(Gott) ist, dann ist er doch wohl Gott wie ein Abbild des wahrhaftigen Gottes, da ">das
Wort auch Gott war"<, freilich nicht so wie der einzige, wahrhaftige Gott.

\begin{footnotesize}
\kapnum{4}Er erdreistete sich, das Wort von Gott zu trennen und das Wort einen
anderen Gott zu nennen, der im Blick auf sein Wesen und die Macht vom Vater unterschieden
ist. Und in welch gro�e Blasphemie er dabei verfiel, kann man klar und leicht an
den von ihm geschriebenen Worten lernen.
Denn er hat mit eben diesen Worten geschrieben:
\end{footnotesize}

Freilich wird das Bild und das, dessen Abbild es ist, nicht f�r ein und dasselbe gehalten,
sondern (es sind) zwei Wesen, zwei Dinge und zwei M�chte, wie es auch so viele Bezeichnungen gibt. 

\begin{footnotesize}
\kapnum{5}Da er den Heiland nur als Menschen erweisen wollte, sagte er folgendes, als ob
er uns das gr��te, unsagbare Geheimnis des Apostels aufdeckte:
\end{footnotesize}

Deswegen hat ganz sicher auch der g�ttliche Apostel uns die unsagbare und mystische Lehre �ber Gott
gegeben und rief und schrie ">Einer ist Gott"<, und danach sagt er, nach dem einen
Gott ist ">ein Mittler Gottes und der Menschen, der Mensch Christus Jesus"<.