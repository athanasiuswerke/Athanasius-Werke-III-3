% \section{Fragment des Briefes des Euseb von Nikomedien an die Synode von Nic�a}
\kapitel{Fragment eines Briefes des Eusebius von Nikomedien (Urk.~21)}
\label{ch:21}
\thispagestyle{empty}

\begin{footnotesize}
Wie ihr Urheber, Eusebius von Nikomedien, in seinem Brief verr�t und schreibt:
\end{footnotesize}

Wenn wir wirklich den Sohn Gottes ebenfalls ">ungeworden"< nennen, dann k�nnen wir auch gleich anfangen, 
ihn als ">wesenseins"< zu bekennen.

\begin{footnotesize}
Als dieser Brief auf der Synode von Nicaea verlesen wurde, nahmen die V�ter dieses Wort in die
Glaubenserkl�rung auf, da sie sahen, da� es ein Schreckgespenst f�r ihre Gegner war.
\end{footnotesize}