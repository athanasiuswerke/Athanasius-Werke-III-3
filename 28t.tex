% \section{Brief Kaiser Konstantins an Theodot von Laodicea}
\kapitel{Brief des Kaisers Konstantin an Theodot von Laodicea (Urk.~28)}
\label{ch:28}

Der Sieger Konstantin Augustus an Theodot.

\kapnum{1}Wie gro� die Gewalt des g�ttlichen Zorns ist, kannst auch du aus dem lernen, was Eusebius
und Theognis erlitten haben, die sich bei der heiligen Gottesverehrung wie Betrunkene aufgef�hrt und
den Namen Gottes, des Erl�sers, auch nachdem ihnen verziehen wurde, mit der Bildung einer eigenen Bande
besudelt haben. Als es n�mlich nach der einm�tigen �bereinkunft auf der Synode besonders wichtig gewesen
w�re, den fr�heren Irrtum zurechtzur�cken, da wurden sie �berf�hrt, wie sie an ihren unm�glichen
Positionen festhielten.

\kapnum{2}Deswegen hat also die g�ttliche Vorsehung sie aus ihrem eigenen Volk ausgesto�en.
Da sie es auch nicht ertrug mitanzusehen, wie der Unsinn von wenigen unschuldige Seelen verdarb,
forderte sie von ihnen f�r den gegenw�rtigen Zeitpunkt eine angemessene Strafe; eine gr��ere
Bestrafung f�r die Zukunft die ganze Ewigkeit hindurch steht aber noch aus.

\kapnum{3}Ich war der Meinung, da� dies deinem Scharfsinn mitgeteilt werden mu�, damit du darauf
achtest, falls irgendein schlechter Rat von diesen, was ich aber nicht glaube, vor deine
Entscheidung gebracht wird, diesen von der Seele fernzuhalten und dir, wie es sich geziemt, eine
reine Gesinnung und einen makellosen, heiligen und unbefleckten Glauben an Gott, den Erl�ser, zu
bewahren. Und so mu� jemand reagieren, der sich entschlie�t, den unversehrten Kampfpreis des ewigen
Lebens zu verdienen.

\textit{Und mit anderer Hand:}
M�ge Gott dich, geliebter Bruder, bewahren.