% \section{Brief des Euseb von Nikomedien und Theognis von Niz�a an die zweite Synode von
% Nic�a}
\kapitel{Brief des Eusebius von Nikomedien und des Theognis von Nicaea an eine Synode (Urk.~31)}
\label{ch:31}


\kapnum{1}Als bereits Verurteilte vor dem Gericht eurer Fr�mmigkeit mu�ten wir bei dem Urteil eurer
heiligen Entscheidung Ruhe bewahren. Da es aber unangemessen ist, durch Schweigen gegen sich selbst
den Verleumdern einen Beweis in die H�nde zu spielen, deswegen geben wir bekannt, da� wir mit dem
Glauben �bereingestimmt und nach einer Pr�fung des Sinnes von ">wesenseins"< ganz und gar den Frieden
unterst�tzt haben; wir sind n�mlich keineswegs Anh�nger der H�resie gewesen.

\kapnum{2}Nachdem wir zur Sicherheit der Kirchen noch einmal dargelegt haben, wie unsere
Gedankeng�nge verlaufen, und die �berzeugt haben, die von uns �berzeugt werden mu�ten, haben wir
dem Glauben zugestimmt, die Verurteilung haben wir aber nicht unterschrieben,
nicht als ob wir damit den Glauben tadelten, sondern da wir nicht der Ansicht sind, derartiges gelte
f�r den Angeklagten, von dem �berzeugt, was er selbst uns mit Briefen oder in pers�nlichen Gespr�chen
vorbrachte, da� derartiges nicht auf ihn zutreffe.

\kapnum{3}Wenn aber eure heilige Synode davon �berzeugt war, werden wir uns nicht widersetzen,
sondern unterst�tzen die Entscheidungen von euch und geben durch dieses Schreiben unsere Zustimmung
bekannt, nicht als ob wir schwer an der Verbannung tragen, sondern weil wir den Verdacht der H�resie
absch�tteln wollen.

\kapnum{4}Denn wenn ihr euch jetzt pers�nlich dazu entschlie�t, uns wieder aufzunehmen, werdet ihr
welche gewinnen, die euch in allem unterst�tzen und sich euren Entschl�ssen anschlie�en,
insbesondere da es eurer Fr�mmigkeit gefiel, dem in diesen Angelegenheiten Beschuldigten mit
Freundlichkeit zu begegnen und ihn zur�ckzurufen. Es ist aber unangemessen, da� wir, wenn der, der
schuldig zu sein schien, zur�ckgerufen worden ist und sich gerechtfertigt hat in den Dingen, derer
er beschuldigt wurde, im Schweigen verharren und so Beweise gegen uns selbst liefern.

\kapnum{5}Entschlie�t euch also, wie es eurer christusliebenden Fr�mmigkeit entspricht, den
gottgeliebtesten Kaiser zu erinnern, unsere Bitten auszuh�ndigen und baldigst einen Entschlu� �ber
uns zu fassen, der euch gef�llt.
