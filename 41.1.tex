%% Erstellt von uh
%% �nderungen:
%%%% 13.5.2004 Layout-Korrekturen (avs) %%%%
\chapter{Briefwechsel und Synoden in Rom und Antiochien der Jahre 340 und 341}
\thispagestyle{empty}
\section{Regest eines Briefes des Julius von Rom an die �stlichen Bisch�fe}
% \label{sec:41.1}
\label{sec:BriefJulius}
\begin{praefatio}
  \begin{description}
  \item[Fr�hjahr 340]Hintergrund sind die seit
    Nicaea\index[synoden]{Nicaea!a. 325} anhaltenden Differenzen um
    Personalien und Bischofssitze, womit sich aufgrund gegenseitiger
    H�resievorw�rfe zunehmend auch theologische Fragen verkn�pften,
    insbesondere nach der Aufnahme
    Markells\index[namen]{Markell!Bischof von Ancyra} in die
    Kirchengemeinschaft durch den Westen, der zuvor wegen H�resie im
    Osten verurteilt worden war
    (vgl. Dok. \ref{ch:Konstantinopel336}). Gegen die R�ckkehr des
    Athanasius\index[namen]{Athanasius!Bischof von Alexandrien} aus
    seinem ersten Exil nach Alexandrien\index[namen]{Alexandrien} im
    November 337 (ind.\,ep.\,fest. 10 [a. 338]) protestierten die
    Bisch�fe des Ostens, die ihn in auf der Synode von
    Tyrus\index[synoden]{Tyrus!a. 335} im Jahr 335 abgesetzt
    hatten. F�r sie war Pistus\index[namen]{Pistus!Bischof von
      Alexandrien} der rechtm��ige alexandrinische Bischof (er war von
    Alexander von Alexandrien\index[namen]{Alexander!Bischof von
      Alexandrien} und auf der Synode von
    Nicaea\index[synoden]{Nicaea!a. 325} 325 exkommuniziert und von
    Secundus von Ptolema"is\index[namen]{Secundus!Bischof von
      Ptolema"is} ordiniert worden [vgl. Dok. \ref{sec:4a} = Urk. 4a;
    Dok. \ref{ch:6} = Urk. 6; Dok. \ref{sec:BriefJuliusII},16;
    ep.\,encycl. 6,2; Epiph., haer. 69,8,5]). Sie sandten Schreiben
    sowohl an Julius von Rom\index[namen]{Julius!Bischof von Rom} als
    auch an die Kaiser zusammen mit den Akten der Synode von
    Tyrus\index[synoden]{Tyrus!a. 335} (h.\,Ar. 9,1; apol.\,sec. 19,3;
    20,1; 83,4; Dok. \ref{sec:BriefJuliusII},8~f.; Socr., h.\,e. II
    17,4). In Reaktion darauf berief
    Athanasius\index[namen]{Athanasius!Bischof von Alexandrien} eine
    Synode ein (338), und die ca. 80 in
    Alexandrien\index[synoden]{Alexandrien!a. 338} versammelten
    �gyptischen Bisch�fe verfa�ten einen apologetischen Brief
    (apol.\,sec. 3--19) und verschickten ihn ebenfalls. In
    Rom\index[namen]{Rom} trafen nun die alexandrinischen Gesandten
    auf die aus Antiochien\index[namen]{Antiochien} (Socr., h.\,e. II
    17,5~f.; Ath., apol.\,sec. 20,1;
    Dok. \ref{sec:BriefJuliusII},8~f.; ep.\,encycl. 7,2).  Angesichts
    der tiefgreifenden theologischen Differenzen zwischen den Kirchen
    des Ostens und denen des Westens, die in den von den westlichen
    Kirchen nicht rezipierten Exkommunikationen von Athanasius und
    Markell ihren Niederschlag fanden, entstand die Notwendigkeit
    einer synodalen Kl�rung der Lage.  Auf einen entsprechenden Brief
    des Julius\index[namen]{Julius!Bischof von Rom} reagierten die
    Antiochener jedoch mit einer eigenen
    Synode\index[synoden]{Antiochien!a. 339}, auf der
    Gregor\index[namen]{Gregor!Bischof von Alexandrien} anstelle des
    umstrittenen Pistus\index[namen]{Pistus!Bischof von Alexandrien}
    als Bischof f�r Alexandrien bestimmt wurde
    (Dok. \ref{sec:BriefJuliusII},41--45;
    Dok. \ref{sec:SerdicaRundbrief},15 mit Anm.), der dann kurz vor
    Ostern 339 (ind.\,ep.\,fest. 11) mit milit�rischer Hilfe in
    Alexandrien\index[namen]{Alexandrien} einziehen konnte.
    Athanasius\index[namen]{Athanasius!Bischof von Alexandrien}
    protestierte gegen diese Ereignisse in seiner ep.\,encycl. und
    wandte sich nun nach Rom\index[namen]{Rom}, einerseits wegen der
    traditionell guten Beziehungen zwischen
    Alexandrien\index[namen]{Alexandrien} und Rom\index[namen]{Rom},
    andererseits auch, um dem Herrschaftsbereich des
    Constantius\index[namen]{Constantius, Kaiser} zu entfliehen
    (apol.\,Const. 4,1--4; das genaue Datum ist unbekannt, da es
    schwierig zu beurteilen ist, ob
    Athanasius\index[namen]{Athanasius!Bischof von Alexandrien}
    tats�chlich direkt, wie er es in apol.\,Const. beschreibt, nach
    Rom\index[namen]{Rom} reiste oder auf Umwegen, wie es seine Gegner
    behaupteten [Dok. \ref{sec:RundbriefSerdikaOst},11]). In
    Rom\index[namen]{Rom} fanden sich weitere aus dem Osten exilierte
    Bisch�fe ein wie Markell von Ancyra\index[namen]{Markell!Bischof
      von Ancyra} (vgl. Dok. \ref{sec:MarkellJulius}), Paulus von
    Konstantinopel\index[namen]{Paulus!Bischof von Konstantinopel} und
    Asclepas von Gaza\index[namen]{Asclepas!Bischof von Gaza}
    (vgl. Dok. \ref{sec:BriefJuliusII},52; Dok.
    \ref{sec:RundbriefSerdikaOst},10.15.21.28; Ath.,
    apol.\,Const. 4,1; Socr., h.\,e. II 15,1~f.; Soz., h.\,e. III
    8,1). Daraufhin lud der r�mische Bischof
    Julius\index[namen]{Julius!Bischof von Rom} mit einem Schreiben,
    von dem nur das vorliegende Regest erhalten ist, erneut zu einer
    Synode nach Rom\index[synoden]{Rom!a. 339} ein und setzte zugleich
    schon einen Termin fest, zu dem eine Delegation aus dem Osten
    erscheinen sollte.
  \item[�berlieferung]Der Brief des
    Julius\index[namen]{Julius!Bischof von Rom} ist nicht
    �berliefert. Neben dem Regest bei Soz., h.\,e. III
    8,3~f. (vgl. Socr., h.\,e. II 15,3) gibt es nur eine Erw�hnung
    durch Athanasius in h.\,Ar. 11,1 (vgl. apol.\,sec. 20,1).
  \item[Fundstelle] \refpassage{soz1}{soz2} Soz., h.\,e. III
    8,3~f. (\editioncite[110,27--111,4]{Hansen:Soz}); \refpassage{ath1}{ath2} Ath.,
    h.\,Ar. 11,1 (\editioncite[188,30--32]{Opitz1935})
  \end{description}
\end{praefatio}
\autor{Das Regest bei Sozomenus}
\begin{pairs}
\selectlanguage{polutonikogreek}
\begin{Leftside}
\beginnumbering
\pstart
\hskip -1.25em\edtext{\abb{}}{\xxref{soz1}{soz2}\Cfootnote{\latintext Soz.
(BC=b)}}\specialindex{quellen}{section}{Sozomenus!h.\,e.!III 8,3~f.}
\kap{1}ka`i\edlabel{soz1} to~ic >an`a t`hn <'ew >episk'opoic >'egraye memf'omenoc <wc
o>uk >orj~wc bouleusam'enoic per`i to`uc >'andrac ka`i t`ac >ekklhs'iac tar'attousi t~w|
m`h >emm'enein to~ic >en Nika'ia|\edindex[synoden]{Nicaea!a. 325} d'oxasin. >ol'igouc
d`e >ek p'antwn e>ic <rht`hn
<hm'eran pare~inai >ek'eleuse diel'egxontac dika'ian >ep'' a>uto~ic >enhnoq'enai t`hn
y~hfon; >`h to~u loipo~u o>uk >an'exesjai >hpe'ilhsen, e>i m`h pa'usointo
newter'izontec.
\pend
\pstart
Ka`i <o m`en t'ade >'egrayen.\edlabel{soz2}
\pend
\end{Leftside}
\begin{Rightside}
\begin{translatio}
\beginnumbering
\pstart
% \autor{Sozomenus}
\noindent Und er (Julius) schrieb an die Bisch�fe aus dem Osten und machte ihnen Vorw�rfe, da� sie nicht
korrekt mit den M�nnern verfahren w�ren und die Kirchen dadurch verwirren w�rden, da� sie sich
nicht an die Beschl�sse von Nicaea hielten. Also ordnete er an, da� einige stellvertretend
f�r alle an dem festgesetzten Tag erscheinen sollten, um klarzustellen, ob die Entscheidung
bei ihnen zu Recht gefallen sei; andernfalls drohte er, sie nicht l�nger ertragen zu
k�nnen, wenn sie nicht mit den Neuerungen aufh�rten.
\pend
\pstart
Derart fiel also sein Brief
aus\footnoteA{Das Antwortschreiben der Antiochener nach Rom best�tigt (vgl. Dok.
\ref{sec:BriefSynode341}), da� Julius in anscheinend barschem Tonfall nicht nur zu einer
Synode eingeladen, sondern den Antiochenern auch ">Arianismus"< vorgeworfen hatte. Zur
Frage nach der Berechtigung des Julius, zu einer Synode einzuladen, um die bereits
gefallenen Synodalurteile erneut aufzurollen, vgl. Dok. \ref{sec:BriefSynode341} und
\ref{sec:BriefJuliusII}.}.
\pend
\endnumbering
\end{translatio}
\end{Rightside}
\Columns
\end{pairs}
\autor{Die Erw�hnung bei Athanasius}
\begin{pairs}
\selectlanguage{polutonikogreek}
\begin{Leftside}
\pstart
\hskip -1.25em\edtext{\abb{}}{\xxref{ath1}{ath2}\Cfootnote{\latintext Ath. (BKPO
R)}}\specialindex{quellen}{section}{Athanasius!h.\,Ar.!11,1} %%
\kap{2}<o\edlabel{ath1} d`e >Io'ulioc\edindex[namen]{Julius!Bischof von
Rom} gr'afei ka`i p'empei presv\-bu\-t'e\-rouc,
>Elp'idion\edindex[namen]{Elpidius!Presbyter in Rom} ka`i
Fil'oxenon\edindex[namen]{Philoxenus!Presbyter in Rom}, <or'isac ka`i
\edtext{projesm'ian}{\Dfootnote{projum'ian
\latintext B}}, <'ina >`h >'eljwsin >`h gin'wskoien <eauto`uc <up'optouc e>~inai kat`a
p'anta.\edlabel{ath2}%% Ath., h.Ar. 11,1%%%
\pend
% \endnumbering
\end{Leftside}
\begin{Rightside}
\begin{translatio}
\beginnumbering
\pstart
% \autor{Athanasius}
\noindent Und Julius schrieb einen Brief\footnoteA{Sowohl an dieser Stelle als auch in apol.sec.
20,1 erw�hnt Athanasius diesen Brief des Julius, um seine Flucht nach Rom 339/340 als Reise
zur Synode darzustellen.} und schickte Presbyter, Elpidius und Philoxenus, und setzte dabei auch eine Frist, damit sie entweder k�men oder erkannten, da� sie selbst in jeder Hinsicht
verd�chtig sind.
\pend
\endnumbering
\end{translatio}
\end{Rightside}
\Columns
\end{pairs}
% \thispagestyle{empty}
%%% Local Variables: 
%%% mode: latex
%%% TeX-master: "dokumente_master"
%%% End: 
