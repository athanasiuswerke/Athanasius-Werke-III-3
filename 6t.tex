% \section{Glaubensbekenntnis des Arius und seiner Genossen an Alexander von Alexandrien}
\kapitel{Theologische Erkl�rung alexandrinischer Kleriker an Alexander von Alexandrien (Urk. 6)}
\label{ch:6}
\thispagestyle{empty}

\kapnum{1}Unseren seligen Papas und Bischof Alexander gr��en die Presbyter und Diakone im Herrn!

\kapnum{2}Unser von den Vorfahren �berlieferter Glaube, den wir auch von dir gelernt haben, seliger
Papas, ist folgender:  Wir kennen einen Gott, allein ungeworden, allein ewig, allein
anfangslos, allein wahrhaftig, allein im Besitz der Unsterblichkeit, allein weise, allein
gut, allein Herrscher, Richter aller Dinge, Ordner, Verwalter, unver�nderlich und
unwandelbar, gerecht und gut, Gott des Gesetzes und der Propheten und des neuen Bundes,
der den eingeborenen Sohn vor ewigen Zeiten gezeugt hat, durch den er auch die Zeiten und
das All geschaffen hat. Er hat ihn aber gezeugt nicht dem Anschein nach, sondern in Wahrheit, ins Dasein
gerufen durch seinen eigenen Willen, als Unver�nderlichen und Unwandelbaren, als ein vollkommenes
Gesch�pf Gottes, doch nicht wie eines der Gesch�pfe, als ein Erzeugnis, aber nicht wie eines
der gezeugten Dinge, 

\kapnum{3}auch nicht wie Valentinus lehrte, da� das vom Vater Gezeugte eine Emanation sei, auch
nicht wie Mani das Gezeugte als wesenseinen Teil des Vaters einf�hrte, und auch nicht wie
Sabellius es Sohnvater nannte, indem er die Monade unterteilte und auch nicht wie Hieracas
als Licht von einem Leuchter oder wie eine in zwei Teile geteilte Fackel. Auch wurde
der vorher Existierende nicht sp�ter gezeugt und dazu erschaffen als Sohn, wie ja auch du selbst,
seliger Papas, sehr oft die, die derartiges einf�hren wollten, mitten in der Kirche und
w�hrend der Versammlung zurechtgewiesen hast, sondern, wie wir sagen, nach dem Willen
Gottes wurde er vor Zeiten und �onen geschaffen und empfing das Leben und das
Sein und die Ehren vom Vater, wobei der Vater zusammen mit ihm existiert.

\kapnum{4}Denn der Vater hat, als er ihm alles zum Erbe gab, sich nicht dessen beraubt, was er als
Ungezeugter in sich tr�gt; denn er ist die Quelle aller Dinge. \ladd{Daher gibt es drei Hypostasen.}\footnote{Vermutlich handelt es sich um eine in den Text eingedrungene Glosse.}
Gott n�mlich, die Ursache von allem, ist einzig und allein anfangslos, der Sohn aber, der zeitlos
von dem Vater gezeugt und vor allen Zeiten geschaffen und gegr�ndet wurde, war nicht, bevor er
gezeugt wurde, sondern, da er zeitlos vor allen Dingen gezeugt wurde, wurde er allein vom Vater ins
Dasein gerufen. Denn er ist nicht ewig oder gemeinsam mit dem Vater ewig und ungeworden, auch besitzt
er nicht zusammen mit dem Vater das Sein, wie einige das Verh�ltnis bestimmen und zwei ungezeugte
Anf�nge einf�hren, sondern wie Gott eine Einheit und der Anfang aller Dinge ist, so ist er auch vor
allen Dingen. Daher ist er auch vor dem Sohn, wie wir es auch bei dir gelernt haben, als du in der
Mitte der Kirche predigtest.

\kapnum{5}So wie er nun von Gott das Sein hat und ihm die Ehren und das Leben und alle
Dinge �bergeben wurden, so ist Gott sein Ursprung. Denn er herrscht �ber ihn, da er
sein Gott und vor ihm ist. Wenn aber Aussagen wie ">aus ihm"< und ">aus dem Scho�"< und
">vom Vater bin ich ausgegangen und gekommen"< von einigen so gedeutet werden, als sei er
sein wesenseines Teil oder eine Emanation, dann ist der Vater nach ihrer Vorstellung
zusammengesetzt, teilbar, wandelbar und ein K�rper, und der
k�rperlose Gott erleidet ihrer Meinung nach genau das, was einem K�rper zukommt.

Ich bete, da� es dir im Herren wohlergeht, seliger Papas. 
Arius, Aeithales, Achilleus, Carpones, Sarmatus, Arius, die Presbyter.
Die Diakone Euzoius, Lucius, Julius, Menas, Helladius, Gaius.
Die Bisch�fe Secundus aus der Pentapolis, Theonas aus Libyen, Pistus.
\clearpage