% \section{Brief des Euseb von Caesarea an Alexander von Alexandrien}
\kapitel{Fragment eines Briefes des Eusebius von Caesarea an Alexander von Alexandrien (Urk. 7)}
\label{ch:7}
\thispagestyle{empty}

\begin{footnotesize}
\kapnum{1}Desgleichen aber auch im Brief an den heiligen Alexander, den Lehrer des gro�en
Athanasius, dessen Anfang lautet:
\end{footnotesize}

Unter so gro�er Aufregung und Besorgnis dar�ber traf der Brief ein.

\begin{footnotesize}
Aufs eindeutigste l�sternd sagt er �ber Arius und seine Anh�nger folgendes:
\end{footnotesize}

\kapnum{2}Dein Brief klagt sie an, da� sie sagen, der Sohn sei aus nichts geworden wie eines
von allen Gesch�pfen. Sie aber brachten ihr eigenes Schriftst�ck vor, das sie an dich
gerichtet und in dem sie ihren eigenen Glauben dargelegt haben und mit folgenden Worten so bekannten:
">Der Gott des Gesetzes und der Propheten und des neuen Bundes zeugte den einziggeborenen
Sohn vor ewigen Zeiten, durch den er auch die Zeiten und das All gemacht hat, gezeugt nicht dem Anschein
nach, sondern in Wahrheit, ins Dasein gerufen durch seinen eigenen Willen, unver�nderlich und
unwandelbar, ein vollkommenes Gesch�pf Gottes, doch nicht wie eins der Gesch�pfe."< Wenn also nun
ihr Schriftst�ck die Wahrheit sagt, in dem sie den Sohn Gottes  vor ewigen Zeiten bekennen,
durch den er auch die Zeiten gemacht hat, dann mu� dir auch vorgelegen haben, da� er unwandelbar und
vollkommenes Gesch�pf Gottes ist, aber nicht wie eines der Gesch�pfe.

\kapnum{3}Dein Brief beschuldigt sie aber, da� sie sagen, der Sohn sei geworden wie eines
der Gesch�pfe. Da sie das aber nicht sagen, sondern deutlich definieren, da� er nicht
">wie eines der Gesch�pfe ist"<, sieh zu, da� ihnen damit nicht wiederum sogleich ein
Anla� gegeben wird, sich daran zu machen, etwas zu unternehmen und Unruhe zu stiften,
sooft sie nur wollen.

\kapnum{4}Ferner hast du sie beschuldigt, sie w�rden sagen, da� ">der, der ist, den, der
nicht ist, gezeugt hat"<. Ich wundere mich aber, ob jemand das anders formulieren kann.
Denn wenn der, der ist, einer ist, ist es doch offensichtlich, da� alles aus ihm wurde,
was auch nach ihm ist. Falls aber nicht er allein der ist, der ist, sondern auch der Sohn
der war, der ist, wie zeugte dann der, der ist, den, der ist? Denn dann d�rften es wohl
zwei sein, die sind.

\kapnum{5}
\begin{footnotesize} So also Eusebius an den ber�hmten Alexander; aber auch weitere Briefe
von ihm an denselben heiligen Mann werden vorgelegt, in denen vielerlei L�sterungen zu
finden sind, die Arius und seine Anh�nger rechtfertigen.
\end{footnotesize}