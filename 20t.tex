\kapitel{Brief des Kaisers Konstantin mit der Einberufung zur Synode von Nicaea (Urk.~20)}
\label{ch:20}
\thispagestyle{empty}

\begin{center}
Brief des Kaisers Konstantin, der die Bisch�fe von Ancyra nach Nicaea rief.\end{center}

Da� in meinen Augen nichts gewichtiger ist als die Furcht vor Gott, glaube ich, ist
jedermann offenbar. Weil aber zun�chst darin �bereinstimmung bestand, da� die Bischofssynode in Ancyra in Galatia stattfinden sollte, so scheint es uns jetzt aus vielen Gr�nden gut, da� es passend sei, da� sie sich in Nicaea in Bithynia versammle: Sowohl wegen
der Bisch�fe, die aus Italia und den �brigen Gegenden Europas sind, als auch wegen der
guten Mischung der Luft und damit ich als Augenzeuge und Teilnehmer dem Geschehen nahe sei. Deshalb, liebe Br�der, lasse ich euch wissen, da� ihr euch alle geflissentlich in jener genannten Stadt, d.\,h. in Nicaea, versammelt. Jeder einzelne von euch
also, wenn er denkt, da� dies vorz�glich sei, wie ich bereits gesagt habe, bem�he
sich, ohne irgendeinen Verzug schnell zu kommen, um aus der N�he in eigener Person Augenzeuge von dem zu werden, was geschieht.

Gott m�ge euch bewahren, liebe Br�der.

