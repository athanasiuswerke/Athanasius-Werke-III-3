% erstellt von AvS 22.11.04
% \section{Brief des Arius an Eusebius von Nikomedien}
\kapitel{Brief des Arius an Eusebius von Nikomedien (Urk.~1)}
\label{ch:1}
\kapnum{1}Dem hei�ersehnten Herrn, dem treuen Menschen Gottes, dem rechtgl�ubigen Eusebius entbietet
Arius, der vom Papas Alexander ungerechterweise um der Wahrheit willen, die alles besiegt und 
die auch du mit dem Schild deckst, verfolgt wird, im Herrn seinen Gru�.

\kapnum{2}Als mein Vater Ammonius im Begriff war, nach Nikomedien zu reisen, schien es mir vern�nftig und geziemend zu sein, dich durch ihn zu gr��en, zugleich aber
auch die dir angeborene Liebe und Gesinnung, die du den Br�dern gegen�ber um Gott und seines Christus willen hegst, daran zu erinnern, da� der Bischof uns arg zusetzt, verfolgt und alle
Hebel gegen uns in Bewegung setzt, so da� er uns aus der Stadt vertreibt wie gottlose Menschen, weil
wir nicht mit ihm �bereinstimmen, wenn er �ffentlich sagt: ">Immer Gott immer Sohn, zugleich Vater
zugleich Sohn, der Sohn existiert ungezeugt zusammen mit dem Vater, ewig-gezeugt, ungezeugt
geworden, weder durch einen Gedanken noch durch irgendeinen kleinsten Moment geht Gott dem
Sohn voran, ewig Gott ewig Sohn, aus Gott selbst ist der Sohn."<

\kapnum{3}Und da ja dein Bruder Eusebius in Caesarea, Theodotus, Paulinus, Athanasius, Gregorius,
A"etius und alle im Osten sagen, da� Gott vor dem Sohn ohne Anfang existiert, wurden sie
ausgeschlossen, abgesehen von Philogonius, Hellanicus und Macarius, h�retischen Menschen, die nicht einmal Taufunterricht empfangen haben, von denen die einen den Sohn ">R�lpser"< nennen, die anderen
">Hervorgeworfenes"<, wieder andere ihn aber zusammen mit dem Vater ungezeugt nennen.

\kapnum{4}Und diese Gottlosigkeiten k�nnten wir nicht einmal dann h�ren, wenn uns die H�retiker
unz�hlige Tode androhten. Wir aber, was sagen, denken, lehrten und lehren wir auch heute? Da� der Sohn nicht
ungezeugt und nicht Teil eines Ungezeugten auf irgendeine Art und Weise ist, noch da� er aus
irgendetwas Existierendem entstanden ist, sondern auf Grund von Wollen und Wunsch vor
Zeiten und Ewigkeiten, voll der Gnade und der Wahrheit, Gott, Eingeborener,
unver�nderlich.

\kapnum{5}Und bevor er gezeugt, geschaffen, beschlossen oder gegr�ndet wurde, war er nicht.
Denn er war nicht ungezeugt. Wir werden aber verfolgt, weil wir sagen, der Sohn hat einen Ursprung,
Gott aber ist ursprungslos. Deswegen werden wir verfolgt, weil wir auch sagen, er ist aus
nichts. Wir haben aber so gesprochen, weil er kein Teil Gottes ist und weil
er nicht aus irgendetwas Existierendem stammt. Deswegen werden wir verfolgt, das �brige wei�t 
Du.

Ich bete, da� es dir im Herrn wohl ergehen m�ge -- eingedenk unserer Betr�bnisse --, mein
Mit-Lukianist, wahrhaftig ">Frommer"<\footnote{Hier liegt ein Wortspiel mit dem Namen
Eusebius vor.}.