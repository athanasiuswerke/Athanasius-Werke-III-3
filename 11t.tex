% \section{Brief des Athanasius von Anazarbos an Alexander von Alexandrien}
\kapitel{Fragment eines Briefes des Athanasius von Anazarba an Alexander von Alexandrien (Urk.~11)}
\label{ch:11}

\begin{footnotesize}
Er schrieb an den Bischof Alexander und wagte folgenderma�en zu reden:
\end{footnotesize}

\kapnum{1}Was tadelst du Arius und seine Mitstreiter, wenn sie sagen, der Sohn Gottes wurde
aus Nichts als ein Gesch�pf gemacht und ist eines von allen? Unter den hundert Schafen, mit denen alle
Gesch�pfe im Gleichnis verglichen werden, ist n�mlich eines von ihnen auch der Sohn.

\kapnum{2}Wenn also die Hundert keine Gesch�pfe und nicht geworden sind oder wenn es noch
etwas gibt jenseits dieser Hundert, dann ist klar, da� der Sohn weder ein Gesch�pf noch eines
wie alle anderen sein soll. Wenn aber alle Hundert geworden sind und es nichts jenseits der Hundert
gibt au�er Gott allein, was sagen Arius und seine Mitstreiter Verkehrtes, wenn sie ihn als einen
von den Hundert ansehen und Christus dazurechnen und sagen: Er ist einer von ihnen allen!