% \section{Brief des Paulin von Tyrus}
\kapitel{Referate eines (?) Briefes des Paulinus von Tyrus (Urk. 9)}
\label{ch:9}

\kapnum{1}Paulinus dachte nicht an die evangelische Lehre und schrieb folgendes, wobei er aber zugab, da� einige zwar aus sich selbst dazu bewegt worden sind,
andere aber durch die Lekt�re der Schriften der oben genannten M�nner zu dieser Denkweise gebracht worden sind.

Schlie�lich unterschrieb er zum Schlu�, gleichsam als ob er dem Beweis eine Art
Schlu�stein aufsetzen wollte, seinen eigenen Brief mit einer Passage aus Origenes, als ob
dieser besser �berzeugen k�nne als die Evangelisten und Apostel. Die Passage aber ist
folgende:

">Es ist an der Zeit, da� wir, wenn wir das Thema des Vaters und des Sohnes und
des heiligen Geistes wieder aufgreifen, einige der damals ausgelassenen Punkte behandeln.
�ber den Vater (ist zu sagen), da� er, der ungetrennt und ungeteilt ist, Vater des Sohnes wird, nicht indem er ihn hervorbringt, wie einige meinen. Denn wenn der Sohn eine Emanation des
Vaters und ein aus ihm Gezeugtes ist, genau wie die Kinder der Lebewesen, dann mu� der
Hervorbringende und der Hervorgebrachte ein K�rper sein."<

\vspace*{\baselineskip}
\kapnum{2}\hskip -1em Von diesen Worten ist auch sein (sc. des Asterius) geistiger Vater Paulinus �berzeugt und z�gert nicht, dasselbe zu sagen und zu schreiben, sagt er doch einmal, Christus sei ein
zweiter Gott und dieser sei ein menschlicherer Gott geworden, ein anderes Mal bestimmt er ihn aber als Gesch�pf.

\vspace*{\baselineskip}
\kapnum{3}\hskip -1em Danach verleumdet er (sc. Markell) den Seligen, als ob er von vielen G�ttern gesprochen habe.

\vspace*{\baselineskip}
\kapnum{4}\hskip -1em Daraus hatte auch Paulinus, der Vater des Asterius, gelernt und glaubte, es gebe j�ngere G�tter.
